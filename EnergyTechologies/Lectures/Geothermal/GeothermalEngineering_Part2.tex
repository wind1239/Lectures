% Aberdeen style guide should be followed when using this
% layout. Their template powerpoint slide is used to extract the
% Aberdeen color and logo but is otherwise ignored (it has little or
% no formatting in it anyway).
%
% http://www.abdn.ac.uk/documents/style-guide.pdf

%%%%%%%%%%%%%%%%%%%% Document Class Settings %%%%%%%%%%%%%%%%%%%%%%%%%
% Pick if you want slides, or draft slides (no animations)
%%%%%%%%%%%%%%%%%%%%%%%%%%%%%%%%%%%%%%%%%%%%%%%%%%%%%%%%%%%%%%%%%%%%%%
%Normal document mode
\documentclass[10pt,compress]{beamer}
%Draft or handout mode
%\documentclass[10pt,compress,handout]{beamer}
%\documentclass[10pt,compress,handout,ignorenonframetext]{beamer} 
 
%%%%%%%%%%%%%%%%%%%% General Document settings %%%%%%%%%%%%%%%%%%%%%%%
% These settings must be set for each presentation
%%%%%%%%%%%%%%%%%%%%%%%%%%%%%%%%%%%%%%%%%%%%%%%%%%%%%%%%%%%%%%%%%%%%%%

\newcommand{\shortname}{Dr Jeff Gomes}
\newcommand{\fullname}{Dr Jeff Gomes}
\institute{School of Engineering}
\newcommand{\emailaddress}{jefferson.gomes@abdn.ac.uk}
\newcommand{\logoimage}{../FigBanner/UoAHorizBanner}
\title{Renewable Energy 1: Solar and Geothermal (EG501J)}
\subtitle{Module 4: Introduction to Porous Media Fluid Flow $\&$ Heat Transfer in Geothermal Systems}
%\date[2014-15]{2014-15}
\date[]{}

%%%%%%%%%%%%%%%%%%%% Template settings %%%%%%%%%%%%%%%%%%%%%%%%%%%%%%%
% You shouldn't have to change below this line, unless you want to.
%%%%%%%%%%%%%%%%%%%%%%%%%%%%%%%%%%%%%%%%%%%%%%%%%%%%%%%%%%%%%%%%%%%%%%
\usecolortheme{whale}
\useoutertheme{infolines}

% Use the fading effect for items that are covered on the current
% slide.
\beamertemplatetransparentcovered

% We abuse the author command to place all of the slide information on
% the title page.
\author[\shortname]{%
  \fullname\\\ttfamily{\emailaddress}
}


%At the start of every section, put a slide indicating the contents of the current section.
\AtBeginSection[] {
  \begin{frame}
    \frametitle{Section Outline}
    \tableofcontents[currentsection]
  \end{frame}
}

% Allow the inclusion of movies into the Presentation! At present,
% only the Okular program is capable of playing the movies *IN* the
% presentation.
\usepackage{multimedia}
\usepackage{animate}
\usepackage{comment} 

%%%%% Color settings
\usepackage{color}
%% The background color for code listings (i.e. example programs)
\definecolor{lbcolor}{rgb}{0.9,0.9,0.9}%
\definecolor{UoARed}{rgb}{0.64706, 0.0, 0.12941}
\definecolor{UoALight}{rgb}{0.85, 0.85, 0.85}
\definecolor{UoALighter}{rgb}{0.92, 0.92, 0.92}
\setbeamercolor{structure}{fg=UoARed} % General background and higlight color
\setbeamercolor{frametitle}{bg=black} % General color
\setbeamercolor{frametitle right}{bg=black} % General color
\setbeamercolor{block body}{bg=UoALighter} % For blocks
\setbeamercolor{structure}{bg=UoALight} % For blocks
% Rounded boxes for blocks
\setbeamertemplate{blocks}[rounded]

%%%%% Font settings
% Aberdeen requires the use of Arial in slides. We can use the
% Helvetica font as its widely available like so
% \usepackage{helvet}
% \renewcommand{\familydefault}{\sfdefault}
% But beamer already uses a sans font, so we will stick with that.

% The size of the font used for the code listings.
\newcommand{\goodsize}{\fontsize{6}{7}\selectfont}

% Extra math packages, symbols and colors. If you're using Latex you
% must be using it for formatting the math!
\usepackage{amscd,amssymb} \usepackage{amsfonts}
\usepackage[mathscr]{eucal} \usepackage{mathrsfs}
\usepackage{latexsym} \usepackage{amsmath} \usepackage{bm}
\usepackage{amsthm} \usepackage{textcomp} \usepackage{eurosym}
% This package provides \cancel{a} and \cancelto{a}{b} to "cancel"
% expressions in math.
\usepackage{cancel}

\usepackage{comment} 

% Get rid of font warnings as modern LaTaX installations have scalable
% fonts
\usepackage{type1cm} 

%\usepackage{enumitem} % continuous numbering throughout enumerate commands

% For exact placement of images/text on the cover page
\usepackage[absolute]{textpos}
\setlength{\TPHorizModule}{1mm}%sets the textpos unit
\setlength{\TPVertModule}{\TPHorizModule} 

% Source code formatting package
\usepackage{listings}%
\lstset{ backgroundcolor=\color{lbcolor}, tabsize=4,
  numberstyle=\tiny, rulecolor=, language=C++, basicstyle=\goodsize,
  upquote=true, aboveskip={1.5\baselineskip}, columns=fixed,
  showstringspaces=false, extendedchars=true, breaklines=false,
  prebreak = \raisebox{0ex}[0ex][0ex]{\ensuremath{\hookleftarrow}},
  frame=single, showtabs=false, showspaces=false,
  showstringspaces=false, identifierstyle=\ttfamily,
  keywordstyle=\color[rgb]{0,0,1},
  commentstyle=\color[rgb]{0.133,0.545,0.133},
  stringstyle=\color[rgb]{0.627,0.126,0.941}}

% Allows the inclusion of other PDF's into the final PDF. Great for
% attaching tutorial sheets etc.
\usepackage{pdfpages}
\setbeamercolor{background canvas}{bg=}  

% Remove foot note horizontal rules, they occupy too much space on the slide
\renewcommand{\footnoterule}{}

% Force the driver to fix the colors on PDF's which include mixed
% colorspaces and transparency.
\pdfpageattr {/Group << /S /Transparency /I true /CS /DeviceRGB>>}

% Include a graphics, reserve space for it but
% show it on the next frame.
% Parameters:
% #1 Which slide you want it on
% #2 Previous slides
% #3 Options to \includegraphics (optional)
% #4 Name of graphic
\newcommand{\reserveandshow}[4]{%
\phantom{\includegraphics<#2|handout:0>[#3]{#4}}%
\includegraphics<#1>[#3]{#4}%
}

\newcommand{\frc}{\displaystyle\frac}
\newcommand{\red}{\textcolor{red}}
\newcommand{\blue}{\textcolor{blue}}
\newcommand{\green}{\textcolor{green}}
\newcommand{\purple}{\textcolor{purple}}
 
%% Handsout -- comment out the lines below to create handstout with 4 slides in a page with space for comments
\usepackage{handoutWithNotes}

\mode<handout>
{
\usepackage{pgf,pgfpages}

\pgfpagesdeclarelayout{2 on 1 boxed with notes}
{
\edef\pgfpageoptionheight{\the\paperheight} 
\edef\pgfpageoptionwidth{\the\paperwidth}
\edef\pgfpageoptionborder{0pt}
}
{
\setkeys{pgfpagesuselayoutoption}{landscape}
\pgfpagesphysicalpageoptions
    {%
        logical pages=4,%
        physical height=\pgfpageoptionheight,%
        physical width=\pgfpageoptionwidth,%
        last logical shipout=2%
    } 
\pgfpageslogicalpageoptions{1}
    {%
    border code=\pgfsetlinewidth{1pt}\pgfstroke,%
    scale=1,
    center=\pgfpoint{.25\pgfphysicalwidth}{.75\pgfphysicalheight}%
    }%
\pgfpageslogicalpageoptions{2}
    {%
    border code=\pgfsetlinewidth{1pt}\pgfstroke,%
    scale=1,
    center=\pgfpoint{.25\pgfphysicalwidth}{.25\pgfphysicalheight}%
    }%
\pgfpageslogicalpageoptions{3}
    {%
    border shrink=\pgfpageoptionborder,%
    resized width=.7\pgfphysicalwidth,%
    resized height=.5\pgfphysicalheight,%
    center=\pgfpoint{.75\pgfphysicalwidth}{.29\pgfphysicalheight},%
    copy from=3
    }%
\pgfpageslogicalpageoptions{4}
    {%
    border shrink=\pgfpageoptionborder,%
    resized width=.7\pgfphysicalwidth,%
    resized height=.5\pgfphysicalheight,%
    center=\pgfpoint{.75\pgfphysicalwidth}{.79\pgfphysicalheight},%
    copy from=4
    }%

\AtBeginDocument
    {
    \newbox\notesbox
    \setbox\notesbox=\vbox
        {
            \hsize=\paperwidth
            \vskip-1in\hskip-1in\vbox
            {
                \vskip1cm
                Notes\vskip1cm
                        \hrule width\paperwidth\vskip1cm
                    \hrule width\paperwidth\vskip1cm
                        \hrule width\paperwidth\vskip1cm
                    \hrule width\paperwidth\vskip1cm
                        \hrule width\paperwidth\vskip1cm
                    \hrule width\paperwidth\vskip1cm
                    \hrule width\paperwidth\vskip1cm
                    \hrule width\paperwidth\vskip1cm
                        \hrule width\paperwidth
            }
        }
        \pgfpagesshipoutlogicalpage{3}\copy\notesbox
        \pgfpagesshipoutlogicalpage{4}\copy\notesbox
    }
}
}

%\pgfpagesuselayout{2 on 1 boxed with notes}[letterpaper,border shrink=5mm]
%\pgfpagesuselayout{2 on 1 boxed with notes}[letterpaper,border shrink=5mm]
 
\begin{document}

% Title page layout
\begin{frame}
  \titlepage
  \vfill%
  \begin{center}
    \includegraphics[clip,width=0.8\textwidth]{\logoimage}
  \end{center}
\end{frame}


%%%
%%% Summary 
%%%
%\begin{frame}
%\frametitle{Overview} % Table of contents slide, comment this block out to remove it
%\tableofcontents % Throughout your presentation, if you choose to use \section{} and \subsection{} commands, these will automatically be printed on this slide as an overview of your presentation
%\end{frame}

%%%%%%%%%%%%%%%%%%%% The Presentation Proper %%%%%%%%%%%%%%%%%%%%%%%%%
% Fill below this line with \begin{frame} commands! It's best to
% always add the fragile option incase you're going to use the
% verbatim environment.
%%%%%%%%%%%%%%%%%%%%%%%%%%%%%%%%%%%%%%%%%%%%%%%%%%%%%%%%%%%%%%%%%%%%%%


%%%           %%%
%%%  SECTION  %%% 
%%%           %%%
\section{Introduction}

% SUBSECTION
 \subsection{Aims and Objectives}
%%%
%%% Slide
%%%
   \begin{frame}
     \frametitle{Aims and Objectives}
     \begin{enumerate}[(i)]
       \item <1-> XXXXXX;
 \end{enumerate}
   \end{frame}


% SECTION
 \subsection{Bibliography} 
%%%
%%% Slide
%%%
   \begin{frame}
     \frametitle{Suggested References}\scriptsize
       Literature relevant for this module:
     \begin{enumerate}[(a)]\scriptsize
       \item E. Barbier (2002) $\lq$Geothermal Energy Technology and Current Status: An Overview', Renewable $\&$ Sustainable Energy Reviews 6:3-65;
       \item H.N. Pollack, S.J. Hurter, J.R. Johnson (1993) $\lq$Heat Flow from the Earth's Interior: Analysis of the Global Data Set', Reviews of Geophysics 31:267-280;
       \item G.S. Bodvarsson, P.A. Witherspoon (1989) $\lq$Geothermal Reservoir Engineering Part 1', Geotherm. Science and Technology 2:1-68;
       \item H.K. Gupta (1980) $\lq$Geothermal Resources: An Energy Alternative', {\it In} Developments in Economic Geology 12, Chapters 3-5;
       \item K. Pruess (2002) $\lq$Mathematical Modelling of Fluid Flow and Heat Transfer in Geothermal Systems -- An Introduction in Five Lectures', United Nations University;
       \item L. Georgsson (2010) $\lq$Geophysical Methods used in Geothermal Exploration', Short Course V on Exploration for Geothermal Resources, Kenya;
       \item Documentation in \href{http://en.openei.org/wiki/Geothermal_Exploration_Best_Practices:_A_Guide_to_Resource_Data_Collection,_Analysis_and_Presentation_for_Geothermal_Projects}{OpenEI Report: Geothermal Exploration Best Practices: A Guide to Resource Data Collection, Analysis and Presentation for Geothermal Projects (2013)};
       \item Z. Chen (2006) $\lq$Computational Methods for Multiphase Flows in Porous Media', SIAM, Chapters: 1-3, 11-13;
       \item J. Finger, D. Blankenship (2010) $\lq$Handbook of Best Practices for Geothermal Drilling', Sandia National Laboratories Report, \href{http://www1.eere.energy.gov/geothermal/pdfs/drillinghandbook.pdf}{SAND2010-6048}.
     \end{enumerate}
\end{frame}


%%%           %%%
%%%  SECTION  %%% 
%%%           %%%
 \section{Geothermal Energy Science}






%%%           %%%
%%%  SECTION  %%% 
%%%           %%%
 \section{Geothermal Project Development} % From IFC-IGA_Geothermal_Exploration_Best_Practices-March2013

%%%
%%% Slide
%%%
\begin{frame}
 \frametitle{Phases for Geothermal Project Developments} 
    There is usually 7 phases for any \blue{geothermal} (or subsurface) \blue{project development}: 
    \begin{enumerate}[1.]
       \item <1-> \blue{Preliminary Survey;}
       \item <1-> Exploration; 
       \item <1-> Test Drilling;
       \item <1-> Project Review and Planning;
       \item <1-> Field Development; 
       \item <1-> Power Plant Construction;
       \item <1-> Commissioning and Operation;
    \end{enumerate}
\end{frame}

%%%
%%% Slide
%%%
\begin{frame}
 \frametitle{Phases for Geothermal Project Developments: \\1. Preliminary Survey} 

    Initial assessment (nationally and/or internationally) of the available information on:
    \begin{enumerate}[{1.}1]
       \item <1-> Finances, Economics and Legal issues:
          \begin{enumerate}[(a)]
             \item<2-> Power (electricity) market (and/or power purchase agreements, PPA, and/or feed in tariff);
             \item<2-> Resources ownership (i.e., land/water right, mining laws etc);
             \item<2-> Institutional and regulatory framework;
          \end{enumerate}
       \item <3-> Environmental regulations; 
       \item <4-> Infrastructure (e.g., roads, water, communication, electrical transmission).
    \end{enumerate}
\end{frame}

%%%
%%% Slide
%%%
\begin{frame}
 \frametitle{Phases for Geothermal Project Developments} 
    \begin{enumerate}[1.]
       \item <1-> Preliminary Survey;
       \item <2-> \blue{Exploration}; 
       \item <1-> Test Drilling;
       \item <1-> Project Review and Planning;
       \item <1-> Field Development; 
       \item <1-> Power Plant Construction;
       \item <1-> Commissioning and Operation;
    \end{enumerate}
\end{frame}

%%%
%%% Slide
%%%
\begin{frame}
 \frametitle{Phases for Geothermal Project Developments: \\ 2. Exploration} 

    \begin{enumerate}[{2.}1]\scriptsize
       \item <1-> The \blue{exploration phase} aims to minimise risks related to resource temperature, depth, productivity and sustainability before drilling assessment;
       \item <2-> \blue{Exploration} usually starts by collecting data from existing surrounding surface, subsurface and wells sensors through geological. geochemical and geophysical methods.
       \item <3-> Usual \blue{exploration} surveys include:
          \begin{enumerate}[(a)]\scriptsize
             \item<4-> Geochemical assessment
                \begin{enumerate}[i)]\scriptsize
                   \item<4-> Geothermometry;
                   \item<4-> Electrical conductivity;
                   \item<4-> pH;
                   \item<4-> Fluid flow rate;
                   \item<4-> Soil morphology.
                \end{enumerate}
             \item<5-> Geophysical assessment
                \begin{enumerate}[i)]\scriptsize
                   \item<5-> Gravity;
                   \item<5-> Electrical resistivity;
                   \item<5-> Temperature gradient drilling;
                   \item<5-> 2D and 3D seismics.
                \end{enumerate}
             \item<6-> Surface assessment
                \begin{enumerate}[i)]\scriptsize
                   \item<6-> Surface geology;
                   \item<6-> Geothermal surface features.
                \end{enumerate}
             \item<7-> etc.
          \end{enumerate}
       \item <8-> With this info, models (conceptual and numerical) are built to support the \blue{pre-feasibility study}. 
    \end{enumerate}
\end{frame}
 

%%%
%%% Slide
%%%
\begin{frame}
 \frametitle{Phases for Geothermal Project Developments} 
    \begin{enumerate}[1.]
       \item <1-> Preliminary Survey;
       \item <1-> Exploration; 
       \item <2-> \blue{Test Drilling};
       \item <1-> Project Review and Planning;
       \item <1-> Field Development; 
       \item <1-> Power Plant Construction;
       \item <1-> Commissioning and Operation;
    \end{enumerate}
\end{frame}


%%%
%%% Slide
%%%
\begin{frame}
 \frametitle{Phases for Geothermal Project Developments: \\ 3. Test Drilling}
  \begin{columns}
    \begin{column}[l]{0.45\linewidth}     
       \begin{enumerate}[{3.}1]\scriptsize
          \item <1-> Usually 2-3 wells are drilled to demonstrate the commercial feasibility of production and injection;
          \item <2-> Drilling, logging and testing will improve the understanding of the subsurface resources enabling,  
          \begin{enumerate}[(a)]\scriptsize
             \item<3-> Further estimate of heat resource;
             \item<4-> Determination of the average well productivity;
             \item<5-> Selection of the wells sites, targets, well path and design for the remaining production and injection wells;
             \item<6-> Development of a \blue{preliminary design for the power plant}.
          \end{enumerate}
       \end{enumerate}
       %\begin{center}
           \includegraphics[width=5.6cm,height=2.7cm,clip]{./Pics/Uncased-Drilling-1.jpg}
       %\end{center}
    \end{column}
    \begin{column}[l]{0.55\linewidth} 
       \vbox{
          \hbox{
             \includegraphics[width=0.5\columnwidth,clip]{./Pics/Geothermal_drilling_at_Te_Mihi_NZ.jpg}
             \includegraphics[width=0.5\columnwidth,clip]{./Pics/Drilling.jpg}
          }
          \hbox{
             \includegraphics[width=0.35\columnwidth,clip]{./Pics/DTRC_Diagram.jpg}\hspace{.53cm}
             \includegraphics[width=0.55\columnwidth,clip]{./Pics/web_Drillpipe_injection_well_profile-copy.jpg}
          }
       }
    \end{column}
  \end{columns}
\end{frame}
 

 

%%%
%%% Slide
%%%
\begin{frame}
 \frametitle{Phases for Geothermal Project Developments} 
    \begin{enumerate}[1.]
       \item <1-> Preliminary Survey;
       \item <1-> Exploration; 
       \item <1-> Test Drilling;
       \item <2-> \blue{Project Review and Planning};
       \item <1-> Field Development; 
       \item <1-> Power Plant Construction;
       \item <1-> Commissioning and Operation;
    \end{enumerate}
\end{frame}


%%%
%%% Slide
%%%
\begin{frame}
 \frametitle{Phases for Geothermal Project Developments: \\ 4. Project Review and Planning}
  \begin{columns}
    \begin{column}[l]{0.45\linewidth}     
       \begin{enumerate}[{3.}1]\scriptsize
          \item <1-> Usually 2-3 wells are drilled to demonstrate the commercial feasibility of production and injection;
          \item <2-> Drilling, logging and testing will improve the understanding of the subsurface resources enabling,  
          \begin{enumerate}[(a)]\scriptsize
             \item<3-> Further estimate of heat resource;
             \item<4-> Determination of the average well productivity;
             \item<5-> Selection of the wells sites, targets, well path and design for the remaining production and injection wells;
             \item<6-> Development of a \blue{preliminary design for the power plant}.
          \end{enumerate}
       \end{enumerate}
       %\begin{center}
           \includegraphics[width=5.6cm,height=2.7cm,clip]{./Pics/Uncased-Drilling-1.jpg}
       %\end{center}
    \end{column}
    \begin{column}[l]{0.55\linewidth} 
       \vbox{
          \hbox{
             \includegraphics[width=0.5\columnwidth,clip]{./Pics/Geothermal_drilling_at_Te_Mihi_NZ.jpg}
             \includegraphics[width=0.5\columnwidth,clip]{./Pics/Drilling.jpg}
          }
          \hbox{
             \includegraphics[width=0.35\columnwidth,clip]{./Pics/DTRC_Diagram.jpg}\hspace{.53cm}
             \includegraphics[width=0.55\columnwidth,clip]{./Pics/web_Drillpipe_injection_well_profile-copy.jpg}
          }
       }
    \end{column}
  \end{columns}
\end{frame}
 


%%%           %%%
%%%  SECTION  %%% 
%%%           %%%
\section{Summary}
%%%
%%% Slide
%%%
\begin{frame}
 \frametitle{Summary}
  \begin{enumerate}
%
     \item <1-> Components 
%
  \end{enumerate}
\end{frame}


\end{document}
