% Aberdeen style guide should be followed when using this
% layout. Their template powerpoint slide is used to extract the
% Aberdeen color and logo but is otherwise ignored (it has little or
% no formatting in it anyway).   
% 
% http://www.abdn.ac.uk/documents/style-guide.pdf

%%%%%%%%%%%%%%%%%%%% Document Class Settings %%%%%%%%%%%%%%%%%%%%%%%%%
% Pick if you want slides, or draft slides (no animations)
%%%%%%%%%%%%%%%%%%%%%%%%%%%%%%%%%%%%%%%%%%%%%%%%%%%%%%%%%%%%%%%%%%%%%%
%Normal document mode%
\documentclass[10pt,compress,unknownkeysallowed]{beamer}
%Draft or handout mode 
%\documentclass[10pt,compress,handout,unknownkeysallowed]{beamer}
%\documentclass[10pt,compress,handout,ignorenonframetext,unknownkeysallowed]{beamer}

%%%%%%%%%%%%%%%%%%%% General Document settings %%%%%%%%%%%%%%%%%%%%%%%
% These settings must be set for each presentation
%%%%%%%%%%%%%%%%%%%%%%%%%%%%%%%%%%%%%%%%%%%%%%%%%%%%%%%%%%%%%%%%%%%%%%
\newcommand{\shortname}{jefferson.gomes@abdn.ac.uk}
\newcommand{\fullname}{Dr Jeff Gomes}
\institute{School of Engineering}
\newcommand{\emailaddress}{}%jefferson.gomes@abdn.ac.uk}
\newcommand{\logoimage}{../FigBanner/UoAHorizBanner}
\title{Renewable Energy 1: Solar and Geothermal (EG50M2/EG504C)}
\subtitle{Module 3: Introduction to Porous Media Fluid Flow $\&$ Heat Transfer in Geothermal Systems}
\date[ ]{ }



%%%%%%%%%%%%%%%%%%%%%%%%%%%%%%%%%%%%%%%%%%%%%%%%%%%%%%%%%%%%%%%%%%%%%%%%%%%%%%%
% BABEL and LANGUAGES %%%%%%%%%%%%%%%%%%%%%%%%%%%%%%%%%%%%%%%%%%%%%%%%%%%%%%%%%
%%%%%%%%%%%%%%%%%%%%%%%%%%%%%%%%%%%%%%%%%%%%%%%%%%%%%%%%%%%%%%%%%%%%%%%%%%%%%%%
% \usepackage{listings}                   % it is a source code printer for LATEX
                                          % \lstset{language=Python}
                                          % \lstinputlisting{source.py}   % command used to pretty-print stand alone files
\usepackage[english]{babel}               % [french, frenchb, english, ]
    % http://forum.mathematex.net/latex-f6/les-puces-avec-babel-t4256.html
    % http://www.grappa.univ-lille3.fr/FAQ-LaTeX/11.1.html


%%%%%%%%%%%%%%%%%%%%%%%%%%%%%%%%%%%%%%%%%%%%%%%%%%%%%%%%%%%%%%%%%%%%%%%%%%%%%%%
% FONTS and ENCODING %%%%%%%%%%%%%%%%%%%%%%%%%%%%%%%%%%%%%%%%%%%%%%%%%%%%%%%%%%
%%%%%%%%%%%%%%%%%%%%%%%%%%%%%%%%%%%%%%%%%%%%%%%%%%%%%%%%%%%%%%%%%%%%%%%%%%%%%%%
%
% See:
% http://tex.stackexchange.com/questions/59702/suggest-a-nice-font-family-for-my-basic-latex-template-text-and-math-i-am
%

\usepackage{lmodern}        % Latin Modern family of fonts. Very much like Computer Modern, but with many more glyphs 
                            % (e.g., for characters with accents, glyphs, cedillas, etc)
\usepackage[T1]{fontenc}    % fontenc is oriented to output, that is, what fonts to use for printing characters. 
                            % http://tex.stackexchange.com/questions/44694/fontenc-vs-inputenc 
                            % http://tex.stackexchange.com/questions/664/why-should-i-use-usepackaget1fontenc

% Change some fonts or the whole font family (i.e. serif, sans serif, monospace, and 'math')
    % \usepackage[varg, cmintegrals, cmbraces, ]{newtxtext,newtxmath}  % Other options: libertine, uprightGreek (U.S.) or slantedGreek (ISO), etc...
     \usepackage{tgtermes}                                            % Only serif ("TeX-Gyre" text)
    % \usepackage{kpfonts}                                             % "Kepler" fonts
    % \usepackage{mathpazo}                                            % Based on Hermann Zapf's Palatino font
    % \usepackage{txfonts}                                             % More than a decade old
    % \usepackage{pslatex}                                             % Obsolete?
    %  - \usepackage{mathptmx}
    %  - \usepackage[scaled=.90]{helvet}
    %  - \usepackage{courier}

% \usepackage{textcomp}     % required for special glyphs
% \usepackage{bm}           % load after all math to give access to bold math
\usepackage[utf8]{inputenc} % inputenc allows the user to input accented characters directly from the keyboard; 
                            % utf8x : much broader but less compatible ; latin1 : old?
                            % http://tex.stackexchange.com/questions/44694/fontenc-vs-inputenc

% See:
% http://tex.stackexchange.com/questions/59626/nicely-force-66-characters-per-line
%
% pslatex is a very obsolete package and that its descendant mathptmx is rather inadequate for serious typesetting involving math.
% If you don't need mathematics, other choices based on (Linotype) Times Roman are
%  - tgtermes
%  - newtxtext (based on txfonts, but with corrected metrics) (with its companion math package newtxmath)
%
%
% See:
% http://www.latex-community.org/forum/viewtopic.php?f=8&t=6637
%
% (times, helvet, courier)
% pslatex and txfonts produce (almost) same resutls.
% pslatex supposedly obsolete
% txfonts supposedly up-to-date
%
%
% See:
% ftp://ftp.rrzn.uni-hannover.de/pub/mirror/tex-archive/info/l2tabu/english/l2tabuen.pdf
% or 
% ftp://ftp.dante.de/tex-archive/info/l2tabu/english/l2tabuen.pdf
% in
% 2.3.3 pslatex.sty
%
% pslatex uses a Courier font scaled too narrowly.
% Its main disadvantage is that it does not work with T1 and TS1 encodings.
% So replace:
% \usepackage{pslatex} or \usepackage{txfonts}
% by all three:
% - \usepackage{mathptmx}
% - \usepackage[scaled=.90]{helvet}
% - \usepackage{courier}
%
%
% See:
% http://xpt.sourceforge.net/techdocs/language/latex/latex32-LaTeXAndFonts/single/
% or http://thirteen-01.stat.iastate.edu/wiki/LaTeXFonts
% or http://www.tex.ac.uk/tex-archive/info/beginlatex/html/chapter8.html
%
% When changing fonts, you can change all of the default fonts at once with the following commands:
% 
% Command     Changes the defaults to
% 
% times       Times, Helvetica, Courier
% pslatex     same as Times, but uses a specially narrowed Courier. This is preferred over Times because of the way it handles Courier.
% newcent     New Century Schoolbook, Avant Garde, Courier
% palatino    Palatino, Helevetica, Courier
% palatcm     changes the Roman to Palatino only, but uses CM mathematics
% kpfonts     "Kepler" fonts. A very nicely evolved set of fonts also based originally on Palatino, but with many special features.
%
%
% See:
% http://tex.stackexchange.com/questions/59702/suggest-a-nice-font-family-for-my-basic-latex-template-text-and-math-i-am
%
% There are, of course, many other font packages, most of which provide "only" text-mode fonts.
% Among these are the "TeX-Gyre" font families: 
%  - Termes (a Times Roman clone), 
%  - Pagella (a Palatino clone), and 
%  - Schola (a Century Schoolbook clone); 
% one would load the packages tgtermes, tgpagella, and tgschola, respectively, to access these fonts.
% However, as these are text fonts, you still need to choose a suitable math font.
% 
% Still another possibility you may want to look into is the Linux Libertine font family, to be loaded via the libertine-legacy package.
% If you like this text font and wish to employ the newtxmath package, be sure to load the newtxmath package with the libertine option set;
% doing so will set up a special set of math-mode fonts that harmonizes well with the libertine text fonts.
% 
%
% See also:
% http://tex.stackexchange.com/questions/56876/times-new-roman-fonts-and-maths-without-mathptmx
%
%
% For a comparison, in:
% /home/christophe/Personal/Truc_Et_Astuce_Informatik/LaTeX/comparison_font_types/,
% see: 
% computer.pdf  lmodern.pdf  pslatex.pdf  test_font_type.pdf  three_replacements.pdf  txfonts.pdf
%


%%%%%%%%%%%%%%%%%%%%%%%%%%%%%%%%%%%%%%%%%%%%%%%%%%%%%%%%%%%%%%%%%%%%%%%%%%%%%%%
% AMS MATH %%%%%%%%%%%%%%%%%%%%%%%%%%%%%%%%%%%%%%%%%%%%%%%%%%%%%%%%%%%%%%%%%%%%
%%%%%%%%%%%%%%%%%%%%%%%%%%%%%%%%%%%%%%%%%%%%%%%%%%%%%%%%%%%%%%%%%%%%%%%%%%%%%%%
% \usepackage{amsmath}      % loads amstext, amsbsy, amsopn but not amssymb
                            % equation stuff (eqref, subequations, equation, align, gather, flalign, multline, alignat, split...)
% \usepackage{amsfonts}     % may be redundant with amsmath
% \usepackage{amssymb}      % may be redundant with amsmath
% \numberwithin{equation}{section}  % reset equation counters at start of each "section" and prefix numbers by section number
% \numberwithin{figure}{section}    % reset figure   counters at start of each "section" and prefix numbers by section number


%%%%%%%%%%%%%%%%%%%%%%%%%%%%%%%%%%%%%%%%%%%%%%%%%%%%%%%%%%%%%%%%%%%%%%%%%%%%%%%
% LAY OUT %%%%%%%%%%%%%%%%%%%%%%%%%%%%%%%%%%%%%%%%%%%%%%%%%%%%%%%%%%%%%%%%%%%%%
%%%%%%%%%%%%%%%%%%%%%%%%%%%%%%%%%%%%%%%%%%%%%%%%%%%%%%%%%%%%%%%%%%%%%%%%%%%%%%%
%
% See:
% http://tex.stackexchange.com/questions/59626/nicely-force-66-characters-per-line
% (must be after pslatex, tgterms, etc...)
%
% a) (but works mostly for a4paper, and changes top and bottom margin too...)
% \usepackage[DIV=calc]{typearea}
%
% or
%
% b) (but you have to choose the value and the margin ratio depending on the class...)
% \newlength{\alphabet}
% \settowidth{\alphabet}{\normalfont abcdefghijklmnopqrstuvwxyz}
% \usepackage{geometry}
% \geometry{%
% textwidth=2.5\alphabet,% (Note: 2.5 * 26 = 65)
% hmarginratio={2:3}}    % (Problem: geometry uses 2:3 as default for twoside and 1:1 for oneside,
%                        % independently of what the class thinks about the margins)

% \usepackage{layout}       % use \layout in the tex file to see the values
% \usepackage{layouts}      % it extends the functionality of layout, allowing you to do much, much more
                            % some commands: \pagelayout, \pagevalues, \pagedesign, ...
% \usepackage[cm]{fullpage} % set 'default' full page
% \usepackage{geometry}     % very customizable margins. Under some (rare) circumstances, should be loaded after hyperref
% \usepackage{anysize}      % \marginsize{left}{right}{top}{bottom}
% \usepackage{pdflscape}    % include landscape layout pages (automatically rotate pages in pdf file for easier reading)
% \usepackage{multicol}     % for multi column environment
\usepackage{lipsum}         % to fill in with arbitrary text
\widowpenalty = 4000        % help suppress widows,  default = 4,000 (?), from 0 to 10 000 (from 300 to 1 000 recommended, 10 000 not recommended)
\clubpenalty  = 4000        % help suppress orphans, default = 4,000 (?), from 0 to 10 000 (from 300 to 1 000 recommended, 10 000 not recommended)
\usepackage[final, babel]{microtype} % many good lay-out/justification effects, see:
                                     % texblog.net/latex-archive/layout/pdflatex-microtype/


%%%%%%%%%%%%%%%%%%%%%%%%%%%%%%%%%%%%%%%%%%%%%%%%%%%%%%%%%%%%%%%%%%%%%%%%%%%%%%%
% EMBED FILEs %%%%%%%%%%%%%%%%%%%%%%%%%%%%%%%%%%%%%%%%%%%%%%%%%%%%%%%%%%%%%%%%%
%%%%%%%%%%%%%%%%%%%%%%%%%%%%%%%%%%%%%%%%%%%%%%%%%%%%%%%%%%%%%%%%%%%%%%%%%%%%%%%
\usepackage{embedfile}    % embed (attach) any files (eg tex source) to a PDF document.
                          % Currently only supported driver is pdfTEX >= 1.30 in PDF mode
%\embedfile{to_post.tex}


%%%%%%%%%%%%%%%%%%%%%%%%%%%%%%%%%%%%%%%%%%%%%%%%%%%%%%%%%%%%%%%%%%%%%%%%%%%%%%%
% EASY EDITS %%%%%%%%%%%%%%%%%%%%%%%%%%%%%%%%%%%%%%%%%%%%%%%%%%%%%%%%%%%%%%%%%%
%%%%%%%%%%%%%%%%%%%%%%%%%%%%%%%%%%%%%%%%%%%%%%%%%%%%%%%%%%%%%%%%%%%%%%%%%%%%%%%
\usepackage{ifdraft}        % ask for selective behavior depending on the draft option (used for waterdraftmark, not draftmark)
% \usepackage{comment}      % provide new {comment} environment: all text inside the environment is ignored.
% \usepackage{fixme}        % allow nice comment / warning system, displayed in draft mode in right margin ; % [status=draft]
% \usepackage{lineno}       % number all lines in left margin if activated with \linenumbers
% \linenumbers


%%%%%%%%%%%%%%%%%%%%%%%%%%%%%%%%%%%%%%%%%%%%%%%%%%%%%%%%%%%%%%%%%%%%%%%%%%%%%%%
% GRAPHICX %%%%%%%%%%%%%%%%%%%%%%%%%%%%%%%%%%%%%%%%%%%%%%%%%%%%%%%%%%%%%%%%%%%%
%%%%%%%%%%%%%%%%%%%%%%%%%%%%%%%%%%%%%%%%%%%%%%%%%%%%%%%%%%%%%%%%%%%%%%%%%%%%%%%
% \usepackage[final]{graphicx} % options = [final]  = all graphics displayed, regardless of draft option in class
                               % options = [pdftex] = necessary (?) if import PDF files
                               % no option : when importing ps- and eps-files (?)
% \graphicspath{{../images/}}  % tell LaTeX where to look for images
% \DeclareGraphicsExtensions{.pdf, .PDF, .jpg, .JPG, .jpeg, .JPEG, .png, .PNG, .bmp, .BMP, .eps, .ps}
\usepackage{float}                      % Improved interface for floating objects ; add [H] option


%%%%%%%%%%%%%%%%%%%%%%%%%%%%%%%%%%%%%%%%%%%%%%%%%%%%%%%%%%%%%%%%%%%%%%%%%%%%%%%
% FILIGREE %%%%%%%%%%%%%%%%%%%%%%%%%%%%%%%%%%%%%%%%%%%%%%%%%%%%%%%%%%%%%%%%%%%%
%%%%%%%%%%%%%%%%%%%%%%%%%%%%%%%%%%%%%%%%%%%%%%%%%%%%%%%%%%%%%%%%%%%%%%%%%%%%%%%
% draftmark : newer and better package but not on Phil's computers,
% in particular, draftmark has a "ifdraft" option included...
%
\ifdraft{
\usepackage{draftwatermark} % add watermark ("draft", "confidential"...)
                            % option: [firstpage] (insert on only the first page)
\SetWatermarkText{COPY~---~DRAFT}
\SetWatermarkAngle{55}
\SetWatermarkScale{6.0}
\SetWatermarkLightness{0.85}
\SetWatermarkFontSize{12 pt}
}{}


\renewcommand{\insertframenumber}{\theframenumber}
\renewcommand{\theframenumber}{\thesection-\arabic{framenumber}}
\renewcommand{\thesubsectionslide}{\thesection-\arabic{framenumber}}
\setbeamertemplate{headline}[text line]{This is frame: \insertframenumber}
\AtBeginSection{\setcounter{framenumber}{0}}


%%%%%%%%%%%%%%%%%%%% Template settings %%%%%%%%%%%%%%%%%%%%%%%%%%%%%%%
% You shouldn't have to change below this line, unless you want to.
%%%%%%%%%%%%%%%%%%%%%%%%%%%%%%%%%%%%%%%%%%%%%%%%%%%%%%%%%%%%%%%%%%%%%%
\usecolortheme{whale}
\useoutertheme{infolines}

% Use the fading effect for items that are covered on the current
% slide.
\beamertemplatetransparentcovered

% We abuse the author command to place all of the slide information on
% the title page.
\author[\shortname]{%
  \fullname\\\ttfamily{\emailaddress}
}


%At the start of every section, put a slide indicating the contents of the current section.
\AtBeginSection[] {
  \begin{frame}
    \frametitle{Section Outline}
    \tableofcontents[currentsection]
  \end{frame}
}

% Allow the inclusion of movies into the Presentation! At present,
% only the Okular program is capable of playing the movies *IN* the
% presentation.
\usepackage{multimedia}
\usepackage{animate}

%% Handsout -- comment out the lines below to create handstout with 4 slides in a page with space for comments
\usepackage{handoutWithNotes}

\mode<handout>
{
\usepackage{pgf,pgfpages}

\pgfpagesdeclarelayout{2 on 1 boxed with notes}
{
\edef\pgfpageoptionheight{\the\paperheight} 
\edef\pgfpageoptionwidth{\the\paperwidth}
\edef\pgfpageoptionborder{0pt}
}
{
\setkeys{pgfpagesuselayoutoption}{landscape}
\pgfpagesphysicalpageoptions
    {%
        logical pages=4,%
        physical height=\pgfpageoptionheight,%
        physical width=\pgfpageoptionwidth,%
        last logical shipout=2%
    } 
\pgfpageslogicalpageoptions{1}
    {%
    border code=\pgfsetlinewidth{1pt}\pgfstroke,%
    scale=1,
    center=\pgfpoint{.25\pgfphysicalwidth}{.75\pgfphysicalheight}%
    }%
\pgfpageslogicalpageoptions{2}
    {%
    border code=\pgfsetlinewidth{1pt}\pgfstroke,%
    scale=1,
    center=\pgfpoint{.25\pgfphysicalwidth}{.25\pgfphysicalheight}%
    }%
\pgfpageslogicalpageoptions{3}
    {%
    border shrink=\pgfpageoptionborder,%
    resized width=.7\pgfphysicalwidth,%
    resized height=.5\pgfphysicalheight,%
    center=\pgfpoint{.75\pgfphysicalwidth}{.29\pgfphysicalheight},%
    copy from=3
    }%
\pgfpageslogicalpageoptions{4}
    {%
    border shrink=\pgfpageoptionborder,%
    resized width=.7\pgfphysicalwidth,%
    resized height=.5\pgfphysicalheight,%
    center=\pgfpoint{.75\pgfphysicalwidth}{.79\pgfphysicalheight},%
    copy from=4
    }%

\AtBeginDocument
    {
    \newbox\notesbox
    \setbox\notesbox=\vbox
        {
            \hsize=\paperwidth
            \vskip-1in\hskip-1in\vbox
            {
                \vskip1cm
                Notes\vskip1cm
                        \hrule width\paperwidth\vskip1cm
                    \hrule width\paperwidth\vskip1cm
                        \hrule width\paperwidth\vskip1cm
                    \hrule width\paperwidth\vskip1cm
                        \hrule width\paperwidth\vskip1cm
                    \hrule width\paperwidth\vskip1cm
                    \hrule width\paperwidth\vskip1cm
                    \hrule width\paperwidth\vskip1cm
                        \hrule width\paperwidth
            }
        }
        \pgfpagesshipoutlogicalpage{3}\copy\notesbox
        \pgfpagesshipoutlogicalpage{4}\copy\notesbox
    }
}
}

%\pgfpagesuselayout{2 on 1 boxed with notes}[letterpaper,border shrink=5mm]
%\pgfpagesuselayout{2 on 1 boxed with notes}[letterpaper,border shrink=5mm]


%%%%%%%%%% Chemical Reactions %%%%%%%%%%%%%%%%

\usepackage[T1]{fontenc}
\usepackage[utf8]{inputenc}
\usepackage{lmodern}
\usepackage[version=3]{mhchem}
\makeatletter
\newcounter{reaction}
%%% >> for article <<
%\renewcommand\thereaction{C\,\arabic{reaction}}
%%% << for article <<
%%% >> for report and book >>
%\renewcommand\thereaction{C\,\thechapter.\arabic{reaction}}
%\@addtoreset{reaction}{chapter}
%%% << for report and book <<
\newcommand\reactiontag{\refstepcounter{reaction}\tag{\thereaction}}
\newcommand\reaction@[2][]{\begin{equation}\ce{#2}%
\ifx\@empty#1\@empty\else\label{#1}\fi%
\reactiontag\end{equation}}
\newcommand\reaction@nonumber[1]{\begin{equation*}\ce{#1}%
\end{equation*}}
\newcommand\reaction{\@ifstar{\reaction@nonumber}{\reaction@}}
\makeatother

%%%%%%%%%%%%%%%%%%%%%%%%%%%%%%%%%%%%%%%%%%%%%%


%%%%% Color settings
\usepackage{color}
%% The background color for code listings (i.e. example programs)
\definecolor{lbcolor}{rgb}{0.9,0.9,0.9}%
\definecolor{UoARed}{rgb}{0.64706, 0.0, 0.12941}
\definecolor{UoALight}{rgb}{0.85, 0.85, 0.85}
\definecolor{UoALighter}{rgb}{0.92, 0.92, 0.92}
\setbeamercolor{structure}{fg=UoARed} % General background and higlight color
\setbeamercolor{frametitle}{bg=black} % General color
\setbeamercolor{frametitle right}{bg=black} % General color
\setbeamercolor{block body}{bg=UoALighter} % For blocks
\setbeamercolor{structure}{bg=UoALight} % For blocks
% Rounded boxes for blocks
\setbeamertemplate{blocks}[rounded]

%%%%% Font settings
% Aberdeen requires the use of Arial in slides. We can use the
% Helvetica font as its widely available like so
% \usepackage{helvet}
% \renewcommand{\familydefault}{\sfdefault}
% But beamer already uses a sans font, so we will stick with that.

% The size of the font used for the code listings.
\newcommand{\goodsize}{\fontsize{6}{7}\selectfont}

% Extra math packages, symbols and colors. If you're using Latex you
% must be using it for formatting the math!
\usepackage{amscd,amssymb} \usepackage{amsfonts}
\usepackage[mathscr]{eucal} \usepackage{mathrsfs}
\usepackage{latexsym} \usepackage{amsmath} \usepackage{bm}
\usepackage{amsthm} \usepackage{textcomp} \usepackage{eurosym}
% This package provides \cancel{a} and \cancelto{a}{b} to "cancel"
% expressions in math.
\usepackage{cancel}

\usepackage{comment} 

% Get rid of font warnings as modern LaTaX installations have scalable
% fonts
\usepackage{type1cm} 

%\usepackage{enumitem} % continuous numbering throughout enumerate commands

% For exact placement of images/text on the cover page
\usepackage[absolute]{textpos}
\setlength{\TPHorizModule}{1mm}%sets the textpos unit
\setlength{\TPVertModule}{\TPHorizModule} 

% Source code formatting package
\usepackage{listings}%
\lstset{ backgroundcolor=\color{lbcolor}, tabsize=4,
  numberstyle=\tiny, rulecolor=, language=C++, basicstyle=\goodsize,
  upquote=true, aboveskip={1.5\baselineskip}, columns=fixed,
  showstringspaces=false, extendedchars=true, breaklines=false,
  prebreak = \raisebox{0ex}[0ex][0ex]{\ensuremath{\hookleftarrow}},
  frame=single, showtabs=false, showspaces=false,
  showstringspaces=false, identifierstyle=\ttfamily,
  keywordstyle=\color[rgb]{0,0,1},
  commentstyle=\color[rgb]{0.133,0.545,0.133},
  stringstyle=\color[rgb]{0.627,0.126,0.941}}

% Allows the inclusion of other PDF's into the final PDF. Great for
% attaching tutorial sheets etc.
\usepackage{pdfpages}
\setbeamercolor{background canvas}{bg=}  

% Remove foot note horizontal rules, they occupy too much space on the slide
\renewcommand{\footnoterule}{}

% Force the driver to fix the colors on PDF's which include mixed
% colorspaces and transparency.
\pdfpageattr {/Group << /S /Transparency /I true /CS /DeviceRGB>>}

% Include a graphics, reserve space for it but
% show it on the next frame.
% Parameters:
% #1 Which slide you want it on
% #2 Previous slides
% #3 Options to \includegraphics (optional)
% #4 Name of graphic
\newcommand{\reserveandshow}[4]{%
\phantom{\includegraphics<#2|handout:0>[#3]{#4}}%
\includegraphics<#1>[#3]{#4}%
}

\newcommand{\frc}{\displaystyle\frac}
\newcommand{\red}{\textcolor{red}}
\newcommand{\blue}{\textcolor{blue}}
\newcommand{\green}{\textcolor{green}}
\newcommand{\purple}{\textcolor{purple}}
\newcommand{\eg}{{\it e.g., }}
\newcommand{\ie}{{\it i.e., }}
\newcommand{\wrt}{{\it wrt }}
\newcommand{\Partial}[3][error]{\left(\frc{\partial #1}{\partial #2}\right)_{#3}}
\newcommand{\mfr}[3][error]{#1_{#2}^{\left(#3\right)}} 
\newcommand{\summation}[3][error]{\sum\limits_{#2}^{#3}#1}

 
\begin{document}

% Title page layout
\begin{frame}
  \titlepage
  \vfill%
  \begin{center}
    \includegraphics[clip,width=0.8\textwidth]{\logoimage}
  \end{center}
\end{frame}

% Table of contents
\frame{ \frametitle{Slides Outline}
  \tableofcontents
}


%%%%%%%%%%%%%%%%%%%% The Presentation Proper %%%%%%%%%%%%%%%%%%%%%%%%%
% Fill below this line with \begin{frame} commands! It's best to
% always add the fragile option incase you're going to use the
% verbatim environment.
%%%%%%%%%%%%%%%%%%%%%%%%%%%%%%%%%%%%%%%%%%%%%%%%%%%%%%%%%%%%%%%%%%%%%%


%%%           %%%
%%%  SECTION  %%% 
%%%           %%%
\section{Introduction}

% SUBSECTION
 \subsection{Aims and Objectives}
%%%
%%% Slide
%%%
   \begin{frame}
     \frametitle{Aims and Objectives} 
     \begin{enumerate}[A.]%\scriptsize
       \item <1-> In {\bf Module 1}, we introduced the role of geothermal power in the world's energy mix and;
       \item <2-> The different geothermal-based power plant configurations;
       \item <3-> {\bf Module 2} summarised thermodynamic cycles that represent the heat and power conversion in geothermal plants;
       \item <4-> In this \blue{Module}, we will focus on subsurface dynamics;
       \item <5-> Thus, the aims of this Module are:
          \begin{enumerate}[i)]%\scriptsize
               \item <6-> Identify fluid flow and heat transfer mechanisms from the mantle to geological formations, involving injection and production of geothermal fluids used in the thermodynamic cycles;
               \item <7-> Identify the main stages of geothermal project developments.
          \end{enumerate} 
 \end{enumerate}
   \end{frame}

% SECTION
 \subsection{Bibliography} 
%%%
%%% Slide
%%%
   \begin{frame}
     \frametitle{Suggested References}\scriptsize
       Literature relevant for this module:
     \begin{enumerate}[(a)]\scriptsize
       \item E. Barbier (2002) $\lq$Geothermal Energy Technology and Current Status: An Overview', Renewable $\&$ Sustainable Energy Reviews 6:3-65;
       \item H.N. Pollack, S.J. Hurter, J.R. Johnson (1993) $\lq$Heat Flow from the Earth's Interior: Analysis of the Global Data Set', Reviews of Geophysics 31:267-280;
       \item G.S. Bodvarsson, P.A. Witherspoon (1989) $\lq$Geothermal Reservoir Engineering Part 1', Geotherm. Science and Technology 2:1-68;
       \item H.K. Gupta (1980) $\lq$Geothermal Resources: An Energy Alternative', {\it In} Developments in Economic Geology 12, Chapters 3-5;
       \item K. Pruess (2002) $\lq$Mathematical Modelling of Fluid Flow and Heat Transfer in Geothermal Systems -- An Introduction in Five Lectures', United Nations University;
       \item L. Georgsson (2010) $\lq$Geophysical Methods used in Geothermal Exploration', Short Course V on Exploration for Geothermal Resources, Kenya;
       \item Documentation in \href{http://en.openei.org/wiki/Geothermal_Exploration_Best_Practices:_A_Guide_to_Resource_Data_Collection,_Analysis_and_Presentation_for_Geothermal_Projects}{OpenEI Report: Geothermal Exploration Best Practices: A Guide to Resource Data Collection, Analysis and Presentation for Geothermal Projects (2013)};
       \item Z. Chen (2006) $\lq$Computational Methods for Multiphase Flows in Porous Media', SIAM, Chapters: 1-3, 11-13;
       \item J. Finger, D. Blankenship (2010) $\lq$Handbook of Best Practices for Geothermal Drilling', Sandia National Laboratories Report, \href{http://www1.eere.energy.gov/geothermal/pdfs/drillinghandbook.pdf}{SAND2010-6048}.
     \end{enumerate}
\end{frame}


%%%           %%%
%%%  SECTION  %%% 
%%%           %%%
 \section{Geothermal Energy Science}

%%% SUBSECTION
\subsection{Natural Geothermal Gradients}

%%%
%%% Slide
%%%
\begin{frame}
 \frametitle{Fourier's Law: Thermal Conduction}
  \begin{columns}
   \begin{column}[c]{0.45\linewidth}
    \begin{enumerate}[1.] \scriptsize
       \item <1-> \blue{Gradient} is a vector operator (symbol: $\nabla$) used to represent the slope of function with respect to a variable, i.e., 
          \begin{displaymath}
             \nabla \phi \left(x, y, z\right) = \left( \frc{\partial \phi}{\partial x}, \frc{\partial \phi}{\partial y}, \frc{\partial \phi}{\partial z}\right)
          \end{displaymath} 
       \item <2-> Temperature rises approximately 2.5$^{\circ}$C at every 100 meters (we may consider negligible oscillations in radial and axial directions),
          \begin{eqnarray}
             \nabla T  &=& \frc{\partial T}{\partial x} = \frc{\Delta T}{\Delta x} \approx \frc{2.5}{100} \frc{^{\circ}\text{C}}{m}  \nonumber \\
                       &=& 0.025 \frc{^{\circ}\text{C}}{m} = 0.025 \frc{K}{m} \nonumber
          \end{eqnarray} 
       \item <3-> This means that to reach \blue{125$^{\circ}$C}, we would need to drill wells at a \blue{depth of 5 km}
       \item <4-> Geothermal gradients can be enhanced through special geologic conditions, e.g., regions where molten rock has raised to shallower depths or;
       \item <4-> In regions with \blue{deep fluid circulation}.
    \end{enumerate}
   \end{column}
   \begin{column}[c]{0.55\linewidth}
     \vbox{
        \hbox{\hspace{1cm}\includegraphics[width=5.cm,clip]{./Pics/geothermal_gradient.png}}
        \hbox{\includegraphics[width=3.5cm,clip]{./Pics/geothermalgradient.jpg}
              \includegraphics[width=3.3cm,clip]{./Pics/geotherms.png}}
     }
   \end{column}  
  \end{columns}
\end{frame}




%%%
%%% Slide
%%%
\begin{frame}
 \frametitle{Fourier's Law: Thermal Conduction}
  \begin{columns}
   \begin{column}[c]{0.5\linewidth}
       \includegraphics[width=\linewidth,clip]{./Pics/geothermal_gradient.png}
   \end{column}
   \begin{column}[c]{0.5\linewidth}
     \vbox{
        \hbox{\includegraphics[width=\linewidth,clip]{./Pics/geothermalgradient.jpg}}
        \hbox{\includegraphics[width=0.8\linewidth,clip]{./Pics/geotherms.png}}
     }
   \end{column}  
  \end{columns}
\end{frame}


%%%
%%% Slide
%%%
\begin{frame}
 \frametitle{Fourier's Law: Thermal Conduction}
      \begin{enumerate}[1.]\setcounter{enumi}{5}% \scriptsize
        \item <1-> The fundamental law of {\it heat conduction} states that heat flows in a \underline{continuous} rock due to temperature differences throughout the medium;
        \item <2-> The transfer of energy occurs from zones of higher temperature to zones of lower temperature ({\it Zeroth Law of Thermodynamics});
        \item <3-> The Fourier's law states that the energy flow (i.e., heat flux, $q$) is proportional to the temperature gradient, $q\propto \nabla T$,
           \visible<3->{\begin{equation}
              q = - \kappa\nabla T
           \end{equation}
           where $\kappa$ is the constant of proportionality and is called \blue{thermal conductivity coefficient};}
        \item <4-> $\kappa \left[\text{in }W.\left(m.K\right)^{-1}\right]$ is a parameter obtained experimentally that may vary in space, i.e., $\kappa = \kappa\left(x, y, z\right)$ and is also a function of temperature -- $\kappa = \kappa\left(x, y, z, T\right)$;
        %\item <5-> For an averaged rock thermal conductivity, $\kappa_{\text{rock}}\approx 2.0\;\frc{W}{m.K}\rightarrow q=-0.05\frc{W}{m^{2}}$;
        %\item <6-> Thus over an area of 1 km$^{2}$, the crustal \blue{heat flow rate} would be of \blue{50 kW};
      \end{enumerate}
\end{frame}





%%%
%%% Slide
%%%
\begin{frame}
 \frametitle{Fourier's Law: Thermal Conduction}
  \begin{columns}
    \begin{column}[c]{0.4\linewidth}
      \begin{enumerate}[1.]\setcounter{enumi}{9}% \scriptsize
        \item <1-> For an averaged rock thermal conductivity, $\kappa_{\text{rock}}\approx 2.0\;\frc{W}{m.K}\rightarrow q=-0.05\frc{W}{m^{2}}$;
        \item <2-> Thus over an area of 1 km$^{2}$, the crustal \blue{heat flow rate} would be of \blue{50 kW};
      \end{enumerate}
    \end{column}
    \begin{column}[c]{0.6\linewidth}
          \href{http://www.springer.com/cda/content/document/cda_downloaddocument/9783642340222-c2.pdf?SGWID=0-0-45-1456221-p174676272}{\hspace{-1.cm}\includegraphics[width=1.2\columnwidth]{./Pics/ListRocksConductivity.png}\\
          \scriptsize From L. Eppelbaum, I. Kutasov, A. Pilchin, Applied Geothermics, Chapter 2.}
    \end{column}  
  \end{columns}
\end{frame} 



%%%
%%% Slide
%%%
\begin{frame}
 \frametitle{Fourier's Law: Thermal Conduction}
    \begin{enumerate}[1.]\setcounter{enumi}{11} \scriptsize
       \item <1-> This is very \underline{small} if compared with common \blue{heat extraction rates} in geothermal fields of \red{10-10$^{2}$ MW};
       \item <2-> Therefore \underline{\blue{Thermal Conduction}} is \red{not the only} heat transfer mechanism.
    \end{enumerate}

    \visible<3->{\begin{center}
       \href{http://www.gly.uga.edu/railsback/PGSG/PGSGmain.html}{\includegraphics[width=8.cm, height=6.5cm,clip]{./Pics/ThermalCond_Geothermal02.jpg}\\
             \scriptsize From L.B. Railsback (Georgia, USA), Petroleum Geoscience and Subsurface Geology.}
    \end{center}}
\end{frame}


%%%
%%% SUBSECTION
%%%
\subsection{Hydrothermal Convection Systems}

%%%
%%% Slide
%%%
\begin{frame}
 \frametitle{Thermal Convection}
    \begin{enumerate}[1.] \scriptsize
       \item <1-> Let's consider a reservoir containing a geothermal fluid of density $\rho$; 
       \item <2-> Under static conditions (i.e., no flow), the pressure is expressed as the weight of the fluid column per unit area, and increases with depth $z$ as,
            \visible<2->{\begin{equation}
                \int\limits_{P_{0}}^{P}dP = \int\limits_{z_{0}}^{z}\rho g dz \Longrightarrow \blue{P(z) = P_{0} + \rho g z},
            \end{equation} 
            where $g$ is the gravity acceleration. Here we assume that pressure does not change in axial and radial directions;}
       \item <3-> Densities of geothermal fluids can be expressed as functions of temperature, pressure (and therefore depth) and salinity $\left(\mathcal{C}_{S}\right)$, i.e., $\rho=\rho\left(T,P,\mathcal{C}_{S}\right)$
       \item <4-> Algebraic expressions that correlate pressure, temperature and specific volume (PvT) are commonly called \underline{\it Equations of State} (EOS);
       \item <5-> An example of EOS that represents the impact of pressure, temperature in the fluid density is, 
            \visible<5->{\begin{equation}
                \rho = \rho_{0}\left[1 - \alpha\left(T-T_{0}\right)+\beta\left(P-P_{0}\right)\right]
            \end{equation}
            where $\alpha$ and $\beta$ are the expansivity and compressibility coefficients, respectively;}
    \end{enumerate}
\end{frame}


%%%
%%% Slide
%%%
\begin{frame}
 \frametitle{Thermal Convection}
  \begin{columns}
   \begin{column}[c]{0.45\linewidth}
    \begin{enumerate}[1.]\setcounter{enumi}{5} \scriptsize
       \item <1-> As warmer fluid (e.g., brine) is less dense than cold fluid $\Longrightarrow$ column of warmer fluid weights less than the cold fluid column;
       \item <2-> \red{Thermal Buoyancy} is a physical phenomena in which cold $\&$ denser fluid displace warmer $\&$ lighter fluid;
       \item <3-> This can be extended to geothermal sources, if we assume that the cold and denser water is the \underline{discharge fluid}, {\it whereas} denser warmer $\&$ lighter fluid represents the conditions in the centre of the field; 
       \item <4-> For \blue{hydrothermal circulation} to occur, we only need a \blue{temperature gradient}. 
    \end{enumerate}
   \end{column}
   \begin{column}[c]{0.55\linewidth}
     \begin{center}
        \includegraphics[width=\columnwidth,clip]{./Pics/GeothermalConvection.png}
     \end{center}
   \end{column}  
  \end{columns}
\end{frame}



%%%
%%% SUBSECTION
%%%
\subsection{Porous Media Flow}

%%%
%%% Slide
%%%
\begin{frame}
 \frametitle{Geothermal Darcy Flows}
  \begin{columns}
   \begin{column}[c]{0.45\linewidth}
    \begin{enumerate}[1.] \scriptsize
       \item <1-> Fluid motion is due to forces, in particular pressure and gravity forces;
       \item <2-> Henry Darcy (1986) established the correlation between fluid flow and forces based on experiments of water flowing through a sandpack;
       \item <3-> The rate of fluid flow depends on \blue{pressure gradient, $\nabla P$},
          \visible<3->{\begin{equation}
             \dot{Q} = - \mathcal{K} \frc{\rho}{\mu}\nabla P
          \end{equation}
          where $\dot{Q}$ is the mass flow rate per cross-sectional area, $\mu$ is the fluid viscosity. $\mathcal{K}$ is the absolute permeability;}
       \item <4-> In 3D, this equation can be expressed as, 
          \begin{displaymath}
            \begin{pmatrix} \dot{Q_{x}}\\ \\\dot{Q_{y}}\\ \\\dot{Q_{z}}\end{pmatrix} = - \mathcal{K} \frc{\rho}{\mu} \begin{pmatrix} \frc{\Delta P}{\Delta x} \\ \\\frc{\Delta P}{\Delta y} \\ \\\frc{\Delta P}{\Delta z}-\rho g \end{pmatrix};
          \end{displaymath}
    \end{enumerate}
   \end{column}
   \begin{column}[c]{0.55\linewidth}
    \begin{enumerate}[1.]\setcounter{enumi}{4} \scriptsize
       \item <5-> Or in matricial form,
          \visible<3->{\begin{equation}
            \underline{\dot{Q}} = - \mathcal{K} \frc{\rho}{\mu} \left(\nabla P - \rho \underline{g}\right),\label{eqn:preDarcy}
          \end{equation}
          with $\underline{g} = \left( 0\;\;\; 0\;\;\; g \right)^{T}$.}
    \end{enumerate}
     \begin{center}
        \visible<3->{\includegraphics[width=1.1\columnwidth,clip]{./Pics/Geothermal_PorousMediaFlow.png}}
     \end{center}
   \end{column}  
  \end{columns}
\end{frame}


%%%
%%% Slide
%%%
\begin{frame}
 \frametitle{Geothermal Darcy Flows}
   \begin{columns}
      \begin{column}[c]{0.5\linewidth}
         \begin{enumerate}[1.]\setcounter{enumi}{10} \scriptsize
            \item <1-> Dividing Eqn.~\ref{eqn:preDarcy} by $\rho$ leads to the volumetric flux, $\underline{u}$,
               \visible<1->{\begin{equation}
                  \frc{\underline{\dot{Q}}}{\rho} = \red{\underline{u} = - \frc{\mathcal{K}}{\mu}\left(\nabla P - \rho \underline{g}\right)},\label{eqn:Darcy}
               \end{equation}
               where $\underline{u}$ is often referred as \blue{\it Darcy velocity};}
            \item <2-> However this is not really the velocity in which the fluid flows -- which is called \blue{\it pore velocity}, \blue{$\underline{v}$}; 
            \item <3-> $\underline{u}$ and $\underline{v}$ are related by the following relationship,
               \visible<3->{\begin{displaymath}
                  \underline{u} = \phi \underline{v}
               \end{displaymath} 
               where $\phi$ is the rock porosity.}
            \item <4-> Before we continue, we need to establish a few very important definitions:
               \begin{enumerate}[(a)]\scriptsize
                  \item <4-> \blue{(Absolute) Porosity $\left(\phi,\text{ dimensionless}\right)$} is the ratio between total volume of voidage in the rock pore and the volume of the rock,
                     \begin{equation}
                        \phi = \frc{V_{\text{void}}}{V_{\text{total}}};
                     \end{equation}
                  \item <5-> \blue{Saturation $\left(S,\text{ dimensionless}\right)$} is a variable relevant \underline{only} when \underline{more than one} fluid 
               \end{enumerate}
         \end{enumerate}  
      \end{column}
      \begin{column}[c]{0.5\linewidth}
         %\begin{enumerate}[(a)]\setcounter{enumi}{1} \scriptsize
             %\item <5-> 
           \visible<5->{\scriptsize (or phase) is present (multiphase systems). Saturation can be  defined as the volume fraction of the occupied pores by fluid $\alpha$,  
                     \begin{equation}
                        S_{\alpha} = \frc{V_{\alpha}}{V_{\text{void}}}
                     \end{equation}}
         %\end{enumerate}  
         \begin{figure}%
            \vspace{-.5cm}
            \vbox{
                  \visible<4->{\includegraphics[width=3.5cm,clip]{./Pics/two-phase-method-flow-in-grain-pack-e1262722458668.jpg}}
                  \visible<5->{\includegraphics[width=3.5cm,clip]{./Pics/Unsat1.jpg}}
            }
         \end{figure}
      \end{column}
   \end{columns}
\end{frame}


%%%
%%% Slide 
%%%
\begin{frame}
 \frametitle{Geothermal Darcy Flows}
    \begin{enumerate}[(a)]\setcounter{enumi}{2} \scriptsize
       \item <1-> \blue{Absolute Permeability $\left(\mathcal{K}\right)$} is a \underline{rock property} that indicates the ability to flow or transmit fluids through rocks. Absolute permeability is often measured in units of {\it darcy} or {\it milidarcy} $\Longrightarrow$ 1 darcy = 9.869233$\times$10$^{-13}$ m$^{2}$;
       \item <2-> Effective Permeability $\left(\mathcal{K}_{\alpha},\text{ darcy or milidarcy}\right)$ is a \underline{rock-fluid property} that measures the conductance of the porous medium for one fluid phase when the medium is saturated with another fluid(s);
       \item <3->  \blue{Relative Permeability $\left(\mathcal{K}_{r\alpha},\text{ dimensionless}\right)$} is the ratio of the effective permeability of a fluid at a given saturation to the absolute permeability,  
          \begin{equation}
             \mathcal{K}_{r\alpha} = \frc{\mathcal{K}_{\alpha}}{\mathcal{K}};
          \end{equation}
    \end{enumerate}

    \begin{enumerate}[1.]\setcounter{enumi}{14} \scriptsize
       \item <4-> The Darcy law represented by Eqn.~\ref{eqn:preDarcy} for a \underline{single phase} can be extended to represent the flow of an arbitrary number of phases,
          \visible<4->{\begin{equation}
            \underline{\dot{Q}}_{\alpha} = - \mathcal{K} \frc{\mathcal{K}_{r\alpha}\rho_{\alpha}}{\mu_{\alpha}} \left(\nabla P_{\alpha} - \rho_{\alpha} \underline{g}\right);\label{eqn:ExtendedDarcy}
          \end{equation}}
    \end{enumerate}

\end{frame}


%%%
%%% SUBSECTION
%%%
\subsection{Mass, Momentum and Energy Conservation}

%%%
%%% Slide 
%%%
\begin{frame}
 \frametitle{Mass Transport}
    \begin{enumerate}[1.]\scriptsize
       \item <1-> Hydrothermal fluids are {\bf \underline{not}} pure fluids;
       \item <2-> Aqueous phase is mixture of a number of components, $\mathcal{N}_{c}$:
          \begin{enumerate}[(a)]\scriptsize
             \item <2-> Water;
             \item <2-> Sodium chloride $\left(NaCl\right)$;
             \item <2-> Carbonates $\left(Na_{2}CO_{3}, K_{2}CO_{3}, CaCO_{3}\right)$ and other salts and;
             \item <3-> Dissolved non-condensable gases: $CO_{2}$, $H_{2}S$, N$_{2}$, noble gases, and;
             \item <4-> Constituents of rocks $\left(\text{e.g., }SiO_{2}\right)$;
          \end{enumerate}
       \item <5-> Hydrothermal fluids may coexist in $\mathcal{N}_{p}$ phases (e.g., gaseous, liquid), in thermodynamic equilibrium; 
       \item <6-> The \blue{total mass flux} of component $i$ split into $\mathcal{N}_{p}$ phases $\left(\underline{\mathcal{F}}_{i}\right)$ is defined as
          \visible<6->{\begin{equation}
              \underline{\mathcal{F}}_{i} = \sum\limits_{j=1}^{\mathcal{N}_{p}} x_{i}^{j}\mathcal{F}_{i} \label{eqn:advectionlike}
          \end{equation}
          where $x_{i}^{j}$ is the mass fraction of component $i\left(\in\left\{1,2,\cdots,\mathcal{N}_{c}\right\}\right)$ in phase $j\left(\in\left\{1,2,\cdots,\mathcal{N}_{p}\right\}\right)$;}
       \item <7-> In addition to the advection-like contribution, Eqn.~\ref{eqn:advectionlike}, mass transport of individual components also occurs due to molecular diffusion. The \blue{diffusive mass flux} $\left(\underline{f}_{i}^{j}\right)$ is expressed as,
          \visible<7->{\begin{equation}
              \underline{f}_{i}^{j} = -\Phi\rho^{j}\kappa_{i}^{j}\nabla x_{i}^{j}\label{eqn:moleculardiffusion}
          \end{equation}
          where $\kappa$ is the mass diffusivity coefficient and $\Phi=\phi\tau_{i}$. $\tau$ is the tortuousity coefficient and is defined as the reduction of diffusivity rate due to tortuous paths taken by the species during transport;}
       \item <8-> Under geothermal reservoir conditions, $\kappa\left(\text{aqueous solutes}\right) = 10^{-9}-10^{-10}\;m^{2}.s^{-1}$ and $\kappa\left(\text{gasses}\right) = 10^{-8}\;m^{2}.s^{-1}$.
    \end{enumerate}

\end{frame}


%%% SUBSECTION
\subsection{Overview of Geothermal Reservoir}

%%%
%%% Slide
%%%
\begin{frame}
 \frametitle{Overview of Geothermal Reservoir}
   \begin{figure}%
     \includegraphics[width=12.cm, height=7.5cm, clip]{./Pics/GeothermalOverviewDiagram}
   \end{figure}  
\end{frame}


%%% SUBSECTION
\subsection{Modelling and Simulation}

%%%
%%% Slide 
%%%
\begin{frame}
 \frametitle{Modelling Fluid Flows in Geothermal Porous Media}
    \begin{enumerate}[1.]\scriptsize
       \item <1-> The first step on modelling $\&$ simulation of fluid flows in porous media is to gather the conservative equations (where $\alpha$ refers to phases):
          \begin{enumerate}[(a)]\scriptsize
             \item <2-> Mass conservation (or continuity):
                \begin{equation}
                   \frc{\partial}{\partial t} \left(\phi\rho_{\alpha}S_{\alpha}\right) + \nabla\cdot\left(\underline{u}_{\alpha}\rho_{\alpha}S_{\alpha}\right) = \mathcal{S}_{\text{cty},\alpha};\label{eqn:continuity}
                \end{equation}
             \item <3-> Momentum conservation (or extended Darcy equation or force-balance):
                \begin{equation}
                   \underline{u}_{\alpha} = - \frc{\mathcal{K}\mathcal{K}_{r\alpha}}{\mu_{\alpha}S_{\alpha}}\left(\nabla P_{\alpha} - \rho_{\alpha}\underline{g} + \mathcal{S}_{\text{mom},\alpha}\right);\label{eqn:momentum}
                \end{equation}
             \item <4-> Thermal energy conservation:
                \begin{equation}
                   \frc{\partial}{\partial t} \left(C_{p,\alpha}\rho_{\alpha}S_{\alpha}T_{\alpha}\right) + \nabla\cdot\left(\underline{u}_{\alpha}C_{p,\alpha}\rho_{\alpha}S_{\alpha}T_{\alpha}\right) = \nabla\cdot\left(S_{\alpha}\kappa_{\alpha}\nabla T_{\alpha}\right) + \mathcal{H}\left(T_{\alpha}-T_{\alpha'}\right) + \mathcal{S}_{\text{thermal},\alpha};\label{eqn:energy}
                \end{equation}
              \visible<4->{where $C_{p}$, $\kappa$ and $\mathcal{H}$ are heat capacity at constant pressure, thermal conductivity coefficient and volumetric interphase (or convective) heat transfer coefficient, respectively. $\mathcal{S}$ is a source and/or sink term;}
             \item <5-> Conservative equations for species and;
             \item <5-> Constitutive equations, i.e., expressions for individual components of the equations above. Example: equations of state are used for individual component and phases to determine density as a function of pressure, temperature, salinity etc. Relative permeability $\left(\mathcal{K}_{r\alpha}\right)$, thermal conductivity $\left(\kappa_{\alpha}\right)$ and interphase heat transfer $\left(\mathcal{H}\right)$ coefficients are often obtained through either experimental or empirical relations;
          \end{enumerate}
    \end{enumerate}

\end{frame}




%%%
%%% Slide 
%%%
\begin{frame}
 \frametitle{Modelling Fluid Flows in Geothermal Porous Media}
         \begin{enumerate}[1.]\setcounter{enumi}{1}\scriptsize
            \item <1-> This set of differential equations (Eqns.~\ref{eqn:continuity}-\ref{eqn:energy}) need to be solved to help in the design and commissioning of power units;
            \item <2-> Reservoir simulators (e.g., \href{http://esd1.lbl.gov/research/projects/tough/software/tough2.html}{TOUGH2}, \href{http://www.software.slb.com/products/foundation/pages/eclipse.aspx}{ECLIPSE}, \href{http://www.uk.comsol.com/}{COMSOL}, \href{http://www.openfoam.com/} {OpenFoam}, etc) are computational tools commonly used in energy industry:
               \begin{enumerate}[(a)]\scriptsize
                   \item <3-> Critical Questions $\&$ Answers:
                       \begin{enumerate}[{a.}i)]\scriptsize
                          \item <4-> {\bf Input data:} number of phases, initial pressure and temperature conditions, properties of fluids and rocks (e.g., $\mathcal{K}$, $\phi$, $\rho$, C$_{p}$, $\mu$, etc.) and initial $\lq$geometry' of the geothermal reservoir field;
                          \item <5-> {\bf Output data:} spatial- and time-distribution of pressure $\left(P_{\alpha}\left(\underline{x},t\right)\right)$, saturation $\left(S_{\alpha}\left(\underline{x},t\right)\right)$, velocity $\left(u_{\alpha}\left(\underline{x},t\right)\right)$ and  temperature $\left(T_{\alpha}\left(\underline{x},t\right)\right)$, and geothermal fluid production rate;
                       \end{enumerate}
                   \item <6-> Geometry:
                       \begin{enumerate}[{b.}i)]\scriptsize
                          \item <6-> Dimension: 1-, 2- or 3-D;
                          \item <6-> Cell/Mesh Shape: 1-D (line), 2-D (triangular and quadrilateral) and 3-D (tetrahedron, hexahedron, pyramid, prism with triangular base, etc );
                          \item <6-> Cell/Mesh Grid: structured or unstructured;
                       \end{enumerate}
                   \item <7-> Geological formation parameters: morphology (porous (e.g., sandstone) or fractured (e.g., carbonates) media), $\phi$, $\mathcal{K}$, $\mathcal{K}_{r\alpha}$, $P_{\text{capillary}}$, C$_{p,\alpha}$, $\kappa_{\alpha}$, etc;
                   \item <8-> Initial Conditions: $T_{\text{reservoir}}$, $P_{\text{reservoir}}$, $S_{\alpha}$, $x_{\alpha}$, homogeneous / heterogeneous (i.e., $\mathcal{K}$ spatial distribution);
                   \item <9-> Boundary Conditions: prescribed (i) mass and heat flows, (ii) injection pressure and temperature conditions;
                   \item <10-> Sink and Sources: (i) mass $\&$ heat, (ii) steady-state or transient (i.e., time-dependent), (iii) injection, (iv) production;
                   \item <11-> Computational Parameters: (i) time-step, (ii) convergence criteria, (iii) linear solvers and pre-conditioners, (iv) etc
               \end{enumerate}
         \end{enumerate}
\end{frame}


%%%
%%% Slide
%%%
\begin{frame}
 \frametitle{Modelling Fluid Flows in Geothermal Porous Media: Mesh Grid Classification}

   \begin{figure}%
     \includegraphics[width=9.cm, height=6.5cm, clip]{./Pics/MeshGrid_Examples.pdf}\label{xx}
    %\caption{Main geometrical elements in reservoir simulators}
   \end{figure}  

\end{frame}




%%%
%%% Slide
%%%
\begin{frame}
 \frametitle{Modelling Fluid Flows in Geothermal Porous Media: Numerical Simulation Algorithm}

   \begin{figure}%
     \includegraphics[width=10.cm, height=7.cm, clip]{./Pics/NumericalSimulation_Fluxogram}
   \end{figure}  

\end{frame}



%%%%%%%%%%%%%%%%%%%%%%%%%%%%%%%%%%%%%%%%%%%%%%%%%%%%%%%%%%%%%%%%%%%%%%%%%
%%%%%%%%%%%%%%%%%%%%%%%%%%%%%%%%%%%%%%%%%%%%%%%%%%%%%%%%%%%%%%%%%%%%%%%%%
%%%%%%%%%%%%%%%%%%%%%%%%%%%%%%%%%%%%%%%%%%%%%%%%%%%%%%%%%%%%%%%%%%%%%%%%%


%%%           %%%
%%%  SECTION  %%% 
%%%           %%%
 \section{Geothermal Project Development} % From IFC-IGA_Geothermal_Exploration_Best_Practices-March2013

%%%
%%% Slide
%%%
\begin{frame}
 \frametitle{Phases for Geothermal Project Developments} 
    There are (usually) 7 phases for any \blue{geothermal} (or subsurface) \blue{project development}: 
    \begin{enumerate}[1.]
       \item <1-> \blue{Preliminary Survey;}
       \item <1-> Exploration; 
       \item <1-> Test Drilling;
       \item <1-> Project Review and Planning;
       \item <1-> Field Development; 
       \item <1-> Power Plant Construction;
       \item <1-> Commissioning and Operation;
    \end{enumerate}
\end{frame}

%%%
%%% Slide
%%%
\begin{frame}
 \frametitle{Phases for Geothermal Project Developments: \\1. Preliminary Survey} 

    Initial assessment (nationally and/or internationally) of the available information on:
    \begin{enumerate}[{1.}1]
       \item <1-> Finances, Economics and Legal issues:
          \begin{enumerate}[(a)]
             \item<2-> Power (electricity) market (and/or power purchase agreements, PPA, and/or feed in tariff);
             \item<2-> Resources ownership (i.e., land/water right, mining laws etc);
             \item<2-> Institutional and regulatory framework;
          \end{enumerate}
       \item <3-> Environmental regulations; 
       \item <4-> Infrastructure (e.g., roads, water, communication, electrical transmission).
    \end{enumerate}
\end{frame}

%%%
%%% Slide
%%%
\begin{frame}
 \frametitle{Phases for Geothermal Project Developments} 
    \begin{enumerate}[1.]
       \item <1-> Preliminary Survey;
       \item <2-> \blue{Exploration}; 
       \item <1-> Test Drilling;
       \item <1-> Project Review and Planning;
       \item <1-> Field Development; 
       \item <1-> Power Plant Construction;
       \item <1-> Commissioning and Operation;
    \end{enumerate}
\end{frame}

%%%
%%% Slide
%%%
\begin{frame}
 \frametitle{Phases for Geothermal Project Developments: \\ 2. Exploration} 

    \begin{enumerate}[{2.}1]\scriptsize
       \item <1-> The \blue{exploration phase} aims to minimise risks related to resource temperature, depth, productivity and sustainability before drilling assessment;
       \item <2-> \blue{Exploration} usually starts by collecting data from existing surrounding surface, subsurface and wells sensors through geological. geochemical and geophysical methods.
       \item <3-> Usual \blue{exploration} surveys include:
          \begin{enumerate}[(a)]\scriptsize
             \item<4-> Geochemical assessment
                \begin{enumerate}[i)]\scriptsize
                   \item<4-> Geothermometry;
                   \item<4-> Electrical conductivity;
                   \item<4-> pH;
                   \item<4-> Fluid flow rate;
                   \item<4-> Soil morphology.
                \end{enumerate}
             \item<5-> Geophysical assessment
                \begin{enumerate}[i)]\scriptsize
                   \item<5-> Gravity;
                   \item<5-> Electrical resistivity;
                   \item<5-> Temperature gradient drilling;
                   \item<5-> 2D and 3D seismics.
                \end{enumerate}
             \item<6-> Surface assessment
                \begin{enumerate}[i)]\scriptsize
                   \item<6-> Surface geology;
                   \item<6-> Geothermal surface features.
                \end{enumerate}
             \item<7-> etc.
          \end{enumerate}
       \item <8-> With this info, models (conceptual and numerical) are built to support the \blue{pre-feasibility study}. 
    \end{enumerate}
\end{frame}
 

%%%
%%% Slide
%%%
\begin{frame}
 \frametitle{Phases for Geothermal Project Developments} 
    \begin{enumerate}[1.]
       \item <1-> Preliminary Survey;
       \item <1-> Exploration; 
       \item <2-> \blue{Test Drilling};
       \item <1-> Project Review and Planning;
       \item <1-> Field Development; 
       \item <1-> Power Plant Construction;
       \item <1-> Commissioning and Operation;
    \end{enumerate}
\end{frame}


%%%
%%% Slide
%%%
\begin{frame}
 \frametitle{Phases for Geothermal Project Developments: \\ 3. Test Drilling}
  \begin{columns}
    \begin{column}[l]{0.45\linewidth}     
       \begin{enumerate}[{3.}1]\scriptsize
          \item <1-> Usually 2-3 wells are drilled to demonstrate the commercial feasibility of production and injection;
          \item <2-> Drilling, logging and testing will improve the understanding of the subsurface resources enabling,  
          \begin{enumerate}[(a)]\scriptsize
             \item<3-> Further estimate of heat resource;
             \item<4-> Determination of the average well productivity;
             \item<5-> Selection of the wells sites, targets, well path and design for the remaining production and injection wells;
             \item<6-> Development of a \blue{preliminary design for the power plant}.
          \end{enumerate}
       \end{enumerate}
       %\begin{center}
           \includegraphics[width=5.6cm,height=2.7cm,clip]{./Pics/Uncased-Drilling-1.jpg}
       %\end{center}
    \end{column}
    \begin{column}[l]{0.55\linewidth} 
       \vbox{
          \hbox{
             \includegraphics[width=0.5\columnwidth,clip]{./Pics/Geothermal_drilling_at_Te_Mihi_NZ.jpg}
             \includegraphics[width=0.5\columnwidth,clip]{./Pics/Drilling.jpg}
          }
          \hbox{
             \includegraphics[width=0.35\columnwidth,clip]{./Pics/DTRC_Diagram.jpg}\hspace{.53cm}
             \includegraphics[width=0.55\columnwidth,clip]{./Pics/web_Drillpipe_injection_well_profile-copy.jpg}
          }
       }
    \end{column}
  \end{columns}
\end{frame}
 
 

%%%
%%% Slide
%%%
\begin{frame}
 \frametitle{Phases for Geothermal Project Developments} 
    \begin{enumerate}[1.]
       \item <1-> Preliminary Survey;
       \item <1-> Exploration; 
       \item <1-> Test Drilling;
       \item <2-> \blue{Project Review and Planning};
       \item <1-> Field Development; 
       \item <1-> Power Plant Construction;
       \item <1-> Commissioning and Operation;
    \end{enumerate}
\end{frame}


%%%
%%% Slide
%%%
\begin{frame}
 \frametitle{Phases for Geothermal Project Developments: \\ 4. Project Review and Planning}
    \begin{enumerate}[{4.}1]\scriptsize
       \item <1-> With all information gathered in the previous phases, the project is reassessed;
       \item <2-> This will enable the developer to:
          \begin{enumerate}[(a)]\scriptsize
             \item<2-> Update the numerical reservoir model;
             \item<2-> (Re)Size the development;
             \item<2-> \blue{Secure power purchase agreement (PPA) to help the financial (including risk analysis) model};
          \end{enumerate}
       \item <3-> The \blue{updated} \underline{numerical reservoir and financial models} will allow the feasibility analysis of:
          \begin{enumerate}[(a)]\scriptsize
             \item<3-> Location and design of drilling pads ;
             \item<3-> Design of development wells;
             \item<3-> Specification of drilling targets for remaining production and reinjection wells;
             \item<3-> Forecasts of performance based on simulated reservoir models (including uncertainty quantification);
             \item<3-> Power plant design;
             \item<3-> Transmission access plan;
             \item<3-> Construction budget and costs;
             \item<3-> Terms of the PPA, \blue{budget and revenue projections.}
          \end{enumerate}
    \end{enumerate}
\end{frame}
 


%%%
%%% Slide
%%%
\begin{frame}
 \frametitle{Phases for Geothermal Project Developments} 
    \begin{enumerate}[1.]
       \item <1-> Preliminary Survey;
       \item <1-> Exploration; 
       \item <1-> Test Drilling;
       \item <1-> Project Review and Planning;
       \item <2-> \blue{Field Development}; 
       \item <1-> Power Plant Construction;
       \item <1-> Commissioning and Operation;
    \end{enumerate}
\end{frame}


%%%
%%% Slide
%%%
\begin{frame}
 \frametitle{Phases for Geothermal Project Developments: \\ 5. Field Development }
    \begin{enumerate}[{5.}1]%\scriptsize
       \item <1-> Drilling of a prescribed number of test, production and (re-)injection wells;
       \item <2-> During the drilling, more data is obtained and the model and planning are updated to take into account new data (e.g., better measurement of permeability, porosity, salinity, pH, etc);
       \item <3-> \underline{For an average well of 2 km depth, a drilling time of 40 to 50 days} \underline{(24 hour operation)} is often required;
       \item <4-> In order to \blue{reduce the risk of depletion}, re-injection wells are required to return the geothermal fluids to the reservoir; 
       \item <5-> The \blue{ratio of reinjection} to production wells ranges from as low as 1:4 in high enthalpy resources to as high as 1:1 in lower enthalpy resources; 
       \item <6-> The enthalpy associated with the geothermal fluids, fluid-steam ratio and the power plant technology will also \blue{determine the number of required wells};
       \item <7-> Another key issues are \blue{location and depths of reinjection wells} -- the decision is based on both drilling experience and numerical simulations.
    \end{enumerate}
\end{frame}
 


%%%
%%% Slide
%%%
\begin{frame}
 \frametitle{Phases for Geothermal Project Developments} 
    \begin{enumerate}[1.]
       \item <1-> Preliminary Survey;
       \item <1-> Exploration; 
       \item <1-> Test Drilling;
       \item <1-> Project Review and Planning;
       \item <1-> Field Development; 
       \item <2-> \blue{Power Plant Construction};
       \item <2-> \blue{Commissioning and Operation};
    \end{enumerate}
\end{frame}


%%%
%%% Slide
%%%
\begin{frame}
 \frametitle{Phases for Geothermal Project Developments: \\ 6. Power Plant Construction \\ 7. Commissioning and Operation }
    \begin{enumerate}[1]\setcounter{enumi}{5}
       \item <1-> Power Plant Construction
         \begin{enumerate}[{6.}1]
           \item<1-> Completion of the steam gathering system, coordinated with infrastructure work;
           \item<1-> Power plant construction and well testing.
         \end{enumerate}
       \item <2-> Commissioning and Operation
         \begin{enumerate}[{7.}1]
           \item<2-> Optimisation of the production and injection procedures to enable the most efficient energy recovery and utilisation;
           \item<2-> This will enable the minimisation of operational costs, maximisation investment returns, and to ensure the reliable delivery of geothermal power.
         \end{enumerate}
    \end{enumerate}
\end{frame}
 












%%%           %%%
%%%  SECTION  %%% 
%%%           %%%
\section{Summary}
%%%
%%% Slide
%%%
\begin{frame}
 \frametitle{Summary}
  \begin{enumerate}[1.]
%
     \item <1-> Revision of:
         \begin{enumerate}[{1.}a)]
             \item <1-> Heat and mass transfer mechanisms;
             \item <1-> Fluid mechanics applied to porous media flows (Darcy's law);
             \item <1-> Main components for project development;
         \end{enumerate}
     \item <2-> Description of the main differential equations that describe mass, momentum and energy conservation in multiphase flows in porous media;
     \item <3-> Summary of the main elements of flow dynamics modelling and simulation;
     \item <4-> Identify the main elements of geothermal project developments.
%
  \end{enumerate}
\end{frame}


\end{document}
