
%\documentclass[11pts,a4paper,amsmath,amssymb,floatfix]{article}%{report}%{book}
\documentclass[12pts,a4paper,amsmath,amssymb,floatfix]{article}%{report}%{book}
\usepackage{graphicx,wrapfig,pdfpages}% Include figure files
%\usepackage{dcolumn,enumerate}% Align table columns on decimal point
\usepackage{enumerate,enumitem}% Align table columns on decimal point
\usepackage{bm,dpfloat}% bold math
\usepackage[pdftex,bookmarks,colorlinks=true,urlcolor=rltblue,citecolor=blue]{hyperref}
\usepackage{amsfonts,amsmath,amssymb,stmaryrd,indentfirst}
\usepackage{times,psfrag}
\usepackage{natbib}
\usepackage{color}
\usepackage{units}
\usepackage{rotating}
\usepackage{multirow}


\usepackage{pifont}
\usepackage{subfigure}
\usepackage{subeqnarray}
\usepackage{ifthen}

\usepackage{supertabular}
\usepackage{moreverb}
\usepackage{listings}
\usepackage{palatino}
%\usepackage{doi}
\usepackage{longtable}
\usepackage{float}
\usepackage{perpage}
\MakeSorted{figure}
%\usepackage{pdflscape}


%\usepackage{booktabs}
%\newcommand{\ra}[1]{\renewcommand{\arraystretch}{#1}}


\definecolor{rltblue}{rgb}{0,0,0.75}


%\usepackage{natbib}
\usepackage{fancyhdr} %%%%
\pagestyle{fancy}%%%%
% with this we ensure that the chapter and section
% headings are in lowercase
%%%%\renewcommand{\chaptermark}[1]{\markboth{#1}{}}
\renewcommand{\sectionmark}[1]{\markright{\thesection\ #1}}
\fancyhf{} %delete the current section for header and footer
\fancyhead[LE,RO]{\bfseries\thepage}
\fancyhead[LO]{\bfseries\rightmark}
\fancyhead[RE]{\bfseries\leftmark}
\renewcommand{\headrulewidth}{0.5pt}
% make space for the rule
\fancypagestyle{plain}{%
\fancyhead{} %get rid of the headers on plain pages
\renewcommand{\headrulewidth}{0pt} % and the line
}

\def\newblock{\hskip .11em plus .33em minus .07em}
\usepackage{color}

%\usepackage{makeidx}
%\makeindex

\setlength\textwidth      {16.cm}
\setlength\textheight     {22.6cm}
\setlength\oddsidemargin  {-0.3cm}
\setlength\evensidemargin {0.3cm}

\setlength\headheight{14.49998pt} 
\setlength\topmargin{0.0cm}
\setlength\headsep{1.cm}
\setlength\footskip{1.cm}
\setlength\parskip{0pt}
\setlength\parindent{0pt}


%%%
%%% Headers and Footers
\lhead[] {\text{\small{EG501J -- Geothermal Energy}}} 
\rhead[] {{\text{\small{Tutorial 01}}}}
%\chead[] {\text{\small{Session 2012/13}}} 
\lfoot[]{Dr Jeff Gomes}
%\cfoot[\thepage]{\thepage}
\rfoot[\text{\small{\thepage}}]{\thepage}
\renewcommand{\headrulewidth}{0.8pt}


%%%
%%% space between lines
%%%
\renewcommand{\baselinestretch}{1.5}

\newenvironment{VarDescription}[1]%
  {\begin{list}{}{\renewcommand{\makelabel}[1]{\textbf{##1:}\hfil}%
    \settowidth{\labelwidth}{\textbf{#1:}}%
    \setlength{\leftmargin}{\labelwidth}\addtolength{\leftmargin}{\labelsep}}}%
  {\end{list}}

%%%%%%%%%%%%%%%%%%%%%%%%%%%%%%%%%%%%%%%%%%%
%%%%%%                              %%%%%%%
%%%%%%      NOTATION SECTION        %%%%%%%
%%%%%%                              %%%%%%%
%%%%%%%%%%%%%%%%%%%%%%%%%%%%%%%%%%%%%%%%%%%

% Text abbreviations.
\newcommand{\ie}{{\em{i.e., }}}
\newcommand{\eg}{{\em{e.g., }}}
\newcommand{\cf}{{\em{cf., }}}
\newcommand{\wrt}{with respect to}
\newcommand{\lhs}{left hand side}
\newcommand{\rhs}{right hand side}
% Commands definining mathematical notation.

% This is for quantities which are physically vectors.
\renewcommand{\vec}[1]{{\mbox{\boldmath$#1$}}}
% Physical rank 2 tensors
\newcommand{\tensor}[1]{\overline{\overline{#1}}}
% This is for vectors formed of the value of a quantity at each node.
\newcommand{\dvec}[1]{\underline{#1}}
% This is for matrices in the discrete system.
\newcommand{\mat}[1]{\mathrm{#1}}


\DeclareMathOperator{\sgn}{sgn}
\newtheorem{thm}{Theorem}[section]
\newtheorem{lemma}[thm]{Lemma}

%\newcommand\qed{\hfill\mbox{$\Box$}}
\newcommand{\re}{{\mathrm{I}\hspace{-0.2em}\mathrm{R}}}
\newcommand{\inner}[2]{\langle#1,#2\rangle}
\renewcommand\leq{\leqslant}
\renewcommand\geq{\geqslant}
\renewcommand\le{\leqslant}
\renewcommand\ge{\geqslant}
\renewcommand\epsilon{\varepsilon}
\newcommand\eps{\varepsilon}
\renewcommand\phi{\varphi}
\newcommand{\bmF}{\vec{F}}
\newcommand{\bmphi}{\vec{\phi}}
\newcommand{\bmn}{\vec{n}}
\newcommand{\bmns}{{\textrm{\scriptsize{\boldmath $n$}}}}
\newcommand{\bmi}{\vec{i}}
\newcommand{\bmj}{\vec{j}}
\newcommand{\bmk}{\vec{k}}
\newcommand{\bmx}{\vec{x}}
\newcommand{\bmu}{\vec{u}}
\newcommand{\bmv}{\vec{v}}
\newcommand{\bmr}{\vec{r}}
\newcommand{\bma}{\vec{a}}
\newcommand{\bmg}{\vec{g}}
\newcommand{\bmU}{\vec{U}}
\newcommand{\bmI}{\vec{I}}
\newcommand{\bmq}{\vec{q}}
\newcommand{\bmT}{\vec{T}}
\newcommand{\bmM}{\vec{M}}
\newcommand{\bmtau}{\vec{\tau}}
\newcommand{\bmOmega}{\vec{\Omega}}
\newcommand{\pp}{\partial}
\newcommand{\kaptens}{\tensor{\kappa}}
\newcommand{\tautens}{\tensor{\tau}}
\newcommand{\sigtens}{\tensor{\sigma}}
\newcommand{\etens}{\tensor{\dot\epsilon}}
\newcommand{\ktens}{\tensor{k}}
\newcommand{\half}{{\textstyle \frac{1}{2}}}
\newcommand{\tote}{E}
\newcommand{\inte}{e}
\newcommand{\strt}{\dot\epsilon}
\newcommand{\modu}{|\bmu|}
% Derivatives
\renewcommand{\d}{\mathrm{d}}
\newcommand{\D}{\mathrm{D}}
\newcommand{\ddx}[2][x]{\frac{\d#2}{\d#1}}
\newcommand{\ddxx}[2][x]{\frac{\d^2#2}{\d#1^2}}
\newcommand{\ddt}[2][t]{\frac{\d#2}{\d#1}}
\newcommand{\ddtt}[2][t]{\frac{\d^2#2}{\d#1^2}}
\newcommand{\ppx}[2][x]{\frac{\partial#2}{\partial#1}}
\newcommand{\ppxx}[2][x]{\frac{\partial^2#2}{\partial#1^2}}
\newcommand{\ppt}[2][t]{\frac{\partial#2}{\partial#1}}
\newcommand{\pptt}[2][t]{\frac{\partial^2#2}{\partial#1^2}}
\newcommand{\DDx}[2][x]{\frac{\D#2}{\D#1}}
\newcommand{\DDxx}[2][x]{\frac{\D^2#2}{\D#1^2}}
\newcommand{\DDt}[2][t]{\frac{\D#2}{\D#1}}
\newcommand{\DDtt}[2][t]{\frac{\D^2#2}{\D#1^2}}
% Norms
\newcommand{\Ltwo}{\ensuremath{L_2} }
% Basis functions
\newcommand{\Qone}{\ensuremath{Q_1} }
\newcommand{\Qtwo}{\ensuremath{Q_2} }
\newcommand{\Qthree}{\ensuremath{Q_3} }
\newcommand{\QN}{\ensuremath{Q_N} }
\newcommand{\Pzero}{\ensuremath{P_0} }
\newcommand{\Pone}{\ensuremath{P_1} }
\newcommand{\Ptwo}{\ensuremath{P_2} }
\newcommand{\Pthree}{\ensuremath{P_3} }
\newcommand{\PN}{\ensuremath{P_N} }
\newcommand{\Poo}{\ensuremath{P_1P_1} }
\newcommand{\PoDGPt}{\ensuremath{P_{-1}P_2} }

\newcommand{\metric}{\tensor{M}}
\newcommand{\configureflag}[1]{\texttt{#1}}

% Units
\newcommand{\m}[1][]{\unit[#1]{m}}
\newcommand{\km}[1][]{\unit[#1]{km}}
\newcommand{\s}[1][]{\unit[#1]{s}}
\newcommand{\invs}[1][]{\unit[#1]{s}\ensuremath{^{-1}}}
\newcommand{\ms}[1][]{\unit[#1]{m\ensuremath{\,}s\ensuremath{^{-1}}}}
\newcommand{\mss}[1][]{\unit[#1]{m\ensuremath{\,}s\ensuremath{^{-2}}}}
\newcommand{\K}[1][]{\unit[#1]{K}}
\newcommand{\PSU}[1][]{\unit[#1]{PSU}}
\newcommand{\Pa}[1][]{\unit[#1]{Pa}}
\newcommand{\kg}[1][]{\unit[#1]{kg}}
\newcommand{\rads}[1][]{\unit[#1]{rad\ensuremath{\,}s\ensuremath{^{-1}}}}
\newcommand{\kgmm}[1][]{\unit[#1]{kg\ensuremath{\,}m\ensuremath{^{-2}}}}
\newcommand{\kgmmm}[1][]{\unit[#1]{kg\ensuremath{\,}m\ensuremath{^{-3}}}}
\newcommand{\Nmm}[1][]{\unit[#1]{N\ensuremath{\,}m\ensuremath{^{-2}}}}

% Dimensionless numbers
\newcommand{\dimensionless}[1]{\mathrm{#1}}
\renewcommand{\Re}{\dimensionless{Re}}
\newcommand{\Ro}{\dimensionless{Ro}}
\newcommand{\Fr}{\dimensionless{Fr}}
\newcommand{\Bu}{\dimensionless{Bu}}
\newcommand{\Ri}{\dimensionless{Ri}}
\renewcommand{\Pr}{\dimensionless{Pr}}
\newcommand{\Pe}{\dimensionless{Pe}}
\newcommand{\Ek}{\dimensionless{Ek}}
\newcommand{\Gr}{\dimensionless{Gr}}
\newcommand{\Ra}{\dimensionless{Ra}}
\newcommand{\Sh}{\dimensionless{Sh}}
\newcommand{\Sc}{\dimensionless{Sc}}


% Journals
\newcommand{\IJHMT}{{\it International Journal of Heat and Mass Transfer}}
\newcommand{\NED}{{\it Nuclear Engineering and Design}}
\newcommand{\ICHMT}{{\it International Communications in Heat and Mass Transfer}}
\newcommand{\NET}{{\it Nuclear Engineering and Technology}}
\newcommand{\HT}{{\it Heat Transfer}}   
\newcommand{\IJHT}{{\it International Journal for Heat Transfer}}

\newcommand{\frc}{\displaystyle\frac}

\newlist{ExList}{enumerate}{1}
\setlist[ExList,1]{label={\bf Example 1.} {\bf \arabic*}}

\newlist{ProbList}{enumerate}{1}
\setlist[ProbList,1]{label={\bf Problem 1.} {\bf \arabic*}}

%%%%%%%%%%%%%%%%%%%%%%%%%%%%%%%%%%%%%%%%%%%
%%%%%%                              %%%%%%%
%%%%%% END OF THE NOTATION SECTION  %%%%%%%
%%%%%%                              %%%%%%%
%%%%%%%%%%%%%%%%%%%%%%%%%%%%%%%%%%%%%%%%%%%


% Cause numbering of subsubsections. 
%\setcounter{secnumdepth}{8}
%\setcounter{tocdepth}{8}

\setcounter{secnumdepth}{4}%
\setcounter{tocdepth}{4}%


\begin{document}



\begin{enumerate}[label=\bfseries Problem \arabic*]
%

%%%
%%%
%%%
\item\label{EasyQuestion} {\it Data below indicates how temperature varies with depth in a hypothetical geothermal electricity site. 
\begin{center}
\begin{tabular}{||c | c | c | c| c | c| c ||}
\hline\hline
{\bf Temperature $\left(^{o}C\right)$} & 25 & 40 & 63 & 100 & 155 & 245 \\
{\bf Depth (m)}                        & 0 & 200 & 400 & 600 & 800 & 1000 \\
\hline\hline
\end{tabular}
\end{center}
\begin{enumerate}
\item Plot the diagram depth $\times$ temperature. 
\item The power plant you are designing is operated with groundwater at 225$^{o}$C. Assuming there is no heat loss from the geothermal source to the environment, determine the depth that it is necessary to drill to produce hot stream for the plant facility.
\item Part of the water used to generate electricity is also used in a district heating system with flow and return temperatures of 80$^{o}$C and 60$^{o}$C respectively.  10$\%$ of the energy supplied is lost in distribution.  A total heat demand on a typical winters day from all building connected to the system is 20 MW. The current heating is supplied by a coal fired boiler operating at an efficiency of 80$\%$.  How much coal is consumed each day if the calorific value of the coal is 24 GJ.tonne$^{-1}$. 
\item It is proposed to supplement the boiler with a single geothermal well which will extract hot water at 80$^{o}$C and discharge the effluent into the sea.  If the maximum flow rate is 71.65 litres.s$^{-1}$,  how much coal will be saved each day. Given, heat capacity of water at constant pressure, $C_{p}$, is 4.18 kJ.$\left(\text{kg.K}\right)^{-1}$.
%\item Briefly explain how a heat pump may be used to increase the potential output from the geothermal resource.
\end{enumerate}}

\begin{comment}
Total heat to be supplied (allowing for distribution loss) =   20 / 0.9		=   22.22 MW
Total net energy supplied in day =   22.22 x 10 6 x 24 x 60 x 60   / 1012	              =     1.92 TJ
Coal consumed in a day (allowing for efficiency)  =   1.92 x 1000 / 24 / 0.8         =    100 tonnes
									     =========
Energy supplied by geothermal heat =  flow rate * temperature difference * specific heat of water

		=  71.65 * (80 - 60) * 4.1868 * 103    =   6.00 MW

hence saving in coal consumed per day =   6 x 10 6 *  86400  / (24 x 109) /0.8   =  27 tonnes
                                                                                          |                              |         =========
						  seconds in day	efficiency

[ alternatively since 6 MW of 22.22 MW is supplied,  saving is 6 / 22.22 * 100 = 27 tonnes again]

\end{comment}


%%%
%%%
%%%
\item\label{P2} {\it Briefly describe the 3 technologies commonly used to generate power in geothermal systems.}

%%%
%%%
%%%
\item {\it Geothermal energy, although considered derived from a renewable source, still has a strong impact in the environment. Table~\ref{table:EnvironmentalImpact}\footnote{T. Hunt (2000) $\lq$Five Lectures on Environmental Effects of Geothermal Utilization', {\it The United Nations University}. ISBN: 9979-68-070-9. In attahcment.} shows a few potential environmental impacts of geothermal energy with both high- and low-temperature stream source (Paper 1, page 5). Discuss these impacts and compare them with those arising from fossil, nuclear and biomass fuels.}
\begin{table}[h]
\begin{center}
  \begin{tabular}{l | l | l | l }
    \hline\hline
    \multirow{2}{*}{} & {\bf Low-temperature systems} & \multicolumn{2}{c}{\bf High-temperature systems} \\
                      &                               & {\bf Vapour-dominated}       & {\bf Liquid-dominated} \\
    \hline
    \multirow{1}{*}{\bf Driling operations} \\
    \hline
    Destruction of forests and erosion & $\bullet$            & $\bullet\bullet$             & $\bullet\bullet$  \\
    Noise                              & $\bullet\bullet$     & $\bullet\bullet$             & $\bullet\bullet$  \\
    Bright lights                      & $\bullet$            & $\bullet$                    & $\bullet$         \\
    Contamination of groundwater by    & $\bullet$            & $\bullet\bullet$             & $\bullet\bullet$  \\
    \;\;\;drilling fluid               &                      &                              &                   \\
    \hline
    \multirow{1}{*}{\bf Mass withdrawal} \\
    \hline
    Degradation of thermal features    & $\bullet$            & $\bullet\bullet$             & $\bullet\bullet\bullet$ \\
    Ground subsidence                  & $\bullet$            & $\bullet\bullet$             & $\bullet\bullet\bullet$  \\
    Depletion of groundwater           & $\circ$              & $\bullet$                    & $\bullet\bullet$         \\
    Hydrothermal eruptions             & $\circ$              & $\bullet$                    & $\bullet\bullet$         \\
    Ground temperature changes         & $\circ$              & $\bullet$                    & $\bullet\bullet$         \\
    \hline
    \multirow{1}{*}{\bf Waste liquid disposal} \\
    \hline
    Effects on living organisms        &                      &                              &                         \\
    \;\;\;surface disposal             & $\bullet$            & $\bullet$                    & $\bullet\bullet\bullet$ \\
    \;\;\; re-injection                & $\circ$              & $\circ$                      & $\circ$                 \\
    Effects on waterways               &                      &                              &                         \\
    \;\;\;surface disposal             & $\bullet$            & $\bullet$                    & $\bullet\bullet$        \\
    \;\;\; re-injection                & $\circ$              & $\circ$                      & $\circ$                 \\
    Contamination of groundwater       & $\bullet$            & $\bullet$                    & $\bullet$               \\
    Induced seismicity                 & $\circ$              & $\bullet\bullet$             & $\bullet\bullet$        \\
    \hline
    \multirow{1}{*}{\bf Waste gas disposal} \\
    \hline
    Effects on living organisms        & $\circ$              & $\bullet$                    & $\bullet\bullet$         \\
    Micro-climatic effects             & $\circ$              & $\bullet$                    & $\bullet$                \\
    \hline
    \hline
  \end{tabular}
\end{center}
  \caption{Possibilities of environmental effects of geothermal development. Symbols: $\circ:$ No effect; $\bullet:$ Little effect; $\bullet\bullet:$ Moderate effect; $\bullet\bullet\bullet:$ High effect.}
\label{table:EnvironmentalImpact}
\end{table}


%%%
%%%
%%%
\item\label{MiningHeatPaper} {\it Paper 2 -- Mining Heat from Schlumberger (page 16) describes the production of heat and power from geothermal sources. Read the paper and get ready to discuss the main aspects and technologies summarised in the manuscript.}




%
\end{enumerate}



\clearpage

{
\includepdf[pages=-,fitpaper, angle=0]{./Tutorials/HuntSelect.pdf}
}

\clearpage

\begin{center}
\bigskip


\bigskip


\vspace{3cm}
{\huge Paper for Discussion:}
\vspace{5cm} 

{\Huge Mining Heat}
\vfill
\end{center}

{
\includepdf[pages=-,fitpaper, angle=0]{./Tutorials/01_mining_heat.pdf}
}

\begin{comment}
\clearpage

{
\includepdf[pages=-,fitpaper, angle=0]{./Pics/20150916162208.pdf}
}

{\bf \ref{P2} Solution:} Geothermal power plants employ three main (family of) technologies: dry steam, flash steam and binary cycle systems. Different from momentum based- (e.g., hydro-power, tides, wind etc) and solar based-power plants, geothermal power plants are independent of fluctuation in daily and seasonal weather, however they must be placed in close proximity of the geothermal source. We can summarise the three main technologies as:
\begin{enumerate}
%
    \item {\bf Dry Steam:} This was the first technology developed to harness energy from geothermal reservoirs. The process consists on injecting cold fluid (i.e., water/brine) into the reservoir and extract (high temperature and pressure) {\it dry steam} (i.e., steam with nearly or no liquid water). The steam, after the removal of water droplets and any impurity is used to drive a set of turbines (i.e., to convert from thermal to mechanical energy) and to generate electrical power through generators (i.e., to convert from mechanical to electrical energy). At the turbine, the fluid is expanded isentropically (i.e., at constant entropy) and then condensed before being re-injected into the geothermal reservoir.
%
    \item {\bf Flash Steam:} This is the most common type of geothermal power generation plants in operation today. They produce liquid brine at high temperature $\left(\ge\text{ 182}^{\circ}\text{C}\right)$ which is pumped at (relatively) high pressure into separation vessels. At this vessels, pressure is abruptly reduced and the fluid splits into two phases ({\it isothermal flash}), the vapour content (still at relatively high pressure) is driven into a set of turbines to generate electrical energy. The remaining of the fluid (after the flash) along with the condensed water/brine are re-injected.
%
    \item {\bf Bynary Cycle:} In this system, the extracted brine temperature is often low (i.e., 107$\le$T$\le$182$^{\circ}$C) to allow efficient power conversion by directly drive the turbines. Thus, heat is extracted from the brine to vaporise a fluid of lower boiling point (often organic Rankine fluids). The vaporised {\it working fluid} at high pressure moves the turbine to generate electrical power, whereas brine is re-injected into the reservoir.
%
\end{enumerate}
%
{\it Flash steam} and {\it binary cycle} are coupled to become the {\bf Combined Cycle}, where liquid brine undertakes isothermal flash and the vapour phase produces part of the electrical power. The thermal energy contained in the liquid fraction after the phase separation (i.e., flash) is transferred to a {\it working fluid} with lower boiling temperature that is vaporised to produce additional electrical power.
%
\end{comment}
\end{document}
