\chapter{Solutions of the Problems}\label{Appendix_Solutions}


%%%% ETOC
\localtableofcontents



%%%%
%%%%
%%%%
\section{Chapter 5}\label{Appendix_Solutions:Chapter5}

%
  \begin{probsol}\label{Chapter:VolumetricPropertiesPureSubstances:Problem:01:solution} % Johannes T4Q2
     \begin{displaymath}\text{Sulfur hexafluoride:} \begin{cases}
             P_{c} = \text{37.6 bar}, \\
             T_{c} = \text{318.7 K}, \\
             V_{c} = \text{198 cm}^{3}\text{.mol}^{-1},\\
             \omega = \text{0.286}.
             \end{cases} \hspace{2cm} \begin{cases} V:? \\ Z:? \end{cases}\end{displaymath}
  \end{probsol}
%
  \begin{probsol}\label{Chapter:VolumetricPropertiesPureSubstances:Problem:02}\citep{Cengel_Book} % Cengel Example 3.13
     Predict the pressure of N$_{2}$ gas at 175 K and $V$ = 3.75$\times$10$^{-3}$ m$^{3}$.kg$^{-1}$ through the following equations of state:
        \begin{enumerate}
          \item Ideal gas equation;
          \item van der Waals.
          %\item Benedict-Webb-Rubin,
             %\begin{displaymath}
                %P = \frc{R T}{V} + \left(B_{0} R T - A_{0} - \frc{C_{0}}{T^{2}}\right) V^{-2} + \frc{ b R T - a}{V^{3}} + \frc{a \alpha}{V^{6}} + \frc{c}{V^{3}T^{2}}\left(1 + \frc{\gamma}{V^{2}}\right) e^{-\gamma/V^{2}} 
             %\end{displaymath}
             %with [P] = kPa, [V] = m$^{3}$.kgmol$^{-1}$ and [T] = K,
             %\begin{center}
                %\begin{tabular}{ l l l l }
                  %\hline
                  %a = 2.54 & b = 2.328$\times$10$^{-3}$ & c = 7.379$\times$10$^{4}$ & $\alpha$ = 1.272$\times$10$^{-4}$ \\
                  %A$_{0}$ = 106.73 & B$_{0}$ = 0.04074 & C$_{0}$ = 8.164$\times$10$^{5}$ & $\gamma$ = 0.0053 \\ 
                  %\hline
                %\end{tabular}
             %\end{center
        \end{enumerate} 
     Compare the values obtained to the experimentally determined value of 10$^{4}$ kPa. Given $T_{c}=$ 126.2 K, $P_{c}=$ 34 bar and $MW=$ 28 g.mol$^{-1}$.
  \end{probsol}
%
  \begin{probsol}\label{Chapter:VolumetricPropertiesPureSubstances:Problem:03}\citep{SmithVanNess_Book} % SM&VN (P3.5)
     Calculate the reversible work done (in kJ) in compressing 0.0283 m$^{3}$ of liquid mercury at a constant temperature of  0$^{\circ}$C from 1 atm to 3000 atm. The isothermal compressibility of mercury at 0$^{\circ}$C is,
     \begin{displaymath}
          \kappa = 3.9\times10^{-6} - 0.1\times 10^{-9}P,
     \end{displaymath}
where [P] = atm and [$\kappa$] = atm$^{-1}$.
  \end{probsol}
%
  \begin{probsol}\label{Chapter:VolumetricPropertiesPureSubstances:Problem:04} 
     Liquid water at 25$^{\circ}$C and 1 bar fills a rigid vessel. If heat is added to the water until its temperature reaches 50$^{\circ}$C, what pressure is developed? The average value of $\beta$ between 25 and 50$^{\circ}$C is 36.2$\times$10$^{-5}$ K$^{-1}$, and the value of $\kappa$ at 1 bar and 50$^{\circ}$C is 4.42$\times$ 10$^{-5}$ bar$^{-1}$  and may be assumed independent of P. The specific volume of liquid water at 25$^{\circ}$C  is 1.0030 cm$^{3}$.g$^{-1}$.
  \end{probsol}
%
