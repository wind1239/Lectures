\chapter{Solutions of the Problems}\label{Appendix_Solutions}


%%%% ETOC
\localtableofcontents



%%%%
%%%%
%%%%
\section{Chapter 5}\label{Appendix_Solutions:Chapter5}

%
  \begin{probsol}\label{Chapter:VolumetricPropertiesPureSubstances:Problem:01:solution} % Johannes T4Q2
     \begin{displaymath}
        \text{Sulfur hexafluoride:} \begin{cases}
             P_{c} = \text{37.6 bar}, \\
             T_{c} = \text{318.7 K}, \\
             V_{c} = \text{198 cm}^{3}\text{.mol}^{-1},\\
             \omega = \text{0.286}.
             \end{cases} \hspace{2cm} \begin{cases} V:? \\ Z:? \end{cases}
     \end{displaymath}
   
     \begin{enumerate}[1.]
%
        \item truncated virial equation:
            \begin{displaymath}
               Z =  \frc{PV}{RT}=1+\frc{B}{V}+\frc{C}{V^{2}}, \;\;\;\text{with } B=-194\frc{\text{cm}^{3}}{\text{mol}},\;\;C=15300 \frc{\text{cm}^{6}}{\text{mol}^{2}}
            \end{displaymath}
            Solving for $V$ (calculator) with the initial guess of (based on ideal gas equation of state),
            \begin{displaymath}
               V^{\text{guess}} = \frc{RT}{P} = 8.314 \frc{\text{J}}{\text{mol.K}} \times 348.15\text{ K} \times \frc{1}{15\text{ bar}} = 1929.68 \frc{\text{cm}^{3}}{\text{mol}},
            \end{displaymath}
            leads to
            \begin{eqnarray}
               &&\frc{PV}{RT}=1+\frc{B}{V}+\frc{C}{V^{2}} \;\;\Longrightarrow \;\; V = 1722.2698 \frc{\text{cm}^{3}}{\text{mol}}\;\;\;\text{ and } \nonumber \\
               && Z =  \frc{PV}{RT} \;\;\Longrightarrow \;\; Z = 0.8925 \nonumber
            \end{eqnarray}
%
        \item Redlich-Kwong EOS:
            \begin{displaymath}
               P = \frc{RT}{V-b} - \frc{a}{V\sqrt{T}(V+b)}\text{ with } a = 0.42748\frc{\left(R T_{c}\right)^{2}}{P_{c}},\; b = 0.08664\frc{R T_{c}}{P_{c}}
            \end{displaymath}
            As this is a non-linear equation on $V$, there are 2 ways to solve this problem:
            \begin{enumerate}[a)]
               \item Calculating $V^{\text{vap}}$ from the SRK EOS through an iterative method and then obtain $Z^{\text{vap}}$;
               \item Calculating $Z^{\text{vap}}$ through Eqn.~\ref{Chapter:VolumetricPropertiesPureSubstances:Eqn:GeneralCubicEOS1_Zvap} and then obtain $V^{\text{vap}}$.
            \end{enumerate}
%
     \end{enumerate}


  \end{probsol}
%
  \begin{probsol}\label{Chapter:VolumetricPropertiesPureSubstances:Problem:02:solution}
     Predict 
  \end{probsol}
%
  \begin{probsol}\label{Chapter:VolumetricPropertiesPureSubstances:Problem:03:solution}
     Calculate
  \end{probsol}
%
  \begin{probsol}\label{Chapter:VolumetricPropertiesPureSubstances:Problem:04:solution} 
     Liquid 
  \end{probsol}
%
