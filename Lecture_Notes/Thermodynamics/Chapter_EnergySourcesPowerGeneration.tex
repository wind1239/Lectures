
%%%
%%% CHAPTER
%%%
\chapter{Energy Sources and Drivers for Power Generation}\label{Chapter:EnergySourcesDrivers}

   \begin{LearningObjectivesBlock}{Learning Objectives}
      Upon completion of this chapter, you will be able to
        \begin{enumerate}
           \item Define 
        \end{enumerate}
\medskip
     Recommended reading: \citet{DiPippo_Book,Armstead_Book,Barbier_2002,Zarrouk_2014,Rybach_2003,Pollack_1993}
   \end{LearningObjectivesBlock}


%%%%%%%%%%%%%%%%%%%%%%%%%%%%%%%%%%%%%%%%%%%%%%%%%%%%%%%%%%%%%%%%%
\begin{comment}
   \begin{LearningObjectivesBlock}{Learning Objectives}
      Upon completion of this chapter, you will be able to
        \begin{enumerate}
           \item {\bf Knowledge:} Define, Name, Select, State 
           \item {\bf Comprehension:} Describe, Identify, Discuss
           \item {\bf Application:} Apply, Demonstrate, Employ, Sketch
           \item {\bf Analysis:} Analyse, Compare, Calculate, Solve
           \item {\bf Synthesis:} Determine, Formulate
           \item {\bf Evaluation:} Assess, Check, Estimate, Compare, Measure, Monitor
        \end{enumerate}
\end{comment}
%%%%%%%%%%%%%%%%%%%%%%%%%%%%%%%%%%%%%%%%%%%%%%%%%%%%%%%%%%%%%%%%%
  

%%%% ETOC
\localtableofcontents


%%%
%%% SECTION
%%%
\section{Introduction}\label{Chapter:EnergySourcesDrivers:Section:Intro}

Electricity is a critical commodity in the modern world -- illumination lighting, TVs, computers, electronic devices (and even cars) etc, they all rely on extensive sources of electricity. In the developed world, electricity is taken for granted, often at relatively low costs, whereas in developing countries cost, quality and availability of electricity are still a daily challenge. 

Coal and wood were burned in the earliest power plants to produce electricity. These rudimentary power plants were initially based on reciprocating steam engines\index{Reciprocating engines} with low overall efficiency, and later replaced by power plants operating with steam turbines. Hydro-power were introduced in the energy mix in the second-half of the nineteenth century as a natural development of its use as mechanical power in the new industrial age. Other energy technologies were further developed along the twentieth century with the introduction of refined hydrocarbons (diesel, natural gas etc) to (a) produce relatively cheap electricity; (b) transport of people and goods, and; (c) heat closed environments. Nuclear energy was first used to commercially generate electricity in 1957 (SM-1 Nuclear Power Plant at Virginia, USA), producing 2 MWe. Since then, nuclear technology has improved in power generation, efficiency and safety, and nowadays the world net capacity is of approximately 370 GWe. This represents nearly 6$\%$ of the world primary energy production, whereas other renewable energy resources such as biomass (\ie wood and dry crop wastes) and hydropower contribute about 10$\%$ and 7$\%$, respectively \citep{AER_2011}.\footnote{See also \href{https://www.imperial.ac.uk/people/r.bryan/document/2407/200barrelsleft2009/?200barrelsleft2009.pdf}{M. Blunt (2009) 'Two hundred barrels left: an analysis of population growth. oil reserves and carbon dioxide emissions'.}}
\medskip

Heat is a form of energy source that can be ``harvested'' from Earth core by geothermal power plants. 




blabla\footnote{See \href{https://www.imperial.ac.uk/people/r.bryan/document/2407/200barrelsleft2009/?200barrelsleft2009.pdf}{M. Blunt (2009) 'Two hundred barrels left: an analysis of population growth. oil reserves and carbon dioxide emissions'.}}


Geothermal energy is simply the natural heat that exists within our planet. In some parts of the
world the existence of a geothermal energy resource is made obvious by the presence of hot
springs, and such resources have been exploited in various ways for millennia. More usually,
there is no direct evidence at Earth‘s surface of the vast reservoir of stored heat below, and
geothermal energy has remained largely ignored and untapped in most parts of the world. Now,
its potential as a renewable source of energy is being recognised increasingly, and technologies
and concepts for exploiting it are developing rapidly along two lines: low temperature resources,
which exploit warm water in the shallow subsurface to provide heat either directly (as warm
water) or indirectly (via heat exchange systems); and high temperature resources, which yield
hot water, usually from deeper levels, that can be used to generate electricity.
The potential for harnessing electricity from geothermal energy has long been recognised; the
potentially substantial reserves, minimal environmental impact, and capacity to contribute
continuously to base load electricity supply make it an extremely attractive prospect. The
ongoing drive to develop renewable sources of energy, coupled with anticipated technological
developments that will in future reduce the depth at which heat reservoirs are considered
economically viable, means there is now a pressing need to know more about the deep
geothermal energy potential in Scotland.






The Energy and Climate Change Directorate (ECCD) of the Scottish Government (the Client)
have identified deep geothermal energy as a particularly important emerging renewable energy
technology that could have the potential to play a significant role in Scotland‘s future energy
provision.
Exploitation of deep geothermal heat energy could potentially deliver many benefits to Scotland,
including:
Reducing carbon emissions and helping Scotland build a sustainable low-carbon economy
in order to meet the legislative requirements for emissions reductions;
Increase the use of renewable heat to help exceed the targets set out in the 2020
Routemap for Renewable Energy in Scotland;
Potentially help to exceed the targets for renewable electricity production;
Become a viable alternative source of energy, improving local and national energy security
and reducing reliance on external sources of energy;
Help reduce fuel poverty through the use of district heating networks;
Regenerate brownfield sites, including in former mining and industrial areas;
Provide skilled employment opportunities, with cross-over with the oil and gas and
manufacturing sectors; and
Push Scotland towards the forefront in the technology required for exploiting deep
geothermal resources, particularly in areas previously considered as marginal or even not
viable.
To date, the extent and location of the potential deep geothermal resources has not been well
defined. In addition, potential commercial investment in development of deep geothermal energy
requires greater certainty on the current administrative framework, including clarification of legal
ownership of resources legal ownership, resource licensing, supportive planning and permitting
regimes, and financing.
