
%%%
%%% CHAPTER
%%%
\chapter{Second Law of Thermodynamics}\label{Chapter:FirstLaw}

   \begin{LearningObjectivesBlock}{Learning Objectives}
      Upon completion of this chapter, you will be able to
        \begin{enumerate}
           \item Demonstrate understanding of key concepts of energy and the first law of thermodynamics;
           \item Apply the first law of thermodynamics to assess of heat transfer and power cycles;
           \item Conduct energy analysis of thermodynamic systems;
           \item Employ energy and mass balances into thermodynamic systems to assess efficiency, and correctly observe sign conventions for work and heat transfer.
        \end{enumerate}
\medskip
     Recommended reading: Chapters 5 of \citet{SmithVanNess_Book,Moran_Book,Borgnakke_Book} or 3 of \citet{Atkins_Book}.
   \end{LearningObjectivesBlock}


  
%%%
%%% SECTION
%%%
   \section{Introduction}\label{Chapter:SecondLaw:Section:Intro}
   The first law (Chapters~\ref{Chapter:Introduction} and \ref{Chapter:FirstLaw}) demonstrated that energy can flow either from or to a system in the form of heat or work, however it does not indicate the direction of process (\ie energy flow). The second law of thermodynamics concerns about feasibility, direction and spontaneity of processes and entropy..
   
The second law of thermodynamics is a general principle which makes constraints upon the direction of heat transfer and the efficiency of heat engines.



   
     In Section~\ref{Chapter:Introduction:Section:ThermodAnalysis}, the main elements in the thermodynamic analysis were introduced, namely {\bf open, closed and isolated systems}, {\bf surroundings} and {\bf boundaries}. The concept of {\bf energy}, {\bf work} and {\bf heat}, pivotal entities in the study of thermodynamics systems, were also defined as,
     \begin{itemize}
        \item {\bf Work} is motion against an opposing force (Eqn.~\ref{Chpt01_Work1});
        \item {\bf Energy} of a system is its capacity to produce work, and; 
        \item {\bf Heat} is the transfer of energy across boundaries caused by temperature gradient \citep{Devoe_Book}.
     \end{itemize}
     These definitions are based on observations of systems in a macro-scale, and are critical for mass and energy balances necessary for this chapter. 
