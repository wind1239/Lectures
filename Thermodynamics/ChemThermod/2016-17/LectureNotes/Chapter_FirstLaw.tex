
%%%
%%% CHAPTER
%%%
\chapter{First Law of Thermodynamics}\label{Chapter:FirstLaw}

   \begin{LearningObjectivesBlock}{Learning Objectives}
      Upon completion of this chapter, you will be able to
        \begin{enumerate}
           \item {\bf Knowledge:} Define, Name, Select, State 
           \item {\bf Comprehension:} Describe, Identify, Discuss
           \item {\bf Application:} Apply, Demonstrate, Employ, Sketch
           \item {\bf Analysis:} Analyse, Compare, Calculate, Solve
           \item {\bf Synthesis:} Determine, Formulate
           \item {\bf Evaluation:} Assess, Check, Estimate, Compare, Measure, Monitor
        \end{enumerate}
\medskip
     Recommended reading: Chapters 2 of \citet{Atkins_Book,SmithVanNess_Book,Moran_Book} or 3 of \citet{Borgnakke_Book}.
   \end{LearningObjectivesBlock}

   
%%%
%%% SECTION
%%%
     \section{Introduction}\label{Chapter:FirstLaw:Section:Intro}
     In Section~\ref{Chapter:Introduction:Section:ThermodAnalysis}, the main elements in the thermodynamic analysis were introduced, namely {\bf open, closed and isolated systems}, {\bf surroundings} and {\bf boundaries}. The concept of {\bf energy}, {\bf work} and {\bf heat} were also defined,
     \begin{description}
        \item[Work] is motion against an opposing force (Eqn.~\ref{Chpt01_Work1});
        \item[Energy] of a system is its capacity to produce work, and; 
        \item[Heat] is the transfer of energy across the boundary caused by a temperature gradient at the boundary \citep{Devoe_Book}.
     \end{description}
