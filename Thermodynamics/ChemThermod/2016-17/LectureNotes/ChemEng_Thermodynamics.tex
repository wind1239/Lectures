
\PassOptionsToPackage{svgnames}{xcolor}
\documentclass[11pts,a4paper,amsmath,amssymb,floatfix]{book}
%\usepackage[gather]{chapterbib}

\usepackage{graphicx,wrapfig}% Include figure files
%\usepackage{dcolumn,enumerate}% Align table columns on decimal point
\usepackage{enumerate}% Align table columns on decimal point
\usepackage{bm,dpfloat}% bold math
\usepackage[pdftex,bookmarks,colorlinks=true,urlcolor=rltblue,citecolor=blue]{hyperref}
\usepackage{amsfonts,amsthm,amssymb,stmaryrd,indentfirst}
\usepackage{amsmath}% ,example}
%\numberwithin{Answer}{chapter}
%\numberwithin{Exercise}{chapter}

\usepackage{times,psfrag,pdfpages}
%\usepackage{natbib} 
\usepackage{color}
\usepackage{units}
\usepackage{rotating}
\usepackage{multirow}
% This package provides \cancel{a} and \cancelto{a}{b} to "cancel"
% expressions in math.
\usepackage{cancel}

%\usepackage[sort&compress,square,comma,authoryear]{natbib}
\usepackage[sort&compress,comma,round]{natbib}

% \usepackage{algorithm}
\usepackage[linesnumbered,lined,boxed,commentsnumbered]{algorithm2e}
\usepackage[noend]{algpseudocode} 
\makeatletter
\def\BState{\State\hskip-\ALG@thistlm}
\makeatother

\usepackage{pifont}
\usepackage{subfigure}
\usepackage{subeqnarray}
\usepackage{ifthen}

\usepackage{pgfplots}
\pgfplotsset{every axis/.append style={
                    axis x line=middle,    % put the x axis in the middle
                    axis y line=middle,    % put the y axis in the middle
                    axis line style={<->}, % arrows on the axis
                    xlabel={$x$},          % default put x on x-axis
                    ylabel={$y$},          % default put y on y-axis
                }}



\usepackage{supertabular}
\usepackage{moreverb}
\usepackage{fancyvrb}
\usepackage{listings}
\usepackage{palatino}
%\usepackage{doi}
\usepackage{longtable}
\usepackage{float}
%\usepackage{perpage}
%\MakeSorted{figure}
%\usepackage{pdflscape}
\usepackage{framed,comment,lscape}
\definecolor{shadecolor}{gray}{0.9}

% Appendix allows the inclusion of a line for "appendices", which otherwise won't get the right page number. You must
% use the [toc] option.
%\usepackage[toc]{appendix}

\definecolor{rltblue}{rgb}{0,0,0.75}

%\usepackage{natbib}
\usepackage{fancyhdr} %%%%
\pagestyle{fancy}%%%%
% with this we ensure that the chapter and section
% headings are in lowercase
\renewcommand{\chaptermark}[1]{\markboth{#1}{}}
\renewcommand{\sectionmark}[1]{\markright{\thesection\ #1}}
%\renewcommand*\thesection{\arabic{section}}


\fancyhf{} %delete the current section for header and footer
%\fancyhead[LE,RO]{\bfseries\thepage}
%\fancyfoot[LE,RO]{\bfseries\thepage}
\fancyhead[LO]{\bfseries\rightmark}
\fancyhead[RE]{\bfseries\leftmark}
\renewcommand{\headrulewidth}{0.5pt}
% make space for the rule
\fancypagestyle{plain}{%
\fancyhead{} %get rid of the headers on plain pages
\renewcommand{\headrulewidth}{0pt} % and the line
}

\def\newblock{\hskip .11em plus .33em minus .07em} 
\usepackage{color}

\theoremstyle{definition}
\newtheorem{exmp}{Example}[chapter]

% For theorems
\newtheorem{theorem}{Theorem}[section]
\newtheorem{corollary}{Corollary}[theorem]
\newtheorem{lemma}[theorem]{Lemma}

\usepackage{makeidx}
\makeindex

\setlength\textwidth      {16.cm}
\setlength\textheight     {23.0cm}
\setlength\oddsidemargin  {-0.3cm}
\setlength\evensidemargin {0.3cm}

\setlength\headheight{14.49998pt} 
\setlength\topmargin{0.0cm}
\setlength\headsep{.5cm}
\setlength\footskip{.8cm}
\setlength\parskip{0pt}
%\setlength\parindent{0pt} % no indentation

%%%
%%% Headers and Footers
%\lhead[\text{\small{WORK IN PROGRESS}}] {\text{\small{ICON/AMCG}}} 
%-\lhead[] {\text{\small{ICON/AMCG}}} 
%-\chead[\text{\small{FETCH Manual}}]  {\text{\small{FETCH Manual}}} %{\text{\small{JAERI Analysis Report Issue}}}
%\rhead[]{}
\rfoot[\thepage]{\thepage}
%-\rfoot[]{}%{\text{\small{\today}}}
%\cfoot[\text{\small{\today}}]{}% {\text{\small{\today}}}
%\cfoot[\text{\small{January 2006}}] {\text{\small{January 2006}}}
%-\lfoot []{}%{\text{\small{Numerical Analysis}}}
%-\renewcommand{\headrulewidth}{0.8pt}

%%%
%%% space between lines
%%%
\renewcommand{\baselinestretch}{1.5}

\newenvironment{VarDescription}[1]%
  {\begin{list}{}{\renewcommand{\makelabel}[1]{\textbf{##1:}\hfil}% 
    \settowidth{\labelwidth}{\textbf{#1:}}%
    \setlength{\leftmargin}{\labelwidth}\addtolength{\leftmargin}{\labelsep}}}%
  {\end{list}}

%%%%%%%%%%%%%%%%%%%%%%%%%%%%%%%%%%%%%%%%%%%
%%%%%%                              %%%%%%%
%%%%%%      NOTATION SECTION        %%%%%%%
%%%%%%                              %%%%%%%
%%%%%%%%%%%%%%%%%%%%%%%%%%%%%%%%%%%%%%%%%%%


% This is for quantities which are physically vectors.
\renewcommand{\vec}[1]{{\mbox{\boldmath$#1$}}}
% Physical rank 2 tensors
\newcommand{\tensor}[1]{\overline{\overline{#1}}}
% This is for vectors formed of the value of a quantity at each node.
\newcommand{\dvec}[1]{\underline{#1}}
% This is for matrices in the discrete system.
\newcommand{\mat}[1]{\mathrm{#1}}


\DeclareMathOperator{\sgn}{sgn}

%\newcommand\qed{\hfill\mbox{$\Box$}}
% Derivatives
\renewcommand{\d}{\mathrm{d}}
\newcommand{\D}{\mathrm{D}}
\newcommand{\ddx}[2][x]{\frac{\d#2}{\d#1}}
\newcommand{\ddxx}[2][x]{\frac{\d^2#2}{\d#1^2}}
\newcommand{\ddt}[2][t]{\frac{\d#2}{\d#1}}
\newcommand{\ddtt}[2][t]{\frac{\d^2#2}{\d#1^2}}
\newcommand{\ppx}[2][x]{\frac{\partial#2}{\partial#1}}
\newcommand{\ppxx}[2][x]{\frac{\partial^2#2}{\partial#1^2}}
\newcommand{\ppt}[2][t]{\frac{\partial#2}{\partial#1}}
\newcommand{\pptt}[2][t]{\frac{\partial^2#2}{\partial#1^2}}
\newcommand{\DDx}[2][x]{\frac{\D#2}{\D#1}}
\newcommand{\DDxx}[2][x]{\frac{\D^2#2}{\D#1^2}}
\newcommand{\DDt}[2][t]{\frac{\D#2}{\D#1}}
\newcommand{\DDtt}[2][t]{\frac{\D^2#2}{\D#1^2}}
% Norms

% Units
\newcommand{\m}[1][]{\unit[#1]{m}}
\newcommand{\km}[1][]{\unit[#1]{km}}
\newcommand{\s}[1][]{\unit[#1]{s}}
\newcommand{\invs}[1][]{\unit[#1]{s}\ensuremath{^{-1}}}
\newcommand{\ms}[1][]{\unit[#1]{m\ensuremath{\,}s\ensuremath{^{-1}}}}
\newcommand{\mss}[1][]{\unit[#1]{m\ensuremath{\,}s\ensuremath{^{-2}}}}
\newcommand{\K}[1][]{\unit[#1]{K}}
\newcommand{\PSU}[1][]{\unit[#1]{PSU}}
\newcommand{\Pa}[1][]{\unit[#1]{Pa}}
\newcommand{\kg}[1][]{\unit[#1]{kg}}
\newcommand{\rads}[1][]{\unit[#1]{rad\ensuremath{\,}s\ensuremath{^{-1}}}}
\newcommand{\kgmm}[1][]{\unit[#1]{kg\ensuremath{\,}m\ensuremath{^{-2}}}}
\newcommand{\kgmmm}[1][]{\unit[#1]{kg\ensuremath{\,}m\ensuremath{^{-3}}}}
\newcommand{\Nmm}[1][]{\unit[#1]{N\ensuremath{\,}m\ensuremath{^{-2}}}}

% Dimensionless numbers
\newcommand{\dimensionless}[1]{\mathrm{#1}}
\renewcommand{\Re}{\dimensionless{Re}}

% Other symbols
\newcommand{\frc}{\displaystyle\frac}
\newcommand{\red}{\textcolor{red}}
\newcommand{\blue}{\textcolor{blue}}
\newcommand{\green}{\textcolor{green}}
\newcommand{\purple}{\textcolor{purple}}
\newcommand{\eg}{{\it e.g., }}
\newcommand{\ie}{{\it i.e., }}
\newcommand{\wrt}{{\it wrt }}
\newcommand{\Partial}[3][error]{\left(\frc{\partial #1}{\partial #2}\right)_{#3}}
\newcommand{\mfr}[3][error]{#1_{#2}^{\left(#3\right)}} 
\newcommand{\summation}[3][error]{\sum\limits_{#2}^{#3}#1}

%%%%%%%%%%%%%%%%%%%%%%%%%%%%%%%%%%%%%%%%%%%
%%%%%%                              %%%%%%%
%%%%%% END OF THE NOTATION SECTION  %%%%%%%
%%%%%%                              %%%%%%%
%%%%%%%%%%%%%%%%%%%%%%%%%%%%%%%%%%%%%%%%%%%

% Cause numbering of subsubsections. 
%\setcounter{secnumdepth}{8}
%\setcounter{tocdepth}{8}

\setcounter{secnumdepth}{3}%
\setcounter{tocdepth}{3}%

\usepackage{chngcntr}
%\counterwithout{figure}{chapter}

\newcounter{qcounter}
\newcounter{mcounter}
\DeclareMathAlphabet{\mathpzc}{OT1}{pzc}{m}{it} 

%%%%%%%%%% Chemical Reactions %%%%%%%%%%%%%%%% 

\usepackage[T1]{fontenc}
\usepackage[utf8]{inputenc}
\usepackage{lmodern}
\usepackage[version=3]{mhchem}
\makeatletter
\newcounter{reaction}
%%% >> for article <<
%\renewcommand\thereaction{R6.\,\arabic{reaction}}
%%% << for article <<
%%% >> for report and book >>
\renewcommand\thereaction{C\,\thechapter.\arabic{reaction}}
%\@addtoreset{reaction}{chapter}
%%% << for report and book <<
\newcommand\reactiontag{\refstepcounter{reaction}\tag{\thereaction}}
\newcommand\reaction@[2][]{\begin{equation}\ce{#2}%
\ifx\@empty#1\@empty\else\label{#1}\fi%
\reactiontag\end{equation}}
\newcommand\reaction@nonumber[1]{\begin{equation*}\ce{#1}%
\end{equation*}}
\newcommand\reaction{\@ifstar{\reaction@nonumber}{\reaction@}}
\makeatother

%%%%%%%%%%%%%%%%%%%%%%%%%%%%%%%%%%%%%%%%%%%%%%%%%%%%%%%%%%%%%

%%%%%%%  ENVIRONMENTAL VARIABLES FOR  EXAMPLES     %%%%%%%%%%
\newcounter{examplecounter}
\newenvironment{example}{\begin{quote}%
    \refstepcounter{examplecounter}%
  \textbf{Example \thechapter.\arabic{examplecounter}}% 
  \quad
}{%
\end{quote}%
}
%%%%%%%  ENVIRONMENTAL VARIABLES FOR Tutorial Problems   %%%%%%%%%%
\newcounter{problemcounter}
\newenvironment{problem}{\begin{quote}%
    \refstepcounter{problemcounter}%
  \textbf{Problem \arabic{problemcounter}}%
  \quad
}{%
\end{quote}%
}

%%%%%%%  ENVIRONMENTAL VARIABLES FOR  BLOCKS   %%%%%%%%%%
\usepackage{tcolorbox}
\usepackage{lipsum}
\tcbuselibrary{skins,breakable}
\usetikzlibrary{shadings,shadows}
\newenvironment{LearningObjectivesBlock}[1]{%
    \tcolorbox[beamer,%
    noparskip,breakable,
    colback=LightGreen,colframe=DarkGreen,%
    colbacklower=LimeGreen!75!LightGreen,%
    title=#1]}%
    {\endtcolorbox}

\newenvironment{FinalSummaryBlock}[1]{%
    \tcolorbox[beamer,%
    noparskip,breakable,
    colback=LightCoral,colframe=DarkRed,%
    colbacklower=Tomato!75!LightCoral,%
    title=#1]}%
    {\endtcolorbox}

\newenvironment{MyBlock}[1]{%
    \tcolorbox[beamer,%
    noparskip,breakable,
    colback=LightGray,colframe=DarkGray,%
    colbacklower=DarkGray!75!LightGray,%
    title=#1]}%
    {\endtcolorbox}

\newenvironment{MyExample}[1]{%
    \tcolorbox[beamer,%
    noparskip,breakable,
    colback=LightGray,colframe=DarkGray,%
    colbacklower=DarkGray!75!LightGray,%
    title=#1]}%
    {\endtcolorbox}

%%%% ETOC Package to introduce Local Table of Contents
\usepackage{etoc}
\makeatletter
\def\etocarticlestyle{%
    \etocsettocstyle
    {\section *{\contentsname
%                \@mkboth {\MakeUppercase \contentsname}
%                         {\MakeUppercase \contentsname}
         }
         }
    {}}
    \makeatother

\begin{document}
\etocsettocdepth{subsection}

\vspace{4cm}

\begin{titlepage}
\begin{center}
\bigskip

  % \includegraphics[width=15cm,clip]{./../FigBanner/UoAHorizBanner}

\bigskip
\vspace{3.5cm}

   {\bf{\Huge Fundamentals of Applied Thermodynamics }}

\bigskip
   {\bf{\huge Lecture Notes}}

\vspace{3.5cm}

\end{center}

\begin{tabular}{l c l}
{\bf{\large Author:}}                          &     & J.L.M.A. Gomes \\
%                                               &     & Environmental and Industrial Fluid Mechanics Group \\
                                               &     & School of Engineering \\
                                               &     & University of Aberdeen \\
\bigskip
{\bf{\large Date:}}                            &     &   \today\\
                                               &     &         \\
{\bf{\large Revision 1:}}                      &     &   \today\\
                                               &     &         \\
\end{tabular}
\vspace{3.5cm}

\bigskip

\bigskip
\end{titlepage}

\pagenumbering{gobble}% Remove page numbers (and reset to 1)

\pagenumbering{roman}% Arabic page numbers (and reset to 1)
\begin{center}
  \Large{ Revisions}

\bigskip

\begin{tabular}{ c c}
\hline
{\bf Revision number}  & {\bf Date } \\
\hline
  1                    & \today \\
\hline 
\end{tabular}
\end{center}

\setcounter{page}{1}

\tableofcontents
\vfill
%%%% ETOC
\etocarticlestyle

\pagebreak
\listoftables
\vfill
\pagebreak
\listoffigures
\vfill
\pagebreak


\pagenumbering{arabic}% Arabic page numbers (and reset to 1)

\part{Introduction and Review of Basic Concepts}

   \begin{shaded}
      Contents of this Part are not examinable. They were designed to help with some of the notations and definitions used in the remaining of this Notes. 
   \end{shaded}
   
  
%%%
%%% CHAPTER
%%%
\chapter{Thermodynamics: Introduction and Principles}\label{Chapter:Introduction}

   \begin{LearningObjectivesBlock}{Learning Objectives}
      Upon completion of this chapter, you will
        \begin{enumerate}
           \item be able to identify the main elements in a thermodynamic system;
           \item understand the concept of thermodynamic equilibrium;
           \item be able to state the zeroth law of thermodynamics.
        \end{enumerate}
\medskip
     Recommended reading: Chapter 2 of \citet{Atkins_Book,Devoe_Book,Borgnakke_Book}.
   \end{LearningObjectivesBlock}


%%%
%%% SECTION
%%%
   \section{Introduction}\label{Chapter:Introduction:Section:Introduction}

   The word `{\it thermodynamics}' stems from Greek roots, {\it therme}: heat and {\it dynamis}: power -- `movement of heat', and was used by the first time by Lord Kelvin \citep{Thomson_1849}. Thermodynamics studies global properties of the matter and the process (\eg thermal, chemical, mechanical, nuclear etc) in which these properties may be altered. In other words, the quantification of the inter-relation between energy and the change of properties of any physical system.

   For most practical applications, thermodynamics deals with interactions between thermal (\ie heat and temperature), mechanical (\ie work) and/or chemical (\ie chemical potential) energies. \citet{Borgnakke_Book} defined thermodynamics as the `the science that deals with heat and work and those properties of matter that relate to heat and work.'

   \medskip
   
   The study of thermodynamics may be divided into two main areas, {\it classical} and {\it statistical} thermodynamics. The latter investigates the macroscopic properties of systems comprising of a large number of subsystems (\ie particles or atoms). Properties of such systems are controlled by the motion of each set (or assembly) of particles, which can be determined by applying probabilistic theories and methods to laws of motion.

   {\it Classical thermodynamics} studies the macroscopic change of properties, \ie it is assumed that matter is formed by large quantity of particle assemblies that have properties representing the interactions between these assemblies. The extent of such changes due to transfer of energy to or from the system is described by fundamental equations of thermodynamics which are derived from observations known as `{\it Laws of Thermodynamics}'. These laws are postulates that describe the nature of interactions of systems and energy.

   {\bf This document will focus only on the study of fundamentals of \underline{classical thermodynamics} and its application on environmental and industrial problems. }

%%%
%%% SECTION
%%%
   \section{Main Elements of Thermodynamics Analysis}\label{Chapter:Introduction:Section:ThermodAnalysis}

   The first stage in the analysis of any thermodynamic problem is to identify the domain in which all energy and/or forces are transferred to or from.
   
% Figure
   \begin{figure}[h]
     \begin{center}
       \includegraphics[width=8cm, height=8cm]{./../Pics/Fig_SystemDefinition}
       \caption{Elements of a thermodynamic problem: system and surroundings separated by well-defined borders.}\label{Chapter:Introduction:Fig:Domain}
     \end{center}
   \end{figure}
   
   
%%% Subsection
   \subsection{System, Surroundings and Boundaries}\label{Chapter:Introduction:Section:Introduction:SystemSurroundingsBoundaries}\index{System}\index{System!Boundaries}\index{System!Surroundings}
   In practice, any thermodynamic analysis starts by defining the domain of interest, which can be a volume in space or quantity of matter (Fig.~\ref{Chapter:Introduction:Fig:Domain}). This domain is called {\it system}, \ie any 3-D region of physical space with prescribed mass; the remaining of the domain is called {\it surroundings} (or {\it neighbourhood}) which is limited by {\it boundaries}. The {\it boundary} is a surface that encloses the {\it system} and separates it from the {\it surroundings}. For example, in Fig.~\ref{Chapter:Introduction:Fig:Domain2}, liquid nitrogen is contained in a cylinder with prescribed wall thickness. In this case, the interior of the vessel with N$_{2}$ is the {\it system}, whereas the cylinder wall is the border of the system. 

% Figure
   \begin{figure}[h]
     \begin{center}
       \includegraphics[width=9cm, height=7cm]{./../Pics/Fig_SystemDefinition2}
        \caption{Example of a well-defined thermodynamic problem: cylinder stored in a room. Pressurised liquid N$_{2}$ contained in a cylinder is the system, whereas the remaining of the room are the surroundings. Cylinder's wall is the border of the system.}\label{Chapter:Introduction:Fig:Domain2}
     \end{center}
   \end{figure}

   For convenience, sometimes we may want to divide the {\it system} into multiple {\it sub-systems} and analyse them individually, or to combine several small {\it systems} into larger {\it super-systems}. The choice depends on the conditions of the domain of interest and how mass and energy flow across the {\it sub-systems}. For example, in Fig.~\ref{Chapter:Introduction:Fig:Domain2}, if the valve is opened to the room (at atmospheric pressure) would be vaporised ({\it phase change}) and occupy the room. In such scenario, the room and the cylinder become the {\it system} bounded by the room's walls; the area outside the room is now the surroundings. Multiple different configurations can be drawn from this rather simple cylinder-room set.

%%% Table
   \begin{table}[h]
     \begin{center}
      \begin{tabular}{|c|c|c|}
         \hline
                      & {\bf Mass} & {\bf Energy} \\
                      & {\bf Exchange} & {\bf Exchange} \\
         \hline
         {\bf Open}   & {\it yes}  & {\it yes}    \\
         {\bf Closed} & {\it no}   & {\it yes}    \\
         {\bf Isolated}&{\it no}   & {\it no}     \\
         \hline 
      \end{tabular}  
        \caption{System and control volumes: energy and mass transfer.}\label{Chapter:Introduction:Table:System}
     \end{center}
   \end{table}
   
   If mass and energy are allowed to flow across the {\it boundaries}, we say that the {\it system} is {\bf open}, otherwise if only the energy is allowed to flow (\ie be transferred) across the {\it boundaries}, the {\it system} is assumed to be {\bf closed}. If both energy and mass can not be transferred across the {\it boundaries} the system is assumed {\bf isolated}, in such case, where there is no energy flow, the boundary is called {\bf adiabatic} (Table~\ref{Chapter:Introduction:Table:System}).\index{System!Open}\index{System!Closed}\index{System!Isolated}\index{System!Adiabatic}\index{Adiabatic}

% Figure
   \begin{figure}[h]
     \begin{center}
        \includegraphics[width=0.7\columnwidth,clip]{./../Pics/Fig_SystemDefinition3}
        \caption{Cylinder-piston system with extensive/intensive properties.}\label{Chapter:Introduction:Fig:Domain3}
     \end{center}
   \end{figure}

   In the example depicted in Fig.~\ref{Chapter:Introduction:Fig:Domain2}, assuming an ordinary industrial liquid N$_{2}$ (at subzero temperature) cylinder, if the valve is closed, then there is no fluid flow from the cylinder to the room, but heat is flowing from the environment to the cylinder cavity. Such system is said to be {\bf closed}.
   
\medskip
% Example
\begin{MyExample}{\begin{center}{\bf Example}\end{center}}
\begin{example}\label{Chapter:Introduction:Example1}
  \citep{Reisel_Book} For the following systems, determine whether the system described is best modelled as an isolated, closed or open system:
  \begin{enumerate}[a)]
     \item steam flowing through a turbine\;\;$\rightarrow$\;\; {\it Open.}
     \item an incandescent light bulb\;\;$\rightarrow$\;\; {\it Closed.}
     \item an inflated tire\;\;$\rightarrow$\;\; {\it Isolated if the tire is at rest, but closed if it is in movement.}
     \item a rock formation 200 m below the surface of the earth\;\;$\rightarrow$\;\; {\it Open.}
     \item a tea kettle containing boiling water\;\;$\rightarrow$\;\; {\it Open as water steam can still leave the system.}
     \item a human body\;\;$\rightarrow$\;\;{\it Depending on the circumstances, a human body can be either open (\eg during meals, physical exercises etc) or closed.}
     \item an engine's radiator\;\;$\rightarrow$\;\; {\it Closed}.
  \end{enumerate}
\end{example}
\end{MyExample}

%%% Subsection
   \subsection{Properties and State of Substances}\label{Chapter:Introduction:Section:Introduction:ExtensiveIntensiveProperties}\index{Extensive Properties}\index{Intensive Properties}\index{System!Extensive Properties}\index{System!Intensive Properties}
   The {\it material} in a system is composed of phases (e.g., solid, liquid, gas) with distinct physical and chemical properties, thus with explicit {\it boundaries} (\ie interfaces) between phases. A quantitative property of a system (\eg temperature and pressure) describes macroscopic characteristics, which may vary with time (\ie time-dependent property). Two states of the matter are equivalent if they have the same properties, \eg in a cylinder-piston system (Fig.~\ref{Chapter:Introduction:Fig:Domain3}) containing {\it n} moles of pure gas, if {\it state 1} is defined by temperature $T_{1}$, pressure $P_{1}$ and volume $V_{1}$, and {\it state 3} is defined is by temperature $T_{3}$, pressure $P_{3}$ and volume $V_{3}$, state 1 is {\it equivalent} to state 3 {\it if and only if} $T_{1} = T_{3}$ and $P_{1} = P_{2}$. 
\medskip

   Thermodynamic properties may be classified as either {\bf extensive} or {\bf intensive}. An extensive property is a property that depends on the mass (or extent) of the substance (\ie size) in the system. Examples of extensive properties are total mass, total volume, total internal energy etc. An intensive property is a property that is independent of the mass of the substance, examples are temperature and pressure.

   Thus, for example, if a system is cut in half, its intensive properties remain unchanged, while extensive properties are cut in half. The ratio of an extensive property to the mass (\ie property per unit mass) is called {\bf specific property}, and this is an {\bf intensive} property.. The ratio of an extensive property to the number of moles of the substance in the system (\ie property per mole) is referred as {\bf molar property}, also this is an {\bf intensive} property.

   
%%%
%%% SECTION
%%%
   \section{Thermodynamic Work and Heat}\label{Chapter:Introduction:Section:ThermodynamicWorkHeat}\index{Work}\index{Heat}\index{Energy}
   \begin{subequations}
     Work can be defined as a form of energy transfer due to changes in external macroscopic physical properties of a thermodynamic system. It can be expressed in several forms: magnetic, mechanical, electrical etc.

     For example, in a piston-cylinder system (Fig.~\ref{Chapter:Introduction:Fig:Domain3}) work is produced by the system when the gas volume expands against an external force (states 2-3). Similarly, an external force is responsible for the compression of the gas, \ie work is given to the system (states 1-2). In these cases (expansion and compression of a gas), work transfer (to or from the system) is due to the application of a finite force on the system boundary (piston).

     It is clear that the boundary (\ie volume limited by the cylinder wall and the piston-head) either contracts or expands due to external and internal forces acting on it. In other words, applied forces acting over a distance (piston length) result in mechanical energy transfer (\ie work). For an infinitesimal displacement of the piston within a cylinder, {\it dx}, the work ($W$) can be defined by
     \begin{equation}
        dW = F dx,\label{Chpt01_Work1}
     \end{equation}
     where $F$ is the force acting vertically upon the piston. If the movement occurs over a finite distance, the resulting work can be obtained by integrating Eqn.~\ref{Chpt01_Work1}. By convention, {\bf work} is assumed {\bf positive} if the displacement is in the same direction as the force applied, and {\bf negative} when the force and the displacement are in opposite directions. Thus, from stage 2 to 3 (Fig.~\ref{Chapter:Introduction:Fig:Domain3}), the force is acting upon the piston with contraction of the volume of the gas $\left(V^{t}\right)$,
     \begin{displaymath}
       dW = -PAd\left(\frc{V^{t}}{A}\right),
     \end{displaymath}
     where $A$ is the area of the piston (constant), then
     \begin{shaded}
        \begin{equation}
           dW = -PdV^{t}.\label{Chpt01_Work2}
        \end{equation}
     \end{shaded}
     Equation~\ref{Chpt01_Work2} describes the work undertaken by any process when the volume changes due to transfer of energy from or to the system. If the fluid undertakes a compression (thus reduction of volume) due to pressure over the system, the work is positive, otherwise when the system produces work (\ie transfer energy to the surroundings through expansion of the boundaries), the work is considered as negative. The concept of work leads to the definition of {\bf energy} as the capacity of the system to produce work.

     \citet{Devoe_Book} defined {\bf heat} as `the transfer of energy across the boundary caused by a temperature gradient at the boundary'. This concept will naturally lead to the {\it First Law of Thermodynamics} (Chapter~\ref{Chapter:Chapter:FirstLaw}).     

   \end{subequations}
   
%%%
%%% SECTION
%%%
   \section{Thermodynamic Equilibrium and the Zeroth Law}\label{Chapter:Introduction:Section:Equilibrium_ZerothLaw}\index{Equilibrium!Mechanical}\index{Equilibrium!Chemical}\index{Equilibrium!Thermal }\index{Laws of Thermodynamics!Zeroth law}
   During thermodynamic processes, the state of the system may change due to gradients of different variables within or across boundaries, \ie
   \begin{enumerate}[a)]
        \item pressure gradients result in momentum transfer and/or convective mass transport;
        \item temperature gradients produce heat exchange, and;
        \item concentration gradients yields to diffusive mass transfer.
   \end{enumerate}
   Changes in the state of the system will continue until all internal or cross-boundary gradients vanish. When all gradients are non-existent the system exhibits no further changes and at such conditions, the system is said to be in {\bf thermodynamic equilibrium}.\index{Equilibrium!Thermodynamic}
      A system is in {\bf thermodynamic equilibrium} if it satisfies the criteria for mechanical, thermal and chemical equilibrium.  
% Figure
   \begin{figure}[h]
     \begin{center}
        \includegraphics[width=0.7\columnwidth,clip]{./../Pics/Fig_SystemDefinition4}
        \caption{Potential energy variation in a particle motion.}\label{Chapter:Introduction:Fig:Domain4}
     \end{center}
   \end{figure}

\medskip

   Let's consider a particle initially at rest (state I in Fig.~\ref{Chapter:Introduction:Fig:Domain4}). The total energy associated with this particle is the sum of potential and kinetic energies (assuming that the particle is chemically inert and is kept at a constant temperature). If the particle is perturbed by a mechanical force of very small magnitude, it will eventually return to its initial state (\ie at a finite time), however if the perturbation is sufficiently large the particle is unlikely to return to the original state. In this scenario, the particle is said to be in a {\it unstable equilibrium}\index{Equilibrium!Mechanical!Unstable}. Now, let's assume that the particle is at state II, where any perturbation can move it to either state I or III. In such conditions, the particle is said to be in a {\it meta-stable equilibrium}\index{Equilibrium!Mechanical!Meta-stable}. Finally, if the particle is at state III, it will remain at this condition even under the influence of large perturbation. At such conditions, the particle is said to be in a {\it stable equilibrium}\index{Equilibrium!Mechanical!Stable}. If $E_{p}$ is the potential energy of the particle and $x$ is the displacement in the vertical direction, the equilibrium states can be described as
   \begin{equation}
      \begin{cases}
         \text{Stable equilibrium (III):}  & \frc{\partial E_{p}}{\partial x} = 0 \text{ and } \frc{\partial^{2} E_{p}}{\partial x^{2}} > 0; \\
          \\
         \text{Unstable equilibrium (II):}  & \frc{\partial E_{p}}{\partial x} = 0 \text{ and } \frc{\partial^{2} E_{p}}{\partial x^{2}} < 0; \\
          \\
         \text{Meta-stable equilibrium (I):}  & \frc{\partial E_{p}}{\partial x} = 0 \text{ and } \frc{\partial^{2} E_{p}}{\partial x^{2}} = 0; \\
      \end{cases}
   \end{equation}
   These mechanical equilibrium states can be extended to thermodynamic systems during phase changes, where the potential energy and the spatial coordinate are replaced by the {\it Gibbs free energy} and intensive/extensive properties, respectively.

   \bigskip

   The concept of thermal equilibrium is intuitively simple: if two or more bodies at distinct temperatures are in physical contact, the bodies will tend to a single temperature at a finite time. This principle is called the {\bf Zeroth Law} of thermodynamics and was first stated by J. C. Maxwell in 1872:
   \begin{MyBlock}{{\bf Zeroth Law of Thermodynamics (Maxwell, 1872) } }
     ``Bodies whose temperatures are equal to that of the same body have themselves equal temperatures.”
   \end{MyBlock}
   This definition enables the use of thermometers as devices to measure the temperature of bodies. Traditional thermometers have two components, a bulb containing mercury and a linear temperature scale. The mercury bulb is maintained at a relatively low temperature $\left(\text{\ie } T_{\text{th}}\le 35^{\circ}\text{C}\right)$, whereas a body is at temperature $T>T_{\text{th}}$. When the thermometer and the body are in contact, from the {\it zeroth law}, both will reach the same temperature $T$ at a finite time. The temperature difference triggers a volumetric expansion of the mercury that can be readily observed in the scaled glass column.

\clearpage   
\begin{FinalSummaryBlock}{Summary}
    In this chapter, some fundamental concepts of thermodynamic properties were revised and the their relationships with energy were introduced, also:
    \begin{itemize}
       \item Thermodynamics is the study of the transformations of energy;
       \item Energy can be defined as the capacity to produce work;
       \item Work is the transfer of energy by motion against an opposing force (Eqns.~\ref{Chpt01_Work1}-~\ref{Chpt01_Work2};
       \item Heat is the transfer of energy  as a result of a temperature difference between the system and the surroundings;
       \item Definitions of system, surroundings and boundaries were stated in Section~\ref{Chapter:Introduction:Section:Introduction:SystemSurroundingsBoundaries};
       \item In open systems, mass and energy are allowed to freely flow across the boundaries of the system, whereas in closed system only energy is able to cross the borders at constant mass. In isolated (or adiabatic) system, neither energy nor mass can be trabsported across the borders;
       \item A state function is a property that depends only on the current state of the system and is independent of the origin of the state; 
       \item Thermal equilibrium is a condition in which no change of state occurs when two (or more) bodies are in contact with each other;
       \item Mechanical equilibrium is the condition of equality of pressure across the boundary of the system;
       \item The Zeroth Law of thermodynamics states that if a body {\it A} is in thermal equilibrium with a body {\it B}, and {\it B} is in thermal equilibrium with {\it C}, then {\it C} and {\it A} are also in thermal equilibrium;
    \end{itemize} 
   
     Most of these fundamentals concepts are familiar to you through other engineering courses. However, the remaining of this document strongly relies on this concepts and ideas, and you should understand all these concepts before moving forward.
\end{FinalSummaryBlock}

   
\begin{comment}
%%%
%%% SECTION
%%%
\section{Laws of Thermodynamics}\label{Chapter:Introduction:Section:LawsThermodynamics}\index{Laws of Thermodynamics}

%%%
%%% SECTION
%%%
\section{Zeroth Law}\label{zeroth_law}\index{Laws of Thermodynamics!Zeroth law}



%%%
%%% SECTION
%%%
\section{First Law}\label{Chapter:LawsOfThermodynamics:FirstLaw}\index{Laws of Thermodynamics!First law}


%%%
%%% SECTION
%%%
\section{Second Law}\label{second_law}\index{Laws of Thermodynamics!Second law}




blablabla \cite{batchelor_1967} \cite{SmithVanNess_Book}
\index{Reynolds Transport theorem}

\begin{exmp}
This is the example.
\end{exmp}

%%%
%%% SECTION 1
%%%
\section{Module 01: Introduction and Principles}\label{Section:01}

%%% SUBSECTION
\subsection{A Few Important Definitions}
  
   \begin{enumerate}[i)]
%
       \item The thermodynamic system is the part of the universe we are considering. We are free to choose boundary conditions that best represent the problem.
%
       \item \red{System} is defined as a quantity of matter or a region in space chosen for study. The mass or region outside the system is called the {\it surroundings};
%
       \item Real or imaginary surfaces that separate the system from its surroundings is called the {\it boundary};
%
      \item Systems may be considered to be {\it closed} or {\it open}, depending on whether a fixed mass or a fixed volume in space is chosen for study; 
%
      \item A closed system (also known as a {\it control mass}) consists of a fixed amount of mass, and no mass can cross its boundary. However, energy (in the form of heat or work) may cross the boundary -- and the volume of a closed system does not have to be fixed; 
%
      \item When neither energy nor mass is allowed to cross the boundary, that system is called an {\it isolated system};
%
      \item An open system (or {\it control volume}) is a properly selected region in space. It usually encloses a device that involves mass flow such as a compressor, turbine, or nozzle.
%
      \item The {\it material} in a system is composed of phases (e.g., solid, liquid, gas) with distinct physical and chemical properties;
%
      \item The {\it composition} of each phase is described by a series of discrete chemical formula units (i.e., chemical components) -- e.g., water/steam $\left(\right.$H$_{2}$O$\left.\right)$, ammonia $\left(\right.$NH$_{3}\left.\right)$, carbon dioxide $\left(\right.$CO$_{2}\left.\right)$, etc;
%
      \item {\it Properties} are macroscopic quantities associated with the system and may be defined experimentally (\eg P, V, T etc). These quantities are either intensive or extensive, \ie either independent or linearly dependent on the amount of matter.
%
      \item {\it State functions} are function of any thermodynamic property, and as such they can be either extensive or intensive (\eg internal energy, enthalpy, entropy, Gibbs free energy, Helmholtz free energy etc).
%
      \item In an arbitrary thermodynamic transformation where $Q$ is the net amount of heat absorbed by the system, and $W$ is the net amount of work done on the system, the 1$^{\text{st}}$ Law states that
          \begin{displaymath}
             \Delta U = Q + W,
          \end{displaymath}
where $U$ is the internal energy.
%
      \item In thermally isolated system (\ie contained within adiabatic walls):
          \begin{displaymath}
             Q = 0 \Longrightarrow \Delta U = W. 
          \end{displaymath}
%
      \item For mechanically isolated system,
          \begin{displaymath}
             W = 0 \Longrightarrow \Delta U = Q.
          \end{displaymath}
%
      \item The 1$^{\text{st}}$ Law is a statement of energy conservation and defines $U$ as an extensive state function. In an infinitesimal transformation, the first law can be expressed in differential form as,
          \begin{displaymath}
             dU = \delta Q + \delta W.
          \end{displaymath}
This expression states that d$U$ is a total (\ie exact) differential for an infinitesimal transformation. $Q$ and $W$ are process-dependent and are not state functions, therefore $\delta Q$ and $\delta W$ are approximations (\ie not exact). 
%
      \item In thermodynamic cycles, changes in the system may lead to a number of equilibrium and non-equilibrium states, but ending in exactly the same state as the start. By definition, all state variables (\eg $U$) are unchanged in a cycle thus,
          \begin{displaymath}
             \displaystyle\oint\limits_{C} d U = 0,
          \end{displaymath}
around any closed cycle $C$, or
          \begin{displaymath}
             \displaystyle\oint\limits_{C} \delta Q + \displaystyle\oint\limits_{C} \delta W = 0.
          \end{displaymath}
Although in general $\displaystyle\oint\limits_{C} \delta Q\ne 0$ and $\displaystyle\oint\limits_{C} \delta W\ne 0$, these line integrals depend on the closed path $C$.
%
      \item For a process to be {\it reversible} two conditions must be satisfied: (a) it must be quasistatic, and (b) there must be no friction. A quasistatic process is a successive set of equilibrium states of the system. It is an idealisation as it is required to be carried out infinitely slowly. As reversible processes are infinitely slow, it is always essentially in equilibrium.
%
      \item In {\it irreversible} processes, variables and state functions continuously change through a number of non-equilibrium states until reaching equilibrium.
%
      \item If the PVT behaviour of a fluid is represented by $PV^{n}=$ constant, then
           \begin{itemize}
              \item If $n = 0\;\;\Longrightarrow P =$ constant, and the process is {\it isobaric}; 
              \item If $n = 1\;\;\Longrightarrow PV =$ constant, and the fluid is an {\it ideal gas};
              \item If $n = \infty\;\;\Longrightarrow$ the process is {\it isochoric} (\ie constant volume);
              \item If $n = \gamma=\frc{C_{p}}{C_{v}}\;\;\Longrightarrow$ the process is {\it adiabatic} (\ie isentropic).
           \end{itemize}
%
   \end{enumerate}

%%% SUBSECTION
\subsection{A Few Important Derivations}
  
   \begin{enumerate}[i)]
%
      \item Derive \blue{$C_{p}-C_{v}=R$}:

           From the statement of the First Law $dU = dQ + dW$ with $dU=C_{v}dT$ and $dW = -PdV$,
              \begin{equation}
                  \red{dQ = C_{v}dT + PdV},\label{Mod01_1Law_1}
              \end{equation}
           where $V$ is the molar volume. For constant external pressure,
              \begin{displaymath}
                  C_{p}dT = dQ = C_{v}dT + PdV,
              \end{displaymath}
           And assuming \underline{ideal gas}, the equation of state, $PV=RT$, can be differentiated,
              \begin{displaymath}
                  dT = d\left(\frc{P V}{R}\right) \Longrightarrow dT =\frc{P}{R}dV + \frc{V}{R}\cancelto{=0\text{ (constant pressure)}}{dP}  \Longrightarrow PdV = RdT
              \end{displaymath}
           Now, replacing in the previous equation,
              \begin{eqnarray}
                  C_{p}dT &=& C_{v}dT + PdV \nonumber \\
                  C_{p}dT &=& C_{v}dT + RdT \;\;\;\blue{\left(\times \frc{1}{dT}\right)} \nonumber\\
                  C_{p} &=& C_{v} + R \Longrightarrow \red{C_{p}-C_{v}=R}  \label{Mod01_1Law_CpCv}
              \end{eqnarray}
%
      \item Derive \blue{$dQ=C_{p}dT-\frc{RT}{P}dP$}: 

           Again from $dU = dQ + dW$,
           \begin{displaymath}
               dQ = dU -dW = dU + PdV = C_{v}dT + PdV,
           \end{displaymath}
           however as $C_{p}+C_{v}=R$,
           \begin{eqnarray}
               dQ &=& \left(C_{p}-R\right)dT + PdV \nonumber \\
                  &=& C_{p}dT - RdT + PdV. \nonumber
           \end{eqnarray}
           Differentiating the ideal gas equation of state,
           \begin{eqnarray}
               PV &=& RT \;\;\text{ (differentiating both sides)} \nonumber \\
               d(PV) &=& d(RT) \nonumber \\
               PdV + VdP &=& RdT \nonumber
           \end{eqnarray}
           Replacing $RdT$ in the relation above for $dQ$,
           \begin{eqnarray}
               \red{dQ} &=& C_{p}dT - RdT + PdV. \nonumber \\
                  &=& C_{p}dT - PdV - VdP + PdV \nonumber \\
                  &=& C_{p}dT - VdP \red{= C_{p}dT - \frc{RT}{P}dP} \label{Mod01_1Law_2}
           \end{eqnarray}
%
      \item Derive \blue{$dQ=\frc{C_{p}}{R} PdV + \frc{C_{v}}{R} VdP$}:

           Again from $dU = dQ + dW$,
           \begin{displaymath}
               dQ = dU -dW = dU + PdV = C_{v}dT + PdV,
           \end{displaymath}
           Differentiating the ideal gas equation of state, $T=\frc{PV}{R}$
           \begin{displaymath}
                dT = \frc{P}{R}dV + \frc{V}{R}dP.
           \end{displaymath}
           Replacing it in the previous relation, and with $C_{p}-C_{v}=R$,
           \begin{eqnarray}
             \red{dQ} &=& C_{v}\frc{P}{R}dV + C_{v}\frc{V}{R}dP + PdV = \left(\frc{C_{v}}{R}+1\right)PdV + \frc{C_{v}}{R}VdP \nonumber \\
                &=& \left(\frc{C_{p}-R}{R}+1\right)PdV + \frc{C_{v}}{R}dP \nonumber \\
                      &=&  \red{\frc{C_{p}}{R} PdV + \frc{C_{v}}{R} VdP } \label{Mod01_1Law_3}
           \end{eqnarray}
%
      \item We just derived 3 fundamental relations based on the 1$^{\text{st}}$ Law -- Eqns.~\ref{Mod01_1Law_1},~\ref{Mod01_1Law_2} and ~\ref{Mod01_1Law_3}.
%
      \item Isentropic/Polytropic Relations: in adiabatic processes, no heat exchange is allowed between the system and the surroundings ($dQ=0$). For mechanically reversible adiabatic (\ie isentropic) \blue{compression / expansion} of ideal gasses (Eqn.~\ref{Mod01_1Law_1}),
           \begin{eqnarray}
             dQ &=& C_{v}dT + PdV =0 \nonumber \\
             dT &=& -\frc{P}{C_{v}}dV  \;\; \left(\text{Constraint: } C_{v}\ne 0\right) \nonumber \\
             dT &=& -\frc{RT}{V C_{v}}dV \;\; \Rightarrow \;\; \frc{dT}{T} = - \frc{R}{C_{v}}\frc{dV}{V}. \label{Mod01_1Law_4}
           \end{eqnarray}
           Integrating Eqn.~\ref{Mod01_1Law_4} and assuming $C_{v}$ is constant,
           \begin{eqnarray}
              \int\limits_{T_{1}}^{T_{2}} \frc{dT}{T} &=& \frc{R}{C_{v}}\int\limits_{V_{1}}^{V_{2}}\frc{dV}{V} \nonumber \\
              \left.\ln{T}\right|_{T_{1}}^{T_{2}} &=& \left.-\frc{R}{C_{v}}\ln{V}\right|_{V_{1}}^{V_{2}} \nonumber \\
              \ln{\frc{T_{2}}{T_{1}}} &=& -\frc{R}{C_{v}}\ln{\frc{V_{2}}{V_{1}}} = \ln{\left(\frc{V_{1}}{V_{2}}\right)^{\frac{R}{C_{v}}}} \nonumber \\
              \frc{T_{2}}{T_{1}} &=& \left(\frc{V_{1}}{V_{2}}\right)^{\frac{R}{C_{v}}} \nonumber
           \end{eqnarray}

           Now, defining the heat capacity ratio (or isentropic index), $\gamma\equiv\frc{C_{p}}{C_{v}}$, and using the relation $C_{p}-C_{v}=R$,
           \begin{equation}
              \gamma = \frc{C_{p}}{C_{v}} = \frc{C_{v}+R}{C_{v}} = 1 + \frc{R}{C_{v}},\label{Mod01_Gamma}
           \end{equation}
           the relation above becomes
           \begin{equation}
              \red{TV^{\gamma-1} = \text{ constant}}\label{Mod01_1Law_5}
           \end{equation}

           Now, from Eqn.~\ref{Mod01_1Law_2} and assuming $C_{p}$ is constant and different from zero,
           \begin{eqnarray}
             dQ &=& C_{p}dT - \frc{RT}{P} = 0 \nonumber \\
             \int\limits_{T_{1}}^{T_{2}}\frc{dT}{T} &=& \frc{R}{C_{p}}\int\limits_{P_{1}}^{P_{2}}\frc{dP}{P} \nonumber \\
             \left.\ln{T}\right|_{T_{1}}^{T_{2}} &=& \left.\frc{R}{C_{p}} \ln{P}\right|_{P_{1}}^{P_{2}} \nonumber \\
             \ln{\frc{T_{2}}{T_{1}}} &=& \ln{\left(\frc{P_{2}}{P_{1}}\right)^{\frac{R}{C_{p}}}} \nonumber \\
             \frc{T_{2}}{T_{1}} &=& \left(\frc{P_{2}}{P_{1}}\right)^{\frac{R}{C_{p}}}.\nonumber
           \end{eqnarray}
            Using the $\gamma$ relation, Eqn.~\ref{Mod01_Gamma},
           \begin{equation}
              \red{TP^{\frac{1-\gamma}{\gamma}} = \text{ constant}}\label{Mod01_1Law_6}
           \end{equation}

           Finally, from Eqn.~\ref{Mod01_1Law_3},
           \begin{eqnarray}
             dQ &=& \frc{C_{v}}{R}VdP + \frc{C_{p}}{R}PdV = 0 \nonumber \\
              \frc{C_{v}}{\cancel{R}}VdP = -\frc{C_{p}}{\cancel{R}}PdV &\Longrightarrow& C_{v}\int\limits_{P_{1}}^{P_{2}} \frc{dP}{P} = -C_{p}\int\limits_{V_{1}}^{V_{2}}\frc{dV}{V} \nonumber \\
              \left.\ln{P}\right|_{P_{1}}^{P_{2}} &=& -\left.\frc{C_{p}}{C_{v}}\ln{V}\right|_{V_{1}}^{V_{2}} \nonumber \\
              \frc{P_{2}}{P_{1}} &=& \left(\frc{V_{1}}{V_{2}}\right)^{\frac{C_{p}}{C_{v}}} \nonumber
           \end{eqnarray}
            Using the $\gamma$ relation, Eqn.~\ref{Mod01_Gamma},
           \begin{equation}
              \red{PV^{\gamma} = \text{ constant}}\label{Mod01_1Law_7}
           \end{equation}
%
      \item {\bf Relation for Entropy Changes:} From the First Law equation,
           \begin{equation}
              dU = dQ - PdV,\label{Mod01_1Law_Eqn}
           \end{equation} 
           If we differentiate the enthalpy equation -- $H = U + PV$.
                \begin{displaymath}
                    dH = dU + d(PV) = dU + PdV +VdP,
                \end{displaymath}
           and replace in Eqn.~\ref{Mod01_1Law_Eqn}:
                \begin{displaymath}
                    dH - \cancel{PdV} - VdP = dQ - \cancel{PdV} \;\;\Rightarrow \;\; dQ = dH - VdP
                \end{displaymath}
           For ideal gas, $C_{p}=\left(\frac{dH}{dT}\right)_{P}$ and $V=\frc{RT}{P}$,
                \begin{eqnarray}
                  dQ &=& C_{p}dT - \frc{RT}{P}dP\;\;\;\;\;\times\left(\frc{1}{T}\right) \nonumber \\
                  \frc{dQ}{T} &=& \frc{C_{p}}{T}dT - \frc{R}{P}dP \nonumber \\
                  dS &=& \frc{C_{p}}{T}dT - \frc{R}{P}dP, \nonumber
                \end{eqnarray}
           where $S$ is the molar entropy of ideal gas. Integrating from state 0 to state 1,
                \begin{eqnarray}
                    \int\limits_{S_{0}}^{S_{1}} dS &=& \int\limits_{T_{0}}^{T_{1}} \frc{C_{p}}{T}dT - R\int\limits_{P_{0}}^{P_{1}}\frc{dP}{P} \nonumber \\
                    \left(S_{1}-S_{0}\right) &=& \int\limits_{T_{0}}^{T_{1}} \frc{C_{p}}{T}dT - R\ln{\frc{P_{1}}{P_{0}}} \;\;\;\;\times\left(\frc{1}{R}\right) \nonumber \\
                    \red{\frc{\Delta S}{R}} &=& \red{\int\limits_{T_{0}}^{T_{1}} \frc{C_{p}}{R}\frc{dT}{T} - \ln{\frc{P_{1}}{P_{0}}} }.
                \end{eqnarray}
           Although this equation was derived for mechanically reversible processes, it focuses on \underline{properties only} and is independent of the process. Thus it can be used to calculate of {\it ideal gasses}.

               
%
   \end{enumerate}

%%% SUBSECTION
\subsection{General Remarks for the Course}

\begin{enumerate}[(i)]
%
   \item Do always use \blue{SI units} for calculations:
       \begin{itemize}
          \item second ($s$), meter ($m$), gram ($g$), Kelvin ($K$), mole ({\it mol});
       \end{itemize}
%
   \item Or those based on them:
       \begin{itemize}
          \item Newton ($N=kg.m.s^{-2}$), Joule ($J=N.m=kg.m^{2}.s^{-2}$), Pascal ($Pa=N.m^{-2}=kg.m^{-1}.s^{-2}$).
       \end{itemize}
%
   \item And the appropriated prefix:
      \begin{center}
        \begin{tabular}{c c c | c c c}
             \hline
             {\it Multiple} & {\it Prefix} & {\it Symbol} & {\it Multiple} & {\it Prefix} & {\it Symbol} \\
             \hline
             10$^{-15}$      & femto        & f            &   10$^{2}$     &  hecto       & h            \\
             10$^{-12}$      & pico         & p            &   10$^{3}$     &  kilo        & k            \\
             10$^{-9}$       & nano         & n            &   10$^{6}$     &  mega        & M            \\
             10$^{-6}$       & micro        & $\mu$        &   10$^{9}$     &  giga        & G            \\
             10$^{-3}$       & milli        & m            &   10$^{12}$    &  tera        & T            \\
             10$^{-2}$       & centi        & c            &   10$^{15}$    &  peta        & P            \\
             \hline
        \end{tabular}
      \end{center}
%
   \item Most of the time, we need to convert units during our calculations. Thus if we want to convert pressure ($P$) from {\it atm} to {\it psi} (pounds per square inch):
      \begin{displaymath}
        P = 5\;\cancel{\text{atm}} \times \textcolor{red}{\displaystyle\frac{14.70\;\text{psi}}{1\;\cancel{\text{atm}}}} = 73.50\;psi
      \end{displaymath}
%
   \item Or, in a more complex example:
      \begin{eqnarray}
        h_{7} &=& h_{6} + v_{6}\left(P_{7}-P_{6}\right) \nonumber \\
              &=& 706.9\textcolor{red}{\frac{kJ}{kg}} + 1.1111\times 10^{-3}\textcolor{blue}{\frac{m^{3}}{kg}}\left(210.0-7.4\right)\textcolor{blue}{bar} \nonumber \\
              &=& 706.9\textcolor{red}{\frac{kJ}{kg}} + 1.1111\times 10^{-3}\textcolor{blue}{\frac{\cancel{m^{3}}}{\cancel{kg}}} 202.6\;\textcolor{blue}{\cancel{bar}} \textcolor{red}{\frac{10^{5}\;\frac{\cancel{kg}}{\cancel{m}.\cancel{s^{2}}}}{1\; \cancel{bar}}} \textcolor{red}{\frac{10^{-3}\; \frac{kJ}{kg}}{1\;\frac{\cancel{m^{2}}}{\cancel{s^{2}}}}} \nonumber \\
              &=& 729.41\textcolor{red}{\frac{kJ}{kg}} \nonumber 
      \end{eqnarray} 
%
\end{enumerate}


\clearpage

%%% SUBSECTION
\subsection{Examples}

\begin{enumerate}[1)]
%%%
%%% EXAMPLE 
%%%
   \item\label{Mod01Ex01} If $P_{1}$ = 3.00 atm, $V_{1}$ = 500 cm$^{3}$, $P_{2}$ = 1.00 atm and $V_{2}$ = 2000 cm$^{3}$. Calculate the work, $W_{\text{rev}}$ (in $J$), for the expansion processes shown in Figs.~\ref{Mod01Fig01} (a) and (b).
      \begin{figure}[h]
         \begin{center}
           \includegraphics[width=.6\columnwidth,clip]{./Figs/Mod1Ex1}
           \vspace{-.1cm}\caption{Expansion processes (Example~\ref{Mod01Ex01}).}\label{Mod01Fig01}
         \end{center}
       \end{figure}

% SOLUTION
       \noindent{\bf Solution:} $W_{\text{rev}}$ is related to the area below the curve. Thus, we should use the $PV$ work equation:
           \begin{itemize}
              \item For process {\bf (a)}: 
                 \begin{eqnarray}
                    d W = -PdV \Longrightarrow W &=& -P_{2}\left(V_{2}-V_{1}\right) \nonumber \\
                                                 &=& - 1\text{ atm}\left(2000-500\right)\text{ cm}^{3} = -1500\text{ atm.cm}^{3} \nonumber 
                 \end{eqnarray}
                 Now we need to convert {\it atm.cm}$^{3}$ to $J$, thus using the unit conversion table:
                 \begin{eqnarray}
                     W &=& -1500\blue{\cancel{\text{ atm}}}.\red{\cancel{\text{cm}^{3}}} \frc{1.01325\times 10^{5}\blue{\cancel{\text{ Pa}}}}{1\blue{\cancel{\text{ atm}}}} \frc{1 \frc{\text{kg}}{\text{m.s}^{2}}}{1\blue{\cancel{\text{ Pa}}}} \frc{1\text{ m}^{3}}{ 100^{3} \red{\cancel{\text{ cm}^{3}}}} \nonumber \\
                       &=& -151.9875 \blue{\cancel{\frc{\text{ kg.m}^{2}}{\text{s}^{2}}}} \frc{ 1 \text{ J}}{ \blue{\cancel{\frc{\text{ kg.m}^{2}}{\text{s}^{2}}}}} \nonumber\\
                       &=& -151.9875\text{ J} \nonumber
                 \end{eqnarray}
%
              \item For process {\bf (b)}, since $P$ is constant, i.e., $P_{1}=P_{2}$:
                 \begin{eqnarray}
                    d W = -PdV \Longrightarrow W &=& -P_{1}\left(V_{2}-V_{1}\right) \nonumber \\
                                                 &=& - 3\text{ atm}\left(2000-500\right)\text{ cm}^{3} = -4500\text{ atm.cm}^{3} \nonumber \\
                                                 &=& -455.9625\text{ J} \nonumber
                 \end{eqnarray}
           \end{itemize}
\clearpage

%%%
%%% EXAMPLE
%%%
   \item Calculate the internal energy (in $J$) when 1 mol of water is isobarically heated from 25$^{\circ}$C to 30$^{\circ}$C at 1 atm. Given: densities of water are 0.9970 g.cm$^{-3}$ at 0$^{\circ}$C and 0.9956 g.cm$^{-3}$ at 100$^{\circ}$C. Molar mass and heat capacity at constant pressure of water are 18 g.mol$^{-1}$ and 1 cal.$\left(\text{g.}^{\circ}\text{C}\right)^{-1}$, respectively.

% SOLUTION
       \noindent{\bf Solution:} From the 1$^{\text{st}}$ Law, $U=Q+W$ ,and in order to calculate $U$ we first need to obtain heat ($Q$) and work ($W$). $Q$ can be obtained from the heat capacity equation
          \begin{displaymath}
             Q = m C_{p} \Delta T,
          \end{displaymath}  
          where $m$ is the mass of water can be obtained from
          \begin{displaymath}
             n = \frc{m}{MW} \Longrightarrow  m = n.MW = 1\text{ mol} . 18 \frc{\text{g}}{\text{mol}} = 18 \text{ g}
          \end{displaymath}
          $n$ and $MW$ are number of moles and molar mass, respectively. Now,
          \begin{displaymath}
             Q = m C_{p} \Delta T = 18\text{ g} . 1 \frc{\text{ cal}}{\text{g.}^{\circ}\text{C}}.\left(30-25\right)^{\circ}\text{C} = 90\text{ cal}
          \end{displaymath}  
          Now, we should calculate the work through $W=-P\Delta V$, however $V$ is not known, but we can obtain it from the density relation $V=m/\rho$, thus
          \begin{displaymath}
             W = - P\Delta V = -P\left(V_{2}-V_{1}\right) = -P\left(\frc{m}{\rho_{2}} - \frc{m}{\rho_{1}}\right) = -0.025\text{atm.cm}^{3} = -0.0006\text{ cal} 
          \end{displaymath}
          Now, calculating the internal energy,
          \begin{displaymath}
             U = Q + W = 89.9994\cancel{\text{ cal}} . \frc{ 4.186\text{ J}}{1\cancel{\text{ cal}}} = 376.7375 \text{ J}
          \end{displaymath}

\clearpage
%%%
%%% EXAMPLE
%%%
   \item A 0.3 m$^{3}$  tank contains oxygen initially at 100 kPa and 300 K. A paddle wheel within the tank is rotated until the pressure inside rise to 150 kPa. During the process 2 kJ of heat is lost to the surroundings. Determine the paddle-wheel work done (in kJ). Neglect the energy stored in the paddle wheel and assume the heat capacity at constant volume of oxygen is 0.6745 kJ.$\left(\text{kg.K}\right)^{1}$. Given molar mass of oxygen of 32 g.mol$^{-1}$.

% SOLUTION
       \noindent{\bf Solution:} The volume of the tank remains constant $V_{2}=V_{1}=V=$ 0.3 m$^{3}$ during the whole process. Thus, we can define:
            \begin{center}
              \begin{tabular}{l l l l}
                 Initial Condition & $P_{1}=$ 100 kPa   & $T_{1}=$ 300 K         & $V_{1}=$ 0.3 m$^{3}$  \\
                 Final Condition   & $P_{2}=$ 150 kPa   &                       & $V_{2}=$ 0.3 m$^{3}$ \\
                 Heat Loss         & $Q=$ -2 kJ        &                       &                    
              \end{tabular}
            \end{center}
       The compression occurs in a closed system (i.e., constant mass) and, as there is no further information, we may consider that oxygen behaves as an ideal gas. The work added to the system by the paddle can be expressed by $U=Q+W$. Our first step is to calculate $T_{2}$,
       \begin{displaymath}
          \frc{P V}{T} = \text{ constant} \Rightarrow \frc{P_{1}V_{1}}{T_{1}} = \frc{P_{2}V_{2}}{T_{2}} \Rightarrow \frc{P_{1}}{T_{1}}=\frc{P_{2}}{T_{2}} \Rightarrow T_{2} = 450\text{ K}
       \end{displaymath}
        The specific internal energy can defined by the fundamental relation, $du=C_{v}dT$ (check the units for this relation!), where $C_{v}$ is the heat capacity at constant volume thus,
       \begin{displaymath}
          \Delta U = m\Delta u = C_{v}\Delta T = Q + W,
       \end{displaymath}
       therefore, we need to obtain the mass of oxygen in the tank through the ideal gas equation of state,
       \begin{eqnarray}
         P V = n R T \Rightarrow n = \frc{m}{MW} = \frc{P V}{R T} \Rightarrow m &=& \frc{ MW P V}{R T} \nonumber \\
                                                      &=& \frc{ 32\frc{\text{ g}}{\text{mol}} 100\text{ kPa} . 0.3\text{ m}^{3}}{ 8.3143 \frc{\text{J}}{\text{mol.K}} 300\text{ K}} \nonumber \\
                                                      &=& 0.3848 \frc{\text{ g.kPa.m}^{3}}{\text{J}} \frc{1 \text{ J}}{1\text{ N.m}} \frc{1000 \text{ Pa}}{1 \text{ kPa}} \frc{ 1 \text{ N.m}^{-2}}{1\text{ Pa}} \nonumber\\
                                                      &=& 0.3848\text{ kg}\nonumber
       \end{eqnarray}
       Thus
       \begin{eqnarray}
          \Delta u = m C_{v}\Delta T = Q + W \Rightarrow W &=& m C_{v}\left(T_{2}-T_{1}\right) - Q \nonumber \\
                                                          &=& 0.3848\text{ kg} \times 0.6745 \frc{\text{kJ}}{\text{kg.K}}\times (450-300)\text{ K} - (-2 \text{ kJ}) \nonumber \\
                                                          &=& 40.9321\text{ kJ}. \nonumber
       \end{eqnarray}
       The paddle-wheel executed 40.9321 kJ of work to the system.

\clearpage
%%%
%%% EXAMPLE
%%%
   \item  Air initially occupying 1 m$^{3}$ at 1.5 bar and 20$^{\circ}$C undergoes an internally reversible compression for which $P V^{\gamma}=$ constant to a final state where the pressure is 6 bar and the temperature is 120$^{\circ}$C. Determine:
        \begin{enumerate}[(a)]
           \item Value of $\gamma$;
           \item Work and the heat transfer (in kJ).
       \end{enumerate}
       Assume that heat capacity at constant volume and molar mass of air are 0.718 kJ.$\left(\text{kg.K}\right)^{-1}$ and 29 g.mol$^{-1}$. 

% SOLUTION
       \noindent{\bf Solution:}
        \begin{center}
           \begin{tabular}{c c c c}
               {\bf State}  &   $P$ (atm)  &   $T\;\left(^{\circ}\text{C}\right)$ & $V\;\left(\text{m}^{3}\right)$ \\
                    1       &     1.5      &            20                      & 1  \\
                    2       &     6.0      &            120                     &     \\              
           \end{tabular}
        \end{center}
        \begin{enumerate}[(a)]
           \item In order to calculate the isentropic index $\gamma$, we can use any of the relations learned in the lecture,
               \begin{displaymath} 
                   P V^{\gamma} = C, \hspace{1cm} T V^{\gamma-1} = C \hspace{1cm} \text{ and/or } \hspace{1cm} T P^{\frac{1-\gamma}{\gamma}} = C.
               \end{displaymath}
               Although $P$ and $V$ is the relation initially given in the problem, we do not know the final volume of air, $V_{2}$. However, initial and final pressure and temperature are known and we can make use of this relationship to obtain $\gamma$,
               \begin{eqnarray} 
                   T P^{\frac{1-\gamma}{\gamma}} = C \Rightarrow T_{1}P_{1}^{\frac{1-\gamma}{\gamma}} = T_{2}P_{2}^{\frac{1-\gamma}{\gamma}} \nonumber \\
                   \left(\frc{P_{1}}{P_{2}}\right)^{\frac{1-\gamma}{\gamma}} = \frc{T_{2}}{T_{1}} \Rightarrow \frc{1-\gamma}{\gamma} = \frc{\ln{\frac{T_{2}}{T_{1}}}}{\ln{\frac{P_{1}}{P_{2}}}} \Longrightarrow \gamma = 1.2686  \nonumber
               \end{eqnarray}
                 
           \item Compression work can be obtained by integrating $\d W = -P dV$ from state 1 to state 2 with $P=\frac{C}{V^{\gamma}}$,
               \begin{eqnarray}
                 W_{1-2} &=& -\int\limits_{V_{1}}^{V_{2}} P dV = -\int\limits_{V_{1}}^{V_{2}}\frc{C}{V^{\gamma}} dV = -\left.\frc{C}{1-\gamma}V^{1-\gamma}\right|_{V_{1}}^{V_{2}} \nonumber \\
                         &=& -\frc{C}{1-\gamma}\left(V_{2}^{1-\gamma}-V_{1}^{1-\gamma}\right),\;\;\text{ however, as } C=P_{1}V_{1}^{\gamma}=P_{2}V_{2}^{\gamma} \nonumber \\
                         &=& -\frc{P_{2}V_{2}^{\gamma}V_{2}^{1-\gamma} - P_{1}V_{1}^{\gamma}V_{1}^{1-\gamma}}{1-\gamma} = \frc{P_{1}V_{1}-P_{2}V_{2}}{1-\gamma} \nonumber
               \end{eqnarray}
               The relation above, although important, can not be used as we do not know $V_{2}$, however we can change variables through the ideal gas relation
               \begin{eqnarray}
                  P V = n R T &\Rightarrow& n = \frc{P V}{R T } = \frc{1.5\text{ atm} \times 1\text{ m}^{3}}{ 8.3143 \frc{\text{J}}{\text{mol.K}} 293.15\text{ K}} \frc{1.01325\times 10^{5} \text{ Pa}}{1 \text{ atm}} \frc{ 1 \text{ N.m}^{-2}}{1\text{ Pa}}\frc{1 \text{ J}}{1\text{ N.m}} = 62.3580\text{ moles}\nonumber \\
                   % m &=& \frc{P_{1} V_{1} MW}{R T_{1}} = \frc{1.5\text{ atm} \times 1\text{ m}^{3} \times 29 \frac{\text{g}}{\text{mol}} } { 8.3143 \frc{\text{J}}{\text{mol.K}} 293.15\text{ K}} \frc{1 \text{ J}}{1\text{ N.m}} \frc{1.01325\times 10^{5} \text{ Pa}}{1 \text{ atm}} \frc{ 1 \text{ N.m}^{-2}}{1\text{ Pa}} = 1808.38 \text{ g}\nonumber\\
                    W &=& \frc{P_{1}V_{1}-P_{2}V_{2}}{1-\gamma} = nR\frc{T_{1}- T_{2}}{1-\gamma} \nonumber \\
                      &=& 62.3580\text{ mol} \times 8.3143 \frc{\text{J}}{\text{mol.K}} \times \frc{\left(293.15-393.15\right)\text{ K}}{1-1.2686} \nonumber \\
                      &=& 193024.2440 \text{ J} \Longrightarrow W = -193.02 \text{ kJ} \nonumber
               \end{eqnarray}
              Heat ($Q$) can be obtained from
               \begin{eqnarray}
          \Delta u &=& m C_{v}\Delta T = n MW C_{v}\Delta T = Q + W  \nonumber \\
             Q &=& n MW C_{v}\left(T_{2}-T_{1}\right) - W \nonumber \\
               &=& 62.3580\text{ mol}\times 29 \frc{\text{g}}{\text{mol}} \red{\frc{1\text{ kg}}{1000\text{ g}}}\times 0.718\frc{\text{kJ}}{\text{kg.K}}\times (393.15-293.15)\text{ K} - 193.30 \text{ kJ} \nonumber \\
                                                          &=& -63.46\text{ kJ}. \nonumber                   
               \end{eqnarray}
       \end{enumerate}


\clearpage
%%%
%%% EXAMPLE
%%%
   \item Two kg of water at 80$^{\circ}$C is mixed adiabatically with 3 kg of water at 30$^{\circ}$C in a constant pressure process of 1 atm. Find the increase in entropy $\left(\text{in kJ.K}^{-1}\right)$ of the total mass of water due to the mixing process. Assume that $C_{p}$ of water is 4.187 kJ.$\left(\text{kg.K}\right)^{-1}$. 

% SOLUTION
       \noindent{\bf Solution:} From the classic thermodynamic definition of reversible entropy processes (second law),
         \begin{displaymath}
            d S = \frc{d Q}{T}
         \end{displaymath}
         with $dQ = m C_{p}dT$, then integrating 
         \begin{eqnarray}
            \int\limits_{S_{i}}^{S_{f}} dS &=& m C_{p}\int\limits_{T_{i}}^{T_{f}}\frc{dT}{T} \nonumber \\
            \Delta S &=& S_{f}-S_{i} = m C_{p} \ln{\frc{T_{f}}{T_{i}}} \nonumber
         \end{eqnarray}
         Assuming that the system is adiabatic, we can calculate the final temperature $\left(T_{f}\right)$ from thermal energy conservation principles: 
         \begin{displaymath}
            m_{1}C_{p,1}T_{1} + m_{2}C_{p,2}T_{2} = \left(m_{1}+m_{2}\right)C_{p}T_{f} \rightarrow T_{f} = 50^{\circ}\text{C} = 323.15\text{ K}
         \end{displaymath}
         And the entropies are
         \begin{eqnarray}
             \Delta S_{1} &=& m_{1}C_{p,1}\ln{\frc{T_{f}}{T_{1}}} =  -0.7434\frc{\text{kJ}}{\text{K}} \nonumber \\
             \Delta S_{2} &=& m_{2}C_{p,2}\ln{\frc{T_{f}}{T_{2}}} =  0.8025\frc{\text{kJ}}{\text{K}} \nonumber 
         \end{eqnarray}
         Thus the increase in entropy of the total mass is
         \begin{displaymath}
              \Delta S_{\text{mix}} = \Delta S_{1} + \Delta S_{2} = 0.0591\frc{\text{kJ}}{\text{K}}
         \end{displaymath}
%
\end{enumerate}
\end{comment}


%%%
%%%  BIBLIOGRAPHY
%%%
%\bibliography{refbib}
 % Introduction and Review of Basic Concepts of Thermodynamics
     \setcounter{examplecounter}{0}

  
%%%
%%% CHAPTER
%%%
\chapter{Introduction to Properties of Gases}\label{Chapter:Intro_Property_of_Gases}

   \begin{LearningObjectivesBlock}{Learning Objectives}
      Upon completion of this chapter, you will be able to
        \begin{enumerate}
           \item define ideal gas and identify the main assumptions;
           \item differentiate an ideal from a real gas through state conditions;
           \item explain Boyle's, Gay-Lussac's and Charles' laws for ideal gases;
           \item define and explain the equation of state for ideal gases;
           \item state the Dalton's law for gaseous mixtures.
        \end{enumerate}
\medskip
     Recommended reading: Chapter 1 of \citet{Atkins_Book,Adamson_BookChapter}.
   \end{LearningObjectivesBlock}

   
%%%
%%% SECTION
%%%
     \section{Introduction}\label{Chapter:Intro_Property_of_Gases:Section:Intro}

   The physical condition (\ie state) of a substance is defined by its physical properties. For example, the state of a pure fluid is specified by volume ($V$), mass (through the number of moles, $n$), pressure ($P$) and temperature ($T$). Laboratory experiments with fluids revealed that if three of these properties (\eg $V, n, T$) are specified, the state of the fluid is naturally defined, and the forth property (\ie $P$) is fixed. Mathematically, this can be represented by a functional often referred as {\bf equation of state}\index{Equation of State} (EOS)\index{EOS|see {Equation of State}},
     \begin{displaymath}
       P = f(T,V,n),
     \end{displaymath}
     which states that if $T$, $n$ and $V$ are known for a particular fluid, then the pressure of the system for the fluid at this state can be readily determined. The state and properties of any chemical species can be described by an specific equation of state -- EOS are the main focus of Chapter~\ref{Chapter:VolumetricPropertiesPureFluids}
   
%%%
%%% SECTION
%%%
     \section{Ideal Gases}\label{Chapter:Intro_Property_of_Gases:Section:IdealGases}\index{Gases!Ideal gas}

     \noindent A fluid is assumed to behave as an {\it ideal gas} if the following two assumptions are true:
     \begin{enumerate}[i)]
       \item molecules of gases are considered as {\bf massless} particles;
       \item there are {\bf no} interaction between the gas particles due to,
         \begin{enumerate}[a)]
           \item volume of the molecules is negligibly small compared with the volume occupied by the gas, and/or;
           \item the distance between gaseous molecules are infinitely large $\left(d\rightarrow \infty\right)$, except during elastic collisions over negligible duration.
         \end{enumerate}
     \end{enumerate}
     \noindent These lead to an obvious conclusion that for a gas to behave as an ideal gas the pressure should be infinitely small, \ie $P\rightarrow 0$. In practical engineering calculations, we assume that a fluid behaves as an ideal gas at low to moderate pressures. The $PVT$ behaviour of fluids was initially investigated through experimental observations of gases at low pressures that led to three intuitive relationships:
     \begin{itemize}
       \item For isothermal processes in closed systems (\ie $T$ and $n$ are constant), $P$ and $V$ are inversely proportional to each other, \ie an increase in pressure leads to a decrease in volume,
          \begin{displaymath}
             P \propto \frc{1}{V},
          \end{displaymath}          
%          
       \item For isochoric processes in closed systems (\ie $V$ and $n$ are constant), $T$ and $P$ are directly proportional to each other, \ie an increase in temperature leads to an increase in pressure,
          \begin{displaymath}
            T \propto P,
          \end{displaymath}         
%
       \item For isobaric processes in closed systems (\ie $P$ and $n$ are constant), $T$ and $V$ are directly proportional to each other, \ie an increase in temperature leads to an increase in volume.
          \begin{displaymath}
            T\propto V.
          \end{displaymath}          
     \end{itemize}
     \begin{shaded}
        These proportionality relations can be merged into a single expression,\index{Gases!Boyle's law}\index{Gases!Gay-Lussac's law}\index{Gases!Charles' law}\index{Gay-Lussac's law|see {Gases}}\index{Charles' law|see {Gases}}\index{Boyle's law|see {Gases}}
          \begin{equation}
            \frc{P V}{n T} = R \;\;\Longleftrightarrow \;\;
              \begin{cases}
                P_{1}V_{1} = P_{2}V_{2}, & \text{if } T \text{ and } n \text{ are constant (Boyle's law)},  \\
         \\
                P_{1}T_{1}^{-1} = P_{2}T_{2}^{-1}, & \text{if } V \text{ and } n \text{ are constant (Gay-Lussac's law)}, \\
         \\
                V_{1}T_{1}^{-1} = V_{2}T_{2}^{-1}, & \text{if } P \text{ and } n \text{ are constant (Charles' law)}.
             \end{cases}\label{Chapter:Intro_Property_of_Gases:Eqn:IdealEOS}\index{Equation of State!Ideal gas}
          \end{equation}
        The constant of proportionality, $R$, which is found experimentally to be the same for all gases, is called {\bf universal gas constant} (Table~\ref{Chapter:Intro_Property_of_Gases:Table:RConst}). This expression, 
          \begin{equation}
             P=\frc{n R T}{V},\label{Chapter:Intro_Property_of_Gases:Eqn:IdealEOS}\index{Equation of State!Ideal gas}
          \end{equation}\index{Equation of State!Ideal gas}
        is called {\bf ideal gas equation of state} and becomes increasingly accurate for any gas as pressure tends to zero $\left(P\rightarrow 0\right)$.
     \end{shaded}
     
   \begin{table}[h]
     \begin{center}
     \begin{tabular}{||c c||}
       \hline\hline
           $\mathbf{R}$   & {\bf Units} $\mathbf{\left(\text{V.P.T}^{-1}\text{n}^{-1}\right)}$ \\
           \hline\hline
           8.3145    &  J.K$^{-1}$.mol$^{-1}$  \\
           8.3145    &  kJ.K$^{-1}$.kmol$^{-1}$  \\
           8.3145    &  l.kPa.K$^{-1}$.mol$^{-1}$  \\
           8.3145$\times$10$^{-3}$    & cm$^{3}$.kPa.K$^{-1}$.mol$^{-1}$  \\
           8.3145    &  m$^{3}$.Pa.K$^{-1}$.mol$^{-1}$  \\
           8.3145$\times$10$^{-5}$    &  m$^{3}$.bar.K$^{-1}$.mol$^{-1}$  \\
           8.2057$\times$10$^{-2}$ &  l.atm.K$^{-1}$.mol$^{-1}$  \\
           \hline\hline           
     \end{tabular}
     \caption{Gas constant, $R$.}\label{Chapter:Intro_Property_of_Gases:Table:RConst}\index{Universal gas constant}\index{R|see {Universal gas constant}}\index{Gas constant|see {Universal gas constant}}
     \end{center}
   \end{table}
   
   % Example
   \begin{MyExample}{\begin{center}{\bf Example}\end{center}}
     \begin{example}\label{Chapter:Intro_Property_of_Gases:Example1}
       In an industrial process, nitrogen is heated to 650.15 K in a vessel of constant volume. If it enters the vessel at 43 atm and 298.15 K, what pressure would it exert at the working temperature if it behaved as an ideal gas?

       {\it This problem deals with a gas that undertakes an isochoric (\ie constant volume) process. Nitrogen gas at $P_{1} = 43$ atm and $T_{1} = 298.15$ K is compressed to pressure $P_{2}$ 'till temperature reaches $T_{2} = 650.15$ K.

         Using the ideal gas equation for states 1 and 2,}
       \begin{eqnarray}
         && \left(\frc{PV}{nT}\right)_{1} = R = \left(\frc{PV}{nT}\right)_{2},\;\;\;\text{ with } V_{1}=V_{2} \text{ and }\;\; n_{1}=n_{2}\nonumber \\
         && \frc{P_{1}}{T_{1}} = \frc{P_{2}}{T_{2}} \;\;\Rightarrow \;\; P_{2}= 93.7664 \text{ atm } \nonumber
         \end{eqnarray}
     \end{example}
   \end{MyExample}

\medskip
   % Example
   \begin{MyExample}{\begin{center}{\bf Example}\end{center}}
     \begin{example}\label{Chapter:Intro_Property_of_Gases:Example2}
       500 kg of helium gas is stored in a tank at 50$^{\circ}$C and 2.5 bar. The fluid is then transferred to a pressure vessel where it is isothermically compressed to 73 bar.  What is the volume occupied by {\it He} in such conditions? Assume ideal gas behaviour. Molar mass ({\it MW}) of helium is 4.0026 g.mol$^{-1}$.

       {\it In this problem, we need to calculate the volume of helium at the pressure vessel after isothermal compression. However, not all necessary conditions are known, \ie}
       \begin{displaymath}
         \frc{P_{1}V_{1}}{n_{1} R T_{1}} = \frc{P_{2}V_{2}}{n_{2} R T_{2}} \Longrightarrow P_{1}V_{1} = P_{2}V_{2},\;\;\;\;\text{where }R\text{ and } n_{i}\text{ are constants},
       \end{displaymath}
           {\it thus there is 1 equation and 2 unknowns, $V_{1}$ and $V_{2}$. In order to solve this problem, we need to first compute $n_{1}$ and $V_{1}$ via, }
           \begin{displaymath}
              V_{1} = \frc{n_{1}R T_{1}}{P_{1}} = 1342.5427\;\text{m}^{3},\;\;\;\text{ where } n_{1} = \frc{m_{1}}{MW_{1}} =  124918.8028\text{ moles}.
           \end{displaymath}
           {\it With $V_{1}$, we can now calculate the volume after compression, $V_{2}$}
           \begin{displaymath}
              P_{1}V_{1} = P_{2}V_{2} \;\;\; \Longrightarrow\;\;\; V_{2} = 45.9775 \text{m}^{3}
           \end{displaymath}
           
     \end{example}
   \end{MyExample}
   
   % Example
   \begin{MyExample}{\begin{center}{\bf Example}\end{center}}
     \begin{example}\label{Chapter:Intro_Property_of_Gases:Example3}
       \citep{Atkins_Book} 
       \begin{enumerate}[a)]
           \item Deduce a relation between pressure and mass density $\left(\rho\right)$ of an ideal gas of molar mass $MW$;
           \item After careful measurement of the density of dimethyl ether at relatively low pressure conditions, a chemical engineering student obtained the following experimental data at 25$^{\circ}$C,
               \begin{center}
                  \begin{tabular}{c c c c c c}
                     $\mathbf{P}$ [kPa]  & 12.223 & 25.20 & 36.97 & 85.23 & 101.3 \\
                     $\mathbf{\rho}\;\left[\text{kg.m}^{-3}\right]$ & 0.225  &  0.456  &  0.664 & 1.468 & 1.734  
                  \end{tabular}
               \end{center}
               Obtain the molar mass of dimethyl ether at each pressure coordinate and compare with the real value of 46.07 g.mol$^{-1}$. What conclusions could be drawn from this data?
       \end{enumerate}
\medskip

       \begin{enumerate}[a)]
           \item {\it The first part of the problem requires the development of a mathematical relation between $P$ and $\rho$. Defining} $\rho=\frc{m}{V}$ {\it which can be allocated in the ideal EOS} $\left(\text{ and }n=\frc{m}{MW}\right)$,
              \begin{eqnarray}
                 P &=& \frc{n R T }{V} = \frc{m}{MW}\frc{R T}{V} \\ 
                   &=& \frc{\rho R T}{MW}
              \end{eqnarray}
%
           \item {\it We can use the relation derived in part (a) to calculate the molar mass of dimethyl ether at each pressure coordinate, thus}
              \begin{displaymath}
                  \begin{cases}
                     MW_{1} = \frc{\rho_{1} R T}{P_{1}} = \frc{0.225\text{ kg.m}^{-3}\cdot 8.3145\text{ m}^{3}\text{.Pa.}\left(\text{K.mol}\right)^{-1}\cdot 298.15\text{ K}}{12.223\times 10^{3}\text{ Pa}} = 45.6326 \text{ g.mol}^{-1} \\
                     MW_{2} = \frc{\rho_{2} R T}{P_{2}} = \frc{0.456\text{ kg.m}^{-3}\cdot 8.3145\text{ m}^{3}\text{.Pa.}\left(\text{K.mol}\right)^{-1}\cdot 298.15\text{ K}}{25.20\times 10^{3}\text{ Pa}} = 44.8575 \text{ g.mol}^{-1} \\
                     MW_{3} = \frc{\rho_{3} R T}{P_{3}} = \frc{0.664\text{ kg.m}^{-3}\cdot 8.3145\text{ m}^{3}\text{.Pa.}\left(\text{K.mol}\right)^{-1}\cdot 298.15\text{ K}}{36.97\times 10^{3}\text{ Pa}} = 44.5235 \text{ g.mol}^{-1} \\
                     MW_{4} = \frc{\rho_{4} R T}{P_{4}} = \frc{1.468\text{ kg.m}^{-3}\cdot 8.3145\text{ m}^{3}\text{.Pa.}\left(\text{K.mol}\right)^{-1}\cdot 298.15\text{ K}}{85.23\times 10^{3}\text{ Pa}} = 42.6977 \text{ g.mol}^{-1} \\
                     MW_{5} = \frc{\rho_{5} R T}{P_{5}} = \frc{1.734\text{ kg.m}^{-3}\cdot 8.3145\text{ m}^{3}\text{.Pa.}\left(\text{K.mol}\right)^{-1}\cdot 298.15\text{ K}}{101.3\times 10^{3}\text{ Pa}} = 42.4337 \text{ g.mol}^{-1} \\
                  \end{cases} 
              \end{displaymath}
              {\it We can compare these values with the real molar mass through the relative error ($\%$),}
                 \begin{displaymath}
                      \epsilon = \frc{\left| MW_{i} - MW^{\text{exp}}\right|}{MW^{\text{exp}}}\times 100,
                 \end{displaymath}
                 {\it leading to}
               \begin{center}
                  \begin{tabular}{c |c c c c c}
                     $\mathbf{P}$ [kPa]                            & 12.223  & 25.20   & 36.97   & 85.23   & 101.3    \\
                     $\mathbf{\rho}\;\left[\text{kg.m}^{-3}\right]$ &  0.225  &  0.456  &  0.664  &  1.468  &   1.734  \\
                     $\mathbf{MW}\;\left[\text{g.mol}^{-1}\right]$  & 45.6326 & 44.8575 & 44.5235 & 42.6977 &  45.4337 \\
                     $\mathbf{\varepsilon}\;\left[\%\right]$       &  0.9494 &  2.6319 &  3.3568 &  7.3199 &   7.8930
                  \end{tabular}
               \end{center}
               {\it From this data, it is clear that as $P\rightarrow 0$ dimethyl ether behaves similar to an ideal gas. However, as pressure increases the ideal gas EOS becomes less accurate.} 

                
       \end{enumerate}

           
     \end{example}
   \end{MyExample}
   

%%%
%%% SECTION
%%%
   \section{Gas Mixtures}\label{Chapter:Intro_Property_of_Gases:Section:MixtureGases}\index{Gases!Mixture }\index{Gases!Dalton's law}\index{Dalton's law|see {Gases}}\index{Pressure!Partial}\index{Partial pressure|see {Pressure}}

   \begin{subequations}
     Let's consider a gaseous mixture containing $\mathcal{N}$ chemical species. In several applications it is necessary to determine the pressure that each gas exerts in the whole system, \ie the contribution of each gas in the total pressure of the mixture. This contribution is often referred as {\it partial pressure}, $P_{i}$, of the gas {\it i} in a gas mixture and is defined as
     \begin{equation}
        P_{i} = y_{i}P,\label{Chapter:Intro_Property_of_Gases:Eqn:PartialPressure_1}
     \end{equation}
     where $P$ is the total pressure and $y_{i}$ is the {\it mole fraction} of component $i$,\index{Mole fraction}
     \begin{displaymath}
        y_{i} = \frc{n_{i}}{n},\;\;\;\text{ with }\;\;i=1,\cdots,\mathcal{N}.
     \end{displaymath}
     The mole fraction is a normalised quantity and as such it must sum to unity, 
     \begin{displaymath}
        \summation[y_{i}]{i=1}{\mathcal{N}} = 1
     \end{displaymath}
     If we sum up the partial pressures of all components in the gaseous mixture, the total pressure of the system is recovered, \ie
     \begin{equation}
       \summation[P_{i}]{i=1}{\mathcal{N}} = \summation[y_{i}P]{i=1}{\mathcal{N}} = P\label{Chapter:Intro_Property_of_Gases:Eqn:PartialPressure_2}
     \end{equation}
     This equation is also known as {\it Dalton's law} and it is valid for any ideal or real gaseous mixtures. It can be stated as
     \begin{shaded}
       ``The pressure exerted by a mixture of gases is the sum of the pressures that each one would exist if it occupied the container alone'' \citep{Atkins_Book}.
     \end{shaded}

   \end{subequations}
   

\medskip
   % Example
   \begin{MyExample}{\begin{center}{\bf Example}\end{center}}
     \begin{example}\label{Chapter:Intro_Property_of_Gases:Example4}
       A gaseous mixture of methane, ethane and ethylene is stored in a tank at 6 atm. The composition (in weight) of the mixture is 42.16$\%$ of CH$_{4}$ and 32.11$\%$ of C$_{2}$H$_{6}$. What is the partial pressure of each gas in the mixture? Molar mass ({\it MW}) of methane, ethane and ethylene are 16.04 g.mol$^{-1}$, 30.07 g.mol$^{-1}$ and 28.05 g.mol$^{-1}$, respectively.
\medskip

       {\it Partial pressure of CH$_{4}$, C$_{2}$H$_{6}$ and C$_{2}$H$_{4}$ can be obtained from Eqn.~\ref{Chapter:Intro_Property_of_Gases:Eqn:PartialPressure_1}, however we first need to calculate the number of moles of each species in the mixture. Assuming a 100 g mixture,}
       \begin{displaymath}
         \begin{cases}
           n_{\text{CH}_{4}} = \frc{m_{\text{CH}_{4}}}{MW_{\text{CH}_{4}}} = \frc{42.16\text{ g}}{16.04\text{ g.mol}^{-1}}= 2.6284\text{ moles},  \\
           n_{\text{C}_{2}\text{H}_{6}} = \frc{m_{\text{C}_{2}\text{H}_{6}}}{MW_{\text{C}_{2}\text{H}_{6}}} = \frc{32.11\text{ g}}{30.07\text{ g.mol}^{-1}}= 1.0678\text{ moles},  \\
           n_{\text{C}_{2}\text{H}_{4}} = \frc{m_{\text{C}_{2}\text{H}_{4}}}{MW_{\text{C}_{2}\text{H}_{4}}} = \frc{25.73\text{ g}}{28.05\text{ g.mol}^{-1}}= 0.9173\text{ moles}. 
         \end{cases}
       \end{displaymath}
       {\it The mole fraction of each species can be obtained, considering}
       \begin{displaymath}
         n = \summation[n_{i}]{i=1}{3} = n_{\text{CH}_{4}}+n_{\text{C}_{2}\text{H}_{6}}+n_{\text{C}_{2}\text{H}_{4}} = 4.6135\text{ moles}
       \end{displaymath}
       {\it thus,}
       \begin{displaymath}
         \begin{cases}
           y_{\text{CH}_{4}} = \frc{n_{\text{CH}_{4}}}{n} = 0.5697  \\
           y_{\text{C}_{2}\text{H}_{6}} = \frc{n_{\text{C}_{2}\text{H}_{6}}}{n} = 0.2315,  \\
           y_{\text{C}_{2}\text{H}_{4}} = \frc{n_{\text{C}_{2}\text{H}_{4}}}{n} = 0.1988. 
         \end{cases}
       \end{displaymath}
       {\it With $y_{i}$ we can finally calculate the partial pressure of each gas:}
       \begin{displaymath}
         \begin{cases}
           P_{\text{CH}_{4}} = P\cdot y_{\text{CH}_{4}} = 3.4182\text{ atm}  \\
           P_{\text{C}_{2}\text{H}_{6}} = P\cdot y_{\text{C}_{2}\text{H}_{6}} = 1.3890\text{ atm},  \\
           P_{\text{C}_{2}\text{H}_{4}} = P\cdot y_{\text{C}_{2}\text{H}_{4}} = 1.1928\text{ atm}. 
         \end{cases}
       \end{displaymath}
       {\it We can check if the calculations are correct by}
       \begin{displaymath}
         \begin{cases}
           \summation[y_{i}]{i=1}{3} = 1.0000  & \textit{ and } \\
           \summation[P_{i}]{i=1}{3} = P = 6\text{ atm} & \\
         \end{cases}
       \end{displaymath}
           
     \end{example}
   \end{MyExample}
\medskip

   % Example
   \begin{MyExample}{\begin{center}{\bf Example}\end{center}}
     \begin{example}\label{Chapter:Intro_Property_of_Gases:Example5}
       \citep{Atkins_Book} A vessel of volume 22.4 dm$^{3}$ contains 2.0 mol H$_{2}$ and 1.0 mol N$_{2}$ at 273.15 K. Calculate (a) the mole fractions of each component, (b) their partial pressures, and (c) their total pressure. Assume ideal gas behaviour.
\medskip

      {\it Here we want to calculate the partial pressure of gaseous mixture of hydrogen and nitrogen contained in a vessel of 22.4 dm$^{3}$ at 273.15 K. Partial pressures can be obtained from the total pressure of the system and the mole fraction of the gases, } $P_{i}=y_{i}P$, {\it thus we first need to obtain $P$ through the ideal gas EOS, with} $n=n_{H_{2}}+n_{N_{2}}=3$,
      \begin{displaymath}
        P = \frc{n R T }{V} = \frc{3\text{ mol } \cdot 8.3145\times 10^{-5}\text{ m}^{3}\text{.bar.}\left(\text{K.mol}\right)^{-1}\cdot 273.15\text{ K}}{22.4\times 10^{-3}\text{ m}^{3}} = 3.0417\text{ bar, }
      \end{displaymath}
      {\it and the mole fraction of both gasses,}
       \begin{displaymath}
         \begin{cases}
           y_{\text{H}_{2}} = \frc{n_{\text{H}_{2}}}{n} = 0.6667  \\
           y_{\text{N}_{2}} = \frc{n_{\text{N}_{2}}}{n} = 0.3333  
         \end{cases}
       \end{displaymath}
       {\it Now we can finally compute partial pressures:}
       \begin{displaymath}
         \begin{cases}
           P_{\text{H}_{2}} = Py_{\text{H}_{2}} = 2.0279\text{ bar},  \\
           P_{\text{N}_{2}} = Py_{\text{N}_{2}} = 1.0138\text{ bar}.
         \end{cases}
       \end{displaymath}


       {\it We can check if the calculations are correct by}
       \begin{displaymath}
         \begin{cases}
           \summation[y_{i}]{i=1}{2} = 1.0000  & \textit{ and } \\
           \summation[P_{i}]{i=1}{2} = P = 3.0417\text{ bar} & \\
         \end{cases}
       \end{displaymath}
           
     \end{example}
   \end{MyExample}
\medskip



   
\clearpage   
\begin{FinalSummaryBlock}{Summary}
    \begin{itemize}
       \item A fluid behaves as an ideal gas at low to moderate pressures $\left(\text{\ie } P\rightarrow 0\right)$
       \item Equation of state is function that correlates pressure, volume, temperature and the amount of chemical component(s);
       \item An ideal gas obeys the ideal gas equation of state (Eqn.~\ref{Chapter:Intro_Property_of_Gases:Eqn:IdealEOS});
       \item Partial pressure of a gas in a gaseous mixture is defined as the pressure that the gas would exerted by itself (Eqn.~\ref{Chapter:Intro_Property_of_Gases:Eqn:PartialPressure_1});
       \item Dalton's law states that the pressure exerted by a mixture of gases is the sum of the partial pressures of the gases.
    \end{itemize}
\end{FinalSummaryBlock}
 % Introduction to gas properties
     \setcounter{examplecounter}{0}
  
%%%
%%% CHAPTER
%%%
\chapter{First Law of Thermodynamics}\label{Chapter:FirstLaw}

   \begin{LearningObjectivesBlock}{Learning Objectives}
      Upon completion of this chapter, you will be able to
        \begin{enumerate}
           \item Demonstrate understanding of key concepts of energy and the first law of thermodynamics;
           \item Apply the first law of thermodynamics to assess of heat transfer and power cycles;
           \item Conduct energy analysis of thermodynamic systems;
           \item Employ energy and mass balances into thermodynamic systems to assess efficiency, and correctly observe sign conventions for work and heat transfer.
        \end{enumerate}
\medskip
     Recommended reading: Chapters 2 of \citet{Atkins_Book,SmithVanNess_Book,Moran_Book} or 3 of \citet{Borgnakke_Book}.
   \end{LearningObjectivesBlock}

%%%%%%%%%%%%%%%%%%%%%%%%%%%%%%%%%%%%%%%%%%%%%%%%%%%%%%%%%%%%%%%%%
\begin{comment}
   \begin{LearningObjectivesBlock}{Learning Objectives}
      Upon completion of this chapter, you will be able to
        \begin{enumerate}
           \item {\bf Knowledge:} Define, Name, Select, State 
           \item {\bf Comprehension:} Describe, Identify, Discuss
           \item {\bf Application:} Apply, Demonstrate, Employ, Sketch
           \item {\bf Analysis:} Analyse, Compare, Calculate, Solve
           \item {\bf Synthesis:} Determine, Formulate
           \item {\bf Evaluation:} Assess, Check, Estimate, Compare, Measure, Monitor
        \end{enumerate}
\medskip
     Recommended reading: Chapters 2 of \citet{Atkins_Book,SmithVanNess_Book,Moran_Book} or 3 of \citet{Borgnakke_Book}.
   \end{LearningObjectivesBlock}
\end{comment}

   
%%%
%%% SECTION
%%%
     \section{Introduction}\label{Chapter:FirstLaw:Section:Intro}\index{Work}\index{Heat}\index{Energy}
     In Section~\ref{Chapter:Introduction:Section:ThermodAnalysis}, the main elements in the thermodynamic analysis were introduced, namely {\bf open, closed and isolated systems}, {\bf surroundings} and {\bf boundaries}. The concept of {\bf energy}, {\bf work} and {\bf heat}, pivotal entities in the study of thermodynamics systems, were also defined as,
     \begin{description}
        \item[Work] is motion against an opposing force (Eqn.~\ref{Chpt01_Work1});
        \item[Energy] of a system is its capacity to produce work, and; 
        \item[Heat] is the transfer of energy across the boundary caused by a temperature gradient at the boundary \citep{Devoe_Book}.
     \end{description}
     These definitions are based on observations of systems in a macro-scale, and are critical for mass and energy balances necessary for this chapter. ADD MORE TEXT HERE !!

%%%
%%% SECTION
%%%
     \section{The Internal Energy}\label{Chapter:FirstLaw:Section:ThermalEnergy}\index{Internal Energy}\index{Energy!Internal|see{Internal Energy}}
     \begin{subequations}
        A system, with a prescribed amount of mass, contains energy in the form of {\bf internal energy} ($U$, inherent in the internal structure), kinetic energy (linked to the motion) and potential energy (associated with external forces acting upon the mass). The total energy, $E$, associated of the system can then be expressed as 
        \begin{displaymath}
            E = \text{Internal} + \text{Kinetic} + \text{Potential} = U + E_{\text{K}} + E_{\text{P}},
        \end{displaymath}
        and the specific energy, $e$, becomes
        \begin{equation}
            e = \frc{E}{m} = u + e_{\text{K}} + e_{\text{P}} = u + \frc{1}{2}v^{2} + gz,\label{Chapter:FirstLaw:Eqn:TotalEnergy1}
        \end{equation}
        where the kinetic energy\footnote{Kinetic energy has three components: vibrational (due to the energy associated with vibration of the body), rotational (associated with the rotation motion) and translational (associated with the motion from one spatial coordinate to another).} is assumed to be due to the translational motion (thus vibrational and rotational motion are neglected) and the potential energy to be due to the constant gravitational force. In Eqn.~\ref{Chapter:FirstLaw:Eqn:TotalEnergy1}, $u$, $e_{\text{K}}$ and $e_{\text{P}}$ are specific internal, kinetic and potential energies, respectively. Kinetic and potential energy are associated with the physical state and location (spatial coordinates) of the system, and are commonly named {\it mechanical energy}\index{Energy!Mechanical}.  The internal energy is a characteristic of the thermodynamic state of the mass and is often labelled as {\it thermal energy}\index{Energy!Thermal}.
      
       \begin{shaded}
          The internal energy is a {\it state function} as its value depends only on the current state of the system and is independent of processes undertook by the system, \ie it is a function of the properties that determine the current state of the system.
       \end{shaded}

       Let's consider a {\it control volume} with a prescribed mass; an {\it energy balance} can be performed over this control volume assuming that energy can not be created or destroyed but just transformed. Thus, any change in energy must be due to an energy transfer into or out of the control volume, which can be represented as work ($W$) or heat ($Q$) transfers,
      \begin{equation}
        \frc{d\mfr[E]{}{\text{cv}}}{dt} = \mfr[\dot{E}]{}{\text{cv}} = \dot{Q} + \dot{W},\label{Chapter:FirstLaw:Eqn:TotalEnergy2}
      \end{equation}
      where the {\it dot} symbol $\left(\dot{ }\right)$ over the variables represents the rate of change, \ie $\frc{d\left[\; \right]}{dt}$. Equation~\ref{Chapter:FirstLaw:Eqn:TotalEnergy2} represents the rate of change (\ie {\it instantaneous rate}) of the total energy stored in the control volume, where part of the system's energy can be transferred from or into the surroundings of the control volume. In most cases, we are interested in finite changes of properties from the beginning of the process to its end, rather than instantaneous rate evaluations. In such cases, we just need to integrate the energy equation (Eqn.~\ref{Chapter:FirstLaw:Eqn:TotalEnergy1}) \wrt time, \ie from time $t_{1}$ to $t_{2}$, then after multiplying it by $dt$,
      \begin{equation}
        \mfr[dE]{}{\text{cv}} = dU + dE_{\text{K}} + d E_{\text{P}} = \delta Q + \delta W,\label{Chapter:FirstLaw:Eqn:TotalEnergy3}\footnote{Here, it is important to differentiate three mathematical symbols commonly used in thermodynamics: $d$, $\partial$ and $\delta$. $d$ and $\partial$ represent {\it exact (or total)} (Appendix~\ref{Appendix_Calculus:TotalDifferential}) and {\it partial} (Appendix~\ref{Appendix_Calculus:PartialDifferential}) differentials, respectively. $\delta$ is often used in thermodynamics study to represent {\it inexact differential} for heat and work as these variables are path-dependent. For simplicity, throughout this document we will only use of $d$ and $\partial$ symbols.}
      \end{equation}
      we can integrate it from {\it state 1} $\left(\text{\ie at time }t_{1}\right)$ to {\it state 2} as,
      \begin{displaymath}
        \text{\bf Left-hand side: } \int\limits_{\mfr[E]{1}{cv}}^{\mfr[E]{2}{cv}}d\mfr[E]{}{cv} = \mfr[E]{t_{2}}{cv} - \mfr[E]{t_{1}}{cv} = \mfr[E]{2}{cv} - \mfr[E]{1}{cv},
      \end{displaymath}
      \begin{displaymath}
        \text{\bf Right-hand side: } \int\limits_{t_{1}}^{t_{2}} \left|\dot{Q} + \dot{W}\right|dt = \int\limits_{\text{path}}\delta Q +  \int\limits_{\text{path}}\delta W = Q_{1-2} + W_{1-2},
      \end{displaymath}
      leading to
      \begin{equation}
          \mfr[E]{2}{cv} - \mfr[E]{1}{cv} = \left[U_{2}-U_{1}\right] + \left[\frc{1}{2}m\left(v_{2}^{2}-v_{1}^{2}\right)\right] + \left[m g \left(z_{2}-z_{1}\right)\right] = Q_{1-2} + W_{1-2}.\label{Chapter:FirstLaw:Eqn:TotalEnergy3}
      \end{equation}
      Equation~\ref{Chapter:FirstLaw:Eqn:TotalEnergy3} describes the energy balance of a system with the surroundings, where the state functions (here represented by the total, internal, kinetic and potential energies) are path-independent, whereas {\bf changes in heat and work depend on the path used in the process}. 
     \end{subequations}

%%%
%%% SECTION
%%%
     \section{Exact and Inexact Differentials}\label{Chapter:FirstLaw:Section:ExactInexactDiff}\index{Exact differential}
     \begin{subequations}
         In Chapter~\ref{Chapter:Introduction}, we briefly defined {\it state functions}\index{State function} as thermodynamic properties that are independent of the process. For example, when water is boiled isobarically from 15$^{\circ}$C to 100$^{\circ}$C, there are infinite ways that the boiling can progress (\eg intense heat in the first 5 mins and a moderate one afterwards 'till boiling or continuous moderate heating, etc). The rate of change of the internal energy is calculated based {\bf only} on the initial and final states. The way in which the heating of water is conducted does not affect the calculation of $\Delta U$. Processes that describe the changes of state (\eg from state 1 at 15$^{\circ}$C to state 2 at 100$^{\circ}$C) are often called {\bf path functions}\index{Path function}. Examples of path functions are heat ($Q$) and work ($W$), whereas $U$ is a state function. 

        Figure~\ref{Chapter:FirstLaw:Fig:StateFunctions} shows three distinct processes (paths I-III) in which the internal energy dropped from $U_{1}$ to $U_{2}$ due to changes in heat and work. If the system is taken through any these paths, the overall change from $U_{1}$ to $U_{2}$ is the sum (\ie integral) of all the infinitesimal changes along the path,
        \begin{displaymath}
             \Delta U = \int\limits_{1}^{2} dU.
        \end{displaymath}
        The value of $\Delta U$ depends on the initial (1) and final (2) states of the system but is independent of the path between them. Such path-independence of the integral is represented by $dU$ which is said to be an {\bf exact differential} (\ie an infinitesimal quantity in which the results after integration are independent of the path between states 1 and 2.)\index{Exact differential}. If the system is heated , the total energy transferred as heat is the sum (\ie integral) of all individual contributions throughout the pathway,
        \begin{displaymath}
             Q = \int\limits_{1,\text{path}}^{2}dQ.
        \end{displaymath}
        As heat is not a state function and the path affects the integration, the left-hand side is written as $Q$ rather than $\Delta Q$, \ie heat {\bf can not} be expressed as just $Q_{2}-Q_{1}$. Such path-dependent quantity is expressed by saying that $dQ$ (or $\delta Q$) is an {\bf inexact differential}\index{Inexact differential} (\ie an infinitesimal quantity in which the results after integration between initial and final states depends on the path).

        In a similar way, the work done in a system (or produced by) is also a path function and as such can be represented as either $dW$ or $\delta W$.

     \end{subequations}

% Figure
   \begin{figure}[h]
     \begin{center}
       \includegraphics[width=9cm, height=9cm]{./Figs/Chp3_State-PathFunctions}
        \caption{State function: Change of internal function as a result of work and heat being exerted into the system by 3 distinct paths. Note that regardless the chosen path (from state 1 to state 2), $U_{1}$ and $U_{2}$ remains the same, although the amount of heat (represented here by changes in temperature, $T$) and work (\ie changes in volume, $V$) may vary significantly.}\label{Chapter:FirstLaw:Fig:StateFunctions}
     \end{center}
   \end{figure}
%%%
%%% SECTION
%%%
     \section{The First Law of Thermodynamics}\label{Chapter:FirstLaw:Section:FirstLaw}\index{Laws of Thermodynamics ! First law}
     \begin{subequations}
         Thermal energy is macro-scale representation of micro-scale changes in mechanical energy, namely work and heat. In a molecular scale, atoms and molecules are in random motion that can be associated with the kinetic energy of these `particles'. Tracking the motion and energy of each particle is addressed by a field of sciences called statistical (or quantum) thermodynamics, here we are interested in the consequences of the particles' motion -- \ie oscillations in temperature as a measure of the average molecular-scale kinetic energy.

         \begin{shaded}
            \begin{center} {\bf Sign Notation}\end{center} 
              Before we proceed stating the {\it First Law}, we should define a sign notation used for all quantities in this document. Thus any form of energy:
              \begin{itemize}
                  \item Added to the system is assumed {\bf positive}, and;
                  \item Removed from the system is assumed {\bf negative}.
              \end{itemize}
              Hence:
              \begin{itemize}
                 \renewcommand{\labelitemi}{$\star$}
                 \item heat added to the system by the neighbourhood is \blue{positive};
                 \item heat removed from the system to the neighbourhood is \red{negative};
                 \item work produced by the system and transferred to the neighbourhood is \red{negative};
                 \item work produced by the neighbourhood and transferred to the system is \blue{positive}.
              \end{itemize}
         \end{shaded}

         In most thermodynamic systems, changes in kinetic and potential energies are often assumed negligible\footnote{Here, we are assuming that such systems are not movable \wrt frames of reference.}, and Eqn.~\ref{Chapter:FirstLaw:Eqn:TotalEnergy3} becomes
            \begin{equation}
               \mfr[E]{2}{cv} - \mfr[E]{1}{cv} = U_{2}-U_{1} = Q_{1-2} + W_{1-2},\label{Chapter:FirstLaw:Eqn:FirstLaw1}
            \end{equation}
         in such cases, the internal energy of a system may be changed in any of the following ways: producing or receiving work and/or have heat being removed or added to the system. Heat and work are equivalent ways of changing a system's internal energy. It was experimentally observed that in isolated systems there is {\bf no} change in the internal energy.

         \begin{shaded}
            The {\it First Law of Thermodynamics} is effectively a statement of energy conservation: `the only way the energy of a closed system can be changed are through transfer of energy by work or heat' \citep{Moran_Book}. In other words: `the internal energy of an isolated system is constant' \citep{Atkins_Book}. Mathematically, these statements can be readily represented in differential form (\ie during infinitesimal changes in the state of the system) by,
            \begin{equation}
               d U = \delta Q + \delta W,\label{Chapter:FirstLaw:Eqn:FirstLaw2}
            \end{equation}
            
         \end{shaded}

   
   % Example
   \begin{MyExample}{\begin{center}{\bf Example}\end{center}}
     \begin{example}\label{Chapter:FirstLaw:Example1}\citep{Atkins_Book}
        In a power station, the water pump is located in an isolated room. The electric motor of this pump produces 15 kJ of energy as mechanical work and loose 2 kJ of heat to the surroundings. Calculate the change in internal energy of the motor.  

       {\it Here, the system is the electrical motor whereas the surroundings is the room. The work produced by the motor is $\delta W= -15$ kJ and the heat transferred from the engine to the room is $\delta Q = -2$ kJ, therefore }
          \begin{eqnarray}
             dU = \Delta U &=&  \delta Q + \delta W \nonumber \\
                &=& -15 -2 = -17\text{ kJ} \nonumber
          \end{eqnarray}
     \end{example}
   \end{MyExample}

     \end{subequations}


%%%
%%% SECTION
%%%
     \section{Work Done at Moving Boundary}\label{Chapter:FirstLaw:Section:Work}
     \begin{subequations}
        In Section~\ref{Chapter:Introduction:Section:ThermodynamicWorkHeat}, the work produced by (or exerted on) the system was defined by Eqn.~\ref{Chpt01_Work2},
           \begin{displaymath}
              dW = -PdV,
           \end{displaymath}
        where $P$ is the pressure and $V$ is the variable volume

     \end{subequations}

% Figure
   \begin{figure}[h]
     \begin{center}
        \includegraphics[width=0.7\columnwidth,clip]{./Figs/Chp3_PistonCylinder1}
        \caption{Cylinder-piston system: work into the system as a result of an external force.}\label{Chapter:FirstLaw:Fig:Work1}
     \end{center}
   \end{figure}
 % Introduction to First Law of Thermodynamics
     \setcounter{examplecounter}{0}
  
%%%
%%% CHAPTER
%%%
\chapter{Second Law of Thermodynamics}\label{Chapter:SecondLaw}

   \begin{LearningObjectivesBlock}{Learning Objectives}
      Upon completion of this chapter, you will be able to
        \begin{enumerate}
           \item Demonstrate understanding of key concepts of entropy and the second law of thermodynamics;
           \item Apply first and second laws of thermodynamics to assess of heat transfer and process reversibility;
           \item Know that Clausius inequality is an alternative statement of the second law;
           \item Demonstrate how to apply the entropy balance in a thermodynamic system;
           \item Recognise processes that generate entropy.
        \end{enumerate}
\medskip
     Recommended reading: Chapters 5 of \citet{SmithVanNess_Book,Moran_Book,Borgnakke_Book} or 3 of \citet{Atkins_Book}.
   \end{LearningObjectivesBlock}

%%%% ETOC
\localtableofcontents


  
%%%
%%% SECTION
%%%
   \section{Introduction}\label{Chapter:SecondLaw:Section:Intro}
   The first law (Chapters~\ref{Chapter:Introduction} and \ref{Chapter:FirstLaw}) demonstrated that energy can flow either from or to a system in the form of heat or work, however it does not indicate the direction of process (\ie energy flow). The second law of thermodynamics concerns about feasibility, direction and spontaneity of processes and entropy..
   
The second law of thermodynamics is a general principle which makes constraints upon spontaneity of the process, direction of heat transfer and general efficiency of heat engines, \eg a hot body cools to the temperature of the surroundings or a chemical reaction that may run in one direction instead of the reverse (\eg combustion of ethane producing carbon dioxide and water). The reverse of such processes may occur but they will not be spontaneous.  
  
%%%
%%% SECTION
%%%
   \section{Statement of the Second Law}\label{Chapter:SecondLaw:Section:SecondLawStatement}\index{Laws of Thermodynamics ! Second law}
The second law of thermodynamics was stated separately by \citet{Clausius_Book}, Kelvin \citep{Thomson_1851} and \citet{Planck_Book} in slightly different ways. Each statement is based on irreversible processes.
\begin{shaded}
  \begin{center}
    {\bf Clausius Statement}
  \end{center}
  'It is impossible for a self-acting machine working in a cyclic process unaided by any external agency, to convey heat from a body at a lower temperature to a body at a higher temperature.'
\end{shaded}
This statement clearly indicates that heat cannot spontaneously flow from a colder to a hotter body. This can only be realised if other effects play some role, \eg in a refrigeration process in which external work is used to extract heat from low temperature body and reject it into a high temperature body (Fig.~\ref{Chapter:SecondLaw:Fig:SecondLawStatement})

   %%% FIGURE
   \begin{figure}[h]
     \begin{center}
        \includegraphics[width=.8\columnwidth,clip]{./Figs/2ndLaw_Schem}
     \caption{All spontaneous processes are irreversible -- heat flows from hot to cold spontaneously and irreversibly. }\label{Chapter:SecondLaw:Fig:SecondLawStatement}
     \end{center}
   \end{figure} 

\begin{shaded}
  \begin{center}
    {\bf Kelvin-Planck Statement}
  \end{center}
  'It is impossible to construct an engine, which while operating in a cycle produces no other effect except to extract heat from a single reservoir and do equivalent amount of work.'
\end{shaded}
This statement indicates that in order to achieve net work from a heat engine (\ie a device operating in cycle), there should be heat interaction between two bodies (\ie thermal reservoirs) at different temperatures (\ie thermal source or sink). 

   
   % Example
   \begin{MyExample}{\begin{center}{\bf Example}\end{center}}
     \begin{example}\label{Chapter:SecondLaw:Example1}\citep{Rajput_Book}
        A heat engine receives heat at the rate of 1500 kJ.min$^{-1}$ and gives an output of 8.2 kW. Determine:
        \begin{enumerate}[a)]
          \item Thermal efficiency;
          \item Rate of heat rejection $\left(\text{in kJ.s}^{-1}\right)$.
        \end{enumerate}
     \end{example}

% SOLUTION
       \noindent{\bf Solution:}
       The heat received by the engine (Fig.~\ref{Chapter:FirstLaw:Fig:PowerRefrigSystems}) is $Q_{1}=$ 1500 kJ.min$^{-1}$ = 25 kJ.s$^{-1}$, and the output work is $W=$ 8.2 kW = 8.2 kJ.s$^{-1}$. Thus
       \begin{enumerate}[a)]
         \item the thermal efficiency is defined as (Eqn.~\ref{Chapter:FirstLaw:Eqn:FirstLawCycle:PowerEfficiency}),
           \begin{displaymath}
             \eta_{\text{th}} = \frc{W}{Q_{1}} = 0.3280 \;\;\Longrightarrow\;\; 32.80\%;
           \end{displaymath}
         \item the rate of heat rejection is effectively a result of the energy balance on the system,
           \begin{displaymath}
             Q_{2}=Q_{1}-W = 16.80 \text{ kJ.s}^{-1}
           \end{displaymath}
           
       \end{enumerate}
   \end{MyExample}
   
   % Example
   \begin{MyExample}{\begin{center}{\bf Example}\end{center}}
     \begin{example}\label{Chapter:SecondLaw:Example2}\citep{Rajput_Book}
        A domestic food refrigerator maintains a temperature of -12$^{\circ}$C. The ambient air temperature is 35$^{\circ}$C. If heat leaks into the freezer at the continuous rate of 2 kJ.s$^{-1}$, determine the least power necessary to pump this heat out continuously.
     \end{example}

% SOLUTION
       \noindent{\bf Solution:}
       For freezer and ambient temperatures of 261.15 K and 308.15 K and assuming reversible process (see Eqn.~\ref{Chapter:FirstLaw:Eqn:FirstLawCycle:PowerEfficiency2}) based on the energy balance of the refrigeration cycle (Fig.~\ref{Chapter:FirstLaw:Fig:PowerRefrigSystems}b), 
       \begin{displaymath}
          \frc{Q_{2}}{Q_{1}} = \frc{T_{2}}{T_{1}} \;\;\Longrightarrow \;\;Q_{1} = \frc{Q_{2}}{T_{2}}T_{1} = \frc{2.00}{261.15}\times 308.15 = 2.3599\text{ kJ.s}^{-1},
       \end{displaymath}
       where $Q_{2}$ and $Q_{1}$ are the heat flowing from the cold reservoir and to the hot reservoir, respectively. Now, calculating the power (based on the energy balance in the system)
       \begin{displaymath}
          W = Q_{1}-Q_{2} = 0.3599\text{ kJ.s}^{-1}.
       \end{displaymath}
   \end{MyExample}
   

%%%
%%% SECTION
%%%
   \section{Mathematical Statement of the Second Law}\label{Chapter:SecondLaw:Section:SecondLawStatement_Maths}\index{Laws of Thermodynamics ! Second law}
     \begin{subequations}
        In Fig.~\ref{Chapter:SecondLaw:Fig:SecondLawStatement2}, an infinitesimal amount of heat, $\delta Q^{\prime}$, is transferred from the thermal reservoir (with temperature $T_{res}$) to a reversible cyclic engine (1). The engine produces a small amount of work, $\delta W^{\prime}$, and releases an infinitesimal amount of heat, $\delta Q$ to another reservoir (at variable temperature $T$) that also releases energy in form of work $\left(\delta W\right)$ to the surroundings.

%%% FIGURE
   \begin{figure}[h]
     \begin{center}
        \includegraphics[width=.8\columnwidth,clip]{./Figs/2ndLaw_Schem2}
     \caption{Thermal power device for mathematical derivation of the Second law. }\label{Chapter:SecondLaw:Fig:SecondLawStatement2}
     \end{center}
   \end{figure} 

        Using the analogy of heat transfer ratio and temperature ratio (see power cycle systems in Section~\ref{Chapter:FirstLaw:Section:FirstLaw_Cycle}),
           \begin{displaymath}
              \frc{\delta Q^{\prime}}{\delta Q} = \frc{T_{res}}{T} \;\; \Longrightarrow \frc{\delta Q^{\prime}}{T_{res}} = \frc{\delta Q}{T}.
           \end{displaymath}
        An energy balance (in differential form) for the combined cycle (within the dotted box) is
           \begin{eqnarray}
              && dU = \text{Energy In} - \text{Energy Out} \nonumber \\
              && dU = \delta Q^{\prime} - \left(\delta W + \delta W^{\prime}\right) \Longrightarrow \delta W + \delta W^{\prime} = \delta Q^{\prime} - dU. \nonumber 
           \end{eqnarray}
        Note that the first law is explicitly applied in the second equation above using the sign convention (\ie energy removed from the system is negative whereas energy added to the system is positive). The process is not required to be cyclic and the heat transfer $\delta Q$ is internal and does not cross the boundaries of the combined system. Thus, eliminating $\delta Q^{\prime}$,
           \begin{displaymath}
              \delta W + \delta W^{\prime} = T_{res}\frc{\delta Q}{T} - dU.
           \end{displaymath}
        If the power configuration undergoes a cyclic process,
           \begin{displaymath}
              \oint\delta W + \oint\delta W^{\prime} = \oint T_{res}\frc{\delta Q}{T} -\oint dU,
           \end{displaymath}
        and as $U$ is a thermodynamic property, the cyclic integral is equal to zero.  Integrating the equation above, and assuming that $T_{res}$ is (by definition) constant,
           \begin{displaymath}
              W + W^{\prime} = T_{res}\oint \frc{\delta Q}{T}. 
           \end{displaymath}
        From Kelvin-Planck statement of the Second Law, not all heat can be converted into work, but all work can be converted into heat, thus $W + W^{\prime}\le 0$ (\ie work is always produced in a power cycle and exported to the surroundings) and therefore 

       \begin{shaded}
           \begin{displaymath}
             T_{res}\oint \frc{\delta Q}{T} \le 0.\label{Chapter:SecondLaw:Eqn:SecondLawMathExpression1}
           \end{displaymath}
           Since $T_{res}>0$ (absolute temperature), this integral equation can be divided by $T_{res}$ without changing the meaning of the inequality to obtain the mathematical representation of the Second Law,
           \begin{equation}
             \begin{cases}
                \displaystyle\oint\frc{\delta Q}{T} < 0, & \;\;\text{ for irreversible processes} \\
                \displaystyle\oint\frc{\delta Q}{T} = 0, & \;\;\text{ for reversible processes} 
             \end{cases}\label{Chapter:SecondLaw:Eqn:SecondLawMathExpression2}
           \end{equation}
           This is called {\bf Clausius inequality}.\index{Clausius inequality}
       \end{shaded}
     \end{subequations}

%%%
%%% SECTION
%%%
   \section{Entropy}\label{Chapter:SecondLaw:Section:Entropy}\index{Laws of Thermodynamics ! Second law}
     \begin{subequations}
       Let's consider the reversible process shown in Fig.~\ref{Chapter:Introduction:Fig:CyclesSchematic}, from state 1 to 2, via path {\it A}, and returning to state 2 from either path {\it B} or {\it C}. The Clausius inequality, $\displaystyle\oint\frc{\delta Q}{T} = 0$, can be split as,
           \begin{displaymath}
             \begin{cases}
                \left(\displaystyle\int_{1}^{2}\frc{\delta Q}{T}\right)_{A} + \left(\displaystyle\int_{2}^{1}\frc{\delta Q}{T}\right)_{B} = 0, \\
                \left(\displaystyle\int_{1}^{2}\frc{\delta Q}{T}\right)_{A} + \left(\displaystyle\int_{2}^{1}\frc{\delta Q}{T}\right)_{C} = 0,
             \end{cases}
           \end{displaymath}
       leading to,
           \begin{displaymath}
                \left(\displaystyle\int_{2}^{1}\frc{\delta Q}{T}\right)_{B} = \left(\displaystyle\int_{2}^{1}\frc{\delta Q}{T}\right)_{C} = 0.
           \end{displaymath}
           Since paths {\it B} and {\it C} are different and arbitrary, but $\displaystyle\int\limits_{2}^{1}\displaystyle\frac{\delta Q}{T}$ is the same on either paths -- therefore {\bf the integral is path-independent}. 
           \begin{shaded}
               This defines another thermodynamic property, the {\bf entropy}\index{Entropy},
               \begin{equation}
                  S_{2}-S_{1} = \displaystyle\int_{1}^{2}\frc{\delta Q}{T}.\label{Chapter:SecondLaw:Eqn:Entropy}
               \end{equation}
           \end{shaded}
          Assuming constant mass ($m$), the intensive property entropy $\left(s=S/m\right)$, the differential form is, 
           \begin{displaymath}
                ds = \frc{\delta q}{T} \;\;\;\; \Longrightarrow \;\;\;\; \delta q = Tds,
           \end{displaymath}
           which is the heat transfer equivalent of 
           \begin{displaymath}
                \delta w = \displaystyle\int_{1}^{2}PdV.
           \end{displaymath}
 
           For a reversible process,
           \begin{displaymath}
                S_{2} - S_{1} = \displaystyle\int\limits_{1}^{2}\frc{\delta Q}{T}\;\;\text{ and } \;\; S_{1} - S_{2} = \displaystyle\int\limits_{2}^{1}\frc{\delta Q}{T},
           \end{displaymath}
           and from the Second law
           \begin{displaymath}
                0 \geq \left( \displaystyle\int\limits_{1}^{2}\frc{\delta Q}{T} \right)_{A} + \left( \displaystyle\int\limits_{2}^{1}\frc{\delta Q}{T} \right)_{B}.
           \end{displaymath}
       \begin{shaded}
           These integral expressions can be combined to eliminate path $B$ and obtain a general expression of entropy of arbitrary processes
           \begin{displaymath}
                dS = S_{2}-S_{1} \geq \displaystyle\int\limits_{1}^{2}\frc{\delta Q}{T}
           \end{displaymath}
           where
           \begin{equation}
               dS
                \begin{cases}
                      = \frc{dQ}{T} & \;\;\text{ for reversible processes} \\
                      > \frc{dQ}{T} & \;\;\text{ for irreversible processes}.\label{Chapter:SecondLaw:Eqn:Entropy2}
                \end{cases}
           \end{equation}           
       \end{shaded}
     \end{subequations}


%%%
%%% SECTION
%%%
   \section{Entropy Changes}\label{Chapter:SecondLaw:Section:EntropyChanges}
     \begin{subequations}
       From the First Law equation,
           \begin{displaymath}
              dU = dQ - PdV
           \end{displaymath} 
           If we differentiate the enthalpy equation -- $H = U + PV$.
                \begin{displaymath}
                    dH = dU + d(PV) = dU + PdV +VdP,
                \end{displaymath}
           and replace the previous equation,
                \begin{displaymath}
                    dH - \cancel{PdV} - VdP = dQ - \cancel{PdV} \;\;\Rightarrow \;\; dQ = dH - VdP
                \end{displaymath}
           For {\it ideal gas}, $C_{p}^{\text{ig}}=\left(\frac{dH}{dT}\right)_{P}$ and $V=\frc{RT}{P}$,
                \begin{eqnarray}
                  dQ &=& C_{p}^{\text{ig}} dT - \frc{RT}{P}dP\;\;\;\;\;\times\left(\frc{1}{T}\right) \nonumber \\
                  \frc{dQ}{T} &=& \frc{C_{p}^{\text{ig}}}{T}dT - \frc{R}{P}dP \nonumber \\
                  dS &=& \frc{C_{p}^{\text{ig}}}{T}dT - \frc{R}{P}dP, \nonumber
                \end{eqnarray}
           where $S$ is the molar entropy of ideal gas. 
           \begin{shaded}
              Integrating from state 0 to state 1,
                \begin{eqnarray}
                    \int\limits_{S_{0}}^{S_{1}} dS &=& \int\limits_{T_{0}}^{T_{1}} \frc{C_{p}^{\text{ig}}}{T}dT - R\int\limits_{P_{0}}^{P_{1}}\frc{dP}{P} \nonumber \\
                    \left(S_{1}-S_{0}\right) &=& \int\limits_{T_{0}}^{T_{1}} \frc{C_{p}^{\text{ig}}}{T}dT - R\ln{\frc{P_{1}}{P_{0}}} \;\;\;\;\times\left(\frc{1}{R}\right) \nonumber \\
                    \frc{\Delta S}{R} &=& \int\limits_{T_{0}}^{T_{1}} \frc{C_{p}^{\text{ig}}}{R}\frc{dT}{T} - \ln{\frc{P_{1}}{P_{0}}}.\label{Chapter:SecondLaw:Eqn:EntropyChanges1}
                \end{eqnarray}
                Although this equation was derived for mechanically reversible processes, it focuses on \underline{properties only} and is independent of the process. Thus it can be used to calculate entropy changes of {\it ideal gases}. A similar expression can be obtained as a function of $C_{v}$,
                \begin{equation}
                   \frc{\Delta S}{R} = \int\limits_{T_{0}}^{T_{1}} \frc{C_{v}}{R}\frc{dT}{T} +  \ln{\frc{V_{1}}{V_{0}}}\label{Chapter:SecondLaw:Eqn:EntropyChanges2}
                \end{equation}               
           \end{shaded}


   \begin{MyExample}{\begin{center}{\bf Example}\end{center}}
     \begin{example}\label{Chapter:SecondLaw:Example3}\citep{SmithVanNess_Book}
       For an ideal gas with constant heat capacity undergoing a reversible adiabatic (and therefore isentropic) process, Eqn.~\ref{Chapter:FirstLaw:Eqn:AdiabaticProcesses_TPGamma} can be written as
         \begin{displaymath}
             \frc{T_{2}}{T_{1}} = \left(\frc{P_{2}}{P_{1}}\right)^{\frac{\gamma-1}{\gamma}}.
         \end{displaymath}
         Show that these same equation results from application of Eqn.~\ref{Chapter:SecondLaw:Eqn:EntropyChanges1} with $\Delta S=$ 0.
     \end{example}

% SOLUTION
       \noindent{\bf Solution:} As $C_{p}^{\text{ig}}$ is constant and different from zero, Eqn.~\ref{Chapter:SecondLaw:Eqn:EntropyChanges1} can be rewritten as
          \begin{displaymath}
             \Delta S = 0 = \ln{\frc{T_{2}}{T_{1}}} - \frc{R}{C_{p}^{\text{ig}}}\ln{\frc{P_{2}}{P_{1}}}.
          \end{displaymath}
           Using Eqn.~\ref{Chapter:FirstLaw:Eqn:HeatCapacitiesRelation_IdealGas1} for an ideal gas with $\gamma = \frac{C_{p}^{\text{ig}}}{C_{v}^{\text{ig}}}$ (Eqn.~\ref{Chapter:FirstLaw:Eqn:HeatCapacitiesRelation1}):
          \begin{displaymath}
             C_{p}^{\text{ig}} = C_{v}^{\text{ig}} + R \;\;\Longrightarrow\;\; \frc{R}{C_{v}^{\text{ig}}} = \frc{\gamma-1}{\gamma},
          \end{displaymath}
          leading to
          \begin{displaymath}
             \Delta S = 0 = \ln{\frc{T_{2}}{T_{1}}} = \frc{\gamma-1}{\gamma}\ln{\frc{P_{2}}{P_{1}}} \;\;\Longrightarrow \;\; \frc{T_{2}}{T_{1}} = \left(\frc{P_{2}}{P_{1}}\right)^{\frac{\gamma-1}{\gamma}}
          \end{displaymath}

   \end{MyExample}


           \begin{shaded}
             Other important entropy relations for {\it any} gas:
             \begin{enumerate}[a)]
                \item Heating a gas at constant volume:
                  \begin{equation}
                    dS = C_{v}\frc{dT}{T}\label{Chapter:SecondLaw:Eqn:EntropyChanges3}
                  \end{equation}
                \item Heating a gas at constant pressure:
                  \begin{equation}
                    dS = C_{p}\frc{dT}{T}\label{Chapter:SecondLaw:Eqn:EntropyChanges4}
                  \end{equation}
                \item Isothermal process:
                  \begin{equation}
                    dS = R\frc{dV}{V}\label{Chapter:SecondLaw:Eqn:EntropyChanges5}
                  \end{equation}
                \item Adiabatic process (reversible):
                  \begin{equation}
                    dS = 0\label{Chapter:SecondLaw:Eqn:EntropyChanges6}
                  \end{equation}
             \end{enumerate}            
           \end{shaded}
           
     \end{subequations}

  
   % Example
   \begin{MyExample}{\begin{center}{\bf Example}\end{center}}
     \begin{example}\label{Chapter:SecondLaw:Example4}\citep{Singh_Book}
        Calculate the change in entropy of air $\left(\text{in kJ.kg}^{-1}\text{.K}^{-1}\right)$, if it is throttled (\ie isenthalpic process, or in other words at constant enthalpy) from 5 bar, 27$^{\circ}$C to 2 bar adiabatically. Given $C_{p}=$ 1.004 kJ.(kg.K)$^{-1}$, $R=$ 8.3145 J.(mol.K)$^{-1}$. Assume ideal gas behaviour and the molar mass of air is 29 g.mol$^{-1}$.
     \end{example}

% SOLUTION
       \noindent{\bf Solution:}
       \begin{center}
          \begin{tabular}{l c l}
             $P_{1}=$ 5 bar     &                           & $P_{2}=$ 2 bar \\
                               & \;\;$\Longrightarrow$\;\; &                \\
             $T_{1}=$ 300.15 K  &                           & $T_{2}$
       \end{tabular}
       \end{center}
       As there is no information \wrt the type of process, we can not assume a reversible adiabatic process, in which $\Delta S=$ =0. Therefore, we could use Eqn.~\ref{Chapter:SecondLaw:Eqn:EntropyChanges1} to calculate $\Delta S$, 
       \begin{displaymath}
          \frc{\Delta S}{R} = \frc{C_{p}^{\text{ig}}}{R}\ln{\frc{T_{2}}{T_{1}}} - \ln{\frc{P_{2}}{P_{1}}},
       \end{displaymath}
       although we do not know $T_{2}$. However, the expansion occurs isenthalpically, \ie 
       \begin{eqnarray}
            h_{1} &=& h_{2} \nonumber \\
            C_{p}^{\text{ig}}T_{1} &=& C_{p}^{\text{ig}}T_{2} \nonumber \\
            T_{1} &=& T_{2}, \nonumber
       \end{eqnarray}
       thus,
       \begin{eqnarray}
          \Delta S &=& C_{p}^{\text{ig}}\ln{\frc{T_{2}}{T_{1}}} - R\ln{\frc{P_{2}}{P_{1}}} \nonumber \\
                   &=& 0.2627\text{ J.g}^{-1}\text{.K}^{-1} = 0.2627\text{ kJ.kg}^{-1}\text{.K}^{-1} \nonumber
       \end{eqnarray}
   \end{MyExample}

   % Example
   \begin{MyExample}{\begin{center}{\bf Example}\end{center}}
     \begin{example}\label{Chapter:SecondLaw:Example5}\citep{SmithVanNess_Book}
         A rigid vessel of volume of 0.06 m$^{3}$ contains an ideal gas with $C_{v}=\frac{5}{2}R$, at 500 K and 1 bar. If 15 kJ of heat is transferred to the gas, determine the entropy of change $\left(\text{in J.K}^{-1}\right)$.
     \end{example}

% SOLUTION
       \noindent{\bf Solution:} 
            We can use Eqn.~\ref{Chapter:SecondLaw:Eqn:EntropyChanges2} to calculate the entropy change,
               \begin{displaymath}
                  \Delta S = C_{v}\ln{\frc{T_{2}}{T_{1}}} + R\ln{\frc{V_{2}}{V_{1}}},
               \end{displaymath}
               as the process take place in a rigid vessel, $V_{2}=V_{1}$, and the second term vanishes. However, in order to obtain the entropy change, $T_{2}$ must be calculated. As the volume is kept constant, there is no work produced or received by the system, $\delta W=$ 0, thus
              \begin{eqnarray}
                  \Delta U = \delta Q = 15000\text{ J} &=& C_{v}\left(T_{2}-T_{1}\right) =  \frc{5}{2}R\left(T_{2}-T_{1}\right) \nonumber \\
                                                       &=& \frc{5}{2} \times 8.3145\text{ J.mol}^{-1}\text{.K}^{-1}\left(T_{2}- 500\right)\text{ K}. \nonumber  
              \end{eqnarray}
              From the expression above, it is clear that there is an unbalance on the units as the right-hand side results in J.mol$^{-1}$ whereas the left-hand side is in J. This can be fixed by multiplying the right-hand side by the total number of moles in the vessel (\ie total energy of the system rather than the energy per mole), which can be easily calculated from the initial conditions using the ideal gas equation of state:  
              \begin{displaymath}
                  n = \frc{P_{1}V_{1}}{RT_{1}} = \frc{1\text{ bar}\times 0.06\text{ m}^{3}}{8.3145\times 10^{-5}\text{ bar.m}^{3}\text{.mol.K}^{-1} \times 500\text{ K}} = 1.4432\text{ moles}.
              \end{displaymath}
              Hence, $T_{2}=$ 1000.02 K. Now, using Eqn.~\ref{Chapter:SecondLaw:Eqn:EntropyChanges2}
               \begin{displaymath}
                  \Delta S = C_{v}\ln{\frc{T_{2}}{T_{1}}} + \cancelto{0}{R\ln{\frc{V_{2}}{V_{1}}}} = 14.4083\text{ J.mol}^{-1}\text{.K}^{-1}.
               \end{displaymath}
              $\Delta S$ can be converted to J.K$^{-1}$ by multiplying it by the number of moles, leading to 20.7941 J.K$^{-1}$.

   \end{MyExample}

   
\clearpage   
\begin{FinalSummaryBlock}{Summary}
    \begin{itemize}
       \item Clausius inequality:
            \begin{displaymath}
                \displaystyle\int \frc{\delta Q}{T} \le 0
            \end{displaymath}
       \item Both statements of the Second law establish that in order to achieve net work from a device operating in cycle, there should be heat interaction between the thermal reservoirs;
       \item 
           \begin{displaymath}
               dS
                \begin{cases}
                      = \frc{dQ}{T} & \;\;\text{ for reversible processes} \\
                      > \frc{dQ}{T} & \;\;\text{ for irreversible processes}.
                \end{cases}
           \end{displaymath}
    \end{itemize}
\end{FinalSummaryBlock}



%%%
%%% TUTORIAL 
%%%  
\clearpage
\section{Tutorial}
\begin{MyTutorial}{\begin{center}{\bf Tutorial for Part I}\end{center}}
%
  \begin{problem}\label{Tut01:UnitConversion}
     A closed system contains 1 mol of nitrogen $\left(\text{MW = 28 g. mol}^{-1}\right)$. Using the ideal gas law, calculate the missing {\it PVT} parameter for the following data given. Give your results in {\bf SI units}. The universal gas constant is $\text{R = 8.314 J.mol}^{-1}\text{.K}^{-1}$.
     \begin{enumerate}[a)]   
       \item P = 1 atm, T = 0$^{\circ}$C
       \item V$^{t}$ = 12.85 ft$^{3}$; T = 59$^{\circ}$F
       \item P = 2.5$\times$10$^{9}$ g.m$^{-1}$.s$^{-2}$; T = 650.50$^{\circ}$R
       \item V$^{t}$ = 1.3$\times$10$^{-12}$Gl; P= 500 psi
    \end{enumerate}
  \end{problem}
%
  \begin{problem}\label{Tut01:FirstLawIdealGas1}
     Given argon at P$_{1}$ = 140 kPa, T$_{1}$ = 10$^{o}$C, V$_{1}$ = 200 liters which undergoes a polytropic compression to P$_{2}$ = 700 kPa, T$_{2}$ = 180 $^{o}$C, find Q$_{1-2}$. Given MW = 39.948 kg.mol$^{-1}$ and C$_{V}$ = 0.312 kJ.kg$^{-1}$.K$^{-1}$.
  \end{problem}
%
  \begin{problem}\label{Tut01:FirstLawIdealGas2}
     Given air (assuming ideal gas behaviour) expanding reversibly and adiabatically from $T_{1}=450 K$ and $V_{1}=3.0\times 10^{-3}m^{3}$ to the final volume, $V_{2}=5.0\times 10^{-3}m^{3}$. $T$ and $V$ relationship for constant heat capacities is represented by $\displaystyle\frc{T_{2}}{T_{1}} = \left(\frc{V_{1}}{V_{2}}\right)^{\gamma-1}$.
     \begin{enumerate}[a)]  
        \item Derive a relationship between $T$ and $P$; 
        \item Calculate $T_{2}$ (in K);
        \item Calculate the work $\left(\text{in kJ.mol}^{-1}\right)$ done during the process and;
        \item Determine the enthalpy change $\left(\text{in kJ.mol}^{-1}\right)$.
     \end{enumerate}
     Assume that $C_{p}= 5.0\; cal.\left(mol.K\right)^{-1}$ and $C_{v}=3.0\; cal.\left(mol.K\right)^{-1}$.
  \end{problem}
%
  \begin{problem}\label{Tut01:FirstLawIdealGas3} % Problem 2.33 (Saphiro) 
      CO gas contained within a piston-cylinder assembly undergoes three processes in series:
     \begin{itemize}
        \item Process 1-2: expansion from p$_{1}$ = 5 bar, V$_{1}$ = 0.2 m$^{3}$ to V$_{2}$ = 1.0 m$^{3}$, during which the pressure-volume relationship is pV = constant.
        \item Process 2-3: constant volume heating from state 2 to state 3, where p$_{3}$ = 5 bar.
        \item Process 3-1: constant pressure compression to the initial state.
     \end{itemize}
     Sketch the processes in series on p-V coordinates and evaluate the work for each process, in kJ.
  \end{problem}
%
  \begin{problem}\label{Tut02:IdealGas1} % Johannes T2Q2
     In a closed system (kinetic and potential energy are constant) three consecutive processes are done by an ideal gas $\left(\right.$10 moles, MW = 24.945 g.mol$\left.^{-1}\right)$:
     \begin{center}
        \begin{tabular}{c l}
          \hline
          Initial conditions: & P$_{1}$ = 1 bar, T$_{1}$ = 300K \\
          Process 1$\rightarrow$2: & Reversible isothermal compression, v$_{2}$ = 0.1m$^{3}$.kg$^{-1}$ \\
          Process 2$\rightarrow$3: & Isochoric colling, P$_{3}$ = 2 bar \\
          Process 3$\rightarrow$4: & Isobaric heating, T$_{4}$ = 600 K \\
          \hline
        \end{tabular}
     \end{center}
     \begin{enumerate}[a)]  
        \item Calculate the initial volume V$^{t}_{1}$ $\left(\text{in m}^{3}\right)$ and specific volume v$_{1}$ $\left(\text{in m}^{3}\text{.kg}^{-1}\right)$ of the gas.% (0.2494 m3, 1 m3 kg-1)
        \item Calculate the pressure P$_{2}$ after the first process. %(10 bar)
        \item Calculate the temperature T$_{3}$ after the second process.% (60 K)
        \item Calculate the final specific volume V$_{4}$ $\left(\text{in m}^{3}\text{.kg}^{-1}\right)$. %(1 m3 kg-1)
        \item What forms of energy are present in transit across the system's boundary during the first process? Calculate the values (in kJ). %(57.4 kJ)
        \item Draw a PV diagram with all processes 1$\rightarrow$2$\rightarrow$3$\rightarrow$4$\rightarrow$1.
     \end{enumerate}
  \end{problem}
%
  \begin{problem}\label{Tut02:IdealGas2} % Johannes T2Q3
     One mole of an ideal gas with C$_{p}$ = (7/2)R and C$_{v}$ = (5/2)R expands from P$_{1}$ = 8 bar and T$_{1}$ = 600 K to P$_{2}$ = 1 bar by each of the following paths:
     \begin{enumerate}[a)] 
        \item Constant volume.% (0 kJ, -10.91 kJ, -15.28 kJ)
        \item Constant temperature.% (-10.37 kJ, 0 kJ)
        \item Adiabatically.% (-5.586 kJ, 0 kJ, -7.821 kJ)
     \end{enumerate}
Assuming mechanical reversibility, calculate $W$, $Q$, $\Delta U$, $\Delta H$ $\left(\text{all in kJ.mol}^{-1}\right)$ for each process. Sketch each path on a single PV diagram.
  \end{problem}
%
  \begin{problem}\label{Tut02:Demonstration} % Johannes T3Q4
     Assuming $S = S\left(P,V\right)$ and taking into consideration that,
\begin{displaymath}
\left(\frc{\partial S}{\partial T}\right)_{V} = \frc{C_{v}}{T}\;\;\;\text{ and }\;\;\; \left(\frc{\partial S}{\partial T}\right)_{P} = \frc{C_{p}}{T}
\end{displaymath}
Prove that 
\begin{displaymath}
\d S = \frc{C_{v}}{T}\left(\frc{\partial T}{\partial P}\right)_{V}\d P + \frc{C_{p}}{T}\left(\frc{\partial T}{\partial V}\right)_{P}\d V
\end{displaymath}
  \end{problem}
%
  %\begin{problem}\label{Tut01:FirstLawDerivation}% Problem 5.78 (Saphiro)
     %The pressure-volume diagram of a Carnot power cycle executed by an ideal gas with constant specific heat ratio $\kappa$ is shown in Fig. \ref{Prob_Saphiro_5.78}. Demonstrate that: (a) $V_{4}V_{2}=V_{1}V_{3}$, (b) $\frc{T_{2}}{T_{3}} = \left(\frc{P_{2}}{P_{3}}\right)^{\frc{\kappa - 1}{\kappa}}$ and (c) $\frc{T_{2}}{T_{3}}=\left(\frc{V_{3}}{V_{2}}\right)^{\kappa - 1}$
%     \includegraphics[width=6.cm,clip]{./../Pics/Problem_5_78}
  %\end{problem}
%
  %\begin{problem}
     
  %\end{problem}
%
\end{MyTutorial}
 % Introduction to Second Law of Thermodynamics

\part{Thermodynamic Properties of Fluids}  
     \setcounter{examplecounter}{0}
  \chapter{Volumetric Properties of Pure Substances}\label{Chapter:VolumetricPropertiesPureSubstances}

   \begin{LearningObjectivesBlock}{Learning Objectives}
      Upon completion of this chapter, you will be able to
        \begin{enumerate}
           \item Demonstrate understanding of key concepts of entropy and the second law of thermodynamics;
           \item Apply first and second laws of thermodynamics to assess of heat transfer and process reversibility;
           \item Know that Clausius inequality is an alternative statement of the second law;
           \item Demonstrate how to apply the entropy balance in a thermodynamic system;
           \item Recognise processes that generate entropy.
        \end{enumerate}
\medskip
     Recommended reading: Chapters 3 of \citet{SmithVanNess_Book}, 6 of \citet{Sandler_Book}, 2 of \citet{Borgnakke_Book} or 4 of \citet{Atkins_Book}.
   \end{LearningObjectivesBlock}

  
%%%
%%% SECTION
%%%
   \section{Introduction}\label{Chapter:VolumetricPropertiesPureSubstances:Section:Intro}
   Definitions and assumptions for ideal gas behaviour were introduced in Section~\ref{Chapter:Intro_Property_of_Gases:Section:IdealGases} along with the corresponding equation of state (Eqn.~\ref{Chapter:Intro_Property_of_Gases:Eqn:IdealEOS}). A fluid behaves as an ideal gas at low to moderate pressure (often below atmospheric pressure) and at high temperatures. Under different conditions, fluids may not behave as ideal gases and therefore other equations of state (not just for gases) were designed to describe the {\it PVT} behaviour of fluids.

Phase transition of pure substances is one of the main topics in industrial chemical engineering as it is of paramount importance to calculate fluid properties for the design of equipment and processes. Therefore, the main aim of this chapter is to introduce the concept of phase diagram of pure substances and its mathematical representation as equations of state. 

  
%%%
%%% SECTION
%%%
   \section{Phase Diagrams of Pure Substances}\label{Chapter:VolumetricPropertiesPureSubstances:Section:PhaseDiagrams}\index{Phase diagram}\index{Triple point}\index{Supercritical state}

Pure substances/fluids can be defined as materials of homogeneous and constant composition. For example, systems containing water-ice, water-steam or water-ice-steam are considered pure fluids (or substances) at different phases, whereas (a) air-water, (b) air-steam and (c) gaseous mixture containing N$_{2}$, H$_{2}$, O$_{2}$ and NO$_{2}$ are considered as either homogeneous (b-c) or heterogeneous (a) mixtures.

A pure substance can exist and coexist in different phases\footnote{Phase of a substance can be defined as the form of matter that is uniform in chemical composition and physical state.}  (\ie solid, liquid and vapour -- S, L and V). Phase behaviour of substances is often represented by {\it PVT}\footnote{{\it PVT} stands for pressure, specific (or molar) volume and temperature.} phase diagrams that describe phase transitions (\ie spontaneous conversion -- or mass/heat transfer of one phase to another) and coordinates. A 3D {\it PVT} diagram of an arbitrary substance is shown in Figure~\ref{Chapter:VolumetricPropertiesPureSubstances:Fig:PVT_Surfaces}a, where solid, liquid and vapour phases are represented by continuous volumes and, the regions between these volumes are representations of phase equilibrium, \ie regions where phases coexist in thermodynamic equilibrium.

   %%% FIGURE
           \begin{figure}[h]
              \begin{center}
                 \includegraphics[width=10.cm,clip]{./Figs/PVT_Surface.jpg}
                 \caption{$PVT$ volume (top) and projections onto (b) $PT$ and (c) $PV$ diagrams for a pure substance \citep[Extracted from][]{Borgnakke_Book}.}\label{Chapter:VolumetricPropertiesPureSubstances:Fig:PVT_Surfaces}
              \end{center}
           \end{figure}    

Such 3D representation helps determining (qualitatively) phases (S, L and V) at prescribed coordinates (temperature, specific/molar volume and pressure) of fluids. However, it is not convenient dealing with 3D plots and, most of the time, we may want to extract quantitative values of fluids (Chapter~\ref{Chapter:ThermodynamicPropertiesPureFluids}). A better approach is to project 3D plots into surfaces, \ie through $PT$, $PV$ and $VT$ phase diagrams as shown in Figs.~\ref{Chapter:VolumetricPropertiesPureSubstances:Fig:PVT_Surfaces}b and \ref{Chapter:VolumetricPropertiesPureSubstances:Fig:PVT_Surfaces}c.

$PV$ diagram (Fig.~\ref{Chapter:VolumetricPropertiesPureSubstances:Fig:PV-PT_Diagrams}a) is a representation of the PVT volume, Fig.~\ref{Chapter:VolumetricPropertiesPureSubstances:Fig:PVT_Surfaces}a, where surfaces (\ie areas) represent both single phases and phases in equilibrium, and lines represent transition between phases. Here, there are three properties that we need to define:
           \begin{enumerate}[a)]
              \item Critical point (or state): coordinates in which two phases of a fluid become indistinguishable. Beyond this coordinate, a fluid is neither completely liquid nor completely gaseous, \ie exhibits properties of both the liquid phase and the gas phase and is referred to as a {\it supercritical fluid}. All fluids have distinct critical coordinates, known as {\it critical pressure} $\left(\text{P}_{c}\right)$, {\it critical temperature} $\left(\text{T}_{c}\right)$ and {\it critical volume} $\left(\text{V}_{c}\right)$. Interesting videos about critical state can be seen in 
                  \begin{center}
                     \href{https://www.youtube.com/watch?v=bJjcTpRzXpM}{https://www.youtube.com/watch?v=bJjcTpRzXpM} and \\
                     \href{https://www.youtube.com/watch?v=RmaJVxafesU}{https://www.youtube.com/watch?v=RmaJVxafesU}.
                  \end{center}
              \item Saturation lines (blue lines in Fig.~\ref{Chapter:VolumetricPropertiesPureSubstances:Fig:PV-PT_Diagrams}a) are coordinates in which phase transition starts to occur.
              \item Isotherms: Lines of constant temperature.
           \end{enumerate}

   %%% FIGURE
           \begin{figure}[h]
              \vbox{
                    \hbox{\includegraphics[width=.5\columnwidth,clip]{./Figs/PV_Diagram1}
                          \includegraphics[width=.5\columnwidth,clip]{./Figs/PT_Diagram}}
                    \vspace{-.1cm}
                    \hbox{\hspace{4cm}(a)\hspace{8cm}(b)}}
              \caption{ (a) $PV$ and (b) $PT$ diagrams for a pure substance. Dotted line in (a) represents the isotherm at T=T$_{c}$.}\label{Chapter:VolumetricPropertiesPureSubstances:Fig:PV-PT_Diagrams}
           \end{figure}


           In vapour-liquid equilibrium (VLE) systems there are two main saturation lines: {\it saturated liquid line} (left-hand side of $C$ in Fig.~\ref{Chapter:VolumetricPropertiesPureSubstances:Fig:PV-PT_Diagrams}a)  and {\it saturated vapour line} (rhs of $C$) that represent the initial transition from a single phase system to a two or three phases system. 
%

$PT$ diagrams (Fig.~\ref{Chapter:VolumetricPropertiesPureSubstances:Fig:PV-PT_Diagrams}b) are representations of Fig.~\ref{Chapter:VolumetricPropertiesPureSubstances:Fig:PVT_Surfaces}a, where phases are defined by surfaces (\ie areas) with continuous lines representing phases transitions (\ie in equilibrium). The {\it triple point} is coordinate in which all three phases (S, L and V) coexist in equilibrium.

A fluid in the compressed liquid state is often called {\it sub-cooled fluid} (central region of Fig.~\ref{Chapter:VolumetricPropertiesPureSubstances:Fig:PV-PT_Diagrams}b), while a gas at a pressure lower than its saturation vapour pressure for a given temperature is said to be at {\it superheated state}.


%%% SECTION
%%%
   \section{Gibbs Phase Rule for Pure Substance}\label{Chapter:VolumetricPropertiesPureSubstances:Section:GibbsPhaseRule}





%%% SECTION
\section{PVT Behaviour of Pure Substances}\label{Chapter:VolumetricPropertiesPureSubstances:PVTBehaviour}
  
   \begin{enumerate}[i)]
%
%
%
        \item The \underline{Gibbs phase rule} is a relation that determines the number of independent variables that must be specified to establish the {\it intensive state of any system at equilibrium},
                \begin{equation}
                    \Psi = 2 + \mathcal{C} - \mathcal{P} -\mathcal{R},\label{Mod02_GibbsPhaseRuleReactive}
                \end{equation}
                where $\Psi$, $\mathcal{C}$, $\mathcal{P}$ and $\mathcal{R}$ are the number of degrees of freedom of the thermodynamic system, number of chemical components, number of co-existing phases and number of independent reactions, respectively. For non-reactive systems\footnote{The Gibbs phase rule will be revisited again in Module~\ref{Section:06} -- Chemical Reaction Equilibrium.}, this relation becomes,
                \begin{equation}
                    \Psi = 2 + \mathcal{C} - \mathcal{P}.\label{Mod02_GibbsPhaseRule}
                \end{equation}
                The degrees of freedom ($\Psi$, \ie number of intensive properties such as temperature and pressure) determines the number of variables that must be specified to fix all other remaining phase rule variables. For example, for a pure liquid component, the phase rule yields two degrees of freedom, \ie if temperature and pressure are specified, all other intensive properties (\eg enthalpy, entropy etc) are uniquely determined. However, if the same liquid component is in equilibrium with its vapour phase (\eg water and steam) there is {\it only} one degree of freedom. This means that either pressure or temperature may be specified to fix all other intensive properties of the system. At the triple point (\eg water, steam and ice), the number of degrees of freedom is {\it zero}, \ie any change from such state (\eg for water-steam-ice at $\sim$ 273.15 K and 0.0061 bar) causes at least one of the phases to vanish.
%
        \item The ideal gas {\it equation of state} (EOS),
                \begin{equation}
                   P V = R T,\label{Mod02_IdealEOS}
                \end{equation}
                where $V$ is the molar volume. This is a relationship between the macroscopic intensive properties, and is based in two main assumptions with respect to the microscopic behaviour of molecules:
            \begin{enumerate}[(a)]
                \item Molecules have no extension in space (\ie zero volume), and;
                \item Molecules \underline{do not interact with each other}.
            \end{enumerate}
            The second assumption is particularly important as it considers that atoms and molecules either do not interact or do not have electric charge or have an infinite distance between them (\ie low density conditions).
%
        \item However, these assumptions are rarely met in real conditions (both in the environment and in industry) and several mathematical relations have been developed to better represent the PVT behaviour of real fluids:
            \begin{enumerate}[a)]
%
                \item The PVT behaviour of real pure fluids can be expressed as functional $f(P,V,T) = 0$. However, from the Gibbs phase rule for a single phase pure component the number of degrees of freedom is equal to 2. Therefore, we can write this function in its simplest way (or EOS), $V=V(T,P)$, or in differential form,
                   \begin{displaymath}
                      dV = \left(\frc{\partial V}{\partial T}\right)_{P}dT + \left(\frc{\partial V}{\partial P}\right)_{T}dP
                   \end{displaymath}
                   defining the {\it coefficient of thermal expansion} (or {\it volume expansivity coefficient}), $\beta$, and the {\it coefficient of isothermal compressibility}, $\kappa$ as,
                   \begin{equation}
                      \beta \equiv \frc{1}{V}\left(\frc{\partial V}{\partial T}\right)_{P}\;\;\;\text{ and }\;\;\;\kappa \equiv -\frc{1}{V}\left(\frc{\partial V}{\partial P}\right)_{T},\label{Mod02_Compressibilityexpansivity}
                   \end{equation}
                   respectively, leading to
                   \begin{equation}
                       \frc{dV}{V} = \beta dT - \kappa dP.\label{Mod02_Compressibilityexpansivity2}
                   \end{equation}
%
                \item The {\it Virial EOS} is a relation that can be derived from statistical mechanics, and is represented by power series in terms of $\frac{1}{V}$ with two alternate forms:
                   \begin{eqnarray}
                       \frc{P V}{R T} &=& 1 + \frc{B}{V} + \frc{C}{V^{2}} + \cdots \text{ or} \label{Mod02_Virial1}\\
                       \frc{P V}{R T} &=& 1 + B^{\prime}P + C^{\prime}P^{2} + \cdots,\label{Mod02_Virial2} 
                   \end{eqnarray}
                   where $B$ and $C$ are the second and third virial coefficients and,
                   \begin{displaymath}
                      B = \frc{B^{\prime}}{R T}\;\;\;\text{ and }\;\;\; C^{\prime}=\frc{C-B^{2}}{\left(R T\right)^{2}}.
                   \end{displaymath}
                   Second and third terms of Eqns.~\ref{Mod02_Virial1} and~\ref{Mod02_Virial2} are \blue{corrections of the non-ideal behaviour of a gas}. Virial coefficients are strongly dependent on the temperature (\ie $B=B(T)$, $C=C(T)$, $D=D(T)$ etc) and the more the number of coefficients (\ie the more terms in the power series -- Eqns.~\ref{Mod02_Virial1} and~\ref{Mod02_Virial2}) the better is the predictions of the gas molar volume. The second virial coefficient is readily found for a large number of fluids in any chemical engineering handbook (and several textbooks), however the third and further coefficients are more difficult to obtain/calculate. Therefore the Virial EOS is often used for moderate deviations from the ideal gas behaviour through the truncated forms of Eqns.~\ref{Mod02_Virial1} and~\ref{Mod02_Virial2}
                   \begin{eqnarray}
                      \frc{P V}{R T} &=& 1 + \frc{B}{V} \;\;\;\;\text{ or } \label{Mod02_Virial1b}\\
                      Z &=& 1 + \frc{B P}{R T} = 1 + \frc{B P_{c}}{R T_{c}}\frc{P_{r}}{T_{r}},\label{Mod02_Virial1c}
                   \end{eqnarray}
                   where,
                   \begin{equation}
                      T_{r} = \frc{T}{T_{c}}\;\;\;\;\text{ and }\;\;\;\; P_{r} = \frc{P}{P_{c}}\label{Mod02_ReducedT-P},
                   \end{equation}
                   are {\it reduced temperature and pressure}, respectively. $Z = \frac{P V}{R T}$ is the \underline{compressibility factor} and can be defined as the ratio of the molar volume of a gas to the molar volume of an ideal gas at the same temperature and pressure conditions -- for an ideal gas, $Z=1$. 

                   A number of {\it generalised relations} have been developed to calculate the {\it second virial coefficients}, \eg
                   \begin{displaymath}
                      \frc{B P_{c}}{R T_{c}} = B^{0} + \omega B^{1},
                   \end{displaymath}
                   with terms $B^{0}$ and $B^{1}$ defined by,
                   \begin{displaymath}
                      B^{0} = 0.083 - \frc{0.422}{T_{r}^{1.6}}\;\;\text{ and }\;\; B^{1} = 0.139 - \frc{0.172}{T_{r}^{4.2}}.
                   \end{displaymath}
                   $\omega$ is a parameter known as \underline{acentric factor} that measures the non-sphericity of molecules,
                   \begin{displaymath}
                      \omega \equiv -1 - \log\limits_{10}{\left(P_{r}^{\text{sat}}\right)_{T_{r}=0.7}},
                   \end{displaymath}
                   where $\left(P_{r}^{\text{sat}}\right)_{T_{r}=0.7}$ is the reduced saturation vapour pressure obtained at reduced temperature of 0.7. Tabulated acentric factor for a number of chemical species can be found in any thermodynamic textbook.
%
                \item The truncated form of the {\it Virial EOS} can be used to represent PVT behaviour of gases with reasonable accuracy at relatively low pressures. At moderate and high pressures, predicted volumetric properties deviate from expected and alternative EOS formulations have been developed. \underline{Cubic EOS} are widely used in \blue{flow and process simulators} to represent PVT behaviour of fluids, and were developed as a cubic function of the molar volume (or the compressibility factor, $Z$). Cubic equations of state can result in (reasonable) accurate prediction of both gas and liquid (saturated) molar volumes and are relatively easy to implement. Since the development of the first cubic EOS in the 19$^{\text{th}}$ century, several EOS have been proposed and used by industry. Four of the most important cubic EOS are listed below:
                   \begin{enumerate}[c.1)]
%
                      \item The \underline{van der Walls} (vdW) EOS was originally developed in 1873 and has the form,
                          \begin{equation}
                             P = \frc{R T}{V-b} - \frc{a}{V^{2}},\label{Mod02_vdWEOS}
                          \end{equation}
                          where $a$ is called the {\it attraction parameter} and $b$ is the {\it repulsive parameter} (or {\it effective molecular volume} or {\it co-volume}), 
                          \begin{displaymath}
                              a = \frc{27}{64}\frc{R^{2}T_{c}^{2}}{P_{c}},\;\;\; b = \frc{1}{8}\frc{R T_{c}}{P_{c}},
                          \end{displaymath}
                          and they take into account interactions between molecules. Although this EOS is able to predict volumetric properties of gasses better than the ideal EOS, it is still very inaccurate for liquids and for fluids at high pressure.  Redlich-Kwong and Soave-Redlich-Kwong EOS were formulated in the 40's and 70's and became increasingly popular in the oil $\&$ gas and petrochemical sectors. 
%
                      \item Redlich-Kwong (RK EOS):
                           \begin{equation}
                               P = \frc{R T}{V-b} - \frc{a}{V\sqrt{T}\left(V+b\right)},\label{Mod02_RKEOS}
                           \end{equation}
                           with,
                          \begin{displaymath}
                              a = \frc{0.42748 R^{2}T_{c}^{2}}{P_{c}},\; \text{ and }\;\;\; b = \frc{0.08664 R T_{c} }{P_{c}}.
                          \end{displaymath}
%
                      \item Soave-Redlich-Kwong (SRK EOS):
                           \begin{equation}
                               P = \frc{R T}{V-b} - \frc{a\alpha}{V\left(V+b\right)},\label{Mod02_SRKEOS}
                           \end{equation}
                           with,
                          \begin{eqnarray}
                              &&a = \frc{0.427 R^{2}T_{c}^{2}}{P_{c}},\;\;\;b = \frc{0.08664 R T_{c} }{P_{c}} \nonumber \\
                              &&\text{and }\;\;\; \alpha = \left[1 + \left( 0.48508 + 1.55171\omega - 0.15613\omega^{2}\right)\left(1-\sqrt{T_{r}}\right)\right]^{2}\nonumber
                          \end{eqnarray}                          
%
                      \item Peng-Robinson (PR EOS): it is by far the most used EOS used in simulators and is able to predict molar volume with good accuracy,
                           \begin{equation}
                               P = \frc{R T}{V-b} - \frc{a\alpha}{V\left(V+b\right)+b\left(V-b\right)},\label{Mod02_PREOS}
                           \end{equation}
                           with,
                          \begin{eqnarray}
                              && a = \frc{0.45724 R^{2}T_{c}^{2}}{P_{c}},\;\;\;b = \frc{0.07780 R T_{c} }{P_{c}},\;\;\alpha = \left[1 + \kappa\left(1-\sqrt{T_{r}}\right)\right]^{2}  \nonumber \\
                              &&\text{ and }\; \kappa = 0.37464 + 1.54226\omega - 0.26992\omega^{2} \nonumber
                          \end{eqnarray}                
%
                   \end{enumerate}
                   These EOS can be generalised and manipulated to appear as a cubic function of the compressibility factor, $Z$,
                   \begin{equation}
                      Z^{3} + k_{1} Z^{2} +k_{2} Z + k_{3} = 0,\label{Mod02_GeneralEOS}
                   \end{equation}
                   where
                   \begin{eqnarray}
                     && A = \frc{aP}{\left(RT\right)^{2}},\;\; B = \frc{bP}{RT},\;\; k_{1} = -1 -B + uB, \nonumber \\
                     && k_{2} = A + w B^{2} - uB -uB^{2}\;\;\text{ and }\;\; k_{3} = - AB -w B^{2} -w B^{3}.\nonumber
                   \end{eqnarray}

\begin{table}[h]
  \begin{center}
  \begin{tabular}{ c | c c c c c }
    \hline
      {\bf EOS} & {\bf $u$} & {\bf $w$} & {\bf k$_{1}$} & {\bf k$_{2}$} & {\bf k$_{3}$} \\
    \hline
        {\bf vdW} &    0    &    0    & -1-B       &     A         & -AB         \\
        {\bf RK}  &    1    &    0    & -1         &  A-B-B$^{2}$   & -AB         \\
        {\bf SRK} &    1    &    0    & -1         &  A-B-B$^{2}$   & -AB         \\
        {\bf PR}  &    2    &   -1    & -1+B       & A-2B-3B$^{2}$  & -AB+B$^{2}$+B$^{3}$ \\
    \hline
  \end{tabular}
  \caption{Values for $u$, $w$ and k$_{i}$ for vdW, RK, SRK and PR EOS --  Eqn.~\ref{Mod02_GeneralEOS}.}\label{Mod02:Table1}
  \end{center}
\end{table}
                   Coefficients for these expressions are listed in Table~\ref{Mod02:Table1}. Equation~\ref{Mod02_GeneralEOS} has three roots ($Z_{1}$, $Z_{2}$ and $Z_{3}$), the largest \underline{real positive root} represents the {\it vapour phase}, $Z_{\text{vap}}$, whereas the \underline{smallest real positive root} represents the {\it liquid phase}, $Z_{\text{liq}}$, the intermediate root has no physical meaning. The relevant roots for this cubic polynomial can be obtained from the following relations:
                   \begin{eqnarray}
                       Z_{\text{vap}} &=& 1 + \beta - q\beta \frc{Z_{\text{vap}} - \beta} {\left(Z_{\text{vap}}+\varepsilon\beta\right)\left(Z_{\text{vap}} +\sigma\beta\right)},\label{Mod02_Zvap} \\
                       Z_{\text{liq}} &=& 1 + \beta + \left(Z_{\text{liq}} + \epsilon\beta\right)\left(Z_{\text{liq}}+\sigma\beta\right)\left(\frc{1+\beta-Z_{\text{liq}}}{q\beta}\right)\label{Mod02_Zliq}
                   \end{eqnarray}
                   with 
                   \begin{displaymath}
                      \beta=\Omega\frc{P_{r}}{T_{r}},\;\;\; \text{ and }\;\;\; q=\frc{\Psi\alpha}{\Omega T_{r}}.
                   \end{displaymath}
                   Parameters for these expressions are listed in Table~\ref{Mod02:Table2} with
                   \begin{eqnarray}
                         \alpha_{\text{SRK}} &=& \left[ 1 + \left( 0.480 + 1.574 \omega - 0.176\omega^{2}\right)\left(1-\sqrt{T_{r}}\right)\right]^{2}, \nonumber \\
                         \alpha_{\text{PR}} &=& \left[ 1 + \left( 0.37464 + 1.54226 \omega - 0.26992\omega^{2}\right)\left(1-\sqrt{T_{r}}\right)\right]^{2}. \nonumber
                   \end{eqnarray}
                   Equations~\ref{Mod02_Zvap} and ~\ref{Mod02_Zliq} can be numerically solved either by a {\it calculator} or applying any method described in Appendix \red{E.2} (see Example~\ref{Mod02Ex01}).

\begin{table}[h]
    \begin{center}
       \begin{tabular}{| l | c c c c c| }
       \hline
          {\bf EOS}  & {\bf $\alpha$} & {\bf $\sigma$}  & {\bf $\varepsilon$} & {\bf $\Omega$} & {\bf $\Psi$ } \\
       \hline
            vdW      & 1              & 0               & 0                  & 1/8            & 27/64          \\
            RK       & T$_{r}^{-1/2}$  & 1                & 0                  & 0.08664       & 0.42748        \\
           SRK       &$\alpha_{\text{SRK}}$& 1            & 0                   & 0.08664       & 0.42748        \\
            PR       &$\alpha_{\text{PR}}$& 1+$\sqrt{2}$   & 1-$\sqrt{2}$        & 0.07780        & 0.45724  \\
       \hline
       \end{tabular}
  \caption{Parameters for Eqns.~\ref{Mod02_Zvap}-~\ref{Mod02_Zliq}.}\label{Mod02:Table2}
  \end{center}
\end{table}

                  

%
            \end{enumerate}
% 
   \end{enumerate}

\clearpage

%%% SUBSECTION
\subsection{Examples}

\begin{enumerate}[1)]
%%%
%%% EXAMPLE 
%%%
\item\label{Mod02Ex01} For gaseous methane at 298K and 2 MPa, compute the molar volume $\left(\text{in cm}^{3}.\text{mol}^{-1}\right)$ using the SRK EOS. Given $T_{c}=$ 190.7 K, $P_{c}=$ 46.41 bar and $\omega=$ 0.011.

% SOLUTION
        \noindent{\bf Solution:} We can calculate the molar volume of a real gas using the relation $PV=ZRT$, where the compressibility factor for the gaseous fluid can be obtained from Eqn.~\ref{Mod02_Zvap},
           \begin{displaymath}
                Z_{\text{vap}} = 1 + \beta - q\beta \frc{Z_{\text{vap}} - \beta} {\left(Z_{\text{vap}}+\varepsilon\beta\right)\left(Z_{\text{vap}} +\sigma\beta\right)},
           \end{displaymath}
           where 
           \begin{eqnarray}
               && T_{r}=\frc{T}{T_{c}} = 1.5627,\;\;P_{r}=\frc{P}{P_{c}}=0.4309,\;\;\omega=0.011,\;\;\Omega = 0.08664, \nonumber \\
               && \Psi = 0.42748,\;\; \sigma=1.0,\;\;\epsilon=0.0,\;\;\beta=\Omega\frc{P_{r}}{T_{r}}=2.3890\times 10^{-2},\nonumber \\
               && \alpha_{\text{SRK}} = 0.7667\;\;\text{ and }\;\; q = \frc{\Psi\alpha_{\text{SRK}}}{\Omega T_{r}} = 2.4207, \nonumber 
           \end{eqnarray}

           There are three ways to solve this non-linear equation, all of them involve iterative methods for root-finder
           \begin{enumerate}[a)]
%
              \item Using a calculator, leading to $Z_{\text{vap}} = $ 0.9670. 
%
              \item {\bf Substitution method:} In this method we rewrite the $Z_{\text{vap}}$ equation as,
                 \begin{displaymath}
                     Z_{\text{vap}}^{(i+1)} = 1 + \beta - q\beta \frc{Z_{\text{vap}}^{(i)} - \beta} {\left(Z_{\text{vap}}^{(i)}+\varepsilon\beta\right)\left(Z_{\text{vap}}^{(i)} +\sigma\beta\right)},
                 \end{displaymath}
                 where $i(=1, 2, \cdots n_{\text{max}})$ is an index for the number of iterations and $n_{\text{max}}$ is the total number of iterations. Taking the ideal gas condition as the initial guess in this iterative method, \ie $Z_{\text{vap}}^{(1)}=1$, 
                 \begin{eqnarray}
                    i=1 &\Rightarrow& Z_{\text{vap}}^{(2)} = 1 + \beta - q\beta \frc{Z_{\text{vap}}^{(1)} - \beta} {\left(Z_{\text{vap}}^{(1)}+\varepsilon\beta\right)\left(Z_{\text{vap}}^{(1)} +\sigma\beta\right)} = 0.96876 \nonumber \\
                    i=2 &\Rightarrow& Z_{\text{vap}}^{(3)} = 1 + \beta - q\beta \frc{Z_{\text{vap}}^{(2)} - \beta} {\left(Z_{\text{vap}}^{(2)}+\varepsilon\beta\right)\left(Z_{\text{vap}}^{(2)} +\sigma\beta\right)} = 0.96707 \nonumber \\
                    i=3 &\Rightarrow& Z_{\text{vap}}^{(4)} = 1 + \beta - q\beta \frc{Z_{\text{vap}}^{(3)} - \beta} {\left(Z_{\text{vap}}^{(3)}+\varepsilon\beta\right)\left(Z_{\text{vap}}^{(3)} +\sigma\beta\right)} = 0.96697 \nonumber \\
                    i=4 &\Rightarrow& Z_{\text{vap}}^{(5)} = 1 + \beta - q\beta \frc{Z_{\text{vap}}^{(4)} - \beta} {\left(Z_{\text{vap}}^{(4)}+\varepsilon\beta\right)\left(Z_{\text{vap}}^{(4)} +\sigma\beta\right)} = 0.96696 \nonumber \\
                    \vdots && \vdots \nonumber
                 \end{eqnarray}
                 these iterations can continue 'till either $i=n_{\text{max}}$ (\ie we reached the maximum number of iterations) or solution has converged (\ie there is no significant change in the solution after k$^{\text{th}}$ iteration). We can create a {\it stoppage criteria} to acess if the solution converged,
                 \begin{displaymath}
                      \mathcal{E} = \frc{\| Z_{\text{vap}}^{(i+1)} - Z_{\text{vap}}^{(i)}\|}{Z_{\text{vap}}^{(i)}}.
                 \end{displaymath}
                 If $\mathcal{E}$ is smaller than a prescribed tolerance, than we say that the method converged, otherwise we should continue our iterations. Assuming that $\mathcal{E} \leq 1.0\times 10^{-4}$,
                 \begin{center}
                     \begin{tabular}{c c c c}
                        \hline
                            $\mathbf{i}$  &  $\mathbf{Z_\text{vap}^{(i)}}$  & $\mathbf{Z_\text{vap}^{(i+1)}}$  & $\mathbf{\mathcal{E}}$ \\  
                               1         &           1.0                 &  0.96876                      &  3.12$\times$10$^{-2}$ \\
                               2         &       0.96876                 &  0.96707                      &  1.74$\times$10$^{-3}$ \\
                               3         &       0.96707                 &  0.96697                      &  1.03$\times$10$^{-4}$ \\
                               4         &       0.96697                 &  0.96696                      &  1.03$\times$10$^{-5}$ \\  
                        \hline
                     \end{tabular}
                 \end{center}
                 Thus after the forth iteration the solution converged to $Z_{\text{vap}}^{(4)} = 0.96696 \sim 0.9670$.
%
              \item {\bf Newton-Raphson Method:} This iterative method (see Appendix E.2) is based on dividing the domain in infinitesimal segments and smoothly `walking towards the solution'. It can be generalised as,
                  \begin{displaymath}
                     x^{(i+1)} = x^{(i)} - \frc{F\left(x^{(i)}\right)}{F^{\prime}\left(x^{(i)}\right)},
                  \end{displaymath}  
                  where $F^{\prime}\left(x^{(i)}\right) = \frc{d F\left(x^{(i)}\right)}{dx}$, \ie the first derivative of the function. In our case, the first step is to manipulate our $Z_{\text{vap}}$ equation as,
                  \begin{displaymath}
                     F\left(Z_{\text{vap}}\right) = Z_{\text{vap}} - 1 - \beta + q\beta \frc{Z_{\text{vap}} - \beta} {\left(Z_{\text{vap}}+\varepsilon\beta\right)\left(Z_{\text{vap}} +\sigma\beta\right)},
                  \end{displaymath}
                  and the Newton-Raphson Method becomes,
                  \begin{displaymath}
                     Z_{\text{vap}}^{(i+1)} = Z_{\text{vap}}^{(i)} - \frc{F\left(Z_{\text{vap}}^{(i)}\right)}{F^{\prime}\left(Z_{\text{vap}}^{(i)}\right)}.
                  \end{displaymath}
                  The next step is to obtain the derivative of the function $F\left(Z_{\text{vap}}\right)$ with respect to $Z_{\text{vap}}$, \ie (for $\epsilon=0$)
                  \begin{eqnarray}
                     F^{\prime}\left(Z_{\text{vap}}\right) &=& \frc{d F^{\prime}\left(Z_{\text{vap}}\right)}{d Z_{\text{vap}}} = \frc{d}{dZ_{\text{vap}}} \left[Z_{\text{vap}} - 1 - \beta + q\beta \frc{Z_{\text{vap}}-\beta}{Z_{\text{vap}}\left(Z_{\text{vap}}+\sigma\beta\right)}\right]\nonumber \\
                                                       &=& 1 - \frc{q\beta}{\left(Z_{\text{vap}}+\sigma\beta\right)^{2}} + \frc{q\beta^{2}\left(2Z_{\text{vap}}+\sigma\beta\right)}{\left[Z_{\text{vap}}\left(Z_{\text{vap}}+\sigma\beta\right)\right]^{2}}. \nonumber
                  \end{eqnarray}
                  This leads to,
                  \begin{eqnarray}
                     Z_{\text{vap}}^{(i+1)} &=& Z_{\text{vap}}^{(i)} - \frc{F\left(Z_{\text{vap}}^{(i)}\right)}{F^{\prime}\left(Z_{\text{vap}}^{(i)}\right)} \nonumber \\
                     &=& Z_{\text{vap}}^{(i)} - \frc{Z_{\text{vap}}^{(i)} - 1 - \beta + q\beta \frc{Z_{\text{vap}}^{(i)} - \beta} {Z_{\text{vap}}^{(i)}\left(Z_{\text{vap}}^{(i)} +\sigma\beta\right)}}{1 - \frc{q\beta}{\left(Z_{\text{vap}}^{(i)}+\sigma\beta\right)^{2}} + \frc{q\beta^{2}\left(2Z_{\text{vap}}^{(i)}+\sigma\beta\right)}{\left[Z_{\text{vap}}^{(i)}\left(Z_{\text{vap}}^{(i)}+\sigma\beta\right)\right]^{2}}}. \nonumber
                  \end{eqnarray}
                  Using ideal gas as initial guess, \ie $Z_{\text{vap}}^{(1)}=1$ and $\mathcal{E} \leq 1.0\times 10^{-4}$,
                 \begin{center}
                     \begin{tabular}{c c c c}
                        \hline
                            $\mathbf{i}$  &  $\mathbf{Z_\text{vap}^{(i)}}$  & $\mathbf{Z_\text{vap}^{(i+1)}}$  & $\mathbf{\epsilon}$ \\  
                               1         &           1.0                 &  0.96703                      &  3.30$\times$10$^{-2}$ \\
                               2         &       0.96703                 &  0.96697                      &  6.20$\times$10$^{-5}$ \\
                               3         &       0.96697                 &  0.96697                      & $<$ 1$\times$10$^{-5}$ \\ 
                        \hline
                     \end{tabular}
                 \end{center}
                 The solution rapidly converged after the second iteration to $Z_{\text{vap}}^{(2)}=0.96697\sim 0.9670$. 
%
           \end{enumerate}
\medskip

           Now replacing $Z_{\text{vap}} = 0.9670$ in 
           \begin{displaymath}
              V=\frc{Z_{\text{vap}}R T}{P} = 1197.9493 \text{ cm}^{3}.\text{mol}^{-1}\sim 1197.95 \text{ cm}^{3}.\text{mol}^{-1}
           \end{displaymath}
\clearpage
%%%
%%% EXAMPLE 
%%%
\item\label{Mod02Ex02} Calculate the molar volume $\left(\text{in cm}^{3}.\text{mol}^{-1}\right)$ for gaseous butane at 2.5 bar and 298 K using (a) the ideal gas equation, (b) the truncated virial EOS and (c) the PR EOS. Given $T_{c}=$ 425.1 K, $P_{c}=$ 37.96 bar and $\omega=$ 0.20.

% SOLUTION
        \noindent{\bf Solution:} 
           \begin{enumerate}[a)]
%
               \item Assuming that butane behaves as an ideal gas,
                    \begin{displaymath}
                        V^{\text{IG}} = \frc{RT}{P} = 9910.6456 \text{ cm}^{3}.\text{mol}^{-1}
                    \end{displaymath}
%
               \item From Eqn.~\ref{Mod02_Virial1c},
                  \begin{displaymath}
                     Z = \frc{P V}{R T} = 1 + \frc{B P_{c}}{R T_{c}}\frc{P_{r}}{T_{r}},
                  \end{displaymath}
                  manipulating this equation with 
                  \begin{displaymath}
                     \frc{B P_{c}}{R T_{c}} = B^{0} + \omega B^{1},
                  \end{displaymath}
                  leads to
                  \begin{displaymath}
                     V = \frc{R T}{P} + \left(B^{0} + \omega B^{1}\right) \frc{P_{r}R T}{T_{r}P} = \frc{R T}{P} + \left(B^{0} + \omega B^{1}\right)\frc{RT_{c}}{P_{c}},
                  \end{displaymath}
                  where $P_{r}=\frac{P}{P_{c}}$ and $T_{r}=\frac{T}{T_{c}}$. Calculating $B^{0}$ and $B^{1}$ from
                  \begin{displaymath}
                     B^{0} = 0.083 - \frc{0.422}{T_{r}^{1.6}} = -0.6620\;\;\text{ and }\;\; B^{1} = 0.139 - \frc{0.172}{T_{r}^{4.2}}=-0.6257.
                  \end{displaymath}
                  Thus $V^{\text{virial}} = 9430.6465$ cm$^{3}$.mol$^{-1}$.
%
               \item Now, in order to calculate the molar volume using the PR EOS, we first need to calculate the compressibility factor $Z_{\text{vap}}$, Eqn.~\ref{Mod02_Zvap},
                  \begin{displaymath}
                     Z_{\text{vap}} = 1 + \beta - q\beta \frc{Z - \beta} {\left(Z+\varepsilon\beta\right)\left(Z+\sigma\beta\right)},
                  \end{displaymath}
                  with
                  \begin{eqnarray}
                     && T_{r}=\frc{T}{T_{c}} = 0.7010,\;\;P_{r}=\frc{P}{P_{c}}=6.5859\times 10^{-2},\;\;\omega=0.20,\;\;\Omega = 0.07880, \nonumber \\
                     && \Psi = 0.45724,\;\; \sigma=1+\sqrt{2},\;\;\epsilon=1-\sqrt{2},\;\;\beta=\Omega\frc{P_{r}}{T_{r}}=7.3281\times 10^{-4},\nonumber \\
                     && \alpha_{\text{PR}} = 1.2308\;\;\text{ and }\;\; q = \frc{\Psi\alpha_{\text{PR}}}{\Omega T_{r}} = 102.9246, \nonumber 
                  \end{eqnarray}
                  Solving numerically leads to $Z_{\text{vap}}=0.9188$ and $V=\frc{Z_{\text{vap}}RT}{P} = 9105.9012$ cm$^{3}$.mol$^{-1}$.
%
           \end{enumerate}
\clearpage
%%%
%%% EXAMPLE 
%%%
   \item\label{Mod02Ex03} Derive the relations for coefficient of thermal expansion and the isothermal compressibility coefficient for
              \begin{enumerate}[(a)]
                  \item Ideal gas EOS;
                  \item $V=\frc{a}{RT}+\frc{bT}{PR}$ 
              \end{enumerate}

% SOLUTION
        \noindent{\bf Solution:} Here we need to obtain expressions for 
           \begin{eqnarray}
                && \beta = \frc{1}{V}\left(\frc{\partial V}{\partial T}\right)_{P} \nonumber \\
                && \kappa = -\frc{1}{V}\left(\frc{\partial V}{\partial P}\right)_{T} \nonumber
           \end{eqnarray}
           for 
           \begin{enumerate}[a)]
%
               \item Ideal gas EOS, $V=\frc{R T}{P}$,
                    \begin{displaymath}
                       \beta =\frc{P}{R T}\frc{R}{P} = \frc{1}{T},
                    \end{displaymath}
                    and 
                    \begin{displaymath}
                       \kappa = -\frc{P}{R T}\left(-\frc{R T}{P^{2}}\right) = \frc{1}{P}
                    \end{displaymath}
%
               \item $V=\frc{a}{RT}+\frc{bT}{PR}$: Solving the partial differentials,
                    \begin{displaymath}
                       \left(\frc{\partial V}{\partial T}\right)_{P} = \frc{b}{PR}-\frc{a}{RT^{2}}\;\;\text{ and }\;\; \left(\frc{\partial V}{\partial P}\right)_{T} = -\frc{bT}{P^{2}R},
                    \end{displaymath}
                    thus
                    \begin{eqnarray}
                       && \beta =\frc{1}{\frc{a}{RT}+\frc{bT}{PR}}\left(\frc{b}{PR}-\frc{a}{RT^{2}}\right) = \frc{\frc{bT^{2}-aP}{PRT^{2}}}{\frc{aP+bT^{2}}{PRT}} = \frc{bT^{2}-aP}{T\left(aP+bT^{2}\right)}
 \nonumber \\
                       && \kappa = -\frc{1}{\frc{a}{RT}+\frc{bT}{PR}} \left(-\frc{bT}{P^{2}R}\right) = \frc{bT^{2}}{P\left(aP+bT^{2}\right)} \nonumber
                    \end{eqnarray}
                    
%
           \end{enumerate}
%     
\end{enumerate}
 % Volumetric Properties of Pure Substances
     \setcounter{examplecounter}{0}
  \chapter{Thermodynamic Properties of Pure Fluids}\label{Chapter:ThermodynamicPropertiesPureFluids}

   \begin{LearningObjectivesBlock}{Learning Objectives}
      Upon completion of this chapter, you will be able to
        \begin{enumerate}
           \item Define pure substances and thermodynamic phases;
           \item State the Gibbs phase rule and its use to define the number of degrees of freedom of a system;
           \item Identify phases and phase transitions in a diagram;
           \item Formulate the solution for thermodynamic problems involving the calculation of volume of pure substances;
           \item Select appropriate equation of state for a given problem/application;
        \end{enumerate}
\medskip
     Recommended reading: Chapters 3 of \citet{SmithVanNess_Book}, 6 of \citet{Sandler_Book}, 2 of \citet{Borgnakke_Book} or 4 of \citet{Atkins_Book}.
   \end{LearningObjectivesBlock}


%%%%%%%%%%%%%%%%%%%%%%%%%%%%%%%%%%%%%%%%%%%%%%%%%%%%%%%%%%%%%%%%%
\begin{comment}
   \begin{LearningObjectivesBlock}{Learning Objectives}
      Upon completion of this chapter, you will be able to
        \begin{enumerate}
           \item {\bf Knowledge:} Define, Name, Select, State 
           \item {\bf Comprehension:} Describe, Identify, Discuss
           \item {\bf Application:} Apply, Demonstrate, Employ, Sketch
           \item {\bf Analysis:} Analyse, Compare, Calculate, Solve
           \item {\bf Synthesis:} Determine, Formulate
           \item {\bf Evaluation:} Assess, Check, Estimate, Compare, Measure, Monitor
        \end{enumerate}
\end{comment}
%%%%%%%%%%%%%%%%%%%%%%%%%%%%%%%%%%%%%%%%%%%%%%%%%%%%%%%%%%%%%%%%%

%%%% ETOC
\localtableofcontents
   

%%% SECTION
\section{Introduction}\label{Chapter:ThermodynamicPropertiesPureFluids:Section:Introduction}
   This chapter is an introduction to some of the mathematical underpinnings of chemical thermodynamics. Practical applications of most of the contents of this chapters may not be promptly obvious, however this is critical to fully understand concepts, theories and practices that will be introduced in later chapters. MORE!!

%%% SECTION
\section{Thermodynamic Property Relations for Single Phase Systems}\label{Chapter:ThermodynamicPropertiesPureFluids:Section:ThermodynamicPropertiesSinglePhase}
     \begin{subequations}

In Chapter~\ref{Chapter:FirstLaw}, we have learnt that the First Law for reversible processes in closed systems can be written as (for $n$ moles),
   \begin{equation}
       d\left(n U\right) = d Q_{\text{rev}} + d W_{\text{rev}},\label{Chapter:ThermodynamicPropertiesPureFluids:Eqn:FirstLaw}
   \end{equation} 
with $d W_{\text{rev}} = -Pd(nV)$ and $d Q_{\text{rev}}=Td(nS)$,
   \begin{equation}
       d\left(n U\right) = T d(nS) - Pd(nV).\label{Chapter:ThermodynamicPropertiesPureFluids:Eqn:FirstSecondLaw}
   \end{equation} 
This relation involves {\it only state functions}, therefore it is not restricted to reversible processes. The only constraint of this relation is that it is defined for closed systems and it assumes that changes may occur between equilibrium states. Finally, Eqn.~\ref{Chapter:ThermodynamicPropertiesPureFluids:Eqn:FirstSecondLaw} involves five thermodynamic properties: $P, T, V, U$ and $S$, and for $n=1$ mol, it becomes,
   \begin{shaded}
     \begin{equation}
        TdS = dU + PdV,\label{Chapter:ThermodynamicPropertiesPureFluids:Eqn:FundamentalRelation01}
     \end{equation}
   \end{shaded}
however, we know the definition of enthalpy,
     \begin{displaymath}
        dH = dU + d(PV) = dU + PdV + VdP.
     \end{displaymath}
Replacing this relation in Eqn.~\ref{Chapter:ThermodynamicPropertiesPureFluids:Eqn:FundamentalRelation01},
   \begin{shaded}
     \begin{equation}
        TdS = dH - VdP.\label{Chapter:ThermodynamicPropertiesPureFluids:Eqn:FundamentalRelation02}
     \end{equation}
   \end{shaded}
Equations~\ref{Chapter:ThermodynamicPropertiesPureFluids:Eqn:FundamentalRelation01}-\ref{Chapter:ThermodynamicPropertiesPureFluids:Eqn:FundamentalRelation02} correlate entropy changes to changes in other thermodynamic properties
      \begin{eqnarray}
         && dU = TdS - PdV \nonumber \\
         && dH = TdS + VdP. \nonumber 
      \end{eqnarray}
These two fundamental relations can be used to define the two remaining thermodynamic potentials, Gibbs ($G$) free and Helmholtz free ($A$) energy functions as:
      \begin{eqnarray}
         && A \equiv U -TS \nonumber \\
         && G \equiv H -TS, \nonumber 
      \end{eqnarray}
or in differential form,
      \begin{eqnarray}
        && dA = dU -TdS - SdT \nonumber \\
        && dG = dH -TdS - SdT.\nonumber 
      \end{eqnarray}
Using Eqns.~\ref{Chapter:ThermodynamicPropertiesPureFluids:Eqn:FundamentalRelation01}-\ref{Chapter:ThermodynamicPropertiesPureFluids:Eqn:FundamentalRelation02}:
   \begin{shaded}
      \begin{eqnarray}
        && dA = - SdT - PdV, \label{Chapter:ThermodynamicPropertiesPureFluids:Eqn:HelmholtzFundamentalRelation01}\\ 
        && dG = - SdT + VdP. \label{Chapter:ThermodynamicPropertiesPureFluids:Eqn:GibbsFundamentalRelation01}
      \end{eqnarray}
   \end{shaded}
These four equations, Eqns.~\ref{Chapter:ThermodynamicPropertiesPureFluids:Eqn:FundamentalRelation01}-~\ref{Chapter:ThermodynamicPropertiesPureFluids:Eqn:GibbsFundamentalRelation01} are known as {\it fundamental thermodynamic relations} as they correlate the five thermodynamic potentials -- $U$, $H$, $S$, $G$, and $A$ with $PVT$ properties.


     \end{subequations}

%%% SECTION
\section{Maxwell Relations}\label{Section:03:MaxwellRelations}
     \begin{subequations}

The thermodynamic potentials expressed in the four fundamental relations (Eqns.~\ref{Chapter:ThermodynamicPropertiesPureFluids:Eqn:FundamentalRelation01}-~\ref{Chapter:ThermodynamicPropertiesPureFluids:Eqn:GibbsFundamentalRelation01}) can be defined as a general functional, 
   \begin{displaymath}
    f = f(a,b),
   \end{displaymath}
with the assumption that the thermodynamic potentials, $f$, are continuous (and differentiable) throughout the domain. Therefore, we can write the total derivative\footnote{Total derivative along with the main elements of Calculus is briefly defined in Appendix~\ref{Appendix_Calculus:TotalDifferential}.} of function $f$ as
   \begin{equation}
         df = \underbrace{\left(\frc{\partial f}{\partial a}\right)_{b}}_{M}da + \underbrace{\left(\frc{\partial f}{\partial b}\right)_{a}}_{N}db = Mda + Ndb.\label{Chapter:ThermodynamicPropertiesPureFluids:Eqn:EqualRelation0}
   \end{equation}
Now, if we differentiate $M$ \wrt $b$ and $N$ \wrt $a$,
   \begin{displaymath}
         \left(\frc{\partial M}{\partial b}\right)_{a} = \frc{\partial^{2}f}{\partial a\partial b} \;\;\;\text{ and }\;\;\; \left(\frc{\partial N}{\partial a}\right)_{b} = \frc{\partial^{2}f}{\partial b\partial a},
   \end{displaymath}
since we required that $f(a,b)$ to be \underline{continuous}, 
   \begin{equation}
         \frc{\partial^{2}f}{\partial a\partial b} = \frc{\partial^{2}f}{\partial b\partial a}\;\;\Longrightarrow \left(\frc{\partial M}{\partial b}\right)_{a} = \left(\frc{\partial N}{\partial a}\right)_{b}\label{Chapter:ThermodynamicPropertiesPureFluids:Eqn:EqualRelation}
   \end{equation}
Now applying Eqn.~\ref{Chapter:ThermodynamicPropertiesPureFluids:Eqn:EqualRelation} into ~\ref{Chapter:ThermodynamicPropertiesPureFluids:Eqn:FundamentalRelation01},
   \begin{displaymath}
       \begin{cases}
          df = Mda + Ndb, & \\
          dU = TdS - PdV, &
       \end{cases}
   \end{displaymath}
with $M=T$, $da=dS$, $N=-P$, $db = dV$ and $df = dU$ leading to,
      \begin{shaded}
         \begin{equation}
            \left(\frc{\partial M}{\partial b}\right)_{a} = \left(\frc{\partial N}{\partial a}\right)_{b} \;\;\;\Longrightarrow \left(\frc{\partial T}{\partial V}\right)_{S} = - \left(\frc{\partial P}{\partial S}\right)_{V}\label{Chapter:ThermodynamicPropertiesPureFluids:Eqn:MaxwellRelation1}
         \end{equation}
Using the same procedure with  Eqn.~\ref{Chapter:ThermodynamicPropertiesPureFluids:Eqn:FundamentalRelation02}-~\ref{Chapter:ThermodynamicPropertiesPureFluids:Eqn:GibbsFundamentalRelation01},
           \begin{eqnarray}
              \left(\frc{\partial T}{\partial P}\right)_{S} &=& \left(\frc{\partial V}{\partial S}\right)_{P},\label{Chapter:ThermodynamicPropertiesPureFluids:Eqn:MaxwellRelation2} \\
              \left(\frc{\partial S}{\partial V}\right)_{T} &=& \left(\frc{\partial P}{\partial T}\right)_{V},\label{Chapter:ThermodynamicPropertiesPureFluids:Eqn:MaxwellRelation3} \\
              -\left(\frc{\partial S}{\partial P}\right)_{T} &=& \left(\frc{\partial V}{\partial T}\right)_{P},\label{Chapter:ThermodynamicPropertiesPureFluids:Eqn:MaxwellRelation4} 
           \end{eqnarray}
      \end{shaded}
Equations~\ref{Chapter:ThermodynamicPropertiesPureFluids:Eqn:MaxwellRelation1}-~\ref{Chapter:ThermodynamicPropertiesPureFluids:Eqn:MaxwellRelation4} are called as \blue{Maxwell relations}, and they allow determining changes in entropy without directly measurement, but only with changes on the PVT properties. We can also make use of the total derivative definition and apply it into Eqns.~\ref{Chapter:ThermodynamicPropertiesPureFluids:Eqn:FundamentalRelation01}-~\ref{Chapter:ThermodynamicPropertiesPureFluids:Eqn:GibbsFundamentalRelation01}, thus
   \begin{eqnarray}
                      &                            df =  \left(\frc{\partial f}{\partial a}\right)_{b}da + \left(\frc{\partial f}{\partial b}\right)_{a}db& \nonumber \\
      dU = TdS - PdV  &\;\;\;\Longleftrightarrow \;\;\;& dU =  \underbrace{\left(\frc{\partial U}{\partial S}\right)_{V}}_{T}dS + \underbrace{\left(\frc{\partial U}{\partial V}\right)_{S}}_{-P}dV \nonumber \\
      dH = TdS + VdP  &\;\;\;\Longleftrightarrow \;\;\;& dH =  \underbrace{\left(\frc{\partial H}{\partial S}\right)_{P}}_{T}dS + \underbrace{\left(\frc{\partial H}{\partial P}\right)_{S}}_{V}dP \nonumber \\
      dA = -SdT - PdV &\;\;\;\Longleftrightarrow \;\;\;& dA =  \underbrace{\left(\frc{\partial A}{\partial T}\right)_{V}}_{-S}dT + \underbrace{\left(\frc{\partial A}{\partial V}\right)_{T}}_{-P}dV \nonumber \\
      dG = -SdT + VdP &\;\;\;\Longleftrightarrow \;\;\;& dG =  \underbrace{\left(\frc{\partial G}{\partial T}\right)_{P}}_{-S}dT + \underbrace{\left(\frc{\partial G}{\partial P}\right)_{T}}_{V}dP \nonumber 
   \end{eqnarray}
Therefore
   \begin{shaded}
         \begin{eqnarray}
             \left(\frc{\partial U}{\partial S}\right)_{V} & =  T = & \left(\frc{\partial H}{\partial S}\right)_{P}\label{Chapter:ThermodynamicPropertiesPureFluids:Eqn:MaxwellRelation5} \\
             \left(\frc{\partial U}{\partial V}\right)_{S} & = -P = & \left(\frc{\partial A}{\partial V}\right)_{T}\label{Chapter:ThermodynamicPropertiesPureFluids:Eqn:MaxwellRelation6} \\
             \left(\frc{\partial H}{\partial P}\right)_{S} & =  V = & \left(\frc{\partial G}{\partial P}\right)_{T}\label{Chapter:ThermodynamicPropertiesPureFluids:Eqn:MaxwellRelation7} \\
             \left(\frc{\partial A}{\partial T}\right)_{V} & = -S = & \left(\frc{\partial G}{\partial T}\right)_{P}\label{Chapter:ThermodynamicPropertiesPureFluids:Eqn:MaxwellRelation8}
         \end{eqnarray}
   \end{shaded}
These property relations, Eqns.~\ref{Chapter:ThermodynamicPropertiesPureFluids:Eqn:MaxwellRelation5}-~\ref{Chapter:ThermodynamicPropertiesPureFluids:Eqn:MaxwellRelation8}, easily correlate measurable properties ($P$, $V$ and $T$) with non-measurable potentials ($U$, $H$, $S$, $A$ and $G$).

      \end{subequations}


   % Example
   \begin{MyExample}{\begin{center}{\bf Example}\end{center}}
     \begin{example}\label{Chapter:ThermodynamicPropertiesPureFluids:Example1} \citep{Balmer_Book}
         Suppose we make a series of measurements in the laboratory and think we discovered a new thermodynamic property, call it $\eta$. Our experimental data provide an empirical equation of the form,
          \begin{displyamth}
             d\eta = PdV + V^{2}dP
          \end{displaymath}
          Is $\eta$ a new property?
     \end{example}

% SOLUTION
       \noindent{\bf Solution:}
        The unknown is whether or not $\eta$ is a new thermodynamic property. Using Eqn.~\ref{Chapter:ThermodynamicPropertiesPureFluids:Eqn:EqualRelation0},
        \begin{displaymath}
            d\eta = Mda + Ndb = PdV + V^{2}dP
        \end{displaymath}
          
   \end{MyExample}
   

%%% SECTION
\section{Relations for Internal Energy, Enthalpy and Entropy}\label{Section:03:U_H_S_Relations}

Maxwell relations can help to develop relations for changes in $U$, $H$ and $S$ that reflect in heat and work interactions. Given
    \begin{center}
       $\begin{cases}
           H = H(T,P), \text{ and } \\
           S = S(T,P),
        \end{cases}$ 
    \end{center}
\ie defining {\it enthalpy} and {\it entropy} as functions of temperature and pressure. From Eqn.~\ref{Chapter:ThermodynamicPropertiesPureFluids:Eqn:FundamentalRelation02},
    \begin{eqnarray}
        dH &=& TdS + VdP\;\;\text{ at constant } P\hspace{1cm} \blue{\times\left(1/dT\right)} \nonumber \\
        \Partial[H]{T}{P} &=& C_{p} = T\Partial[S]{T}{P} \nonumber \\
        \Partial[S]{T}{P} &=& \frc{C_{p}}{T}\label{Mod03_DerivedEntropyRelation1}
    \end{eqnarray}
From Eqn.~\ref{Chapter:ThermodynamicPropertiesPureFluids:Eqn:FundamentalRelation02} at constant $T$, multiplied by $(1/dP)$,
    \begin{displaymath}
       \Partial[H]{P}{T} = T\underbrace{\Partial[S]{P}{T}}_{\text{Eqn.~\ref{Chapter:ThermodynamicPropertiesPureFluids:Eqn:MaxwellRelation4}}} + V = -T\Partial[V]{T}{P} + V.
    \end{displaymath}
Now if we write the derivatives of functions $H$ and $S$,
    \begin{displaymath}
       dH = \Partial[H]{T}{P}dT + \Partial[H]{P}{T}dP 
    \end{displaymath}
    \begin{subequations}
      \begin{shaded}
         \begin{equation}
            dH = C_{p}dT + \left[V - T\Partial[V]{T}{P}\right]dP,\label{Mod03_DerivedEnthalpyRelation1}
         \end{equation}
      \end{shaded}
and using Eqn.~\ref{Mod03_DerivedEntropyRelation1},
      \begin{displaymath}
         dS = \underbrace{\Partial[S]{T}{P}}_{\text{Eqn.~\ref{Mod03_DerivedEntropyRelation1}}}dT + \overbrace{\Partial[S]{P}{T}}^{\text{Eqn.~\ref{Chapter:ThermodynamicPropertiesPureFluids:Eqn:MaxwellRelation4}}}dP 
       \end{displaymath}
       \begin{shaded}
          \begin{equation}
             dS = C_{p}\frc{dT}{T} - \Partial[V]{T}{P}dP\label{Mod03_DerivedEntropyRelations}
          \end{equation}
       \end{shaded}
   \end{subequations}

\medskip

\begin{shaded}
      \noindent
      For an \blue{ideal gas}, $V=\frac{RT}{P}$ and $\Partial[V]{T}{P} = \frc{R}{P}$, therefore Eqn.~\ref{Mod03_DerivedEnthalpyRelation1} becomes
         \begin{eqnarray}
            dH &=& C_{p}dT + \left[V-\overbrace{T\frc{R}{P}}^{=V}\right]dP \nonumber \\
            dH &=& C_{p}dT,\label{Mod03_DerivedEnthalpyRelation2}
         \end{eqnarray}
      and Eqn.~\ref{Mod03_DerivedEntropyRelations} becomes,
         \begin{equation}
            dS = C_{p}\frc{dT}{T} -\frc{R}{P}dP.\label{Mod03_DerivedEntropyRelations2}
         \end{equation}
\end{shaded}

\bigskip

Similarly, we can express $U$ and $S$ as functions of $T$ and $V$,
    \begin{center}
       $\begin{cases}
           U = U(T,V), \text{ and } \\
           S = S(T,V),
        \end{cases}$
    \end{center} 
using the same procedure as above starting from the derivatives of functions $U$ and $S$,
    \begin{eqnarray}
        dU &=& \Partial[U]{T}{V}dT + \Partial[U]{V}{T}dV, \text{ and } \nonumber \\
        dS &=& \Partial[S]{T}{V}dT + \Partial[S]{V}{T}dV. \nonumber
    \end{eqnarray} 
From Eqn.~\ref{Chapter:ThermodynamicPropertiesPureFluids:Eqn:FundamentalRelation01}, 
    \begin{displaymath}
        dU = TdS - PdV,
        \begin{cases}
            \red{\times\Partial[]{T}{V}} \Longrightarrow \Partial[U]{T}{V} = T\Partial[S]{T}{V} = C_{v} \Longrightarrow  \Partial[S]{T}{V} = \frc{C_{v}}{T}\\
            \red{\times\Partial[]{V}{T}} \Longrightarrow \underbrace{\Partial[U]{V}{T}}_{\text{Eqn.~\ref{Chapter:ThermodynamicPropertiesPureFluids:Eqn:MaxwellRelation3}}} = T\Partial[S]{V}{T} - P \Longrightarrow \Partial[U]{V}{T} = T\Partial[P]{T}{V}-P   \\
        \end{cases}
    \end{displaymath}
Substituting these relations in the total differentials,
    \begin{shaded}
       \begin{subequations}
          \begin{eqnarray}
             dU = \Partial[U]{T}{V}dT + \Partial[U]{V}{T}dV \;\;&\Longrightarrow&\;\; dU = C_{v}dT + \left[T\Partial[P]{T}{V}-P\right]dV\label{Mod03_DerivedIntEnergyRelation1} \\
             dS = \Partial[S]{T}{V}dT + \Partial[S]{V}{T}dV \;\;&\Longrightarrow&\;\; dS = \frc{C_{v}}{T}dT + \Partial[P]{T}{V}dV.\label{Mod03_DerivedEntropyRelations3}
          \end{eqnarray} 
       \end{subequations}
    \end{shaded}

Equation~\ref{Mod02_Compressibilityexpansivity2} (Module~\ref{Section:02}) relates the isothermal compressibility, $\kappa$, expansivity, $\beta$ coefficients to $T$, $P$ and $V$,
     \begin{eqnarray}
        \frc{dV}{V} &=& \beta dT - \kappa dP \;\;\;\red{\times\Partial[]{T}{V} \nonumber \text{ \ie at }V\text{ constant}} \nonumber \\
            0       &=& \beta - \kappa\Partial[P]{T}{V} \Longrightarrow \Partial[P]{T}{V} = \frc{\beta}{\kappa} \nonumber
     \end{eqnarray}
This relation can be used in Eqns.~\ref{Mod03_DerivedIntEnergyRelation1} and~\ref{Mod03_DerivedEntropyRelations3},
     \begin{eqnarray}
        dU &=& C_{v}dT + \left[T\frc{\beta}{\kappa}-P\right]dV\label{Mod03_DerivedIntEnergyRelation2} \\
        dS &=& \frc{C_{v}}{T}dT + \frc{\beta}{\kappa}dV.\label{Mod03_DerivedEntropyRelations4}
     \end{eqnarray}

\begin{table}[h]
   \begin{shaded} 
       \begin{center}
           {\bf $\kappa$ and $\beta$ Relations for Liquids}
       \end{center}
       \begin{subequations}
          For liquids, using $\beta$ and $\kappa$ in Eqns.~\ref{Mod03_DerivedEnthalpyRelation1}-~\ref{Mod03_DerivedEntropyRelations},
            \begin{eqnarray}
               \Partial[S]{P}{T} &=& -\Partial[V]{T}{P} = -\beta V \nonumber \\
               \Partial[H]{P}{T} &=& V - T\Partial[V]{T}{P} = V - TV\beta = \left(1-\beta T\right)V, \nonumber
            \end{eqnarray}
          leading to
            \begin{eqnarray}
               dH &=& C_{p}dT + \left(1-\beta T\right)VdP \label{Mod03_DerivedEnthalpyLiquid}\\
               dS &=& C_{p}\frc{dT}{T} - \beta V dP.\label{Mod03_DerivedEntropyRelationsLiquid}
            \end{eqnarray}
          For internal energy, we can write,
            \begin{eqnarray}
               dU &=& dH - d(PV) = dH -PdV -VdP\;\;\;\;\blue{\times\Partial[]{P}{T}} \nonumber \\
               \Partial[U]{P}{T} &=& \underbrace{\Partial[H]{P}{T}}_{\red{\cancel{V}}-T\Partial[V]{T}{P}} - P\Partial[V]{P}{T} - \red{\cancel{V}} \nonumber \\
                                 &=& - T\underbrace{\Partial[V]{T}{P}}_{\beta V} - P \underbrace{\Partial[V]{P}{T}}_{\kappa V} \nonumber \\
               \Partial[U]{P}{T} &=& \left(\kappa P - \beta T\right)V \label{Mod03_DerivedInternalEnergyLiquid}
            \end{eqnarray}
          \underline{These relations are only applied to {\it liquids}.}
       \end{subequations}
    \end{shaded}
\end{table}


%%% SECTION
\section{Gibbs Free Energy as Generating Function}\label{Section:03:GibbsGeneratingFunction}

The fundamental relation for the Gibbs free energy, Eqn.~\ref{Chapter:ThermodynamicPropertiesPureFluids:Eqn:GibbsFundamentalRelation01}, demonstrated that this potential can be expressed as a function of temperature and pressure, $G=G(T,P)$. In chemical industrial (or lab) processes, both properties, $T$ and $P$, can be readily measured and controlled, and therefore used to obtain $G$. A convenient way to deal with this relation is rewriting it in dimensionless form
    \begin{eqnarray}
         d\left(\frc{G}{RT}\right) &\equiv& \frc{1}{RT}dG - \frc{GR}{\left(RT\right)^{2}}dT = \frc{1}{RT}\underbrace{dG}_{\text{using Eqn.~\ref{Chapter:ThermodynamicPropertiesPureFluids:Eqn:GibbsFundamentalRelation01}}} - \overbrace{\frc{G}{RT^{2}}}^{\text{using }G=H-TS}dT \nonumber \\
                                    & =& \frc{V}{RT}dP - \red{\cancel{\frc{S}{RT}dT}} - \frc{H}{RT^{2}}dT + \red{\cancel{\frc{S}{RT}dT}} \nonumber
    \end{eqnarray}

    \begin{subequations}
       \begin{shaded} 
          \begin{eqnarray}
              && d\left(\frc{G}{RT}\right) = \frc{V}{RT}dP - \frc{H}{RT^{2}}dT\label{Mod03_GibbsGeneratingFunction} \\
              && \frc{V}{RT} = \Partial[(G/RT)]{P}{T} \;\;\blue{\text{ for } T\text{ constant}},\label{Mod03_GibbsGeneratingFunctionTConst} \\
              && \frc{H}{RT} = -T\Partial[(G/RT)]{T}{P} \;\;\blue{\text{ for } P\text{ constant}}.\label{Mod03_GibbsGeneratingFunctionPConst}
          \end{eqnarray}
        \end{shaded}
        With a similar procedure we can obtain generating functions from
        \begin{shaded}
           \begin{eqnarray}
              \frc{S}{R} &=& \frc{H}{RT} - \frc{G}{RT}   \;\;\blue{\Longleftarrow \; G=H-TS \;\;\times(1/RT)}\label{Mod03_EntropyGeneratingFunction} \\
              \frc{U}{RT} &=& \frc{H}{RT} - \frc{PV}{RT} \;\;\blue{\Longleftarrow \; U=H-PV \;\;\times(1/RT)}\label{Mod03_IntEnergyGeneratingFunction}
           \end{eqnarray}
        \end{shaded}
    \end{subequations}


%%% SECTION
\section{Residual Properties}\label{Section:03:ResidualProperties}

Another way to calculate energy and entropy changes for real gases is by defining {\it residual properties}, \ie the difference between any extensive thermodynamic property, $M$ (\eg $V$, $U$, $H$, $S$, $G$ and $A$), in real gases and its equivalent assuming ideal gas behaviour, $M^{\text{ig}}$,
   \begin{shaded}
      \begin{displaymath}
         M^{R} \equiv M - M^{\text{ig}},
      \end{displaymath}
   \end{shaded}
thus \eg
   \begin{subequations}
      \begin{equation}
         V^{R} = V - V^{\text{ig}} = \underbrace{V}_{=\frac{zRT}{P}} - \frc{RT}{P} = \frc{RT}{P}\left(Z-1\right).\label{Mod03_ResidualProperties_V}
      \end{equation}
      The {\it residual properties} concept is \underline{only} used for gases, and it is based in the idea that properties of real gases are often close of ideal gas behaviour at similar conditions. We can use the {\it generating function} concept, defined in Section~\ref{Section:03:GibbsGeneratingFunction}, to compute changes in properties based on the Gibbs free energy (Eqn.~\ref{Mod03_GibbsGeneratingFunction}),
      \begin{shaded}
         \begin{equation}
            d\left(G^{R}/RT\right) = d\left(G/RT\right) - d\left(G^{\text{ig}}/RT\right) = \frc{V^{R}}{RT}dP - \frc{H^{R}}{RT^{2}}dT\label{Mod03_ResidualProperties_GeneratingGibbsFunction1}
         \end{equation}
      \end{shaded}
    \end{subequations}

%%% SUBSECTION
   \subsection{Properties obtained at Constant Temperature}

      For processes at constant temperature, \ie, $dT=0$,
          \begin{displaymath}
              d\left(G^{R}/RT\right) = \frc{V^{R}}{RT}dP,
          \end{displaymath}
          integrating from $0$ (ideal gas state) to $P$,
          \begin{eqnarray}
              \left(\frc{G^{R}}{RT}\right)_{P} &-&  \underbrace{\left(\frc{G^{R}}{RT}\right)_{P=0}}_{\Phi \text{ (integration constant)}} =  \int\limits_{0}^{P}\overbrace{\frc{V^{R}}{RT}}^{\text{replacing by Eqn.~\ref{Mod03_ResidualProperties_V}}}dP, \nonumber \\
              \left(\frc{G^{R}}{RT}\right)_{P} &=& \Phi + \int\limits_{0}^{P}\left(Z-1\right)\frc{dP}{P},\label{Mod03_ResidualProperties_GeneratingGibbsFunction2}
          \end{eqnarray}
          Differentiating this expression \wrt $T$ $\left(\text{\ie }\Partial[]{T}{}\right)$ 
          \begin{displaymath}
              \Partial[\left(G^{R}/RT\right)]{T}{} = \int\limits_{0}^{P} \left.\Partial[Z]{T}{}\right|_{P}\frc{dP}{P},
          \end{displaymath}
          here it is important to note that in $\left.\Partial[Z]{T}{}\right|_{P}$, the index $P$ indicates that the differential in brackets is only valid for $P\ne0$, as for $P=0$ the gas is assumed ideal and $Z$ is constant equal to $1$. Now using Eqn.~\ref{Mod03_GibbsGeneratingFunctionPConst},
          \begin{subequations}
          \begin{shaded}
             \begin{equation}
                \frc{H^{R}}{RT} = - T\int\limits_{0}^{P}\left.\Partial[Z]{T}{}\right|_{P}\frc{dP}{P}.\label{Mod03_ResidualProperties_GeneratingEnthalpyFunction2}
             \end{equation}
          \end{shaded}
          With this expression, one can readily obtain the residual enthalpy of a real gas assuming that $\frac{\partial Z}{\partial T}$ at a prescribed pressure can be obtained by experiments or by differentiating the cubic form in $Z$ of any EOS (Eqns.~\ref{Mod02_GeneralEOS}-~\ref{Mod02_Zvap}). Now, using the fundamental relation $G = H - TS$,
          \begin{eqnarray}
                 G^{\text{ig}} &=& H^{\text{ig}} - TS^{\text{ig}} \nonumber \\
                 G^{R} &=& H^{R} - TS^{R}\;\;\;\;\;\blue{\times\frc{1}{RT}} \;\;\;\Longrightarrow \;\;\; \frc{S^{R}}{R} = \frc{H^{R}}{RT} - \frc{G^{R}}{RT} \nonumber \\
                 \frc{S^{R}}{R} &=& \underbrace{-T\int\limits_{0}^{P}\left.\Partial[Z]{T}{}\right|_{P}\frc{dP}{P}}_{\text{Eqn.~\ref{Mod03_ResidualProperties_GeneratingEnthalpyFunction2}}} \overbrace{- \Phi - \int\limits_{0}^{P}\left(Z-1\right)\frc{dP}{P}}^{\text{Eqn.~\ref{Mod03_ResidualProperties_GeneratingGibbsFunction2}}}\;\;\;\blue{(\text{at constant}\;\; T)}.\nonumber %\label{Mod03_ResidualProperties_GeneratingEntropyFunction1}
          \end{eqnarray}
          For practical applications we often require $\Delta S$,
          \begin{displaymath}
             \Delta S = S_{2} - S_{1} = \left(S_{2}^{\text{ig}}-S_{1}^{\text{ig}}\right) + \underbrace{\overbrace{\left(S_{2}^{R}-S_{1}^{R}\right)}^{\Phi\text{ cancels out !!}}}_{\text{Thus we can arbitrarily set }\Phi=0}
          \end{displaymath}
          Thus
          \begin{shaded}
              \begin{eqnarray}
                  \frc{S^{R}}{R} &=& -T\int\limits_{0}^{P}\left.\Partial[Z]{T}{}\right|_{P}\frc{dP}{P} - \int\limits_{0}^{P}\left(Z-1\right)\frc{dP}{P},\label{Mod03_ResidualProperties_GeneratingEntropyFunction2} \\
                  \frc{G^{R}}{RT} &=&  \int\limits_{0}^{P}\left(Z-1\right)\frc{dP}{P}.\label{Mod03_ResidualProperties_GeneratingGibbsFunction3} 
              \end{eqnarray}
          \end{shaded}
       \end{subequations}
          
%%% SUBSECTION
   \subsection{Residual Properties in the Zero-Pressure Limit}
      
       By definition, at pressures near to zero, \ie $P\rightarrow 0$, the gas behaves as an ideal gas and $Z\rightarrow 1$. However, \underline{not all} residual properties become zero near the near-zero pressure condition. Molar volume for example,
       \begin{displaymath}
           \lim\limits_{P\rightarrow 0}V^{R} = \lim\limits_{P\rightarrow 0}V - \lim\limits_{P\rightarrow 0}V^{\text{ig}},
       \end{displaymath}
both terms in the r.h.s. tends to $+\infty$ and the difference is assumed undetermined. For other properties:
       \begin{displaymath}
          \begin{cases}
             \lim\limits_{P\rightarrow 0}H^{R} = 0, & \forall T \text{ (for all $T$)}, \\
             \lim\limits_{P\rightarrow 0}\left(G^{R}/RT\right) = \infty-\infty,& \text{undetermined but not necessarily zero}.  
          \end{cases}
       \end{displaymath}


%%% SUBSECTION
\section{Two-Phase Systems}\label{Section:03:Two_Phase}

%%% SUBSECTION
   \subsection{Clapeyron Relations}\label{Section:03:ClapeyronRelations}

\begin{subequations}
When a pure component is in phase equilibrium, {\it all coexisting phases have the same temperature and pressure}. Therefore, the following criterion is applied,
      \begin{displaymath}
          dG = VdP - SdT = 0 \;\;\;\;\text{ at constant } T \text{ and } P.
      \end{displaymath}
This means that changes in Gibbs free energy in all coexisting phases are the same, \ie,
      \begin{displaymath}
          dG^{\alpha} = dG^{\beta} = dG^{\gamma} = \cdots,
      \end{displaymath}
where $\alpha$, $\beta$, $\gamma$, $\cdots$, are phases in thermodynamic equilibrium. Therefore we can write Gibbs change for two given phases as,
      \begin{displaymath}
          V^{\alpha}dP^{\text{sat}} - S^{\alpha}dT = V^{\beta}dP^{\text{sat}} - S^{\beta}dT.
      \end{displaymath}
$P^{\text{sat}}$ is the saturated pressure, \ie pressure in which phase change occurs. We can manipulate this expression,
      \begin{displaymath}
          \frc{dP^{\text{sat}}}{dT} = \frc{S^{\beta}-S^{\alpha}}{V^{\beta}-V^{\alpha}} = \frc{\Delta S^{\alpha\beta}}{\Delta V^{\alpha\beta}}.
      \end{displaymath}
We can define the {\it latent heat of phase transition} (\eg vaporisation, solidification etc), $\Delta H^{\alpha\beta}$ from Eqn.~\ref{Chapter:ThermodynamicPropertiesPureFluids:Eqn:FundamentalRelation02} integrating with constant pressure,
      \begin{displaymath}
          dH = TdS + VdP \;\;\Rightarrow \;\; \int\limits_{H^{\alpha}}^{H^{\beta}} dH = \int\limits_{S^{\alpha}}^{S^{\beta}} TdS \;\;\Rightarrow \;\;  H^{\beta}-H^{\alpha} = T\left(S^{\beta}-S^{\alpha}\right)  \;\;\Rightarrow \;\; \Delta H^{\alpha\beta} = T\Delta S^{\alpha\beta},
      \end{displaymath}
therefore,
      \begin{shaded}
          \begin{equation}
              \frc{dP^{\text{sat}}}{dT} = \frc{\Delta H^{\alpha\beta}}{T\Delta V^{\alpha\beta}},\label{Mod03_ClapeyronEqn} 
          \end{equation}
          this expression is known as {\it Clapeyron Equation} and it expresses the relation between a change in temperature and a change in pressure under the conditions of equilibrium between two phases. 
      \end{shaded}

We can determine the enthalpy of vaporisation (\ie latent heat of vaporisation), $H^{\alpha\beta}=H^{\text{fg}}$, at a given $T$ by simply measuring the slope of the saturation curve on a $PT$ diagram (Fig.~\ref{Mod03Fig01}) and the specific volume of saturated liquid and saturated vapour at the given $T$. The {\it Clapeyron Equation} can be simplified for liquid-vapour (and for solid-vapour) phase changes, at low pressures
      \begin{displaymath}
         V^{\text{g}} >>>> V^{\text{f}} \;\;\Longrightarrow \;\; V^{\text{fg}} = V^{\text{g}},
      \end{displaymath}
if we treat the vapour as an ideal gas, $V^{\text{g}} = \frc{RT}{P}$, and replacing in Eqn.~\ref{Mod03_ClapeyronEqn} 
      \begin{displaymath}
          \frc{dP^{\text{sat}}}{dT} = \frc{P^{\text{sat}}\Delta H^{\text{fg}}}{RT^{2}} \;\;\;\Longrightarrow \;\;\; \left(\frc{dP}{P}\right)_{\text{sat}} = \frc{\Delta H^{\text{fg}}}{R}\left(\frc{dT}{T^{2}}\right)_{\text{sat}}.
      \end{displaymath}
For infinitesimal intervals of $T$, \underline{$\Delta H^{\text{fg}}$ may be considered constant}, thus
%
           \begin{figure}[h]
               \begin{center}
                   \includegraphics[width=.5\columnwidth,clip]{./../Pics/PT_Diagram2}
               \end{center} 
               \caption{ $PT$ diagrams for a pure substance. Graphical representation of the Clapeyron relation.}\label{Mod03Fig01}
           \end{figure}
      \begin{shaded}
          \begin{equation}
             \frc{d\left(\ln{P^{\text{sat}}}\right)}{dT} = \frc{\Delta H^{\text{fg}}}{RT^{2}} \;\;\Longrightarrow \;\;\; \ln{\left(\frc{P_{2}}{P_{1}}\right)_{\text{sat}}} = \frc{\Delta H^{\text{fg}}}{R}\left(\frc{1}{T_{1}}-\frc{1}{T_{2}}\right)_{\text{sat}}.\label{Mod03_ClausiusClapeyronEqn} 
          \end{equation} 
This expression is known as {\it Clausius-Clapeyron equation} and is a good approximation when describing temperature and pressure dependence at boiling/condensation and at sublimation/gas deposition.
      \end{shaded}
For the dependence of the saturated vapour pressure on $T$, a number of empirical relations have been developed. The simplest expression is,
    \begin{displaymath}
       \ln{P^{\text{sat}}} = A - \frc{B}{T},%\label{Mod03_AntoineSimplest}
    \end{displaymath}
where $A$ and $B$ are constants obtained from experiments. A more `popular' relation is,
    \begin{shaded}
       \begin{equation}
          \ln{P^{\text{sat}}} = A - \frc{B}{T+C},\label{Mod03_Antoine}
       \end{equation}
       this relation is known as {\it Antoine Equation}, where $A$, $B$ and $C$ are constants obtained experimentally.
    \end{shaded}
This two relations, although still widely used by the fluids community are plagued with strong inaccuracy. Due to better accuracy, high-order polynomial relations have become commonly used in flow and process simulators,
    \begin{displaymath}
       \ln{P^{\text{sat}}} = \frc{A\tau + B\tau^{1.5} + C\tau^{3} + D\tau^{6}}{1-\tau}\;\;\;\;\text{ with }\;\;\tau = 1 - T_{r}.
    \end{displaymath}
\end{subequations}


%%% SUBSECTION
   \subsection{Vapour-Liquid Equilibrium Systems}

Several processes of engineering relevance occur with fluids in phase equilibria -- either saturated vapour (\ie vapour saturated with liquid droplets) and saturated liquid (\ie liquid saturated with bubbles of vapour). In most cases, it is important to know the actual quantities of both phases in thermodynamic equilibrium, \ie the amount of vapour and liquid present in a constrained system at prescribed temperature and pressure conditions. Let assume that a closed system contains $n$ moles of a chemical species split into $\mathcal{P}$ phases,
    \begin{displaymath}
      n = \sum\limits_{j=1}^{\mathcal{P}} \mfr[n]{}{j} = \mfr[n]{}{1} + \mfr[n]{}{2} + \cdots + \mfr[n]{}{\mathcal{P}}.
    \end{displaymath}
The mass balance acroos all $\mathcal{P}$ phases can be represented as
    \begin{displaymath}
       nV = \mfr[n]{}{1}\mfr[V]{}{1} + \mfr[n]{}{2}\mfr[V]{}{2} + \cdots + \mfr[n]{}{\mathcal{P}}\mfr[V]{}{\mathcal{P}}  = \sum\limits_{j=1}^{\mathcal{P}}\left(nV\right)^{\left(j\right)},
    \end{displaymath}
where $V$ is the molar volume. Dividing by $n$
    \begin{displaymath}
       V = \frc{\mfr[n]{}{1}}{n}\mfr[V]{}{1} + \frc{\mfr[n]{}{2}}{n}\mfr[V]{}{2} + \cdots + \frc{\mfr[n]{}{\mathcal{P}}}{n}\mfr[V]{}{\mathcal{P}}.
    \end{displaymath}
Defining molar (or mole) fraction, $\mfr[x]{}{j}=\frc{\mfr[n]{}{j}}{n}$, where
    \begin{eqnarray}
         && \sum\limits_{j=1}^{\mathcal{P}}\mfr[x]{}{j} = 1,  \nonumber \\
         && V = \mfr[x]{}{1}\mfr[V]{}{1} + \mfr[x]{}{2}\mfr[V]{}{2} + \cdots + \mfr[x]{}{\mathcal{P}}\mfr[V]{}{\mathcal{P}}.  \nonumber
    \end{eqnarray}
For vapour-liquid systems,
    \begin{eqnarray}
         && \mfr[x]{}{L} + \mfr[x]{}{V} = 1,  \nonumber \\
         && V = \mfr[x]{}{L}\mfr[V]{}{L} + \mfr[x]{}{V}\mfr[V]{}{V}. \nonumber
    \end{eqnarray}
For a generic thermodynamic potential $M$ (= $V$, $U$, $H$, $S$ etc),
    \begin{shaded}
       \begin{subequations}
           \begin{equation}
              M = \left(1-\mfr[x]{}{V}\right)\mfr[M]{}{L} + \mfr[x]{}{V}\mfr[M]{}{V}
           \end{equation}
           \begin{equation}
              M = \mfr[M]{}{L} + \mfr[x]{}{V}\Delta\mfr[M]{}{LV}\label{Mod03_QualityVapour}
           \end{equation}
       \end{subequations}
    \end{shaded}
$\mfr[x]{}{V}$ is called \underline{\it vapour quality}. Thermodynamic potentials of pure substances are graphically represented by $Ph$ (pressure $\times$ specific enthalpy) and $Ts$ (temperature $\times$ specific entropy) diagrams, Fig.~\ref{Mod03Fig02}, where information on $P$, $T$, $s$, $h$, $x$ and $v$ (specific volume) can be readily extracted.
%
           \begin{figure}[h]
              \vbox{
                    \hbox{\includegraphics[width=.5\columnwidth,clip]{./Figs/Mod3PHDiagram}
                          \includegraphics[width=.5\columnwidth,clip]{./Figs/Mod3TSDiagram}}
                    \vspace{-.1cm}
                    \hbox{\hspace{4cm}(a)\hspace{8cm}(b)}}
              \caption{ (a) $Ph$ and (b) $Ts$ diagrams for a pure substance.}\label{Mod03Fig02}
           \end{figure}
%
In the $Ph$ diagram, Fig~\ref{Mod03Fig02}a, isotherms (i.e., lines representing constant temperature) and pressure conditions determine the phase of the fluid. At the left hand-side of the {\it dome} all fluid is at liquid state, whereas at the right hand-side of the {\it dome}, all fluid is at vapour state. The region within the {\it dome} is a two-phase region, where liquid and vapour coexist in thermodynamic equilibrium. As the fluid conditions `move' from the {\it saturated liquid line} to the {\it saturated vapour line} through the {\it isotherm}, the fluid is continuously vaporised `till there is no droplets of liquid fluid. The total amount of heat given to the system -- $\Delta H^{\text{fg}}$, is the latent heat of vaporisation. In a similar way, the $Ts$ diagram, Fig~\ref{Mod03Fig02}b, shows similar features over different {\it isobars}. In addition, the {\it quality} of the vapour can also be graphically represented. In a reversible expansion from $P_{1}$ to $P_{2}$ $\left(P_{1}>P_{2}\right)$, entropy change is null, $\Delta s=0$, and is represented by a vertical line, however during irreversible expansion, $\Delta s >0$, represented by an inclined line.  

 $Ph$ and $Ts$ diagrams for common substances are no longer used by industry but it helps to qualitatively understand phase (and associated thermodynamic potentials) behaviour of pure substances. Quantitative information can be obtained from either saturated and superheated (Fig.~\ref{Mod03Fig03}) fluid tables or dedicated software, \eg
\begin{itemize}
   \item \href{http://www.weatherford.com/doc/wft183650}{PVTflex$^{TM}$};
   \item \href{http://www.kbcat.com/infochem-software/flow-assurance-software-multiflash/pvt-simulation}{Multiflash$^{TM}$};
   \item \href{https://www.honeywellprocess.com/en-US/explore/products/advanced-applications/unisim/Pages/default.aspx}{UniSim – Software for Process Design and Simulation};
   \item \href{http://webbook.nist.gov/chemistry/fluid/}{NIST Website}
   \item etc.
\end{itemize}
In general, table of {\it saturated fluid properties}, Fig.~\ref{Mod03Fig03}a, contains information of the fluid within the {\it dome}, whereas the table of {\it superheated fluid properties} refer to the region outside (rhs) the {\it dome}. 
%
   \begin{figure}[h]
      \vbox{
         \hbox{\includegraphics[width=.5\columnwidth,clip]{./Figs/WaterSatTable}
               \includegraphics[width=.5\columnwidth,clip]{./Figs/Water_SuperheatedTable}}
         \vspace{-1.5cm}
         \hbox{\hspace{4cm}(a)\hspace{7cm}(b)}
      }
      \caption{ Table of properties of (a) saturated water-steam and (b) superheated vapour (Extracted from Moran $\&$ Saphiro, see Appendix B).}\label{Mod03Fig03}  
   \end{figure}
%
    

%%% SUBSECTION
   \subsection{Industrial Applications: Power System}
Regardless the energy source (fossil fuel, nuclear or geothermal), power plants are good applications for VLE systems as fluids are continuously vaporised and condensed by addition and extraction of heat and volume expansion. {\it Rankine thermal cycles} are system configurations for generating power and consist of four processes (Fig.~\ref{Mod03Fig04}) with associated energy balances:
     \begin{itemize}
      \item \textcolor{red}{Process 1-2}: reversible adiabatic (i.e., \blue{isentropic}) expansion in the turbine (or steam engine),
            \begin{displaymath}
               \left(h_{2} + \dot{W}_{T}\right)-h_{1} = 0 \Rightarrow \dot{W}_{T} = h_{1}-h_{2}
            \end{displaymath}
      \item \textcolor{red}{Process 2-3}: constant-pressure heat transfer (to the environment) in the condenser,
            \begin{displaymath}
               \left(h_{3} + \dot{Q}_{C}\right)-h_{2} = 0 \Rightarrow \dot{Q}_{C} = h_{2}-h_{3}
            \end{displaymath}
      \item \textcolor{red}{Process 3-4}: reversible adiabatic (i.e., \blue{isentropic}) pumping process in the feed pump,
            \begin{displaymath}
               h_{4} - \left(h_{3} + \dot{W}_{P}\right) = 0 \Rightarrow \dot{W}_{P} = h_{4}-h_{3}
            \end{displaymath}
      \item \textcolor{red}{Process 4-1}: constant-pressure heat transfer (to the fluid) in the boiler,
            \begin{displaymath}
               h_{1} - \left(h_{4} + \dot{Q}_{B}\right) = 0 \Rightarrow \dot{Q}_{B} = h_{1}-h_{4}
            \end{displaymath} 
     \end{itemize}
     The efficiency $\left(\eta\right)$ of the Rankine cycle is given by
           \begin{displaymath}
               \eta_{\text{Rankine}} = \frc{\sum W_{i}}{Q_{B}} = \frc{\left|W_{\text{net}}\right|}{Q_{B}} = \frc{\left|\left(h_{1}-h_{2}\right)+\left(h_{4}-h_{3}\right)\right|}{h_{1}-h_{4}}.
           \end{displaymath}
     Due to engineering constraints, fluids entering and leaving the pump \underline{must be} at liquid phase, thus $h_{4}=h_{f4}$ and $h_{3}=h_{f3}$, \ie the fluid has the enthalpy of the liquid phase (from the saturated fluid table) at the prescribed temperature and pressure conditions. {\it Pumps} are able to induce the transport of liquid fluids that are often assumed incompressible, therefore
          \begin{displaymath}
                Tds = dh - v dP
          \end{displaymath}
as the compression occurs isentropically, \ie $ds=0$,
          \begin{displaymath}
                dh = v dP \Rightarrow h_{f4} = h_{f3} + v_{3}\left(P_{4}-P_{3}\right).
          \end{displaymath}
However, as $\left(h_{f4}-h_{f3}\right) <<<<< \left(h_{1}-h_{2}\right)$, the efficiency can be considered as
           \begin{displaymath}
               \eta_{\text{Rankine}} = \frc{\left|\left(h_{1}-h_{2}\right)\right|}{h_{1}-h_{f4}}.
           \end{displaymath}     
%
   \begin{figure}[h]
      \begin{center}
         \includegraphics[width=\columnwidth,clip]{./Figs/Mod3PowerSystemDiagram}
      \end{center}
      \caption{ (a) Diagram of Rankine thermal cycle for power generation and associated $Ts$ diagram.}\label{Mod03Fig04}
   \end{figure}


\clearpage

%%% SECTION
\section{Examples}

\begin{enumerate}[1)]
%%%
%%% EXAMPLE 
%%%
\item\label{Mod03Ex01} A block of copper of 1 kg undertakes a reversible compression from 0.1 MPa to 100 MPa at constant temperature of 15$^{\circ}$C. Calculate:
    \begin{enumerate}[a)]
       \item Work done on the copper block during the process;
       \item Change in entropy {\it per} kg of copper;
       \item Heat transfer and;
       \item Change of internal energy {\it per} kg.
    \end{enumerate}
    Given, 
    \begin{itemize}
       \item Volume expansivity coefficient: $\beta = 5\times 10^{-5}$ K$^{-1}$;
       \item Isothermal compressibility coefficient: $\kappa = 8.6\times 10^{-12}$ m$^{2}$.N$^{-1}$;
       \item specific volume: $v=1.14\times 10^{-4}$ m$^{3}$.kg$^{-1}$.
    \end{itemize} 

% SOLUTION
    \noindent{\bf Solution:} 
       \begin{enumerate}[a)]
%
            \item The work done during the compression,
                \begin{displaymath}
                   w = -\int P dv,
                \end{displaymath}
                where $v$ is the specific volume. $\kappa$ was defined in Module~\ref{Section:02} as,
                \begin{displaymath}
                   \kappa = \frc{1}{v}\left(\frc{\partial v}{\partial P}\right)_{T}\;\Longrightarrow \; v\kappa dP = - dv \;\;\text{ (with constant T)}
                \end{displaymath}
                For isothermal processes
                \begin{displaymath}
                   w = -\int P dv = - \int P\left(-v\kappa dP\right) = \frc{v}{2}\kappa\left(P_{2}^{2}-P_{1}^{2}\right) = 4.90 \frc{\text{J}}{\text{kg}}
                \end{displaymath}
%
            \item $ds$ = ? (specific entropy).\\
                  From the Maxwell relations -- Eqn.~\ref{Chapter:ThermodynamicPropertiesPureFluids:Eqn:MaxwellRelation4}, 
                \begin{displaymath}
                   -\left(\frc{\partial s}{\partial P}\right)_{T} = \left(\frc{\partial v}{\partial T}\right)_{P},
                \end{displaymath}
                and from the definition of $\beta$,
                \begin{eqnarray}
                    && \beta = \frc{1}{v}\left(\frc{\partial v}{\partial T}\right)_{P} \;\;\Longrightarrow\;\; -\left(\frc{\partial s}{\partial P}\right)_{T} = \beta v \nonumber \\
                    && ds = -\beta v dP \;\;\Longrightarrow ds = s_{2}-s_{1} = -\beta v \left(P_{2}-P_{1}\right) = -0.5694 \frc{\text{J}}{\text{kg.K}} \nonumber
                \end{eqnarray}
%
            \item The heat transferred in such reversible isothermal process is
                \begin{displaymath}
                   dq = Tds \;\;\Longrightarrow q = T\left(s_{2}-s_{1}\right) = -164.07 \frc{\text{J}}{\text{kg}}.
                \end{displaymath}
%
            \item The specific internal energy,
                \begin{eqnarray}
                   &&  du = q + w \nonumber \\
                   && \left(u_{2}-u_{1}\right) = \underbrace{-164.07}_{\text{heat removed from the system}} + \overbrace{4.90}^{\text{work given to the system}} = -159.17 \frc{\text{J}}{\text{kg}}. \nonumber
                \end{eqnarray}
 
%
       \end{enumerate} 

\clearpage
%%%
%%% EXAMPLE 
%%%
\item\label{Mod03Ex02} Demonstrate that the derivative of molar volume \wrt temperature at constant pressure is
     \begin{displaymath}
         \Partial[V]{T}{P} = -\frc{\Partial[P]{T}{V}}{\Partial[P]{V}{T}},
     \end{displaymath}
     and obtain an expression for $\Partial[V]{T}{P}$ for the van der Waals EOS. {\bf Hint:} You should start the proof from the total differential of a continuous function $f(a,b)$,
     \begin{displaymath}
         df = \Partial[f]{a}{b}da + \Partial[f]{b}{a}db.
     \end{displaymath}

% SOLUTION
    \noindent{\bf Solution:} The total differential of a generic continuous function $f(a,b)$ is
     \begin{displaymath}
         df = \Partial[f]{a}{b}da + \Partial[f]{b}{a}db.
     \end{displaymath}
     where (from the given thermodynamic function) $f=P$, $a=T$ and $b=V$, \ie
     \begin{displaymath}
         dP = \Partial[P]{T}{V}dT + \Partial[P]{V}{T}dV.
     \end{displaymath}
     However we want a differential expression in which $P$ is constant, therefore $dP = 0$,
     \begin{eqnarray}
         0 &=& \Partial[P]{T}{V}dT + \Partial[P]{V}{T}dV \;\;\;\text{ at } P \text{ constant},\nonumber \\
         \Partial[V]{T}{P} &=& -\frc{\Partial[P]{T}{V}}{\Partial[P]{V}{T}}.\nonumber
     \end{eqnarray}

     \medskip\noindent
     The vdW-EOS is,
     \begin{displaymath}
          P = \frc{RT}{V-b} - \frc{a}{V^{2}},
     \end{displaymath}
     where $V$ is the molar volume and $a$ and $b$ are constants that {\it depends only on critical properties}, $P_{c}$ and $T_{c}$. Due to the non-linearity of this EOS, obtaining $\Partial[V]{T}{P}$ from a direct differentiation would be difficult. However, we can use the expression that we just derived,
     \begin{displaymath}
         \Partial[V]{T}{P} = -\frc{\Partial[P]{T}{V}}{\Partial[P]{V}{T}} = -\frc{\frc{R}{V-b}}{-\frc{RT}{\left(V-b\right)^{2}}+\frc{2a}{V^{3}}}
     \end{displaymath} 

\clearpage
%%%
%%% EXAMPLE 
%%%
\item\label{Mod03Ex03} The Antoine equation constants for toluene are $A=14.01415$, $B=3106.46$ K and $C=-53.15$ K (for pressure given in kPa). At 1.01325$\times$10$^{5}$ Pa, calculate the boiling temperature and the enthalpy of vaporisation at this temperature.

% SOLUTION
    \noindent{\bf Solution:} Boiling temperature can be calculated from the Antoine equation,
       \begin{displaymath}
          \ln{P^{\text{sat}}} = A - \frc{B}{T+C} \;\;\;\Rightarrow \;\;\; T = \frc{B}{A-\ln{P^{\text{sat}}}} - C = \red{383.77 K}
       \end{displaymath}
The enthalpy of vaporisation, $\Delta H^{\text{fg}}$, can be obtained from the Clausius-Clapeyron equation,
         \begin{eqnarray}
            \frc{d}{dT} \left(\ln{P^{\text{sat}}}\right) &=& \frc{\Delta H^{\text{fg}}}{RT^{2}} \nonumber \\
             \frc{B}{\left(T+C\right)^{2}} &=&  \frc{\Delta H^{\text{fg}}}{RT^{2}} \;\;\Longrightarrow \Delta H^{\text{fg}} = 34.7984 \text{kJ.mol}^{-1}. \nonumber
         \end{eqnarray}
 
\clearpage

%%%
%%% EXAMPLE 
%%%  
\item\label{Mod03Ex04} Derive an expression for enthalpy change of a gas during an isothermal process assuming using the following EOS: $P\left(V-b\right)=RT$

% SOLUTION
    \noindent{\bf Solution:} We have seen that enthalpy change is given by Eqn.~\ref{Mod03_DerivedEnthalpyRelation1},
    \begin{displaymath}
       dH = C_{p}dT + \left[V - T\Partial[V]{T}{P}\right]dP.
    \end{displaymath}
    We can rearrange the given EOS and obtain $\Partial[V]{T}{P}$,
    \begin{eqnarray}
       && P\left(V-b\right)=RT \;\;\;\rightarrow\;\;\; V = \frc{RT}{P} + b \;\;\;\rightarrow\;\;\; \Partial[V]{T}{P} = \frc{R}{P}\;\;\text{ thus, } \nonumber \\
       && dH = C_{p}dT + \left(V - \frc{RT}{P}\right)dP = \blue{C_{p}dT + bdP}. \nonumber 
    \end{eqnarray}
    
\clearpage
    
\clearpage
 
%%%
%%% EXAMPLE 
%%%  
\item\label{Mod03Ex06} Steam (dry and saturated) is supplied by the boiler at 15 bar and the condenser inlet pressure is 0.4 bar. Calculate the Rankine efficiency of the cycle. Neglect the pump work, assume the enthalpy of fluid leaving the pump is 317.58 kJ.kg$^{-1}$

% SOLUTION  
    \noindent{\bf Solution:} At 15 bar, dry and saturated $\left(\ie x_{1}=1\right)$ steam has the following properties (from saturated table)\footnote{Using the same numbering as in Fig.~\ref{Mod03Fig04}.},
          \begin{eqnarray}
             T_{1} &=& T_{\text{sat}} = 198.3^{\circ}\text{C},\nonumber \\
             h_{1} &=& h_{\text{g}} = 2792.2\; \text{kJ.kg}^{-1} \nonumber \\
             s_{1} &=& s_{\text{g}} = 6.4448\; \text{kJ.(kg.K)}^{-1} \nonumber
          \end{eqnarray} 
    In the condenser, $P_{2}=0.4$ bar,
          \begin{eqnarray}
              T_{2} &=& T_{\text{sat}} = 75.87^{\circ}\text{C}, \nonumber \\
              h_{\text{g}2} &=& 2636.8\;\text{kJ.kg}^{-1},\;\;\; h_{\text{f}2} = 317.58\;\text{kJ.kg}^{-1},  \nonumber \\
              s_{\text{g}2} &=& 7.6700 \;\text{kJ.(kg.K)}^{-1},\;\;\; s_{\text{f}2} = 1.0259\;\text{kJ.(kg.K)}^{-1}. \nonumber  
          \end{eqnarray}
$h_{2}$ and $s_{2}$ depend on the knowledge of how vaporised the water is, in other words, we need to determine the quality of the steam, $x_{1}$ through Eqn.~\ref{Mod03_QualityVapour},
      \begin{eqnarray}
          h_{2} &=& h_{\text{f}2} + x_{2}\left(h_{\text{g}2} - h_{\text{f}2}\right), \nonumber \\
          s_{2} &=& s_{\text{f}2} + x_{2}\left(s_{\text{g}2} - s_{\text{f}2}\right). \nonumber
      \end{eqnarray}
As we know that water is expanded isentropically in the turbine, \ie $s_{1}=s_{2}$,
      \begin{displaymath}
         s_{2} = s_{\text{f}2} + x_{2}\left(s_{\text{g}2} - s_{\text{f}2}\right) = s_{1} = 6.4448 \;\;\;\Rightarrow \;\;\; x_{2} = 0.8156 \;\;(81.56\% \text{ of vapour}
      \end{displaymath}
Thus replacing in 
      \begin{displaymath}
          h_{2} = h_{\text{f}2} + x_{2}\left(h_{\text{g}2} - h_{\text{f}2}\right) = 2209.14\text{ kJ.kg}^{-1}.
      \end{displaymath}
The Rankine efficiency is given by
      \begin{displaymath}
           \eta_{\text{Rankine}} = \frc{\text{Adiabatic or Isentropic Heat Drop}}{\text{Heat Supplied}} = \frc{\left|h_{1}-h_{2}\right|}{h_{1}-h_{\text{f}4}} = 0.2356\;\;\;\rightarrow \;\;\; 23.56\%
      \end{displaymath}
    


\end{enumerate}


 % Thermodynamic Properties of Pure Fluids
%  \chapter{Vapour-Liquid Equilibrium of Mixtures}\label{Chapter:VLE}


   \begin{LearningObjectivesBlock}{Learning Objectives}
      Upon completion of this chapter, you will be able to
        \begin{enumerate}
           \item 
        \end{enumerate}
\medskip
     Recommended reading: Chapters 6 of \citet{SmithVanNess_Book}, 3 and 5 of \citet{Lue_Book}, 12 of \citet{Balmer_Book}, 12 of \citet{Borgnakke_Book} or 5 of \citet{Atkins_Book}.
   \end{LearningObjectivesBlock}


%%%%%%%%%%%%%%%%%%%%%%%%%%%%%%%%%%%%%%%%%%%%%%%%%%%%%%%%%%%%%%%%%
\begin{comment}
   \begin{LearningObjectivesBlock}{Learning Objectives}
      Upon completion of this chapter, you will be able to
        \begin{enumerate}
           \item {\bf Knowledge:} Define, Name, Select, State 
           \item {\bf Comprehension:} Describe, Identify, Discuss
           \item {\bf Application:} Apply, Demonstrate, Employ, Sketch
           \item {\bf Analysis:} Analyse, Compare, Calculate, Solve
           \item {\bf Synthesis:} Determine, Formulate
           \item {\bf Evaluation:} Assess, Check, Estimate, Compare, Measure, Monitor
        \end{enumerate}
\end{comment}
%%%%%%%%%%%%%%%%%%%%%%%%%%%%%%%%%%%%%%%%%%%%%%%%%%%%%%%%%%%%%%%%%

%%%% ETOC
\localtableofcontents
   

%%% SECTION
\section{Introduction}\label{Chapter:VLE:Section:Introduction}
Up to this point, we have only considered thermodynamic and volumetric properties of pure components at prescribed pressure and temperature conditions. However, most engineering applications deals with mixtures of components that may be present in single or multiple phases, \eg oil in reservoirs, naphtha distillation, steel processing etc. This chapter focuses on understanding PVT behaviour of mixtures and the conditions for vapour-liquid equilibrium (VLE).



%%% SECTION
\section{A Few Important Definitions}
In the previous chapters, we mostly focused on thermodynamic properties of systems containing pure chemical species. This chapter (and the remaining of this notes) will study systems with arbitrary number of components, and the quantification of each component at each phase (Gibbs phase rule, Section~\ref{Chapter:VolumetricPropertiesPureSubstances:Section:GibbsPhaseRule}, is fundamental to determine the number of degrees of freedom in two-phase system with arbitrary number of chemical species).\index{Phase rule}

%%% SUBSECTION
\subsection{Representing Compositions}\label{Chapter:VLE:Section:Compositions}
 For a mixture containing $\mathcal{C}$ chemical species, each of them with an arbitrary number of moles, $n_{i},\;\;\forall i\in\left\{1,2,\cdots,\mathcal{C}\right\}$,
\begin{subequations}
   \begin{enumerate}[a)]\index{Molar fraction}
       \item Molar (or mole) fraction $\left(x_{i}\right)$:
            \begin{eqnarray}
                x_{i} = \frc{n_{i}}{n},\label{Chapter:VLE:Eqn:MolarFraction}
            \end{eqnarray} 
            where $n=\summation[n_{i}]{i=1}{\mathcal{C}}$ is the total number of moles in the system. As the molar fraction is a normalised quantity, the following constraint is imposed:
            \begin{equation}
                  \summation[x_{i}]{i=1}{\mathcal{C}} = 1,\label{Chapter:VLE:Eqn:MolarFractionConstraint}
            \end{equation}
       \item (Average) Molar mass of mixtures:\index{Molar mass}\index{Molar weight|see {Molar mass}}
            \begin{equation}
                \overline{MW} = \summation[\left(x_{i} \cdot MW_{i}\right)]{i=1}{\mathcal{C}}\label{Chapter:VLE:Eqn:MolarMass}
            \end{equation}
         where $MW_{i}$ is the molecular mass (or molar weight) of species $i$.
   \end{enumerate}
\end{subequations}

%%% SUBSECTION
\subsection{Partial Molar Properties}\label{Chapter:VLE:PartialMolarProperties}\index{Partial molar properties}
Many thermodynamic properties of {\it ideal solutions} do not change on mixing, for example, the volume of a mixture (assumed ideal) is equal to the sum of the volume of the original unmixed solutions. However, in real systems the volume and other properties, are not additive, \ie the volume of a mixture is not equal to the sum of the volumes of the individual pure components. {\it Partial molar properties} describe the behaviour of homogeneous multi-component systems. Thus, the concept of partial molar properties are crucial tool to assess individual contribution to thermodynamic properties of mixtures. In Section~\ref{ChapterChapter:SolutionThermodynamics:Section:GibbsDuhem}, an expression will be developed to assess thermodynamic properties of solutions based on this concept. 

 Let's consider a homogeneous (\ie single phase, $\mathcal{P}=$ 1) and open system with $\mathcal{C}$ chemical species that undertakes a change in composition. Thus, the total value of any extensive property $M^{\text{t}}\;\left(M\equiv V, U, H, S, G, A\right)$ is not only a function of pressure and temperature, but it also depends on the number of moles of each species in the system. therefore
  \begin{subequations}
     \begin{equation}
       M^{\text{t}} = nM = M\left(T,P,n_{1},n_{2},\cdots, n_{\mathcal{C}}\right).\label{Chapter:VLE:PartialMolarProperties:Eqn:PartialProperties1}
     \end{equation}
     The total derivative of this property, $M^{\text{t}}$, is,
     \begin{displaymath}  
        d(nM) = \Partial[(nM)]{P}{T,n}dP + \Partial[(nM)]{T}{P,n}dT + \Partial[(nM)]{n_{i}}{T,P,n_{j\ne i}}dn_{i},
     \end{displaymath}
     where the subscript $n$ in the partial derivatives indicates that the number of moles is kept constant, whereas $n_{j\ne i}$ indicates that the number of moles of all components, except component $i$, are kept constant. This expression can be simplified to be a function of the mole fraction, $x_{i}$,
     \begin{shaded}
        \begin{equation}  
           d(nM) = n\Partial[M]{P}{T,x}dP + n\Partial[M]{T}{P,x}dT + \summation[\overline{M}_{i}dn_{i}]{i}{},\label{Chapter:VLE:PartialMolarProperties:PartialProperties2}
        \end{equation}
     \end{shaded}
     \noindent where $\overline{M}_{i} = \Partial[(nM)]{n_{i}}{T,P,n_{j\ne i}}$ defines the {\it partial molar property} $\overline{M}_{i}$ of species $i$ in solution, \ie the change of the total property $nM$ of a mixture of $\mathcal{C}$ species resulting from the addition at constant $T$ and $P$ of infinitesimal amount of species $i$ to a prescribed amount of solution. We will discuss partial molar properties in more details in Module~\ref{Section:05}.
  \end{subequations}

%%% SUBSECTION
\subsection{Excess Properties}\label{Chapter:VLE:ExcessProperties}
  
In Section~\ref{Section:03:ResidualProperties}, we have defined {\it residual properties} for gases as the difference between any extensive thermodynamic property, $M$, in real gases and its equivalent assuming ideal gas behaviour, $M^{\text{ig}}$. The equivalent for liquids is called {\it excess properties}, \ie the deviation from an ideal liquid solution property.

Let's assume that $M$ is any extensive thermodynamic property (\eg $V$, $U$, $H$, $S$, $G$ and $A$), the excess property $M^{\text{E}}$ is defined as the difference between the property value of a solution and the value it would have as an ideal solution at the same $T$, $P$ and composition,
\begin{subequations}
  \begin{shaded}
    \begin{equation}
       M^{\text{E}} \equiv M - M^{\text{id}},\label{Chapter:VLE:PartialMolarProperties:ExcessProperties1a}
    \end{equation}
  \end{shaded}
  \noindent where properties in ideal ({\it id}) solutions of multiple chemical species can be represented as,
    \begin{displaymath}
       M^{\text{id}} = \summation[x_{i}M_{i}]{i}{},
    \end{displaymath}
    where $M_{i}$ is an extensive {\it molar property of the \underline{pure} chemical species}. For example, if we mix equal volumes of two species (\eg water and ethanol), the final volume \underline{is not} the sum of the individual volumes. In fact, the final volume will be slightly larger than the sum of the individual volumes and this is due to the non-ideality behaviour of real liquid fluids. Thus for a binary solution,
    \begin{equation}
      V^{\text{E}} = V - V^{\text{id}} = V - \summation[x_{i}V_{i}]{i}{} = V -\left(x_{1}V_{1}+x_{2}V_{2}\right).\label{Chapter:VLE:PartialMolarProperties:ExcessProperties1b} 
    \end{equation}
    %This equation shows that the total volume of a mixture of two components is different from the simple addition of both individual volumes. 
\end{subequations}



%%% SECTION
\section{Criteria for Chemical Equilibrium}\label{Chapter:VLE:ChemicalEquilibrium}

  
%%% SUBSECTION
\subsection{Chemical Potential $\left(\mu_{i}\right)$}\label{Chapter:VLE:ChemicalPotential}

If we apply the concept of partial molar property to the Gibbs free energy definition, assuming that this thermodynamic potential is a function of $P$, $T$ and $n$, \ie $G^{\text{t}}= nG = G\left(T,P,n_{1},n_{2},\cdots,n_{\mathcal{C}}\right)$,
  \begin{subequations}
      \begin{equation}
         d(nG) = n\Partial[G]{P}{T,x}dP + n\Partial[G]{T}{P,x}dT + \summation[\overline{G}_{i}dn_{i}]{i}{}.\label{Chapter:VLE:PartialMolarProperties:ChemPotentialDef1}
      \end{equation}
      By definition the {\it partial molar Gibbs free energy} is called \blue{chemical potential $\left(\mu_{i}\right)$},
      \begin{shaded}
         \begin{equation}
            \mu_{i} = \Partial[(nG)]{n_{i}}{T,P,n_{j\ne i}} \;\;\Longleftrightarrow\;\; \mu_{i} = \overline{G}_{i}.\label{Chapter:VLE:PartialMolarProperties:ChemPotentialDef1b}
         \end{equation}
      \end{shaded}
      The chemical potential can be understood as an energy associated with interactions of atoms/molecules in a mixture that controls the tendency of molecules to leave the solution (or return to the solution), and to chemically react. We have defined the Gibbs free energy in Eqn.~\ref{Mod03_GibbsFundamentalRelation01} as a function of $T$ and $P$, now we will extend it to be also a function of the number of moles of chemical species -- Eqn.~\ref{Chapter:VLE:PartialMolarProperties:ChemPotentialDef1},
      \begin{equation}
         d(nG) = (nV)dP - (nS)dT + \summation[\mu_{i}dn_{i}]{i}{},\label{Chapter:VLE:PartialMolarProperties:ChemPotentialDef1c}
      \end{equation}
      considering $n=1\;\Rightarrow n_{i}=x_{i}$, and Eqn.~\ref{Chapter:VLE:PartialMolarProperties:ChemPotentialDef1c} becomes
      \begin{shaded}
        \begin{equation}
          \blue{dG = VdP -SdT + \summation[\mu_{i}dx_{i}]{i}{}},\label{Chapter:VLE:PartialMolarProperties:ChemPotentialDef1d}
        \end{equation}
      \end{shaded}
  \end{subequations}
  
%%% SUBSECTION
\subsection{Thermodynamic Equilibrium}\label{Chapter:VLE:thermodynamicEquilibrium}

There are two main conditions for a system to achieve {\it chemical equilibrium}:
  \begin{enumerate}[a)]
     \item all distinct $\mathcal{P}$ phases that may co-exist are in equilibrium with each other, as such as there is no net mass transfer of any chemical species between phases;
     \item all chemical reactions that may occur between species are also in equilibrium, \ie there is no net progress \wrt conversion of reactants to products (and {\it vice-versa}).
  \end{enumerate}

  \begin{subequations}

    \medskip

    Let's initially consider a closed system (either homogeneous or heterogeneous) in thermal and mechanical equilibrium with the surroundings. In addition, let's also assume that the system is not under chemical equilibrium, \ie there are effective mass transfer across the phase boundary. Such transfer of matter occurs `till the point when the system is also at chemical equilibrium. In reality, these changes towards chemical equilibrium occur by infinitesimal gradients and are, therefore, irreversible. Applying the {\it First Law},
    \begin{displaymath}
      dU^{\text{t}} = dQ + dW\;\;\;\text{ with }\;\;\; dW = -PdV^{\text{t}},
    \end{displaymath}
    with the heat exchange between the system and the surroundings,
    \begin{displaymath}
      dS_{\text{surr}} = \frc{dQ_{\text{surr}}}{T_{\text{surr}}} = - \frc{dQ^{\text{t}}}{T},\;\;\text{ where } dQ_{\text{surr}} = - dQ^{\text{t}}.
    \end{displaymath}
    However, by the Second Law, $dS_{\text{surr}}+dS^{\text{t}} \ge 0$, and if we combine the expressions above $ - \frc{dQ^{\text{t}}}{T}+dS^{\text{t}}\ge 0$,
    \begin{displaymath}
      dQ^{\text{t}} \leq TdS^{\text{t}},
    \end{displaymath}
    \ie for every allowed change in state, the system can not spontaneously leave the current state. The system is said to be at \underline{stable equilibrium}.
    \begin{shaded}
      The \underline{entropy} of an adiabatically (\ie $dS<0$) isolated stable equilibrium system is \underline{maximum}.
    \end{shaded}
    We can write the criterion for stable equilibrium in non-adiabatic system from the First Law for $n$ moles,
    \begin{eqnarray}
      dQ = dU + PdV -\mu dn &\leq& TdS \label{Chapter:VLE:PartialMolarProperties:EquilibriumCriteria0} \\
      dU + PdV -\mu dn - TdS &\leq& 0,\label{Chapter:VLE:PartialMolarProperties:EquilibriumCriteria1}
    \end{eqnarray}
    thus,
    \begin{enumerate}[i)]
        \item if we make $U$, $V$ and $n$ constants, $dQ=0$ and the stability condition becomes \underline{$dS \ge 0$} $\Longrightarrow$ \underline{$S$ is maximum};
        \item if $S$, $V$ and $n$ are set as constants, then from Eqn.~\ref{Chapter:VLE:PartialMolarProperties:EquilibriumCriteria1}, the system will be stable if \underline{$dU\le 0$} $\Longrightarrow$ \underline{$U$ is minimum};
        \item if $T$, $V$ and $n$ are kept constants, then Eqn.~\ref{Chapter:VLE:PartialMolarProperties:EquilibriumCriteria1}, becomes
            \begin{eqnarray}
              dU - TdS &\leq& 0 \nonumber \\
              d(U-TS) &\leq& 0 \;\;  \text{(for } T \text{  constant}),
            \end{eqnarray}
            however, we have seen the definition of the Helmholtz free energy, $U-TS =A$, thus $dA\leq0$ $\Longrightarrow$ \underline{$A$ is minimum};
        \item from the Helmholtz free energy definition,
            \begin{eqnarray}
              dA &=& \overbrace{dU}^{\text{from Eqn.~\ref{Chapter:VLE:PartialMolarProperties:EquilibriumCriteria0}}: dU = dQ -dW +\mu dn} -SdT - TdS \nonumber \\
                &=& \red{dQ} -dW + \mu dn -SdT \red{-TdS},\label{Chapter:VLE:PartialMolarProperties:EquilibriumCriteria2}
            \end{eqnarray}
            from the Clausius inequality, $dQ-TdS \leq 0$, and if we consider $T$ and $n$ constants $\Longrightarrow\;\; dA = -dW + (dQ - TdS) \leq 0$, therefore
            \begin{displaymath}
                 dA \leq -dW \;\;\;\text{ or }\;\;\; W \leq -\Delta A.
            \end{displaymath}
            This means that $-\Delta A$ is the maximum work we can obtain from a process performed under constant $T$ and $n$ conditions. Also, as $A$ is a \underline{state function}, it does \underline{not} depend on the path, but only on the initial and final states;
         \item assuming $S$, $P$ and $n$ are kept constant, then Eqn.~\ref{Chapter:VLE:PartialMolarProperties:EquilibriumCriteria1} becomes
            \begin{displaymath}
                 d(U+PV) = dH \leq 0,
            \end{displaymath}
            for stable equilibrium $\Longrightarrow$ \underline{enthalpy is minimum};
         \item now, suppose that $T$, $P$ and $n$ are held constant, Eqn.~\ref{Chapter:VLE:PartialMolarProperties:EquilibriumCriteria1} becomes,
            \begin{displaymath}
                 d(U+PV-TS)  \leq 0,
            \end{displaymath}
            but $U+PV-TS = H-TS = G$, \ie \underline{the Gibbs free energy is minimum} for stable equilibrium with fixed $T$, $P$ and $n$.    
    \end{enumerate}
    In Table~\ref{Chapter:VLE:PartialMolarProperties:TableEquilibriumCriteria}, the \blue{equilibrium criteria} is summarised for all thermodynamic potentials. The stability criteria for \blue{Gibbs free energy} is particularly important for several chemical engineering processes involving closed systems, as so as we can state,

    \begin{shaded}
         `The equilibrium state of a closed system is that state for which the total Gibbs free energy is a minimum \wrt all possible changes at the given $T$ and $P$.'
      \end{shaded}
    
    \begin{table}
    \begin{center}
      \begin{tabular}{c c c c c}
         \hline
          Held            & State    &  Definition & Differential         &  Stable Equilibrium \\
          Fixed           & Function &             &                      &   Criterion          \\
          \hline
          $U$, $V$, $n$   & $S$      & --          & $dS=\frac{dQ}{T}$    & Maximum             \\
          $S$, $V$, $n$   & $U$      & --          & $dU=TdS-PdV+\mu dn$ & Minimum             \\
          $S$, $P$, $n$   & $H$      & $H\equiv U+PV$& $dH=TdS+VdP+\mu dn$ & Minimum             \\
          $T$, $V$, $n$   & $A$      & $A\equiv U-TS$& $dA=-SdT-PdV+\mu dn$ & Minimum             \\
          $T$, $P$, $n$   & $G$      & $G\equiv H-TS$& $dG=-SdT+VdP+\mu dn$ & Minimum             \\
      \end{tabular}
    \end{center}
    \caption{Criteria for thermodynamic equilibria.}\label{Chapter:VLE:PartialMolarProperties:TableEquilibriumCriteria}
    \end{table}
    
  \end{subequations}

%%% SUBSECTION
\subsection{Chemical Potential and Thermodynamic Equilibrium}\label{Chapter:VLE:ChemPotThermEquil}

Now, let's consider a closed system consisting of two phases in equilibrium  (vapour and liquid, or solid and vapour or liquid and solid), $\alpha$ and $\beta$. Each of these phases may be considered as an open system with an interface between them, where mass and energy can flow freely. Also, both systems (or phases) are in mechanical and thermal equilibrium $\left(\ie\; T^{\alpha}=T^{\beta}=T\text{ and } P^{\alpha}=P^{\beta}=P\right)$. We can apply Eqn.~\ref{Chapter:VLE:PartialMolarProperties:ChemPotentialDef1c} to both phases,
  \begin{subequations}

     \begin{eqnarray}
        d(nG)^{\alpha} &=& (nV)^{\alpha}dP - (nS)^{\alpha}dT + \summation[\mu_{i}^{\alpha}dn_{i}^{\alpha}]{i}{},\label{Chapter:VLE:PartialMolarProperties:ChemPotentialDef1c1} \\
        d(nG)^{\beta} &=& (nV)^{\beta}dP - (nS)^{\beta}dT + \summation[\mu_{i}^{\beta}dn_{i}^{\beta}]{i}{}.\label{Chapter:VLE:PartialMolarProperties:ChemPotentialDef1c2} 
     \end{eqnarray}
     The change in total Gibbs free energy of a two-phase system is the sum of the changes in each phase. As mass is transferred across the interface, volume and entropy of each phase may change,
     \begin{eqnarray}
        (nV) &=& (nV)^{\alpha}+ (nV)^{\beta}, \nonumber \\
        (nS) &=& (nS)^{\alpha}+(nS)^{\beta}, \nonumber
     \end{eqnarray}
     Summing up Eqns.~\ref{Chapter:VLE:PartialMolarProperties:ChemPotentialDef1c1} and~\ref{Chapter:VLE:PartialMolarProperties:ChemPotentialDef1c2} and using the above relations,
     \begin{displaymath}
        d(nG) = (nV)dP - (nS)dT + \summation[\mu_{i}^{\alpha}dn_{i}^{\alpha}]{i}{} + \summation[\mu_{i}^{\beta}dn_{i}^{\beta}]{i}{} = 0,
     \end{displaymath}
     and as the whole system is closed and the net mass transfer is zero, $dn_{i}^{\alpha}=-dn_{i}^{\beta}$ thus,
     \begin{displaymath}
        d(nG) = (nV)dP - (nS)dT + \summation[\left(\mu_{i}^{\alpha}-\mu_{i}^{\beta}\right)dn_{i}^{\alpha}]{i}{} = 0,
     \end{displaymath}
     as we are considering that system is already under mechanical and thermal equilibrium,
     \begin{equation}
        \summation[\left(\mu_{i}^{\alpha}-\mu_{i}^{\beta}\right)dn_{i}^{\alpha}]{i}{} = 0,
     \end{equation}
     This equation describes changes in the chemical potential and in the mass of each component in each phase. In fact, throughout this derivation, $dn_{i}^{\alpha}$ is enforced to be independent of each other and therefore \underline{can not be zero}, hence
      \begin{shaded}
         \begin{equation}
            \mu_{i}^{\alpha} = \mu_{i}^{\beta}, \;\;\;\;\forall i=1,2,\cdots, \mathcal{C}.\label{Chapter:VLE:PartialMolarProperties:ChemPotentialDef1c3} 
         \end{equation}
      \end{shaded}
      This rather simple relation is crucial for understanding phase equilibria, as it states that for an arbitrary closed system at fixed $T$, $P$, thermodynamic equilibrium will be reached when the chemical potential of \underline{all} chemical species in one of the phases is \underline{equal} to the counterpart in the other phase. In fact, we can extend this argument for any number of phases, 
      \begin{shaded}
         \begin{equation}
            \blue{\mu_{i}^{\alpha} = \mu_{i}^{\beta} = \cdots = \mu_{i}^{\mathcal{P}} \;\;\;\;\forall i=1,2,\cdots, \mathcal{C}.}\label{Chapter:VLE:PartialMolarProperties:ChemPotentialDef1c3} 
         \end{equation}
      \end{shaded}

  \end{subequations}


\begin{comment}
%%% SUBSECTION
\subsection{Gibbs Phase Rule and Duhem's Theorem}\label{Chapter:VLE:GibbsPhaseRule}

In Section~\ref{Section:02:PVTBehaviour} (Eqns.~\ref{Mod02_GibbsPhaseRuleReactive}-~\ref{Mod02_GibbsPhaseRule}), we introduced the idea of {\it Gibbs phase rule} and how this `rule' can help determining the number of degrees of freedom of any system. Here, a mathematical framework will be used to demonstrate the validity of the {\it Gibbs phase rule}. Consider a non-reactive system closed system in thermodynamic equilibrium with $\mathcal{P}$ co-existing phases and $\mathcal{C}$ independent chemical species, the number of degrees of freedom for such system (\ie number of intensive variables that may vary independently of each other) is given by
\begin{shaded}
  \begin{center}
     \red{Degrees of Freedom} = \blue{Total Number of Intensive Variables} - \purple{Number of Independent Equations}
  \end{center} 
\end{shaded}

Thus,
\begin{enumerate}[a)]
   \item \blue{Total Number of Intensive Variables}: $T$, $P$ and $\mathcal{C}-1$ mole fractions for each of the $\mathcal{P}$ phases;
   \item \purple{Number of Independent Equations}: $\left(\mathcal{P}-1\right)\mathcal{C}$
\end{enumerate}
XXXXXXXXXXXXXXXXXXXXXXXXXXXXXXXXXXXXXXXXXXXXXXXXXXXXXXXXXXXXXXXXXXXXXXXXXXXXXXXXXXXXXXXXXXXXX
\end{comment}


%%% SECTION
\section{Vapour-Liquid Equilibrium}\label{Chapter:VLE:VLE}

One of the main objectives of Module~\ref{Section:02} was to study the thermodynamic equilibrium of pure components that involved phase change (see Figs.~\ref{Mod02Fig01}-~\ref{Mod02Fig02}), and in particular for vapour-liquid equilibrium (VLE). $PVT$ relations of pure components were quantitatively expressed through equations of state. Here we will extend the investigation of VLE to mixtures of arbitrary number of components.

%%% SUBSECTION
\subsection{Qualitative Analysis: Phase Diagrams}\label{Chapter:VLE:PhaseDiagrams}
In Section~\ref{Section:02:PVTBehaviour} (Eqns.~\ref{Mod02_GibbsPhaseRuleReactive}-~\ref{Mod02_GibbsPhaseRule}), the {\it Gibbs phase rule} was introduced to help determining the number of degrees of freedom of any system,
    \begin{displaymath}
        \Psi = 2 + \mathcal{C} - \mathcal{P}.
    \end{displaymath}
For a binary components system in equilibrium, $\mathcal{C}=2$, the phase rule yields a degree of freedom of $\Psi=4-\mathcal{P}$. In VLE system, the number of independent intensive variables becomes $2$, and it will need to be chosen between:
\begin{enumerate}[a)]
   \item temperature ($T$);
   \item pressure ($P$);
   \item molar fraction of the more volatile component\footnote{In this module, we will adopt a notation in which for a mixture of $\mathcal{C}$ components, the most volatile will be assigned as number \blue{$1$}. Also, we will follow here the notation used in most thermodynamic text-books in which the molar fraction of component $i$ in vapour and liquid phases are \blue{$y_{i}$} and \blue{$x_{i}$}, respectively.} in the vapour phase $\left(y_{1}\right)$, and;
   \item molar fraction of the more volatile component in the liquid phase $\left(x_{1}\right)$.
\end{enumerate}
A plot with these 4 variables is not very practical, as shown in Fig.~\ref{Mod04Fig01} for a mixture of hexane and triethylamine. In this $P-T-xy$ diagram, to ensure a single phase system $\left(\mathcal{P}=1\right)$, temperature, pressure and {\bf one} molar fraction need to be fixed. For a two-phase system $\left(\mathcal{P}=2\right)$, two variables need to be fixed and the system is expressed as a surface -- \eg the plane of constant pressure of $0.54$ bar. %Although insightful, this plot is not very convenient to assess compositions and phase behaviour, and is limited to a binary systems. 
      \begin{figure}[h]
         \begin{center}
           \includegraphics[width=.9\columnwidth,clip]{./../Pics/PTxy_diagram}
           \vspace{-.1cm}\caption{$P-T-xy$ diagram for hexane and triethylamine (extracted from Sandler, 2006).}\label{Mod04Fig01}
         \end{center}
       \end{figure}

In Fig.~\ref{Mod04Fig02}, a generic $P-T-xy$ diagram is shown. Vertical planes parallel to \blue{0-1-R-M-U-0} (\eg plane \blue{E-A-B-D-E}) indicates $P-xy$ diagrams at constant temperature, whereas horizontal planes parallel to \blue{K-J-I-H-K} indicates $T-xy$ diagrams at constant pressure. The (subcooled) liquid region lies \underline{above} the upper surface of the solid (grey) projection, and (superheated) vapour region lies \underline{below} the under surface. The interior of the solid projection between the two surfaces is the region of coexistence of both liquid and vapor phases. 

The $x_{1},y_{1}$ axis is limited to \blue{zero} and \blue{one}, thus at $x_{1}$ (or $y_{1}$) equal to zero, $x_{2}$ (or $y_{2}$) is equal to unity, \ie there is just pure component \blue{2} in solution. Similarly at $x_{1}$ (or $y_{1}$) equal to one, $x_{2}$ (or $y_{2}$) is equal to zero, \ie there is just pure component \blue{1} in solution. The diagram also depicts points \blue{$C_{1}$} and \blue{$C_{2}$} along the vertical planes of $x_{1},y_{1}=0$ and $x_{1},y_{1}=1$, respectively, these two points are the coordinates of the critical conditions $\left(P_{c,1}, P_{c,2}, T_{c,1} \text{ and } T_{c,2}\right)$ of both \underline{pure components}. Critical coordinates for mixtures at various compositions of \blue{1} and \blue{2} lie along a line on the rounded edge of the surface between \blue{$C_{1}$} and \blue{$C_{2}$}.
      \begin{figure}[h] 
         \begin{center}
             \includegraphics[width=.75\columnwidth,clip]{./Figs/PTxy_Digram} 
             \vspace{-.1cm}\caption{$P-T-xy$ diagram for binary mixtures (extracted from Smith, Van Ness and Abbott, 2000).}\label{Mod04Fig02}
         \end{center}
       \end{figure}
For each constant temperature plane, the upper line of the solid (grey) projection is the \blue{saturated liquid line} whilst the lower line is the \blue{saturated vapour line} (the reader should remember that these lines are also present for pure substances in the $Ts$ and $Ph$ diagrams , Fig.~\ref{Mod03Fig02}).

Compositions of each phase can be obtained from a parallel line linking $x_{1}$ and $y_{1}$. For example, in the `lens' \blue{U-M-R-N-U} contained in the plane \blue{0-U-M-R-1-Q-0}, the vapour phase $\left(\text{with composition }\red{y_{1}=\alpha}\text{ at } \blue{K_{1}}\right)$ is in equilibrium with the liquid phase $\left(\text{with composition }\red{x_{1}=\beta}\text{ at } \blue{K_{2}}\right)$. This parallel line, \blue{$K_{1}-K_{2}$}, is called \underline{tie-line} (Fig.~\ref{Mod04Fig03}b). The intersection of the tie-line with the saturated liquid line is referred as \blue{bubble pressure}, and the intersection with the saturated vapour line is called \blue{dew pressure} (at constant temperature).

$P-T-xy$ diagram, although insightful, is not very practical. Projections of $P-xy$ plane at constant temperature are shown in Fig.~\ref{Mod04Fig03}. Now, let's assume that the fluid of this diagram, Fig.~\ref{Mod04Fig03}(c), is at liquid phase at coordinate \blue{$M$} with composition \blue{$\left(x_{1}=x_{z};y_{1}=0\right)$}. As pressure is reduced along the \blue{$M-S$} line, the first bubble of vapour appears at \blue{$N$}; composition at this coordinate is \blue{$\left(x_{1}=x_{z};y_{1}=y_{BP}\right)$}, and this is called \blue{\it bubble point}. Further reduction of pressure to coordinate \blue{$Q$} (still at constant temperature), the fluid becomes partially vaporised and the composition at this coordinate is given by the {\it tie-line} \blue{$Q_{L}-Q_{V}$} -- \blue{$\left(x_{1}=x_{Q};,y_{1}=y_{Q}\right)$}. 
      \begin{figure}[h]
        \vbox{\hbox{\hspace{-1.cm}\includegraphics[width=.7\linewidth,clip]{./../Pics/VLE_Pxy_Diagram1}
            \hspace{-4.cm}\includegraphics[width=.7\linewidth,clip]{./../Pics/VLE_Pxy_Diagram2}}
          \vspace{-.5cm}
          \hbox{\hspace{4.5cm}(a)\hspace{7cm}(b)}
          \vspace{-.5cm}\hbox{\hspace{3.cm}\includegraphics[width=.6\linewidth,clip]{./../Pics/VLE_Pxy_Diagram3b}}
          \vspace{-.1cm}
          \hbox{\hspace{7.5cm}(c)}}
             \vspace{-.1cm}\caption{VLE for binary mixtures: $P-xy$ diagrams at constant temperature.}\label{Mod04Fig03}
      \end{figure}
      
As the pressure is continuously reduced along line \blue{$Q-R$}, more liquid is vaporised until \blue{$R$}, when the last droplet of liquid (dew) is vaporised. This coordinate is called \blue{\it dew point}. The composition at $N$ is \blue{$\left(x_{1}=x_{DB};y_{1}=y_{z}\right)$}. Continued reduction of pressure towards \blue{$S$} leads to (pure) vapour region with composition \blue{$\left(x_{1}=0,y_{1}=y_{z}\right)$}.
     \begin{figure}[h]
        \vbox{\hbox{\hspace{3.cm}\includegraphics[width=.6\linewidth,clip]{./../Pics/VLE_xy_DiagramIdeal}}
          \vspace{-.5cm}
          \hbox{\hspace{8.cm}(a)}
          \hbox{\hspace{-1.cm}\includegraphics[width=.5\linewidth,clip]{./../Pics/VLE_xy_DiagramNonIdeal1}
            \hspace{-0.cm}\includegraphics[width=.5\linewidth,clip]{./../Pics/VLE_xy_DiagramNonIdeal2}}
          \vspace{-.5cm}
          \hbox{\hspace{3.5cm}(b)\hspace{7.5cm}(c)}}
          \vspace{-.1cm}\caption{VLE for binary mixtures: $xy$ diagrams at constant pressure of (a) ideal and (b-c) non-ideal solutions.}\label{Mod04Fig04}
      \end{figure}

     Another common phase diagram is the $x-y$, Fig.~\ref{Mod04Fig04}(a), in which either pressure or temperature is kept constant. In this diagram, at constant pressure, there are two curves, the upper curve represents the composition at equilibrium with varying temperature. The straight line represents the equimolar composition, \ie $x_{1}=y_{1}$. However, in several solutions of industrial interest, molecular attractive forces from the heavier component (2) keep molecules of the more volatile component in the liquid solution, constraining its evaporation. This leads into an equilibrium pressure smaller than the expected at the same liquid composition. Such deviation from ideality is called \blue{\it azeotropy}. In practical terms, any component from solutions can often be separated through distillation processes, however azeotropes may lead to vapour-liquid-liquid equilibrium (VLLE) and other separations strategies may need to be used.  
     
%%% SUBSECTION
\subsection{VLE Models: Raoult's Law}\label{Chapter:VLE:RaoultsLaw}
  \begin{subequations}
      The diagrams of binary (as well as multi-component) mixtures shown in the previous section are obtained from experiments and introduce a qualitative analysis of the PVT behaviour of solutions. However, classic thermodynamics provides a mathematical framework for systematic correlation, extension, generalisation and interpretation of data.
      For ideal vapour-liquid systems, the \blue{Raoult's law} provides a relation between the compositions of the vapour and liquid phases,
         \begin{shaded}
           \begin{equation}
             P_{i} = y_{i}P = x_{i}P_{i}^{\text{sat}},\;\;\;\forall i\in\left\{1,2,\cdots,\mathcal{C}\right\}\label{Chapter:VLE:PartialMolarProperties:RaoultLaw} 
           \end{equation}
         \end{shaded}
      $P_{i}=y_{i}P$ in Eqn.~\ref{Chapter:VLE:PartialMolarProperties:RaoultLaw} is known as \blue{\it partial pressure}, and $P_{i}^{\text{sat}}$ is the saturated pressure (or saturated vapour pressure) of pure species $i$ and was defined in Eqn.~\ref{Mod03_Antoine}. This rather simple relation can only be applied to ideal solutions (for both phases). For a gas mixture to be considered ideal the pressure need to be equal or smaller than atmospheric. There are several relations for ideal gas mixtures, \eg
       \begin{enumerate}[a)]
           \begin{shaded}    
               \item Dalton's Law:
                  \begin{equation}
                      P = \summation[P_{i}]{i=1}{\mathcal{C}} = \summation[y_{i}P]{i=1}{\mathcal{C}}\label{Chapter:VLE:PartialMolarProperties:DaltonLaw}
                  \end{equation}
           \end{shaded}
           \item Amagat's Law:
              \begin{displaymath}
                  V^{\text{t}} = \summation[V_{i}^{\text{t}}]{i=1}{\mathcal{C}} = \summation[y_{i}V^{\text{t}}]{i=1}{\mathcal{C}}
              \end{displaymath}
           \item Kay's rule for pseudo-critical temperature and pressure:
              \begin{displaymath}
                  T_{c}^{\text{t}} = \summation[y_{i}T_{c,i}]{i=1}{\mathcal{C}},\;\;\text{ and }\;\;P_{c}^{\text{t}} = \summation[y_{i}P_{c,i}]{i=1}{\mathcal{C}}
              \end{displaymath}
       \end{enumerate}
       Liquid solutions are considered ideals if the molecular interaction between the similar species is identical to that between dissimilar species. Thermodynamics of liquid solutions are the focus of Module~\ref{Section:05}.
       
\end{subequations}

%%% SUBSECTION
\subsection{Henry's Law}\label{Chapter:VLE:HenryLaw}

A typical application of VLE is when gases are solubilised in liquid solutions, \eg O$_{2}$ in water, CO$_{2}$ in soft drinks, air in the blood stream etc. In these cases, gases have relatively low solubility (mole fraction varying between 10$^{-5}$ to 10$^{-2}$) in liquid solvents. Application of Raoult's law is limited to $P^{\text{sat}}$. If the temperature exceeds the critical temperature, Raoult's law is no longer applicable as $P^{\text{sat}}$ can not be defined. For such cases, a new relation is required that relates compositions in both phases and the pressure,
\begin{shaded}
  \begin{equation}
      P_{i} = y_{i}P = x_{i}\mathcal{H}_{i},\label{Chapter:VLE:PartialMolarProperties:HenryLaw} 
  \end{equation}
\end{shaded}
\noindent where $\mathcal{H}$ is the Henry's constant obtained experimentally. Values of Henty's contant for several gases dissolved in water at 25$^{\circ}$C is shown in Table~\ref{Chapter:VLE:PartialMolarProperties:HenryLawTable}
 \begin{table}
  \begin{center}
    \begin{tabular}{l r || l r }
      \hline
       {\bf Gas}    &  ${\bf \mathcal{H}\text{ (bar)}}$ & {\bf Gas}    &  ${\bf \mathcal{H}\text{ (bar)}}$ \\
      \hline
         Acetylene  &   1350                            & He           &  126600 \\
         Air        &   72950                           & H$_{2}$      &  71600  \\
         CO$_{2}$    & 1670                              & H$_{2}$S     & 550 \\
         CO         &  54600                            &  CH$_{4}$    &  41850 \\
         C$_{2}$H$_{6}$ & 30600                          &  N$_{2}$     & 87650  \\
         Ethylene  & 11550                              & O$_{2}$      & 44380 \\
      \hline
    \end{tabular}
    \caption{Henry's constant for gases dissolved in water at 25$^{\circ}$C.}\label{Chapter:VLE:PartialMolarProperties:HenryLawTable}
  \end{center}
\end{table}

  
%%% SECTION
\section{Industrial Applications: Multi-Component VLE}

%%% SUBSECTION
\subsection{Dew-point and Bubble-point Calculations using Raoult’s Law }

Let's consider a vapour-liquid system containing $\mathcal{C}$ chemical species. From the phase rule the number of degrees of freedom is $\mathcal{C}$ that are $T$, $P$, $x_{i}$ and/or $y_{i}$. Four types of problems are possible:
\begin{center}
   \begin{tabular}{|l c c|}
      \hline 
      $\mathbf{VLE}$ {\bf Problem} & {\bf Specified Variables} &  {\bf Computed Variables} \\  
      \hline
          {\bf Bubble Pressure}        &  $T$ and $x_{i}$           &   $P$ and $y_{i}$          \\
          {\bf Dew Pressure}           &  $T$ and $y_{i}$           &   $P$ and $x_{i}$          \\
          {\bf Bubble Temperature}     &  $P$ and $x_{i}$           &   $T$ and $y_{i}$          \\
          {\bf Dew Temperature}        &  $P$ and $y_{i}$           &   $T$ and $x_{i}$          \\     
      \hline
   \end{tabular}
\end{center}
Such problems are calculated using three relations:
\begin{displaymath}
    P_{i} = y_{i}P = x_{i}P_{i}^{\text{sat}}, \;\;\; \summation[x_{i}]{i=1}{\mathcal{C}} = 1, \;\;\text{ and }\;\; \summation[y_{i}]{i=1}{\mathcal{C}} = 1,
\end{displaymath}
bearing in mind that $P_{i}^{\text{sat}}$ is expressed as a function of $T$. Thus,
\begin{enumerate}[a)]
    \item Bubble pressure: Given $T$ and $x_{i}$ $\left(\text{with }i=1,\cdots,\mathcal{C}\right)$, find $P$ that solves,
        \begin{displaymath}
           \summation[y_{i}]{i=1}{\mathcal{C}} = 1 \;\;\Longrightarrow\;\; \summation[\frc{x_{i}P_{i}^{\text{sat}}}{P}]{i=1}{\mathcal{C}} = 1,
        \end{displaymath}
        and then calculate $y_{i}= \frc{x_{i}P_{i}^{\text{sat}}}{P}$.
        
    \item Bubble temperature: Given $P$ and $x_{i}$ $\left(\text{with }i=1,\cdots,\mathcal{C}\right)$, find $T$ that solves,
        \begin{displaymath}
           \summation[y_{i}]{i=1}{\mathcal{C}} = 1 \;\;\Longrightarrow\;\; \summation[\frc{x_{i}P_{i}^{\text{sat}}}{P}]{i=1}{\mathcal{C}} = 1,
        \end{displaymath}
        and then calculate $y_{i}= \frc{x_{i}P_{i}^{\text{sat}}}{P}$.
      
    \item Dew pressure: Given $T$ and $y_{i}$ $\left(\text{with }i=1,\cdots,\mathcal{C}\right)$, find $P$ that solves,
        \begin{displaymath}
           \summation[x_{i}]{i=1}{\mathcal{C}} = 1 \;\;\Longrightarrow\;\; \summation[\frc{y_{i}P}{P_{i}^{\text{sat}}}]{i=1}{\mathcal{C}} = 1,
        \end{displaymath}
        and then calculate $y_{i}= \frc{x_{i}P_{i}^{\text{sat}}}{P}$.
      
    \item Dew temperature: Given $P$ and $y_{i}$ $\left(\text{with }i=1,\cdots,\mathcal{C}\right)$, find $T$ that solves,
        \begin{displaymath}
           \summation[x_{i}]{i=1}{\mathcal{C}} = 1 \;\;\Longrightarrow\;\; \summation[\frc{y_{i}P}{P_{i}^{\text{sat}}}]{i=1}{\mathcal{C}} = 1,
        \end{displaymath}
        and then calculate $y_{i}= \frc{x_{i}P_{i}^{\text{sat}}}{P}$.
\end{enumerate}


%%% SUBSECTION
\subsection{K-Value Correlations for Hydrocarbon Systems}

As discussed in Modules~\ref{Section:02}-~\ref{Section:03}, cubic equations of state are often used to describe the PVT behaviour of vapour and liquid phases of pure chemical species and mixtures. Although the use of cubic EOS is widely spread over the chemical industry, applicability to liquid phase is still limited. Several alternatives have been proposed and developed over the past 50 years, one of them, mostly used in petroleum and petrochemical industry for hydrocarbons, is called {\it K-value correlation}. $K$ is defined as
\begin{shaded}
   \begin{equation}
      K_{i} = \frc{P_{i}^{\text{sat}}}{P} = \frc{y_{i}}{x_{i}}.\label{Chapter:VLE:PartialMolarProperties:KValue}
   \end{equation}
\end{shaded}
Values for $K_{i}$ are tabulated for a large number of hydrocarbons at several pressure and temperature conditions and can be found in any chemical engineering handbook as part of \blue{DePriester chart} (Fig.~\ref{Mod04Fig05}).
  \begin{figure}[h]
     \begin{center}
         \includegraphics[width=1.\linewidth,clip]{./Figs/DePriesterCharts}
     \end{center}
     \caption{DePriester chart for several hydrocarbons (extracted from Smith, Van Ness and Abbott, 2000).}\label{Mod04Fig05}
  \end{figure}


%%% SUBSECTION
\subsection{Flash Distillation}\label{Chapter:VLE:FlashDistillation}

Flash distillation is a common process in chemical industry to obtain solutions with the required enrichment from a feed-stock. Figure~\ref{Mod04Fig06} shows a schematic of this process, where a liquid stream (\ie with pressure equal to or larger than its bubble point pressure) with molar fraction $F$ and overall composition $z_{i}$ is injected into a separation vessel through a pressure reduction valve. The sudden reduction in pressure leads to partial evaporation (\ie {\it flash}) of the liquid feed resulting in the formation of a vapour and a liquid stream. Vapour and Liquid streams have molar fraction $V$ and $L$, respectively, with compositions of $y_{i}$ and $x_{i}$. 
  \begin{figure}[h]
     \begin{center}
         \includegraphics[width=.7\linewidth,clip]{./Figs/FlashDistillation}
     \end{center}
     \caption{Schematic of a flash distillation process.}\label{Mod04Fig06}
  \end{figure}
        
      The overall mass balance of the system is
         \begin{displaymath}
             F = V + L,
         \end{displaymath}
         for convenience, $F$ is assumed equal to unity, $F=1$. Similarly, the mass balance for each component $i$ is
         \begin{displaymath}
            z_{i} = x_{i}L + y_{i}V = x_{i}\left(1-V\right) + y_{i}V.
         \end{displaymath}
         Replacing $x_{i} = \frc{y_{i}}{K_{i}}$,
         \begin{displaymath}
           z_{i} = \frc{y_{i}}{K_{i}}\left(1-V\right) + y_{i}V \;\;\Longrightarrow \;\;y_{i} = \frc{z_{i}K_{i}}{1+V\left(K_{1}-1\right)}. 
         \end{displaymath}
         As $\summation[y_{i}]{i=1}{\mathcal{C}} = 1$,
         \begin{shaded}
           \begin{equation}
              \summation[\frc{z_{i}K_{i}}{1+V\left(K_{i}-1\right)}]{i=1}{\mathcal{C}} = 1.\label{Chapter:VLE:PartialMolarProperties:FlashEquation} 
           \end{equation}
         \end{shaded}
         Solving a $P-T$ flash problem is to \underline{find $V$} that satisfies Eqn.~\ref{Chapter:VLE:PartialMolarProperties:FlashEquation}.
      
\clearpage

%%% SECTION
\section{Examples}

\begin{enumerate}[1)] 
%%%
%%% EXAMPLE
%%%
   \item\label{Mod04Ex01} Assuming a mixture of n-pentane $\left(nC_{5}\right)$ and n-heptane $\left(nC_{7}\right)$ is ideal, draw vapour-liquid $x_{5}\times y_{5}$ and $P-x_{5}y_{5}$ equilibrium diagrams for this mixtures at constant temperature of 50$^{\circ}$C. Given $P^{\text{sat}}$ relation,
    \begin{displaymath}
      \ln{P_{i}^{\text{sat}}} = A_{i} - \frc{B_{i}}{RT}
    \end{displaymath}
    with $A_{nC_{5}}=10.422$, $A_{nC_{7}}=11.431$, $B_{nC_{5}}=26799 \text{ J.mol}^{-1}$ and $B_{nC_{7}}=35200 \text{ J.mol}^{-1}$. Also [$P$] = bar and [$T$] = K.

% SOLUTION
 \noindent{\bf Solution:} At $T=$50$^{\circ}$C = 323.15 K, vapour saturated pressures for $nC_{5}$ and $nC_{7}$ (for short notation, let's use 5 and 7 as $nC_{5}$ and $nC_{7}$, respectively) are
           \begin{displaymath}
              P_{5}^{\text{sat}} = 1.5639\text{ bar}\;\;\text{ and }\;\;P_{7}^{\text{sat}} = 0.1881\text{ bar}.
           \end{displaymath}
           In order to calculate the equilibrium pressure at each liquid $nC_{5}$ composition, $x_{5}$,
           \begin{eqnarray}
               && P_{i} = x_{i}P_{i}^{\text{sat}} \;\;\Longleftrightarrow \;\; P = \summation[P_{i}]{i=1}{\mathcal{C}} = \summation[x_{i}P_{i}^{\text{sat}}]{i=1}{\mathcal{C}} \nonumber \\
               && P = x_{5}P_{5}^{\text{sat}} + x_{7}P_{7}^{\text{sat}} = x_{5}P_{5}^{\text{sat}} + \left(1-x_{5}\right)P_{7}^{\text{sat}} \nonumber 
           \end{eqnarray}
           And for vapour composition (Raoult's law),
           \begin{displaymath}
               y_{i} P = x_{i}P_{i}^{\text{sat}}  \;\;\Longleftrightarrow \;\; y_{5} = \frc{x_{5}P_{5}^{\text{sat}}}{P},
           \end{displaymath}
           replacing pressure, $P$, from the previous equation
           \begin{displaymath}
               y_{5} = \frc{x_{5}P_{5}^{\text{sat}}}{x_{5}P_{5}^{\text{sat}} + \left(1-x_{5}\right)P_{7}^{\text{sat}}}.
           \end{displaymath}
           The $x_{5} \times y_{5}$ diagram may be plot by giving values to $0.0\leq x_{5} \leq 1.0$ and calculating $x_{5}$ through the equation above.

           The $P-x_{5}y_{5}$ diagram may be plot using the relations,
           \begin{displaymath}
               y_{5} = \frc{x_{5}P_{5}^{\text{sat}}}{P}\;\;\;\text{ and }\;\;\; P = x_{5}P_{5}^{\text{sat}} + \left(1-x_{5}\right)P_{7}^{\text{sat}} \;\Longrightarrow\; x_{5} = \frc{P-P_{7}^{\text{sat}}}{P_{5}^{\text{sat}}-P_{7}^{\text{sat}}}.
           \end{displaymath}
           Thus, given values for $P_{7}^{\text{sat}}\leq P \leq P_{5}^{\text{sat}}$ $\Rightarrow$ Calculate $x_{5}$  $\Rightarrow$ Calculate $y_{5}$.
           
      \begin{figure}[h]
         \vbox{
             \hbox{\includegraphics[width=.5\linewidth,clip]{./Figs/Mod4Ex1}
                   \includegraphics[width=.5\linewidth,clip]{./Figs/Mod4Ex1b}}}
         \caption{Example~\ref{Mod04Ex01}}
      \end{figure}
\clearpage
 % 
%%%
%%% EXAMPLE
%%%
   \item\label{Mod04Ex02} Assuming a mixture of n-pentane $\left(nC_{5}\right)$ and n-heptane $\left(nC_{7}\right)$ is ideal, draw vapour-liquid $x_{5}\times y_{5}$ and $T-x_{5}y_{5}$ equilibrium diagrams for this mixtures at constant pressure of 1.013 bar. Given $P^{\text{sat}}$ relation,
    \begin{displaymath}
      \ln{P_{i}^{\text{sat}}} = A_{i} - \frc{B_{i}}{RT}
    \end{displaymath}
    with $A_{nC_{5}}=10.422$, $A_{nC_{7}}=11.431$, $B_{nC_{5}}=26799 \text{ J.mol}^{-1}$ and $B_{nC_{7}}=35200 \text{ J.mol}^{-1}$. Also [$P$] = bar and [$T$] = K.

% SOLUTION
 \noindent{\bf Solution:} In this problem, equilibrium temperature and compositions are not known, thus
          \begin{displaymath}
              P = x_{5}P_{5}^{\text{sat}} + x_{7}P_{7}^{\text{sat}} = x_{5}P_{5}^{\text{sat}} + \left(1 - x_{5}\right)P_{7}^{\text{sat}} = 1.013\text{ bar},
          \end{displaymath}
          where $P_{i}^{\text{sat}}$ is non-linear function of the temperature $T$, \ie
          \begin{displaymath}
              P = x_{5}\exp{\left(A_{5} - \frc{B_{5}}{RT}\right)} + \left(1 - x_{5}\right)\exp{\left(A_{7} - \frc{B_{7}}{RT}\right)}.
          \end{displaymath}
          This equation has 2 unknowns, $x_{5}$ and $T$, however we know that $0.0\leq x_{5} \leq 1.0$, thus we can give values on this range $\Rightarrow$ calculate $T$  $\Rightarrow$  obtain $y_{5}$.
          \begin{center}
              \begin{tabular}{ c c c }
                  $\mathbf{x_{5}}$ & $\mathbf{T}${\bf(K)}  & $\mathbf{y_{5}}$ \\
                    0.0000          & 370.80                & 0.0000   \\
                    0.1000          & 357.80                & 0.4050   \\
                    0.2000          & 347.73                & 0.6249   \\
                    0.3000          & 339.73                & 0.7535   \\
                    0.4000          & 333.21                & 0.8345   \\
                    0.5000          & 327.77                & 0.8881   \\
                    0.6000          & 323.13                & 0.9257   \\
                    0.7000          & 319.12                & 0.9528   \\
                    0.8000          & 315.60                & 0.9728   \\
                    0.9000          & 312.47                & 0.9881   \\
                    1.0000          & 309.67                & 1.0000   
              \end{tabular}
           \end{center}
           
      \begin{figure}[h]
         \vbox{
             \hbox{\includegraphics[width=.5\linewidth,clip]{./Figs/Mod4Ex2a}
                   \includegraphics[width=.5\linewidth,clip]{./Figs/Mod4Ex2b}}}
         \caption{Example~\ref{Mod04Ex02}}
      \end{figure}
\clearpage
 % 
%%%
%%% EXAMPLE
%%%
   \item\label{Mod04Ex03} Estimate the bubble and dew point temperatures of a mixture of 25 mol-$\%$ of n-pentane, 45 mol-$\%$ of n-hexane and 30 mol-$\%$ of n-heptane at 1.013 bar. Given $P^{\text{sat}}$ relation,
    \begin{displaymath}
      \ln{P_{i}^{\text{sat}}} = A_{i} - \frc{B_{i}}{RT}
    \end{displaymath}
    with $A_{nC_{5}}=10.422$, $A_{nC_{6}}=10.456$, $A_{nC_{7}}=11.431$, $B_{nC_{5}}=26799 \text{ J.mol}^{-1}$, $B_{nC_{6}}=29676 \text{ J.mol}^{-1}$ and $B_{nC_{7}}=35200 \text{ J.mol}^{-1}$. Also [$P$] = bar and [$T$] = K.

% SOLUTION
 \noindent{\bf Solution:} Assuming the solution is ideal and the gas phase behaves as an ideal gas (\ie low/room pressure), Raoult's law can be used,
    \begin{displaymath}
       x_{i}P_{i}^{\text{sat}} = y_{i}P = P_{i} \;\;\text{ with }\;\; \summation[x_{i}P_{i}^{\text{sat}}]{i}{} = \summation[P_{i}]{i}{} = P, \;\summation[x_{i}]{i}{}=1=\summation[y_{i}]{i}{}.
    \end{displaymath}
    For the bubble point, the procedure is:
    \begin{enumerate}[i)]
       \item choose an initial guess for the bubble point temperature, $T^{\text{guess}}$;
       \item calculate $y_{i}=\frc{x_{i}P_{i}^{\text{sat}}}{P}$;
       \item then:
           \begin{itemize}
              \item if $\summation[y_{i}]{i}{} = 1 \;\;\Rightarrow \;\; T^{\text{guess}} = T$;
              \item if $\summation[y_{i}]{i}{} > 1 \;\;\Rightarrow \;\; T^{\text{guess}} > T \;\; \Rightarrow\;\; \text{adjust } T^{\text{guess}}$;
              \item if $\summation[y_{i}]{i}{} < 1 \;\;\Rightarrow \;\; T^{\text{guess}} < T \;\; \Rightarrow\;\; \text{adjust } T^{\text{guess}}$.
           \end{itemize}
    \end{enumerate}
    Thus, for this problem,
    \begin{displaymath}
        \summation[y_{i}]{i}{} = 1\;\; \Longleftrightarrow \;\; \frc{x_{5}P_{5}^{\text{sat}}}{P} + \frc{x_{6}P_{6}^{\text{sat}}}{P} + \frc{x_{7}P_{7}^{\text{sat}}}{P} = 1,
    \end{displaymath}
    where $P_{i}^{\text{sat}}$ is a function of $T$, 
    \begin{displaymath}
        \frc{x_{5}\exp{\left(A_{5}-\frc{B_{5}}{RT}\right)}}{P} + \frc{x_{6}\exp{\left(A_{6}-\frc{B_{6}}{RT}\right)}}{P} + \frc{x_{7}\exp{\left(A_{7}-\frc{B_{7}}{RT}\right)}}{P} = 1
    \end{displaymath}
    Using $T^{\text{guess}}=298.15$ K as initial guess, $T=334.9380$ K (bubble point temperature). Now, calculating the vapour composition, $y_{i}=\frc{x_{i}P_{i}^{\text{sat}}}{P}$ leads to,
    \begin{displaymath}
        y_{5} =0.5483,\;\; y_{6} = 0.3634\;\;\text{ and }\;\;y_{7} = 0.0883\;\;\Rightarrow\;\; \summation[y_{i}]{i}{} = 1.0000
    \end{displaymath}

\medskip
    For the dew point, the procedure is:
    \begin{enumerate}[i)]
       \item choose an initial guess for the dew point temperature, $T^{\text{guess}}$;
       \item calculate $x_{i}=\frc{y_{i}P}{P_{i}^{\text{sat}}}$;
       \item then:
           \begin{itemize}
              \item if $\summation[x_{i}]{i}{} = 1 \;\;\Rightarrow \;\; T^{\text{guess}} = T$;
              \item if $\summation[x_{i}]{i}{} > 1 \;\;\Rightarrow \;\; T^{\text{guess}} < T \;\; \Rightarrow\;\; \text{adjust } T^{\text{guess}}$;
              \item if $\summation[x_{i}]{i}{} < 1 \;\;\Rightarrow \;\; T^{\text{guess}} > T \;\; \Rightarrow\;\; \text{adjust } T^{\text{guess}}$.
           \end{itemize}
    \end{enumerate}
    Thus, for this problem,
    \begin{displaymath}
        \summation[x_{i}]{i}{} = 1\;\; \Longleftrightarrow \;\; \frc{y_{5}P}{P_{5}^{\text{sat}}} + \frc{y_{6}P}{P_{6}^{\text{sat}}} + \frc{y_{7}P}{P_{7}^{\text{sat}}} = 1,
    \end{displaymath}
    where $P_{i}^{\text{sat}}$ is a function of $T$, 
    \begin{displaymath}
        \frc{y_{5}P}{\exp{\left(A_{5}-\frc{B_{5}}{RT}\right)}} + \frc{x_{6}P}{\exp{\left(A_{6}-\frc{B_{6}}{RT}\right)}} + \frc{x_{7}P}{\exp{\left(A_{7}-\frc{B_{7}}{RT}\right)}}= 1
    \end{displaymath}
    Using $T^{\text{guess}}=298.15$ K as initial guess, $T=350.5857$ K (dew point temperature). Now, calculating the liquid composition, $y_{i}=\frc{x_{i}P_{i}^{\text{sat}}}{P}$ leads to,
    \begin{displaymath}
        x_{5} =0.0742\;\; x_{6} = 0.3463\;\;\text{ and }\;\;x_{7} = 0.5795\;\;\Rightarrow\;\; \summation[x_{i}]{i}{} = 1.0000
    \end{displaymath}


\clearpage
 % 
 % 
%%%
%%% EXAMPLE
%%%
   \item\label{Mod04Ex04} Estimate the bubble and dew point pressures of a mixture of 25 mol-$\%$ of n-pentane, 45 mol-$\%$ of n-hexane and 30 mol-$\%$ of n-heptane at 73$^{\circ}$C. Given $P^{\text{sat}}$ relation,
    \begin{displaymath}
      \ln{P_{i}^{\text{sat}}} = A_{i} - \frc{B_{i}}{RT}
    \end{displaymath}
    with $A_{nC_{5}}=10.422$, $A_{nC_{6}}=10.456$, $A_{nC_{7}}=11.431$, $B_{nC_{5}}=26799 \text{ J.mol}^{-1}$, $B_{nC_{6}}=29676 \text{ J.mol}^{-1}$ and $B_{nC_{7}}=35200 \text{ J.mol}^{-1}$. Also [$P$] = bar and [$T$] = K.

% SOLUTION
 \noindent{\bf Solution:}  For the bubble point, the mixture follows Raoult's law at 73$^{\circ}$C,
      \begin{displaymath}
          P = \summation[x_{i}P_{i}^{\text{sat}}]{i}{} = x_{5}P_{5}^{\text{sat}} + x_{6}P_{6}^{\text{sat}} + x_{7}P_{7}^{\text{sat}} = 1.413\text{ bar}
      \end{displaymath}
      This is the bubble point pressure at 346.15 K. The composition of the vapour phase is,
      \begin{displaymath}
             y_{i}=\frc{x_{i}P_{i}^{\text{sat}}}{P}
             \begin{cases}
                  y_{5} = 0.5370,\\
                  y_{6} = 0.3680,\\
                  y_{7} = 0.0950,
             \end{cases}
      \end{displaymath}
      leading to $\summation[y_{i}]{i}{} = 1.0000$.

\medskip

     For the dew point, the procedure is:
    \begin{enumerate}[i)]
       \item choose an initial guess for the dew point pressure, $P^{\text{guess}}$;
       \item calculate $x_{i}=\frc{Py_{i}}{P_{i}^{\text{sat}}}$;
       \item then:
           \begin{itemize}
              \item if $\summation[x_{i}]{i}{} = 1 \;\;\Rightarrow \;\; P^{\text{guess}} = P$;
              \item if $\summation[x_{i}]{i}{} > 1 \;\;\Rightarrow \;\; P^{\text{guess}} > P \;\; \Rightarrow\;\; \text{adjust } P^{\text{guess}}$;
              \item if $\summation[x_{i}]{i}{} < 1 \;\;\Rightarrow \;\; P^{\text{guess}} < P \;\; \Rightarrow\;\; \text{adjust } P^{\text{guess}}$.
           \end{itemize}
    \end{enumerate}
    Thus, for this problem,
    \begin{displaymath}
        \summation[x_{i}]{i}{} = 1\;\; \Longleftrightarrow \;\; \frc{y_{5}P}{P_{5}^{\text{sat}}} + \frc{y_{6}P}{P_{6}^{\text{sat}}} + \frc{y_{7}P}{P_{7}^{\text{sat}}} = 1,
    \end{displaymath}
    where $P_{i}^{\text{sat}}$ is a function of $T$ (which is known in this problem!).  Solving this equation leads to $P=0.8774$ bar (dew point pressure) with composition
    \begin{displaymath}
        x_{5} =0.0723\;\; x_{6} = 0.3418\;\;\text{ and }\;\;x_{7} = 0.5859\;\;\Rightarrow\;\; \summation[x_{i}]{i}{} = 1.0000
    \end{displaymath}


\clearpage
 % 
 % 
%%%
%%% EXAMPLE
%%%
   \item\label{Mod04Ex045} A liquid mixture of 25 mol-$\%$ n-pentane $\left(nC_{5}\right)$, 45 mol-$\%$ n-hexane $\left(nC_{6}\right)$ and 30 mol-$\%$ n-heptane $\left(nC_{7}\right)$ initially at 69$^{\circ}$C and a high pressure, is partially vaporised by isothermically lowering the pressure to  1.013 bar. Calculate the relative amounts of vapour and liquid in equilibrium and compositions.

% SOLUTION
 \noindent{\bf Solution:} This is a typical {\it flash} problem described by the figure,
    
  \begin{figure}[h]
     \begin{center}
         \includegraphics[width=.25\linewidth,clip]{./Figs/FlashDistillation}
     \end{center}
     \caption{Schematic of a flash distillation process, Example~\ref{Mod04Ex045}.}
  \end{figure}

From the given $P_{i}^{\text{sat}}$ relation at 69$^{\circ}$C,
   \begin{displaymath}
        \begin{cases}
            P_{5}^{\text{sat}} = 2.721 \text{ bar},\\ 
            P_{6}^{\text{sat}} = 1.024 \text{ bar},\\
            P_{7}^{\text{sat}} = 0.389 \text{ bar}.
        \end{cases}
   \end{displaymath}
Assuming ideal solution, $K_{i}$ for the three hydrocarbons can be readily obtained,
   \begin{displaymath}
        K_{i} = \frc{P_{i}^{\text{sat}}}{P} = \frc{y_{i}}{x_{i}},
          \begin{cases}
             K_{5} = 2.6861; \\
             K_{6} = 1.0109; \\
             K_{7} = 0.3840
          \end{cases}
   \end{displaymath}
   Therefore,
   \begin{displaymath}
       y_{5} = x_{5}K_{5},\;\;\;y_{6} = x_{6}K_{6},\;\;\;y_{7} = x_{7}K_{7},
   \end{displaymath}
    with the molar fraction constraints,
    \begin{displaymath}
        \begin{cases}
            x_{5}+x_{6}+x_{7} = 1,\\ 
            y_{5}+y_{6}+y_{7} = 1, \\
            F = V + L
        \end{cases}
   \end{displaymath}
   Assuming $F=1$, the mass balance of the individual components in equilibrium are
    \begin{displaymath}
        \begin{cases}
            x_{5}L + y_{5}V = z_{5} = 0.25 \\
            x_{6}L + y_{6}V = z_{6} = 0.45 \\
            x_{7}L + y_{7}V = z_{7} = 0.30            
        \end{cases}
   \end{displaymath}
   these relation can be rewritten as $\left(\text{based on }y_{5}+y_{6}+y_{7} = 1,\; y_{i}=K_{i}x_{i} \;\text{ and } V= 1-L\right)$
    \begin{displaymath}
        \begin{cases}
            x_{5}L + y_{5}V = z_{5} = 0.25 \;\Longrightarrow\; x_{5}\left[L\left(1-K_{5}\right)+K_{5}\right] = 0.25 \\
            x_{6}L + y_{6}V = z_{6} = 0.45 \;\Longrightarrow\; x_{6}\left[L\left(1-K_{6}\right)+K_{6}\right] = 0.45 \\
            x_{7}L + y_{7}V = z_{7} = 0.30 \;\Longrightarrow\; x_{7}\left[L\left(1-K_{7}\right)+K_{7}\right] = 0.30
        \end{cases}
   \end{displaymath}


   Now the molar fraction constraint of the liquid phase $x_{5}+x_{6}+x_{7} = 1$, is
   \begin{displaymath}
        \frc{0.25}{L\left(1-K_{5}\right)+K_{5}} + \frc{0.45}{L\left(1-K_{6}\right)+K_{6}} + \frc{0.30}{L\left(1-K_{7}\right)+K_{7}} = 1
   \end{displaymath}
   This expression has just one unknown, $L$, as a cubic polynomial. Solving with initial guess $L^{\text{guess}}=0.50$ leads to $L=0.5748$.  Now using 
    \begin{displaymath}
        \begin{cases}
            x_{5}L + y_{5}V = z_{5} = 0.25 \;\Longrightarrow\; x_{5}\left[L\left(1-K_{5}\right)+K_{5}\right] = 0.25  \;\Longrightarrow\;  x_{5} = 0.1456 \\
            x_{6}L + y_{6}V = z_{6} = 0.45 \;\Longrightarrow\; x_{6}\left[L\left(1-K_{6}\right)+K_{6}\right] = 0.45  \;\Longrightarrow\;  x_{6} = 0.4479 \\
            x_{7}L + y_{7}V = z_{7} = 0.30 \;\Longrightarrow\; x_{7}\left[L\left(1-K_{7}\right)+K_{7}\right] = 0.30  \;\Longrightarrow\;  x_{7} = 0.4065 
        \end{cases}
   \end{displaymath}
   leading to $\summation[x_{i}]{i}{} = 1.0000$. Molar fraction of the vapour phase is $V= 1-L=0.4252$, resulting in
    \begin{displaymath}
        \begin{cases}
            y_{5} = K_{5}x_{5} \;\Longrightarrow\; y_{5} = 0.3911 \\
            y_{6} = K_{6}x_{6} \;\Longrightarrow\; y_{6} = 0.4528 \\
            y_{7} = K_{7}x_{7} \;\Longrightarrow\; y_{7} = 0.1561 
        \end{cases}
   \end{displaymath}
    with $\summation[y_{i}]{i}{} = 1.0000$
 % 

%
\end{enumerate}


%\part{Liquid Solutions}
%  \chapter{Solution Thermodynamics}\label{Chapter:SolutionThermodynamics}


   \begin{LearningObjectivesBlock}{Learning Objectives}
      Upon completion of this chapter, you will be able to
        \begin{enumerate}
           \item Define fugacity and activity;
           \item Perform bubble, dew and flash calculations using the modified Raoult's law;
           \item Identify the requirements for an ideal solution;
           \item Apply activity models to the Raoult's law calculations;
           \item Derive expressions for the activity coefficient based on excess Gibbs energy;
           \item Describe ideal gas mixtures;
           \item Apply Henry's law to estimate fugacities of dilute components;
           \item Estimate fugacity coefficient through residual and excess properties.
        \end{enumerate}
\medskip
     Recommended reading: Chapters 11-12 of \citet{SmithVanNess_Book}, 9-10 of \cite{Sandler_Book}, 6 \citet{Lue_Book}, 11 of \citet{Elliot_Book}, 11-12 of \citet{Devoe_Book} or 5 of \citet{Atkins_Book}.
   \end{LearningObjectivesBlock}


%%%%%%%%%%%%%%%%%%%%%%%%%%%%%%%%%%%%%%%%%%%%%%%%%%%%%%%%%%%%%%%%%
\begin{comment}
   \begin{LearningObjectivesBlock}{Learning Objectives}
      Upon completion of this chapter, you will be able to
        \begin{enumerate}
           \item {\bf Knowledge:} Define, Name, Select, State 
           \item {\bf Comprehension:} Describe, Identify, Discuss
           \item {\bf Application:} Apply, Demonstrate, Employ, Sketch
           \item {\bf Analysis:} Analyse, Compare, Calculate, Solve
           \item {\bf Synthesis:} Determine, Formulate
           \item {\bf Evaluation:} Assess, Check, Estimate, Compare, Measure, Monitor
        \end{enumerate}
\end{comment}
%%%%%%%%%%%%%%%%%%%%%%%%%%%%%%%%%%%%%%%%%%%%%%%%%%%%%%%%%%%%%%%%%

%%%% ETOC
\localtableofcontents
   


%%% SUBSECTION
\section{Introduction}\label{Chapter:SolutionThermodynamics:Section:Introduction}
Vapour-liquid equilibrium of pure components and mixtures were the focus of study in Chapters~\ref{Chapter:VolumetricPropertiesPureSubstances}-\ref{Chapter:VLE}. In these chapters, equations of state were introduced to represent the $PVT$ behaviour of pure components in both vapour and liquid phases. From your readings, you may have noticed that EOS are largely applied, with relatively accuracy, to vapour phases, but rarely used to assess liquid phase behaviour. 

This is mainly due to the fact that EOS were originally (and cubic EOS in particular) to represent weak inter-molecular forces between gaseous molecules at relatively large distances (\ie {\it mean free path}). As pressure increases $\left(P\rightarrow \infty\right)$ and the distance between molecules tends to zero $\left(d\rightarrow 0\right)$, the accuracy of equations of state to predict $PVT$ behaviour also decreases, as the attractive and repulsive forces between molecules become stronger. 

Molecules in liquid phase are held together at close distances by strong attractive forces, however these forces are not strong enough to keep them in fixed positions (as inter-/intra-molecular forces in the solid phase) and the molecules are free to move past, slide over and collide with other molecules. A fraction of the molecules (mostly with higher energy) may be able to break such attractive forces and `escape' towards the vapour phase. 

Kinetic energy in gaseous molecules are larger than the attractive forces, leading to large distances between molecules and a tendency to occupy the whole volume of the system. The large kinetic energy in gaseous molecules also lead to continuous collisions between molecules and with the system (container) walls. During collisions, although total momentum is conserved, individual energies (and velocities) may decrease and molecules may move to a system with lower kinetic energy, \ie liquid phase.

\bigskip

In Chapter~\ref{Chapter:Introduction}, the concept of {\it mechanical} (\ie system at constant or uniform pressure) and {\it thermal} (from the Zeroth law) equilibrium were introduced. In order to define phase equilibria, we established (in Chapter~\ref{Chapter:ThermodynamicPropertiesPureFluids}) that an additional condition was necessary, {\it chemical equilibrium}, \ie equality of Gibbs free energy of all phases (Section~\ref{Chapter:ThermodynamicPropertiesPureFluids:Section:ClapeyronRelations}) for pure fluids. This concept can be extended for mixtures by using the definition of {\it chemical potential}, $\mu_{i}$ (Eqn.~\ref{Chapter:VLE:Eqn:ChemPotentialDef1b}),\index{Chemical potential}
         \begin{displaymath}
            \mu_{i} = \Partial[(nG)]{n_{i}}{T,P,n_{j\ne i}} \;\;\Longleftrightarrow\;\; \mu_{i} = \overline{G}_{i},
         \end{displaymath}
where $\overline{G}_{i}$ is the {\it partial molar Gibbs free energy}. This leads to the final phase equilibria condition in which {\it the chemical potential for each component at all coexisting phases are identical}, \ie 
         \begin{displaymath}
           \mfr[\mu]{i}{\alpha} = \mfr[\mu]{i}{\beta} = \cdots =  \mfr[\mu]{i}{j} = \cdots = \mfr[\mu]{i}{\mathcal{N}_{\mathcal{P}}}, \hspace{1.5cm} \forall i\in\{1,2,\cdots,\mathcal{C}\}\;\; \text{ and }\;\;\; \forall j\in\{\alpha,\beta,\cdots,\mathcal{N}_{\mathcal{P}}\}
         \end{displaymath}
         or for VLE systems:
         \begin{displaymath}
           \mfr[\mu]{i}{L} = \mfr[\mu]{i}{V},\;\;\;\forall i\in\{1,2,\cdots,\mathcal{C}\}
         \end{displaymath}

\medskip

The main assumptions for ideal gas behaviour assume that molecules of gas are massless particles with no interaction between them (\ie volume of the molecules is negligibly small compared with the volume occupied by the gas or simply the distance between gaseous molecules are often infinitely large, $d\rightarrow \infty$), except during elastic collisions over negligible duration. These conditions can only be met by fluids at sufficiently high temperature and relatively low pressure when large distances between molecules and high velocity overcome any interaction. The concept of ideality for liquids also exists and is based on the assumption that inter-molecular forces are the same between all molecules in solution. This assumption may be correct for pure non-polar substances (\ie in solutions with no major inter-/intra-molecular forces).

Thus, the aim of this Chapter is to define thermodynamic properties of real solutions based on properties of ideal solutions. Excess properties (Section~\ref{Chapter:VLE:Section:ExcessProperties}) plays a role similar to that of residual properties (Section~\ref{Chapter:ThermodynamicPropertiesPureFluids:Section:ResidualProperties}) for gases. In order to study solution properties at low to moderate pressures, {\it activity coefficient}, based on the definition of {\it fugacity}, is used. The {\it activity coefficient}, derived from the {\it excess Gibbs free energy}, is a measure of the extent of non-ideality of a real solution and helps describing phase equilibria of mixtures at low and moderate pressures.


%%% SECTION
\section{Fugacity}\label{Chapter:SolutionThermodynamics:Section:FugacitySection}\index{Fugacity}\index{Gibbs free energy!Partial molar}
\begin{subequations}
  A formal definition for {\it partial molar property} was introduced in Section~\ref{Chapter:VLE:Section:PartialMolarProperties} for an arbitrary extensive function $M$, Eqn.~\ref{Chapter:VLE:Eqn:PartialProperties2}. Such formulation was applied to the Gibbs free energy of a system with $n=n_{1}+n_{2}+\cdots+n_{\mathcal{C}}$ moles leading to the definition of the {\it chemical potential}, Eqns.~\ref{Chapter:VLE:Eqn:ChemPotentialDef1}-\ref{Chapter:VLE:Eqn:ChemPotentialDef1b},
  \begin{shaded}
     \begin{eqnarray}
       && dG = \Partial[G]{P}{T,n}dP + \Partial[G]{T}{P,n}dT + \underbrace{\Partial[G]{n}{T,P}}_{\mu}dn, \nonumber \\
       && \mu_{i} = \Partial[(nG)]{n_{i}}{T,P,n_{j\ne i}} \;\;\Longleftrightarrow\;\; \mu_{i} = \overline{G}_{i}, \nonumber
     \end{eqnarray}
  \end{shaded}
  \noindent For pure components,
  \begin{displaymath}
    G = n\mu \;\;\rightarrow \mu=\frc{G}{n}=\overline{g},
  \end{displaymath}
  where $\overline{g}$ is the molar Gibbs free energy. Using the Maxwell relation, Eqn.~\ref{Chapter:ThermodynamicPropertiesPureFluids:Eqn:MaxwellRelation7},
  \begin{eqnarray}
    && \Partial[G]{P}{T} = V = \Partial[n\mu]{P}{T} = n\Partial[\mu]{P}{T} \nonumber \\
    && \Partial[\mu]{P}{T} = \frc{V}{n} = \overline{v}\label{Chapter:SolutionThermodynamics:Eqn:fugacity1}
  \end{eqnarray}
  where $\underline{v}$ is the molar volume. For an ideal gas,
  \begin{displaymath}
     \Partial[\mu^{\text{ig}}]{P}{T} = \frc{RT}{P},
  \end{displaymath}
  where the superscript {\it ig} refers to ideal gas. This differential can be integrated as,
  \begin{equation}
    \mu^{\text{ig}} = RT\ln{P} + C(T),\label{Chapter:SolutionThermodynamics:Eqn:fugacity1a}
  \end{equation}
  where $C(T)$ is an integration constant. At limiting conditions, $P\rightarrow 0$ and $P\rightarrow\infty$, the chemical potential lies within $-\infty < \mu < +\infty$.
  \begin{shaded}
    \noindent For a fluid to be considered an ideal gas, pressure needs to be relatively low, thus we can use Eqn.~\ref{Chapter:SolutionThermodynamics:Eqn:fugacity1a} to represent the \underline{chemical potential of real fluids},
  \begin{equation}
    \mu = RT\ln{f} + C(T),\label{Chapter:SolutionThermodynamics:Eqn:fugacity1b}
  \end{equation}
  where $f$ is the {\it fugacity of a real gas}, \ie \blue{a representation of the pressure of an ideal gas that has the same chemical potential as the real gas.}
  \end{shaded}
  \noindent Since the {\it fugacity} has the \underline{units of pressure}, it is often referred as a `fictitious pressure'.  The definition of fugacity from Eqn.~\ref{Chapter:SolutionThermodynamics:Eqn:fugacity1b} is entirely general and can be readily extended to both liquids and solids. Now, replacing Eqn.~\ref{Chapter:SolutionThermodynamics:Eqn:fugacity1b} in the Maxwell relation, Eqn.~\ref{Chapter:ThermodynamicPropertiesPureFluids:Eqn:MaxwellRelation7}, at constant temperature,
  \begin{equation}
    RT\Partial[\left(\ln{f}\right)]{P}{T} = \overline{v}.\label{Chapter:SolutionThermodynamics:Eqn:fugacity1c}
  \end{equation}
  {\it Fugacity} can be obtained by integrating this equation holding $T$ constant. As pressure tends to zero (lower limit of the left-hand-side integration), the fluid behaves as an ideal gas. We can thus state that the fugacity of a pure component is equal to the pressure in the limit of zero pressure, \ie
      \begin{shaded}
        \begin{equation}
           \lim\limits_{P \rightarrow 0} \frc{f}{P} = 1.\label{Chapter:SolutionThermodynamics:Eqn:fugacity1d}
        \end{equation}
      \end{shaded}

  \noindent  In a \underline{mixture} containing $n$ moles of $\mathcal{C}$ chemical species, Eqn.~\ref{Chapter:SolutionThermodynamics:Eqn:fugacity1c} becomes,
      \begin{eqnarray}
        && RT\Partial[\left(\ln{f_{i}}\right)]{P}{T,n} = \overline{v}_{i},\;\;\;\;\forall i\in\left\{1,2,\cdots,\mathcal{C}\right\}\label{Chapter:SolutionThermodynamics:Eqn:fugacity1e1} \\
        && RT\Partial[\left(\ln{\overline{f}_{i}}\right)]{P}{T,n} = \overline{V}_{i},\label{Chapter:SolutionThermodynamics:Eqn:fugacity1e2}
      \end{eqnarray}
      where $f_{i}$ is the {\it fugacity of pure component $i$} and $\overline{f}_{i}$ is the {\it fugacity of component $i$ in the mixture}. $\overline{v}_{i}$ and $\overline{V}_{i}$ are the molar volume of pure component $i$ and, of component $i$ in the mixture. If we subtract Eqn.~\ref{Chapter:SolutionThermodynamics:Eqn:fugacity1e1} from \ref{Chapter:SolutionThermodynamics:Eqn:fugacity1e2} and integrate the result from $P'$ to $P$ at constant $T$,
      \begin{displaymath}
        RT\Partial[\left(\ln{\frc{\overline{f}_{i}}{f_{i}}}\right)]{P}{T,n} = \overline{V}_{i} - \overline{v}_{i} \;\;\Longrightarrow \;\;  RT\left[\left.\ln{\left(\frc{\overline{f}_{i}}{f_{i}}\right)}\right|_{P'}^{P}\right] = \int\limits_{P'}^{P}\left(\overline{V}_{i} - \overline{v}_{i}\right)dP
      \end{displaymath}
      Assuming that $P'$ tends to $0$,
      \begin{equation}
        RT\left[\ln{\left(\frc{\overline{f}_{i}}{f_{i}}\right)} - \lim\limits_{P'\rightarrow 0}\ln\left(\frc{\overline{f}_{i}}{f_{i}}\right)\right] = \int\limits_{0}^{P}\left(\overline{V}_{i} - \overline{v}_{i}\right)dP.\label{Chapter:SolutionThermodynamics:Eqn:fugacity1f}
      \end{equation}
      When $P'\rightarrow 0$
      \begin{displaymath}
        \begin{cases}
          f_{i} \rightarrow P', \\
          \overline{f}_{i} \rightarrow y_{i}P',
        \end{cases}
      \end{displaymath}
      and the limiting term in the previous equation becomes,
      \begin{displaymath}
        \lim\limits_{P'\rightarrow 0}\ln\left(\frc{\overline{f}_{i}}{f_{i}}\right) \rightarrow \ln{\left(\frc{y_{i}P'}{P'}\right)} = \ln{y_{i}}
      \end{displaymath}
      Thus Eqn.~\ref{Chapter:SolutionThermodynamics:Eqn:fugacity1f} becomes,
        \begin{displaymath}
          RT\left[\ln{\left(\frc{\overline{f}_{i}}{f_{i}}\right)} - \ln{y_{i}}\right] = \int\limits_{0}^{P}\left(\overline{V}_{i} - \overline{v}_{i}\right)dP
        \end{displaymath}
      \begin{shaded}
        \begin{equation}
           RT\ln{\left(\frc{\overline{f}_{i}}{y_{i}f_{i}}\right)} = \int\limits_{0}^{P}\left(\overline{V}_{i} - \overline{v}_{i}\right)dP. \label{Chapter:SolutionThermodynamics:Eqn:fugacity1g}
        \end{equation}
      \end{shaded}
  \noindent This expression relates fugacity of a pure component $i$ and the fugacity of the same component $i$ in a mixture with $\mathcal{C}$ chemical components during continuous change in pressure. The term in brackets in the left-hand side represents the ratio between partial pressures between real and ideal fluids in a mixture.

  \bigskip
        
      For an ideal mixture of ideal gases, the partial pressure of species $i$ is the fugacity of component $i$ in the mixture, \ie
         \begin{displaymath}
            \mfr[\overline{f}]{i}{V} = y_{i}P = P_{i}.
         \end{displaymath}
      Now, for an ideal mixture of real fluids,
         \begin{displaymath}
            \mfr[\overline{f}]{i}{L} = x_{i}\mfr[f]{i}{L},
         \end{displaymath}
      \ie the fugacity of a component in a mixture is a function of the fugacity of the pure component.
      \begin{shaded}
         An \underline{ideal solution} is a mixture in which,
        \begin{equation}
          \mfr[\overline{f}]{i}{V} = y_{i}P = P_{i}\;\;\;\text{ and }\;\;\; \mfr[\overline{f}]{i}{L} = x_{i}\mfr[f]{i}{L}\label{Chapter:SolutionThermodynamics:Eqn:fugacity2a}
        \end{equation}
        These expressions are called {\it Lewis-Randall rules} and they describe relations between phase compositions and fugacities.\index{Lewis-Randall rules|see {Solutions}}\index{Solutions!Lewis-Randall rules}
      \end{shaded}
      
\end{subequations}

      
%%% SECTION
\section{Activity and Activity Coefficient }\label{Chapter:SolutionThermodynamics:Section:ActivitySection}\index{Activity|see {Solutions}}\index{Solutions!Activity}\index{Activity coefficient|see {Solutions}}\index{Solutions!Activity coefficient}
   \begin{subequations}
      
      In a mixture with $\mathcal{C}$ chemical species in $n\;\left(= n_{1}+n_{2}+\cdots +n_{\mathcal{C}}\right)$ moles, the chemical potential of each component is (based on Eqn.~\ref{Chapter:SolutionThermodynamics:Eqn:fugacity1b}),
        \begin{displaymath}
           \mu_{i} = RT\ln{\overline{f}_{i}} + C_{i}(T).
        \end{displaymath}
      Let's consider a reference-state where a species $i$ of a multi-component system is pure at temperature $T$ and at reference-state pressure, $P_{\text{ref}}$,
        \begin{equation}
           \mu_{i} - \mu_{i}^{\circ} = RT\ln{\left(\frc{\overline{f}_{i}}{f_{i}^{\circ}}\right)},\label{Chapter:SolutionThermodynamics:Eqn:activity1a}
        \end{equation}
      where superscript $^{\circ}$ stands for {\it reference state}. 
        \begin{shaded}
           \noindent Now, we can define the term in brackets as the {\it activity} (dimensionless)
           \begin{equation}
              \frc{\overline{f}_{i}}{f_{i}^{\circ}} = a_{i}, \label{Chapter:SolutionThermodynamics:Eqn:activity1b}
           \end{equation}
           which `measures' the deviation from ideal behaviour.
        \end{shaded}
      \noindent Since $\mu^{\circ} = \overline{g}^{\circ}_{i}$, Eqn.~\ref{Chapter:SolutionThermodynamics:Eqn:activity1a} becomes
        \begin{equation}
           \mu_{i} = \overline{g}^{\circ}_{i} + RT\ln{a_{i}}\label{Chapter:SolutionThermodynamics:Eqn:activity1c}
        \end{equation}
      For an ideal solution, the {\it Lewis-Randall relation} can be used,\index{Solutions!Lewis-Randall rules}
        \begin{displaymath}
           a_{i} =  \frc{\overline{f}_{i}}{f_{i}^{\circ}} = \frc{y_{i}f_{i}}{f_{i}^{\circ}},
        \end{displaymath}
      where $f_{i}$ is the fugacity of pure component $i$ at $P$ and $T$, and $f_{i}^{\circ}$ is the fugacity of pure component $i$ at $T$ and the reference-pressure $P_{\text{ref}}$. Thus,
        \begin{displaymath}
           \mu_{i} = \overline{g}^{\circ}_{i} + RT\ln{\left( \frc{y_{i}f_{i}}{f_{i}^{\circ}}\right)}.
        \end{displaymath}

        \begin{shaded}
            \noindent For \underline{\it ideal solutions},
              \begin{equation}
                 \mu_{i} = \overline{g}^{\circ}_{i} + RT\ln{\left[\left(\frc{f_{i}}{P}\right)\left(\frc{P_{\text{ref}}}{f_{i}^{\circ}}\right)\frc{y_{i}P}{P_{\text{ref}}}\right]}.\label{Chapter:SolutionThermodynamics:Eqn:activity1d}
              \end{equation}
            If component $i$ behaves as an \underline{\it ideal gas} at both $\left(T,P\right)$ and $\left(T,P_{\text{ref}}\right)$, thus $\frc{f_{i}}{P} = \frc{f_{i}^{\circ}}{P_{\text{ref}}}=1$ and Eqn.~\ref{Chapter:SolutionThermodynamics:Eqn:activity1d} becomes
              \begin{equation}
                 \mu_{i} = \overline{g}^{\circ}_{i} + RT\ln{\frc{y_{i}P}{P_{\text{ref}}}}.\label{Chapter:SolutionThermodynamics:Eqn:activity1e}
              \end{equation}
             This expression allows the calculation of the chemical potential of the vapour phase in a mixture subjected to a change in pressure based on composition and temperature. {\it Molar Gibbs energy at reference conditions}, $\overline{g}^{\circ}_{i}$, for several chemical species are often tabulated and can be found in any chemical engineering handbook.
             
        \noindent Finally, we can now define the \underline{\blue{activity coefficient}}, $\gamma_{i}$, as
          \begin{equation}
              a_{i} = \frc{y_{i}f_{i}}{f_{i}^{\circ}}\;\;\;\;\Rightarrow \;\;\;\; \gamma_{i} = \frc{a_{i}}{y_{i}} = \frc{f_{i}}{f_{i}^{\circ}}\label{Chapter:SolutionThermodynamics:Eqn:activity1f}
          \end{equation}
         And for solutions,
          \begin{equation}
              \gamma_{i} = \frc{\overline{f}_{i}}{x_{i}f_{i}}\label{Chapter:SolutionThermodynamics:Eqn:activity1g}
          \end{equation}
         The {\it activity coefficient} takes into account non-idealities in the liquid phase. Now, we can properly define the \underline{Raoult's law} (Section~\ref{Chapter:VLE:Section:RaoultsLaw}) at low pressure,\index{Solutions!Raoult's law}
          \begin{equation}
               \gamma_{i} = \frc{\overline{f}_{i}}{x_{i}f_{i}} = \frc{y_{i}P}{x_{i}f_{i}}=\frc{y_{i}P}{x_{i}P_{i}^{\text{sat}}} \;\;\;\Longrightarrow \;\;\;\ x_{i}\gamma_{i}P_{i}^{\text{sat}} = y_{i}P \label{Chapter:SolutionThermodynamics:Eqn:RaoultLaw}
          \end{equation}
          Equation~\ref{Chapter:SolutionThermodynamics:Eqn:RaoultLaw} is the general form for Raoult's law. If the solution is ideal, $\gamma_{i}=1$, Eqn.~\ref{Chapter:VLE:Eqn:RaoultLaw} is recovered.

      \end{shaded}       
%
   \end{subequations}

   % Example
   \begin{MyExample}{\begin{center}{\bf Example}\end{center}}
     \begin{example}\label{Chapter:SolutionThermodynamics:Example1} %Johannes Problem 2 (Tutorial 5)
        A process stream contains light species 1 and heavy species 2. A relatively pure liquid stream containing mostly 2 is obtained through a single-stage liquid/vapour separator. Specifications on the equilibrium composition are: $x_{1}$ = 0.002 and $y_{1}$ = 0.950. Assuming that the modified Raoult's law applies, 
\begin{displaymath}
  y_{i} P = x_{i}\gamma_{i}P_{i}^{\text{sat}}
\end{displaymath} 
Determine $T$ and $P$ for the separator. Given the activity coefficients for the liquid phase,
\begin{displaymath}
\ln\gamma_{1} = 0.93x_{2}^{2} \;\;\;\;\;\text{ and }\;\;\;\;\;\ln\gamma_{2}=0.93x_{1}^{2}
\end{displaymath}
\begin{displaymath}
\ln P_{i}^{\text{sat}} = A_{i} - \frc{B_{i}}{T}\;\;\;\text{with [P] = bar and [T] = K}
\end{displaymath} 
$A_{1}$ =10.08, $B_{1}$ = 2572.0 K, $A_{2}$ = 11.63 and $B_{2}$ = 6254.0 K.
     \end{example}

% SOLUTION
     \noindent{\bf Solution:}
            This problem requires calculating $T$ and $P$ at the vapour-liquid equilibrium of a binary mixture whereas compositions of both phases are given,
            \begin{displaymath}
               \begin{cases}
                  x_{1} = 0.002 \Leftrightarrow x_{2} = 0.998,\\
                  y_{1} = 0.950 \Leftrightarrow y_{2} = 0.050.
               \end{cases}
            \end{displaymath}
            For this composition, we can also obtain the actitivity coefficient, $\gamma_{i}$,
            \begin{displaymath}
               \begin{cases}
                  \ln{\gamma_{1}} = 0.93x_{2}^{2} \Leftrightarrow \gamma_{1} = 2.5251,\\
                   ln{\gamma_{2}} = 0.93x_{1}^{2} \Leftrightarrow \gamma_{2} = 1.0000.
               \end{cases}
            \end{displaymath}
            Now, the current modified Raoult's law results in 2 non-linear equations (for both components in the mixture) and two unknowns ($T$ and $P$). These expresssions can be manipulated to be a function of only one unknown, $T$,
            \begin{eqnarray}
                y_{i} P &=& x_{i}\gamma_{i}P_{i}^{\text{sat}} \nonumber \\
                     P &=& \frc{x_{i}\gamma_{i}P_{i}^{\text{sat}}}{y_{i}} =  \frc{x_{1}\gamma_{1}P_{1}^{\text{sat}}}{y_{1}} = \frc{x_{2}\gamma_{2}P_{2}^{\text{sat}}}{y_{2}} \nonumber \\
                \frc{P_{1}^{\text{sat}}}{P_{2}^{\text{sat}}}  &=& \frc{x_{2}\gamma_{2}y_{1}}{x_{1}\gamma_{1}y_{2}} \nonumber \\
                \exp{\left[A_{1}-\frc{B_{1}}{T}-A_{2}+\frc{B_{2}}{T}\right]} &=& \underbrace{\frc{x_{2}\gamma_{2}y_{1}}{x_{1}\gamma_{1}y_{2}}}_{= 3754.7028} \nonumber \\
                         T&=& 376.4532\text{ K} \nonumber
            \end{eqnarray}
            Pressure can now be easily obtained from,
            \begin{displaymath}
                 P = \frc{x_{i}\gamma_{i}P_{i}^{\text{sat}}}{y_{i}} = 0.1368\text{ bar}.
            \end{displaymath}
   \end{MyExample} 


% Example
   \begin{MyExample}{\begin{center}{\bf Example}\end{center}}
     \begin{example}\label{Chapter:SolutionThermodynamics:Example2}\citep{SmithVanNess_Book}
          For the acetone (Ket) / methanol (MetOH) system, a vapour mixture of $z_{\text{Ket}}$ = 0.25 and $z_{\text{MetOH}}$ = 0.75 is cooled to temperature $T$ in the two-phase region and flows into a separation chamber at a pressure of 1 bar. If the composition of the liquid product is $x_{\text{Ket}}$ = 0.175, calculate $T$  and $y_{\text{Ket}}$. For liquid mixture, assume that
\begin{displaymath}
\ln\gamma_{1} = 0.64x_{2}^{2} \;\;\;\;\;\text{ and }\;\;\;\;\;\ln\gamma_{2}=0.64x_{1}^{2}
\end{displaymath}
For the Antoine equation, 
\begin{displaymath}
\ln P_{i}^{\text{sat}} = A_{i} - \frc{B_{i}}{T + C} \;\;\left(\text{ [P] = kPa and [T] = }^{\circ}\text{C}\right)
\end{displaymath}
$A_{\text{Ket}}$ = 14.3145, $B_{\text{Ket}}$ = 2756.22$^{\circ}$C, $C_{\text{Ket}}$ = 228.060$^{\circ}$C, $A_{\text{MetOH}}$ = 16.5785, $B_{\text{MetOH}}$ = 3638.27$^{\circ}$C, $C_{\text{MetOH}}$ = 239.50$^{\circ}$C.
     \end{example}

% SOLUTION
     \noindent{\bf Solution:}
            This is a typical flash problem for separation of acetone (1) and methanol (2). The feed stream contains 25$\%$-mol of acetone, \ie $z_{1}=0.25$, and after the separation at 1 bar, the resulting liquid stream contains 17.5$\%$-mol, \ie $x_{1}=0.175$. The problem requires temperature of the separation and composition of the gas stream, $y_{i}$. As $\gamma_{i}$ is given, we can consider that the liquid mixture is {\bf not ideal}, therefore the modified Raoult's law can be used,
\begin{displaymath}
    y_{i}P = x_{i}\gamma_{i}P_{i}^{\text{sat}} \;\;\;\longrightarrow y_{1}=\frc{x_{1}\gamma_{1}P_{1}^{\text{sat}}}{P}
\end{displaymath}
     \begin{center}
         \includegraphics[width=.25\linewidth,clip]{./Figs/FlashDistillation}
     \end{center}

 Using the material balance derived in Section~\ref{Chapter:VLE:Section:FlashDistillation},
\begin{eqnarray}
    && Fz_{1} = x_{1}L + y_{1}V\text{ and for } F=1 \text{ (\ie normalised stream fractions)}\;\longrightarrow F = L + V = 1 \nonumber \\
    && z_{1} = x_{1}L + \frc{x_{1}\gamma_{1}P_{1}^{\text{sat}}}{p}\left(1-L\right)\;(\star)\;\; \text{ with }\;\; P=\summation[P_{i}]{i}{} = y_{1}P + y_{2}P = x_{1}\gamma_{1}P_{1}^{\text{sat}} + x_{2}\gamma_{2}P_{2}^{\text{sat}}\;(\dagger),\nonumber
\end{eqnarray}
therefore, we have 2 equations and 2 unknowns, $L$ and $T$. Let's solve ($\dagger$) first for $T$ with $\gamma_{1}=1.5459$ and $\gamma_{2}=1.0198$
   \begin{displaymath}
       P = x_{1}\gamma_{1}P_{1}^{\text{sat}} + x_{2}\gamma_{2}P_{2}^{\text{sat}} = 100 \text{ kPa} \;\;\;\Rightarrow\;\;\; T = 59.5309\text{ K}
   \end{displaymath}
 With $T$, we can solve ($\star$) for $L$,
   \begin{eqnarray}
       && z_{1} = x_{1}L + \frc{x_{1}\gamma_{1}P_{1}^{\text{sat}}}{p}\left(1-L\right) \nonumber \\
       && L = \frc{z_{1} - \frac{x_{1}\gamma_{1}P_{1}^{\text{sat}}}{P}}{x_{1} - \frac{x_{1}\gamma_{1}P_{1}^{\text{sat}}}{P}} = 0.4306\;\;\rightarrow \;\; V = 0.5694 \nonumber \\
       && y_{1} = \frac{x_{1}\gamma_{1}P_{1}^{\text{sat}}}{P} = 0.3067\;\;\rightarrow \;\; y_{2} = 0.6933\nonumber
   \end{eqnarray}
   \end{MyExample} 

      
%%% SECTION
\section{Henry's Law}\label{Chapter:SolutionThermodynamics:Section:HenryLaw}\index{Solutions!Henry's law}
Now that the concepts of fugacity and activity coefficient were introduced, we can return to the special VLE condition in which the solute is very diluted in the mixture (Section~\ref{Chapter:VLE:Section:HenryLaw}). In such cases, the fugacity of a very dilute chemical species in a liquid mixture (\eg dissolved gases or solid of limited solubility) is experimentally found to be a linear function of its low composition (\ie mole fraction)\footnote{It is clear that the division in the limiting term,
  \begin{displaymath}
    \lim\limits_{x_{i}\rightarrow 0}\frc{\overline{f_{i}}}{x_{i}}
  \end{displaymath}
  is undetermined as $f_{i}=0\text{ when }x_{i}\rightarrow 0$, \ie at infinite dilution of a solute $i$ in solution, the fugacity of this component is null. In order to solve this mathematical problem, we can use the L'H\^opital rule (see Appendix~\ref{Appendix:lHopital}) which states that for a limit,
         \begin{displaymath}
                 \lim\limits_{x\rightarrow a}\frc{f(x)}{g(x)},
         \end{displaymath}
         if the numerator and denominator are finite at $a$, and $g(a)=0$, then
         \begin{displaymath}
                 \lim\limits_{x\rightarrow a}\frc{f(x)}{g(x)} =  \lim\limits_{x\rightarrow a}\frc{f'(a)}{g'(a)}.
         \end{displaymath}
         Therefore, the limiting dilution case can be solved as
         \begin{displaymath}
                 \lim\limits_{x_{i}\rightarrow 0}\frc{\overline{f_{i}}}{x_{i}} = \left(\frc{d \overline{f_{i}}}{d x_{i}}\right)_{x_{i}=0}.
         \end{displaymath}
}, \ie   
      \begin{eqnarray}
        && \lim\limits_{x_{i}\rightarrow 0}\frc{\overline{f_{i}}}{x_{i}} = \left(\frc{d \overline{f_{i}}}{d x_{i}}\right)_{x_{i}=0}  \equiv \mathcal{H}_{i}, \nonumber \\
        && \mfr[\overline{f}]{i}{L}(T,P,x) = x_{i}\mathcal{H}_{i}(T,P)\;\;\;\;\text{ as } x_{i}\rightarrow 0.
      \end{eqnarray}      
This proportionality constant, $\mathcal{H}$, is called {\it Henry's law constant}, and is dependent on the solute-solvent pair, temperature and pressure. This experimentally-defined constant was first introduced in Eqn.~\ref{Chapter:VLE:Eqn:HenryLaw} as, in equilibrium conditions (at $P$, $T$ and $n$ constants), $\mfr[\overline{f}]{i}{V}=\mfr[\overline{f}]{i}{L}$,
         \begin{shaded}
           \begin{displaymath}
             P_{i} = y_{i}P = x_{i}\mathcal{H}_{i},
           \end{displaymath}
           with $\mfr[\overline{f}]{i}{V}=y_{i}P$.
        \end{shaded}


%
%%%%%
%%%%%  COMMENTS
%%%%%
\begin{comment}      

\begin{table}[h]
   \begin{center}
      \begin{tabular}{l c l l l}
          \hline
             {\bf Components}   & {\bf Basis}  & {\bf Standard-state}   & {\bf Activity}        & {\bf Limits} \\
          \hdashline
             Solid or liquid    &              &  Pure                  & $a=1$                 &              \\
          \hdashline
             Solvent            &   Raoult     &  Pure solvent          & $a=\frc{P}{P_{\text{ref}}}$,  & $\gamma\rightarrow 1\text{ as } x\rightarrow 1$ \\
                                &              &                        & $a=\gamma x$          &  (pure solvent) \\
          \hdashline
             Solute             & Henry        & (1) hypothetical state of& $a=\frc{P}{\mathcal{H}}$,& $\gamma\rightarrow 1\text{ as } x\rightarrow 0$ \\
                                &              & pure solute            & $a=\gamma x$          &                                     \\
          \hdashline
                                &              & (2) hypothetical state of& $a=\gamma\frc{m}{m^{\circ}}$& $\gamma\rightarrow 1\text{ as } m\rightarrow 0$ \\
                                &              & solute at molality $m$   &                     &                                     \\
         \hline 
       \end{tabular}
       \caption{Standard states for activity \citep[extracted from][]{Atkins_Book}.}\label{Chapter:SolutionThermodynamics:Table:StandarStateActivity}
   \end{center}
\end{table}
\end{comment}

%%% SECTION
\section{Gibbs-Duhem Equation}\label{Chapter:SolutionThermodynamics:Section:GibbsDuhem}\index{Gibbs-Duhem equation|see {Solutions}}\index{Solutions!Gibbs-Duhem equation}
   \begin{subequations}
%
       In Section~\ref{Chapter:VLE:Section:PartialMolarProperties}, the concept of \underline{\it partial molar property} of species $i$ in solution was defined as\index{Partial molar properties}
         \begin{displaymath}
            \overline{M}_{i} = \Partial[(nM)]{n_{i}}{T,P,n_{j\ne i}},
         \end{displaymath}
        \ie the change of the total property $nM$ of a mixture of $\mathcal{C}$ species resulting from the addition at constant $T$ and $P$ of infinitesimal amount of species $i$ to a prescribed amount of solution.  It is important to bear in mind that {\it partial molar property} of a substance is different from {\it molar property} of the same substance in a pure state at the same $T$ and $P$. This is due to inter-/intra-molecular forces that imposes interactions between species in solution.
\medskip

      Equation~\ref{Chapter:VLE:Eqn:PartialProperties2} is a general expression for the change of any extensive property, $M$
         \begin{displaymath}
            d(nM) = n\Partial[M]{P}{T,x}dP + n\Partial[M]{T}{P,x}dT + \summation[\overline{M}_{i}dn_{i}]{i}{}.
         \end{displaymath}
      Mole fraction of a species $i$ was also introduced in Section~\ref{Chapter:VLE:Section:Compositions} as,
         \begin{displaymath}
            x_{i} = \frc{n_{i}}{n} \;\;\Rightarrow\;\; n_{i} = x_{i}n,
         \end{displaymath}
      Applying {\it derivative of a product} rule (Appendix~\ref{Appendix_Calculus:Section:BasicDerivationIntegration}) on $n_{i}$,
         \begin{displaymath}
             dn_{i} = x_{i}dn + n dx_{i},
         \end{displaymath}
      and for the total property $(nM)$,
         \begin{displaymath}
             d(nM) = n d M + M dn.
         \end{displaymath}
      Replacing these expressions in Eqn.~\ref{Chapter:VLE:Eqn:PartialProperties2},
        \begin{displaymath}
           n d M + M d n = n \Partial[M]{T}{P,x}dT + n\Partial[M]{P}{T,x}dP + \summation[\overline{M}_{i}\left(x_{i}dn + ndx_{i}\right)]{i}{}
        \end{displaymath}
      Rearranging this expression,
        \begin{equation}
           \underbrace{\left[dM - \Partial[M]{T}{P,x}dT - \Partial[M]{P}{T,x}dP - \summation[\overline{M}_{i}dx_{i}]{i}{}\right]}_{A}n + \underbrace{\left[M-\summation[\overline{M}_{i}x_{i}]{i}{}\right]}_{B}dn = 0.\label{Chapter:SolutionThermodynamics:Eqn:GibbsDuhem1a}
        \end{equation}
      Equation~\ref{Chapter:SolutionThermodynamics:Eqn:GibbsDuhem1a} is valid for any arbitrary values of $n$ and $dn$, therefore $n$ and $dn$ should be independent of each other. Therefore, this expression is only true if the coefficients $A$ and $B$ are zero, \ie
          \begin{eqnarray}
              A &=& dM - \Partial[M]{T}{P,x}dT - \Partial[M]{P}{T,x}dP - \summation[\overline{M}_{i}dx_{i}]{i}{} = 0 \nonumber \\
              dM &=& \Partial[M]{T}{P,x}dT + \Partial[M]{P}{T,x}dP + \summation[\overline{M}_{i}dx_{i}]{i}{}, \label{Chapter:SolutionThermodynamics:Eqn:GibbsDuhem1b}
          \end{eqnarray}
      and
          \begin{eqnarray}
              B &=& M-\summation[\overline{M}_{i}x_{i}]{i}{} = 0 \nonumber \\
              M &=& \summation[\overline{M}_{i}x_{i}]{i}{}\label{Chapter:SolutionThermodynamics:Eqn:GibbsDuhem1c} \\
             \text{as } nM = n\summation[\overline{M}_{i}x_{i}]{i}{} &\Rightarrow& dM = \summation[x_{i}d \overline{M}_{i}]{i}{} + \summation[\overline{M}_{i}dx_{i}]{i}{}\label{Chapter:SolutionThermodynamics:Eqn:GibbsDuhem1d}
          \end{eqnarray}
      Replacing Eqn.~\ref{Chapter:SolutionThermodynamics:Eqn:GibbsDuhem1d} in Eqn.~\ref{Chapter:SolutionThermodynamics:Eqn:GibbsDuhem1b} leads to the \blue{\it Gibbs-Duhem Equation (GDE)},
          \begin{shaded}
             \begin{equation}
                 \Partial[M]{T}{P,x}dT + \Partial[M]{P}{T,x}dP - \summation[x_{i}d\overline{M}_{i}]{i}{} = 0.\label{Chapter:SolutionThermodynamics:Eqn:GibbsDuhem1e}
             \end{equation}
             The \blue{GDE} must be satisfied for all changes in $P$, $T$ and $\overline{M}_{i}$. For changes at constant temperature and pressure conditions, the GDE becomes
               \begin{equation}
                   \summation[x_{i}d\overline{M}_{i}]{i}{} = 0.\label{Chapter:SolutionThermodynamics:Eqn:GibbsDuhem1f}
               \end{equation}
             Or taking any arbitrary species $j$,
               \begin{equation}
                   \summation[x_{i}\frc{d\overline{M}_{i}}{dx_{j}}]{i}{} = 0, \;\;\;\text{ at constant } T\text{ and } P.\label{Chapter:SolutionThermodynamics:Eqn:GibbsDuhem1g}
               \end{equation}
             Equations~\ref{Chapter:SolutionThermodynamics:Eqn:GibbsDuhem1f}-\ref{Chapter:SolutionThermodynamics:Eqn:GibbsDuhem1g} imply that partial molar properties of various species, $\overline{M}_{i}$, are \underline{not} independent. 

             \noindent Some key-properties,
               \begin{displaymath}
                   \begin{cases}
                       \lim\limits_{x_{i}\rightarrow 0}\overline{M}_{i} = \overline{M}_{i}^{\infty} & \text{\ie at infinite dilutions;} \\
                       \lim\limits_{x_{i}\rightarrow 1}\overline{M}_{i} = M_{i}                   & \text{ \ie nearly pure chemical species.} 
                   \end{cases}
               \end{displaymath}
          \end{shaded}
%
   \end{subequations}


% Example
   \begin{MyExample}{\begin{center}{\bf Example}\end{center}}
     \begin{example}\label{Chapter:SolutionThermodynamics:Example4} %http://people.cst.cmich.edu/teckl1mm/PChemI/Chm351Ch7aF01.htm
          The experimental value of the partial molar volume $\left(\text{cm}^{3}\text{.mol}^{-1}\right)$ of a aqueous solution of K$_{2}$SO$_{4}$ is given by
                \begin{displaymath}
                   \overline{V}_{A} = 32.280 + 18.216 m^{1/2},
                \end{displaymath} 
 where $m$ is the molality (= number of moles per kg of water) of the K$_{2}$SO$_{4}$ . Use the Gibbs-Duhem equation to derive an equation for the partial molar volume of water in the solution. Plot $\overline{V}_{i}\;\;\times m$, with $0.0\leq m\leq 0.1$. The molar volume of pure water at 298.15 K is 18.079 cm$^{3}$.mol$^{-1}$ and the molar mass of pure water is 18 g.mol$^{-1}$.
     \end{example}

% SOLUTION
     \noindent{\bf Solution:}
           For an easy notation, let's take K$_{2}$SO$_{4}$: $A$ and Water: $w$.
\begin{itemize}
  \item The Gibbs-Duhem equation for the partial molar volumes can be expressed as,
     \begin{displaymath}
        \sum\limits x_{i}d\overline{V}_{i} = x_{A} d\overline{V}_{A} + x_{w} d\overline{V}_{w} = 0,
     \end{displaymath}
     that after rearranging and integrating,
     \begin{displaymath}
        \int\limits_{\overline{V}_{w}=V_{w}^{\circ}}^{\overline{V}_{w}}d\overline{V}_{w} = \overline{V}_{w} - V_{w}^{\circ} = -\int\frc{x_{A}}{x_{w}} d\overline{V}_{A},
     \end{displaymath}
     where $V_{w}^{\circ}$ is the molar volume of pure water at reference conditions (\ie at $T=$ 25$^{\circ}$C).

  \item We can change the integration variable from $\overline{V}_{A}$ to $m$,
      \begin{displaymath}
         \overline{V}_{A} = 32.280 + 18.216 m^{1/2} \rightarrow \frc{d\overline{V}_{A}}{dm} = \frc{1}{2}\cdot 18.216 m^{-1/2} 
      \end{displaymath} 
  
  \item Now, replacing it into the integral above,
       \begin{eqnarray}
          && \overline{V}_{w} - V_{w}^{\circ} = -\int\frc{x_{A}}{x_{w}} \frc{1}{2}\cdot 18.216 m^{-1/2} dm \hspace{1cm} \red{\cdot\frc{n}{n}} \nonumber \\
          && \overline{V}_{w} - V_{w}^{\circ} = -\int\frc{n_{A}}{n_{w}} \frc{1}{2}\cdot 18.216 m^{-1/2} dm, \nonumber
       \end{eqnarray}
       where $n$ is the total number of moles in the solution.

  \item From the definition of molality, i.e., number of moles of K$_{2}$SO$_{4}$ per kg of water,
       \begin{displaymath}
           m = \frc{n_{A}}{n_{w}M_{w}} \Longrightarrow m\cdot M_{w} = \frc{n_{A}}{n_{w}},
       \end{displaymath}
       where $n_{A}$, $n_{w}$ are the number of moles of K$_{2}$SO$_{4}$ and water, respectively. $M_{w}$ is the molar mass of water.

  \item Replacing in the equation above, 
      \begin{eqnarray}
         \overline{V}_{w} - V_{w}^{\circ} &=& -\int\limits_{0}^{m} m\cdot M_{w} \frc{1}{2}\cdot 18.216 m^{-1/2} dm \nonumber \\
                                       &=& -9.108\; M_{w} \left. \frc{2}{3} m^{3/2}\right|_{0}^{m} \nonumber \\
         \blue{\left(\text{where }\right.}&& \blue{\left. M_{w} = 18 \frc{\text{g}}{\text{mol}} = 0.018 \frc{\text{kg}}{\text{mol}}\right)} \nonumber \\
         \overline{V}_{w} &=&  18.079 - 0.1093 m^{3/2} \nonumber 
      \end{eqnarray}

  \item In order to plot $\overline{V}_{i}\;\;\times m$, with $0.0\leq m\leq 0.1$, we can use the defined equations above to set  up the table:  
        \begin{center}
          \begin{tabular}{|c | c c |}
            \hline
            ${\bf m (mol/kg)}$ & ${\bf \overline{V}_{A}}$ & ${\bf \overline{V}_{w}}$ \\
            \hline
              0.00  & 32.2800 &  18.0790 \\
              0.05  & 36.6353 &  18.0778 \\
              0.10  & 38.0404 &  18.0755 \\
             \hline
          \end{tabular}
        \end{center}
          \begin{center}
              \includegraphics[width=0.75\columnwidth,clip]{./Pics/Example04_Pic}
          \end{center}
\end{itemize}

   \end{MyExample} 

      
%%% SECTION
\section{Partial Molar Properties in Binary Solutions}\label{Chapter:SolutionThermodynamics:PMP_Binary}\index{Partial molar properties}
   \begin{subequations}
%
       In a mixture with $n$ moles, we can define the {\it isothermal molar property change of mixing}, $\Delta M_{\text{mix}}$ as,\index{Molar property change of mixing!Definition}
          \begin{eqnarray}
              \Delta M_{\text{mix}}(T,P) & = & M(T,P) - \summation[x_{i}M_{i}(T,P)]{i}{} \nonumber \\
                                      &=& \underbrace{\summation[\overline{M}_{i}x_{i}]{i}{}}_{\text{from Eqn.~\ref{Chapter:SolutionThermodynamics:Eqn:GibbsDuhem1c}}} - \summation[x_{i}M_{i}]{i}{} \nonumber \\
                                      &=& \summation[x_{i}\left(\overline{M}_{i}-M_{i}\right)]{i}{}.\label{Chapter:SolutionThermodynamics:Eqn:GibbsDuhem1h}
          \end{eqnarray}
       \begin{shaded}
          \noindent In a \underline{binary system} (from Eqn.~\ref{Chapter:SolutionThermodynamics:Eqn:GibbsDuhem1c}),
          \begin{equation}
              M = \overline{M}_{1} x_{1} + \overline{M}_{2} x_{2},\;\;\;\;\;\left(\text{at constant }P \text{ and } T\right),\label{Chapter:SolutionThermodynamics:Eqn:GibbsDuhem1i}
          \end{equation}
       \end{shaded}
       \noindent and using the {\it derivative of a product} rule on $M$,
          \begin{displaymath}
              dM = \left(x_{1}d\overline{M}_{1} + \overline{M}_{1}dx_{1}\right) + \left(x_{2}d\overline{M}_{2} + \overline{M}_{2}dx_{2}\right), 
          \end{displaymath}
      applying GDE (Eqn.~\ref{Chapter:SolutionThermodynamics:Eqn:GibbsDuhem1f}) in this expression, and using 
          \begin{displaymath}
             \summation[x_{i}]{i}{}=1\;\;\Leftrightarrow\;\; x_{1}+x_{2}=1\;\;\Rightarrow\;\;\; dx_{1}=-dx_{2},
          \end{displaymath}
      leads to 
          \begin{eqnarray}
             && \cancel{\red{x_{1}d\overline{M}_{1}}} + \overline{M}_{1}dx_{1} + \cancelto{\red{=0 \text{ (due to GDE, Eqn.~\ref{Chapter:SolutionThermodynamics:Eqn:GibbsDuhem1f})}}}{\red{x_{2}d\overline{M}_{2}}} - \overline{M}_{2}dx_{1} = dM \nonumber \\
             && dM = \overline{M}_{1}dx_{1} - \overline{M}_{2}dx_{1} \nonumber \\
             && \frc{dM}{dx_{1}} = \overline{M}_{1} - \overline{M}_{2} \label{Chapter:SolutionThermodynamics:Eqn:GibbsDuhem1j}
          \end{eqnarray}
          \begin{shaded}
             Using Eqn.~\ref{Chapter:SolutionThermodynamics:Eqn:GibbsDuhem1i} in Eqn.~\ref{Chapter:SolutionThermodynamics:Eqn:GibbsDuhem1j},
               \begin{eqnarray}
                  && \frc{dM}{dx_{1}} = \overline{M}_{1} - \left(\frc{M-\overline{M}_{1}x_{1}}{x_{2}}\right) \nonumber \\
                 && x_{2}\frc{dM}{dx_{1}} =  \underbrace{x_{2}\overline{M}_{1}}_{\left(1-x_{1}\right)\overline{M}_{1}} - M + \overline{M}_{1}x_{1} \nonumber
               \end{eqnarray}
               \begin{equation}
                 \begin{cases}
                    \overline{M}_{1} = M + x_{2}\frc{dM}{dx_{1}},\\% \label{Chapter:SolutionThermodynamics:Eqn:GibbsDuhem1k} \\
                    \overline{M}_{2} = M - x_{1}\frc{dM}{dx_{1}} \label{Chapter:SolutionThermodynamics:Eqn:GibbsDuhem1k}
                 \end{cases}
              \end{equation}
            In binary solutions, the partial molar properties can be easily calculated from these two  expressions as a function of composition at constant $T$ and $P$. %From the GDE,
              \begin{eqnarray}
                   && x_{1}d\overline{M}_{1} + x_{2}d\overline{M}_{2} = 0 \;\;\;\;\;\blue{\times\frc{1}{dx_{1}}} \nonumber \\
                   && x_{1}\frc{d\overline{M}_{1}}{dx_{1}} + x_{2}\frc{d\overline{M}_{2}}{dx_{1}} = 0 \;\;\;\;\;\blue{\times\frc{1}{x_{1}}}\nonumber \\ %\label{Chapter:SolutionThermodynamics:Eqn:GibbsDuhem1l} \\
                   && \frc{d\overline{M}_{1}}{dx_{1}} + \frc{x_{2}}{x_{1}}\frc{d\overline{M}_{2}}{dx_{1}} = 0. \label{Chapter:SolutionThermodynamics:Eqn:GibbsDuhem1l} 
              \end{eqnarray}
          \end{shaded}
 %
   \end{subequations}



% Example
   \begin{MyExample}{\begin{center}{\bf Example}\end{center}}
     \begin{example}\label{Chapter:SolutionThermodynamics:Example3}
           At 25$^{\circ}$C and atmospheric pressure, the volume change of mixing of a binary liquid mixture of species 1 and 2 is given by,
       \begin{displaymath}
          \Delta V = x_{1}x_{2}\left(45x_{1}+25x_{2}\right)\;\;\;\;\text{ with }\;\;\;\left[\Delta V\right] = \text{cm}^{3}.\text{mol}^{-1}
       \end{displaymath}
       with molar volumes of $V_{1} = 110\;\text{cm}^{3}.\text{mol}^{-1}$ and $V_{2} = 90\;\text{cm}^{3}.\text{mol}^{-1}$. Determine the partial molar volumes, $\overline{V}_{1}$ and $\overline{V}_{2}$, in a mixture containing 40$\%$-mol of species 1.
     \end{example}

% SOLUTION
     \noindent{\bf Solution:} 
          The mixture contains 40$\%$-mol of species 1, \ie $x_{1}=0.40$ and $x_{2}=1-x_{1}=0.60$. The partial molar volumes, $\overline{V}_{i}$, can be obtained from Eqn.~\ref{Chapter:SolutionThermodynamics:Eqn:GibbsDuhem1k}, 
               \begin{displaymath}
                 \begin{cases}
                    \overline{M}_{1} = M + x_{2}\frc{dM}{dx_{1}},\\
                    \overline{M}_{2} = M - x_{1}\frc{dM}{dx_{1}},
                 \end{cases}
              \end{displaymath}
              with $M=V$. Therefore the first step is to calculate the molar volume of the solution, $V$,
                 \begin{displaymath}
                    V^{E} = V - V^{\text{id}} = V - \summation[x_{i}V_{i}]{i=1}{2} = V - \underbrace{\left(x_{1}V_{1} + x_{2}V_{2}\right)}_{98\text{ cm}^{3}.\text{mol}^{-1}},
                 \end{displaymath}
              with the excess volume, $V^{E}=\Delta V$, obtained from the given expression,
                 \begin{displaymath}
                    V^{E} = \Delta V = x_{1}x_{2}\left(45x_{1}+25x_{2}\right) = 7.92\text{ cm}^{3}.\text{mol}^{-1}
                 \end{displaymath}
              therefore $V=105.92\text{ cm}^{3}.\text{mol}^{-1}$.
\medskip

              Now using Eqn.~\ref{Chapter:SolutionThermodynamics:Eqn:GibbsDuhem1k} to calculate the partial molar volume $\overline{V}_{i}$,
                 \begin{eqnarray}
                    \overline{V}_{1} &=& V + x_{2}\frc{dV}{dx_{1}} = V + x_{2}\frc{d}{dx_{1}} \left(\overbrace{\Delta V}^{V^{E}} + x_{1}V_{1} + x_{2}V_{2}\right) \nonumber \\
                         &=& V + x_{2}\frc{d}{dx_{1}} \left[x_{1}x_{2}\left(45x_{1}+25x_{2}\right)+x_{1}V_{1}+x_{2}V_{2}\right] = 190.28\text{ cm}^{3}.\text{mol}^{-1}, \nonumber
                 \end{eqnarray}
              with the same procedure for $\overline{V}_{2}=49.68\text{ cm}^{3}.\text{mol}^{-1}$. We can check if our calculations are correct with,
                 \begin{displaymath}
                      V = \summation[x_{i}\overline{V}_{i}]{}{} = x_{1}\overline{V}_{1} + x_{2}\overline{V}_{2} = 105.92\text{ cm}^{3}.\text{mol}^{-1}.
                 \end{displaymath}
           
   \end{MyExample} 
     

   
%%% SECTION
\section{Ideal Gas Mixture Model}\label{Chapter:SolutionThermodynamics:Section:IGM}\index{Gases!Mixture}
   \begin{subequations}
%
     The ideal gas equation of state of originally introduced in Section~\ref{Chapter:Intro_Property_of_Gases:Section:IdealGases}\index{Gases!Ideal gas}
     \begin{displaymath}
        PV = nRT \;\;\;\Rightarrow\;\;\; P\overline{v} = RT
     \end{displaymath}
     In an {\it ideal gas mixture} (igm), the EOS has a similar format,
     \begin{equation}
       PV^{\text{igm}} = \underbrace{\left(n_{1}+n_{2}+\cdots+n_{\mathcal{C}}\right)}_{N\text{ moles}}RT = \left(\summation[n_{i}]{i=1}{\mathcal{C}}\right)RT = NRT.\label{Chapter:SolutionThermodynamics:Eqn:IGM1} 
     \end{equation}
     The internal energy of the gaseous mixture, $U^{\text{igm}}$, is just the summation of the internal energies of the individual components of the mixture,
     \begin{equation}
       U^{\text{igm}}(T,N) = \summation[n_{i}\overline{u}_{i}^{\text{ig}}(T)]{i=1}{\mathcal{C}},\label{Chapter:SolutionThermodynamics:Eqn:IGM2} 
     \end{equation}
     where $\overline{u}_{i}^{\text{ig}}=\frac{U_{i}^{\text{ig}}}{n_{i}}$ is the molar internal energy. We can change the notation on Eqns.~\ref{Chapter:SolutionThermodynamics:Eqn:IGM1}-\ref{Chapter:SolutionThermodynamics:Eqn:IGM2} to replace the number of moles by molar fraction, $y_{i}$, of the gas mixture, and to obtain the {\it partial molar internal energy}, $\overline{U}_{i}^{\text{igm}}(T,y)$
     \begin{equation}
       \overline{U}_{i}^{\text{igm}}(T,y) = \Partial[U^{\text{igm}}(T,N)]{n_{i}}{T,P,n_{j\ne i}} = \left[\frc{\partial\summation[n_{j}\overline{u}_{j}^{\text{ig}}(T)]{j=1}{\mathcal{C}}}{\partial n_{i}}\right]_{T,P,n_{j\ne i}}  = \overline{u}_{i}^{\text{ig}}(T)\label{Chapter:SolutionThermodynamics:Eqn:IGM3}
     \end{equation} 
     This expression indicates that the partial molar internal energy of species $i$ in an ideal gas mixture at a given tempereature is equal to the pure component molar internal energy of that component behaving as an ideal gas at the same temperature. We can also obtain the {\it partial molar volume}, $\overline{V}_{i}^{\text{igm}}$, with the same conclusion,
     \begin{equation}
       \overline{V}_{i}^{\text{igm}}(T,P,y) = \Partial[V^{\text{igm}}(T,P,N)]{n_{i}}{T,P,n_{j\ne i}} = \left[\frc{\partial\summation[n_{j} \frc{RT}{P}]{j=1}{\mathcal{C}}}{\partial n_{i}}\right]_{T,P,n_{j\ne i}} = \frc{RT}{P} = \overline{v}_{i}^{\text{ig}}(T,P).\label{Chapter:SolutionThermodynamics:Eqn:IGM4}
     \end{equation}
     \medskip

     The partial pressure of species $i$ in a gas mixture, $P_{i}$ (see Dalton's law in Section~\ref{Chapter:Intro_Property_of_Gases:Section:MixtureGases}), is defined for both ideal and real gas mixtures as\index{Gases!Dalton's law}
     \begin{displaymath}
       P_{i} = y_{i}P,
     \end{displaymath}
     and for an ideal gas mixture,
     \begin{equation}
       P_{i}^{\text{igm}}(N,T,V,y) = \frc{n_{i}}{\summation[n_{j}]{j=1}{\mathcal{C}}}P = \frc{n_{i}}{\summation[n_{j}]{j=1}{\mathcal{C}}}\left[\summation[n_{j}\frc{RT}{V}]{j=1}{\mathcal{C}}\right] = \frc{n_{i}RT}{V} = P^{\text{ig}}\left(n_{i},V,T\right),\label{Chapter:SolutionThermodynamics:Eqn:IGM4}
     \end{equation}
     \ie for ideal gas mixtures, the partial pressure of species $i$ is equal to the pressure that would be exerted if the same number of moles of that species, $n_{i}$, alone were contained in the same volume, $V$, at $T$ and $P$. 
\medskip

     As there is {\bf no} effective inter-/intra-molecular interactions in an ideal gas mixture, the effect on each species of the mixture at constant $T$ and $P$ is equivalent to:
     \begin{itemize}
       \item reducing the pressure from $P$ to $P_{i}$, or;
       \item expanding each gas from its initial volume $V_{i}=n_{i}RT/P$ to $V=\summation[n_{i}RT/P]{}{}$.
     \end{itemize}
     Thus using the entropy relations below for ideal gases (Eqns.~\ref{Chapter:SecondLaw:Eqn:EntropyChanges1}-\ref{Chapter:SecondLaw:Eqn:EntropyChanges2}),
       \begin{displaymath}
           dS =
         \begin{cases}
              C_{p}d\left(\ln{T}\right) - Rd\left(\ln{P}\right), \\
              C_{v}d\left(\ln{T}\right) + Rd\left(\ln{V}\right),
         \end{cases}          
     \end{displaymath}
     leading to (assuming constant $T$),
       \begin{displaymath}
           \overline{S}_{i}^{\text{igm}}(T,P,y) - \overline{s}_{i}^{\text{ig}}(T,P) =
         \begin{cases}
              -R\ln{\frc{P_{i}}{P}} = -R\ln{y_{i}}\;\;\;\text{ or }, \\
               R\ln{\frc{V}{V_{i}}} = R\ln{\frc{\summation[n_{j}RT/P]{j}{}}{n_{i}RT/P}} = -R\ln{y_{i}},
         \end{cases}          
     \end{displaymath}
     for the whole mixture, the molar entropy change of mixing, $\Delta_{\text{mix}}\overline{S}^{\text{igm}}$ (see Eqn.~\ref{Chapter:SolutionThermodynamics:Eqn:GibbsDuhem1h}) can be obtained from
     \begin{displaymath}
          \Delta_{\text{mix}}S^{\text{igm}} = \summation[n_{i}\left(\overline{S}_{i}^{\text{igm}}(T,P,y) - \overline{s}_{i}^{\text{ig}}(T,P) \right) ]{i=1}{\mathcal{C}} = -R\summation[n_{i}\ln{y_{i}}]{i=1}{\mathcal{C}}.
     \end{displaymath}
     If this expression is divided by the total number of moles, $\summation[n_{j}]{j=1}{\mathcal{C}}$,
     \begin{equation}
       \Delta_{\text{mix}}\overline{s}^{\text{igm}} = - R \summation[y_{i}\ln{y_{i}}]{i=1}{\mathcal{C}}.
     \end{equation}
\medskip
  
     \noindent With volume, internal energy and entropy change of mixing, partial molar properties of other thermodynamic potentials can be readily obtained. For enthalpy,
     \begin{eqnarray}
          && \overline{H}_{i}^{\text{igm}}(T,P,y) = \overline{h}_{i}^{\text{ig}}(T,P) \nonumber \\
          && \overline{H}^{\text{igm}}(T,P,y) = \summation[y_{i}\overline{h}_{i}^{\text{ig}}(T,P)]{i=1}{\mathcal{C}},
     \end{eqnarray}
     and for Gibbs free energy from $G=H-TS$,
     \begin{eqnarray}
       \overline{G}_{i}^{\text{igm}}(T,P,y) &=& \overline{H}_{i}^{\text{igm}}(T,P,y) - T\overline{S}_{i}^{\text{igm}}(T,P,y) \nonumber \\
                                        &=& \overline{h}_{i}^{\text{ig}}(T,P) - T\left[ \overline{s}_{i}^{\text{ig}}(T,P) - R\ln{y_{i}}\right] \nonumber \\
                                        &=& \overline{g}_{i}^{\text{ig}}(T,P) + RT\ln{y_{i}},
     \end{eqnarray}
     and the Gibbs free energy change of mixing,
     \begin{eqnarray}
        \Delta_{\text{mix}}\overline{g}_{i}^{\text{igm}} &=& \summation[y_{i}\left(\overline{G}_{i}^{\text{igm}}(T,P,y)-\overline{g}_{i}^{\text{ig}}(T,P)\right)]{i=1}{\mathcal{C}} \nonumber \\
                                        &=& RT\summation[y_{i}\ln{y_{i}}]{i=1}{\mathcal{C}} \nonumber 
     \end{eqnarray}


 \begin{table}\scriptsize
     \begin{tabular}{ | c | c | c | c |}%\scriptsize
 \hline
        {\bf Property}    &   {\bf Partial Molar}   &  {\bf Change of Mixing}                    &  {\bf Molar Property} \\
                          &   {\bf Property}        &  $\left(\mathbf{\Delta_{\text{mix}}}\right)$  &                       \\
 \hline
            Volume        &   $\overline{V}_{i}^{\text{igm}}(T,y) = \overline{v}_{i}^{\text{ig}}(T)$ & $\Delta_{\text{mix}} \overline{v}^{\text{igm}}=0 $ & $\overline{v}^{\text{igm}}(T,P,y) = \summation[y_{i}\overline{v}_{i}^{\text{ig}}(T,P)]{}{}$ \\
         Internal Energy  &   $\overline{U}_{i}^{\text{igm}}(T,y) = \overline{u}_{i}^{\text{ig}}(T)$ & $\Delta_{\text{mix}} \overline{u}^{\text{igm}}=0 $ & $\overline{u}^{\text{igm}}(T,y) = \summation[y_{i}\overline{u}_{i}^{\text{ig}}(T)]{}{}$ \\
         Enthalpy         &   $\overline{H}_{i}^{\text{igm}}(T,y) = \overline{h}_{i}^{\text{ig}}(T)$ & $\Delta_{\text{mix}} \overline{h}^{\text{igm}}=0 $ & $\overline{h}^{\text{igm}}(T,P,y) = \summation[y_{i}\overline{h}_{i}^{\text{ig}}(T,P)]{}{}$ \\
         Entropy          &   $\overline{S}_{i}^{\text{igm}}(T,P,y) = \overline{s}_{i}^{\text{ig}}(T,P)-R\ln{y_{i}}$ & $\Delta_{\text{mix}} \overline{s}^{\text{igm}}=-R\summation[y_{i}\ln{y_{i}}]{}{}$ & $\overline{s}^{\text{igm}}(T,P,y) = \summation[y_{i}\overline{s}_{i}^{\text{ig}}(T,P)]{}{}-R\summation[y_{i}\ln{y_{i}}]{}{}$ \\
         Gibbs Energy     &   $\overline{G}_{i}^{\text{igm}}(T,P,y) = \overline{g}_{i}^{\text{ig}}(T,P)+RT\ln{y_{i}}$ & $\Delta_{\text{mix}} \overline{g}^{\text{igm}}=-RT\summation[y_{i}\ln{y_{i}}]{}{}$ & $\overline{g}^{\text{igm}}(T,P,y) = \summation[y_{i}\overline{g}_{i}^{\text{ig}}(T,P)]{}{} + RT\summation[y_{i}\ln{y_{i}}]{}{}$ \\
         Helmholtz Energy &   $\overline{A}_{i}^{\text{igm}}(T,P,y) = \overline{a}_{i}^{\text{ig}}(T,P)+RT\ln{y_{i}}$ & $\Delta_{\text{mix}} \overline{a}^{\text{igm}}=-RT\summation[y_{i}\ln{y_{i}}]{}{}$ & $\overline{a}^{\text{igm}}(T,P,y) = \summation[y_{i}\overline{a}_{i}^{\text{ig}}(T,P)]{}{} + RT\summation[y_{i}\ln{y_{i}}]{}{}$ \\ 
 \hline
     \end{tabular}
     \caption{Properties of ideal gas mixtures \citep[extracted from][]{Sandler_Book}.}\label{Chapter:SolutionThermodynamics:Table:TableIGM}\index{Molar property change of mixing!Volume}\index{Molar property change of mixing!Internal energy}\index{Molar property change of mixing!Enthalpy}\index{Molar property change of mixing!Entropy}\index{Molar property change of mixing!Gibbs free energy}\index{Molar property change of mixing!Helmholtz free energy}\index{Gibbs free energy!change of mixing}\index{Helmholtz free energy!change of mixing}\index{Enthalpy!change of mixing}\index{Entropy!change of mixing}\index{Internal Energy!change of mixing}
 \end{table}
     
     Partial molar properties, molar properties and change of mixing of molar properties for the main thermodynamic functions are summarised in Table~\ref{Chapter:SolutionThermodynamics:Table:TableIGM}.
%
   \end{subequations}




% Example
   \begin{MyExample}{\begin{center}{\bf Example}\end{center}}
     \begin{example}\label{Chapter:SolutionThermodynamics:Example5} % NPTEl (Ex 6.3)
        What is the change in entropy when 0.6 m$^{3}$ of CO$_{2}$ and 0.4 m$^{3}$ of N$_{2}$, each at 1 bar and 25 $^{\circ}$C blend to form a gas mixture at the same conditions. Assume ideal gas behaviour and that {\it mole fraction = volume fraction}. 
     \end{example}

% SOLUTION
     \noindent{\bf Solution:}
           For an ideal gas, with the assumption of {\it mole fraction = volume fraction}, \ie $y_{1} = 0.6$ and $y_{2} = 0.4$, with CO$_{2}$:1 and N$_{2}$:2.
      \begin{displaymath}
          \Delta_{\text{mix}}\overline{s}^{\text{igm}} = -R\summation[y_{i}\ln{y_{i}}]{i=1}{2} = 5.5954 \text{ J.(mol.K)}^{-1}
      \end{displaymath}
   \end{MyExample} 

      
%%% SECTION
\section{Fugacity Coefficient of Species in Mixtures}\label{Chapter:SolutionThermodynamics:Section:FugacityCoefficient}\index{Fugacity coefficient|see {Fugacity}}\index{Fugacity! Coefficient}
   \begin{subequations}
%
      In  Section~\ref{Chapter:SolutionThermodynamics:Section:FugacitySection} (Eqn.~\ref{Chapter:SolutionThermodynamics:Eqn:fugacity1b}), a formal definition of fugacity was introduced,
       \begin{displaymath}
          \mu_{i}= RT\ln{\overline{f}_{i}} + C_{i}(T),
       \end{displaymath}
       where $\overline{f}_{i}$ is the fugacity of species $i$ in the mixture. For pure fluids, the \blue{fugacity coefficient}, $\phi$, can be defined as,
       \begin{equation}
         \phi = \frc{f}{P}.
       \end{equation}
       For pure ideal gases, $f^{\text{ig}}=P$ (obtained from the definition of fugacity $\lim\limits_{P\rightarrow 0}\frac{f}{P}=1$) and $\phi^{\text{ig}}=1.$
       \begin{shaded}
         \noindent The definition of fugacity coefficient can be extended to {\it real gas mixtures} and {\it real liquid solutions} as,
         \begin{eqnarray}
           \mfr[\phi]{i}{V} &=& \frc{\mfr[\overline{f}]{i}{V}}{y_{i}P}\;\;\text{ and, }\label{Chapter:SolutionThermodynamics:Eqn:FugCoeffVap1}\\
           \mfr[\phi]{i}{L} &=& \frc{\mfr[\overline{f}]{i}{L}}{x_{i}P_{i}^{\text{sat}}}.\label{Chapter:SolutionThermodynamics:Eqn:FugCoeffLiq1}
         \end{eqnarray}
       \end{shaded}

\medskip
       For a single component in a closed system behaving as an ideal gas, the fundamental thermodynamic relation is valid,
         \begin{displaymath}
            dG = VdP - SdT.
         \end{displaymath}
         A pure gas $i$ at constant temperature, this relation reduces to,
         \begin{displaymath}
            dG_{i}^{\text{ig}} = V_{i}^{\text{ig}}dP = RT\frc{dP}{P} = RTd\ln{P}.
         \end{displaymath}
         Integrating this expression leads to,
         \begin{displaymath}
             G_{i}^{\text{ig}} = RT\ln{P} + C_{i}(T),
         \end{displaymath}
         where $C_{i}$ is an integration constant. With the same procedure, but now for a real fluid, we can obtain a relation between the Gibbs free energy of species $i$ and the fugacity at constant $T$,
         \begin{displaymath} 
             G_{i} = RT\ln{f_{i}} + C_{i}(T).
         \end{displaymath}
         Subtracting these expressions results in the {\it residual Gibbs free energy} (Section~\ref{Chapter:ThermodynamicPropertiesPureFluids:Section:ResidualProperties})\index{Gases!Residual properties}
         \begin{equation}
            G_{i}^{R} = G_{i} - G_{i}^{\text{ig}} = RT\ln{f_{i}}{P} = RT\ln{\phi_{i}}.
         \end{equation}
         Using the Gibbs free energy generating function, Eqn.~\ref{Chapter:ThermodynamicPropertiesPureFluids:Eqn:ResidualProperties_GeneratingGibbsFunction3}, %$\frc{G^{R}}{RT} =  \int\limits_{0}^{P}\left(Z-1\right)\frc{dP}{P}$,
         \begin{equation}
            \ln{\phi_{i}} = \int\limits_{0}^{P}\left(Z-1\right)\frc{dP}{P}
         \end{equation}
\bigskip

         During phase change from saturated liquid to saturated vapour,\index{Fugacity}
         \begin{equation}
            \mfr[G]{i}{V} - \mfr[G]{i}{L} = RT\ln{\frc{\mfr[f]{i}{V}}{\mfr[f]{i}{L}}},\label{Chapter:SolutionThermodynamics:Eqn:FugacityCoeff1}
         \end{equation}
         however, from Section~\ref{Chapter:ThermodynamicPropertiesPureFluids:Section:ClapeyronRelations}, we studied that during phase transition \blue{$dG=0$}, \ie
         \begin{displaymath}
            \mfr[G]{i}{V} - \mfr[G]{i}{L} = 0 = RT\ln{\frc{\mfr[f]{i}{V}}{\mfr[f]{i}{L}}}
         \end{displaymath}
         \begin{shaded}
            \begin{equation}
               \mfr[f]{i}{V} = \mfr[f]{i}{L} = \mfr[f]{i}{\text{sat}} \;\;\text{ and }\;\; \mfr[\phi]{i}{V} = \mfr[\phi]{i}{L} = \mfr[\phi]{i}{\text{sat}}.
            \end{equation}
         \end{shaded}
%
   \end{subequations}

%%% SECTION
\section{Expression for Fugacity of a Pure Liquid}\label{Chapter:SolutionThermodynamics:Section:FugacityCoefficient_Liquid}\index{Solutions!Activity}\index{Solutions!Activity coefficient}\index{Fugacity! Coefficient}
%
   \begin{subequations}
%
         We can write the fugacity of a pure liquid species $i$ as,
         \begin{displaymath}
            \mfr[f]{i}{L}(P) = \underbrace{\frc{\mfr[f]{i}{V}\left(P_{i}^{\text{sat}}\right)}{P_{i}^{\text{sat}}}}_{\blue{A}} \underbrace{\frc{\mfr[f]{i}{L}\left(P_{i}^{\text{sat}}\right)}{\mfr[f]{i}{V}\left(P_{i}^{\text{sat}}\right)}}_{\blue{B}} \underbrace{\frc{\mfr[f]{i}{L}\left(P\right)}{\mfr[f]{i}{L}\left(P_{i}^{\text{sat}}\right)}}_{\blue{C}} P_{i}^{\text{sat}}.
         \end{displaymath}
         with
         \begin{description}
%
             \item[\blue{(A):}] vapour phase fugacity coefficient, $\phi_{i}^{\text{sat}}$,
                \begin{equation}
                   \ln{\phi_{i}^{\text{sat}}} = \int\limits_{0}^{P_{i}^{\text{sat}}}\left(\mfr[Z]{i}{V}-1\right)\frc{dP}{P},\;\;\;\;\text{ at constant } T;
                \end{equation}
%
             \item[\blue{(B):}] $=1$ as $\mfr[f]{i}{L}\left(P_{i}^{\text{sat}}\right) = \mfr[f]{i}{V}\left(P_{i}^{\text{sat}}\right)$;
%
             \item[\blue{(C):}] effect of pressure on fugacity of pure liquid $i$. From $G_{i} - G_{i}^{\text{sat}} = \int\limits_{P_{i}^{\text{sat}}}^{P}\mfr[V]{i}{L}dP$ (at constant $T$), \blue{C} becomes (based on Eqn.~\ref{Chapter:SolutionThermodynamics:Eqn:FugacityCoeff1}),
                 \begin{eqnarray}
                     \blue{C} = \frc{\mfr[f]{i}{L}\left(P\right)}{\mfr[f]{i}{L}\left(P_{i}^{\text{sat}}\right)} &=& \exp\left[\frc{1}{RT}\int\limits_{P_{i}^{\text{sat}}}{P}\mfr[V]{i}{L}dP\right] \nonumber \\
                              \mfr[f]{i}{L}\left(P\right) &=& \underbrace{\mfr[f]{i}{L}\left(P_{i}^{\text{sat}}\right)}_{\mfr[\phi]{i}{\text{sat}}=\frc{\mfr[f]{i}{\text{sat}}}{P_{i}^{\text{sat}}}}  \underbrace{\exp{\left[\frc{1}{RT}\int\limits_{P_{i}^{\text{sat}}}{P}\mfr[V]{i}{L}dP\right]}}_{\text{Poynting-pressure factor}},\;\;\;\text{ assuming constant } \mfr[V]{i}{L} \nonumber \\
                              \mfr[f]{i}{L}\left(P\right) &=& \mfr[\phi]{i}{\text{sat}}P_{i}^{\text{sat}}\exp{\left[\frc{\mfr[V]{i}{L}\left(P - P_{i}^{\text{sat}}\right)}{RT}\right]}
                 \end{eqnarray}
                 The \blue{Poynting-pressure factor} indicates the increase in fugacity due to the fact that system pressure is larger than the vapour pressure of the liquid. As the molar volume of the liquid phase is much smaller that the molar volume of the vapour phase $\left(\mfr[V]{}{L} <<<< \mfr[V]{}{V}\right)$, the Poynting-pressure factor is only important at high pressure or at very low temperature conditions. The Poynting-pressure factor is the {\it best approximation} for $\mfr[f]{}{L}$.
         \end{description}
         
         \noindent Therefore, in order to calculate the fugacity of pure liquids, $\mfr[f]{i}{L}(P)$, we need:
               \begin{itemize}
                  \item $\mfr[Z]{i}{V}$ (obtained from either EOS, experiments or generalised correlations) to compute $\mfr[\phi]{i}{\text{sat}}$;
                  \item liquid phase molar volume, $\mfr[V]{}{L}$ (usually the value for saturated liquid);
                  \item a value for $P_{i}^{\text{sat}}$.
               \end{itemize}
%
   \end{subequations}

%%% SECTION
\section{Relation between Residual Property and Species Fugacity Coefficients in Mixtures}\label{Chapter:SolutionThermodynamics:Section:FugacityCoefficient_Residual}
%
   \begin{subequations}
%
        We can extend the definition of fugacity coefficient for mixture of gases or liquid solutions, \ie, from the definition of partial residual Gibbs energy,
            \begin{displaymath}
               \overline{G}^{R} = \overline{G}_{i} - \overline{G}_{i}^{\text{ig}} = \mu_{i} - \mu_{i}^{R}
            \end{displaymath}
        and for the chemical potential, $\mu_{i}$,
            \begin{displaymath}
               \mu_{i} - \mu_{i}^{\text{ig}} = RT\ln{\frc{\overline{f}_{i}}{y_{i}P}},
            \end{displaymath}
        thus
            \begin{displaymath}
               \overline{G}^{R} = RT\ln{\overline{\phi}_{i}},\;\;\;\text{ with }\;\; \overline{\phi}_{i} = \frc{\overline{f}_{i}}{y_{i}P}.
            \end{displaymath}
 

%
   \end{subequations}

%%% SECTION
\section{Ideal Solution Model}\label{Chapter:SolutionThermodynamics:Section:IdealSolution}\index{Solutions!Ideal}\index{Gibbs free energy!ideal solution}\index{Helmholtz free energy!ideal solution}\index{Enthalpy!ideal solution}\index{Entropy!ideal solution}\index{Internal Energy!ideal solution}
%
   \begin{subequations}
%
      From Table~\ref{Chapter:SolutionThermodynamics:Table:TableIGM}, the chemical potential of an ideal gas mixture model is defined as
         \begin{displaymath}
              \mu_{i}^{\text{igm}} = \overline{G}_{i}^{\text{igm}}(T,P,y) = \overline{g}_{i}^{\text{ig}}(T,P)+RT\ln{y_{i}}.
         \end{displaymath}
      \begin{shaded}
      Using a similar method, an expression can be derived for ideal solutions,
         \begin{equation}
              \mu_{i}^{\text{id}} = \overline{G}_{i}^{\text{id}}(T,P,x) = \overline{g}_{i}^{\text{ig}}(T,P)+RT\ln{x_{i}}.
         \end{equation}
      Similarly to Section~\ref{Chapter:SolutionThermodynamics:Section:IGM}, the partial molar volume, $\overline{V}_{i}^{\text{id}}$, entropy, $\overline{S}_{i}^{\text{id}}$, and enthalpy, $\overline{H}_{i}^{\text{id}}$ can be defined as
         \begin{eqnarray}
            \overline{V}_{i}^{\text{id}} &=& \Partial[\overline{G}_{i}^{\text{id}}]{P}{T,x} = \Partial[G_{i}]{P}{T} = V_{i},  \\
            \overline{S}_{i}^{\text{id}} &=& -\Partial[\overline{G}_{i}^{\text{id}}]{T}{P,x} = \Partial[G_{i}]{T}{P} - R\ln{x_{i}} = S_{i} -R\ln{x_{i}}, \\
            \overline{H}_{i}^{\text{id}} &=& \overline{G}_{i}^{\text{id}} + R\overline{S}_{i}^{\text{id}} = H_{i}.
         \end{eqnarray}
     And the summability relations for ideal solutions, 
        \begin{displaymath}
            M^{\text{id}} = \summation[x_{i}\overline{M}_{i}^{\text{id}}]{}{},
        \end{displaymath} 
    thus, 
        \begin{eqnarray}
           && G^{\text{id}} = \summation[x_{i}G_{i}]{}{} + RT\summation[x_{i}\ln{x_{i}}]{}{}, \\
           && S^{\text{id}} = \summation[x_{i}S_{i}]{}{} - R\summation[x_{i}\ln{x_{i}}]{}{}, \\
           && V^{\text{id}} = \summation[x_{i}V_{i}]{}{}, \\
           && H^{\text{id}} = \summation[x_{i}H_{i}]{}{}, 
        \end{eqnarray}
   \end{shaded}

%
   \end{subequations}

%%% SECTION
\section{Defining Activity Coefficient based on Excess Properties}\label{Chapter:SolutionThermodynamics:Section:ActivityCoeffExcessProp}\index{Solutions!Activity}\index{Solutions!Excess properties}\index{Gibbs free energy!Excess}\index{Helmholtz free energy!Excess}\index{Enthalpy!Excess}
%
   \begin{subequations}
%
      In Section~\ref{Chapter:ThermodynamicPropertiesPureFluids:Section:GibbsGeneratingFunction}, the generating function of the Gibbs free energy was formally defined. Such definition can be extended (now considering that Gibbs free energy depends on temperature, pressure and number of moles) to study the excess Gibbs free energy,
        \begin{equation}
           d\left(\frc{nG^{E}}{RT}\right) = \frc{nV^{E}}{RT}dP - \frc{nH^{E}}{RT^{2}}dT + \summation[\frc{\overline{G}_{i}^{E}}{RT}dn_{i}]{i}{}.\label{Chapter:SolutionThermodynamics:Eqn:ActivityCoeffExcessProp1}
        \end{equation}

        \begin{shaded}
           The excess Gibbs free energy of species $i$ is obtained from,
             \begin{eqnarray}
                \overline{G}_{i}^{E} &=&  \overline{G}_{i} -G _{i}^{\text{id}} = \left[RT\ln{\overline{f}_{i}}+C_{i}(T)\right] - \left[RT\ln{x_{i}f_{i}}+C_{i}(T)\right] \nonumber \\
                                    &=&  RT\ln{\left(\frc{\overline{f}_{i}}{x_{i}f_{i}}\right)} = RT\ln{\gamma_{i}}\label{Chapter:SolutionThermodynamics:Eqn:ActivityCoeffExcessProp2}
             \end{eqnarray}
            Replacing this expression in Eqn.~\ref{Chapter:SolutionThermodynamics:Eqn:ActivityCoeffExcessProp1},
              \begin{equation}
                 d\left(\frc{nG^{E}}{RT}\right) = \frc{nV^{E}}{RT}dP - \frc{nH^{E}}{RT^{2}}dT + \summation[\ln{\gamma_{i}}dn_{i}]{i}{}.\label{Chapter:SolutionThermodynamics:Eqn:ActivityCoeffExcessProp2}
              \end{equation}
            And we can also define (see Eqns~\ref{Chapter:ThermodynamicPropertiesPureFluids:Eqn:GibbsGeneratingFunctionTConst} and \ref{Chapter:ThermodynamicPropertiesPureFluids:Eqn:GibbsGeneratingFunctionPConst}),
               \begin{eqnarray}
                   && \frc{V^{E}}{RT} = \left[\frc{\partial \left(\frac{G^{E}}{RT}\right)}{\partial P}\right]_{T,x} \\
                   && \frc{H^{E}}{RT} = -T\left[\frc{\partial \left(\frac{G^{E}}{RT}\right)}{\partial P}\right]_{P,x} \\
                   && \ln{\gamma_{i}} = \left[\frc{\partial \left(\frac{G^{E}}{RT}\right)}{\partial P}\right]_{T,P,n_{j\ne i}} 
               \end{eqnarray}
        \end{shaded}
%
   \end{subequations}

%%% SECTION
\section{Activity Coefficient Models}\label{Chapter:SolutionThermodynamics:Section:ActivityCoeffModels}\index{Solutions!Activity coefficient}\index{Solutions!Activity coefficients!Models!Wilson}\index{Solutions!Activity coefficients!Models!Margules}\index{Solutions!Activity coefficients!Models!Van Laar}\index{Solutions!Activity coefficients!Models!NRTL}\index{Solutions!Activity coefficients!Models!UNIQUAC}\index{Solutions!Activity coefficients!Models!UNIFAC}
%
   \begin{subequations}
      The activity coefficient, $\gamma_{i}$, is by far the most critical parameter for the assessment of liquid solutions behaviour, in particular during phase equilibria due to its relationship with the excess Gibbs energy, $G^{E}$ (Eqn.~\ref{Chapter:SolutionThermodynamics:Eqn:ActivityCoeffExcessProp2}). For solutions, the activity coefficient plays a similar role as the compressibility factor, $Z$, for gases. As $G^{E}$ depends on $T$, $P$ and $n$, it is important to define relations for $\gamma_{i}$ with respect to these properties, 
    \begin{shaded}
      \begin{enumerate}[a)]
          \item Temperature:
             \begin{equation}
                \ln{\gamma_{i}}\left(T_{2}\right) - \ln{\gamma_{i}}\left(T_{1}\right) = \int\limits_{T{i}}^{T_{2}} \Partial[\ln{\gamma_{i}}]{T}{P,n}dT,\;\;\text{ where }\;\; \Partial[\ln{\gamma_{i}}]{T}{P,n} = -\frc{\overline{H}_{i}-H_{i}^{\text{id}}}{RT^{2}};
             \end{equation}
          \item Pressure:
             \begin{equation}
                \ln{\gamma_{i}}\left(P_{2}\right) - \ln{\gamma_{i}}\left(P_{1}\right) = \int\limits_{P{i}}^{P_{2}} \Partial[\ln{\gamma_{i}}]{P}{T,n}dT,\;\;\text{ where }\;\; \Partial[\ln{\gamma_{i}}]{P}{T,n} = \frc{\overline{V}_{i}-V_{i}^{\text{id}}}{RT}.
             \end{equation}
      \end{enumerate}
     \end{shaded}

%%% TABLE 1
\begin{table}[h]
  \begin{center}
     \begin{tabular}{|l l|} 
\hline
         {\bf System Type}                    &  {\bf Models} \\
\hline
            Species of similar size and shape &   1-parameter Margules \\
            Moderately non-ideal mixtures     &   2-parameter Margules, Van Laar, Regular Solution\\
            Strongly non-ideal mixtures       &   Wilson, NRTL, UNIQUAC \\
            Solutions with miscibility gaps   &   NRTL, UNIQUAC \\
\hline 
     \end{tabular}
     \caption{Applicability of activity coefficient models.}\label{Chapter:SolutionThermodynamics:Table:TableActivityModels1}
  \end{center}
\end{table}
   
     Similarly to EOS for gases, several empirical and semi-empirical relations were developed for the dependence of $G^{E}$ $\left(\text{and therefore } \gamma_{i}\right)$ with composition. Such relations, often called {\it activity coefficient models}, were designed to predict Gibbs free energy behaviour in solutions regardless the inter-intra-molecular forces involved (\ie taking into account effects of molar mass, polarity, charged/ionic solutions etc). These models are divided into two major groups depending upon (a) molecular size (\ie molar mass) and shape (stereochemistry) of species and (b) inter-/intra-molecular interaction energies:
        \begin{enumerate}[Group 1:]
            \item Homogeneous Mixtures Models are used when inter-/intra-molecular interaction energies between species are moderate. These models assume that molecules are homogeneously distributed over the solution volume with no difference between overall macroscopic composition and microscopic (local) composition around a single central molecule. Examples of such models include Mergules, Van Laar, Regular Solution models etc;
            \item Local Compositions Models are used when the species in solution are of different size and shape (\eg polymers and organic solvents, solutions involving isomers, ionic solutions etc). Examples of such models include Wilson, NRTL (Non-Random Two Liquid), UNIQUAC (Universal Quasi-Chemical), UNIFAC (UNIQUAC Functional Activity Coefficients) models etc.
        \end{enumerate}
        Group 1 models are often used in moderate deviation from ideal solution behaviour, whereas Group 2 models are used for strongly non-ideal solutions (Table~\ref{Chapter:SolutionThermodynamics:Table:TableActivityModels1}). Most of these models contain 2 to 3 parameters, however some may contain larger number of parameters to improve the accuracy of prediction (though are more computationally demanding). Some of these models may be found in Table~\ref{Chapter:SolutionThermodynamics:Table:TableActivityModels2}.

       \begin{shaded}
           Using Eqn.~\ref{Chapter:SolutionThermodynamics:Eqn:ActivityCoeffExcessProp2} for the partial molar Gibbs free energy for component $i$ along with $G^{E}=\summation[x_{i}\overline{G}_{i}^{E}]{}{}$ leads to \blue{GDE},
             \begin{equation}
                  \frc{G^{E}}{RT} = \summation[x_{i}\ln{\gamma_{i}}]{}{}\;\;\;\text{ and } \;\;\; \summation[x_{i}d\left(\ln{\gamma_{i}}\right)]{}{}=0,\;\;\text{ at constant } T\text{ and } P.\label{Chapter:SolutionThermodynamics:Eqn:ActivityCoeffModels1}
             \end{equation}
           For binary mixtures,
             \begin{displaymath}
                 \begin{cases}
                      \frc{G^{E}}{RT} = x_{1}\ln{\gamma_{1}} + x_{1}\ln{\gamma_{1}},  \\
                      x_{1}d\left(\ln{\gamma_{1}}\right) + x_{2}d\left(\ln{\gamma_{2}}\right) = 0.
                 \end{cases}
             \end{displaymath}
       \end{shaded}


%%% TABLE 2
\begin{landscape}
\begin{table}[h]
  \begin{center}
     \begin{tabular}{l |l | l  | c  }
\hline
         {\bf Model}         &  $\mathbf{\frc{G^{E}}{RT}=}$   &   $\mathbf{\ln{\gamma_{i}}}$     & {\bf Binary } \\
                             &                                &                                  & {\bf Parameters} \\
\hline
      1-Parameter Margules   &     $Ax_{1}x_{2}$                &   $\ln{\gamma_{1}} = Ax_{2}^{2}$  &  $A$ \\
                             &                                &   $\ln{\gamma_{2}} = Ax_{1}^{2}$   &\\
\hline
      2-Parameter Margules   &     $x_{1}x_{2}\left(A_{21}x_{1}+A_{12}x_{2}\right)$ &   $\ln{\gamma_{1}} = x_{2}^{2}\left[A_{12}+2x_{1}\left(A_{21}-A_{12}\right)\right]$ & $A_{21}, A_{21}$\\
                             &                                                 &   $\ln{\gamma_{2}} = x_{1}^{2}\left[A_{21}+2x_{2}\left(A_{12}-A_{21}\right)\right]$ & \\
\hline
     van Laar                & $x_{1}x_{2}\frc{A_{12}A_{21}}{A_{12}x_{1}+A_{21}x_{2}}$ & $\ln{\gamma_{1}} = A_{12}\left(1+\frc{A_{12}x_{1}}{A_{21}x_{2}}\right)^{-2}$ & $A_{21}, A_{21}$ \\
                             &                                                  &  $\ln{\gamma_{2}} = A_{21}\left(1+\frc{A_{21}x_{2}}{A_{12}x_{1}}\right)^{-2}$ & \\
\hline
     Wilson                  & $-\summation[x_{i}\ln{\left(x_{i}+\summation[\Lambda_{ij}x_{j}]{j\ne i}{}\right)}]{i}{}$ &  $\ln{\gamma_{1}} = -\ln{\left(x_{1}+x_{2}\Lambda_{12}\right)}+x_{2}\left(\frc{\Lambda_{12}}{x_{1}+x_{2}\Lambda_{12}}-\frc{\Lambda_{21}}{x_{1}\Lambda_{21}+x_{2}}\right)$ & $\Lambda_{21}, \Lambda_{21}$\\
                             &                                                  &  $\ln{\gamma_{2}} = -\ln{\left(x_{1}\Lambda_{21}+x_{2}\right)}+x_{1}\left(\frc{\Lambda_{12}}{x_{1}+x_{2}\Lambda_{12}}-\frc{\Lambda_{21}}{x_{1}\Lambda_{21}+x_{2}}\right)$ & \\
\hline
     NRTL                    &$x_{1}x_{2}\left(\frc{\mathcal{G}_{21}\tau_{21}}{x_{1}+x_{2}\mathcal{G}_{21}}+\frc{\mathcal{G}_{12}\tau_{12}}{x_{2}+x_{1}\mathcal{G}_{12}}\right)$ & $\ln{\gamma_{1}} = x_{2}^{2}\left[\tau_{21}\left(\frc{\mathcal{G}_{21}}{x_{1}+x_{2}\mathcal{G}_{21}}\right)^{2} + \frc{\mathcal{G}_{12}\tau_{12}}{\left(x_{2}+x_{1}\mathcal{G}_{12}\right)^{2}}\right]$ & $\alpha,\; b_{12},\; b_{21}$ \\
                             & where $\mathcal{G}_{ij}=\exp{\left(-\alpha\tau_{ij}\right)},\;\tau_{ij}=\frc{b_{ij}}{RT}$ & $\ln{\gamma_{2}} = x_{1}^{2}\left[\tau_{12}\left(\frc{\mathcal{G}_{12}}{x_{2}+x_{1}\mathcal{G}_{12}}\right)^{2} + \frc{\mathcal{G}_{21}\tau_{21}}{\left(x_{1}+x_{2}\mathcal{G}_{21}\right)^{2}}\right]$ & \\
 
\hline          
     \end{tabular}
     \caption{Selected Activity Models for binary systems \citep[extracted from][]{Sandler_Book}.}\label{Chapter:SolutionThermodynamics:Table:TableActivityModels2}
  \end{center}
\end{table}
\end{landscape}

%      2-Parameter Margules   &     $Ax_{1}x_{2}$                &   $\ln{\gamma_{1}} = Ax_{2}^{2}$ \\
%                             &                                &   $\ln{\gamma_{2}} = Ax_{1}^{2}$ \\
%
   \end{subequations}



% Example
   \begin{MyExample}{\begin{center}{\bf Example}\end{center}}
     \begin{example}\label{Chapter:SolutionThermodynamics:Example6} %Nguyen (Ex 5.3.2)
         Calculate the bubble point temperature and vapor composition for a liquid mixture of 1 mol-$\%$ acetone (1) and water (2) at 101.3 KPa. Given the Wilson activity model:
   \begin{eqnarray}
       &&\ln\gamma_{1} = -\ln\left(x_{1}+x_{2}C_{12}\right) + x_{2}\left(\frc{C_{12}}{x_{1}+x_{2}C_{12}}-\frc{C_{21}}{x_{2}+x_{1}C_{21}}\right) \nonumber \\
       &&\ln\gamma_{2} = -\ln\left(x_{2}+x_{1}C_{21}\right) + x_{1}\left(\frc{C_{21}}{x_{2}+x_{1}C_{21}}-\frc{C_{12}}{x_{1}+x_{2}C_{12}}\right) \nonumber
   \end{eqnarray}
   with $C_{12}=$0.1173 and $C_{21}=$0.4227. Also, the vapour pressure $\left(\text{with } P_{i}^{\text{sat}}\text{ in kPa and } T\text{ in K}\right)$ is given by,
\begin{displaymath}
   \ln P_{1}^{\text{sat}} = 14.71712 - \frc{2975.95}{T-34.5228} \;\;\;\; \ln P_{2}^{\text{sat}} = 16.5362 - \frc{3985.44}{T-38.9974}
\end{displaymath}
     \end{example}

% SOLUTION
     \noindent{\bf Solution:}
            

   \begin{itemize}
      \item Since the vapour mole fractions are unknown, we should start from the main composition constraint, $\sum\limits_{i=1}^{2}y_{i} = 1$.  
      \item Replacing $y_{i}=x_{i}\gamma_{i}P_{i}^{\text{sat}}/P$,
         \begin{displaymath}
            x_{1}\gamma_{1}P_{1}^{\text{sat}} + x_{2}\gamma_{2}P_{2}^{\text{sat}} = P
         \end{displaymath}
      \item For $x_{1} = 0.01$, $x_{2}=0.99$, $C_{12}=$0.1173 and $C_{21}=$0.4227, $\gamma_{1}=$ 13.0695 and $\gamma_{2}=$ 1.0007
      \item With these numerical values, we can replace into the equation above,
         \begin{displaymath}
            (0.01)(13.0695)\exp\left[14.71712 - \frc{2975.95}{T-34.5228}\right] + (0.99)(1.0007)\exp\left[16.5362 - \frc{3985.44}{T-38.9974}\right] =  101.3
         \end{displaymath}
         Resulting in  $T= 361.71$ K -- the bubble point temperature of this acetone-water mixture.
      \item At this temperature, the saturation pressure of acetone and water are $P_{1}^{\text{sat}}= 276.32$ kPa and $P_{2}^{\text{sat}} = 65.78$ kPa, respectively.
      \item The vapour composition is
         \begin{displaymath}
             y_{1} =\frc{x_{1}\gamma_{1}P_{1}^{\text{sat}}}{P} = 0.3565\;\;\;\text{ and }\;\;\;\; y_{2} = 0.6435.
         \end{displaymath} 
   \end{itemize}
   \end{MyExample} 

% Example
   \begin{MyExample}{\begin{center}{\bf Example}\end{center}}
     \begin{example}\label{Chapter:SolutionThermodynamics:Example7} %Nguyen (Ex 5.3.1)  
         Find the bubble point pressure and vapor composition for a liquid mixture of 41.2 mol-$\%$ ethanol (1) and n-hexane (2) at 331 K. Given the Van Laar activity model equations:
\begin{displaymath}
\ln\gamma_{1} = \frc{A}{\left[1+\frc{A x_{1}}{B x_{2}}\right]^{2}}\;\;\;\; \ln\gamma_{2} = \frc{B}{\left[1+\frc{B x_{2}}{A x_{1}}\right]^{2}}\
\end{displaymath}
with A = 2.409 and B = 1.970. Also, the vapour pressure $\left(\text{with } P_{i}^{\text{sat}}\text{ in kPa and } T\text{ in K}\right)$ can be expressed as,
\begin{displaymath}
   \ln P_{1}^{\text{sat}} = 16.1952 - \frc{3423.53}{T-55.7152} \;\;\;\text{ and }\;\;\; \ln P_{2}^{\text{sat}} = 14.0568 - \frc{2825.42}{T-42.7089}
\end{displaymath}
     \end{example}

% SOLUTION
     \noindent{\bf Solution:}

   \begin{itemize}
      \item At 331 K,  $P_{1}^{\text{sat}}=$ 42.90 kPa and $P_{2}^{\text{sat}}=$ 70.54 kPa.
      \item For the liquid solution with $x_{1}=0.412$ and $x_{2}=1-x_{1}=0.588$, the Van Laar equation,
         \begin{eqnarray}
            \ln\gamma_{1} = \frc{2.409}{\left[1+\frc{2.409 x_{1}}{1.970 x_{2}}\right]^{2}} & \Longrightarrow & \gamma_{1} = 2.0111 \nonumber \\
            \ln\gamma_{2} = \frc{1.970}{\left[1+\frc{1.970 x_{1}}{2.409 x_{1}}\right]^{2}} & \Longrightarrow & \gamma_{2} = 1.5244 \nonumber
         \end{eqnarray}
      \item Partial pressure of ethanol and n-hexane are obtained from,
         \begin{displaymath}
             P_{1} = x_{1}\gamma_{1}P_{1}^{\text{sat}} = 35.55 \text{ kPa and }P_{2} = x_{2}\gamma_{2}P_{2}^{\text{sat}} = 63.22 \text{ kPa}
         \end{displaymath}
      \item The bubble point pressure is
         \begin{displaymath}
             P = P_{1} + P_{2} = 98.77 \text{ kPa}
         \end{displaymath}
      \item The composition of the vapour phase is
         \begin{displaymath}
            y_{1} = \frc{P_{1}}{P} = 0.3599\;\;\text{ and }\;\; y_{2} = \frc{P_{2}}{P} = 0.6401
         \end{displaymath}
   \end{itemize}
           
   \end{MyExample} 
 

\clearpage   
\begin{FinalSummaryBlock}{Summary}
     This aim of this chapter is to introduce definitions and applications of fugacity and activity to solution thermodynamics. This will lay down the strategy to solve problems involving vapour-liquid equilibrium (and solutions in particular) and very diluted solutions. 
    \begin{itemize}
       \item Fugacity of a real gas is formally defined as function of the chemical potential in Eqn.~\ref{Chapter:SolutionThermodynamics:Eqn:fugacity1b} as a representation of the pressure of an ideal gas that has the same chemical potential as the real gas;
       \item Lewis-Randall rules (Eqn.~\ref{Chapter:SolutionThermodynamics:Eqn:fugacity2a}) establishes a relationship between phase compositions and fugacities; 
       \item Activity coefficient of a chemical species (Eqn.~\ref{Chapter:SolutionThermodynamics:Eqn:activity1f}-\ref{Chapter:SolutionThermodynamics:Eqn:activity1g}) can be defined as an adjustment that relates the actual behaviour to the ideal behaviour of a solution at the same pressure and temperature conditions;
       \item A modified Raoult's equation was introduced in Eqn.~\ref{Chapter:SolutionThermodynamics:Eqn:RaoultLaw} to take into account non-idealities of the liquid solution;
       \item Gibbs-Duhem equation (Eqn.~\ref{Chapter:SolutionThermodynamics:Eqn:GibbsDuhem1e}) establishes a relationship between any thermodynamic potential, temperature, pressure and composition;
       \item Activity coefficient models are used to represent the thermodynamic behaviour of solutions in a similar way that equations of state are used to represent the PVT behaviour of gases.
    \end{itemize}
\end{FinalSummaryBlock}

%%%
%%% TUTORIAL 
%%%  
\clearpage  
\section{Problems}
\begin{MyTutorial}{}%{\begin{center}{\bf Tutorial}\end{center}}
%
  \begin{problem}\label{Chapter:SolutionThermodynamics:Problem:01} % Johannes T06E01
     For a lab experiment an amount of 2 litres of an antifreeze solution is required. The solution should consist of 30 mol$\%$ methanol in water. Determine the volumes of pure methanol and pure water at 25$^{\circ}$C that must be mixed to yield the 2 litres of solution. Partial molar volumes $\left(\overline{V}_{i}\right)$ for methanol and water in 30 mol$\%$ methanol solution and their pure-species molar volumes $\left(V_{i}\right)$, both at 25$^{\circ}$C, are,
     \begin{center}
     \begin{tabular}{l l c l} 
          Methanol (1): & $\overline{V}_{1}$=38.632 $\frac{\text{cm}^{3}}{\text{mol}}$ & \hspace{1cm} & $V_{1}$=40.727 $\frac{\text{cm}^{3}}{\text{mol}}$ \\
          Water (2):    & $\overline{V}_{2}$=17.765 $\frac{\text{cm}^{3}}{\text{mol}}$ & \hspace{1cm} & $V_{2}$=18.068 $\frac{\text{cm}^{3}}{\text{mol}}$ 
     \end{tabular}
     \end{center} 
     What would be the volume if an ideal solution were formed? 
  \end{problem}
%
  \begin{problem}\label{Chapter:SolutionThermodynamics:Problem:02} % Johannes T0602
      What is the change in entropy when 700 litres of CO$_{2}$ and 300 litres of N$_{2}$, each at 1 bar and 25$^{\circ}$C blend to form a gas mixture at the same conditions? Assume ideal gas behaviour and that {\it mole fraction = volume fraction}.
  \end{problem}
%
  \begin{problem}\label{Chapter:SolutionThermodynamics:Problem:03} % Johannes T0701
    The following expressions have been proposed for the partial molar properties of a particular binary mixture:
    \begin{displaymath}
        \overline{M}_{1} = M_{1} + A x_{2} \hspace{2cm} \overline{M}_{2} = M_{2} + A x_{1} 
    \end{displaymath}
    Here, parameter $A$ is a constant. Can these expressions possibly be correct? Explain.
  \end{problem}
%
  \begin{problem}\label{Chapter:SolutionThermodynamics:Problem:04} % Johannes T0702
     The volume change of mixing $\left(\text{in cm}^{3}\text{.mol}^{-1}\right)$ for the system ethanol (1) and methy-butyl ether (2) at 25$^{\circ}$C is given by the equation
     \begin{displaymath}
          \Delta V_{\text{mix}} = x_{1} x_{2}\left[-1.026+0.220\left(x_{1}-x_{2}\right)\right]
     \end{displaymath} 
     Given the molar volume of pure species, V$_{1}$=58.63 cm$^{3}$.mol$^{-1}$ and V$_{2}$=118.46 cm$^{3}$.mol$^{-1}$, what volume of mixture is formed when 750 cm$^{3}$ of pure species 1 is mixed with 1500 cm$^{3}$ of species 2 at 25$^{\circ}$C? What would be the volume if an ideal solution were formed?
  \end{problem}
%
  \begin{problem}\label{Chapter:SolutionThermodynamics:Problem:05}\citep{SmithVanNess_Book} % SM & VN 11.13
     The molar volume $\left(\text{in cm}^{3}\text{.mol}^{-1}\right)$ of a binary liquid mixture at $T$ and $P$ is given by:
     \begin{displaymath}
          V = 120 x_{1} + 70 x_{2} + \left( 15x_{1} + 8x_{2}\right)x_{1}x_{2} 
     \end{displaymath}
     \begin{enumerate}[a)]
          \item\label{first} Find expressions for the partial molar volumes of species 1 and 2 at $T$ and $P$.
          \item Show that when these expressions are combined in accord with $M=\sum\limits_{i}x_{i}\overline{M}_{i}$, the given equation for $V$ is recovered;
          \item Show that these expressions satisfy the Gibbs-Duhem equation, $\sum\limits_{i}x_{i}d\overline{M}_{i}=0$;
          \item Show that 
             \begin{displaymath}
                  \left(\frc{d\overline{V}_{1}}{dx_{1}}\right)_{x_{1}=1} = \left(\frc{d\overline{V}_{2}}{dx_{1}}\right)_{x_{1}=0} = 0
             \end{displaymath}
         \item Plot values of $V$, $\overline{V}_{1}$ and $\overline{V}_{2}$ calculated by the given equation for $V$ and by the equations developed in (a) {\it vs} $x_{1}$. Label points $\overline{V}_{1}^{\infty}$ and $\overline{V}_{2}^{\infty}$ and show their values.
     \end{enumerate} 
  \end{problem}
%
  \begin{problem}\label{Chapter:SolutionThermodynamics:Problem:06} %
     In generating expressions from $G^{E}/RT$ from VLE data, a convenient approach is to plot values of $G^{E}/\left(x_{1}x_{2}RT\right)$ {\it vs} $x_{1}$ and fitting results with an appropriate function. Consider if such data were fit by the expression,
   \begin{displaymath}
       \frc{G^{E}}{x_{1}x_{2} R T} = A + B x_{1}^{2}.
   \end{displaymath}
         From the expression $G^{E}/\left(x_{1}x_{2}RT\right)$, provide equations for the activity coefficient, $\ln{\gamma_{i}}$, as a function of $A$, $B$ and $x_{1}$, given
       \begin{displaymath}
               \ln{\gamma_{i}} = \frc{\overline{G}_{i}^{E}}{RT}.
       \end{displaymath} 
  \end{problem}
%
\end{MyTutorial}


%\part{Fundamental of Chemical Reactions Thermodynamics}
%  \chapter{Chemical Reaction Equilibrium}\label{Section:06}

%%% SECTION
\section{Introduction}\label{Section:06:Introduction}
In previous Modules, we have studied thermodynamic behaviour of gases and liquids at equilibrium conditions assuming that species do not react. However, in many industrial, environmental and bio-medical applications, chemical reactions are the driving force behind important processes. 

      \begin{wrapfigure}{r}{0.5\textwidth}%{figure}[hpt]
         \begin{center}
           \includegraphics[width=0.5\columnwidth,clip]{./Figs/Mod06_SchematicEqReaction_Temp_b}
           \caption{Sketch of equilibrium reaction and temperature.}\label{Mod06Fig01}
         \end{center}
       \end{wrapfigure}%{figure}
      Reactive systems are often characterised in terms of the maximum possible yield of the target product(s) at prescribed conditions, starting from the reactants (\ie raw materials). From {\it chemical kinetics} theory, rates of reaction and the reaction mechanisms\footnote{Here, it is important to define two terms commonly used in chemical kinetics theory:
        \begin{enumerate}[a)]
           \item {\it rate of reaction}: changes in the reactants concentrations with time, $\mfr[C]{}{\text{react}}=\mfr[C]{}{\text{react}}(t)$ (\ie `speed' of the reaction) and,
           \item {\it mechanisms of reactions}: stages of the reaction (in molecular level) that leads to the consumption of reactants and intermediate products, and production of final and intermediate products.
        \end{enumerate}
         }
        are often strongly influenced by temperature and pressure conditions. In several cases, reaction rates rise as temperature increases and in reactions involving gaseous phases, increasing of pressure may also lead to `faster' reactive conversions. In fact, reaction data from several applications (from combustion and geochemistry to biochemistry) show that conversion rates from reactants to products do not increase {\it monotonically}, but rather reach a {\it maximum} of conversion and smoothly decrease as schematically shown in Fig.~\ref{Mod06Fig01}.

Consider a chemical reaction in the gaseous phase,
     \reaction[chemreaction:reaction1]{A (g) + B (g) <=> C (g) + D (g)}

\noindent where initially $\left(\text{\ie at time } t_{0}=0\right)$ there are 1 mol of reactants $A$ and $B$ (each) in a {\it closed system}. Molecules of $A$ and $B$ are in continuous and random movement and collide with each other forming products $C$ and $D$. At this stage, the consumption/depletion of molecules of $A$ and $B$ dominates the dynamics in the system. However, as the reaction progresses $\left(\text{\ie at } t > t_{0}\right)$, the amount of molecules (\ie number of moles) of $C$ and $D$ increases and these species become dominant in the system. Collisions between $A$ and $B$ may still occur, but as they are no longer abundant, conversion from reactants to products get smaller, however molecules of $C$ and $D$ (both also in gaseous form) also collide and may produce $A$ and $B$. 

Thus, while initially the {\it forward} reaction dominates $\left(\text{\ie } A (g) + B (g) \rightarrow C (g) + D (g)\right)$, as reaction progress the {\it backward} reaction $\left(\text{\ie } A (g) + B (g) \leftarrow C (g) + D (g)\right)$ becomes increasingly relevant, which eventually leads into two reaction rates of equal value. After this stage, the concentration of all species in the system is invariant (\ie constant) with no tendency to change except if a perturbation is imposed to the system (\eg addition/removal of heat, increase/decrease of pressure, addition/removal of species etc). At such conditions, the reaction is said to be in {\it equilibrium}. The aim of the study of {\it chemical kinetics} is to investigate the mechanisms that leads to optimal reactions (\ie `faster' and with larger conversion ratio), whereas \blue{reaction thermodynamics} investigates the conditions at {\it equilibrium}.

In the application of {\it chemical reaction equilibrium} to problems relevant to industry, the aim is to ensure {\it maximum} conversion at prescribed pressure and temperature conditions. Reactions for which the conversion is (approximately) 100$\%$ are often referred as \blue{\it irreversible}, whereas reactions that have lower conversion rates are called \blue{\it reversible}. Chemical reactions that occur in a single phase (either solid, liquid or gas) are called \blue{\it homogeneous}, \eg
\begin{enumerate}[a)]
    \item formation of NO$_{\text{x}} \left(\text{NO and NO}_{2}\right)$ in atmospheric pollution:
        \begin{eqnarray}
           N_{2} (g) + O_{2} (g) &\Longleftrightarrow& 2 NO (g), \nonumber \\
           \frac{1}{2} N_{2} (g) + O_{2} (g) &\Longleftrightarrow& NO_{2} (g), \nonumber
        \end{eqnarray}
    \item {\it Fischer esterification} reaction of ethanol and acetic acid to produce ethyl acetate:
        \begin{displaymath}
           CH_{3}CH_{2}OH (l) + CH_{3}COOH (l) \Longleftrightarrow CH_{3}COOCH_{2}CH_{3} (l),
        \end{displaymath}
\end{enumerate}
Reactions that progress at different phases are called \blue{\it heterogeneous}, \eg
\begin{enumerate}[a)]
    \item strong acid-alkali reactions:
        \begin{eqnarray}
           H_{2}SO_{4} (\text{aq.}) + 2 NaOH (\text{aq.}) &\Longleftrightarrow&  Na_{2}SO_{4} (s) + H_{2}O (l) \nonumber \\
           HCl (\text{aq.}) + KOH (\text{aq.}) &\Longleftrightarrow& KCl (s) + H_{2}O (l), \nonumber \\
           2 HCl (\text{aq.}) + Mg(OH)_{2} (s)  &\Longleftrightarrow& MgCl_{2} (s) + 2 H_{2}O (l), \nonumber 
        \end{eqnarray}
    \item combustion of natural gas:
        \begin{displaymath}
           CH_{4} (g) + O_{2} (g) \Longleftrightarrow CO_{2} (g) + H_{2}O (l),
        \end{displaymath}
    \item reduction of solid oxides:
        \begin{displaymath}
           NiO (s) + H_{2} (g) \Longleftrightarrow Ni (s) + H_{2}O (l). 
        \end{displaymath}        
\end{enumerate}

Reaction conditions are, thus, critical to achieve the maximum possible conversion rate. The main aim of this module is to study the thermodynamic relations necessary to predict equilibrium conversion of {\it homogeneous chemical reactions}. 

%%% SECTION
\section{Reaction Coordinate}\label{Section:06:ReactionCoordinate}
\begin{subequations}
    For a chemical reaction in general form:
        \reaction[chemreaction:reaction]{\nu_{1} A_{1} + \nu_{2} A_{2}  + ... \nu_{k} A_{k} <=> \nu_{l} A_{l} + \nu_{m} A_{m} + ... + \nu_{\mathcal{C}} A_{\mathcal{C}} }
    where $\nu_{i}$ is the \blue{\it molar stoichiometric coefficient}, $A_{i}$ are chemical species and $\mathcal{C}$ is the total number of species in the system. For convention, \underline{positive $\nu_{i}$ stands for products} whereas \underline{negative $\nu_{i}$ is for reactants}, and the reaction can be expressed as an algebraic equation,
    \begin{displaymath}
       \nu_{l} A_{l} + \nu_{m} A_{m} + ... + \nu_{\mathcal{C}} A_{\mathcal{C}} - \left(\nu_{1} A_{1} + \nu_{2} A_{2}  + ... \nu_{k} A_{k}\right) = 0,
    \end{displaymath}
    or simply
    \begin{equation}
      \summation[\nu_{i}A_{i}]{i=1}{\mathcal{C}} = 0.\label{Mod6:Eqn:2}
    \end{equation}
\bigskip

    Let's consider, for example, the reaction for production of synthesis gas (or syngas, a mixture of hydrogen and carbon monoxide) from natural gas, 
        \reaction[chemreaction:syngas]{ CH_{4} + H_{2}O <=> CO + 3 H_{2} }
    the molar stoichiometric coefficients are $\nu_{CH_{4}}=\nu_{H_{2}O}=-1$, $\nu_{CO}=1$ and $\nu_{H_{2}}=3$, and the reaction can be written in equation form as,
    \begin{displaymath}
       CO + 3 H_{2} - CH_{4} - H_{2}O = 0.
    \end{displaymath}
    This {\it steam reforming reaction} indicates that as CH$_{4}$ is consumed (along with the water), CO and H$_{2}$ are produced. More specifically, if initially there are 1 mol of CH$_{4}$ and 1 mol of H$_{2}$O and 0 mol of CO and H$_{2}$,
     \begin{center}
        \begin{tabular}{ l | c c c c c c c}
                           & CH$_{4}$  & + & H$_{2}$O & $\Longleftrightarrow$ &  CO & + & 3 H$_{2}$ \\
                 $t = 0 $  &   1      &   &    1     &                       &  0 &    &   0      \\
              $t = t_{i} $  &   X      &   &    X     &                       &  X &    &   3X      \\
        \hline
                           &  1-X     &   &  1-X     &                       &   X  &    & 3X
     \end{tabular}
    \end{center}
    Thus, as reaction progress $X$ moles of CH$_{4}$ and H$_{2}$O are consumed whereas $X$ and $3X$ moles of CO and H$_{2}$, respectively, are formed. As the reaction needs to be stoichiometrically balanced,
    \begin{displaymath}
       \frc{d n_{CH_{4}}}{\nu_{CH_{4}}} = \frc{d n_{H_{2}O}}{\nu_{H_{2}O}} = \frc{d n_{CO}}{\nu_{CO}} = \frc{d n_{H_{2}}}{\nu_{H_{2}}},
    \end{displaymath}
    where $n_{i}$ is the number of moles of species $i$. This relation can be generalised for {\it any} chemical reaction as,
    \begin{displaymath}
       \frc{d n_{1}}{\nu_{1}} = \frc{d n_{2}}{\nu_{2}} = \cdots = \frc{d n_{\mathcal{C}}}{\nu_{\mathcal{C}}}.
    \end{displaymath}
    \begin{shaded}
       These equalities lead to the definition of \underline{reaction coordinate}, \blue{$\varepsilon$}, the {\it extension in which a reaction has progressed},
       \begin{equation}
          \frc{d n_{1}}{\nu_{1}} = \frc{d n_{2}}{\nu_{2}} = \cdots = \frc{d n_{\mathcal{C}}}{\nu_{\mathcal{C}}} = d\varepsilon.\label{Mod6:Eqn:2a}
       \end{equation}
       Assuming  that at $t=t_{0}=0$ the initial composition is $n_{i} = n_{i,0}$ and the reaction has not started yet, \ie $\varepsilon=0$. Integrating Eqn.~\ref{Mod6:Eqn:2a} from $t_{0}$ to $t_{i}$,
       \begin{eqnarray}
           && \int\limits_{n_{i,0}}^{n_{i}} dn_{i} = \nu_{i}\int\limits_{0}^{\varepsilon}d\varepsilon \nonumber \\
           && n_{i} = n_{i,0} + \nu_{i}\varepsilon,\;\;\;\;\forall i=\left\{1,2,\cdots,\mathcal{C}\right\}\label{Mod6:Eqn:2b}
       \end{eqnarray}
       For a total number of moles $n$, 
       \begin{equation}
            n = \summation[n_{i}]{i=1}{\mathcal{C}} = \summation[n_{i,0}]{i=1}{\mathcal{C}} + \varepsilon \summation[\nu_{i}]{i=1}{\mathcal{C}},\label{Mod6:Eqn:2c}
       \end{equation}
       or in compressed notation format,
       \begin{equation}
          n = n_{0} + \nu\varepsilon,\label{Mod6:Eqn:2d}
       \end{equation}
       where $n_{0}=\summation[n_{i,0}]{}{}$, $\nu=\summation[\nu_{i}]{}{}$ and $n=\summation[n_{i}]{}{}$. Now, we can define the composition (as mole fraction) of species $i$,
       \begin{equation}
          y_{i} = \frc{n_{i}}{n} = \frc{n_{i,0} + \nu_{i}\varepsilon}{n_{0} + \nu\varepsilon}.\label{Mod6:Eqn:2e}
       \end{equation}
       \blue{For an \underline{inert species} $k$, \ie chemical compound that does not react but is present in the reactive system, $\nu_{k}=0$.} 
    \end{shaded}

    From the initial steam reforming reaction example, let's assume that initially there are 2 moles of CH$_{4}$, 1 mol of H$_{2}$O, 1 mol of CO and 4 moles of H$_{2}$ in a closed system. Thus, the initial number of moles, $n_{0}=\summation[n_{i,0}]{}{} = 2+1+1+4 = 8$. Note that before the reaction starts there are already products in the system. The overall molar stoichiometric coefficient is
    \begin{displaymath}
         \nu = \summation[\nu_{i}]{}{} = -1 -1 + 1+ 3 = 2.
    \end{displaymath}
    The gaseous composition is expressed by Eqn.~\ref{Mod6:Eqn:2e},
    \begin{displaymath}
          y_{i} = \frc{n_{i,0} + \nu_{i}\varepsilon}{n_{0} + \nu\varepsilon}
          \begin{cases}
               y_{CH_{4}} = \frc{2-\varepsilon}{8+2\varepsilon},\;\;\; y_{H_{2}O} = \frc{1-\varepsilon}{8+2\varepsilon} \\
               y_{CO} = \frc{1 + \varepsilon}{8+2\varepsilon},\;\;\; y_{H_{2}} = \frc{4+\varepsilon}{8+2\varepsilon} \\
          \end{cases}
    \end{displaymath}

\bigskip

Given the set of 3 reactions below for the formation of CO and CO$_{2}$,
          \reaction[chemreaction:FormationCO2]{ C + O_{2} <=> CO_{2}}
          \reaction[chemreaction:FormationCO]{2C + O_{2} <=> 2CO } 
          \reaction[chemreaction:FormationCO-CO2]{2CO + O_{2} <=> 2CO_{2}}
    It is easy to notice that, \ref{chemreaction:FormationCO} + \ref{chemreaction:FormationCO-CO2} = \ref{chemreaction:FormationCO2}. In matricial form (adopting the stoichiometric notation),
    \begin{center}
      \begin{tabular}{ c | c c c c }
                                          & C   & O$_{2}$ & CO & CO$_{2}$ \\
\hline
       \ref{chemreaction:FormationCO2}    & -1  & -1     & 0   &   1 \\
       \ref{chemreaction:FormationCO}     & -2  & -1     & 2   &   0 \\
       \ref{chemreaction:FormationCO-CO2} & 0   & -1     & -2  &   2 
      \end{tabular}
    \end{center}
    Row 1 (\ref{chemreaction:FormationCO2}) is a linear combination of rows 2 and 3 (\ie Row 1 = 0.5 $\times$ Row 2 + 0.5 $\times$ Row 3). This indicates that 2 out of the 3 rows of the matrix are {\it linearly independent} and one of them is {\it linearly dependent}, \ie it can be written as a linear combination of the other two rows. Such linear algebra fundamental concept, linear independency\footnote{A finite set $\mathcal{S}=\left\{\mathbf{x_{1}}, \mathbf{x_{2}}, \cdots, \mathbf{x_{m}}\right\}$ of vectors in $\mathbb{R}^{n}$ space is said to be {\bf linearly dependent} if there exist scalars (\ie real numbers), $\alpha_{1}, \alpha_{2}, \cdots, \alpha_{m}$ (not all of them equal to zero), such that
    \begin{displaymath}
       \alpha_{1}\mathbf{x_{1}} + \alpha_{2}\mathbf{x_{2}} + \cdots + \alpha_{m}\mathbf{x_{m}} = 0.
    \end{displaymath}
    If a set of vectors is said to be linearly dependent, then at least one vector can be expressed as a {\it linear combination} of the others. A good review of linear dependence in vector space can be found in \href{http://linear.ups.edu/html/section-LI.html}{http://linear.ups.edu/html/section-LI.html}.} in vector space, is extended to define independent chemical reactions, \ie the smallest set of reactions that, when linearly combined results in {\it all possible chemical reactions} among the species present in the system. 

    In the following set of reactions,
          \reaction[chemreaction:O2]{ 2 O <=> O_{2}} 
          \reaction[chemreaction:FormationCO-b]{ C + O <=> CO }
          \reaction[chemreaction:FormationCO2-b]{C + 2O <=> CO_{2}}
    none of them can be expressed as a linear combination of the remaining reactions. These are, thus, {\it independent reactions}, and several chemical reactions can be written as combinations of this set. Combining these reactions to eliminate reaction~\ref{chemreaction:O2} (fusion of atomic oxygen),
          \begin{displaymath}
             \text{Independent reactions:}
                \begin{cases}
                   2 C + O_{2} \Longleftrightarrow 2 CO \;\;\;\;\; (2\ref{chemreaction:FormationCO-b} - \ref{chemreaction:O2}) \\
                   C + O_{2} \Longleftrightarrow CO_{2} \;\;\;\;\;\;\; (\ref{chemreaction:FormationCO2-b}-\ref{chemreaction:O2})
                \end{cases}
          \end{displaymath}
    provide a reduction of the order of the system (from three reactions/equations to just 2 independent reactions/equations). Other sets of 2 reactions can be readily obtained from linear combinations of reactions~\ref{chemreaction:O2}-~\ref{chemreaction:FormationCO2-b}, \eg
          \begin{displaymath}
             \text{Independent reactions:}
             \begin{cases}
                CO_{2} \Longleftrightarrow CO + O \;\;\;\; (\ref{chemreaction:FormationCO-b}-\ref{chemreaction:FormationCO2-b}) \\
                2O  \Longleftrightarrow O_{2},
             \end{cases}
          \end{displaymath}
    These reactions results in single reaction: 
          \reaction[chemreaction:CO2-CO-O2]{2CO_{2} <=>  2CO + O_{2}}
    \begin{shaded}
       \noindent Thus, for a set of \underline{independent chemical reactions}, Eqn.~\ref{Mod6:Eqn:2b} can be rewritten as,
          \begin{equation}
             n_{i} = n_{i,0} + \summation[\nu_{ij}\varepsilon_{j}]{j}{},\label{Mod6:Eqn:2f}
          \end{equation}
       where $i$ refers to chemical species, whereas $j$ refers to reactions,
          \begin{displaymath}
             n = \summation[n_{i,0}]{i}{} + \summation[\summation[\nu_{ij}\varepsilon_{j}]{i}{}]{j}{} = n_{0} + \summation[\left(\summation[\nu_{ij}]{j}{}\right)\varepsilon_{j}]{i}{},
          \end{displaymath}
       For simplicity of notation, let's use $\nu_{j}=\summation[\nu_{ij}]{i}{}$, thus
          \begin{equation}
             n = n_{0} + \summation[\nu_{j}\varepsilon_{j}]{j}{}.\label{Mod6:Eqn:2g}
          \end{equation}
       Now, for composition of species $i$ in multiple reactions $j$,
       \begin{equation}
          y_{i} = \frc{n_{i}}{n} = \frc{n_{i,0} + \summation[\nu_{ij}\varepsilon_{j}]{j=1}{\mathcal{R}}}{n_{0} + \summation[\nu_{j}\varepsilon_{j}]{j=1}{\mathcal{R}}},\label{Mod6:Eqn:2h}
       \end{equation}
       where $\mathcal{R}$ is the number of independent chemical reactions.
    \end{shaded}

\bigskip

    As an example, let's consider the steam reforming reactions,
          \begin{displaymath}
              \begin{cases}
                  CH_{4} + H_{2}O \Longleftrightarrow CO + 3 H_{2}, \\
                  CH_{4} + 2H_{2}O \Longleftrightarrow CO_{2} + 4 H_{2}.
              \end{cases}
          \end{displaymath}
    Consider that at $t=0$, there are 2 moles of methane and 3 moles of water in the system $\left(\text{\ie } n_{0}=5\right)$. Let's obtain expressions for the gaseous compositions. As a initial step, a table with species ($i$) and reactions ($j$) may be expressed as,
    \begin{center}
       \begin{tabular}{ c | c c c c c | c} 
          $(i=)$  & CH$_{4}$ & H$_{2}$O & CO & CO$_{2}$ & H$_{2}$ &  \\
\hline
            $j$   &         &         &     &         &        & $\nu_{j}$ \\
\hline
             1    &   -1    &    -1   &   1 &    0    &   3    &    2 \\
             2    &   -1    &    -2   &   0 &    1    &   4    &    2 \\
       \end{tabular}
    \end{center}
    therefore,
      \begin{displaymath}
         n = n_{0} + \summation[\nu_{j}\varepsilon_{j}]{j}{} = 5 + 2\varepsilon_{1} + 2\varepsilon_{2},
      \end{displaymath}
    and compositions are (from Eqn.~\ref{Mod6:Eqn:2h}),
    \begin{displaymath}
        \begin{cases}
            y_{CH_{4}} = \frc{2 - \varepsilon_{1} - \varepsilon_{2}}{5 + 2\varepsilon_{1} + 2\varepsilon_{2}},\;\;\; y_{H_{2}O} = \frc{3 - \varepsilon_{1} - 2\varepsilon_{2}}{5 + 2\varepsilon_{1} + 2\varepsilon_{2}}, \\
            y_{CO}   = \frc{\varepsilon_{1}}{5 + 2\varepsilon_{1} + 2\varepsilon_{2}},\;\;\; y_{CO_{2}} = \frc{\varepsilon_{2}}{5 + 2\varepsilon_{1} + 2\varepsilon_{2}} \;\;\text{ and }\;\;\;y_{H_{2}} = \frc{3\varepsilon_{1}+4\varepsilon_{2}}{5 + 2\varepsilon_{1} + 2\varepsilon_{2}}  \\
        \end{cases}
    \end{displaymath}

\end{subequations}

%%% SECTION
\section{Standard Enthalpy of Reaction}\label{Section:06:EnthalpyGibbsReaction}
\begin{subequations}
   Given the homogeneous chemical reaction,
     \reaction[chemreaction:simplehomogeneousreaction]{\nu_{A}A + \nu_{B}B <=> \nu_{C}C + \nu_{D}D}
   the {\it standard enthalpy of reaction}, $\Delta H^{\circ}_{\text{r}}$ is the heat associated with a chemical reaction that occurs at constant temperature $T$ and at {\it standard pressure} $P$ of 101.325 kPa (\ie 1 atm). In other words, it is the change in enthalpy that occurs when $\nu_{A}$ moles of $A$ and $\nu_{B}$ moles of $B$ at standard state and temperature $T$ are fully converted into $\nu_{C}$ moles of $C$ and $\nu_{D}$ moles of $D$ at standard states and at the same temperature $T$. {\it Standard states} commonly used are:
   \begin{enumerate}[a)]
       \item Gases: pure substance in the ideal gas state at 1 atm;
       \item Liquids and solids: pure liquid or solid at 1 atm.
   \end{enumerate}
   In the literature, data on standard enthalpy of reaction is typically reported at 25$^{\circ}$C. Using the sign convention for molar stoichiometric coefficient, the standard enthalpy of reaction is
   \begin{equation}
      \Delta H^{\circ}_{\text{r}} = \summation[\nu_{i}\Delta H^{\circ}_{i,f}]{i}{},\label{Mod6:Eqn:3a}
   \end{equation}
   where $\Delta H^{\circ}_{i,f}$ is the {\it standard enthalpy of formation}, \ie the standard enthalpy of a reaction in which {\it one mol of a given substance is formed from elements}. The elements are assumed to react at the most stable phase at a given temperature and standard pressure conditions. Data basis of the {\it standard enthalpy of formation} for common chemical species can be found in any chemical engineering handbook and on thermodynamic textbooks.

   In a similar way, we can extend Eqn.~\ref{Mod6:Eqn:3a} to all thermodynamic functions, $M=\left\{U, H, A, G, S\right\}$, as
   \begin{equation}
      \Delta M^{\circ}_{\text{r}} = \summation[\nu_{i}\Delta M^{\circ}_{i,f}]{i}{}.\label{Mod6:Eqn:3b}
   \end{equation}
  
\end{subequations}



%%% SECTION
\section{Criteria for Chemical Reaction Equilibrium}\label{Section:06:CriteriaEquibrium}
\begin{subequations}

   For simplicity of the analysis, let's assume {\it single homogeneous chemical reactions}. The total Gibbs free energy of a system is obtained from
     \begin{equation}
         G^{t} = \summation[n_{i}\overline{G}_{i}]{i=1}{\mathcal{C}} = \summation[\left(n_{i,0}+\nu_{i}\varepsilon\right)\overline{G}_{i}]{i=1}{\mathcal{C}}\label{Mod6:Eqn:4a}
     \end{equation}
   where $\overline{G}_{i}$ is the partial molar Gibbs free energy. At equilibrium (Fig.~\ref{Mod06Fig02}), $\left(dG^{t}\right)_{T,P} = 0$, \ie
       \begin{eqnarray}
          && dG^{t} = d(n\overline{G}) = nd\overline{G} + \overline{G}dn \;\;\;\; \blue{\times \frc{1}{d\varepsilon}} \nonumber \\
          && \frc{dG^{t}}{d\varepsilon} = n\frc{d\overline{G}}{d\varepsilon} + G\frc{\overline{G} dn}{d\varepsilon} = 0 \nonumber \\
          && \text{in a closed system, } dn = 0 \Longrightarrow \frc{dG^{t}}{d\varepsilon} = \frc{d\overline{G}}{d\varepsilon} = 0\;\;\text{ (equilibrium criteria).} \nonumber
       \end{eqnarray}
      \begin{figure}[hpt]
         \begin{center}
           \includegraphics[width=0.5\columnwidth,clip]{./Figs/Mod06_SchematicEquilibriumGibbs_b}
           \caption{Sketch of equilibrium Gibbs free energy.}\label{Mod06Fig02}
         \end{center}
      \end{figure} 
   Differentiating Eqn.~\ref{Mod6:Eqn:4a} \wrt $\varepsilon$,
       \begin{displaymath}
           \frc{dG^{t}}{d\varepsilon} = \cancelto{0}{\frc{d}{d\varepsilon}\summation[n_{i,0}\overline{G}_{i}]{i}{}} + \frc{d}{d\varepsilon}\left[\summation[\left(\nu_{i}\varepsilon\right)\overline{G}_{i}]{i}{}\right] = 0,
       \end{displaymath}
   where the summation in brackets can be simplified as,
       \begin{displaymath}
           \frc{d}{d\varepsilon}\left(\nu_{i}\varepsilon \overline{G}_{i}\right) = \nu_{i}G_{i}\cancelto{1}{\frc{d\varepsilon}{d\varepsilon}} + \nu_{i}\varepsilon\cancelto{0}{\frc{d \overline{G}_{i}}{d\varepsilon}} = \nu_{i}\overline{G}_{i}
       \end{displaymath} 
   with $\nu_{i}$ constant, leading to
        \begin{shaded}
          \begin{equation}
             \summation[\nu_{i}\overline{G}_{i}]{i}{} = 0 = \summation[\nu_{i}\mu_{i}]{i}{}.\label{Mod6:Eqn:4b}
          \end{equation}
        \end{shaded}
\end{subequations}


%%% SECTION
\section{Equilibrium Constant of Reactions}\label{Section:06:EquilibriumConstantReactions}
\begin{subequations}

   The {\it chemical potential} was previously defined for mixtures (see Eqn.~\ref{Mod05_fugacity1b}) as,
       \begin{displaymath}
           \mu_{i} = \Partial[\left(nG\right)]{n_{i}}{T,P,n_{j\ne i}} = \overline{G}_{i} = RT\ln{\overline{f}_{i}} + C_{i}(T),
       \end{displaymath}
   where $\overline{f}_{i}$ is the fugacity of component $i$ in the mixture\footnote{In Module~\ref{Section:05}, we have studied forms to calculate $\mfr[\overline{f}]{i}{j}$:
      \begin{enumerate}[a)]
          \item ideal gases: $\mfr[\overline{f}]{i}{\text{ig}} =  y_{i}P=P_{i}$ (Dalton law);
          \item real gases: $\mfr[\overline{\phi}]{i}{V}  = \frc{\mfr[\overline{f}]{i}{V}}{y_{i}P}$;
          \item ideal solution: $\mfr[\overline{f}]{i}{\text{id}} = x_{i}\mfr[f]{i}{L}$ (Lewis-Randall rule);
          \item real solution: $\mfr[\overline{f}]{i}{L} = x_{i}\gamma_{i}\mfr[f]{i}{L}$.
      \end{enumerate}
}. At {\it standard state} (\ie pure component at 1 atm), the partial molar Gibbs free energy
      \begin{displaymath}
            \overline{G}_{i}^{\circ}\left(T,P=1\text{ atm}, x_{i}^{\circ}\right) = RT\ln{\overline{f}_{i}^{\circ}} + C_{i}(T),
      \end{displaymath}
      where $x_{i}^{\circ}$ is usually assumed to be composition of the pure component $\left(\text{\ie }x_{i}^{\circ}=1\right)$, or infinite diluition state $\left(\text{\ie }x_{i}^{\circ}=0\right)$ or ideal 1 molal solution, depending on the nature of the species\footnote{For simplicity of the notation, in this section, $x_{i}$ will represent the mole fraction of species in liquid and vapour phases.}.  $\overline{f}_{i}^{\circ}$ is the fugacity of component $i$ in the mixture at standard conditions. Subtracting the equation above from the previous one,
      \begin{equation}
         \overline{G}\left(T,P, x_{i}\right)  - \overline{G}_{i}^{\circ}\left(T,P=1\text{ atm}, x_{i}^{\circ}\right) = RT\ln{\frc{\overline{f}_{i}}{\overline{f}_{i}^{\circ}}}.\label{Mod6:Eqn:5a}
      \end{equation}
      The log-term was previously (Eqn.~\ref{Mod05_activity1b}) defined as the activity, \ie $a_{i} = \frc{\overline{f}_{i}}{\overline{f}_{i}^{\circ}}$, and if Eqn.~\ref{Mod6:Eqn:5a} is substituted in Eqn.~\ref{Mod6:Eqn:4b},
      \begin{displaymath}
         0 = \summation[\nu_{i}\overline{G}_{i}]{i}{} = \underbrace{\summation[\nu_{i}\overline{G}_{i}^{\circ}]{i}{}}_{\Delta G_{r}^{\circ}} + RT\summation[\nu_{i}\ln{a_{i}}]{i}{}
      \end{displaymath}
      where $\Delta G_{r}^{\circ} = \summation[\nu_{i}\overline{G}_{i}^{\circ}\left(T,P=1\text{ atm}, x_{i}^{\circ}\right)]{i}{}$ is the Gibbs energy change on reaction with each species (reactants and products) in its standard state or state of unity activity. Using logarithm properties, we can simplify this expression and introduce the {\it product}, $\prod$,\footnote{See Appendix C for log-properties. In summary, for a set of reals $a_{i}$, $b_{i}$ and $n_{i}$ $\forall i\in\left\{1,2,\cdots, z\right\}$, with $b_{i}\ne 0$,
         \begin{eqnarray}
           \summation[n_{i}\ln{\frc{a_{i}}{b_{i}}}]{i=1}{z} =  \summation[\ln{\left(\frc{a_{i}}{b_{i}}\right)^{n_{i}}}]{i=1}{z} &=& \ln{\left(\frc{a_{1}}{b_{1}}\right)^{n_{1}}} + \ln{\left(\frc{a_{2}}{b_{2}}\right)^{n_{2}}}  + \ln{\left(\frc{a_{3}}{b_{3}}\right)^{n_{3}}} + \cdots + \ln{\left(\frc{a_{z}}{b_{z}}\right)^{n_{z}}} \nonumber \\
                                                                                 &=& \ln{\left[\left(\frc{a_{1}}{b_{1}}\right)^{n_{1}} \cdot \left(\frc{a_{2}}{b_{2}}\right)^{n_{2}} \cdot \left(\frc{a_{3}}{b_{3}}\right)^{n_{3}} \cdots \left(\frc{a_{z}}{b_{z}}\right)^{n_{z}} \right]}  \nonumber \\
                                                                                 &=& \ln{\left[\prod\limits_{i=1}^{z} \left(\frc{a_{i}}{b_{i}}\right)^{n_{i}}\right]}  \nonumber
                    \end{eqnarray}
}
      \begin{shaded}
         \begin{displaymath}
             \ln{\prod\limits_{i=1}^{\mathcal{C}} a_{i}^{\nu_{i}}} = - \frc{\Delta G_{r}^{\circ}}{RT} \;\;\;\;\;\; \Longleftrightarrow \;\;\;\;\; \prod\limits_{i=1}^{\mathcal{C}} a_{i}^{\nu_{i}} = \underbrace{\exp{\left[- \frc{\Delta G_{r}^{\circ}}{RT}\right]}}_{K},
         \end{displaymath}
         \blue{$K$} is the \blue{\it equilibrium constant} and is a function of the system temperature, \ie $K=K(T)$,
         \begin{equation}
             K(T) = \exp{\left[- \frc{\Delta G_{r}^{\circ}}{RT}\right]} = \prod\limits_{i=1}^{\mathcal{C}} a_{i}^{\nu_{i}}. \label{Mod6:Eqn:5b}
         \end{equation}
         If the standard state of each chemical species in the reaction is chosen to be $T=298.15$ K and $P= 1$ atm,
         \begin{displaymath}
              \Delta G_{r}^{\circ} \left(T=298.15\text{ K}\right) = \summation[\nu_{i}\Delta G_{i,f}^{\circ}\left(T=298.15\text{ K}\right)]{i}{},
         \end{displaymath}
         where $\Delta G_{i,f}^{\circ}$ is the standard state Gibbs free energy of formation, briefly discussed in Section~\ref{Section:06:EnthalpyGibbsReaction}.
      \end{shaded}


%%% SUBSECTION
\subsection{Dependence of $K$ with Temperature}\label{Section:06:K_Temperature}
      The equilibrium constant $K$ is strongly dependent on the system temperature, and therefore it is important to be able to define it at any reaction conditions. The {\it Van't Hoff} equation provides a way to correlate $K$ and the enthalpy of reaction, using Eqn.~\ref {Mod03_GibbsGeneratingFunctionPConst} (for constant pressure conditions) at standard state, 
      \begin{displaymath}
          \overline{H}_{i}^{\circ} = -RT^{2}\frc{d}{dT}\left(\overline{G}_{i}^{\circ}/RT\right) \;\;\;\;\Longrightarrow\;\;\;\;\; \Delta H^{\circ}_{r} = -RT^{2}\frc{d}{dT}\left(\Delta G^{\circ}_{r}/RT\right),
      \end{displaymath}
      where $\Delta H^{\circ}_{r}$ is the standard heat (or enthalpy) of reaction. 
  
      \begin{shaded}
      \noindent Replacing Eqn.~\ref{Mod6:Eqn:5b} in the relation above leads to the \blue{\it Van't Hoff equation},
      \begin{equation}
         \frc{d\left(\ln{K}\right)}{dT} = \frc{\Delta H^{\circ}_{r}}{RT^{2}} \;\;\;\;\;\text{ or }\;\;\;\;\; \frc{d}{dT}\left(\Delta G^{\circ}_{r}/RT\right) = -\frc{d\left(\ln{K}\right)}{dT} = -\frc{\Delta H^{\circ}_{r}}{RT^{2}}\label{Mod6:Eqn:5c}
      \end{equation}
      \end{shaded}

      \noindent The Van't Hoff equation can be integrated between two temperature levels,
      \begin{displaymath}
           \int\limits_{K_{1}}^{K_{2}}d\left(\ln{K}\right) = \int\limits_{T_{1}}^{T_{2}}\frc{\Delta H^{\circ}}{RT^{2}} dT \;\;\;\Longrightarrow \;\;\; \left(\ln{K_{2}}-\ln{K_{1}}\right) = -\frc{\Delta H^{\circ}}{R}\left(\frc{1}{T_{2}}-\frc{1}{T_{1}}\right).
      \end{displaymath}
      For a qualitative analysis, we can plot $\ln{K} \times 1/T$, and the slope of the linear equation $\left(\text{\ie } y=ax+b,\text{ where }\right.$ $\left.y=\ln{K}\text{ and }x=\frac{1}{T}\right)$ is $-\Delta H^{\circ}/R$. Let's consider reactions for formation of carbon and nitrogen monoxides,
         \reaction[chemreaction:FormationCO]{C + \frac{1}{2}O_{2} <=> CO}
         \reaction[chemreaction:FormationNO]{\frac{1}{2}N_{2} + \frac{1}{2}O_{2} <=> NO}
      with $\ln{K} \times 1/T$ plot for these reactions in Fig.~\ref{Mod06Fig03}. We can qualitatively analyse the behaviour of the equilibrium constant at 1666.67 K and 833.33 K,
      \begin{figure}[hpt] 
         \begin{center}
           \includegraphics[width=0.5\columnwidth,angle=90,clip]{./Figs/Mod06_SchematicVantHoff_b}
           \caption{(Left) Sketch of equilibrium constants as a function of temperature for reactions~\ref{chemreaction:FormationCO} and~\ref{chemreaction:FormationNO}. Data extracted from Smith, Van Ness and Abbott (6$^{\text{th}}$ Edition). (Right) Reaction coordinate $\times$ temperature -- qualitative analysis for endothermic/exothermic reactions based on the standard heat of reaction.}\label{Mod06Fig03}
         \end{center}
      \end{figure} 
     
      \begin{eqnarray}
         && -\left[\frc{\Delta H^{\circ}}{R}\right]_{\ref{chemreaction:FormationCO}} =  \frc{27 - 19}{12-6} = 1.3333, \nonumber \\
         && -\left[\frc{\Delta H^{\circ}}{R}\right]_{\ref{chemreaction:FormationNO}} =  \frc{-12 - (-5)}{12-6} = -1.1667, \nonumber
      \end{eqnarray}
      Reactions that present positive $\Delta H^{\circ}$ (\ie negative slope) are called {\it endothermic}, \ie equilibrium constant rises as the temperature is increased (or in other words, heat is given to the reactive system to promote the reaction). Similarly, negative $\Delta H^{\circ}$ (\ie positive slope) are called {\it exothermic}, \ie an increase in temperature leads to a decrease in the equilibrium constant (heat is transferred from the reactive system to the surroundings).

\begin{comment}
%%% SUBSECTION
\subsection{Dependence of $K$ with Temperature}\label{Section:06:K_Temperature}
\begin{subequations}
    The fundamental thermodynamic relation for the Gibbs free energy at stadard state conditions,
       \begin{displaymath}
           G_{i}^{\circ} = H_{i}^{\circ} - TS_{i}^{\circ},
       \end{displaymath}
   can be extended to chemical reactions at standard state conditions (using Eqn.~\ref{Mod6:Eqn:3b})
       \begin{equation}
           \underbrace{\summation[\nu_{i}G_{i}^{\circ}]{}{}}_{\Delta G^{\circ}} =  \underbrace{\summation[\nu_{i}H_{i}^{\circ}]{}{}}_{\Delta H^{\circ}} - T \underbrace{\summation[\nu_{i}S_{i}^{\circ}]{}{}}_{\Delta S^{\circ}},\label{Mod6:Eqn:6a}
       \end{equation}
   where,
   \begin{enumerate}[i)]
       \item standard heat (enthalpy) of reaction $\left(\Delta H^{\circ}\right)$: starting from the definition of heat capacity at constant pressure, $C_{p}=\Partial[H]{T}{P}$, at standard conditions,
            \begin{eqnarray}
                 && dH_{i}^{\circ} = C_{p_{i}}^{\circ} dT \hspace{4cm}\blue{\left(\times \summation[\nu_{i}]{}{}\right)} \nonumber \\
                 && \summation[\nu_{i}dH_{i}^{\circ}]{}{} = \summation[\nu_{i}C_{p_{i}}^{\circ} dT]{}{} \nonumber \\
                 && \summation[d\left(\nu_{i}H_{i}^{\circ}\right)]{}{} = d\underbrace{\summation[\nu_{i}H_{i}^{\circ}]{}{}}_{\Delta H^{\circ}} = \underbrace{\summation[\nu_{i}C_{p_{i}}^{\circ}]{}{}}_{\Delta C_{p}^{\circ}}dT \nonumber \\
                 && \int d\left(\Delta H^{\circ}\right) = \int \Delta C_{p}^{\circ} dT \nonumber \\
                 && \Delta H^{\circ} - \Delta H^{\circ}_{0} = \blue{R}\int\limits_{T_{0}}^{T} \frc{\Delta C_{p}^{\circ}}{\blue{R}} dT \nonumber
            \end{eqnarray}
            here $R$ is acting as an {\it artifact} that will vanish in a later stage.

       \item standard entropy of reaction $\left(\Delta S^{\circ}\right)$: from the relation $\Partial[S]{T}{P}=\frc{C_{p}}{T}$,
            \begin{eqnarray}
                 && dS_{i}^{\circ} = C_{p_{i}}^{\circ} \frc{dT}{T} \nonumber \\
                 && \int d\left(\Delta S^{\circ}\right) = \int\limits_{T_{0}}^{T}\Delta C_{p}^{\circ}\frc{dT}{T} \nonumber \\
                 && \Delta S^{\circ} - \Delta S^{\circ}_{0} = \blue{R}\int\limits_{T_{0}}^{T}\frc{\Delta C_{p}^{\circ}}{\blue{R}} \frc{dT}{T} \nonumber 
            \end{eqnarray}
   \end{enumerate}
  Replacing these two terms in Eqn.~\ref{Mod6:Eqn:6a} to obtain $\Delta G^{\circ}$
     \begin{eqnarray}
         \Delta G^{\circ} &=&  \Delta H^{\circ}_{0} + \blue{R}\int\limits_{T_{0}}^{T} \frc{\Delta C_{p}^{\circ}}{\blue{R}} dT - T\overbrace{\Delta S^{\circ}_{0}}^{\frc{\Delta H^{\circ}_{0}-\Delta G^{\circ}_{0}}{T_{0}}} - \blue{R}T\int\limits_{T_{0}}^{T}\frc{\Delta C_{p}^{\circ}}{\blue{R}} \frc{dT}{T} \;\;\;\;\; \blue{\times\left(\frc{1}{RT}\right)} \nonumber \\ 
         \underbrace{\frc{\Delta G^{\circ}}{RT}}_{-\ln{K}} &=&  \frc{\Delta G^{\circ}_{0}-\Delta H^{\circ}_{0}}{RT_{0}} + \frc{\Delta H^{\circ}_{0}}{RT} + \frc{1}{T}\int\limits_{T_{0}}^{T} \frc{\Delta C_{p}^{\circ}}{R}dT - \int\limits_{T_{0}}^{T}\frc{\Delta C_{p}^{\circ}}{R}dT \label{Mod6:Eqn:6b}
     \end{eqnarray}
  The standard heat capacities, $\Delta C_{p}^{\circ}=\summation[\nu_{i}C_{p_{i}}^{\circ}]{}{}$ are often expressed as polynomials of the temperature for a range of chemical species. This ensures a sensitivity of this thermophysical parameter with respect to the temperature. In such cases, Eqn.~\ref{Mod6:Eqn:6b} can only be solved numerically with the integrals ranging from the standard (or reference) temperature, $T_{0}$, to $T$. In addition, $\Delta G^{\circ}_{0}$ and $\Delta H^{\circ}_{0}$, Gibbs free energy and enthalpy of reaction at standard pressure conditions (\ie 1 atm) and at reference (or standad) temperature conditions (\ie 298.15 K) are tabulated and readily available in the literature for a number of (relatively) simple chemical reactions.

\end{subequations}
\end{comment}

%%% SUBSECTION
\subsection{Dependence of $K$ with Composition}\label{Section:06:K_Composition}
    Recall that in Eqn.~\ref{Mod6:Eqn:5b}, the equilibrium constant $K$ was defined as function of the activities of the reactive mixture,
    \begin{displaymath}
       K = \prod\limits_{i=1}^{\mathcal{C}} a_{i}^{\nu_{i}} = \prod\limits_{i=1}^{\mathcal{C}} \left(\frc{\overline{f}_{i}}{\overline{f}_{i}^{\circ}}\right)^{\nu_{i}}.
    \end{displaymath}
    As the standard state of a gas is the ideal gas state at $P^{\circ}=1\text{ atm}$, the fugacity of an ideal gas in the standard state is equal to its pressure $\overline{f}_{i}^{\circ}=P^{\circ}$ for each reactive species $i$, thus
    \begin{displaymath}
       K = \prod\limits_{i=1}^{\mathcal{C}} \left(\frc{\overline{f}_{i}}{P^{\circ}}\right)^{\nu_{i}},
    \end{displaymath}
    however, for the gaseous phase $\overline{f}_{i} = \overline{\phi}_{i}y_{i}P$,
    \begin{displaymath}
        K = \prod\limits_{i=1}^{\mathcal{C}} \left(\frc{\overline{\phi}_{i}y_{i}P}{P^{\circ}}\right)^{\nu_{i}} = \left[\prod\limits_{i=1}^{\mathcal{C}} \left(\overline{\phi}_{i}y_{i}\right)^{\nu_{i}}\right]\left(\frc{P}{P^{\circ}}\right)^{\summation[\nu_{i}]{i}{} }.
    \end{displaymath}
    \begin{shaded}    
       \noindent Defining $\nu=\summation[\nu_{i}]{i}{}$,
       \begin{equation}
           \prod\limits_{i=1}^{\mathcal{C}} \left(\overline{\phi}_{i}y_{i}\right)^{\nu_{i}} = K\left(\frc{P}{P^{\circ}}\right)^{-\nu}\;\;\;\;\blue{\text{ (for gaseous phase).}}\label{Mod6:Eqn:5d}
       \end{equation}
       At low pressure or high temperature, the fluid behaves as an ideal gas, \ie $\overline{\phi}_{i}=1$, thus
       \begin{equation}
           \prod\limits_{i=1}^{\mathcal{C}} y_{i}^{\nu_{i}} = K\left(\frc{P}{P^{\circ}}\right)^{-\nu}\;\;\;\;\blue{\text{ (for ideal gases).}}\label{Mod6:Eqn:5e}
       \end{equation}
    \end{shaded}
    Now, for the liquid phase the expression previously defined 
    \begin{displaymath}
       K = \prod\limits_{i=1}^{\mathcal{C}} \left(\frc{\overline{f}_{i}}{\overline{f}_{i}^{\circ}}\right)^{\nu_{i}},
    \end{displaymath}
    still holds, and for {\it pure liquid} $i$ at $T$ and 1 atm, using $\overline{f}_{i} = \gamma_{i}x_{i}f_{i}$ (Eqn.~\ref{Mod05_activity1g}), where $f_{i}$ is the fugacity of pure component $i$ in the liquid phase at $P$ and $T$. Hence,
    \begin{displaymath}
       \frc{\overline{f}_{i}}{\overline{f}_{i}^{\circ}} = \frc{\gamma_{i}x_{i}f_{i}}{\overline{f}_{i}^{\circ}} = \gamma_{i}x_{i}\left(\frc{f_{i}}{\overline{f}_{i}^{\circ}}\right).
    \end{displaymath}
    The fugacity ratio in brackets can be obtained by manipulating the Maxwell relation, $\Partial[G]{P}{T}=V$, and integrating from standard state to $T$ and $P$,
    \begin{eqnarray}
          G_{i}-G_{i}^{\circ} &=& \int\limits_{P^{\circ}}^{P}V_{i}dP \nonumber  \\
          RT\ln{\left(\frc{f_{i}}{f_{i}^{\circ}}\right)} &=& \int\limits_{P^{\circ}}^{P}V_{i}dP \;\;\;\Longrightarrow\;\;\; \ln{\left(\frc{f_{i}}{f_{i}^{\circ}}\right)} = \frc{V_{i}\left(P-P^{\circ}\right)}{RT}, \nonumber
    \end{eqnarray}
    assuming that fluids at liquid phase are incompressible $\left(\text{\ie } V_{i} \text{ is constant}\right)$. Now, replacing in the original equation\footnote{Here, properties of exponents (see Appendix C) are used to operate over the thermodynamic properties. Given the set of reals $a_{i}$ and $b_{i}$ $\forall i\in\left\{1,2,\cdots, z\right\}$ and constants $c$ and $d$:
   \begin{enumerate}[a)]
       \item $\left(e^{c}\right)^{d} = e^{cd}$;
       \item $e^{a_{1}b_{1}c}\cdot e^{a_{2}b_{2}c}\cdots e^{a_{z}b_{z}c} = e^{c\left(a_{1}b_{1}+a_{2}b_{2}+\cdots+a_{z}b_{z}\right)}$.
   \end{enumerate}
},
    \begin{eqnarray}
          K &=& \prod\limits_{i=1}^{\mathcal{C}} \left(\frc{\overline{f}_{i}}{\overline{f}_{i}^{\circ}}\right)^{\nu_{i}} = \prod\limits_{i=1}^{\mathcal{C}} \left[\gamma_{i}x_{i}\left(\frc{f_{i}}{\overline{f}_{i}^{\circ}}\right)\right]^{\nu_{i}} \nonumber \\
            &=& \prod\limits_{i=1}^{\mathcal{C}} \left[\gamma_{i}x_{i}\exp{\left(\frc{V_{i}\left(P-P^{\circ}\right)}{RT}\right)}\right]^{\nu_{i}}  = \prod\limits_{i=1}^{\mathcal{C}}\left(\gamma_{i}x_{i}\right)^{\nu_{i}} \times \left\{\exp{\left[\summation[\frc{V_{i}\left(P-P^{\circ}\right)}{RT}]{i}{}\right]}\right\}^{\nu_{i}} \nonumber \\
            &=& \prod\limits_{i=1}^{\mathcal{C}}\left(\gamma_{i}x_{i}\right)^{\nu_{i}} \times \exp{\left[\frc{P-P^{\circ}}{RT}\summation[V_{i}\nu_{i}]{i=1}{\mathcal{C}}\right]}\nonumber
    \end{eqnarray}
    \begin{shaded}
       \begin{equation}
           \prod\limits_{i=1}^{\mathcal{C}}\left(\gamma_{i}x_{i}\right)^{\nu_{i}} = K\exp{\left[\frc{P^{\circ}-P}{RT}\summation[V_{i}\nu_{i}]{i=1}{\mathcal{C}}\right]}\;\;\;\;\blue{\text{ (for liquid phase).}}\label{Mod6:Eqn:5f}
       \end{equation}
       Simplifications:
       \begin{enumerate}[a)]
           \item At room pressure conditions, $P^{\circ}-P=0$ and
              \begin{equation}
                 K = \prod\limits_{i=1}^{\mathcal{C}}\left(\gamma_{i}x_{i}\right)^{\nu_{i}} \;\;\;\;\blue{\text{ (for liquid phase at room pressure conditions).}}\label{Mod6:Eqn:5g}
              \end{equation}
              Although at first sight this equation may seem relatively simple to solve, activity coefficient models are often necessary to be used making it complex to solve, in such cases iterative methods are commonly employed;
           \item For ideal solutions, $\gamma=1$ and Eqn.~\ref{Mod6:Eqn:5g} becomes,
              \begin{equation}
                 K = \prod\limits_{i=1}^{\mathcal{C}}x_{i}^{\nu_{i}} \;\;\;\;\blue{\text{ (for ideal solutions).}}\label{Mod6:Eqn:5h}
              \end{equation}              
       \end{enumerate}
    \end{shaded}
     
\end{subequations}


%%% SECTION
\section{Equilibrium Constant in Multiple Simultaneous Reactions}\label{Section:06:MultipleReactions}
\begin{subequations}
  Calculations of equilibrium constant of single homogeneous chemical reactions can be extended to multiple independent reactions. Let's consider a set of $\mathcal{R}$ independent reactions involving $\mathcal{C}$ chemical species, the equilibrium constant for each reaction is given by an extension of Eqn.~\ref{Mod6:Eqn:5b}
  \begin{shaded}
    \begin{equation}
        K_{j} = \prod\limits_{i=1}^{\mathcal{C}} \left(\frc{\overline{f}_{i}}{\overline{f}_{i}^{\circ}}\right)^{\nu_{i,j}},\;\;\;\;\forall j\in\left\{1,2,\cdots,\mathcal{R}\right\}.\label{Mod6:Eqn:6a}
    \end{equation}
  \end{shaded}
  For {\it gaseous} reactions,
  \begin{shaded}
    \begin{equation}
      K_{j} =
      \begin{cases}
          \prod\limits_{i=1}^{\mathcal{C}} \left(\frc{\overline{f}_{i}}{P^{\circ}}\right)^{\nu_{i,j}},\;\;\;\;\blue{\text{(gas phase),}}\label{Mod6:Eqn:6b} \\
           \\
          \left(\frc{P}{P^{\circ}}\right)^{-\nu_{i,j}}\prod\limits_{i=1}^{\mathcal{C}} \left(y_{i}\right)^{\nu_{i,j}},\;\;\;\;\blue{\text{(ideal gas mixture).}}
      \end{cases}
      \forall j\in\left\{1,2,\cdots,\mathcal{R}\right\}
    \end{equation}
  \end{shaded}



\end{subequations}


\clearpage

%%% SECTION
\section{Examples}

\begin{enumerate}[1)]

%%%
%%% Elliot & Lira (Ex 3.4)  
%%%
\item\label{Example:1} Five moles of hydrogen, two moles of CO and 1.5 moles of CH$_{3}$OH vapour are combined in a batch methanol synthesis reactor (closed system) at 500 K and 1 MPa. Develop expressions for the mole fractions of the species in terms of the reaction coordinate. The components are known to react with the following stoichiometry:
  \begin{displaymath}
      2 H_{2} (g) + CO (g) \Longleftrightarrow CH_{3}OH (g).
  \end{displaymath}

\bigskip

{\bf Solution:} We need to develop expressions for each species in the gaseous form. In the equilibrium, the molar stoichiometric coefficient and the initial number of moles are:
    \begin{displaymath}
       \begin{cases}
         \nu = \sum\limits_{i}\nu_{i}= (-2)+(-1)+(+1)= -2 \;\;\text{ and } \\
         n_{0} = \sum\limits_{i}n_{i,0}= 5 + 2 + 1.5 = 8.5,
       \end{cases}
    \end{displaymath} 
    respectively. Mole fraction of each species is given by
    \begin{displaymath}
          y_{i} = \frc{n_{i}}{n} = \frc{n_{i,0}+\nu_{i}\epsilon}{n_{0}+\nu\epsilon},
    \end{displaymath}
    where the the total number of moles in equilibrium is $n=8.5-2\epsilon$, thus
    \begin{displaymath}
       \begin{cases}
         y_{\text{H}_{2}} = \frc{5-2\epsilon}{8.5-2\epsilon}, \\
         \\
         y_{\text{CO}} = \frc{2-\epsilon}{8.5-2\epsilon}\\
         \\
          y_{\text{CH}_{3}\text{OH}}=\frc{1.5+\epsilon}{8.5-2\epsilon}
       \end{cases}
    \end{displaymath}
\clearpage

\begin{comment}
%%%
%%%   
%%%
\item\label{Example:2} Consider the chemical reaction,
         \begin{displaymath}
             C_{2}H_{4} (g) + H_{2}O (g) \Longleftrightarrow  C_{2}H_{5}OH (g).
         \end{displaymath}
         If an equimolar mixture of ethylene and water vapour is fed to a reactor which is maintained at 500 K and 40 bar, determine the equilibrium constant, assuming that the reaction mixture behaves like an ideal gas. Assume the following ideal gas specific heat data:
         \begin{displaymath}
             C_{p}^{\text{ig}} = a + bT + cT^{2} + dT^{3} + eT^{-2},
         \end{displaymath}
         where C$_{p}^{\text{ig}}$ is expressed in {\it J/mol} and $T$ in $K$. Also,
         \begin{center}
            \begin{tabular}{|c|c c c c c|}
                \hline
                {\it Species}    & $a$    &  $b\times 10^{3}$  & $c\times 10^{6}$ & $d\times 10^{9}$  & $e\times10^{-5}$ \\
                \hline
                C$_{2}$H$_{4}$    & 20.691 &   205.346          & -99.793         & 18.825           & --               \\
                H$_{2}$O         & 4.196  &  154.565           & -81.076         &  16.813          & --               \\
                C$_{2}$H$_{5}$OH  & 28.850 &  12.055            & --              &  --              & 1.006           \\
                \hline
            \end{tabular}
         \end{center}
         The standard state enthalpy and Gibbs energy of reaction are $\Delta H^{0}_{\text{R},298}=-52.7$ kJ/mol and $\Delta G^{0}_{\text{R},298}=14.5$ kJ/mol. In order to calculate the enthalpy of reaction at temperature $T$, $\Delta H^{0}_{T}$, you should use,
         \begin{displaymath}
            \Delta H^{0}_{T} = \Delta H^{0}_{\text{R},298} + \int\limits_{T_{0}}^{T} \Delta C_{p} dT = \Delta H^{0}_{\text{R},298} + \int\limits_{T_{0}}^{T}\left(\Delta a + \Delta bT + \Delta cT^{2} + \Delta dT^{3} + \Delta eT^{-2}\right) dT
         \end{displaymath} 
         where $\Delta a= \sum\limits_{i} \nu_{i}a_{i}$, $\Delta b= \sum\limits_{i} \nu_{i}b_{i}$, $\Delta c= \sum\limits_{i} \nu_{i}c_{i}$, $\Delta d= \sum\limits_{i} \nu_{i}d_{i}$ and $\Delta e= \sum\limits_{i} \nu_{i}e_{i}$.  Also, given the Van't Hoff equation:
         \begin{displaymath}
            \frc{d}{dT}\left(\Delta G^{0}/RT\right) = -\frc{\Delta H^{0}}{RT^{2}}.
         \end{displaymath}


\bigskip

{\bf Solution:}

   \begin{itemize}
      \item We first need to obtain the integral expression for the heat of reaction at temperature $T$,
         \begin{displaymath}
            \Delta H^{0}_{T} = \Delta H^{0}_{\text{R},298} + \int\limits_{T_{0}}^{T} \Delta C_{p} dT = \Delta H^{0}_{\text{R},298} + \int\limits_{T_{0}}^{T}\left(\Delta a + \Delta bT + \Delta cT^{2} + \Delta dT^{3} + \Delta eT^{-2}\right) dT
         \end{displaymath} 
         with 
         \begin{center}
             \begin{tabular}{r l c}
                $\Delta a =$ & $(+1).(28.850)-\left[(-1).(20.691)+(-1).(4.196)\right]=$ & 3.963 \\
                $\Delta b =$ & $\left\{(+1).(12.055)- \left[(-1).(205.346)+(-1).(154.565)\right]\right\}\times 10^{-3}=$ & 0.37197 \\
                $\Delta c =$ & $\left\{(+1).(0)-\left[(-1).(-99.793)+(-1).(-81.076)\right]\right\}\times 10^{-6}=$       & -1.8087$\times$10$^{-4}$  \\
                $\Delta d =$ & $\left\{(+1).(0)-\left[(-1).(18.825)+(-1).(16.813)\right]\right\}\times 10^{-9}=$         & 3.5638$\times$10$^{-8}$\\
                $\Delta e =$ & $\left\{(+1).(1.006)-\left[(-1).(0)+(-1).(0)\right]\right\}\times 10^{5}=$                & 1.006$\times$10$^{5}$ \\
             \end{tabular}
         \end{center}
         Thus,
         \begin{eqnarray}
            \Delta H^{0}_{T} &=& \Delta H^{0}_{\text{R},298} + \int\limits_{T_{0}}^{T} \Delta C_{p} dT = \Delta H^{0}_{\text{R},298} + \int\limits_{T_{0}}^{T}\left(\Delta a + \Delta bT + \Delta cT^{2} + \Delta dT^{3} + \Delta eT^{-2}\right) dT \nonumber \\
            &=& -52.7 + \int\limits_{T_{0}}^{T} \left(3.963 + 0.37197 T - 1.8087\times 10^{-4} T^{2} + 3.5638\times 10^{-8} T^{3} + \frc{1.006 \times 10^{5}}{T^{-2}} \right)dT \nonumber
         \end{eqnarray}
%
      \item We can calculate the Gibbs energy of the reaction through the Van't Hoff equation:
         \begin{displaymath}
                \frc{d}{dT}\left(\Delta G^{0}/RT\right) = -\frc{\Delta H^{0}}{RT^{2}}.
         \end{displaymath}
         Integrating this equation from the standard state temperature, 298.15 K, to $T=$ 500 K

          
   \end{itemize}
\clearpage
\end{comment}

%%%
%%%   Nguyen (Ex 6.6.1)  
%%%
\item\label{Example:2} Ethylene is produced from the decomposition of ethane,
       \begin{displaymath}
          C_{2}H_{6} (g) \Longleftrightarrow C_{2}H_{4} (g) + H_{2} (g) 
       \end{displaymath} 
       Determine the equilibrium composition at 1000$^{\circ}$C and 1 atm. Assume that, initially, there is 1 mol of ethane. Given,
       \begin{center}
           \begin{tabular}{|c c c c|}
           \hline
                                        &  C$_{2}$H$_{6}$ (g) & C$_{2}$H$_{4}$ (g) &  H$_{2}$ (g)  \\ 
           \hline
             $\Delta G^{\circ}_{\text{f,298}}$  &  -32.84            &  68.15           & 0.0           \\
                   (kJ/mol)             &                    &                  &               \\
           \hline
             $\Delta H^{\circ}_{\text{f,298}}$  &  -84.68            &  52.26           & 0.0           \\
                   (kJ/mol)             &                    &                  &               \\
           \hline 
           \end{tabular}
       \end{center}
       where $\Delta G^{\circ}_{\text{f,298}}$ and $\Delta H^{\circ}_{\text{f,298}}$ are the standard state Gibbs energy and enthalpy of formation. 

\bigskip

{\bf Solution:} First we need to determine the standard state Gibbs energy of reaction,
         \begin{eqnarray}
             \Delta G^{\circ}_{\text{r,298}} &=& \sum\limits_{i}\nu_{i}\left(\Delta G^{\circ}_{\text{f,298}}\right)_{i} \nonumber \\
                                     &=& (+1).(68.15)+ (1).(0.0) + (-1).(-32.84) = 100.99\; \text{kJ/mol} \nonumber
         \end{eqnarray}
       The equilibrium constant can be obtained from,
         \begin{displaymath}
            K = \exp\left(-\frc{\Delta G^{\circ}_{\text{r,298}}}{RT}\right) = exp\left(-\frc{100.99\times 10^{3}}{(8.314).(298.15)}\right) = 2.0246\times 10^{-18}
         \end{displaymath}
       The Van't Hoff equation can be used to calculate the equilibrium constant at temperature $T$,
         \begin{displaymath}
            \frc{\partial\left(\ln K\right)}{\partial T} = \frc{\Delta H^{\circ}_{\text{r}}}{RT^{2}}
         \end{displaymath}
       Before we can integrate the Van't Hoff equation, we first need to determine the standard heat of reaction, $\Delta H^{\circ}_{\text{r}}$,
         \begin{eqnarray}
            \Delta H^{\circ}_{\text{r}} &=& \sum\limits_{i} \nu_{i}\left(\Delta H^{\circ}_{\text{f,298}}\right)_{i} \nonumber \\
                                 &=& (+1).(52.26) + (+1).(0.0) + (-1).(-84.68) = 136.94\;\text{kJ/mol}\nonumber
         \end{eqnarray}
       Now integrating the Van't Hoff equation from 298.15 K to 1273.15 K,
         \begin{eqnarray}
            \ln\frc{K_{1273}}{K_{298}} &=& -\frc{\Delta H^{\circ}_{\text{r}}}{R}\left(\frc{1}{T}-\frc{1}{298.15}\right) \nonumber \\
            \ln\frc{K_{1273}}{2.0246\times 10^{-18}} &=& -\frc{136.94\times 10^{3}}{8.314}\left(\frc{1}{1273.15}-\frc{1}{298.15}\right) \nonumber \\
            K_{1273} &=& 4.7859 \nonumber
         \end{eqnarray} 
       The equilibrium constant $K$ can also be expressed as a function of the components' activities,
         \begin{displaymath}
            K = \prod\limits_{i=1}^{\mathcal{C}} a_{i}^{\nu_{i}} = \frc{a_{\text{C}_{2}\text{H}_{4}}a_{\text{H}_{2}}}{a_{\text{C}_{2}\text{H}_{6}}} = \frc{\left(\frc{\overline{f}_{\text{C}_{2}\text{H}_{4}}}{\overline{f}^{\circ}_{\text{C}_{2}\text{H}_{4}}}\right)\left(\frc{\overline{f}_{\text{H}_{2}}}{\overline{f}^{\circ}_{\text{H}_{2}}}\right)}{\left(\frc{\overline{f}_{\text{C}_{2}\text{H}_{6}}}{\overline{f}^{\circ}_{\text{C}_{2}\text{H}_{6}}}\right)}
         \end{displaymath}
         Assuming ideal gas behaviour, $\overline{f}_{i}=P_{i}$, $\overline{f}_{i}^{\circ}=P^{\circ}_{\text{C}_{2}\text{H}_{6}}=P^{\circ}_{\text{C}_{2}\text{H}_{4}}=P^{\circ}_{\text{H}_{2}}=$ 1 atm. Thus,
         \begin{displaymath}
            K = \frc{a_{\text{C}_{2}\text{H}_{4}}a_{\text{H}_{2}}}{a_{\text{C}_{2}\text{H}_{6}}} = \frc{\left(\frc{\overline{f}_{\text{C}_{2}\text{H}_{4}}}{\overline{f}^{\circ}_{\text{C}_{2}\text{H}_{4}}}\right)\left(\frc{\overline{f}_{\text{H}_{2}}}{\overline{f}^{\circ}_{\text{H}_{2}}}\right)}{\left(\frc{\overline{f}_{\text{C}_{2}\text{H}_{6}}}{\overline{f}^{\circ}_{\text{C}_{2}\text{H}_{6}}}\right)} = \frc{\left(\frc{P_{\text{C}_{2}\text{H}_{4}}}{P^{\circ}_{\text{C}_{2}\text{H}_{4}}}\right)\left(\frc{P_{\text{H}_{2}}}{P^{\circ}_{\text{H}_{2}}}\right)}{\left(\frc{P_{\text{C}_{2}\text{H}_{6}}}{P^{\circ}_{\text{C}_{2}\text{H}_{6}}}\right)} = \frc{\left(\frc{y_{\text{C}_{2}\text{H}_{4}}P}{1\text{ atm}}\right)\left(\frc{y_{\text{H}_{2}}P}{1\text{ atm}}\right)}{\left(\frc{y_{\text{C}_{2}\text{H}_{6}}P}{1\text{ atm}}\right)}          
         \end{displaymath}
         Since $P=$ 1 atm,
         \begin{displaymath}
            K = \frc{\left(\frc{y_{\text{C}_{2}\text{H}_{4}}P}{1\text{ atm}}\right)\left(\frc{y_{\text{H}_{2}}P}{1\text{ atm}}\right)}{\left(\frc{y_{\text{C}_{2}\text{H}_{6}}P}{1\text{ atm}}\right)} = \frc{\left(y_{\text{C}_{2}\text{H}_{4}}\right)\left(y_{\text{H}_{2}}\right)}{\left(y_{\text{C}_{2}\text{H}_{6}}\right)} = 4.7859
         \end{displaymath}
         This equation can not be solved as there are two unknowns $\left(\text{remember that }\summation[y_{i}]{i=1}{3}=1\right)$. The number of unknowns can be decreased if we make use of
         \begin{displaymath}
           y_{i} = \frc{n_{i}}{n} = \frc{n_{i,0}+\nu_{i}\epsilon}{n_{0}+\nu\epsilon}.
         \end{displaymath}
         with 
         \begin{displaymath}
            \nu = \sum\limits_{i}\nu_{i}= (-1)+(+1)+(+1)= 1 \;\;\text{ and }\;\; n_{0} = \sum\limits_{i}n_{i,0}= 1 + 0 + 0 = 1,
         \end{displaymath} 
       Thus,
         \begin{displaymath}
             y_{\text{C}_{2}\text{H}_{6}} = \frc{1-\epsilon}{1+\epsilon},\;\;y_{\text{C}_{2}\text{H}_{4}} = \frc{\epsilon}{1+\epsilon}\;\text{ and }\;y_{\text{H}_{2}} = \frc{\epsilon}{1+\epsilon}
         \end{displaymath}
       Replacing the compositions in the expression for $K$,
         \begin{eqnarray}
           K  &=& \frc{\left(y_{\text{C}_{2}\text{H}_{4}}\right)\left(y_{\text{H}_{2}}\right)}{\left(y_{\text{C}_{2}\text{H}_{6}}\right)} = \frc{\left(\frc{\epsilon}{1+\epsilon}\right)^{2}}{\left(\frc{1-\epsilon}{1+\epsilon}\right)} =4.7859 \nonumber \\
            &&\epsilon = 0.9095 \nonumber
         \end{eqnarray}
        The equilibrium concentration of C$_{2}$H$_{4}$(g) is $y_{\text{C}_{2}\text{H}_{4}} =$ 0.4763 $=y_{\text{H}_{2}}$ and $y_{\text{C}_{2}\text{H}_{6}}=$ 0.0474.

\clearpage


%%%
%%%   Nguyen (Ex 6.7.2) 
%%%
\item\label{Example:3} Determine the equilibrium conversion for the liquid-phase isomerisation reaction of methylcyclopentane $\left(\text{CH}_{3}\text{C}_{5}\text{H}_{9}\right)$ to cyclohexane $\left(\text{C}_{6}\text{H}_{12}\right)$ at 25$^{\circ}$C. Gibbs energies of formation are given at 25$^{\circ}$C as:
  \begin{displaymath}
     \left(\Delta G^{\circ}_{\text{f,298}}\right)_{\text{CH}_{3}\text{C}_{5}\text{H}_{9}} = 31.72 \text{ kJ.mol}^{-1} \text{ and } \left(\Delta G^{\circ}_{\text{f,298}}\right)_{\text{C}_{6}\text{H}_{12}} = 26.89 \text{ kJ.mol}^{-1}.
  \end{displaymath}
  Also, assume that this isomerisation reaction occurs at relatively low pressure, and therefore $\frc{f_{i}}{f^{\circ}_{i}}=1$. Also, assume that solution is ideal.


\bigskip

{\bf Solution:} The isomerisation reaction can be represented by,
    \begin{displaymath}
        CH_{3}C_{5}H_{9} (l) \Longleftrightarrow C_{6}H_{12} (l)
    \end{displaymath}
    We need to determine the equilibrium constant at room temperature
         \begin{eqnarray}
           \Delta G^{\circ}_{r,298} &=& \sum\limits_{i}\nu_{i}\Delta G^{\circ}_{\text{f,298}} \nonumber \\
                             &=& (+1).\left(\Delta G^{\circ}_{\text{f,298}}\right)_{\text{C}_{6}\text{H}_{12}} + (-1).\left(\Delta G^{\circ}_{\text{f,298}}\right)_{\text{CH}_{3}\text{C}_{5}\text{H}_{9}} \nonumber \\ 
                             &=& 26.89-31.72 = -4.83 \text{ kJ/mol} \nonumber
         \end{eqnarray}
       The equilibrium constant is
         \begin{displaymath}
             K = \exp\left(-\frc{\Delta G^{\circ}_{r,298}}{RT}\right) = \exp\left(-\frc{-4830}{(8.314).(298.15)}\right) = 7.0182
         \end{displaymath}
       The equilibrium constant can also be obtained from 
         \begin{displaymath}
            K = \prod\limits_{i}\left(\frc{\overline{f}_{i}}{\overline{f}^{\circ}_{i}}\right)^{\nu_{i}} = \prod\limits_{i}\left(\frc{x_{i}\gamma_{i} f_{i}}{\overline{f}^{\circ}_{i}}\right)^{\nu_{i}}
         \end{displaymath}
       At room pressure conditions, the ratio $\frc{f_{i}}{\overline{f}^{\circ}_{i}} = \frc{f_{i}}{f_{i}^{\circ}} = \exp\left[\frc{V\left(P-P^{\circ}\right)}{RT}\right]=1$, therefore,
         \begin{displaymath}
            K = \prod\limits_{i}\left(\frc{x_{i}\gamma_{i}f_{i}}{\overline{f}^{\circ}_{i}}\right)^{\nu_{i}} = \prod\limits_{i}\left(x_{i}\gamma_{i}\right)^{\nu_{i}}
         \end{displaymath}
       For ideal solution,
         \begin{displaymath}
            K = \prod\limits_{i=1}^{\mathcal{C}}\left(x_{i}\gamma_{i}\right)^{\nu_{i}} = \prod\limits_{i}\left(x_{i}\right)^{\nu_{i}} = \left(x_{\text{C}_{6}\text{H}_{12}}\right).\left(x_{\text{CH}_{3}\text{C}_{5}\text{H}_{9}}\right)^{-1} = 7.0182,
         \end{displaymath}
         as $\summation[x_{i}]{i}{}=1$, we could easily solve the equation above as $x_{\text{C}_{6}\text{H}_{12}}\left(1-x_{\text{C}_{6}\text{H}_{12}}\right)=7.0182$, and obtain $x_{\text{C}_{6}\text{H}_{12}}=0.8753$. Alternatively, one could obtain a relation between composition and the reaction coordinate, $\varepsilon$,
         \begin{displaymath}
            x_{i} = \frc{n_{i,0} + \nu_{i}\varepsilon}{n_{0}+\nu\epsilon},
         \end{displaymath}
         where $\nu = (-1) + (1) = 0$ and assuming that initially there are $n_{\text{CH}_{3}\text{C}_{5}\text{H}_{9},0}=1$ and $n_{\text{C}_{6}\text{H}_{12},0}=0$,
         \begin{displaymath}
            x_{\text{CH}_{3}\text{C}_{5}\text{H}_{9}} = 1-\varepsilon\;\;\;\;\;\text{ and }\;\;\;\;\;\; x_{\text{C}_{6}\text{H}_{12}} = \varepsilon.
         \end{displaymath}
         Now, replacing in the expression above, 
         \begin{displaymath}
             K = \left(x_{\text{C}_{6}\text{H}_{12}}\right).\left(x_{\text{CH}_{3}\text{C}_{5}\text{H}_{9}}\right)^{-1} = \frc{\varepsilon}{1-\varepsilon} = 7.0182 \;\;\;\; \Longrightarrow \;\;\;\;\; \epsilon = 0.8753 
         \end{displaymath}
         Thus, $x_{\text{C}_{6}\text{H}_{12}}=0.8753$ and $x_{\text{CH}_{3}\text{C}_{5}\text{H}_{9}}=0.1247$.

\clearpage

%%%
%%%  Sandler (Ex 13.1.1, pg. 713)
%%%
\item\label{Example:4} Calculate the equilibrium extent of decomposition of nitrogen tetroxide $\left(N_{2}O_{4}\right)$,
  \begin{displaymath}
     N_{2}O_{4} (g) \Longleftrightarrow 2 NO_{2} (g)
  \end{displaymath}
  at 298.15 K and 1 atm. Assume that the gas mixture behaves as an ideal gas and the Gibbs energies of formation at 25$^{\circ}$C are:
  \begin{displaymath}
     \left(\Delta G^{\circ}_{\text{f,298}}\right)_{N_{2}O_{4}} = 97.89 \text{ kJ.mol}^{-1} \text{ and } \left(\Delta G^{\circ}_{\text{f,298}}\right)_{NO_{2}} = 51.31 \text{ kJ.mol}^{-1}.
  \end{displaymath}
     

\bigskip

{\bf Solution:} 
       The equilibrium constant is
         \begin{displaymath}
             K = \exp\left(-\frc{\Delta G^{\circ}_{r,298}}{RT}\right) = \frc{a_{NO_{2}}^{2}}{a_{N_{2}O_{4}}} = \frc{\left(\frc{y_{NO_{2}}P}{P^{\circ}_{NO_{2}}}\right)^{2}}{\left(\frc{y_{N_{2}O_{4}}P}{P^{\circ}_{N_{2}O_{4}}}\right)} = \frac{\left(y_{NO_{2}}\right)^{2}}{\left(y_{N_{2}O_{4}}\right)},
         \end{displaymath}
         where the standard Gibbs energy of reaction is
         \begin{eqnarray}
           \Delta G^{\circ}_{r,298} &=& \sum\limits_{i}\nu_{i}\Delta G^{\circ}_{\text{f,298}} \nonumber \\
                             &=& (+2).\left(\Delta G^{\circ}_{\text{f,298}}\right)_{NO_{2}} + (-1).\left(\Delta G^{\circ}_{\text{f,298}}\right)_{N_{2}O_{4}} \nonumber \\ 
                             &=& 4730 \text{ J/mol}, \nonumber
         \end{eqnarray}
         and the equilibrium constant,
         \begin{displaymath}
             K = \exp\left(-\frc{\Delta G^{\circ}_{r,298}}{RT}\right) = 0.1484 = \frac{\left(y_{NO_{2}}\right)^{2}}{\left(y_{N_{2}O_{4}}\right)}
         \end{displaymath}
         Composition and reaction coordinate are related through
         \begin{displaymath}
            y_{i} = \frc{n_{i,0} + \nu_{i}\varepsilon}{n_{0}+\nu\epsilon},
         \end{displaymath}
         where, assuming that the initial number of moles are $n_{N_{2}O_{4},0}=1$ and $n_{NO_{2},0}=0$, an $\nu= 2-1 = 1$,
         \begin{displaymath}
            y_{N_{2}O_{4}} = \frc{1-\varepsilon}{1+\varepsilon}\;\;\;\;\;\text{ and }\;\;\;\;\;\; y_{NO_{2}} = \frc{2\varepsilon}{1+\varepsilon},
         \end{displaymath}
         leading to
         \begin{displaymath}
             K = \frac{\left(y_{NO_{2}}\right)^{2}}{\left(y_{N_{2}O_{4}}\right)} = \frc{\left(\frc{2\varepsilon}{1+\varepsilon}\right)^{2}}{\left(\frc{1-\varepsilon}{1+\varepsilon}\right)} = 0.1484 \;\;\;\;\Longrightarrow\;\;\;\; \varepsilon = 0.1891
         \end{displaymath}
         Thus: $y_{NO_{2}} = 0.3181$ and $y_{N_{2}O_{4}} = 0.6819$.

         
\clearpage

%%%
%%%  Sandler (Ex 13.1.2, pg. 714)
%%%
\item\label{Example:5} Pure nitrogen tetroxide $\left(N_{2}O_{4}\right)$ at a low temperature is diluted with nitrogen $\left(N_{2}\right)$ and heated to 298.15 K and 1 atm. If the initial mole fraction of $N_{2}O_{4}$ in the $N_{2}O_{4}-N_{2}$ mixture before the dissociation begins is 0.20, what is the extent of the decomposition, 
  \begin{displaymath}
     N_{2}O_{4} (g) \Longleftrightarrow 2 NO_{2} (g)
  \end{displaymath}
  and the mole fractions of $NO_{2}$ and $N_{2}O_{4}$ present at equilibrium? Assume that the gas mixture behaves as an ideal gas and the Gibbs energies of formation at 25$^{\circ}$C are:
  \begin{displaymath}
     \left(\Delta G^{\circ}_{\text{f,298}}\right)_{N_{2}O_{4}} = 97.89 \text{ kJ.mol}^{-1} \text{ and } \left(\Delta G^{\circ}_{\text{f,298}}\right)_{NO_{2}} = 51.31 \text{ kJ.mol}^{-1}.
  \end{displaymath}

\bigskip

{\bf Solution:} Here, nitrogen is an inert species, \ie it is used to dilute the reactant gas, $N_{2}O_{4}$, but do not participate in the decomposition reaction, thus $\nu_{N_{2}}=0$. In Example~\ref{Example:4}, we have already calculated the equilibrium constant, $K = 0.1484$, and used 
         \begin{displaymath}
            y_{i} = \frc{n_{i,0} + \nu_{i}\varepsilon}{n_{0}+\nu\epsilon},
         \end{displaymath}
         to calculate the equilibrium composition. The overall molar stoichiometric coefficient $\nu= 2-1-0 = 1$ remains the same, however, now the initial reactive composition is (assuming 1 mol of the reactant mixture) $n_{N_{2}O_{4},0} = 0.2$, $n_{N_{2},0} = 0.8$ and $n_{NO_{2},0}=0$. 
         \begin{displaymath}
           y_{N_{2}O_{4}} = \frc{0.2-\varepsilon}{1+\varepsilon},\;\;\;\;\; y_{N_{2}} = \frc{0.8}{1+\varepsilon}\;\;\;\text{ and }\;\;\;y_{NO_{2}} = \frc{2\varepsilon}{1+\varepsilon}.
         \end{displaymath}
         Leading to
         \begin{displaymath}
             K = \frac{\left(y_{NO_{2}}\right)^{2}}{\left(y_{N_{2}O_{4}}\right)} = \frc{\left(\frc{2\varepsilon}{1+\varepsilon}\right)^{2}}{\left(\frc{0.2-\varepsilon}{1+\varepsilon}\right)} = 0.1484 \;\;\;\;\Longrightarrow\;\;\;\; \varepsilon = 0.0715,
         \end{displaymath}
         with equilibrium compositions of $y_{NO_{2}} = 0.1335$, $y_{N_{2}O_{4}} = 0.1199$ and $y_{N_{2}} = 0.7466$. 
         
\clearpage


%%%
%%%  Sandler (Ex 13.1.3, pg. 718)
%%%
\item\label{Example:6} Calculate the equilibrium extent of decomposition of nitrogen tetroxide $\left(N_{2}O_{4}\right)$,
  \begin{displaymath}
     N_{2}O_{4} (g) \Longleftrightarrow 2 NO_{2} (g)
  \end{displaymath}
  at 250 K and 400 K and 1 atm. Assume that the gas mixture behaves as an ideal gas and the Gibbs energies of formation at 25$^{\circ}$C are:
  \begin{displaymath}
     \left(\Delta G^{\circ}_{\text{f,298}}\right)_{N_{2}O_{4}} = 97.89 \text{ kJ.mol}^{-1} \text{ and } \left(\Delta G^{\circ}_{\text{f,298}}\right)_{NO_{2}} = 51.31 \text{ kJ.mol}^{-1}.
  \end{displaymath}
  Also, the standard enthalpy of reaction at 25$^{\circ}$C is 56.189 kJ.mol$^{-1}$. Heat capacity $\left(\text{in J.mol}^{-1}\text{.K}^{-1}\right)$ for both gases are expressed in a polynomial form,
  \begin{displaymath}
    C_{p} = a + bT + CT^{2} + dT^{3},
  \end{displaymath}
  where
  \begin{center}
    \begin{tabular}{ l | c c c c }
      \hline
                         &  $a$     &  $b\times 10^{-2}$  & $c\times 10^{-5}$  & $d\times 10^{-9}$ \\
                         &$\left(\text{J.mol}^{-1}\text{.K}^{-1}\right)$& $\left(\text{J.mol}^{-1}\text{.K}^{-2}\right)$& $\left(\text{J.mol}^{-1}\text{.K}^{-3}\right)$& $\left(\text{J.mol}^{-1}\text{.K}^{-4}\right)$ \\
      \hline
      NO$_{2}$             &  22.929 &      5.711          & -3.519            & 7.866 \\
      N$_{2}$O$_{4}$       &   33.054 &      18.661         &    -11.339        &  -- 
    \end{tabular}
  \end{center}
      

\bigskip

{\bf Solution:} In order to calculate composition at equilibrium, we need to calculate the equilibrium constant, $K$, at 250 K and 400 K through the Van't Hoff equation,
\begin{displaymath}
    \frc{d\left(\ln{K}\right)}{dT} = \frc{\Delta H^{\circ}_{r}}{RT^{2}} \;\;\;\Longrightarrow \;\;\; \ln{\left(\frc{K(T)}{K(298.15\text{ K})}\right)} = \int\limits_{298.15\text{ K}}^{T}\frc{\Delta H^{\circ}_{r}}{RT^{2}}dT.
\end{displaymath}
We have already calculated $K$ at 298.15 in Example~\ref{Example:4}, $K= 0.1484$. The standard heat of reaction is obtained from the fundamental enthalpy relation, $dH= C_{p}dT$, and 
\begin{displaymath}
     \Delta H^{\circ}_{r} = \summation[\nu_{i}\Delta H^{\circ}_{i,f}]{i}{} \;\;\;\Longrightarrow\;\;\; \int\limits_{\Delta H^{\circ}_{r,298.15}}^{\Delta H^{\circ}_{r,T}}d\left(\Delta H^{\circ}_{r}\right) = \int\limits_{298.15\text{ K}}^{T}\summation[\nu_{i} C_{p,i}]{i}{}dT.
\end{displaymath}
The first stage in this calculation is to obtain an expression for the right-hand side integration, \ie the summation term $\Delta C_{p}^{\circ} = \summation[\nu_{i} C_{p,i}]{i}{}$, using (for simplicity in the notation) $NO_{2}$:1 and $N_{2}O_{4}$:2,
\begin{eqnarray}
   \Delta C_{p}^{\circ} &=& \summation[\nu_{i} C_{p,i}]{i}{} = (+2) \left(a_{1}+b_{1}T+c_{1}T^{2}+d_{1}T^{3}\right) + (-1)\left(a_{2}+b_{2}T+c_{2}T^{2}+d_{2}T^{3}\right) \nonumber \\
                     &=& 12.804 - 7.239\times 10^{-2}T + 4.301\times 10^{-5} T^{2} + 15.732\times 10^{-9}T^{3} \nonumber 
\end{eqnarray}
And the integration becomes $\left(\text{with }\Delta H^{\circ}_{r,298.15} = 56189\text{ J.mol}^{-1}\right)$
\begin{eqnarray}
    && \Delta H^{\circ}_{r}(T) - \Delta H^{\circ}_{r,298.15} = \int\Delta C_{p}^{\circ}dT \nonumber \\
    && \Delta H^{\circ}_{r}(T) = 56189 + 12.804 T - \frc{7.239\times 10^{-2}}{2}T^{2} + \frc{4.301\times 10^{-5}}{3}T^{3} + \frc{15.732\times 10^{-9}}{4}T^{4} \nonumber
\end{eqnarray}
Now, using the Van't Hoff equation,
\begin{eqnarray}
  \ln{\left(\frc{K(T)}{K_{298.15\text{ K}}}\right)} &=& \int\limits_{298.15\text{ K}}^{T}\frc{\Delta H^{\circ}_{r}}{RT^{2}}dT \nonumber \\
                                                &=& \frc{1}{R}\int\limits_{298.15\text{ K}}^{T} \left[\frc{56189}{T^{2}} + \frc{12.804}{T} - \frc{7.239\times 10^{-2}}{2} + \frc{4.301\times 10^{-5}}{3}T + \frc{15.732\times 10^{-9}}{4}T^{2}\right]dT \nonumber \\
                                                &=& \frc{1}{R}\left[-\frc{56189}{T} + 12.804\ln{T} - \frc{7.239\times 10^{-2}}{2}T + \frc{4.301\times 10^{-5}}{6}T^{2} + \frc{15.732\times 10^{-9}}{12}T^{3}\right]_{298.15\text{ K}}^{T} \nonumber
\end{eqnarray}
Solving this integral for
\begin{displaymath}
  \begin{cases}
      K_{250\text{ K}} = 1.7298\times 10^{-3} \\
      K_{400\text{ K}} = 51.4338
  \end{cases}  
\end{displaymath}
Now, using the expression obtained in Example~\ref{Example:4} for composition, assuming that the initial number of moles are $n_{N_{2}O_{4},0}=1$ and $n_{NO_{2},0}=0$, and $\nu= 2-1 = 1$,
         \begin{displaymath}
            y_{N_{2}O_{4}} = \frc{1-\varepsilon}{1+\varepsilon}\;\;\;\;\;\text{ and }\;\;\;\;\;\; y_{NO_{2}} = \frc{2\varepsilon}{1+\varepsilon},
         \end{displaymath}
         leading to
         \begin{displaymath}
             K = \frac{\left(y_{NO_{2}}\right)^{2}}{\left(y_{N_{2}O_{4}}\right)} = \frc{\left(\frc{2\varepsilon}{1+\varepsilon}\right)^{2}}{\left(\frc{1-\varepsilon}{1+\varepsilon}\right)}
         \end{displaymath}
         Solving for
         \begin{displaymath}
           \begin{cases}
             K_{250\text{ K}}   = 1.7298\times10^{-3} \;\;\Longrightarrow \varepsilon = 0.02079 \;\;\Longrightarrow \;\; y_{NO_{2}} = 0.0407 \text{ and } y_{N_{2}O_{4}} = 0.9593; \\
             K_{298.15\text{ K}} = 0.1484 \hspace{1.cm} \Longrightarrow  \varepsilon = 0.1891  \hspace{.35cm} \Longrightarrow \;\; y_{NO_{2}} = 0.3181 \text{ and } y_{N_{2}O_{4}} = 0.6819; \\
             K_{400\text{ K}}   = 51.4331 \hspace{1.2cm} \Longrightarrow \varepsilon = 0.9632  \hspace{.35cm} \Longrightarrow \;\;  y_{NO_{2}} = 0.9813 \text{ and } y_{N_{2}O_{4}} = 0.0187 \\
           \end{cases}
         \end{displaymath}
         
        Note that at low temperature, 250 K $\left(-23.15^{\circ}\text{C}\right)$, the production of nitrogen dioxide is fairly small (backward reaction), however as the temperature rises the forward reaction becomes dominant with 98.1$\%$ of NO$_{2}$ being produced at 400 K$\left(126.85^{\circ}\text{C}\right)$.

\clearpage

\end{enumerate}


%\part{Power and Refrigeration}

\pagebreak

\cleardoublepage

  \begin{appendix}
     \part{Appendices}
       
\chapter{Unit Conversion}


Extracted from~\cite{Balmer_Book}:\\
  \begin{tabular}{c l}
     \hspace{1cm} & R.T. Balmer \\
                  & Modern Engineering Thermodynamics \\
                  & Academic Press, 2011  \\
  \end{tabular}


\begin{list}{\bf Example \arabic{qcounter}:~}{\usecounter{qcounter}}
%
   \item\label{Example:UnitConversion1} Calculate the volume $\left(\text{in m}^{3}\right)$ of 1 kg of hydrogen gas $\left(\text{molecular mass of 2.016 kg.kgmol}^{-1}\right)$ at 27$^{\circ}$C and 1 bar. Assume ideal gas behaviour.
     \begin{description}
        \item[Solution:] We can use the ideal gas equation of state to calculate the volume of H$_{2}$ gas,
           \begin{displaymath}
              V = \frc{nRT}{P},
           \end{displaymath}
           where $n = m/MW$ is the number of moles, $m$, $MW$ and $R$ are the mass, molecular mass, and universal gas constant, respectively. We should be able to replace the variables with their values,
           \begin{eqnarray}
              V &=& \frc{nRT}{P} = \frc{ \frc{m}{MW} R T }{ P } \nonumber \\
                &=& \frc{ \frc{ 1\text{ kg}}{ 2.016\text{ kg.kgmol}^{-1}}\;\; 0.08314\frc{\text{bar.m}^{3}}{\text{kgmol.K}}\;\; ( 27 + 273.15)\text{ K}}{ 1 \text{ bar}},\nonumber
           \end{eqnarray}
           It is clear that the units above are consistent and can be easily eliminated resulting in m$^{3}$,
           \begin{eqnarray}
              V &=& \frc{ \frc{ 1\cancel{\text{ kg}}}{ 2.016\text{ \cancel{kg}.}\cancel{\text{kgmol}^{-1}}}\;\; 0.08314\frc{\text{\cancel{bar}.m}^{3}}{\text{\cancel{kgmol}.\cancel{K}}}\;\; ( 27 + 273.15)\cancel{\text{ K}}}{ 1 \text{ \cancel{bar}}},\nonumber \\
                &=& 12.3782\text{ m}^{3}.\nonumber
           \end{eqnarray}

     \end{description}
%
   \item\label{Example:UnitConversion2}  Liquid water at 0.70 bar is transferred from a condenser to a boiler through a pump. The pressure in the exit of the pump is 25 bar. Assuming that the water undertakes an isentropic (\ie constant entropy) compression, calculate the specific enthalpy of the water after the pump. Consider that the liquid water as incompressible and
       \begin{displaymath}
          dh = Tds + vdP,
       \end{displaymath}
where h, s, v are specific enthalpy (kJ/kg), entropy (kJ/(kg.K)) and volume (m$^{3}$.kg).
     \begin{description}
        \item[Solution:] If the process is isentropic, therefore $ds=0$ and the fundamental thermodynamic relation is simplified to,
       \begin{displaymath}
          dh = vdP \Longrightarrow h_{2} - h_{1} = v\left(P_{2}-P_{1}\right),
       \end{displaymath}
      or summarising,
       \begin{center}
         \begin{tabular}{c| c c c}
            State  & $P$ (bar)  & $h$ (kJ/kg) & $v$ $\left(\text{m}^{3}\text{.kg}\right)$ \\
\hline
              1    &   0.70    &  376.70   &  1.0360$\times$10$^{-3}$                 \\
              2    &  25.0    & \red{h$_{2}$}& 1.0360$\times$10$^{-3}$ 
         \end{tabular}
       \end{center}
       Values of {\it State 1} were obtained from the water saturated table (Appendix~\ref{Appendix:Saturated_SH_Tables}). Note that as the fluid is assumed incompressible there is no variation in the volume, $v_{1}=v_{2}$.
       \begin{eqnarray}
         h_{2} &=& h_{1} = v\left(P_{2}-P_{1}\right) \nonumber \\
              &=& 376.70\frc{\text{kJ}}{\text{kg}} + 1.0360\times 10^{-3}\frc{\text{m}^{3}}{\text{kg}}\;\left(25 - 0.70\right)\text{ bar}  \nonumber\\
              &=& 376.70\frc{\text{kJ}}{\text{kg}} + 0.02517 \frc{\text{m}^{3}.\text{bar}}{\text{kg}} \nonumber
       \end{eqnarray}
       It is clear that the units in the two terms of the r.h.s. of the equation contain distinct units that \underline{can not} be summed up. Therefore, we need to convert $\left[\text{m}^{3}.\text{bar}\right]$ to $[\text{kJ}]$. Bearing in mind that 1 J = 1 N.m = 1 kg.m$^{2}$.s$^{-2}$, we first can convert $\left[\text{m}^{3}.\text{bar}\right]$ to $\left[\text{kg.m}^{2}.\text{s}^{-2}\right]$
       \begin{displaymath}
          1 \cancelto{\blue{\text{m}^{2}}}{\text{m}^{3}}.\red{\cancel{\text{bar}}} \times \frc{\red{10^{5}\cancel{\text{ Pa}}}}{\red{1 \cancel{\text{ bar}}}} \times \frc{\red{ 1 \text{ kg}/\left(\cancel{\blue{\text{m}}}.\text{s}^{2}\right)}}{\red{ 1 \cancel{\text{ Pa}}}} = 10^{5} \frac{\text{kg.m}^{2}}{\text{s}^{2}}
       \end{displaymath}
       Now, replacing the converted $\left[\text{m}^{3}.\text{bar}\right]$ term in the r.h.s. of the expression for $h_{2}$,
       \begin{eqnarray}
         h_{2} &=& 376.70\frc{\text{kJ}}{\text{kg}} + 0.02517 \frc{\text{m}^{3}.\text{bar}}{\text{kg}} \nonumber \\
              &=& 376.70\frc{\text{kJ}}{\text{kg}} + 0.02517 \frc{\cancel{\left(\text{m}^{3}.\text{bar}\right)}}{\text{kg}} \times \frc{10^{5} \frac{\text{kg.m}^{2}}{\text{s}^{2}}}{1 \cancel{\left(\text{m}^{3}.\text{bar}\right)}} \nonumber \\
              &=& 376.70\frc{\text{kJ}}{\text{kg}} + 2517 \frc{\cancelto{\red{\text{J}}}{\frac{\text{kg.m}^{2}}{\text{s}^{2}}}}{\text{kg}}\nonumber\\
              &=& 376.70\frc{\text{kJ}}{\text{kg}} + 2517 \frc{\text{J}}{\text{kg}} \nonumber
       \end{eqnarray}
       We still \underline{can not} sum the two terms in the r.h.s., as the first term involves $\left[\text{kJ/kg}\right]$ whereas the second term is $\left[\text{J/kg}\right]$, therefore
       \begin{eqnarray}
         h_{2} &=& 376.70\frc{\text{kJ}}{\text{kg}} + 2517 \frc{\text{J}}{\text{kg}} \nonumber\\
              &=& 376.70\frc{\text{kJ}}{\text{kg}} + 2517 \frc{\cancel{\text{J}}}{\text{kg}} \red{\times \frc{1\text{ kJ}}{1000 \cancel{\text{J}}}}\nonumber\\
              &=& 379.22 \frc{\text{kJ}}{\text{kg}}\nonumber
       \end{eqnarray}
       


     \end{description}
%
\end{list}

  \includepdf[scale=1,pages=-,pagecommand={}, fitpaper]{./Pics/ChemEng_UnitConv.pdf}

       
\chapter{Table of Properties of Saturated and Superheated Fluids}\label{Appendix:Saturated_SH_Tables}

Extracted from~\cite{Moran_Book}.

  \includepdf[scale=1,pages=-,pagecommand={}, fitpaper]{./../Pics/ChemEng_AllTables.pdf}

       
\chapter{Summary of Logarithms and Exponential Properties}
{\it This is not examinable} -- it is here so that you can see where some of the notations, operations and results of earlier sections come from. 

%%%% ETOC
\localtableofcontents

%%%
%%% EXPONENTS
%%%
\section{Exponents}
Assuming that $a$, $b$, $m$ and $n$ are real numbers, the following properties of exponents hold:
\begin{center}
  \begin{tabular}{||c l | c l ||}
    \hline\hline
     1 & $a^{m}a^{n} = a^{m+n}$ & 6 & $\frc{a^{m}}{a^{n}} = a^{m-n}$, $\forall a\neq 0$ \\
     2 & $\left(a^{m}\right)^{n} = a^{mn}$ & 7 &  $\left(ab\right)^{m} = a^{m}b^{m}$ \\
     3 & $\left(\frc{a}{b}\right)^{m} = \frc{a^{m}}{b^{m}}$, $\forall b\neq 0$ & 8 & $a^{-m}=\frc{1}{a^{m}}$, $\forall a\neq 0$ \\
     4 & $a^{1/n} = \sqrt[n]{a}$ & 9 & $a^{0} = 1$, $\forall a\neq 0$ \\
     5 & $a^{m/n} = \sqrt[n]{a^{m}} = \left(\sqrt[n]{a}\right)^{m}$ & & \\ 
    \hline\hline
  \end{tabular}
\end{center}


%%%
%%% LOGARITHMS
%%%
\section{Logarithms}
Let's define $y = \log_{a}x$ if ({\it and only if}) $x=a^{y}$, $\forall a>0$. Given the {\it Euler number}, $e$,
\begin{displaymath}
e = \sum\limits_{n=0}^{\infty}\frc{1}{n!}\sim 2.71828
\end{displaymath}
we can define $\ln x = \log_{e}{x}$, referred as the natural logarithm. The following properties of logarithms hold:
\begin{center}
  \begin{tabular}{||c l | c l||}
     \hline\hline
       1  & $\log_{a}b = \frc{\log_{10}{a}}{\log_{10}{b}}$ & 1' & $\log_{a}b = \frc{\ln{a}}{\ln{b}}$\\
     \hline
       2  & $\log_{a}{xy} = \log_{a}{x}+\log_{a}{y}$ & 2' &$\ln{xy} = \ln{x}+\ln{y}$  \\
     \hline
       3  & $\log_{a}{\frc{x}{y}} = \log_{a}{x}-\log_{a}{y}$ & 3' & $\ln{\frc{x}{y}} = \ln{x}-\ln{y}$ \\
     \hline
       4  & $\log_{a}{x^{y}} = y\cdot\log_{a}{x}$ & 4' & $\ln{x^{y}} = y\cdot\ln{x}$ \\
     \hline
       5  & $\log_{a}{a^{x}} = x$ & 5' & $\ln{e^{x}} = x$ \\
     \hline
       6  & $a^{\log_{a}{x}} = x$   & 6' & $e^{\ln{x}} = x$ \\
     \hline
       7  & $\log_{a}{a} = 1$, $\forall a>0$ & 7' & $\ln{e} = 1$ \\ 
     \hline
       8  & $\log_{a}1 = 0$, $\forall a > 0$ & 8' & $\ln{1} = 0$ \\
     \hline\hline
  \end{tabular}  
\end{center}
 
       
\chapter{Calculus' Background for Thermodynamics}\label{Appendix_Calculus}
{\it This is not examinable} -- it is here so that you can see where some of the notations, operations and results of earlier sections came from. Details of the contents of this Appendix can be found in \cite{Leithold_Book,Kallo_1955,Strang_Book} or in any {\it Calculus} text-book.
\bigskip

%%%
%%% SECTION
%%%
\section{Vector Calculus}

The operator {\it del} (or {\it nabla}),\index{{\it del} ($\nabla$) operator}\index{{\it nabla} ($\nabla$) operator}\index{$\nabla$}
\begin{displaymath}
  \nabla \equiv \left(\frc{\partial}{\partial x}, \frac{\partial}{\partial y}, \frac{\partial}{\partial z}\right)
\end{displaymath} 
is both a vector and a differential operator and can be used to define,
\begin{enumerate}
%
  \item Gradient: operates on a scalar field $\phi$, e.g., $T$, $\rho$, $\cdots$\index{Gradient}
     \begin{displaymath}
        \text{grad}\phi \equiv \nabla\phi \equiv \left(\frc{\partial\phi}{\partial x}, \frac{\partial\phi}{\partial y}, \frac{\partial\phi}{\partial z}\right)
     \end{displaymath}
%
  \item Divergence: operates on a vector field  $\theta = \left(\theta_{x}, \theta_{y}, \theta_{z}\right) $, e.g., velocity field.\index{Divergence}
     \begin{displaymath}
        \text{div}\theta \equiv \nabla\cdot\theta \equiv \frc{\partial\theta_{x}}{\partial x} + \frac{\partial\theta_{y}}{\partial y} + \frac{\partial\theta_{z}}{\partial z}
     \end{displaymath}
%
  \item Curl: operates on a vector field  $\theta = \left(\theta_{x}, \theta_{y}, \theta_{z}\right)$,\index{Curl}
     \begin{displaymath}
        \text{curl}\theta \equiv \nabla\times\theta \equiv \begin{pmatrix} i & j & k \\ \frc{\partial}{\partial x} & \frc{\partial}{\partial y} & \frc{\partial}{\partial z} \\ \theta_{x} & \theta_{y} & \theta_{z} \end{pmatrix}
     \end{displaymath}
%
   \item Laplacian: operates on a scalar field $\phi$,\index{Laplacian}
      \begin{displaymath}
         \text{div}\left(\text{grad}\phi\right) \equiv \nabla\cdot\nabla\phi \equiv \nabla^{2}\phi \equiv \frc{\partial^{2}\phi}{\partial x^{2}} + \frc{\partial^{2}\phi}{\partial y^{2}} + \frc{\partial^{2}\phi}{\partial z^{2}}
      \end{displaymath}
%
\end{enumerate}


%%%
%%% SECTION
%%%
\section{Some Basic Derivatives/Integration Operations}

\begin{center}
  \begin{tabular}{|| l l | l l ||}
    \hline\hline
       {\bf f(x)}  & {\bf f'(x)}  & {\bf f(x)}  & {\bf f'(x)}  \\
    \hline\hline
       $x^{n}$      &  $nx^{n-1}$   & $\ln{x}$    & $x^{-1}$      \\
       $e^{x}$      &  $e^{x}$      & $sin(x)$   & $cos(x)$     \\
       $cos(x)$    &  $-sin(x)$   & $tan(x)$    & $sec^{2}(x)$  \\
    \hline\hline
       {\bf f(x)}  &  {\bf $\int$f(x)dx} & {\bf f(x)}  &  {\bf $\int$f(x)dx} \\
    \hline\hline
       $e^{x}$      & $e^{x}+\mathcal{C}$& $x^{n}$ for $n\neq -1$ & $\frc{x^{n+1}}{n+1}+\mathcal{C}$ \\
                    &                  &                        & \\
       $1/x$ for $x\neq 0$& $\ln{|x|}+\mathcal{C}$ & $a^{x}$ for $a\neq 1$, $a>0$ & $\frc{a^{x}}{\ln{a}}+\mathcal{C}$\\
                   &                  &                         & \\
       $e^{ax}$ for $a\neq 0$  & $\frc{e^{ax}}{a}+\mathcal{C}$ & $cos(ax)$ for $a\neq0$ & $\frc{1}{a}sin(ax)+\mathcal{C}$\\
                   &                  &                        & \\
       $sin(ax)$ for $a\neq 0$ & $-\frc{1}{a}cos(ax)+\mathcal{C}$& & \\
    \hline\hline
  \end{tabular}
\end{center}

\begin{itemize}
%
  \item Derivative of a sum:
    \begin{displaymath}
       \frc{d}{dx}\left[f(x)+g(x)\right] = \frc{d}{dx}f(x) + \frc{d}{dx}g(x) = f'(x)+g'(x) 
    \end{displaymath}
%
  \item Derivative with a constant factor $c$:
    \begin{displaymath}
       \frc{d}{dx}\left[c f(x)\right] = c\frc{d}{dx}f(x) = cf'(x)
    \end{displaymath}
%
  \item Derivative of a product:
    \begin{displaymath}
       \frc{d}{dx}\left[f(x)g(x)\right] = f(x)g'(x) + f'(x)g(x)
    \end{displaymath}
%
  \item Derivative of a quotient:
    \begin{displaymath}
      \frc{d}{dx}\left[\frc{f(x)}{g(x)}\right] = \frc{g(x)f'(x)-f(x)g'(x)}{g^{2}(x)}
    \end{displaymath}
%
  \item Chain rule (or function of a function):
    \begin{displaymath}
      \frc{d}{dx}f\left[g(x)\right] = f'\left[g(x)\right]g'(x)
    \end{displaymath}
%
  \item Chain rule of a linear function:
    \begin{displaymath}
      \frc{d}{dx} \left[f(ax+b)\right] = a f'(ax+b)
    \end{displaymath}
%
  \item Integral of a function of a linear function:
    \begin{displaymath}
       \int\left[f'(ax+b)\right]dx = \frc{1}{a}f(ax+b) + \mathcal{C}
    \end{displaymath}
%
  \item Integral of a chain rule derivative:
    \begin{displaymath}
       \int\left\{f'\left[g(x)\right]g'(x)\right\}dx = f\left[g(x)\right] + \mathcal{C}
    \end{displaymath}
%
  \item Integral of a sum:
    \begin{displaymath}
      \int\left[f(x)+g(x)\right]dx = \int f(x)dx + \int g(x)dx
    \end{displaymath}
%
  \item Integral with a constant function:
    \begin{displaymath}
       \int c f(x)dx = c\int f(x)dx
    \end{displaymath}
%
  \item Integration by parts:
    \begin{displaymath}
       \int\left[f(x)g'(x)\right]dx = f(x)g(x) - \int\left[f'(x)g(x)\right]dx
    \end{displaymath}
%
  \item Definite integral (if $f'(x)$ is continuous at $a<x<b$):
    \begin{eqnarray}
       && \int\limits_{a}^{b}f(x)dx = - \int\limits_{b}^{a}f(x)dx \nonumber \\
       && \int\limits_{a}^{b}f'(x)dx = \left.f(x)\right|_{a}^{b} = \lim_{x\rightarrow b^{-}}f(x)-\lim_{x\rightarrow a^{+}}f(x)\nonumber
    \end{eqnarray}
%
  \item Substitution:
    \begin{eqnarray}
        \int f(x)dx = \int f(x(u))\frc{dx}{du}du && \text{(indefinite integral)} \nonumber \\
        \int\limits_{a}^{b} f(x) = \int\limits_{u(a)}^{u(b)}f(x(u))\frc{dx}{du}du  && \text{(definite integral)} \nonumber
    \end{eqnarray}
%
  \item Integration by parts:
    \begin{eqnarray}
       \int f(x)g'(x)dx = f(x)g(x) - \int f'(x)g(x) && \text{(indefinite integral)} \nonumber \\
       \int\limits_{a}^{b} f(x)g'(x)dx = \left.f(x)g(x)\right|_{a}^{b} - \int\limits_{a}^{b} f'(x)g(x) && \text{(definite integral)} \nonumber        
    \end{eqnarray}
%
\end{itemize}


%%%
%%% SECTION
%%%
\section{Partial Derivatives and Total Differentials}

%%% SUBSECTION
\subsection{Partial Derivatives:} Given a function $\phi\left(x_{1},x_{2},x_{3},\cdots,x_{n-1},x_{n}\right)$ of $n$ independent variables, the partial derivative of $\phi$ with respect to $x_{i}$, holding the other $n-1$ independent variables constant, is defined as,
  \begin{displaymath}
    \left(\frc{\partial\phi}{\partial x_{i}}\right)_{x_{j\neq i}} = \lim_{\Delta x_{i}\rightarrow 0}\left\{\frc{\phi\left(x_{1},x_{2},\cdots,x_{i}+\Delta x_{i},\cdots,x_{n}\right)-\phi\left(x_{1},x_{2},\cdots,x_{i},\cdots,x_{n}\right)}{\Delta x_{i}}\right\}
  \end{displaymath}

{\bf Example:} A pure fluid with ideal gas behaviour, the pressure can be expressed as a function of the number of mols ($n$), volume ($V$) and temperature ($T$),
  \begin{displaymath}
     P(n,V,T) = \frc{n R T}{V},
  \end{displaymath}
thus,
  \begin{displaymath}
     \left(\frc{\partial P}{\partial n}\right)_{V,T} = \frc{RT}{V}\hspace{1cm} \left(\frc{\partial P}{\partial V}\right)_{n,T} = -\frc{n R T}{V^{2}} \hspace{1cm} \left(\frc{\partial P}{\partial T}\right)_{n,V} = \frc{n R}{V}\d{T}
  \end{displaymath}


%%% SUBSECTION
\subsection{Total Differentials:}\label{Appendix_Calculus:TotalDifferential} Given a function $\phi\left(x_{1},x_{2},x_{3},\cdots,x_{n-1},x_{n}\right)$ of $n$ independent variables, the {\it total differential} of $\phi$, $d\phi$, is defined as
  \begin{eqnarray}
     d\phi &=& \sum\limits_{i=1}^{n}\left(\frc{\partial \phi}{\partial x_{i}}\right)_{x_{j\neq i}} d x_{i} \nonumber \\
     &=& \left(\frc{\partial\phi}{\partial x_{1}}\right)_{x_{2},\cdots,x_{n}} d x_{1} + \left(\frc{\partial\phi}{\partial x_{2}}\right)_{x_{1},x_{3},\cdots,x_{n}} dx_{2} + \cdots +  \left(\frc{\partial\phi}{\partial x_{n}}\right)_{x_{1},x_{2},\cdots,x_{n-1}} d x_{n} \nonumber 
  \end{eqnarray}
where $d x_{i}$ is an infinitesimal small increment in $x_{i}$.

\noindent
{\bf Example:} Infinitesimal changes in the ideal gas pressure are expressed as,
  \begin{eqnarray}
      d P &=& \left(\frc{\partial P}{\partial n}\right)_{V,T}\d{n} + \left(\frc{\partial P}{\partial V}\right)_{n,T}\d{V} + \left(\frc{\partial P}{\partial T}\right)_{n,V}\d{T} \nonumber \\
     &=& \frc{R T}{V} d n - \frc{n R T}{V^{2}} d V + \frc{n R}{V} d T. \nonumber 
  \end{eqnarray}

%%% SUBSECTION
\subsection{Properties of Partial Derivatives}\label{Appendix_Calculus:Properties}
  \begin{enumerate}[(i)]
%
     \item The order of differentiation in mixed second derivatives is immaterial, i.e.,
        \begin{displaymath}
           \left[\frc{\partial}{\partial y}\left(\frc{\partial\phi}{\partial x}\right)_{y}\right]_{x} = \left[\frc{\partial}{\partial x}\left(\frc{\partial\phi}{\partial y}\right)_{x}\right]_{y} \hspace{1cm}\Longleftrightarrow\hspace{1cm} \frc{\partial^{2}\phi}{\partial x\partial y} = \frc{\partial^{2}\phi}{\partial y\partial x}
        \end{displaymath}
%
     \item Cyclic rule:
        \begin{displaymath}
           \left(\frc{\partial\phi}{\partial x}\right)_{y}\left(\frc{\partial y}{\partial \phi}\right)_{x}\left(\frc{\partial x}{\partial y}\right)_{\phi} = -1
        \end{displaymath}
%
     \item Given $\phi(x,y)$ and $\varphi(x,y)$:
        \begin{enumerate}[(a)]
           \item $\left(\frc{\partial\phi}{\partial\varphi}\right)_{x} = \left(\frc{\partial\phi}{\partial y}\right)_{x}\left(\frc{\partial y}{\partial\varphi}\right)_{x}$  (chain rule);
           \item $\left(\frc{\partial\phi}{\partial x}\right)_{\varphi} = \left(\frc{\partial\phi}{\partial x}\right)_{y} + \left(\frc{\partial\phi}{\partial y}\right)_{x}\left(\frc{\partial y}{\partial x}\right)_{\varphi} $
        \end{enumerate}  
%
  \end{enumerate}

%%%
%%% SECTION
%%%
\section{The Mean Value Theorem and l'H\^opital's Rule}\label{Appendix:lHopital}

\begin{theorem}[Mean value]\index{Mean value theorem}\label{Appendix:MeanValueTheorem}
Suppose $f(x)$ is continuous in the closed interval $a\leq x\leq b$ and has derivatives everywhere in the open interval $a<x<b$. Then,
     \begin{equation}
       \frc{f(a)-f(b)}{b-a} = f'(c)\;\;\;\text{ at some point } a<c<b.
     \end{equation}
\end{theorem}

\begin{theorem}[Rolle's theorem, i.e., extrema of a function]\index{Mean value theorem ! Rolle's theorem}
   Suppose $f(a) = f(b) = 0$ (zero at endpoints). Then $f'(c) = 0$ at some point within $a<c<b$.
\end{theorem}

\begin{theorem}[l'H\^opital rule]\index{Mean value theorem ! L'H\^opital rule}\index{L'H\^opital rule}
   Suppose $f(x)$ and $g(x)$ are differentiable and $g'(x)\neq 0$ near a point $a$ (except possibly at $a$). Suppose that
    \begin{displaymath}
      \lim_{x\rightarrow a} f(x) = 0 \;\;\text{ and }\;\; \lim_{x\rightarrow a} g(x) = 0,
    \end{displaymath}
or that
    \begin{displaymath}
      \lim_{x\rightarrow a} f(x) = \pm\infty \;\;\text{ and }\;\; \lim_{x\rightarrow a} g(x) = \pm\infty,
    \end{displaymath}
$\left(\text{i.e., an indeterminate quotient, }\frc{0}{0} \text{ or }\frc{\infty}{\infty}\right)$. Then
    \begin{equation}
        \lim_{x\rightarrow a}\frc{f(x)}{g(x)} = \lim_{x\rightarrow a}\frc{f'(x)}{g'(x)},
    \end{equation}
if the limit on the right side exists (or is $\infty$ or $-\infty$).
%both approach zero as $x\rightarrow a$. Then $\frc{f(x)}{g(x)}$ approaches the same limit as $\frc{f'(x)}{g'(x)}$, if this second limit exists,
%    \begin{equation}
%        \lim_{x\rightarrow a}\frc{f(x)}{g(x)} = \lim_{x\rightarrow a}\frc{f'(x)}{g'(x)}.
%    \end{equation}
%   This limit often is $\frc{f'(a)}{g'(a)}$.
    \begin{list}{\bf Example \arabic{qcounter}:~}{\usecounter{qcounter}}
%       
       \item Find $\lim\limits_{x\rightarrow\infty} \frc{5x-2}{7x+3}$.
           \begin{eqnarray}
              \lim_{x\rightarrow\infty}\frc{5x-2}{7x+3} &=& \frc{\infty}{\infty} \nonumber \\
                                                   &=& \lim_{x\rightarrow\infty}\frc{\left[5x-2\right]'}{\left[7x+3\right]'} = \lim_{x\rightarrow\infty} \frc{5}{7} = \frc{5}{7}\nonumber
           \end{eqnarray}
%
      \item Find $\lim\limits_{x\rightarrow -2}\frc{x+2}{\ln{(x+3)}}$. 
           \begin{eqnarray}
              \lim_{x\rightarrow -2}\frc{x+2}{\ln{(x+3)}} &=& \frc{0}{0} \nonumber \\                                                   &=& \lim_{x\rightarrow -2}\frc{\left[x+2\right]'}{\left[\ln{(x+3)}\right]} = \lim_{x\rightarrow -2} \frc{1}{\frc{1}{x+3}} = \lim_{x\rightarrow -2} \left(x+3\right) = 1 \nonumber
           \end{eqnarray}
         
%
    \end{list}


\end{theorem}

%%%
%%% SECTION
%%%
\section{Line Integrals}\index{Line integral}

%%%
\subsection{Exact and Inexact Differential}\index{Line integral!Exact differential}\index{Line integral!Inexact differential}
Consider that $\mathbf{\Psi}$ is a function of the independent variables $x_{j}$ (with $j=1,2,\cdots,n$),  $\Psi_{i}=\Psi_{i}\left(x_{1},x_{2},\cdots,x_{n}\right)$. An infinitesimal quantity,
   \begin{displaymath}
      dz = \sum\limits_{i=1}^{n} \Psi_{i}\left(x_{1},x_{2},\cdots,x_{n}\right) d x_{i} =  \Psi_{1} d x_{1} + \Psi_{2} d x_{2} +\cdots + \Psi_{n} d x_{n},
   \end{displaymath}
is called {\it linear differential}\index{Linear differential}. If we focus on a two-dimensional problem, i.e., $\mathbf{\Psi}=\left\{M(x,y),N(x,y)\right\}$,
   \begin{equation}
      dz = M d x + N d y\label{Appendix_Calculus:Eqn:ExactLinearDifferential}
   \end{equation}

\medskip
Equation~\ref{Appendix_Calculus:Eqn:ExactLinearDifferential} is an {\it exact differential}\index{Exact differential} if, and only if, there is a function of $x$ and $y$, $\Phi(x,y)$, such that $d \Phi= d z$ for all values of $x$ and $y$. This is equivalent to
   \begin{displaymath}
        \Partial[M]{y}{x} = \Partial[N]{x}{y}.
   \end{displaymath}
The equation
   \begin{equation}
      dw = M^{\prime} d x + N^{\prime} d y,\label{Appendix_Calculus:Eqn:InexactLinearDifferential}
   \end{equation}
 is an {\it inexact differential}\index{Inexact differential} if, and only if, there is no function $\Phi(x,y)$, such that $d \Phi = d w$ for all values of $x$ and $y$, thus,
   \begin{displaymath}
        \Partial[M^{\prime}]{y}{x} \neq \Partial[N^{\prime}]{x}{y}.
   \end{displaymath}

%%%
\subsection{Fundamental Theorem for Line Integrals}\index{Line integral!Fundamental theorem}

\begin{theorem}
   If {\bf F} is a gradient or conservative vector field, i.e., $\mathbf{F}=\mathbf{\nabla}f(x,y)=\langle f_{x}, f_{y}\rangle$ for a {\it potential function} $f$ for the field, and $\mathcal{C}$ is a curve with endpoints $P_{0}=\left(x_{0},y_{0}\right)$ and $P_{1}=\left(x_{1},y_{1}\right)$,
      \begin{eqnarray}
         \int\limits_{\mathcal{C}}\mathbf{F}\cdot d\mathbf{r} &=& \int\limits_{\mathcal{C}}\mathbf{\nabla}f d \mathbf{r} = \left.f(x,y)\right|_{P_{0}}^{P_{1}}\\
                                                           &=& f\left(P_{1}\right)-f\left(P_{0}\right) = f\left(x_{1},y_{1}\right) - f\left(x_{0},y_{0}\right)\nonumber.
      \end{eqnarray}
\end{theorem} 
That is, for gradient fields the line integral is independent of the path taken, i.e., it depends only on the endpoints of $\mathcal{C}$. We call such a line integral {\it path independent}.
\medskip

The line integral of a vector field over a {\it simple} (i.e., non-intersecting) {\it closed} (i.e., no endpoints) curve $\mathcal{C}$ is denoted as,\index{Line integral}
        \begin{equation}
           \oint_{\mathcal{C}}\mathbf{F}\cdot d \mathbf{r} = 0,
        \end{equation}
i.e., the line integral around all closed paths is 0 $\leftrightarrow$ -- {\it path independence}.
   
       
\chapter{Introduction to Numerical Methods relevant to Thermodynamics}\label{Appendix_NumMethods}


%%%% ETOC
\localtableofcontents

{\it This is not examinable} -- it is here so that you can see where some of the notations, operations and results of earlier sections came from. Details of the contents of this Appendix can be found in \cite{Atkinson_Book_Newton,Atkinson_Book_Interpolation,NumericalRecipes_Interpolation,NumericalRecipes_Newton} or in any text-book of {\it Mathematical or Numerical Methods} for engineering.

%%%
%%% SECTION
%%%
\section{Linear Interpolation}\label{LinearInterpolation}\index{Linear interpolation}

Given a continuous and unknown function $f(x)$, defined at a set of points  $x_{1} < \cdots < x_{i} < \cdots < x_{N}$. Interpolation is the process of determining a polynomial expression to calculate the pair $\left[x_{k}, f\left(x_{k}\right)\right]$ based on neighbours discrete coordinates $\left\{\left[x_{1},f\left(x_{1}\right)\right], \cdots, \left[x_{N},f\left(x_{N}\right)\right]\right\}$. 

Consider a set of discrete data points,
  \begin{center}
    \begin{tabular}{c | c }
        $\mathbf{x}$   & $\mathbf{f\left(x_{i}\right)}$ \\
        \hline
           $x_{1}$ &  $f\left(x_{1}\right)$ \\
           $x_{2}$ &  $f\left(x_{2}\right)$ \\
           $x_{3}$ &  $f\left(x_{3}\right)$ \\
           $x_{4}$ &  $f\left(x_{4}\right)$ \\
    \end{tabular}
  \end{center}
that are a subset of a continuous and smooth function $y=f(x)$ (Fig.~\ref{Appendix:Fig:Interpolation}). Polynomials of order $n\ge 1$ can be generated to represent this function. High-order polynomials can more accurately fit the discrete coordinatess than low-order polynomials. In Fig.~\ref{Appendix:Fig:Interpolation}, let's assume the discrete pairs 
  \begin{displaymath}
     \left\{\left[x_{1},f\left(x_{1}\right)\right], \left[x_{2},f\left(x_{2}\right)\right],\left[x_{3},f\left(x_{3}\right)\right], \left[x_{4},f\left(x_{4}\right)\right]\right\}
  \end{displaymath}
are known, and one wants to determine the value of the function $f$ at $x_{2} < x_{k} < x_{3}$. If the interval $\Delta x= x_{3}-x_{2}$ is sufficiently small, a linear function can be used to fit these coordinates,
   \begin{displaymath}
       f\left(x_{k}\right) = f\left(x_{2}\right) + m\left(x_{k}-x_{2}\right),%\label{LinearInterpolation:Eqn1}
   \end{displaymath}
where 
   \begin{displaymath}
      m = \frc{f\left(x_{3}\right)-f\left(x_{2}\right)}{x_{3}-x_{2}}.
   \end{displaymath}
If $m$ is replaced in the previous equation, %Eqn.~\ref{LinearInterpolation:Eqn1},
   \begin{displaymath}
       f\left(x_{k}\right) = \frc{f\left(x_{2}\right)\left(x_{3}-x_{k}\right) + f\left(x_{3}\right)\left(x_{k}-x_{2}\right)}{x_{3}-x_{2}}.
   \end{displaymath}
   
   \begin{shaded}
      Or for a general case with $x_{a} < x_{k} < x_{b}$,
        \begin{equation}\label{LinearInterpolation:Eqn1}
            f\left(x_{k}\right) = \frc{f\left(x_{a}\right)\left(x_{b}-x_{k}\right) + f\left(x_{b}\right)\left(x_{k}-x_{a}\right)}{x_{b}-x_{a}}.
        \end{equation}
   \end{shaded}

%%% Figure
     \begin{figure}[h]\label{Appendix:Fig:Interpolation}%
        \begin{center}
          \includegraphics[width=\columnwidth,clip]{./../Pics/Interpolation}
           \caption{Smooth function $f(x)$ (solid blue line) may be more accurately interpolated by a high-order polynomial (black dotted line) than by a low-order polynomial (solid black line).} 
        \end{center}
      \end{figure}

   % Example
   \begin{MyExample}{\begin{center}{\bf Example}\end{center}}
      \begin{example}
         Given a table of values for $f(x)=\tan{x}$ for a few values of $x$,
            \begin{center}
               \begin{tabular}{c | c c c c}
                   $x$        & 1.00   & 1.10   & 1.20   & 1.30   \\
                   \hline
                   $\tan{x}$  & 1.5574 & 1.9648 & 2.5722 & 3.6021 \\
               \end{tabular}
            \end{center}
            Estimate $\tan{(1.15)}$ and $\tan{(1.23)}$.
     \end{example}

% SOLUTION
       \noindent{\bf Solution:} For $\left(x_{a}=1.10\right) < \left(x_{k}=1.15\right) < \left(x_{b}=1.20\right)$,
               \begin{eqnarray}
                  f\left(x_{k}\right) &=& \frc{f\left(x_{a}\right)\left(x_{b}-x_{k}\right) + f\left(x_{b}\right)\left(x_{k}-x_{a}\right)}{x_{b}-x_{a}} \nonumber \\
                                     &=& \frc{1.9648\times(1.20-1.15) + 2.5722\times(1.15-1.10)}{1.20-1.10} = 2.2685 \nonumber
               \end{eqnarray}

For  $\left(x_{a}=1.20\right) < \left(x_{k}=1.23\right) < \left(x_{b}=1.30\right)$,
               \begin{eqnarray}
                  f\left(x_{k}\right) &=& \frc{f\left(x_{a}\right)\left(x_{b}-x_{k}\right) + f\left(x_{b}\right)\left(x_{k}-x_{a}\right)}{x_{b}-x_{a}} \nonumber \\
                                     &=& \frc{2.5722\times(1.30-1.23) + 3.6021\times(1.23-1.20)}{1.30-1.20} = 2.8812 \nonumber
               \end{eqnarray}
   \end{MyExample}

   % Example
   \begin{MyExample}{\begin{center}{\bf Example}\end{center}}
      \begin{example}
         Calculate specific volume $\left(v, \text{in m}^{3}\text{.kg}^{-1}\right)$, internal energy $\left(u, \text{in kJ.kg}^{-1}\right)$ and entropy $\left(s, \text{in kJ.(kg.K)}^{-1}\right)$ of saturated water vapour at 133.45$^{\circ}$C.
     \end{example}

% SOLUTION
       \noindent{\bf Solution:} From Appendix~\ref{Appendix:Saturated_SH_Tables} (Table A-2), for $T_{a}(=130.0) < T_{k} (= 133.45) < T_{b} (=140.0)^{\circ}C$, thus:
               \begin{eqnarray}
                  v\left(T_{k}\right) &=& \frc{v\left(T_{a}\right)\left(T_{b}-T_{k}\right) + v\left(T_{b}\right)\left(T_{k}-T_{a}\right)}{T_{b}-T_{a}} \nonumber \\
                                     &=& \frc{0.6685\times(140.0-133.45) + 0.5089\times(133.45-130.0)}{140.0-130.0} = 0.6134 \text{ m}^{3}\text{.kg}^{-1}\nonumber \\
                                     && \nonumber \\
                  u\left(T_{k}\right) &=& \frc{u\left(T_{a}\right)\left(T_{b}-T_{k}\right) + u\left(T_{b}\right)\left(T_{k}-T_{a}\right)}{T_{b}-T_{a}} \nonumber \\
                                     &=& \frc{2539.9\times(140.0-133.45) + 2550.0\times(133.45-130.0)}{140.0-130.0} = 2543.39 \text{ kJ.kg}^{-1}\nonumber \\
                                     && \nonumber \\
                  s\left(T_{k}\right) &=& \frc{s\left(T_{a}\right)\left(T_{b}-T_{k}\right) + s\left(T_{b}\right)\left(T_{k}-T_{a}\right)}{T_{b}-T_{a}} \nonumber \\
                                     &=& \frc{7.0269\times(140.0-133.45) + 6.9299\times(133.45-130.0)}{140.0-130.0} = 6.9934 \text{ kJ.}\left(\text{kg.K}\right)^{-1}\nonumber 
               \end{eqnarray}
   \end{MyExample}


%%%
%%% SECTION
%%%
\section{Root-Finder Methods}\label{Section:RootFinderMethods}\index{Root-Finder Methods}

%%% SUBSECTION
\subsection{Motivation}
Given a smooth, {\it continuous} and {\it fully differentiable} function 
  \begin{displaymath}
     y = f(x) \hspace{3cm} \text{ with } x\in\mathbb{R}.
  \end{displaymath}
We aim to find the root $x=\psi$ of the function 
  \begin{displaymath}
     f(x) = 0.
  \end{displaymath}
The first step is to estimate $x_{0}$ that results in $f\left(x_{0}\right)\neq 0$ and may lead to a new estimate $x_{1}$. The procedure is repeated until $f\left(x_{n}\right)\rightarrow 0$ (\ie $x_{n}\approx\psi$), where $n$ is the number of repetitions (or {\it iterations}). There are several methods designed to solve non-linear equations, i.e., find the rrot of the function, here we will focus on the most popular {\it Newton-Raphson} method that combines simplicity and power.


%%% SUBSECTION
\subsection{Newton-Raphson Iterative Method}\label{Section:RootFinderMethods:NewtonRaphson}\index{Root-Finder Methods!Newton-Raphson method}
Let's assume that $x_{0}$ is a good estimate of the root$\psi$ and $\psi = x_{0} + h$. Since the root of the function $f(x)$ is $\psi$ and $h = \psi 0 x_{0}$, $h$ represents the distance between the initial estimate (or guess) and the root. Assuming $h$ is very (or {\it infinitesimal}) small, we can linearly approximate the function,
   \begin{displaymath}
        f\left(\psi\right) = 0 = f\left(x_{0}+h\right) \approx f\left(x_{0}\right) + h f'\left(x_{0}\right).
   \end{displaymath}
Therefore, except if $f'\left(x_{0}\right)$ is close to $0$, 
   \begin{displaymath}
        h \approx -\frc{f\left(x_{0}\right)}{f'\left(x_{0}\right)} \hspace{.3cm} \Longrightarrow \hspace{.3cm} \psi = x_{0} + h \approx x_{0} -\frc{f\left(x_{0}\right)}{f'\left(x_{0}\right)}.
   \end{displaymath}
%%% Figure
     \begin{figure}[h]\label{Appendix:Fig:NewtonRaphson}%
        \begin{center}
         \vbox{
           \hbox{\includegraphics[width=\columnwidth,height=10cm]{./../Pics/NewtonRaphson2}}
           \hbox{\includegraphics[width=\columnwidth,height=10cm]{./../Pics/NewtonRaphson3}}}
           \vspace{-1cm}
           \caption{Graphic representation of the Newton-Raphson iterative method: (top) solution of the smooth and continuous function $f(x)$ (solid red line) is approximated from the initial estimate $x_{0}$ to the final solution $x=\psi$; (bottom) initial estimate $x_{0}$ is far away from the root $\psi$ and the solution may diverge. Blue solid lines are tangent of the function at $x_{i}$.} 
        \end{center}
      \end{figure}
%
This expression represents an improvement of the original estimate, i.e.,  
   \begin{displaymath}
        x_{1} = x_{0} -\frc{f\left(x_{0}\right)}{f'\left(x_{0}\right)}.
   \end{displaymath}
The next estimate, $x_{2}$, is obtained from $x_{1}$, 
   \begin{displaymath}
        x_{2} = x_{1} -\frc{f\left(x_{1}\right)}{f'\left(x_{1}\right)}.
   \end{displaymath}

   \begin{shaded}
      We can generalise this expression for the $n-${\it th iteration},
         \begin{equation}
            x_{n+1} = x_{n} -\frc{f\left(x_{n}\right)}{f'\left(x_{n}\right)}.\label{NewtonRaphson:Eqn1}
         \end{equation}
   \end{shaded}

Figure~\ref{Appendix:Fig:NewtonRaphson}(a) shows a geometrical representation of the Newton-Raphson iterative method, where $m=f(x)$ is the tangent (blue) line at the coordinate pair $\left[x_{0},f\left(x_{0}\right)\right]$,
   \begin{displaymath}
     m = f\left(x_{0}\right) + \left(x-x_{0}\right)f'\left(x_{0}\right).
   \end{displaymath}
Let $x_{1}$ be the {\it x-intercept} of the tangent line, therefore
   \begin{displaymath}
     x_{1} = x_{0} - \frc{f\left(x_{0}\right)}{f'\left(x_{0}\right)}
   \end{displaymath}
The tangent line is a geometrical representation of the Newton-Raphson iterative method, Eqn.~\ref{NewtonRaphson:Eqn1}, as the estimates gradually tend to the root $\psi$ of the function. As it can be seen in the Fig.~\ref{Appendix:Fig:NewtonRaphson}(b), if the initial estimate $x_{0}$ is not close enough of the root $\psi$, the solution may not {\it converge}. In fact, the Newton-Raphson iterative method works most of the time if the initial estimate is {\it good enough}.

 From the {\it mean value theorem} (Theorem~\ref{Appendix:MeanValueTheorem}), let the function $f(x)$ be such that, 
   \begin{enumerate}[(a)]
      \item it is continuously differentiable in some open interval containing the solution $x=\psi$;
      \item $\left|f'(\psi)\right| < 1$.
   \end{enumerate}
Then there is a number $\epsilon > 0$ such that the iteration $x_{k+1}=f\left(x_{k}\right)$ {\it converges} whenever $x_{0}$ is chosen in $\left|x_{0}-\psi\right|\leq\epsilon$.

For bounded $x\in\mathbb{R}$ (\ie contained in the interval $a\leq x \leq b$), if $f''(x)$ exists and is continuous on $[a,b]$ and $\psi$ is a root of $f(x)$, that is, $f(\psi)=0$ and $f'(\psi)\neq0$. Thus for a function $g(x)$
    \begin{displaymath}
        g(x) = x - \frc{f(x)}{f'(x)}
    \end{displaymath}
with
    \begin{displaymath}
        g'(x) = 1 - \frc{[f'(x)]^{2} - f(x)f''(x)}{[f'(x)]^{2}} = \frc{f(x)f''(x)}{[f'(x)]^{2}},
    \end{displaymath}
and
    \begin{displaymath}
        g'(\psi) = \frc{f(\psi)f''(\psi)}{[f'(\psi)]^{2}}=0, \text{ since } f(\psi) = 0 \text{ and } f'(\psi) \neq 0
    \end{displaymath}
As $g'(x)$ is continuous, this means that there is a small neighbourhood around the root $x=\psi$ such that for all points $x$ in that neighbourhood, $\left|g'(x)\right|<1$.  Therefore, if $g(x)$ is chosen as above and the initial estimate $x_{0}$ is chosen {\it sufficiently close to the root} $x=\psi$, then the Newton-Raphson method is {\it guaranteed to converge}. 

Algorithm~\ref{Algorithm:NewtonRaphson} highlights the steps towards find the root of a function $f(x)$.

\begin{algorithm}[h]%\scriptsize
   \SetKwData{Left}{left}\SetKwData{This}{this}\SetKwData{Up}{up}
   \SetKwFunction{Union}{Union}\SetKwFunction{FindCompress}{FindCompress}
   \SetKwInOut{Input}{Input}\SetKwInOut{Output}{Output}\SetKwInOut{Calculate}{Calculate}\SetKwInOut{Set}{Set}\SetKwInOut{Adjust}{Adjust}\SetKwInOut{Assumption}{Assumption}

      \Input{Given the function $f(x)$, the initial estimate $x_{0}$, the error tolerance $\epsilon$ and the maximum number of iterations $N$:}
      \Output{An approximation to the root $x=\psi$}

      \Assumption{$x=\psi$ is a root of $f(x)$}

      \For{$k \leftarrow 0$ \KwTo $N$}{
             \Calculate{ $f\left(x_{k}\right)$ and $f'\left(x_{k}\right)$ }

             \Calculate{ $x_{k+1} = x_{k} - \frc{f\left(x_{k}\right)}{f'\left(x_{k}\right)}$ }

             \eIf{ $k == N$}{
                             {\it Calculation has \underline{not converged}. Modify the initial estimate $x_{0}$.} 
                   }{
                     \If{ $\left|f\left(x_{k}\right)\right| \leq \epsilon$ {\bf or} $\frc{\left|x_{k+1}-x_{k}\right|}{\left|x_{k}\right|} \leq \epsilon$ }{
                          {\it Stopping criteria} achieved. The root of function $f(x)$ is $\psi = x_{k+1}$
                          } }
          }
 \caption{Newton-Raphson method algorithm.}\label{Algorithm:NewtonRaphson}
\end{algorithm}

%%% SUBSECTION
\subsection{Secant Iterative Method}\label{Section:RootFinderMethods:Secant}\index{Root-Finder Methods!Secant method}
The Secant method is essentially the same as Newton-Raphson, however the derivative $f'(x)$ is approximated by a finite difference based on the current and the previous estimate for the root,
   \begin{displaymath}
       f'\left(x_{n}\right) \approx \frc{f\left(x_{n}\right) - f\left(x_{n-1}\right)}{x_{n}-x_{n-1}}
   \end{displaymath}

   \begin{shaded}
      Replacing the derivative in Eqn.~\ref{NewtonRaphson:Eqn1} for the $(n+1)^{\text{th}}$-{\it iteration},
         \begin{equation}
            x_{n+1} = x_{n} - \frc{ x_{n} - x_{n-1} }{f\left(x_{n}\right) - f\left(x_{n-1}\right)} f\left(x_{n}\right).\label{Secant:Eqn1}
         \end{equation}
   \end{shaded}
The main problem of the Secant iterative method is that it requires two initial estimates $x_{1}$ and $x_{0}$ for the calculations. These estimates must bound the solution, \ie, $x_{0} \leq \psi \leq x_{1}$. Algorithm~\ref{Algorithm:Secant} shows the steps for its implementation.


\begin{algorithm}[h]%\scriptsize
   \SetKwData{Left}{left}\SetKwData{This}{this}\SetKwData{Up}{up}
   \SetKwFunction{Union}{Union}\SetKwFunction{FindCompress}{FindCompress}
   \SetKwInOut{Input}{Input}\SetKwInOut{Output}{Output}\SetKwInOut{Calculate}{Calculate}\SetKwInOut{Set}{Set}\SetKwInOut{Adjust}{Adjust}\SetKwInOut{Assumption}{Assumption}

      \Input{Given the function $f(x)$, the initial estimates $x_{0}$ and $x_{1}$, the error tolerance $\epsilon$ and the maximum number of iterations $N$:}
      \Output{An approximation to the root $x=\psi$}

      \Assumption{$x=\psi$ is a root of $f(x)$}

      \For{$k \leftarrow 1$ \KwTo $N$}{
             \Calculate{ $f\left(x_{k}\right)$ and $f\left(x_{k-1}\right)$ }

             \Calculate{ $x_{k+1} = x_{k} - \frc{f\left(x_{k}\right)\left(x_{k}-x_{k-1}\right)}{f\left(x_{k}\right) - f\left(x_{k-1}\right)}$ }

             \eIf{ $k == N$}{
                             {\it Calculation has \underline{not converged}. Modify the initial estimate $x_{0}$.} 
                   }{
                     \If{ $\left|f\left(x_{k}\right)\right| \leq \epsilon$ {\bf or} $\frc{\left|x_{k+1}-x_{k}\right|}{\left|x_{k}\right|} \leq \epsilon$ }{
                          {\it Stopping criteria} achieved. The root of function $f(x)$ is $\psi = x_{k+1}$
                          } }
          }
 \caption{Secant method algorithm.}\label{Algorithm:Secant}
\end{algorithm}

   % Example
   \begin{MyExample}{\begin{center}{\bf Example}\end{center}}
     \begin{example}\label{Section:RootFinderMethods:Example:Roots:Secant} 
        Calculate the root of the function $f(x) = x^{2}-2$ using the Secant iterative method with initial estimates of $x_{0}=1.5$ and $x_{1}=1.0$. The error tolerance is $\epsilon=10^{-5}$.
     \end{example}

% SOLUTION
       \noindent{\bf Solution:} The Secant method is expressed through Eqn.~\ref{Secant:Eqn1} for the $(k+1)^{\text{th}}$-iteration,
          \begin{displaymath}
            x_{k+1} = x_{k} - \frc{ x_{k} - x_{k-1} }{f\left(x_{k}\right) - f\left(x_{k-1}\right)} f\left(x_{k}\right).
         \end{displaymath}
         \begin{list}{{\bf Iteration \arabic{mcounter}} (k=\arabic{mcounter}):~}{\usecounter{mcounter}}
            \item Calculating $x_{2}$ from $x_{0}$ and $x_{1}$:
                  \begin{eqnarray}
                      x_{2} &=& x_{1} - \frc{ x_{1} - x_{0} }{f\left(x_{1}\right) - f\left(x_{0}\right)} f\left(x_{1}\right) \nonumber \\
                           &=& 1 - \frc{1 - 1.5}{-1-0.25}  (-1) = 1.4 \nonumber 
                  \end{eqnarray}
                  Stoppage criteria:
                    \begin{enumerate}[(a)]
                         \item $\left|f\left(x_{2}\right)\right| = 0.04 \leq \epsilon \hspace{2cm} \Longrightarrow$ \underline{False}
                         \item $\frc{\left|x_{2}-x_{1}\right|}{\left|x_{1}\right|} = 0.0667 \leq \epsilon \hspace{1.4cm} \Longrightarrow$ \underline{False}
                    \end{enumerate}
            \item Calculating $x_{3}$ from $x_{1}$ and $x_{2}$:
                  \begin{eqnarray}
                      x_{3} &=& x_{2} - \frc{ x_{2} - x_{1} }{f\left(x_{2}\right) - f\left(x_{1}\right)} f\left(x_{2}\right) \nonumber \\
                           &=& 1.4 - \frc{1.4 - 1.0}{-0.04-(-1)}  (-0.04) = 1.4167\nonumber
                  \end{eqnarray}
                  Stoppage criteria:
                    \begin{enumerate}[(a)]
                         \item $\left|f\left(x_{3}\right)\right| = 0.0070 \leq \epsilon \hspace{2cm} \Longrightarrow$ \underline{False}
                         \item $\frc{\left|x_{3}-x_{2}\right|}{\left|x_{2}\right|} = 0.0193 \leq \epsilon \hspace{1.65cm} \Longrightarrow$ \underline{False}
                    \end{enumerate}
            \item Calculating $x_{4}$ from $x_{2}$ and $x_{3}$:
                  \begin{eqnarray}
                      x_{4} &=& x_{3} - \frc{ x_{3} - x_{2} }{f\left(x_{3}\right) - f\left(x_{2}\right)} f\left(x_{3}\right) \nonumber \\
                           &=& 1.4167 - \frc{1.4167 - 1.4}{0.0070-(-0.04)}  (0.0070) = 1.4142\nonumber
                  \end{eqnarray}
                  Stoppage criteria:
                    \begin{enumerate}[(a)]
                         \item $\left|f\left(x_{4}\right)\right| = 3.84\times 10^{-5} \leq \epsilon \hspace{2cm} \Longrightarrow$ \underline{False}
                         \item $\frc{\left|x_{4}-x_{3}\right|}{\left|x_{3}\right|} = 1.76\times 10^{-3} \leq \epsilon \hspace{1.65cm} \Longrightarrow$ \underline{False}
                    \end{enumerate}
            \item Calculating $x_{5}$ from $x_{3}$ and $x_{4}$:
                  \begin{eqnarray}
                      x_{5} &=& x_{4} - \frc{ x_{4} - x_{3} }{f\left(x_{4}\right) - f\left(x_{3}\right)} f\left(x_{4}\right) \nonumber \\
                           &=& 1.4142 - \frc{1.4142 - 1.4167}{-3.84\times 10^{-5}-(0.0070)}  (-3.84\times 10^{-5}) = 1.4142\nonumber
                  \end{eqnarray}
                  Stoppage criteria:
                    \begin{enumerate}[(a)]
                         \item $\left|f\left(x_{5}\right)\right| = 3.84\times 10^{-5} \leq \epsilon \hspace{2cm} \Longrightarrow$ \underline{False}
                         \item $\frc{\left|x_{5}-x_{4}\right|}{\left|x_{4}\right|} = 0.0 \leq \epsilon \hspace{3cm} \Longrightarrow$ \red{\underline{True}}
                    \end{enumerate}
         \end{list}
         Thus, \underline{4 iterations} were necessary to calculate the root of the function $f(x)=x^{2}-2$. The root of the function is \underline{$x=\psi=1.4142$}.
   \end{MyExample}
         

   % Example
   \begin{MyExample}{\begin{center}{\bf Example}\end{center}}
     \begin{example}\label{Section:RootFinderMethods:Example:Roots:NewtonRaphson} 
         Calculate the root of the same function of the previous example using the Newton-Raphson method.
     \end{example}

% SOLUTION
       \noindent{\bf Solution:} Now that we know the solution of the function, let's take the initial estimate as $x_{1}=1.5$. Newton-Raphson method is expressed through Eqn.~\ref{NewtonRaphson:Eqn1} for the $(k+1)^{\text{th}}$-iteration,
          \begin{displaymath}
            x_{k+1} = x_{k} - \frc{f\left(x_{k}\right)}{f'\left(x_{k}\right)}, \text{ where } f'\left(x_{k}\right) = 2x_{k}.
         \end{displaymath}
         \begin{list}{{\bf Iteration \arabic{qcounter}} (k=\arabic{qcounter}):~}{\usecounter{qcounter}}
            \item Calculating $x_{2}$ from $x_{1}$:
                  \begin{eqnarray}
                      x_{2} &=& x_{1} - \frc{f\left(x_{1}\right)}{f'\left(x_{1}\right)} = x_{1} - \frc{ x_{1}^{2}-2 }{ 2x_{1}}   \nonumber \\
                           &=& 1.5 - \frc{0.25}{3} = 1.4167\nonumber
                  \end{eqnarray}
                  Stoppage criteria:
                    \begin{enumerate}[(a)]
                         \item $\left|f\left(x_{2}\right)\right| = 0.0070 \leq \epsilon \hspace{2cm} \Longrightarrow$ \underline{False}
                         \item $\frc{\left|x_{2}-x_{1}\right|}{\left|x_{1}\right|} = 0.0555 \leq \epsilon \hspace{1.4cm} \Longrightarrow$ \underline{False}
                    \end{enumerate}
            \item Calculating $x_{3}$ from $x_{2}$:
                  \begin{eqnarray}
                      x_{3} &=& x_{2} - \frc{f\left(x_{2}\right)}{f'\left(x_{2}\right)} = x_{2} - \frc{ x_{2}^{2}-2 }{ 2x_{2}}  \nonumber \\
                           &=& 1.4167 - \frc{0.0070}{2.8334} = 1.4142\nonumber
                  \end{eqnarray}
                  Stoppage criteria:
                    \begin{enumerate}[(a)]
                         \item $\left|f\left(x_{3}\right)\right| = 3.84\times 10^{-5} \leq \epsilon \hspace{2cm} \Longrightarrow$ \underline{False}
                         \item $\frc{\left|x_{3}-x_{2}\right|}{\left|x_{2}\right|} = 1.77\times 10^{-3} \leq \epsilon \hspace{1.65cm} \Longrightarrow$ \underline{False}
                    \end{enumerate}
            \item Calculating $x_{4}$ from $x_{3}$:
                  \begin{eqnarray}
                      x_{4} &=& x_{3} - \frc{f\left(x_{3}\right)}{f'\left(x_{3}\right)} = x_{3} - \frc{ x_{3}^{2}-2 }{ 2x_{3}} \nonumber \\
                           &=& 1.4142 - \frc{-3.84\times 10^{-5}}{2.8284} = 1.4142\nonumber
                  \end{eqnarray}
                  Stoppage criteria:
                    \begin{enumerate}[(a)]
                         \item $\left|f\left(x_{4}\right)\right| = 3.84\times 10^{-5} \leq \epsilon \hspace{2cm} \Longrightarrow$ \underline{False}
                         \item $\frc{\left|x_{4}-x_{3}\right|}{\left|x_{3}\right|} = 0 \leq \epsilon \hspace{3.3cm} \Longrightarrow$ \red{\underline{True}}
                    \end{enumerate}
         \end{list}
         Thus, we need \underline{3 iterations} to calculate the root, \underline{$x=\psi=1.4142$}, of the function. Now try to use $x_{1}=1.0$ as a first estimate and check how many iterations will be necessary to convergence.

    You may have noticed that the function $f(x)=x^{2}-2$ has two real roots, $\sqrt{2}$ and $-\sqrt{2}$, and in the examples we only obtained the positive root. In order to find the negative root, we need to use initial estimates close enough to the solution, \eg $x_{0}=-1.5$·
   \end{MyExample}

%       
\chapter{A Few Examples}

  \begin{list}{\bf Example \arabic{qcounter}:~}{\usecounter{qcounter}}
%
     %%% EXAMPLE 1:
     \item\label{example1} Using the cyclic rule (Appendix~\ref{Appendix_Calculus:Properties}) and the definitions,
    \begin{displaymath}
        \alpha = \frc{1}{V}\left(\frc{\partial V}{\partial T}\right)_{P} \hspace{1cm}\text{ and }\hspace{1cm} \beta = -\frc{1}{V}\left(\frc{\partial V}{\partial P}\right)_{T},
    \end{displaymath}
    \noindent show that 
    \begin{displaymath}
      \left(\frc{\partial P}{\partial T}\right)_{V} = \frc{\alpha}{\beta}.
    \end{displaymath}
%%
\medskip
     {\bf Solution:} From the cyclic rule,
       \begin{displaymath}
          \left(\frc{\partial P}{\partial T}\right)_{V}\left(\frc{\partial V}{\partial P}\right)_{T}\left(\frc{\partial T}{\partial V}\right)_{T} = -1.
       \end{displaymath}
     Thus,
       \begin{displaymath}
          \left(\frc{\partial P}{\partial T}\right)_{V} = \frc{-1}{\left(\frc{\partial V}{\partial P}\right)_{T}\left(\frc{\partial T}{\partial V}\right)_{T}} = \frc{-\left(\frc{\partial V}{\partial T}\right)_{P}}{\left(\frc{\partial V}{\partial P}\right)_{T}} = \frc{-V\alpha}{-V\beta} = \frc{\alpha}{\beta}
       \end{displaymath}
      
%
     %%% EXAMPLE 2:
     \item\label{example2} For a van der Waals gas, the pressure $P$ and the internal energy $U$ can be expressed as functions of the number of mols ($n$), total volume ($V$) and temperature ($T$),
       \begin{displaymath}
         P = \frc{n R T}{V-nb} - \frc{n^{2}a}{V^{2}} \hspace{1cm}\text{ and }\hspace{1cm} U = \frc{3}{2}n R T - \frc{n^{2}a}{V},
       \end{displaymath}
       respectively, where $a$ and $b$ are constants. Use these equations and the chain rule to derive an equation for $\left(\frc{\partial U}{\partial P}\right)_{n,T}$ in terms of $n$, $V$ and $T$.

%%
\medskip
{\bf Solution:}
   \begin{eqnarray}
      \Partial[U]{P}{n,T} &=& \Partial[U]{V}{n,T}\Partial[V]{P}{n,T} = \frc{\Partial[U]{V}{n,T}}{\Partial[P]{V}{n,T}} \nonumber \\
                          &=& \frc{\frc{n^{2}a}{V^{2}}}{\frc{2n^{2}a}{V^{3}}-\frc{n R T}{\left(V-nb\right)^{2}}} = \frc{n a}{\frc{2 n a}{V}-\frc{R T V^{2}}{\left(V-nb\right)^{2}}}\nonumber
   \end{eqnarray}
      
%
     %%% EXAMPLE 3:
     \item\label{example3} The heat capacity at constant volume is defined as $C_{v}\equiv \Partial[U]{T}{V}$. Show that
       \begin{displaymath}
          \Partial[U]{T}{P} = C_{v} + \alpha V\Partial[U]{V}{T},
       \end{displaymath}
       with $\alpha=\frc{1}{V}\Partial[V]{T}{P}$.

%
\medskip
       {\bf Solution:}
          \begin{displaymath}
            \Partial[U]{T}{P} = \Partial[U]{T}{V} + \Partial[U]{V}{T}\Partial[V]{T}{P},
          \end{displaymath}
          however $\Partial[U]{T}{V}=C_{v}$ and $\Partial[V]{T}{P}=V\alpha$. Thus,
          \begin{displaymath}
             \Partial[U]{T}{P} = C_{v} + \alpha V\Partial[U]{V}{T}.
          \end{displaymath}

%
     %%% EXAMPLE 4:
     \item\label{example4} h
%
\end{list}

  \end{appendix}

\cleardoublepage

\pagebreak

\bibliographystyle{plainnat}
\bibliography{refbib}
%\bibliographystyle{unsrt}

\cleardoublepage
\phantomsection
\renewcommand\leftmark{}
\renewcommand\rightmark{Index}
\addcontentsline{toc}{chapter}{Index}

\printindex


\end{document}
