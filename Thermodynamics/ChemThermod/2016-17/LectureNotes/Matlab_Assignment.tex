
%\documentclass[11pts,a4paper,amsmath,amssymb,floatfix]{article}%{report}%{book}
\documentclass[12pts,a4paper,amsmath,amssymb,floatfix]{article}%{report}%{book}
\usepackage{graphicx,wrapfig,pdfpages}% Include figure files
%\usepackage{dcolumn,enumerate}% Align table columns on decimal point
\usepackage{enumerate,enumitem}% Align table columns on decimal point
\usepackage{bm,dpfloat}% bold math
\usepackage[pdftex,bookmarks,colorlinks=true,urlcolor=rltblue,citecolor=blue]{hyperref}
\usepackage{amsfonts,amsmath,amssymb,stmaryrd,indentfirst}
\usepackage{times,psfrag}
\usepackage{natbib}
\usepackage{color}
\usepackage{units}
\usepackage{rotating}
\usepackage{multirow}


\usepackage{pifont}
\usepackage{subfigure}
\usepackage{subeqnarray}
\usepackage{ifthen}

\usepackage{supertabular}
\usepackage{moreverb}
\usepackage{listings}
\usepackage{palatino}
%\usepackage{doi}
\usepackage{longtable}
\usepackage{float}
\usepackage{perpage}
\MakeSorted{figure}
%\usepackage{pdflscape}


%\usepackage{booktabs}
%\newcommand{\ra}[1]{\renewcommand{\arraystretch}{#1}}


\definecolor{rltblue}{rgb}{0,0,0.75}


%\usepackage{natbib}
\usepackage{fancyhdr} %%%%
\pagestyle{fancy}%%%%
% with this we ensure that the chapter and section
% headings are in lowercase
%%%%\renewcommand{\chaptermark}[1]{\markboth{#1}{}}
\renewcommand{\sectionmark}[1]{\markright{\thesection\ #1}}
\fancyhf{} %delete the current section for header and footer
\fancyhead[LE,RO]{\bfseries\thepage}
\fancyhead[LO]{\bfseries\rightmark}
\fancyhead[RE]{\bfseries\leftmark}
\renewcommand{\headrulewidth}{0.5pt}
% make space for the rule
\fancypagestyle{plain}{%
\fancyhead{} %get rid of the headers on plain pages
\renewcommand{\headrulewidth}{0pt} % and the line
}

\def\newblock{\hskip .11em plus .33em minus .07em}
\usepackage{color}

%\usepackage{makeidx}
%\makeindex

\setlength\textwidth      {16.cm}
\setlength\textheight     {22.6cm}
\setlength\oddsidemargin  {-0.3cm}
\setlength\evensidemargin {0.3cm}

\setlength\headheight{14.49998pt} 
\setlength\topmargin{0.0cm}
\setlength\headsep{1.cm}
\setlength\footskip{1.cm}
\setlength\parskip{0pt}
\setlength\parindent{0pt}


%%%
%%% Headers and Footers
\lhead[] {\text{\small{EX3029 -- Chemical Thermodynamics}}} 
\rhead[\text{\small{Matlab Assignment}}]{Matlab Assignment}
%\rfoot[] {{\text{\small{EOS + Mass Conservation using Matlab }}}}
%\chead[] {\text{\small{Session 2012/13}}} 
\lfoot[]{Dr Jeff Gomes}
\rfoot[\thepage]{\thepage}
\renewcommand{\headrulewidth}{0.8pt}


%%%
%%% space between lines
%%%
\renewcommand{\baselinestretch}{1.5}

\newenvironment{VarDescription}[1]%
  {\begin{list}{}{\renewcommand{\makelabel}[1]{\textbf{##1:}\hfil}%
    \settowidth{\labelwidth}{\textbf{#1:}}%
    \setlength{\leftmargin}{\labelwidth}\addtolength{\leftmargin}{\labelsep}}}%
  {\end{list}}

%%%%%%%%%%%%%%%%%%%%%%%%%%%%%%%%%%%%%%%%%%%
%%%%%%                              %%%%%%%
%%%%%%      NOTATION SECTION        %%%%%%%
%%%%%%                              %%%%%%%
%%%%%%%%%%%%%%%%%%%%%%%%%%%%%%%%%%%%%%%%%%%

% Text abbreviations.
\newcommand{\ie}{{\em{i.e., }}}
\newcommand{\eg}{{\em{e.g., }}}
\newcommand{\cf}{{\em{cf., }}}
\newcommand{\wrt}{with respect to}
\newcommand{\lhs}{left hand side}
\newcommand{\rhs}{right hand side}
% Commands definining mathematical notation.

% This is for quantities which are physically vectors.
\renewcommand{\vec}[1]{{\mbox{\boldmath$#1$}}}
% Physical rank 2 tensors
\newcommand{\tensor}[1]{\overline{\overline{#1}}}
% This is for vectors formed of the value of a quantity at each node.
\newcommand{\dvec}[1]{\underline{#1}}
% This is for matrices in the discrete system.
\newcommand{\mat}[1]{\mathrm{#1}}


\DeclareMathOperator{\sgn}{sgn}
\newtheorem{thm}{Theorem}[section]
\newtheorem{lemma}[thm]{Lemma}

%\newcommand\qed{\hfill\mbox{$\Box$}}
\newcommand{\re}{{\mathrm{I}\hspace{-0.2em}\mathrm{R}}}
\newcommand{\inner}[2]{\langle#1,#2\rangle}
\renewcommand\leq{\leqslant}
\renewcommand\geq{\geqslant}
\renewcommand\le{\leqslant}
\renewcommand\ge{\geqslant}
\renewcommand\epsilon{\varepsilon}
\newcommand\eps{\varepsilon}
\renewcommand\phi{\varphi}
\newcommand{\bmF}{\vec{F}}
\newcommand{\bmphi}{\vec{\phi}}
\newcommand{\bmn}{\vec{n}}
\newcommand{\bmns}{{\textrm{\scriptsize{\boldmath $n$}}}}
\newcommand{\bmi}{\vec{i}}
\newcommand{\bmj}{\vec{j}}
\newcommand{\bmk}{\vec{k}}
\newcommand{\bmx}{\vec{x}}
\newcommand{\bmu}{\vec{u}}
\newcommand{\bmv}{\vec{v}}
\newcommand{\bmr}{\vec{r}}
\newcommand{\bma}{\vec{a}}
\newcommand{\bmg}{\vec{g}}
\newcommand{\bmU}{\vec{U}}
\newcommand{\bmI}{\vec{I}}
\newcommand{\bmq}{\vec{q}}
\newcommand{\bmT}{\vec{T}}
\newcommand{\bmM}{\vec{M}}
\newcommand{\bmtau}{\vec{\tau}}
\newcommand{\bmOmega}{\vec{\Omega}}
\newcommand{\pp}{\partial}
\newcommand{\kaptens}{\tensor{\kappa}}
\newcommand{\tautens}{\tensor{\tau}}
\newcommand{\sigtens}{\tensor{\sigma}}
\newcommand{\etens}{\tensor{\dot\epsilon}}
\newcommand{\ktens}{\tensor{k}}
\newcommand{\half}{{\textstyle \frac{1}{2}}}
\newcommand{\tote}{E}
\newcommand{\inte}{e}
\newcommand{\strt}{\dot\epsilon}
\newcommand{\modu}{|\bmu|}
% Derivatives
\renewcommand{\d}{\mathrm{d}}
\newcommand{\D}{\mathrm{D}}
\newcommand{\ddx}[2][x]{\frac{\d#2}{\d#1}}
\newcommand{\ddxx}[2][x]{\frac{\d^2#2}{\d#1^2}}
\newcommand{\ddt}[2][t]{\frac{\d#2}{\d#1}}
\newcommand{\ddtt}[2][t]{\frac{\d^2#2}{\d#1^2}}
\newcommand{\ppx}[2][x]{\frac{\partial#2}{\partial#1}}
\newcommand{\ppxx}[2][x]{\frac{\partial^2#2}{\partial#1^2}}
\newcommand{\ppt}[2][t]{\frac{\partial#2}{\partial#1}}
\newcommand{\pptt}[2][t]{\frac{\partial^2#2}{\partial#1^2}}
\newcommand{\DDx}[2][x]{\frac{\D#2}{\D#1}}
\newcommand{\DDxx}[2][x]{\frac{\D^2#2}{\D#1^2}}
\newcommand{\DDt}[2][t]{\frac{\D#2}{\D#1}}
\newcommand{\DDtt}[2][t]{\frac{\D^2#2}{\D#1^2}}
% Norms
\newcommand{\Ltwo}{\ensuremath{L_2} }
% Basis functions
\newcommand{\Qone}{\ensuremath{Q_1} }
\newcommand{\Qtwo}{\ensuremath{Q_2} }
\newcommand{\Qthree}{\ensuremath{Q_3} }
\newcommand{\QN}{\ensuremath{Q_N} }
\newcommand{\Pzero}{\ensuremath{P_0} }
\newcommand{\Pone}{\ensuremath{P_1} }
\newcommand{\Ptwo}{\ensuremath{P_2} }
\newcommand{\Pthree}{\ensuremath{P_3} }
\newcommand{\PN}{\ensuremath{P_N} }
\newcommand{\Poo}{\ensuremath{P_1P_1} }
\newcommand{\PoDGPt}{\ensuremath{P_{-1}P_2} }

\newcommand{\metric}{\tensor{M}}
\newcommand{\configureflag}[1]{\texttt{#1}}

% Units
\newcommand{\m}[1][]{\unit[#1]{m}}
\newcommand{\km}[1][]{\unit[#1]{km}}
\newcommand{\s}[1][]{\unit[#1]{s}}
\newcommand{\invs}[1][]{\unit[#1]{s}\ensuremath{^{-1}}}
\newcommand{\ms}[1][]{\unit[#1]{m\ensuremath{\,}s\ensuremath{^{-1}}}}
\newcommand{\mss}[1][]{\unit[#1]{m\ensuremath{\,}s\ensuremath{^{-2}}}}
\newcommand{\K}[1][]{\unit[#1]{K}}
\newcommand{\PSU}[1][]{\unit[#1]{PSU}}
\newcommand{\Pa}[1][]{\unit[#1]{Pa}}
\newcommand{\kg}[1][]{\unit[#1]{kg}}
\newcommand{\rads}[1][]{\unit[#1]{rad\ensuremath{\,}s\ensuremath{^{-1}}}}
\newcommand{\kgmm}[1][]{\unit[#1]{kg\ensuremath{\,}m\ensuremath{^{-2}}}}
\newcommand{\kgmmm}[1][]{\unit[#1]{kg\ensuremath{\,}m\ensuremath{^{-3}}}}
\newcommand{\Nmm}[1][]{\unit[#1]{N\ensuremath{\,}m\ensuremath{^{-2}}}}

% Dimensionless numbers
\newcommand{\dimensionless}[1]{\mathrm{#1}}
\renewcommand{\Re}{\dimensionless{Re}}
\newcommand{\Ro}{\dimensionless{Ro}}
\newcommand{\Fr}{\dimensionless{Fr}}
\newcommand{\Bu}{\dimensionless{Bu}}
\newcommand{\Ri}{\dimensionless{Ri}}
\renewcommand{\Pr}{\dimensionless{Pr}}
\newcommand{\Pe}{\dimensionless{Pe}}
\newcommand{\Ek}{\dimensionless{Ek}}
\newcommand{\Gr}{\dimensionless{Gr}}
\newcommand{\Ra}{\dimensionless{Ra}}
\newcommand{\Sh}{\dimensionless{Sh}}
\newcommand{\Sc}{\dimensionless{Sc}}


% Journals
\newcommand{\IJHMT}{{\it International Journal of Heat and Mass Transfer}}
\newcommand{\NED}{{\it Nuclear Engineering and Design}}
\newcommand{\ICHMT}{{\it International Communications in Heat and Mass Transfer}}
\newcommand{\NET}{{\it Nuclear Engineering and Technology}}
\newcommand{\HT}{{\it Heat Transfer}}   
\newcommand{\IJHT}{{\it International Journal for Heat Transfer}}

\newcommand{\frc}{\displaystyle\frac}

\newlist{ExList}{enumerate}{1}
\setlist[ExList,1]{label={\bf Example 1.} {\bf \arabic*}}

\newlist{ProbList}{enumerate}{1}
\setlist[ProbList,1]{label={\bf Problem 1.} {\bf \arabic*}}

%%%%%%%%%%%%%%%%%%%%%%%%%%%%%%%%%%%%%%%%%%%
%%%%%%                              %%%%%%%
%%%%%% END OF THE NOTATION SECTION  %%%%%%%
%%%%%%                              %%%%%%%
%%%%%%%%%%%%%%%%%%%%%%%%%%%%%%%%%%%%%%%%%%%


% Cause numbering of subsubsections. 
%\setcounter{secnumdepth}{8}
%\setcounter{tocdepth}{8}

\setcounter{secnumdepth}{4}%
\setcounter{tocdepth}{4}%


\begin{document}

\begin{enumerate}[label=\bfseries Problem \arabic*:]
%
     \item\label{Prob1} In order to investigate the PVT behaviour of gasses, 10$^{-3}$ m$^{3}$.mol$^{-1}$ of methanol is injected into a reactor cell and kept constant at 276.85$^{\circ}$C. Several properties of the gas were measured at reduced pressures $\left(P_{r}\right)$ r`anging from 10$^{-5}$ to 1.  Similar measurements were obtained for benzene, toluene and carbon tetrachloride at the same conditions. Write a \underline{Matlab code} to calculate compressibility factor ($Z$) as a function of $P_{r}$ for both Soave-Redlich-Kwong (SRK-EOS) and Peng-Robinson (PR-EOS) cubic equations of state and solve the following tasks:
          \begin{enumerate}[label=\bfseries Task \arabic*]
              \item\label{a} Plot $P_{r}\times Z$ for all 4 gasses with both EOS; \hfill{\bf[32 Marks]} 
              \item\label{b} Using the data generated in \ref{a} for PR-EOS, calculate pressures that will lead to $Z=0.750\pm0.001$ for all gasses. \hfill{\bf[6 Marks]} 
          \end{enumerate}
For your calculations, $10^{-5}\leq P_{r}\leq 1$ should be obtained with intervals of 10$^{-3}$, also you should use the thermofluid properties from Table~\ref{Practical1:Table1}. 

\begin{table}[h]
\begin{center}
\begin{tabular}{||c | c c c c c ||} 
\hline\hline
                          & {\bf Molar Mass}           &  {\bf $\omega$}  & {\bf T$_{c}$}  & {\bf P$_{c}$} & {\bf Z$_{c}$}  \\
                          & $\left(\text{g.mol}^{-1}\right)$ &                  &   (K)         &   (bar)       &               \\ 
\hline
{\bf Benzene}             & 78.0                             &  0.212           &  563.0         &  49.2        &    0.271       \\  
{\bf Methanol}            & 32.0                             &  0.559           &  513.0         &  80.8        &    0.224       \\  
{\bf Toluene}             & 92.0                             &  0.266           &  592.0         &  41.3        &    0.284       \\  
{\bf Carbon tetrachloride}& 154.0                            &  0.194           &  556.0         &  45.6        &    0.272       \\  
\hline\hline
\end{tabular}
\caption{Thermofluid properties of the fluids.}
\label{Practical1:Table1}
\end{center}
\end{table}
%
     \item Write a \underline{Matlab code} to calculate the molar volume $\left(\text{in m}^{3}\text{.mol}^{-1}\right)$ and compressibility factor for gaseous ammonia at 450K using the van der Waals (vdW) equation of state for the following conditions:
          \begin{enumerate}[label=\bfseries Task \arabic*]
              \item\label{c} Pressure of 56 atm; \hfill{\bf[8 Marks]}
              \item\label{d} Repeat the calculations for reduced pressure $\left(P_{r}\right)$ of 0.1, 0.2, 0.5, 0.8, 1.0, 2.0 and 5.0 and plot $P_{r}\times Z$. \hfill{\bf[12 Marks]}
          \end{enumerate}
          Critical temperature and pressure of ammonia are 405.5 K and 111.3 atm, respectively.

     \item After distillation, a mixture of light hydrocarbons are stored in two pressure vessels:
         \begin{itemize}
            \item {\bf Vessel 1:} 45 mol-$\%$ of n-hexane, 30 mol-$\%$ of n-heptane and 25 mol-$\%$ of i-octane;
            \item {\bf Vessel 2:} 1.70 mol-$\%$ of n-hexane, 3.75 mol-$\%$ of n-heptane, 5.76 mol-$\%$ of i-octane, 27.45 mol-$\%$ of o-xylene, 37.71 mol-$\%$ of p-xylene and 23.63 mol-$\%$ of chlorobenzene 
         \end{itemize} 
         Vessels 1 and 2 are kept at 1.5 and 5 bar, respectively.
          \begin{enumerate}[label=\bfseries Task \arabic*]
              \item\label{e} {\bf[Hand-calculation]} Estimate the bubble and dew point temperatures for Vessel 1. Also calculate the compositions at bubble and dew points; \hfill{\bf[16 Marks]}
              \item\label{f} Write a \underline{Matlab code} to calculate bubble and dew point temperatures and compositions at Vessel 2.\hfill{\bf[26 Marks]}
          \end{enumerate}

          Bubble and dew point are obtained through the following relations:
          \begin{displaymath}
             \sum\limits_{i=1}^{n} y_{i} = \sum\limits_{i=1}^{n} \frc{x_{i}P_{i}^{\text{sat}}}{P} = 1 \hspace{1cm}\text{ and } \hspace{1cm} \sum\limits_{i=1}^{n} x_{i} = \sum\limits_{i=1}^{n} \frc{y_{i}P}{P_{i}^{\text{sat}}} = 1,
          \end{displaymath}
          where $x_{i}$ and $y_{i}$ are molar fractions of component $i$ at liquid and vapour phases, respectively. $n$ and $P$ are the total number of components in the mixture and the system pressure. Finally, $P^{\text{sat}}$ is the vapour pressure that can be obtained from the Antoine equation,
          \begin{displaymath}
            \ln{P^{\text{sat}}} = A - \frc{B}{T+C}
          \end{displaymath}
          where $\left[P^{\text{sat}}\right]$ = kPa and $[T]$ = $^{\circ}$C, with coefficients given in Table~\ref{Practical1:Table2}.
           

\begin{table}[h]
\begin{center}
\begin{tabular}{||c | c c c ||} 
\hline\hline
                           & {\bf A}    &  {\bf B}    & {\bf C}    \\
\hline
{\bf n-hexane}             & 13.8193    & 2696.04     & 224.317    \\  
{\bf n-heptane}            & 13.8622    & 2910.26     & 216.432    \\  
{\bf i-octane}             & 13.6703    & 2896.31     & 220.767    \\  
{\bf o-xylene}             & 14.0415    & 3381.81     & 216.120    \\  
{\bf p-xylene}             & 14.0579    & 3331.45     & 214.627    \\  
{\bf chlorobenze}           & 13.8635    & 3174.78     & 211.700    \\  
\hline\hline
\end{tabular}
\caption{Constants for the Antoine equation for vapour pressure.}
\label{Practical1:Table2}
\end{center}
\end{table}




\end{enumerate}
\clearpage

{\bf Deliverables:}
\begin{itemize}
\item Write a report containing a brief introduction of EOS and a summary of your results (incl. figures displaying the data), findings along with concluding remarks and \underline{Matlab codes} used in your calculations. 
%
\item {\bf Prepare the report and submit to the UG Office (with the appropriate plagiarism cover sheet) by Monday, October 31$^{st}$ 2016, noon at the latest.}
%
\item Feedback will be provided on November 21$^{st}$ 2016.
%
\item Penalties for late or non-submission are as follows:
\begin{enumerate}%[(a)]
\item Up to one week late, 2 CGS points deducted;
\item Up to two weeks late, 3 CGS point deducted;
\item More than two weeks late no marks awarded.
\end{enumerate}
If late or non-submission is due to medical or other circumstances out with your control you must submit a medical certificate or other formal evidence to the UG Office as soon as is practicable but no later than the end of Revision Week.


\item Note that the submitted work is part of the continuous assessment which will contribute 20$\%$ to your EX3029 mark.

\end{itemize}



\clearpage

\begin{center}
  \Large{\bf Solutions}
\end{center}


\begin{enumerate}[label=\bfseries Problem \arabic*:]
% PROBLEM 1
   \item Use the Matlab code in the {\it Problem 1} directory:
      \begin{enumerate}[label=\bfseries Task \arabic*:]
        \item Each plot is worth \underline{8 marks}. See Matlab code for the solution.
          \begin{figure}[h]
             \vbox{
                   \hbox{ 
                          \includegraphics[width=8.cm,height=6.cm,clip]{./Figs/Benzene.png}
                          \includegraphics[width=8.cm,height=6.cm,clip]{./Figs/Methanol.png}
                         }
                   \hbox{ 
                          \includegraphics[width=8.cm,height=6.cm,clip]{./Figs/Toluene.png}
                          \includegraphics[width=8.cm,height=6.cm,clip]{./Figs/CCl4.png}
                        }
                  } 
             \caption{Problem 1, Task 1. }
             \label{Prob1_Task1}
          \end{figure}
%
        \item Using the Matlab code, one can obtain pressures in the following range (depending on number of points or the method used) with \underline{1.5 marks} each: 
          \begin{itemize}
             \item Methanol:  62.62 - 63.02 bar
             \item Benzene:   25.98 - 26.08 bar
             \item Toluene:   18.30 - 18.38 bar
             \item CCl$_{4}$:  25.13 - 25.26 bar
          \end{itemize}
      \end{enumerate} 
\clearpage
%  PROBLEM 2
   \item Use the Matlab code in the {\it Problem 2} directory to calculate the follow for NH$_{3}$:
      \begin{enumerate}[label=\bfseries Task \arabic*:]
          \item Molar volume is 57.48 m$^{3}$.mol$^{-1}$ and $Z$ is 0.8718. These are worth \underline{8 marks}.
          \item The plot is worth \underline{12 marks}.
            \begin{figure}[h]
               \begin{center}
                   \includegraphics[width=15.cm,height=10.cm,clip]{./Figs/NH3.png}
               \end{center}
               \caption{Problem 2, Task 2. }
               \label{Prob2_Task2}
            \end{figure}

      \end{enumerate} 

\clearpage
%
   \item For Task 2, use the Matlab code in the {\it Problem 3} directory.
      \begin{enumerate}[label=\bfseries Task \arabic*:]
%
         \item Hand calculation:
            \begin{enumerate}
               \item Bubble point:
                    \begin{eqnarray}
                        \sum\limits_{i=1}^{n} y_{i} &=& \sum\limits_{i=1}^{n} \frc{x_{i}P_{i}^{\text{sat}}}{P} = 1 \nonumber \\
                                                 &=& \frc{x_{C6}P_{C6}^{\text{sat}}}{P} + \frc{x_{C7}P_{C7}^{\text{sat}}}{P} + \frc{x_{C8}P_{C8}^{\text{sat}}}{P} = 1 \nonumber \\
                                                 &=& x_{C6}P_{C6} + x_{C7}P_{C7} + x_{C8}P_{C8} = P, \nonumber 
                    \end{eqnarray}
                    with the Antoine relation,
                      \begin{displaymath}
                         \ln{P^{\text{sat}}} = A - \frc{B}{T+C},
                      \end{displaymath}
                     with $P$ 150 kPa. Solving this non-linear equation (using a calculator) leads to the bubble point temperature $\Longrightarrow$ $T_{\text{bubble}}$ = 95.68$^{\circ}$C (\underline{5 marks}).  Vapour phase composition is obtained from:
                      \begin{displaymath}
                         y_{i} = \frc{x_{i}P_{i}^{\text{sat}}}{P}.
                      \end{displaymath}
                      Leading to y = [ 0.6603, 0.1870, 0.1527 ] (\underline{3 marks}).
%
               \item Dew point:
                    \begin{eqnarray}
                        \sum\limits_{i=1}^{n} x_{i} &=& \sum\limits_{i=1}^{n} \frc{y_{i}P}{P_{i}^{\text{sat}}} = 1 \nonumber \\
                                                 &=& \frc{y_{C6}P}{P_{C6}^{\text{sat}}} + \frc{y_{C7}P}{P_{C7}^{\text{sat}}} + \frc{y_{C8}P}{P_{C8}^{\text{sat}}} = 1 \nonumber 
                    \end{eqnarray}
                    Solving this non-linear equation (using a calculator) leads to the dew point temperature $\Longrightarrow$ $T_{\text{dew}}$ = 102.10$^{\circ}$C (\underline{5 marks}).  Liquid phase composition is obtained from:
                      \begin{displaymath}
                         x_{i} = \frc{y_{i}P}{P_{i}^{\text{sat}}}.
                      \end{displaymath}
                      Leading to x = [ 0.2599, 0.3989, 0.3412 ] (\underline{3 marks}).    
            \end{enumerate}
%
          \item From Matlab code:
            \begin{enumerate}
               \item Bubble temperature: 195.95$^{\circ}$C (\underline{7 marks}) with vapour composition of (\underline{1 mark})
                   \begin{center}
                      y = [ 0.0620, 0.0752, 0.1061, 0.2086, 0.3196, 0.2285]
                   \end{center}
                   for n-C$_{6}$, n-C$_{7}$, i-C$_{8}$, o-xylene, p-xylene and chlorobenzene.
               \item Dew temperature: 201.14$^{\circ}$C (\underline{7 marks}) with vapour composition of (\underline{1 mark})
                   \begin{center}
                      y = [ 0.0043, 0.0171, 0.0287, 0.3262, 0.4021, 0.2216].
                   \end{center}
            \end{enumerate}
%
      \end{enumerate} 
%
\end{enumerate}


\end{document}
