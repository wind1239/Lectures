
\chapter{Introduction to Numerical Methods relevant to Thermodynamics}\label{Appendix_NumMethods}

%%%
%%% SECTION
%%%
\section{Linear Interpolation}\label{LinearInterpolation}\index{Linear interpolation}

Given a continuous and unknown function $f(x)$, defined at a set of points  $x_{1} < \cdots < x_{i} < \cdots < x_{N}$. Interpolation is the process of determining a polynomial expression to calculate the pair $\left[x_{k}, f\left(x_{k}\right)\right]$ based on neighbours discrete coordinates $\left\{\left[x_{1},f\left(x_{1}\right)\right], \cdots, \left[x_{N},f\left(x_{N}\right)\right]\right\}$. 

Consider a set of discrete data points,
  \begin{center}
    \begin{tabular}{c | c }
        $\mathbf{x}$   & $\mathbf{f\left(x_{i}\right)}$ \\
        \hline
           $x_{1}$ &  $f\left(x_{1}\right)$ \\
           $x_{2}$ &  $f\left(x_{2}\right)$ \\
           $x_{3}$ &  $f\left(x_{3}\right)$ \\
           $x_{4}$ &  $f\left(x_{4}\right)$ \\
    \end{tabular}
  \end{center}
that are a subset of a continuous and smooth function $y=f(x)$ (Fig.~\ref{Appendix:Fig:Interpolation}). Polynomials of order $n\ge 1$ can be generated to represent this function. High-order polynomials can more accurately fit the discrete coordinatess than low-order polynomials. In Fig.~\ref{Appendix:Fig:Interpolation}, let's assume the discrete pairs 
  \begin{displaymath}
     \left\{\left[x_{1},f\left(x_{1}\right)\right], \left[x_{2},f\left(x_{2}\right)\right],\left[x_{3},f\left(x_{3}\right)\right], \left[x_{4},f\left(x_{4}\right)\right]\right\}
  \end{displaymath}
are known, and one wants to determine the value of the function $f$ at $x_{2} < x_{k} < x_{3}$. If the interval $\Delta x= x_{3}-x_{2}$ is sufficiently small, a linear function can be used to fit these coordinates,
   \begin{equation}
       f\left(x_{k}\right) = f\left(x_{2}\right) + m\left(x_{k}-x_{2}\right),\label{LinearInterpolation:Eqn1}
   \end{equation}
where 
   \begin{displaymath}
      m = \frc{f\left(x_{3}\right)-f\left(x_{2}\right)}{x_{3}-x_{2}}.
   \end{displaymath}
If $m$ is replaced in Eqn.~\ref{LinearInterpolation:Eqn1},

%%% Figure
     \begin{figure}\label{Appendix:Fig:Interpolation}%
        \begin{center}
          \includegraphics[width=\columnwidth,clip]{./Pics/Interpolation}
           \caption{Smooth function $f(x)$ (solid blue line) may be more accurately interpolated by a high-order polynomial (black dotted line) than by a low-order polynomial (solid black line).} 
        \end{center}
      \end{figure}


%%%
%%% SECTION
%%%
\section{Root Finder}
