
\chapter{Introduction to Numerical Methods relevant to Thermodynamics}\label{Appendix_NumMethods}

%%%
%%% SECTION
%%%
\section{Linear Interpolation}\label{LinearInterpolation}\index{Linear interpolation}

Given a continuous and unknown function $f(x)$, defined at a set of points  $x_{1} < \cdots < x_{i} < \cdots < x_{N}$. Interpolation is the process of determining a polynomial expression to calculate the pair $\left[x_{k}, f\left(x_{k}\right)\right]$ based on neighbours discrete coordinates $\left\{\left[x_{1},f\left(x_{1}\right)\right], \cdots, \left[x_{N},f\left(x_{N}\right)\right]\right\}$. 

Consider a set of discrete data points,
  \begin{center}
    \begin{tabular}{c | c }
        $\mathbf{x}$   & $\mathbf{f\left(x_{i}\right)}$ \\
        \hline
           $x_{1}$ &  $f\left(x_{1}\right)$ \\
           $x_{2}$ &  $f\left(x_{2}\right)$ \\
           $x_{3}$ &  $f\left(x_{3}\right)$ \\
           $x_{4}$ &  $f\left(x_{4}\right)$ \\
    \end{tabular}
  \end{center}
that are a subset of a continuous and smooth function $y=f(x)$ (Fig.~\ref{Appendix:Fig:Interpolation}). Polynomials of order $n\ge 1$ can be generated to represent this function. High-order polynomials can more accurately fit the discrete coordinatess than low-order polynomials. In Fig.~\ref{Appendix:Fig:Interpolation}, let's assume the discrete pairs 
  \begin{displaymath}
     \left\{\left[x_{1},f\left(x_{1}\right)\right], \left[x_{2},f\left(x_{2}\right)\right],\left[x_{3},f\left(x_{3}\right)\right], \left[x_{4},f\left(x_{4}\right)\right]\right\}
  \end{displaymath}
are known, and one wants to determine the value of the function $f$ at $x_{2} < x_{k} < x_{3}$. If the interval $\Delta x= x_{3}-x_{2}$ is sufficiently small, a linear function can be used to fit these coordinates,
   \begin{equation}
       f\left(x_{k}\right) = f\left(x_{2}\right) + m\left(x_{k}-x_{2}\right),\label{LinearInterpolation:Eqn1}
   \end{equation}
where 
   \begin{displaymath}
      m = \frc{f\left(x_{3}\right)-f\left(x_{2}\right)}{x_{3}-x_{2}}.
   \end{displaymath}
If $m$ is replaced in Eqn.~\ref{LinearInterpolation:Eqn1},
   \begin{displaymath}
       f\left(x_{k}\right) = \frc{f\left(x_{2}\right)\left(x_{3}-x_{k}\right) + f\left(x_{3}\right)\left(x_{k}-x_{2}\right)}{x_{3}-x_{2}}.
   \end{displaymath}
   
   \begin{shaded}
      Or for a general case with $x_{a} < x_{k} < x_{b}$,
        \begin{equation}\label{LinearInterpolation:Eqn1}
            f\left(x_{k}\right) = \frc{f\left(x_{a}\right)\left(x_{b}-x_{k}\right) + f\left(x_{b}\right)\left(x_{k}-x_{a}\right)}{x_{b}-x_{a}}.
        \end{equation}
   \end{shaded}

%%% Figure
     \begin{figure}[h]\label{Appendix:Fig:Interpolation}%
        \begin{center}
          \includegraphics[width=\columnwidth,clip]{./Pics/Interpolation}
           \caption{Smooth function $f(x)$ (solid blue line) may be more accurately interpolated by a high-order polynomial (black dotted line) than by a low-order polynomial (solid black line).} 
        \end{center}
      \end{figure}

    \begin{list}{\bf Example \arabic{qcounter}:~}{\usecounter{qcounter}}
%       
         \item Given a table of values for $f(x)=\tan{x}$ for a few values of $x$,
            \begin{center}
               \begin{tabular}{c | c c c c}
                   $x$        & 1.00   & 1.10   & 1.20   & 1.30   \\
                   \hline
                   $\tan{x}$  & 1.5574 & 1.9648 & 2.5722 & 3.6021 \\
               \end{tabular}
            \end{center}
            Estimate $\tan{1.15}$ and $\tan{1.23}$.

            {\bf Solution:} For $\left(x_{a}=1.10\right) < \left(x_{k}=1.15\right) < \left(x_{b}=1.20\right)$,
               \begin{eqnarray}
                  f\left(x_{k}\right) &=& \frc{f\left(x_{a}\right)\left(x_{b}-x_{k}\right) + f\left(x_{b}\right)\left(x_{k}-x_{a}\right)}{x_{b}-x_{a}} \nonumber \\
                                     &=& \frc{1.9648\times(1.20-1.15) + 2.5722\times(1.15-1.10)}{1.20-1.10} = 2.2685 \nonumber
               \end{eqnarray}

For  $\left(x_{a}=1.20\right) < \left(x_{k}=1.23\right) < \left(x_{b}=1.30\right)$,
               \begin{eqnarray}
                  f\left(x_{k}\right) &=& \frc{f\left(x_{a}\right)\left(x_{b}-x_{k}\right) + f\left(x_{b}\right)\left(x_{k}-x_{a}\right)}{x_{b}-x_{a}} \nonumber \\
                                     &=& \frc{2.5722\times(1.30-1.23) + 3.6021\times(1.23-1.20)}{1.30-1.20} = 2.8812 \nonumber
               \end{eqnarray}
%
         \item Calculate specific volume $\left(\text{in m}^{3}\text{.kg}^{-1}\right)$ enthalpy $\left(\text{in kJ.kg}^{-1}\right)$ and entropy $\left(\text{in kJ.(kg.K)}^{-1}\right)$ of saturated water vapour at 133.45$^{\circ}$C.

           {\bf Solution:} From Appendix~\ref{Appendix:Saturated_SH_Tables} (Table A-2), for $T_{a}(=130.0) < T_{k} (= 133.45) < T_{b} (=140.0)^{\circ}C$, thus:
               \begin{eqnarray}
                  v\left(T_{k}\right) &=& \frc{v\left(T_{a}\right)\left(T_{b}-T_{k}\right) + v\left(T_{b}\right)\left(T_{k}-T_{a}\right)}{T_{b}-T_{a}} \nonumber \\
                                     &=& \frc{0.6685\times(140.0-133.45) + 0.5089\times(133.45-130.0)}{140.0-130.0} = 2.8812 \nonumber
               \end{eqnarray}


    \end{list}

%%%
%%% SECTION
%%%
\section{Root Finder}
