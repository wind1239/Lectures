
%%%
%%% CHAPTER
%%%
\chapter{Introduction to Properties of Gases}\label{Chapter:Intro_Property_of_Gases}

   \begin{LearningObjectivesBlock}{Learning Objectives}
      Upon completion of this chapter, you will be able to
        \begin{enumerate}
           \item define ideal gas and identify the main assumptions;
           \item differentiate an ideal from a real gas through state conditions;
           \item explain Boyle's, Gay-Lussac's and Charles' laws for ideal gases;
           \item define and explain the equation of state for ideal gases;
           \item state the Dalton's law for gaseous mixtures.
        \end{enumerate}
\medskip
     Recommended reading: Chapter 1 of \citet{Atkins_Book,Adamson_BookChapter}.
   \end{LearningObjectivesBlock}


%%%% ETOC
\localtableofcontents
   
%%%
%%% SECTION
%%%
     \section{Introduction}\label{Chapter:Intro_Property_of_Gases:Section:Intro}

   The physical condition (\ie state) of a substance is defined by its physical properties. For example, the state of a pure fluid is specified by volume ($V$), mass (through the number of moles, $n$), pressure ($P$) and temperature ($T$). Laboratory experiments with fluids revealed that if three of these properties (\eg $V, n, T$) are specified, the state of the fluid is naturally defined, and the forth property (\ie $P$) is fixed. Mathematically, this can be represented by a functional often referred as {\bf equation of state}\index{Equation of State} (EOS)\index{EOS|see {Equation of State}},
     \begin{displaymath}
       P = f(T,V,n),
     \end{displaymath}
     which states that if $T$, $n$ and $V$ are known for a particular fluid, then the pressure of the system for the fluid at this state can be readily determined. The state and properties of any chemical species can be described by an specific equation of state\footnote{Equations of state are the main focus of Chapter~\ref{Chapter:VolumetricPropertiesPureFluids}.}.
   
%%%
%%% SECTION
%%%
     \section{Ideal Gases}\label{Chapter:Intro_Property_of_Gases:Section:IdealGases}\index{Gases!Ideal gas}
     \begin{subequations}

     \noindent A fluid is assumed to behave as an {\it ideal gas} if the following two assumptions are true:
     \begin{enumerate}[i)]
       \item molecules of gases are considered as {\bf massless} particles;
       \item there are {\bf no} interaction between the gas particles due to,
         \begin{enumerate}[a)]
           \item volume of the molecules is negligibly small compared with the volume occupied by the gas, and/or;
           \item the distance between gaseous molecules are infinitely large $\left(d\rightarrow \infty\right)$, except during elastic collisions over negligible duration.
         \end{enumerate}
     \end{enumerate}
     \noindent These lead to an obvious conclusion that for a gas to behave as an ideal gas the pressure should be infinitely small, \ie $P\rightarrow 0$. In practical engineering calculations, we assume that a fluid behaves as an ideal gas at low to moderate pressures. The $PVT$ behaviour of fluids was initially investigated through experimental observations of gases at low pressures that led to three intuitive relationships:
     \begin{itemize}
       \item For isothermal processes in closed systems (\ie $T$ and $n$ are constant), $P$ and $V$ are inversely proportional to each other, \ie an increase in pressure leads to a decrease in volume,
          \begin{displaymath}
             P \propto \frc{1}{V},
          \end{displaymath}          
%          
       \item For isochoric processes in closed systems (\ie $V$ and $n$ are constant), $T$ and $P$ are directly proportional to each other, \ie an increase in temperature leads to an increase in pressure,
          \begin{displaymath}
            T \propto P,
          \end{displaymath}         
%
       \item For isobaric processes in closed systems (\ie $P$ and $n$ are constant), $T$ and $V$ are directly proportional to each other, \ie an increase in temperature leads to an increase in volume.
          \begin{displaymath}
            T\propto V.
          \end{displaymath}          
     \end{itemize}
     \begin{shaded}
        These proportionality relations can be merged into a single expression,\index{Gases!Boyle's law}\index{Gases!Gay-Lussac's law}\index{Gases!Charles' law}\index{Gay-Lussac's law|see {Gases}}\index{Charles' law|see {Gases}}\index{Boyle's law|see {Gases}}
          \begin{equation}
            \frc{P V}{n T} = R \;\;\Longleftrightarrow \;\;
              \begin{cases}
                P_{1}V_{1} = P_{2}V_{2}, & \text{if } T \text{ and } n \text{ are constant (Boyle's law)},  \\
         \\
                P_{1}T_{1}^{-1} = P_{2}T_{2}^{-1}, & \text{if } V \text{ and } n \text{ are constant (Gay-Lussac's law)}, \\
         \\
                V_{1}T_{1}^{-1} = V_{2}T_{2}^{-1}, & \text{if } P \text{ and } n \text{ are constant (Charles' law)}.
             \end{cases}\label{Chapter:Intro_Property_of_Gases:Eqn:IdealEOS}\index{Equation of State!Ideal gas}
          \end{equation}
        The constant of proportionality, $R$, which is found experimentally to be the same for all gases, is called {\bf universal gas constant} (Table~\ref{Chapter:Intro_Property_of_Gases:Table:RConst}). This expression, 
          \begin{equation}
             P=\frc{n R T}{V},\label{Chapter:Intro_Property_of_Gases:Eqn:IdealEOS}\index{Equation of State!Ideal gas}
          \end{equation}\index{Equation of State!Ideal gas}
        is called {\bf ideal gas equation of state} and becomes increasingly accurate for any gas as pressure tends to zero $\left(P\rightarrow 0\right)$.
     \end{shaded}
     \end{subequations}
     
   \begin{table}[h]
     \begin{center}
     \begin{tabular}{||c c||}
       \hline\hline
           $\mathbf{R}$   & {\bf Units} $\mathbf{\left(\text{V.P.T}^{-1}\text{n}^{-1}\right)}$ \\
           \hline\hline
           8.3145    &  J.K$^{-1}$.mol$^{-1}$  \\
           8.3145    &  kJ.K$^{-1}$.kmol$^{-1}$  \\
           8.3145    &  l.kPa.K$^{-1}$.mol$^{-1}$  \\
           8.3145$\times$10$^{-3}$    & cm$^{3}$.kPa.K$^{-1}$.mol$^{-1}$  \\
           8.3145    &  m$^{3}$.Pa.K$^{-1}$.mol$^{-1}$  \\
           8.3145$\times$10$^{-5}$    &  m$^{3}$.bar.K$^{-1}$.mol$^{-1}$  \\
           8.2057$\times$10$^{-2}$ &  l.atm.K$^{-1}$.mol$^{-1}$  \\
           \hline\hline           
     \end{tabular}
     \caption{Gas constant, $R$.}\label{Chapter:Intro_Property_of_Gases:Table:RConst}\index{Universal gas constant}\index{R|see {Universal gas constant}}\index{Gas constant|see {Universal gas constant}}
     \end{center}
   \end{table}
   
   % Example
   \begin{MyExample}{\begin{center}{\bf Example}\end{center}}
     \begin{example}\label{Chapter:Intro_Property_of_Gases:Example1}
       In an industrial process, nitrogen is heated to 650.15 K in a vessel of constant volume. If it enters the vessel at 43 atm and 298.15 K, what pressure would it exert at the working temperature if it behaved as an ideal gas?

       {\it This problem deals with a gas that undertakes an isochoric (\ie constant volume) process. Nitrogen gas at $P_{1} = 43$ atm and $T_{1} = 298.15$ K is compressed to pressure $P_{2}$ 'till temperature reaches $T_{2} = 650.15$ K.

         Using the ideal gas equation for states 1 and 2,}
       \begin{eqnarray}
         && \left(\frc{PV}{nT}\right)_{1} = R = \left(\frc{PV}{nT}\right)_{2},\;\;\;\text{ with } V_{1}=V_{2} \text{ and }\;\; n_{1}=n_{2}\nonumber \\
         && \frc{P_{1}}{T_{1}} = \frc{P_{2}}{T_{2}} \;\;\Rightarrow \;\; P_{2}= 93.7664 \text{ atm } \nonumber
         \end{eqnarray}
     \end{example}
   \end{MyExample}

\medskip
   % Example
   \begin{MyExample}{\begin{center}{\bf Example}\end{center}}
     \begin{example}\label{Chapter:Intro_Property_of_Gases:Example2}
       500 kg of helium gas is stored in a tank at 50$^{\circ}$C and 2.5 bar. The fluid is then transferred to a pressure vessel where it is isothermically compressed to 73 bar.  What is the volume occupied by {\it He} in such conditions? Assume ideal gas behaviour. Molar mass ({\it MW}) of helium is 4.0026 g.mol$^{-1}$.

       {\it In this problem, we need to calculate the volume of helium at the pressure vessel after isothermal compression. However, not all necessary conditions are known, \ie}
       \begin{displaymath}
         \frc{P_{1}V_{1}}{n_{1} R T_{1}} = \frc{P_{2}V_{2}}{n_{2} R T_{2}} \Longrightarrow P_{1}V_{1} = P_{2}V_{2},\;\;\;\;\text{where }R\text{ and } n_{i}\text{ are constants},
       \end{displaymath}
           {\it thus there is 1 equation and 2 unknowns, $V_{1}$ and $V_{2}$. In order to solve this problem, we need to first compute $n_{1}$ and $V_{1}$ via, }
           \begin{displaymath}
              V_{1} = \frc{n_{1}R T_{1}}{P_{1}} = 1342.5427\;\text{m}^{3},\;\;\;\text{ where } n_{1} = \frc{m_{1}}{MW_{1}} =  124918.8028\text{ moles}.
           \end{displaymath}
           {\it With $V_{1}$, we can now calculate the volume after compression, $V_{2}$}
           \begin{displaymath}
              P_{1}V_{1} = P_{2}V_{2} \;\;\; \Longrightarrow\;\;\; V_{2} = 45.9775 \text{m}^{3}
           \end{displaymath}
           
     \end{example}
   \end{MyExample}
   
   % Example
   \begin{MyExample}{\begin{center}{\bf Example}\end{center}}
     \begin{example}\label{Chapter:Intro_Property_of_Gases:Example3}
       \citep{Atkins_Book} 
       \begin{enumerate}[a)]
           \item Deduce a relation between pressure and mass density $\left(\rho\right)$ of an ideal gas of molar mass $MW$;
           \item After careful measurement of the density of dimethyl ether at relatively low pressure conditions, a chemical engineering student obtained the following experimental data at 25$^{\circ}$C,
               \begin{center}
                  \begin{tabular}{c c c c c c}
                     $\mathbf{P}$ [kPa]  & 12.223 & 25.20 & 36.97 & 85.23 & 101.3 \\
                     $\mathbf{\rho}\;\left[\text{kg.m}^{-3}\right]$ & 0.225  &  0.456  &  0.664 & 1.468 & 1.734  
                  \end{tabular}
               \end{center}
               Obtain the molar mass of dimethyl ether at each pressure coordinate and compare with the real value of 46.07 g.mol$^{-1}$. What conclusions could be drawn from this data?
       \end{enumerate}
\medskip

       \begin{enumerate}[a)]
           \item {\it The first part of the problem requires the development of a mathematical relation between $P$ and $\rho$. Defining} $\rho=\frc{m}{V}$ {\it which can be allocated in the ideal EOS} $\left(\text{ and }n=\frc{m}{MW}\right)$,
              \begin{eqnarray}
                 P &=& \frc{n R T }{V} = \frc{m}{MW}\frc{R T}{V} \nonumber\\ 
                   &=& \frc{\rho R T}{MW}\nonumber
              \end{eqnarray}
%
           \item {\it We can use the relation derived in part (a) to calculate the molar mass of dimethyl ether at each pressure coordinate, thus}
              \begin{displaymath}
                  \begin{cases}
                     MW_{1} = \frc{\rho_{1} R T}{P_{1}} = \frc{0.225\text{ kg.m}^{-3}\cdot 8.3145\text{ m}^{3}\text{.Pa.}\left(\text{K.mol}\right)^{-1}\cdot 298.15\text{ K}}{12.223\times 10^{3}\text{ Pa}} = 45.6326 \text{ g.mol}^{-1} \\
                     MW_{2} = \frc{\rho_{2} R T}{P_{2}} = \frc{0.456\text{ kg.m}^{-3}\cdot 8.3145\text{ m}^{3}\text{.Pa.}\left(\text{K.mol}\right)^{-1}\cdot 298.15\text{ K}}{25.20\times 10^{3}\text{ Pa}} = 44.8575 \text{ g.mol}^{-1} \\
                     MW_{3} = \frc{\rho_{3} R T}{P_{3}} = \frc{0.664\text{ kg.m}^{-3}\cdot 8.3145\text{ m}^{3}\text{.Pa.}\left(\text{K.mol}\right)^{-1}\cdot 298.15\text{ K}}{36.97\times 10^{3}\text{ Pa}} = 44.5235 \text{ g.mol}^{-1} \\
                     MW_{4} = \frc{\rho_{4} R T}{P_{4}} = \frc{1.468\text{ kg.m}^{-3}\cdot 8.3145\text{ m}^{3}\text{.Pa.}\left(\text{K.mol}\right)^{-1}\cdot 298.15\text{ K}}{85.23\times 10^{3}\text{ Pa}} = 42.6977 \text{ g.mol}^{-1} \\
                     MW_{5} = \frc{\rho_{5} R T}{P_{5}} = \frc{1.734\text{ kg.m}^{-3}\cdot 8.3145\text{ m}^{3}\text{.Pa.}\left(\text{K.mol}\right)^{-1}\cdot 298.15\text{ K}}{101.3\times 10^{3}\text{ Pa}} = 42.4337 \text{ g.mol}^{-1} \\
                  \end{cases} 
              \end{displaymath}
              {\it We can compare these values with the real molar mass through the relative error ($\%$),}
                 \begin{displaymath}
                      \epsilon = \frc{\left| MW_{i} - MW^{\text{exp}}\right|}{MW^{\text{exp}}}\times 100,
                 \end{displaymath}
                 {\it leading to}
               \begin{center}
                  \begin{tabular}{c |c c c c c}
                     $\mathbf{P}$ [kPa]                            & 12.223  & 25.20   & 36.97   & 85.23   & 101.3    \\
                     $\mathbf{\rho}\;\left[\text{kg.m}^{-3}\right]$ &  0.225  &  0.456  &  0.664  &  1.468  &   1.734  \\
                     $\mathbf{MW}\;\left[\text{g.mol}^{-1}\right]$  & 45.6326 & 44.8575 & 44.5235 & 42.6977 &  45.4337 \\
                     $\mathbf{\varepsilon}\;\left[\%\right]$       &  0.9494 &  2.6319 &  3.3568 &  7.3199 &   7.8930
                  \end{tabular}
               \end{center}
               {\it From this data, it is clear that as $P\rightarrow 0$ dimethyl ether behaves similar to an ideal gas. However, as pressure increases the ideal gas EOS becomes less accurate.} 

                
       \end{enumerate}

           
     \end{example}
   \end{MyExample}
   

%%%
%%% SECTION
%%%
   \section{Gas Mixtures}\label{Chapter:Intro_Property_of_Gases:Section:MixtureGases}\index{Gases!Mixture }\index{Gases!Dalton's law}\index{Dalton's law|see {Gases}}\index{Pressure!Partial}\index{Partial pressure|see {Pressure}}

   \begin{subequations}
     Let's consider a gaseous mixture containing $\mathcal{N}$ chemical species. In several applications it is necessary to determine the pressure that each gas exerts in the whole system, \ie the contribution of each gas in the total pressure of the mixture. This contribution is often referred as {\it partial pressure}, $P_{i}$, of the gas {\it i} in a gas mixture and is defined as
     \begin{equation}
        P_{i} \equiv y_{i}P,\label{Chapter:Intro_Property_of_Gases:Eqn:PartialPressure_1}
     \end{equation}
     where $P$ is the total pressure and $y_{i}$ is the {\it mole fraction} of component $i$,\index{Mole fraction}
     \begin{displaymath}
        y_{i} = \frc{n_{i}}{n},\;\;\;\text{ with }\;\;i=1,\cdots,\mathcal{N}.
     \end{displaymath}
     The mole fraction is a normalised quantity and as such it must sum to unity, 
     \begin{displaymath}
        \summation[y_{i}]{i=1}{\mathcal{N}} = 1
     \end{displaymath}
     If we sum up the partial pressures of all components in the gaseous mixture, the total pressure of the system is recovered, \ie
     \begin{equation}
       \summation[P_{i}]{i=1}{\mathcal{N}} = \summation[y_{i}P]{i=1}{\mathcal{N}} = P\label{Chapter:Intro_Property_of_Gases:Eqn:PartialPressure_2}
     \end{equation}
     This equation is also known as {\it Dalton's law} and it is valid for any ideal or real gaseous mixtures. It can be stated as
     \begin{shaded}
       ``The pressure exerted by a mixture of gases is the sum of the pressures that each one would exist if it occupied the container alone'' \citep{Atkins_Book}.
     \end{shaded}

   \end{subequations}
   

\medskip
   % Example
   \begin{MyExample}{\begin{center}{\bf Example}\end{center}}
     \begin{example}\label{Chapter:Intro_Property_of_Gases:Example4}
       A gaseous mixture of methane, ethane and ethylene is stored in a tank at 6 atm. The composition (in weight) of the mixture is 42.16$\%$ of CH$_{4}$ and 32.11$\%$ of C$_{2}$H$_{6}$. What is the partial pressure of each gas in the mixture? Molar mass ({\it MW}) of methane, ethane and ethylene are 16.04 g.mol$^{-1}$, 30.07 g.mol$^{-1}$ and 28.05 g.mol$^{-1}$, respectively.
\medskip

       {\it Partial pressure of CH$_{4}$, C$_{2}$H$_{6}$ and C$_{2}$H$_{4}$ can be obtained from Eqn.~\ref{Chapter:Intro_Property_of_Gases:Eqn:PartialPressure_1}, however we first need to calculate the number of moles of each species in the mixture. Assuming a 100 g mixture,}
       \begin{displaymath}
         \begin{cases}
           n_{\text{CH}_{4}} = \frc{m_{\text{CH}_{4}}}{MW_{\text{CH}_{4}}} = \frc{42.16\text{ g}}{16.04\text{ g.mol}^{-1}}= 2.6284\text{ moles},  \\
           n_{\text{C}_{2}\text{H}_{6}} = \frc{m_{\text{C}_{2}\text{H}_{6}}}{MW_{\text{C}_{2}\text{H}_{6}}} = \frc{32.11\text{ g}}{30.07\text{ g.mol}^{-1}}= 1.0678\text{ moles},  \\
           n_{\text{C}_{2}\text{H}_{4}} = \frc{m_{\text{C}_{2}\text{H}_{4}}}{MW_{\text{C}_{2}\text{H}_{4}}} = \frc{25.73\text{ g}}{28.05\text{ g.mol}^{-1}}= 0.9173\text{ moles}. 
         \end{cases}
       \end{displaymath}
       {\it The mole fraction of each species can be obtained, considering}
       \begin{displaymath}
         n = \summation[n_{i}]{i=1}{3} = n_{\text{CH}_{4}}+n_{\text{C}_{2}\text{H}_{6}}+n_{\text{C}_{2}\text{H}_{4}} = 4.6135\text{ moles}
       \end{displaymath}
       {\it thus,}
       \begin{displaymath}
         \begin{cases}
           y_{\text{CH}_{4}} = \frc{n_{\text{CH}_{4}}}{n} = 0.5697  \\
           y_{\text{C}_{2}\text{H}_{6}} = \frc{n_{\text{C}_{2}\text{H}_{6}}}{n} = 0.2315,  \\
           y_{\text{C}_{2}\text{H}_{4}} = \frc{n_{\text{C}_{2}\text{H}_{4}}}{n} = 0.1988. 
         \end{cases}
       \end{displaymath}
       {\it With $y_{i}$ we can finally calculate the partial pressure of each gas:}
       \begin{displaymath}
         \begin{cases}
           P_{\text{CH}_{4}} = P\cdot y_{\text{CH}_{4}} = 3.4182\text{ atm}  \\
           P_{\text{C}_{2}\text{H}_{6}} = P\cdot y_{\text{C}_{2}\text{H}_{6}} = 1.3890\text{ atm},  \\
           P_{\text{C}_{2}\text{H}_{4}} = P\cdot y_{\text{C}_{2}\text{H}_{4}} = 1.1928\text{ atm}. 
         \end{cases}
       \end{displaymath}
       {\it We can check if the calculations are correct by}
       \begin{displaymath}
         \begin{cases}
           \summation[y_{i}]{i=1}{3} = 1.0000  & \textit{ and } \\
           \summation[P_{i}]{i=1}{3} = P = 6\text{ atm} & \\
         \end{cases}
       \end{displaymath}
           
     \end{example}
   \end{MyExample}
\medskip

   % Example
   \begin{MyExample}{\begin{center}{\bf Example}\end{center}}
     \begin{example}\label{Chapter:Intro_Property_of_Gases:Example5}
       \citep{Atkins_Book} A vessel of volume 22.4 dm$^{3}$ contains 2.0 mol H$_{2}$ and 1.0 mol N$_{2}$ at 273.15 K. Calculate (a) the mole fractions of each component, (b) their partial pressures, and (c) their total pressure. Assume ideal gas behaviour.
\medskip

      {\it Here we want to calculate the partial pressure of gaseous mixture of hydrogen and nitrogen contained in a vessel of 22.4 dm$^{3}$ at 273.15 K. Partial pressures can be obtained from the total pressure of the system and the mole fraction of the gases, } $P_{i}=y_{i}P$, {\it thus we first need to obtain $P$ through the ideal gas EOS, with} $n=n_{H_{2}}+n_{N_{2}}=3$,
      \begin{displaymath}
        P = \frc{n R T }{V} = \frc{3\text{ mol } \cdot 8.3145\times 10^{-5}\text{ m}^{3}\text{.bar.}\left(\text{K.mol}\right)^{-1}\cdot 273.15\text{ K}}{22.4\times 10^{-3}\text{ m}^{3}} = 3.0417\text{ bar, }
      \end{displaymath}
      {\it and the mole fraction of both gasses,}
       \begin{displaymath}
         \begin{cases}
           y_{\text{H}_{2}} = \frc{n_{\text{H}_{2}}}{n} = 0.6667  \\
           y_{\text{N}_{2}} = \frc{n_{\text{N}_{2}}}{n} = 0.3333  
         \end{cases}
       \end{displaymath}
       {\it Now we can finally compute partial pressures:}
       \begin{displaymath}
         \begin{cases}
           P_{\text{H}_{2}} = Py_{\text{H}_{2}} = 2.0279\text{ bar},  \\
           P_{\text{N}_{2}} = Py_{\text{N}_{2}} = 1.0138\text{ bar}.
         \end{cases}
       \end{displaymath}


       {\it We can check if the calculations are correct by}
       \begin{displaymath}
         \begin{cases}
           \summation[y_{i}]{i=1}{2} = 1.0000  & \textit{ and } \\
           \summation[P_{i}]{i=1}{2} = P = 3.0417\text{ bar} & \\
         \end{cases}
       \end{displaymath}
           
     \end{example}
   \end{MyExample}
\medskip



   
\clearpage   
\begin{FinalSummaryBlock}{Summary}
    \begin{itemize}
       \item A fluid behaves as an ideal gas at low to moderate pressures $\left(\text{\ie } P\rightarrow 0\right)$
       \item Equation of state is function that correlates pressure, volume, temperature and the amount of chemical component(s);
       \item An ideal gas obeys the ideal gas equation of state (Eqn.~\ref{Chapter:Intro_Property_of_Gases:Eqn:IdealEOS});
       \item Partial pressure of a gas in a gaseous mixture is defined as the pressure that the gas would exerted by itself (Eqn.~\ref{Chapter:Intro_Property_of_Gases:Eqn:PartialPressure_1});
       \item Dalton's law states that the pressure exerted by a mixture of gases is the sum of the partial pressures of the gases.
    \end{itemize}
\end{FinalSummaryBlock}
