% Aberdeen style guide should be followed when using this
% layout. Their template powerpoint slide is used to extract the
% Aberdeen color and logo but is otherwise ignored (it has little or
% no formatting in it anyway).
%
% http://www.abdn.ac.uk/documents/style-guide.pdf

%%%%%%%%%%%%%%%%%%%% Document Class Settings %%%%%%%%%%%%%%%%%%%%%%%%%
% Pick if you want slides, or draft slides (no animations)
%%%%%%%%%%%%%%%%%%%%%%%%%%%%%%%%%%%%%%%%%%%%%%%%%%%%%%%%%%%%%%%%%%%%%%
%Normal document mode%
\documentclass[10pt,compress]{beamer}
%Draft or handout mode
%\documentclass[10pt,compress,handout]{beamer}
%\documentclass[10pt,compress,handout,ignorenonframetext]{beamer}

%%%%%%%%%%%%%%%%%%%% General Document settings %%%%%%%%%%%%%%%%%%%%%%%
% These settings must be set for each presentation
%%%%%%%%%%%%%%%%%%%%%%%%%%%%%%%%%%%%%%%%%%%%%%%%%%%%%%%%%%%%%%%%%%%%%%
\newcommand{\shortname}{jefferson.gomes@abdn.ac.uk}
\newcommand{\fullname}{Dr Jeff Gomes}
\institute{School of Engineering}
\newcommand{\emailaddress}{}%jefferson.gomes@abdn.ac.uk}
\newcommand{\logoimage}{../../FigBanner/UoAHorizBanner}
\title{Chemical Thermodynamics (EX3029)}
\subtitle{Module 3: Thermodynamic Properties of Pure Fluids}
\date[ ]{ }

%%%%%%%%%%%%%%%%%%%% Template settings %%%%%%%%%%%%%%%%%%%%%%%%%%%%%%%
% You shouldn't have to change below this line, unless you want to.
%%%%%%%%%%%%%%%%%%%%%%%%%%%%%%%%%%%%%%%%%%%%%%%%%%%%%%%%%%%%%%%%%%%%%%
\usecolortheme{whale}
\useoutertheme{infolines}

% Use the fading effect for items that are covered on the current
% slide.
\beamertemplatetransparentcovered

% We abuse the author command to place all of the slide information on
% the title page.
\author[\shortname]{%
  \fullname\\\ttfamily{\emailaddress}
}


%At the start of every section, put a slide indicating the contents of the current section.
\AtBeginSection[] {
  \begin{frame}
    \frametitle{Section Outline}
    \tableofcontents[currentsection]
  \end{frame}
}

% Allow the inclusion of movies into the Presentation! At present,
% only the Okular program is capable of playing the movies *IN* the
% presentation.
\usepackage{multimedia}
\usepackage{animate}

%% Handsout -- comment out the lines below to create handstout with 4 slides in a page with space for comments
\usepackage{handoutWithNotes}

\mode<handout>
{
\usepackage{pgf,pgfpages}

\pgfpagesdeclarelayout{2 on 1 boxed with notes}
{
\edef\pgfpageoptionheight{\the\paperheight} 
\edef\pgfpageoptionwidth{\the\paperwidth}
\edef\pgfpageoptionborder{0pt}
}
{
\setkeys{pgfpagesuselayoutoption}{landscape}
\pgfpagesphysicalpageoptions
    {%
        logical pages=4,%
        physical height=\pgfpageoptionheight,%
        physical width=\pgfpageoptionwidth,%
        last logical shipout=2%
    } 
\pgfpageslogicalpageoptions{1}
    {%
    border code=\pgfsetlinewidth{1pt}\pgfstroke,%
    scale=1,
    center=\pgfpoint{.25\pgfphysicalwidth}{.75\pgfphysicalheight}%
    }%
\pgfpageslogicalpageoptions{2}
    {%
    border code=\pgfsetlinewidth{1pt}\pgfstroke,%
    scale=1,
    center=\pgfpoint{.25\pgfphysicalwidth}{.25\pgfphysicalheight}%
    }%
\pgfpageslogicalpageoptions{3}
    {%
    border shrink=\pgfpageoptionborder,%
    resized width=.7\pgfphysicalwidth,%
    resized height=.5\pgfphysicalheight,%
    center=\pgfpoint{.75\pgfphysicalwidth}{.29\pgfphysicalheight},%
    copy from=3
    }%
\pgfpageslogicalpageoptions{4}
    {%
    border shrink=\pgfpageoptionborder,%
    resized width=.7\pgfphysicalwidth,%
    resized height=.5\pgfphysicalheight,%
    center=\pgfpoint{.75\pgfphysicalwidth}{.79\pgfphysicalheight},%
    copy from=4
    }%

\AtBeginDocument
    {
    \newbox\notesbox
    \setbox\notesbox=\vbox
        {
            \hsize=\paperwidth
            \vskip-1in\hskip-1in\vbox
            {
                \vskip1cm
                Notes\vskip1cm
                        \hrule width\paperwidth\vskip1cm
                    \hrule width\paperwidth\vskip1cm
                        \hrule width\paperwidth\vskip1cm
                    \hrule width\paperwidth\vskip1cm
                        \hrule width\paperwidth\vskip1cm
                    \hrule width\paperwidth\vskip1cm
                    \hrule width\paperwidth\vskip1cm
                    \hrule width\paperwidth\vskip1cm
                        \hrule width\paperwidth
            }
        }
        \pgfpagesshipoutlogicalpage{3}\copy\notesbox
        \pgfpagesshipoutlogicalpage{4}\copy\notesbox
    }
}
}

%\pgfpagesuselayout{2 on 1 boxed with notes}[letterpaper,border shrink=5mm]

%%%%% Color settings
\usepackage{color}
%% The background color for code listings (i.e. example programs)
\definecolor{lbcolor}{rgb}{0.9,0.9,0.9}%
\definecolor{UoARed}{rgb}{0.64706, 0.0, 0.12941}
\definecolor{UoALight}{rgb}{0.85, 0.85, 0.85}
\definecolor{UoALighter}{rgb}{0.92, 0.92, 0.92}
\setbeamercolor{structure}{fg=UoARed} % General background and higlight color
\setbeamercolor{frametitle}{bg=black} % General color
\setbeamercolor{frametitle right}{bg=black} % General color
\setbeamercolor{block body}{bg=UoALighter} % For blocks
\setbeamercolor{structure}{bg=UoALight} % For blocks
% Rounded boxes for blocks
\setbeamertemplate{blocks}[rounded]

%%%%% Font settings
% Aberdeen requires the use of Arial in slides. We can use the
% Helvetica font as its widely available like so
% \usepackage{helvet}
% \renewcommand{\familydefault}{\sfdefault}
% But beamer already uses a sans font, so we will stick with that.

% The size of the font used for the code listings.
\newcommand{\goodsize}{\fontsize{6}{7}\selectfont}

% Extra math packages, symbols and colors. If you're using Latex you
% must be using it for formatting the math!
\usepackage{amscd,amssymb} \usepackage{amsfonts}
\usepackage[mathscr]{eucal} \usepackage{mathrsfs}
\usepackage{latexsym} \usepackage{amsmath} \usepackage{bm}
\usepackage{amsthm} \usepackage{textcomp} \usepackage{eurosym}
% This package provides \cancel{a} and \cancelto{a}{b} to "cancel"
% expressions in math.
\usepackage{cancel}

\usepackage{comment} 

% Get rid of font warnings as modern LaTaX installations have scalable
% fonts
\usepackage{type1cm} 

%\usepackage{enumitem} % continuous numbering throughout enumerate commands

% For exact placement of images/text on the cover page
\usepackage[absolute]{textpos}
\setlength{\TPHorizModule}{1mm}%sets the textpos unit
\setlength{\TPVertModule}{\TPHorizModule} 

% Source code formatting package
\usepackage{listings}%
\lstset{ backgroundcolor=\color{lbcolor}, tabsize=4,
  numberstyle=\tiny, rulecolor=, language=C++, basicstyle=\goodsize,
  upquote=true, aboveskip={1.5\baselineskip}, columns=fixed,
  showstringspaces=false, extendedchars=true, breaklines=false,
  prebreak = \raisebox{0ex}[0ex][0ex]{\ensuremath{\hookleftarrow}},
  frame=single, showtabs=false, showspaces=false,
  showstringspaces=false, identifierstyle=\ttfamily,
  keywordstyle=\color[rgb]{0,0,1},
  commentstyle=\color[rgb]{0.133,0.545,0.133},
  stringstyle=\color[rgb]{0.627,0.126,0.941}}

% Allows the inclusion of other PDF's into the final PDF. Great for
% attaching tutorial sheets etc.
\usepackage{pdfpages}
\setbeamercolor{background canvas}{bg=}  

% Remove foot note horizontal rules, they occupy too much space on the slide
\renewcommand{\footnoterule}{}

% Force the driver to fix the colors on PDF's which include mixed
% colorspaces and transparency.
\pdfpageattr {/Group << /S /Transparency /I true /CS /DeviceRGB>>}

% Include a graphics, reserve space for it but
% show it on the next frame.
% Parameters:
% #1 Which slide you want it on
% #2 Previous slides
% #3 Options to \includegraphics (optional)
% #4 Name of graphic
\newcommand{\reserveandshow}[4]{%
\phantom{\includegraphics<#2|handout:0>[#3]{#4}}%
\includegraphics<#1>[#3]{#4}%
}

\newcommand{\frc}{\displaystyle\frac}
\newcommand{\red}{\textcolor{red}}
\newcommand{\blue}{\textcolor{blue}}
\newcommand{\green}{\textcolor{green}}
\newcommand{\purple}{\textcolor{purple}}
\newcommand{\eg}{{\it e.g., }}
\newcommand{\ie}{{\it i.e., }}
\newcommand{\wrt}{{\it wrt }}
\newcommand{\Partial}[3][error]{\left(\frc{\partial #1}{\partial #2}\right)_{#3}}
\newcommand{\mfr}[3][error]{#1_{#2}^{\left(#3\right)}} 
\newcommand{\summation}[3][error]{\sum\limits_{#2}^{#3}#1}
 
\begin{document}

% Title page layout
\begin{frame}
  \titlepage
  \vfill%
  \begin{center}
    \includegraphics[clip,width=0.8\textwidth]{\logoimage}
  \end{center}
\end{frame}

% Table of contents
\frame{ \frametitle{Slides Outline}
  \tableofcontents
}


%%%%%%%%%%%%%%%%%%%% The Presentation Proper %%%%%%%%%%%%%%%%%%%%%%%%%
% Fill below this line with \begin{frame} commands! It's best to
% always add the fragile option incase you're going to use the
% verbatim environment.
%%%%%%%%%%%%%%%%%%%%%%%%%%%%%%%%%%%%%%%%%%%%%%%%%%%%%%%%%%%%%%%%%%%%%%


%%%
%%% SECTION
%%%
%\section{General Remarks}

%%%
%%% Slides
%%%
\begin{frame}
 \frametitle{Aims and Objectives}
   \begin{enumerate}
     \item<1-> In Modules 1-2, we learnt:
       \begin{enumerate}
         \item<1-> the laws of Thermodynamics and how they describe thermal equilibrium of species in closed and opened systems;
         \item<1-> how to calculate thermodynamics properties -- internal energy, enthalpy and entropy for pure chemicallspecies;
         \item<1-> PVT behaviour of pure species in equilibrium and;
         \item<1-> Equations of state.
       \end{enumerate} 
     \item<2-> This Module focuses on 
         \begin{enumerate}
           \item<2-> Thermodynamic properties of pure fluids;
           \item<2-> Introduce two remaining thermodynamic properties: Gibbs and Helmholtz free energies;
           \item<2-> Maxwell relations.
         \end{enumerate}
   \end{enumerate}

\end{frame}


%%%
%%% SECTION
%%%
%\section{Bibliography}
\begin{frame}
 \frametitle{Suggested References}
  Literature relevant for this module:
  \begin{enumerate}[(a)]
   \item\label{SVN_Book} J.M. Smith, H.C. Van Ness, M.M. Abbott, $\lq$Introduction to Chemical Engineering Thermodynamics', 6$^{th}$ Edition: Chapter 6;
   \item Y.A. Cengel, M.A. Boles, $\lq$Thermodynamics -- An Engineering Approach', 5$^{th}$ Edition: Chapter 12.1-4; 
   %\item M.J. Moran, H.N. Saphiro, D.D. Boettner, M.B. Bailey, $\lq$Principles of Engineering Thermodynamics', 7$^{th}$ Edition: Chapters 3;
   %\item C. Borgnakke, R.E. Sonntag,$\lq$Fundamentals of Thermodynamics',8$^{th}$ Edition: Chapter 2.
   \item S.I. Sandler, $\lq$Chemical, Biochemical and Engineering Thermodynamics', 4$^{th}$ Edition: Chapter 6.
  \end{enumerate}
\end{frame}


%%%
%%% SECTION
%%%
\section{Property Relations for Homogeneous Phases}

%%%
%%% SUBSECTION
%%%
\subsection{Thermodynamic Potentials} 


%%%
%%% Slide
%%%
%\scriptsize
\begin{frame}
  \frametitle{New State Functions}
   \visible<1->{\begin{block}{Enthalpy}
      \begin{displaymath}
         H = U + P V
      \end{displaymath}
   \end{block}
   }
   \visible<2->{\begin{block}{Gibbs Free Energy}
      \begin{displaymath}
         G = H - T S
      \end{displaymath}
   \end{block}
   }
   \visible<3->{\begin{block}{Helmholtz Free Energy}
      \begin{displaymath}
         A = U - T S
      \end{displaymath}
   \end{block}
   }

\end{frame}
\normalsize

%%%
%%% SUBSECTION
%%%
\subsection{Maxwell's Relations}
%%%
%%% Slide
%%%
%\scriptsize
\begin{frame}
  \frametitle{Maxwell's Relations}\label{Module03:Section:MaxwellRelation}
   \visible<1->{\begin{block}{For 1 mol of homogeneous fluid at $T=$ constant}
      \begin{center}
        \begin{tabular}{l c  l}
           $dU = T dS - PdV$  &  \hspace{1cm} & $dH = TdS + VdP$ \\
           $dA = -PdV - SdT$  &  \hspace{1cm} & $dG = VdP - SdT$ 
        \end{tabular}
      \end{center}
   \end{block}
   }
   \visible<2->{\begin{block}{Maxwell's Equations}
      \begin{center}
        \begin{tabular}{l c  l}
           $\left(\frc{\partial T}{\partial V}\right)_{S} = -\left(\frc{\partial P}{\partial S}\right)_{v}$  &  \hspace{1cm} & $\left(\frc{\partial T}{\partial P}\right)_{S} =  \left(\frc{\partial V}{\partial S}\right)_{P}$  \\
           $\left(\frc{\partial P}{\partial T}\right)_{V} =  \left(\frc{\partial S}{\partial V}\right)_{T}$  &  \hspace{1cm} & $\left(\frc{\partial V}{\partial T}\right)_{\red{P}} = -\left(\frc{\partial S}{\partial P}\right)_{T}$ 
        \end{tabular}
      \end{center}
   \end{block}
   }

\end{frame}
\normalsize


%%%
%%% SUBSECTION
%%%
\subsection{Thermodynamic Potentials: Dependence on $T$ and $P$}

%%%
%%% Slide
%%%
%\scriptsize
\begin{frame}
  \frametitle{Internal Energy, Enthalpy and Entropy}
   \visible<1->{\begin{block}{$H$ and $S$ as functions of $T$ and $P$}
      \begin{center}
        \begin{tabular}{l c  l}
           $dH = C_{p}dT + \left[ V - T\left(\frc{\partial V}{\partial T}\right)_{P}\right]dP$ &  \hspace{1cm} & $dS=C_{p}\frc{dT}{T} - \left(\frc{\partial V}{\partial T}\right)_{P}dP$
        \end{tabular}
      \end{center}
   \end{block}
   }
   \visible<2->{\begin{block}{Ideal Gas}
      \begin{center}
        \begin{tabular}{l c  l}
           $dH = C_{p}dT$  &  \hspace{1cm} & $dS=C_{p}\frc{dT}{T} - R\frc{dP}{P}$  
        \end{tabular}
      \end{center}
   \end{block}
   }
   \visible<3->{\begin{block}{Liquid}
      \begin{center}
        \begin{tabular}{l c  l}
           $dH = C_{p}dT + \left(1-\beta\right)VdP$  &  \hspace{1cm} & $dS = C_{p}\frc{dT}{T} - \beta V{dP}$  
        \end{tabular}
      \end{center}
   \end{block}
   }
   \visible<4->{\begin{block}{$U$ and $S$ as functions of $T$ and $V$}
      \begin{center}
        \begin{tabular}{l c  l}
           $dU = C_{v}dT + \left[T\left(\frc{\partial P}{\partial T}\right)_{V} - P\right]dV$  &  \hspace{1cm} & $dS = C_{v}\frc{dT}{T} - \left(\frc{\partial P}{\partial T}\right)_{V}dV$  
        \end{tabular}
      \end{center}
   \end{block}
   }

\end{frame}
\normalsize


%%%
%%% Slide
%%%
%\scriptsize
\begin{frame}
  \frametitle{Gibbs Free Energy as Differential Operator}
     \begin{enumerate}[(a)]
        \item<1-> Gibbs Energy as generating function:
              \begin{displaymath}
                 \visible<1->{dG = VdP - SdT} \visible<2->{\Longrightarrow d\left(\frc{G}{RT}\right) = \frc{1}{RT}dG - \frc{G}{RT^{2}}dT} \nonumber 
              \end{displaymath}

        \item<3-> After substitution $\&$ algebraic reduction:
              \begin{displaymath}
                  \visible<3->{\frc{V}{RT} = \left\{ \frc{\partial\left(\frc{G}{RT}\right)}{\partial P}\right\}_{T} \hspace{1cm}\text{and}\hspace{1cm} \frc{H}{RT} = -T\left\{ \frc{\partial\left(\frc{G}{RT}\right)}{\partial T}\right\}_{P}}
              \end{displaymath}

        \item<4-> \textcolor{blue}{The Gibbs free energy when expressed as $G\left(T,P\right)$ operates as a generating function for other thermodynamic properties, and implicitly represents complete property information.}
     \end{enumerate}

\end{frame}
\normalsize

%%%
%%% SECTION
%%%
\section{Residual Properties}

\subsection{General Approach}
%%%
%%% Slide
%%%
%\scriptsize
\begin{frame}
  \frametitle{Residual Properties: General Approach}
     \begin{enumerate}[(a)]
        \item<1-> \textcolor{blue}{Residual Gibbs energy} is defined as:
              \visible<1->{\begin{displaymath}
                 \textcolor{red}{G^{R} = G - G^{ig}}
              \end{displaymath}}
            where \textcolor{blue}{$G$} is the actual Gibbs energy and \textcolor{blue}{$G^{id}$} is the correspondent thermodynamic function for ideal gas at same $T$ and $P$;
        \item<2-> This leads to,
              \visible<2->{\begin{displaymath}
                 d\left(\frc{G^{R}}{RT}\right) =\frc{V^{R}}{RT}dP - \frc{H^{R}}{RT^{2}}dT
              \end{displaymath}}
             
              \visible<3->{with,\begin{displaymath}
                 \frc{V^{R}}{RT} = \left\{ \frc{\partial\left(\frc{G^{R}}{RT}\right)}{\partial P}\right\}_{T} \hspace{0.5cm}\text{ and }\hspace{0.5cm} \frc{H^{R}}{RT} = -T\left\{ \frc{\partial\left(\frc{G^{R}}{RT}\right)}{\partial T}\right\}_{P}
              \end{displaymath}}
     \end{enumerate}

\end{frame}
\normalsize


%%%
%%% SUBSECTION
%%%
\subsection{Equations of State}
%%%
%%% Slide
%%%
%\scriptsize
\begin{frame}
  \frametitle{Residual Properties by EOS}
       Alternative approach to numerical integration: analytical solution by EOS:
            \begin{enumerate}[(a)]
                \item<1-> Virial EOS: \textcolor{blue}{$Z = 1 + \frc{BP}{RT}$}, 
                   \visible<2->{leading to \begin{displaymath}
                      \frc{G^{R}}{RT} = \displaystyle\int\limits_{0}^{\rho}\left(Z - 1\right)\frc{d\rho}{\rho} + Z - 1 - \ln Z \hspace{1cm} \frc{H^{R}}{RT} = -T \displaystyle\int\limits_{0}^{\rho}\left(\frc{\partial Z}{\partial T}\right)_{\rho}\frc{d\rho}{\rho} + Z - 1
                   \end{displaymath}
                   \begin{displaymath}
                      \frc{S^{R}}{RT} = \ln Z - T \displaystyle\int\limits_{0}^{\rho} \left(\frc{\partial Z}{\partial T}\right)_{\rho} \frc{d\rho}{\rho} - \displaystyle\int\limits_{0}^{\rho}\left(Z - 1\right)\frc{d\rho}{\rho} 
                   \end{displaymath}
}
             \end{enumerate}
\end{frame}
\normalsize



%%%
%%% Slide
%%%
%\scriptsize
\begin{frame}
  \frametitle{Residual Properties by EOS}
            \begin{enumerate}[(a)]\setcounter{enumi}{1}
               \item<1-> Cubic EOS: \textcolor{blue}{$P = \frc{RT}{V - b} - \frc{a\left(T\right)}{\left(V + \epsilon b\right)\left(V + \sigma b\right)}$}
                   \visible<2->{leading to\begin{displaymath}
                      \frc{G^{R}}{RT} = Z - 1 - \ln\left(Z - \beta\right) - q\mathcal{I} \hspace{1cm} \frc{H^{R}}{RT} = Z - 1 + \left[\frc{d\ln\alpha\left(T_{r}\right)}{d\ln T_{r}} - 1\right]q\mathcal{I}
                   \end{displaymath}
                   \begin{displaymath}
                       \frc{S^{R}}{R} = \ln\left(Z - \beta \right) + \frc{d\ln\alpha\left(T_{r}\right)}{d\ln T_{r}}q\mathcal{I} 
                   \end{displaymath}
                   with
                   \begin{displaymath}
                      \mathcal{I} \equiv \displaystyle\int\limits_{0}^{P} \frc{d\left(\rho b\right)}{\left(1 + \epsilon\rho b\right)\left(1 + \sigma\rho b\right)}\;\;\;\;\left(T\text{ const.}\right)
                   \end{displaymath}
                   See Section 6.3 of Smith, Van Ness $\&$ Abbott -- \textcolor{blue}{Reference (\ref{SVN_Book})}.
}
             \end{enumerate}

\end{frame}
\normalsize





%%%
%%% SECTION
%%%
\section{Two-Phase Systems}

\subsection{General Remarks}
%%%
%%% Slide
%%%
%\scriptsize
\begin{frame}
  \frametitle{General Remarks}
     \begin{enumerate}[(a)]
         \item<1-> \textcolor{blue}{Phase transition}: many extensive properties change abruptly during phase transition at given $P$ and $T$: specific volume, internal energy, enthalpy and entropy;
         \item<2-> \textcolor{blue}{Exception:} \textcolor{red}{molar Gibbs energy};
         \item<3-> For 2 phases \textcolor{blue}{$\alpha$} and \textcolor{blue}{$\beta$} of pure species at equilibrium,
            \visible<3->{\begin{block}{\textcolor{blue}{Equilibrium $\Rightarrow$ Equality of Gibbs energy}}
                     \begin{displaymath}
                        \textcolor{red}{G^{\alpha} = G^{\beta}} 
                     \end{displaymath}
            \end{block}}
         \item<4-> Clapeyron equation:
            \visible<4->{\begin{displaymath}
              \frc{d P^{sat}}{d T} = \frc{\Delta H^{lv}}{T \Delta V^{lv}}
            \end{displaymath}}
     \end{enumerate}

\end{frame}
\normalsize


%%%
%%% Slide
%%%
%\scriptsize
\begin{frame}
  \frametitle{General Remarks}
     \begin{enumerate}[(a)]\setcounter{enumi}{4}
         \item<1-> Temperature dependence of vapour pressure \blue{$\Rightarrow$} Empirical approaches for practical applications:
         \begin{enumerate}[(i)]
            \item<2-> Simplest case:
                \visible<2->{\begin{displaymath}
                   \ln P^{sat} = A - \frc{B}{T}
                \end{displaymath}}
            \item<3-> Antoine Equation:
                \visible<3->{\begin{displaymath}
                   \ln P^{sat} = A - \frc{B}{T+C}
                \end{displaymath}}
            \item<4-> Wagner Equation \blue{(more accurate)}:
                \visible<4->{\begin{displaymath}
                   \ln P_{r}^{sat} = \frc{A\tau + B\tau^{1.5} + C\tau^{3} + D\tau^{6}}{1-\tau}\;\;\;\text{ with }\;\;\; \tau = 1 - T_{r}
                \end{displaymath}}
         \end{enumerate}
     \end{enumerate}

\end{frame}
\normalsize


%%%
%%% SUBSECTION 
%%%
\subsection{Liquid/Vapour Systems}
%%%
%%% Slide
%%%
%\scriptsize
\begin{frame}
  \frametitle{Liquid/Vapour Systems}
     \begin{enumerate}[(a)]
         \item<1-> Systems with \blue{saturated vapour} and \blue{saturated liquid} in \red{equilibrium};
         \item<2-> \blue{Mass/Energy Balance} for any extensive property:
            \visible<3->{\begin{displaymath} 
                nV = n^{(l)}V^{(l)} + n^{(v)}V^{(v)} \;\;\; \Leftrightarrow \;\;\; V = x^{(l)}V^{(l)} + x^{(v)}V^{(v)}
             \end{displaymath}
             where $x^{(j)}$ is the \blue{molar/mass fraction} of phase \blue{j = l, v} $\rightarrow$  $x^{(l)} + x^{(v)} = 1$. 
            }
         \item<4-> The mass/molar volume fraction of vapour, \blue{$x^{(v)}$}, is also called \blue{vapour quality}. 
     \end{enumerate}

\end{frame}
\normalsize


%%%
%%%
%%% SUBSECTION 
%%%
\subsection{Thermodynamic Diagrams}

%%%
%%% Slide
%%%
%\scriptsize
\begin{frame}
  \frametitle{Pressure $\times$ Enthalpy ({\it PH}) Diagram}
      \begin{figure}%
        \begin{center}
          \includegraphics[width=1\columnwidth,clip]{./../Pics/LnP_H_Diagram}
        \end{center}
      \end{figure}
\end{frame}
\normalsize


%%%
%%% Slide
%%%
%\scriptsize
\begin{frame}
  \frametitle{Temperature $\times$ Entropy ({\it TS}) Diagram}
      \begin{figure}%
        \begin{center}
          \includegraphics[width=1\columnwidth,clip]{./../Pics/T_S_Diagram}
        \end{center}
      \end{figure}
\end{frame}
\normalsize

%%%
%%% Slide
%%%
%\scriptsize
\begin{frame}
  \frametitle{Enthalpy $\times$ Entropy ({\it Moiller}) Diagram}
      \begin{figure}%
        \begin{center} 
          \includegraphics[width=1\columnwidth,clip]{./../Pics/MoillerDiagram}
        \end{center}
      \end{figure}
\end{frame}
\normalsize


\section{Summary}

%%%
%%% Slide
%%%
%\scriptsize
\begin{frame}
 \frametitle{Summary}
   \begin{enumerate}[(i)]
     \item New thermodynamic potential properties: Gibbs and Helmholtz free energies;
     \item Introduction of Maxwell's relations and applications;
     \item Internal energy, enthalpy, entropy Gibbs and Helmholtz energies described as functions of pressure, volume and temperature (PVT);
     \item Introduction of residual properties and applications;
     \item Two-phase systems.
   \end{enumerate}
\end{frame}

\section{Examples}

%%%
%%% Slide
%%%
%\scriptsize
\begin{frame}
   \frametitle{Example 1}%[label=Module03:Example01]
    \blue{A block of copper of 1 kg undertakes a reversible compression from 0.1 MPa to 100 MPa at constant temperature of 15$^{\circ}$C. Calculate:}
    \begin{enumerate}[a)]
       \item \blue{Work done on the copper block during the process;}
       \item \blue{Change in entropy {\it per} kg of copper;}
       \item \blue{Heat transfer and;}
       \item \blue{Change of internal energy {\it per} kg.}
    \end{enumerate}
    \blue{Given, }
    \begin{itemize}
       \item \blue{Volume expansivity coefficient: $\beta = 5\times 10^{-5}$ K$^{-1}$;}
       \item \blue{Isothermal compressibility coefficient: $\kappa = 8.6\times 10^{-12}$ m$^{2}$.N$^{-1}$;}
       \item \blue{specific volume: $v=1.14\times 10^{-4}$ m$^{3}$.kg$^{-1}$.}
    \end{itemize} 

    \noindent{\bf Solution:} 

    \visible<2->{{\bf(a)} The work done during the compression,
                \begin{displaymath}
                   w = -\int P dv,
                \end{displaymath}
                where $v$ is the specific volume.}\visible<3->{ $\kappa$ was defined in Module 2 as,
                \begin{displaymath}
                   \kappa = \frc{1}{v}\left(\frc{\partial v}{\partial P}\right)_{T}\;\Longrightarrow \; v\kappa dP = - dv \;\;\text{ (with constant T)}
                \end{displaymath}}
                For isothermal processes
                \begin{displaymath}
                   w = -\int P dv = - \int P\left(-v\kappa dP\right) = \frc{v}{2}\kappa\left(P_{2}^{2}-P_{1}^{2}\right) = 4.90 \frc{\text{J}}{\text{kg}}
                \end{displaymath}

\end{frame}

%%%
%%% Slide
%%%
%\scriptsize
\begin{frame}
   \frametitle{Example 1}

    \visible<1->{For isothermal processes
                \begin{displaymath}
                   w = -\int P dv = - \int P\left(-v\kappa dP\right) = \frc{v}{2}\kappa\left(P_{2}^{2}-P_{1}^{2}\right) = 4.90 \frc{\text{J}}{\text{kg}}
                \end{displaymath}}
    
   \visible<2->{{\bf (b)} $ds$ = ? (specific entropy).}

   \visible<3->{From the Maxwell relations, 
                \begin{displaymath}
                   -\left(\frc{\partial s}{\partial P}\right)_{T} = \left(\frc{\partial v}{\partial T}\right)_{P},
                \end{displaymath}} 

   \visible<4->{and from the definition of $\beta$,
                \begin{eqnarray}
                    && \beta = \frc{1}{v}\left(\frc{\partial v}{\partial T}\right)_{P} \;\;\Longrightarrow\;\; -\left(\frc{\partial s}{\partial P}\right)_{T} = \beta v \nonumber \\
                    && ds = -\beta v dP \;\;\Longrightarrow ds = s_{2}-s_{1} = -\beta v \left(P_{2}-P_{1}\right) = -0.5694 \frc{\text{J}}{\text{kg.K}} \nonumber
                \end{eqnarray}}

\end{frame}

%%%
%%% Slide
%%%
%\scriptsize
\begin{frame}
   \frametitle{Example 1}

    \visible<1->{{\bf (c)} The heat transferred in such reversible isothermal process is
                \begin{displaymath}
                   dq = Tds \;\;\Longrightarrow q = T\left(s_{2}-s_{1}\right) = -164.07 \frc{\text{J}}{\text{kg}}.
                \end{displaymath}}

    \visible<1->{{\bf (d)} The specific internal energy,
                \begin{eqnarray}
                   &&  du = q + w \nonumber \\
                   && \left(u_{2}-u_{1}\right) = \underbrace{-164.07}_{\text{heat removed from the system}} + \overbrace{4.90}^{\text{work given to the system}} = -159.17 \frc{\text{J}}{\text{kg}}. \nonumber
                \end{eqnarray}}

\end{frame}



%%%
%%% Slide
%%%
%\scriptsize
\begin{frame}
   \frametitle{Example 2}
    \blue{Demonstrate that the derivative of molar volume \wrt temperature at constant pressure is}
     \begin{displaymath}
         \blue{\Partial[V]{T}{P} = -\frc{\Partial[P]{T}{V}}{\Partial[P]{V}{T}},}
     \end{displaymath}
     \blue{and obtain an expression for $\Partial[V]{T}{P}$ for the van der Waals EOS.} \\

   \blue{ {\bf Hint:} You should start the proof from the total differential of a continuous function $f(a,b)$,}
     \begin{displaymath}
         \blue{df = \Partial[f]{a}{b}da + \Partial[f]{b}{a}db.}
     \end{displaymath}

\end{frame}


%%%
%%% Slide
%%%
%\scriptsize
\begin{frame}
   \frametitle{Example 2}

    \noindent{\bf Solution:} 
    \visible<1->{The total differential of a generic continuous function $f(a,b)$ is
       \begin{displaymath}
           df = \Partial[f]{a}{b}da + \Partial[f]{b}{a}db.
       \end{displaymath}
       where (from the given thermodynamic function) $f=P$, $a=T$ and $b=V$, \ie}

    \visible<2->{\begin{displaymath}
          dP = \Partial[P]{T}{V}dT + \Partial[P]{V}{T}dV.
       \end{displaymath}}

    \visible<3->{However we want a differential expression in which $P$ is constant, therefore} \visible<4->{$\mathbf{dP = 0}$,}
       \begin{eqnarray}
           \visible<5->{0 &=& \Partial[P]{T}{V}dT + \Partial[P]{V}{T}dV \;\;\;\text{ at } P \text{ constant},}\nonumber \\
           \visible<6->{\Partial[V]{T}{P} &=& -\frc{\Partial[P]{T}{V}}{\Partial[P]{V}{T}}.} \nonumber
       \end{eqnarray}

\end{frame}

%%%
%%% Slide
%%%
%\scriptsize
\begin{frame}
   \frametitle{Example 2}

    \visible<1->{Now we want to obtain a differential expression for the vdW-EOS,}
    \visible<2->{\begin{displaymath}
          P = \frc{RT}{V-b} - \frc{a}{V^{2}},
       \end{displaymath}
       where $V$ is the molar volume and $a$ and $b$ are constants that {\it depends only on critical properties}, $P_{c}$ and $T_{c}$.} 

    \visible<3->{Due to the non-linearity of this EOS, obtaining $\Partial[V]{T}{P}$ from a direct differentiation would be difficult. However, now we know an expression that cn help us,}

    \visible<4->{\begin{displaymath}
         \Partial[V]{T}{P} = -\frc{\Partial[P]{T}{V}}{\Partial[P]{V}{T}} = -\frc{\frc{R}{V-b}}{-\frc{RT}{\left(V-b\right)^{2}+\frc{2a}{V^{3}}}}
     \end{displaymath} }

\end{frame}


%%%
%%% Slide
%%%
%\scriptsize
\begin{frame}
   \frametitle{Example 3}
    \blue{Derive an expression for enthalpy change of a gas during an isothermal process assuming using the following EOS: $P\left(V-b\right)=RT$}

    \noindent\visible<2->{{\bf Solution:} We have seen that enthalpy change is given by (see Slide 8 or Eqn. 36a from the {\it Notes})
    \begin{displaymath}
       dH = C_{p}dT + \left[V - T\Partial[V]{T}{P}\right]dP.
    \end{displaymath}}

    \visible<3->{We can rearrange the EOS and obtain $\Partial[V]{T}{P}$, (or, for a more complex EOS, we could use the procedure of Exmaple 2 to obtain this differential)}
    \begin{eqnarray}
       \visible<3->{&& P\left(V-b\right)=RT \;\;\;\rightarrow\;\;\; V = \frc{RT}{P} + b \;\;\;\rightarrow\;\;\; \Partial[V]{T}{P} = \frc{R}{P}\;\;\text{ thus, }} \nonumber \\
       \visible<3->{&& dH = C_{p}dT + \left(V - \frc{RT}{P}\right)dP = \blue{C_{p}dT + bdP}.} \nonumber 
    \end{eqnarray}

\end{frame}

%%%
%%% Slide
%%%
%\scriptsize
\begin{frame}
   \frametitle{Example 4}
    \blue{The Antoine equation constants for toluene are $A=14.01415$, $B=3106.46$ K and $C=-53.15$ K (for pressure given in kPa). At 1.01325$\times$10$^{5}$ Pa, calculate:}
        \begin{enumerate}[(a)]
           \item \blue{the boiling temperature and;}
           \item \blue{the enthalpy of vaporisation at these conditions.}
        \end{enumerate}

    \noindent\visible<2->{{\bf Solution:} }
       \begin{enumerate}[a)]
%
           \item<2->Boiling temperature can be calculated from the Antoine equation,
               \begin{eqnarray}
                   \visible<3->{&& \ln{P^{\text{sat}}} = A - \frc{B}{T+C}} \nonumber \\
                   \visible<4->{&& T = \frc{B}{A-\ln{P^{\text{sat}}}} - C = \blue{383.77 \text{ K}}}  \nonumber          
               \end{eqnarray}
%
           \item<5-> The enthalpy of vaporisation, $\Delta H^{\text{fg}}$, can be obtained from the Clausius-Clapeyron equation,
               \begin{eqnarray}
                   && \visible<5->{\frc{d}{dT} \left(\ln{P^{\text{sat}}}\right) = \frc{\Delta H^{\text{fg}}}{RT^{2}}} \nonumber \\
                   && \visible<6->{\frc{B}{\left(T+C\right)^{2}} =  \frc{\Delta H^{\text{fg}}}{RT^{2}} \;\;\Longrightarrow \Delta H^{\text{fg}} = \blue{34.7984 \text{ kJ.mol}^{-1}}.} \nonumber
               \end{eqnarray}
%
        \end{enumerate}

\end{frame}

%%%
%%% Slide
%%%
%\scriptsize
\begin{frame}
   \frametitle{Example 5}
    \blue{Steam (dry and saturated) is supplied by the boiler at 15 bar and the condenser inlet pressure is 0.4 bar. Calculate the Rankine efficiency of the cycle. Neglect the pump work, assume the enthalpy of fluid leaving the pump is 317.58 kJ.kg$^{-1}$.}
    \noindent\visible<2->{{\bf Solution:} For this problem, let's assume the same numbering of Fig. 7a in the Lecture-Notes.}\visible<3->{ At 15 bar, dry and saturated $\left(\ie\; x_{1}=1\right)$ steam} \visible<4->{has the following properties (from saturated table),}
           \visible<4->{\begin{eqnarray}
             T_{1} &=& T_{\text{sat}} = 198.3^{\circ}\text{C},\nonumber \\
             h_{1} &=& h_{\text{g}} = 2792.2\; \text{kJ.kg}^{-1} \nonumber \\
             s_{1} &=& s_{\text{g}} = 6.4448\; \text{kJ.(kg.K)}^{-1} \nonumber
          \end{eqnarray} }
          \visible<5->{In the condenser, $P_{2}=0.4$ bar,
          \begin{eqnarray}
              T_{2} &=& T_{\text{sat}} = 75.87^{\circ}\text{C}, \nonumber \\
              h_{\text{g}2} &=& 2636.8\;\text{kJ.kg}^{-1},\;\;\; h_{\text{f}2} = 317.58\;\text{kJ.kg}^{-1},  \nonumber \\
              s_{\text{g}2} &=& 7.6700 \;\text{kJ.(kg.K)}^{-1},\;\;\; s_{\text{f}2} = 1.0259\;\text{kJ.(kg.K)}^{-1}. \nonumber  
          \end{eqnarray}}
\end{frame}

%%%
%%% Slide
%%%
%\scriptsize
\begin{frame}
   \frametitle{Example 5}
         \visible<1-> {$h_{2}$ and $s_{2}$ depend on the knowledge of how vaporised the water is, in other words, we need to determine the quality of the steam, $x_{1}$ through},
         \begin{displaymath}
              \visible<2->{M = \mfr[M]{}{L} + \mfr[x]{}{V}\Delta\mfr[M]{}{LV}}
              \visible<3->{ \begin{cases}
                      h_{2} = h_{\text{f}2} + x_{2}\left(h_{\text{g}2} - h_{\text{f}2}\right), \\
                      s_{2} = s_{\text{f}2} + x_{2}\left(s_{\text{g}2} - s_{\text{f}2}\right).  
              \end{cases}}
         \end{displaymath}

         \visible<4-> {As we know that water is expanded isentropically in the turbine, \ie \blue{$s_{1}=s_{2}$},}
         \begin{eqnarray}
             && \visible<5-> {s_{2} = s_{\text{f}2} + x_{2}\left(s_{\text{g}2} - s_{\text{f}2}\right) = s_{1} = 6.4448}  \nonumber \\
             && \visible<6-> {x_{2} = 0.8156 \;\;(81.56\% \text{ of vapour})}\nonumber
         \end{eqnarray}

         \visible<7-> {Thus replacing in 
             \begin{displaymath}
                h_{2} = h_{\text{f}2} + x_{2}\left(h_{\text{g}2} - h_{\text{f}2}\right) = 2209.14\text{ kJ.kg}^{-1}.
             \end{displaymath}}

         \visible<8-> {The Rankine efficiency is given by}
             \begin{displaymath}
                 \visible<8-> {\eta_{\text{Rankine}} = \frc{\text{Adiabatic or Isentropic Heat Drop}}{\text{Heat Supplied}} =} \visible<9-> {\frc{\left|h_{1}-h_{2}\right|}{h_{1}-h_{\text{f}4}} = 0.2356\;\;\;\rightarrow \;\;\; 23.56\%}
             \end{displaymath}
\end{frame}



\end{document}
 
