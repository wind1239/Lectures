% Aberdeen style guide should be followed when using this
% layout. Their template powerpoint slide is used to extract the
% Aberdeen color and logo but is otherwise ignored (it has little or
% no formatting in it anyway).
%
% http://www.abdn.ac.uk/documents/style-guide.pdf

%%%%%%%%%%%%%%%%%%%% Document Class Settings %%%%%%%%%%%%%%%%%%%%%%%%%
% Pick if you want slides, or draft slides (no animations)
%%%%%%%%%%%%%%%%%%%%%%%%%%%%%%%%%%%%%%%%%%%%%%%%%%%%%%%%%%%%%%%%%%%%%%
%Normal document mode%
\documentclass[10pt,compress]{beamer}
%Draft or handout mode
%\documentclass[10pt,compress,handout]{beamer}
%\documentclass[10pt,compress,handout,ignorenonframetext]{beamer}

\renewcommand{\insertframenumber}{\theframenumber}
\renewcommand{\theframenumber}{\thesection-\arabic{framenumber}}
\renewcommand{\thesubsectionslide}{\thesection-\arabic{framenumber}}
\setbeamertemplate{headline}[text line]{This is frame: \insertframenumber}
\AtBeginSection{\setcounter{framenumber}{0}}


%%%%%%%%%%%%%%%%%%%% General Document settings %%%%%%%%%%%%%%%%%%%%%%%
% These settings must be set for each presentation
%%%%%%%%%%%%%%%%%%%%%%%%%%%%%%%%%%%%%%%%%%%%%%%%%%%%%%%%%%%%%%%%%%%%%%
\newcommand{\shortname}{jefferson.gomes@abdn.ac.uk}
\newcommand{\fullname}{Dr Jeff Gomes}
\institute{School of Engineering}
\newcommand{\emailaddress}{}%jefferson.gomes@abdn.ac.uk}
\newcommand{\logoimage}{../../FigBanner/UoAHorizBanner}
\title{Chemical Thermodynamics (EX3029)}
\subtitle{Module 5: Solution Thermodynamics}
\date[ ]{ }

%%%%%%%%%%%%%%%%%%%% Template settings %%%%%%%%%%%%%%%%%%%%%%%%%%%%%%%
% You shouldn't have to change below this line, unless you want to.
%%%%%%%%%%%%%%%%%%%%%%%%%%%%%%%%%%%%%%%%%%%%%%%%%%%%%%%%%%%%%%%%%%%%%%
\usecolortheme{whale}
\useoutertheme{infolines}

% Use the fading effect for items that are covered on the current
% slide.
\beamertemplatetransparentcovered

% We abuse the author command to place all of the slide information on
% the title page.
\author[\shortname]{%
  \fullname\\\ttfamily{\emailaddress}
}


%At the start of every section, put a slide indicating the contents of the current section.
\AtBeginSection[] {
  \begin{frame}
    \frametitle{Section Outline}
    \tableofcontents[currentsection]
  \end{frame}
}

% Allow the inclusion of movies into the Presentation! At present,
% only the Okular program is capable of playing the movies *IN* the
% presentation.
\usepackage{multimedia}
\usepackage{animate}

%% Handsout -- comment out the lines below to create handstout with 4 slides in a page with space for comments
\usepackage{handoutWithNotes}

\mode<handout>
{
\usepackage{pgf,pgfpages}

\pgfpagesdeclarelayout{2 on 1 boxed with notes}
{
\edef\pgfpageoptionheight{\the\paperheight} 
\edef\pgfpageoptionwidth{\the\paperwidth}
\edef\pgfpageoptionborder{0pt}
}
{
\setkeys{pgfpagesuselayoutoption}{landscape}
\pgfpagesphysicalpageoptions
    {%
        logical pages=4,%
        physical height=\pgfpageoptionheight,%
        physical width=\pgfpageoptionwidth,%
        last logical shipout=2%
    } 
\pgfpageslogicalpageoptions{1}
    {%
    border code=\pgfsetlinewidth{1pt}\pgfstroke,%
    scale=1,
    center=\pgfpoint{.25\pgfphysicalwidth}{.75\pgfphysicalheight}%
    }%
\pgfpageslogicalpageoptions{2}
    {%
    border code=\pgfsetlinewidth{1pt}\pgfstroke,%
    scale=1,
    center=\pgfpoint{.25\pgfphysicalwidth}{.25\pgfphysicalheight}%
    }%
\pgfpageslogicalpageoptions{3}
    {%
    border shrink=\pgfpageoptionborder,%
    resized width=.7\pgfphysicalwidth,%
    resized height=.5\pgfphysicalheight,%
    center=\pgfpoint{.75\pgfphysicalwidth}{.29\pgfphysicalheight},%
    copy from=3
    }%
\pgfpageslogicalpageoptions{4}
    {%
    border shrink=\pgfpageoptionborder,%
    resized width=.7\pgfphysicalwidth,%
    resized height=.5\pgfphysicalheight,%
    center=\pgfpoint{.75\pgfphysicalwidth}{.79\pgfphysicalheight},%
    copy from=4
    }%

\AtBeginDocument
    {
    \newbox\notesbox
    \setbox\notesbox=\vbox
        {
            \hsize=\paperwidth
            \vskip-1in\hskip-1in\vbox
            {
                \vskip1cm
                Notes\vskip1cm
                        \hrule width\paperwidth\vskip1cm
                    \hrule width\paperwidth\vskip1cm
                        \hrule width\paperwidth\vskip1cm
                    \hrule width\paperwidth\vskip1cm
                        \hrule width\paperwidth\vskip1cm
                    \hrule width\paperwidth\vskip1cm
                    \hrule width\paperwidth\vskip1cm
                    \hrule width\paperwidth\vskip1cm
                        \hrule width\paperwidth
            }
        }
        \pgfpagesshipoutlogicalpage{3}\copy\notesbox
        \pgfpagesshipoutlogicalpage{4}\copy\notesbox
    }
}
}

%\pgfpagesuselayout{2 on 1 boxed with notes}[letterpaper,border shrink=5mm]
%\pgfpagesuselayout{2 on 1 boxed with notes}[letterpaper,border shrink=5mm]

%%%%% Color settings
\usepackage{color}
%% The background color for code listings (i.e. example programs)
\definecolor{lbcolor}{rgb}{0.9,0.9,0.9}%
\definecolor{UoARed}{rgb}{0.64706, 0.0, 0.12941}
\definecolor{UoALight}{rgb}{0.85, 0.85, 0.85}
\definecolor{UoALighter}{rgb}{0.92, 0.92, 0.92}
\setbeamercolor{structure}{fg=UoARed} % General background and higlight color
\setbeamercolor{frametitle}{bg=black} % General color
\setbeamercolor{frametitle right}{bg=black} % General color
\setbeamercolor{block body}{bg=UoALighter} % For blocks
\setbeamercolor{structure}{bg=UoALight} % For blocks
% Rounded boxes for blocks
\setbeamertemplate{blocks}[rounded]

%%%%% Font settings
% Aberdeen requires the use of Arial in slides. We can use the
% Helvetica font as its widely available like so
% \usepackage{helvet}
% \renewcommand{\familydefault}{\sfdefault}
% But beamer already uses a sans font, so we will stick with that.

% The size of the font used for the code listings.
\newcommand{\goodsize}{\fontsize{6}{7}\selectfont}

% Extra math packages, symbols and colors. If you're using Latex you
% must be using it for formatting the math!
\usepackage{amscd,amssymb} \usepackage{amsfonts}
\usepackage[mathscr]{eucal} \usepackage{mathrsfs}
\usepackage{latexsym} \usepackage{amsmath} \usepackage{bm}
\usepackage{amsthm} \usepackage{textcomp} \usepackage{eurosym}
% This package provides \cancel{a} and \cancelto{a}{b} to "cancel"
% expressions in math.
\usepackage{cancel}

\usepackage{comment} 

% Get rid of font warnings as modern LaTaX installations have scalable
% fonts
\usepackage{type1cm} 

%\usepackage{enumitem} % continuous numbering throughout enumerate commands

% For exact placement of images/text on the cover page
\usepackage[absolute]{textpos}
\setlength{\TPHorizModule}{1mm}%sets the textpos unit
\setlength{\TPVertModule}{\TPHorizModule} 

% Source code formatting package
\usepackage{listings}%
\lstset{ backgroundcolor=\color{lbcolor}, tabsize=4,
  numberstyle=\tiny, rulecolor=, language=C++, basicstyle=\goodsize,
  upquote=true, aboveskip={1.5\baselineskip}, columns=fixed,
  showstringspaces=false, extendedchars=true, breaklines=false,
  prebreak = \raisebox{0ex}[0ex][0ex]{\ensuremath{\hookleftarrow}},
  frame=single, showtabs=false, showspaces=false,
  showstringspaces=false, identifierstyle=\ttfamily,
  keywordstyle=\color[rgb]{0,0,1},
  commentstyle=\color[rgb]{0.133,0.545,0.133},
  stringstyle=\color[rgb]{0.627,0.126,0.941}}

% Allows the inclusion of other PDF's into the final PDF. Great for
% attaching tutorial sheets etc.
\usepackage{pdfpages}
\setbeamercolor{background canvas}{bg=}  

% Remove foot note horizontal rules, they occupy too much space on the slide
\renewcommand{\footnoterule}{}

% Force the driver to fix the colors on PDF's which include mixed
% colorspaces and transparency.
\pdfpageattr {/Group << /S /Transparency /I true /CS /DeviceRGB>>}

% Include a graphics, reserve space for it but
% show it on the next frame.
% Parameters:
% #1 Which slide you want it on
% #2 Previous slides
% #3 Options to \includegraphics (optional)
% #4 Name of graphic
\newcommand{\reserveandshow}[4]{%
\phantom{\includegraphics<#2|handout:0>[#3]{#4}}%
\includegraphics<#1>[#3]{#4}%
}

\newcommand{\frc}{\displaystyle\frac}
\newcommand{\red}{\textcolor{red}}
\newcommand{\blue}{\textcolor{blue}}
\newcommand{\green}{\textcolor{green}}
\newcommand{\purple}{\textcolor{purple}}
 
\begin{document}

% Title page layout
\begin{frame}
  \titlepage
  \vfill%
  \begin{center}
    \includegraphics[clip,width=0.8\textwidth]{\logoimage}
  \end{center}
\end{frame}

% Table of contents
\frame{ \frametitle{Slides Outline}
  \tableofcontents
}


%%%%%%%%%%%%%%%%%%%% The Presentation Proper %%%%%%%%%%%%%%%%%%%%%%%%%
% Fill below this line with \begin{frame} commands! It's best to
% always add the fragile option incase you're going to use the
% verbatim environment.
%%%%%%%%%%%%%%%%%%%%%%%%%%%%%%%%%%%%%%%%%%%%%%%%%%%%%%%%%%%%%%%%%%%%%%

%%%
%%% SECTION
%%%
\section{General Remarks}

%%%
%%% Slides
%%%
\begin{frame}
 \frametitle{Aims and Objectives}
   \begin{enumerate}
     \item<1-> In the previous Modules we have focused primarily on thermodynamic systems comprised of pure substances and mixtures in VLE;
     \item<1-> Last lecture we derived chemical potential and fugacity expressions for pure chemical species and mixtures of gases;
     \item<1-> From this, we also derived the activity and activity coefficient;
     \item<1-> This module will focus on thermodynamic description of solutions and how the aforementioned properties can be used to determine the \underline{properties of ideal and real liquid solutions}. 
   \end{enumerate}
\end{frame}


%%%
%%% SECTION
%%%
\subsection{Bibliography}
\begin{frame}
 \frametitle{Suggested References}
  Literature relevant for this module:
  \begin{enumerate}[(i)]
   \item\label{SVN_Book} J.M. Smith, H.C. Van Ness, M.M. Abbott, $\lq$Introduction to Chemical Engineering Thermodynamics', 6$^{th}$ Edition: Chapters 11 and 12;
   %\item Y.V.C. Rao, $\lq$Chemical Engineering Thermodynamics',4$^{th}$ Edition: Chapters 10 and 12.
   \item\label{Sandle_Book} S.I. Sandler, $\lq$Chemical, Biochemical and Engineering Thermodynamics', 4$^{th}$ Edition: Chapter 9.
  \end{enumerate}
\end{frame}


%%%
%%% SECTION
%%%
\section{Introduction}


%%%
%%% SUBSECTION
%%%
\subsection{Extending Thermodynamic Properties to Multi-Component Systems} 

%%%
%%% Slide
%%%
%\scriptsize
\begin{frame}
  \frametitle{Fundamental Property Relation}
  \begin{enumerate}
    \item<1-> Total Gibbs energy for closed system:
       \visible<1->{\begin{displaymath}
         d\left(n G\right) = \left(n V\right)dP - \left(n S\right)dT
       \end{displaymath}}
    \item<2-> General case of a single phase and open system:
       \visible<2->{\begin{displaymath}
         d\left(n G\right) = \left[\frc{\partial \left(n G\right)}{\partial P}\right]_{T,n}dP + \left[\frc{\partial \left(n G\right)}{\partial T}\right]_{P,n}dT + \left.\sum\limits_{i}\left[\frc{\partial \left(n G\right)}{\partial n_{i}}\right]_{P,T,n_{j}}dn_{i}\right.
       \end{displaymath}}
    \item<3-> Chemical Potential of Species {\it i}:
       \visible<3->{\begin{displaymath}
         \mu_{i} = \left[\frc{\partial\left(n G\right)}{\partial n_{i}}\right]_{P,T,n_{j}}  
       \end{displaymath}}
  \end{enumerate}
\end{frame}
\normalsize



%%%
%%% Slide
%%%
%\scriptsize
\begin{frame}
  \frametitle{Fundamental Property Relation}
  \begin{enumerate}\setcounter{enumi}{3}
    \item<1-> For single phase systems of variable mass and composition
       \visible<1->{\begin{displaymath}
         d\left(n G\right) = \left(n V\right)dP - \left(n S\right)dT + \sum\limits_{i}\mu_{i} dn_{i}
       \end{displaymath}}
    \item<2-> For {\bf one} mole of solution:
       \visible<2->{\begin{displaymath}
         V = \left(\frc{\partial G}{\partial P}\right)_{T,x} \;\;\;\text{ and }\;\;\; S = -\left(\frc{\partial G}{\partial T}\right)_{P,x} 
       \end{displaymath}}
    \item<3-> And the enthalpy can be expressed as,
       \visible<3->{\begin{displaymath}
          H = G + TS = G - T\left(\frc{\partial G}{\partial T}\right)_{P,x}
       \end{displaymath}}
  \end{enumerate}
\end{frame}
\normalsize

%%%
%%% SUBSECTION
%%%
\subsection{Chemical Potential}
%%%
%%% Slide
%%%
%\scriptsize
\begin{frame}
  \frametitle{Chemical Potential}
  \begin{enumerate}\setcounter{enumi}{6}
    \item<1-> From the last module, the stability criteria for \textcolor{blue}{phase equilibria} can be written as a function of the Gibbs energy if the system is held at constant $T$, $P$ and number of moles:
       \visible<1->{\begin{displaymath}
         d\left(n G\right)^{k} = \left(n V\right)^{k}dP - \left(n S\right)^{k}dT + \sum\limits_{i}\mu_{i}^{k} dn_{i}^{k}  \;\;\;\text{ with } k = 1,2,\cdots,\mathcal{P}
       \end{displaymath}
       where $k$ represents the phase.}
    \item<2-> Therefore
       \visible<2->{\begin{displaymath}
          \mu_{i}^{\alpha} = \mu_{i}^{\beta} = \cdots = \mu_{i}^{\mathcal{P}}
       \end{displaymath}}
  \end{enumerate}
  \visible<3->{\begin{block}{}
     \textcolor{blue}{\underline{Multiphase systems} at $T$ and $P$ are in equilibrium when the chemical potential $\left(\mu\right)$ of each species is the the same in {\bf all phases}.}
  \end{block}}
\end{frame}
\normalsize

%%%
%%% SECTION
%%%
\section{Partial Properties}
%%%
%%% SUBSECTION
%%%
\subsection{Introduction}
%%%
%%% Slide
%%%
%\scriptsize
\begin{frame}
  \frametitle{Partial Properties: Definition}
  \begin{enumerate}%\setcounter{enumi}{8}
    \item<1-> In a mixture the total value of any \textcolor{blue}{extensive property, $M^{t}$} $\left(M\equiv V,U, H, A, G\right)$ is not {\bf only} a function of $T$ and $P$, but also of the number of moles of each species present in the system.
    \item<2-> We can thus write in a functional form as $M^{t}=nM = M\left(T,P,n_{1}, n_{2}, \cdots n_{\mathcal{N}}\right)$, where $\mathcal{N}$ is the total number of chemical species in the system. $n_{i}$ is the number of moles of chemical species $i$ and $n\left(=\displaystyle\sum\limits_{i} n_{i}\right)$ is the total number of moles.
    \item<3-> With the total derivative of $M^{t}$,
        \visible<3->{\begin{equation}\label{totalderivative}
           d M^{t} = d\left(n M\right) = \left[\frc{\partial\left(nM\right)}{\partial P}\right]_{T,n}dP + \left[\frc{\partial\left(nM\right)}{\partial T}\right]_{P,n}dT +\textcolor{blue}{\left[\frc{\partial\left(nM\right)}{\partial n_{i}}\right]_{T,P,n_{j}}} dn_{i}
        \end{equation}}
     \item<4-> The last term in the rhs is called \textcolor{blue}{\it{partial molar property}, $\overline{M}_{i}=\left[\frc{\partial\left(nM\right)}{\partial n_{i}}\right]_{T,P,n_{j}}$}
  \end{enumerate}
\end{frame}
\normalsize

%%%
%%% Slide
%%%
%\scriptsize
\begin{frame}
  \frametitle{Partial Properties: Definition}
  \begin{enumerate}\setcounter{enumi}{4}
    \item<1-> The \textcolor{blue}{partial molar property} represents the change of total property $nM$ of a mixture resulting from addition (at constant $T$ and $P$) of a differential amount of species $i$ to a finite amount of solution.
    \item<2-> In general, partial molar properties of a chemical species differs from the molar property of the same species in a pure state at the same $T$ and $P$ as the mixture or solution. 
    \item<3-> This is because in a pure state the molecules interact with its own species, however,
    \item<4-> In a solution it may be subjected to different molecular interactions with other (dissimilar) molecules. 
  \end{enumerate}
\end{frame}
\normalsize

%%%
%%% SUBSECTION
%%%
\subsection{Fundamental Relations}
%%%
%%% Slide
%%%
%\scriptsize
\begin{frame}
  \frametitle{Partial Properties: Relations between Molar and Partial Molar Properties}
  \begin{enumerate}%\setcounter{enumi}{8}
    \item<1->For the total property $M$:
      \visible<1->{\begin{displaymath}
          n M = \sum\limits_{i} n_{i}\overline{M}_{i}\;\;\;\text{ and }\;\;\; M = \sum\limits_{i}x_{i}\overline{M}_{i}
      \end{displaymath}}
    \item<2->General expression for $dM$:
      \visible<2->{\begin{displaymath}
          dM = \sum\limits_{i}x_{i}d\overline{M}_{i} + \sum\limits_{i}\overline{M}_{i}dx_{i}
      \end{displaymath}}
  \end{enumerate}
  \visible<3->{\begin{block}{\textcolor{blue}{Gibbs-Duhem Equation}}
               \begin{displaymath}
                   \left(\frc{\partial M}{\partial P}\right)_{T,x} dP + \left(\frc{\pa