
%\documentclass[11pts,a4paper,amsmath,amssymb,floatfix]{article}%{report}%{book}
\documentclass[12pts,a4paper,amsmath,amssymb,floatfix]{article}%{report}%{book}
\usepackage{graphicx,wrapfig,pdfpages}% Include figure files
%\usepackage{dcolumn,enumerate}% Align table columns on decimal point
\usepackage{enumerate,enumitem}% Align table columns on decimal point
\usepackage{bm,dpfloat}% bold math
\usepackage[pdftex,bookmarks,colorlinks=true,urlcolor=rltblue,citecolor=blue]{hyperref}
\usepackage{amsfonts,amsmath,amssymb,stmaryrd,indentfirst}
\usepackage{times,psfrag}
\usepackage{natbib}
\usepackage{color}
\usepackage{units}
\usepackage{rotating}
\usepackage{multirow,mathtools}


\usepackage{pifont}
\usepackage{subfigure}
\usepackage{subeqnarray}
\usepackage{ifthen}

\usepackage{supertabular}
\usepackage{moreverb}
\usepackage{listings}
\usepackage{palatino}
%\usepackage{doi}
\usepackage{longtable}
\usepackage{float}
\usepackage{perpage}
\MakeSorted{figure}
%\usepackage{pdflscape}


%\usepackage{booktabs}
%\newcommand{\ra}[1]{\renewcommand{\arraystretch}{#1}}


\definecolor{rltblue}{rgb}{0,0,0.75}


%\usepackage{natbib}
\usepackage{fancyhdr} %%%%
\pagestyle{fancy}%%%%
% with this we ensure that the chapter and section
% headings are in lowercase
%%%%\renewcommand{\chaptermark}[1]{\markboth{#1}{}}
\renewcommand{\sectionmark}[1]{\markright{\thesection\ #1}}
\fancyhf{} %delete the current section for header and footer
\fancyhead[LE,RO]{\bfseries\thepage}
\fancyhead[LO]{\bfseries\rightmark}
\fancyhead[RE]{\bfseries\leftmark}
\renewcommand{\headrulewidth}{0.5pt}
% make space for the rule
\fancypagestyle{plain}{%
\fancyhead{} %get rid of the headers on plain pages
\renewcommand{\headrulewidth}{0pt} % and the line
}

\def\newblock{\hskip .11em plus .33em minus .07em}
\usepackage{color}

%\usepackage{makeidx}
%\makeindex

\setlength\textwidth      {16.cm}
\setlength\textheight     {22.6cm}
\setlength\oddsidemargin  {-0.3cm}
\setlength\evensidemargin {0.3cm}

\setlength\headheight{14.49998pt} 
\setlength\topmargin{0.0cm}
\setlength\headsep{1.cm}
\setlength\footskip{1.cm}
\setlength\parskip{0pt}
\setlength\parindent{0pt}


%%%
%%% Headers and Footers
\lhead[] {\text{\small{EX3029 -- Chemical Thermodynamics}}} 
\rhead[] {{\text{\small{Tutorial 06}}}}
%\chead[] {\text{\small{Session 2012/13}}} 
\lfoot[]{Dr Jeff Gomes}
%\cfoot[\thepage]{\thepage}
\rfoot[\text{\small{\thepage}}]{\thepage}
\renewcommand{\headrulewidth}{0.8pt}


%%%
%%% space between lines
%%%
\renewcommand{\baselinestretch}{1.5}

\newenvironment{VarDescription}[1]%
  {\begin{list}{}{\renewcommand{\makelabel}[1]{\textbf{##1:}\hfil}%
    \settowidth{\labelwidth}{\textbf{#1:}}%
    \setlength{\leftmargin}{\labelwidth}\addtolength{\leftmargin}{\labelsep}}}%
  {\end{list}}

%%%%%%%%%%%%%%%%%%%%%%%%%%%%%%%%%%%%%%%%%%%
%%%%%%                              %%%%%%%
%%%%%%      NOTATION SECTION        %%%%%%%
%%%%%%                              %%%%%%%
%%%%%%%%%%%%%%%%%%%%%%%%%%%%%%%%%%%%%%%%%%%

% Text abbreviations.
\newcommand{\ie}{{\em{i.e., }}}
\newcommand{\eg}{{\em{e.g., }}}
\newcommand{\cf}{{\em{cf., }}}
\newcommand{\wrt}{with respect to}
\newcommand{\lhs}{left hand side}
\newcommand{\rhs}{right hand side}
% Commands definining mathematical notation.

% This is for quantities which are physically vectors.
\renewcommand{\vec}[1]{{\mbox{\boldmath$#1$}}}
% Physical rank 2 tensors
\newcommand{\tensor}[1]{\overline{\overline{#1}}}
% This is for vectors formed of the value of a quantity at each node.
\newcommand{\dvec}[1]{\underline{#1}}
% This is for matrices in the discrete system.
\newcommand{\mat}[1]{\mathrm{#1}}


\DeclareMathOperator{\sgn}{sgn}
\newtheorem{thm}{Theorem}[section]
\newtheorem{lemma}[thm]{Lemma}

%\newcommand\qed{\hfill\mbox{$\Box$}}
\newcommand{\re}{{\mathrm{I}\hspace{-0.2em}\mathrm{R}}}
\newcommand{\inner}[2]{\langle#1,#2\rangle}
\renewcommand\leq{\leqslant}
\renewcommand\geq{\geqslant}
\renewcommand\le{\leqslant}
\renewcommand\ge{\geqslant}
\renewcommand\epsilon{\varepsilon}
\newcommand\eps{\varepsilon}
\renewcommand\phi{\varphi}
\newcommand{\bmF}{\vec{F}}
\newcommand{\bmphi}{\vec{\phi}}
\newcommand{\bmn}{\vec{n}}
\newcommand{\bmns}{{\textrm{\scriptsize{\boldmath $n$}}}}
\newcommand{\bmi}{\vec{i}}
\newcommand{\bmj}{\vec{j}}
\newcommand{\bmk}{\vec{k}}
\newcommand{\bmx}{\vec{x}}
\newcommand{\bmu}{\vec{u}}
\newcommand{\bmv}{\vec{v}}
\newcommand{\bmr}{\vec{r}}
\newcommand{\bma}{\vec{a}}
\newcommand{\bmg}{\vec{g}}
\newcommand{\bmU}{\vec{U}}
\newcommand{\bmI}{\vec{I}}
\newcommand{\bmq}{\vec{q}}
\newcommand{\bmT}{\vec{T}}
\newcommand{\bmM}{\vec{M}}
\newcommand{\bmtau}{\vec{\tau}}
\newcommand{\bmOmega}{\vec{\Omega}}
\newcommand{\pp}{\partial}
\newcommand{\kaptens}{\tensor{\kappa}}
\newcommand{\tautens}{\tensor{\tau}}
\newcommand{\sigtens}{\tensor{\sigma}}
\newcommand{\etens}{\tensor{\dot\epsilon}}
\newcommand{\ktens}{\tensor{k}}
\newcommand{\half}{{\textstyle \frac{1}{2}}}
\newcommand{\tote}{E}
\newcommand{\inte}{e}
\newcommand{\strt}{\dot\epsilon}
\newcommand{\modu}{|\bmu|}
% Derivatives
\renewcommand{\d}{\mathrm{d}}
\newcommand{\D}{\mathrm{D}}
\newcommand{\ddx}[2][x]{\frac{\d#2}{\d#1}}
\newcommand{\ddxx}[2][x]{\frac{\d^2#2}{\d#1^2}}
\newcommand{\ddt}[2][t]{\frac{\d#2}{\d#1}}
\newcommand{\ddtt}[2][t]{\frac{\d^2#2}{\d#1^2}}
\newcommand{\ppx}[2][x]{\frac{\partial#2}{\partial#1}}
\newcommand{\ppxx}[2][x]{\frac{\partial^2#2}{\partial#1^2}}
\newcommand{\ppt}[2][t]{\frac{\partial#2}{\partial#1}}
\newcommand{\pptt}[2][t]{\frac{\partial^2#2}{\partial#1^2}}
\newcommand{\DDx}[2][x]{\frac{\D#2}{\D#1}}
\newcommand{\DDxx}[2][x]{\frac{\D^2#2}{\D#1^2}}
\newcommand{\DDt}[2][t]{\frac{\D#2}{\D#1}}
\newcommand{\DDtt}[2][t]{\frac{\D^2#2}{\D#1^2}}
% Norms
\newcommand{\Ltwo}{\ensuremath{L_2} }
% Basis functions
\newcommand{\Qone}{\ensuremath{Q_1} }
\newcommand{\Qtwo}{\ensuremath{Q_2} }
\newcommand{\Qthree}{\ensuremath{Q_3} }
\newcommand{\QN}{\ensuremath{Q_N} }
\newcommand{\Pzero}{\ensuremath{P_0} }
\newcommand{\Pone}{\ensuremath{P_1} }
\newcommand{\Ptwo}{\ensuremath{P_2} }
\newcommand{\Pthree}{\ensuremath{P_3} }
\newcommand{\PN}{\ensuremath{P_N} }
\newcommand{\Poo}{\ensuremath{P_1P_1} }
\newcommand{\PoDGPt}{\ensuremath{P_{-1}P_2} }

\newcommand{\metric}{\tensor{M}}
\newcommand{\configureflag}[1]{\texttt{#1}}

% Units
\newcommand{\m}[1][]{\unit[#1]{m}}
\newcommand{\km}[1][]{\unit[#1]{km}}
\newcommand{\s}[1][]{\unit[#1]{s}}
\newcommand{\invs}[1][]{\unit[#1]{s}\ensuremath{^{-1}}}
\newcommand{\ms}[1][]{\unit[#1]{m\ensuremath{\,}s\ensuremath{^{-1}}}}
\newcommand{\mss}[1][]{\unit[#1]{m\ensuremath{\,}s\ensuremath{^{-2}}}}
\newcommand{\K}[1][]{\unit[#1]{K}}
\newcommand{\PSU}[1][]{\unit[#1]{PSU}}
\newcommand{\Pa}[1][]{\unit[#1]{Pa}}
\newcommand{\kg}[1][]{\unit[#1]{kg}}
\newcommand{\rads}[1][]{\unit[#1]{rad\ensuremath{\,}s\ensuremath{^{-1}}}}
\newcommand{\kgmm}[1][]{\unit[#1]{kg\ensuremath{\,}m\ensuremath{^{-2}}}}
\newcommand{\kgmmm}[1][]{\unit[#1]{kg\ensuremath{\,}m\ensuremath{^{-3}}}}
\newcommand{\Nmm}[1][]{\unit[#1]{N\ensuremath{\,}m\ensuremath{^{-2}}}}

% Dimensionless numbers
\newcommand{\dimensionless}[1]{\mathrm{#1}}
\renewcommand{\Re}{\dimensionless{Re}}
\newcommand{\Ro}{\dimensionless{Ro}}
\newcommand{\Fr}{\dimensionless{Fr}}
\newcommand{\Bu}{\dimensionless{Bu}}
\newcommand{\Ri}{\dimensionless{Ri}}
\renewcommand{\Pr}{\dimensionless{Pr}}
\newcommand{\Pe}{\dimensionless{Pe}}
\newcommand{\Ek}{\dimensionless{Ek}}
\newcommand{\Gr}{\dimensionless{Gr}}
\newcommand{\Ra}{\dimensionless{Ra}}
\newcommand{\Sh}{\dimensionless{Sh}}
\newcommand{\Sc}{\dimensionless{Sc}}


% Journals
\newcommand{\IJHMT}{{\it International Journal of Heat and Mass Transfer}}
\newcommand{\NED}{{\it Nuclear Engineering and Design}}
\newcommand{\ICHMT}{{\it International Communications in Heat and Mass Transfer}}
\newcommand{\NET}{{\it Nuclear Engineering and Technology}}
\newcommand{\HT}{{\it Heat Transfer}}   
\newcommand{\IJHT}{{\it International Journal for Heat Transfer}}

\newcommand{\frc}{\displaystyle\frac}

\newlist{ExList}{enumerate}{1}
\setlist[ExList,1]{label={\bf Example 1.} {\bf \arabic*}}

\newlist{ProbList}{enumerate}{1}
\setlist[ProbList,1]{label={\bf Problem 1.} {\bf \arabic*}}

%%%%%%%%%%%%%%%%%%%%%%%%%%%%%%%%%%%%%%%%%%%
%%%%%%                              %%%%%%%
%%%%%% END OF THE NOTATION SECTION  %%%%%%%
%%%%%%                              %%%%%%%
%%%%%%%%%%%%%%%%%%%%%%%%%%%%%%%%%%%%%%%%%%%


% Cause numbering of subsubsections. 
%\setcounter{secnumdepth}{8}
%\setcounter{tocdepth}{8}

\setcounter{secnumdepth}{4}%
\setcounter{tocdepth}{4}%


\begin{document}



\begin{enumerate}[label=\bfseries Problem \arabic*:]

%%%
%%% Johannes T08E01
%%%
\item\label{T08E01}Develop expressions for the mole fractions of reacting species as function of the reaction coordinate, $\varepsilon$, for:
\begin{enumerate}
\item A system initially containing 2 moles of NH$_{3}$ and 5 moles of O$_{2}$:
\begin{center}
$4 NH_{3} + 5 O_{2} \Longleftrightarrow 4 NO + 6 H_{2}O$
\end{center}
\item A system initially containing 3 moles of H$_{2}$S and 5 moles of O$_{2}$:
\begin{center}
$2 H_{2}S + 3 O_{2} \Longleftrightarrow 2 H_{2}O + 2 SO_{2}$
\end{center}
\item A system initially containing 3 moles of NO$_{2}$, 4 moles of NH$_{3}$ and 1 mole of N$_{2}$:
\begin{center}
$6 NO_{2} + 8 NH_{3} \Longleftrightarrow 7 N_{2} + 12 H_{2}O$
\end{center}
\end{enumerate}

%%%
%%% Johannes Lecture Example
%%%
\item\label{LectExample} The Fischer esterification reaction of acetic acid with ethanol,
\begin{displaymath}
CH_{3}COOH (l) + C_{2}H_{5}OH (l)  \xLeftrightarrow[]{H_{2}SO_{4} (l)} CH_{3}COOC_{2}H_{5} (l) + H_{2}O (l),
\end{displaymath}
is catalysed by a sulphuric acid solution  at 100$^{\circ}$C. Initially, 1 mole of acetic acid, 1 mol of ethanol and 1 mol of sulphuric acid (catalyst) are mixed. Calculate the equilibrium composition assuming that the reaction enthalpy and Gibbs energy at standard state (25$^{\circ}$C and 1 atm) are $\Delta H_{298}^{\circ}=-3640$ J.mol$^{-1}$ and $\Delta G_{298}^{\circ}=-4650$ J.mol$^{-1}$.

%%%
%%% Johannes T0802
%%%
\item\label{T0802} Give the equation for the stoichiometric combustion of 1 mole of general hydrocarbons, C$_{x}$H$_{y}$ with O$_{2}$ producing CO$_{2}$ and water. Develop expressions for gas phase mole fractions of each species as a function of the reaction coordinate. Assume that initially 1 mole of hydrocarbon and the stoichiometric amount of oxygen for complete conversion are present. 

%%%
%%% Johannes T0803
%%%
\item\label{T0803} In a reactor, 2 moles of CO$_{2}$, 5 moles of H$_{2}$ and 1 mole of CO are mixed and start to undergo the following reactions:
\begin{eqnarray}
CO_{2} + 3 H_{2} &\Longleftrightarrow& CH_{3}OH + H_{2}O \nonumber \\
CO_{2} + H_{2} &\Longleftrightarrow& CO + H_{2}O\nonumber
\end{eqnarray}
Develop expressions for the mole fractions of the reacting species as functions of the reaction coordinates for the two reactions.

%%%
%%% Johannes T0901
%%%
\item\label{T0901} For the ammonia synthesis reaction,
\begin{displaymath}
\frc{1}{2} N_{2}+ \frc{3}{2} H_{2} \Longleftrightarrow NH_{3}
\end{displaymath}
with 0.5 mole of N$_{2}$ and 1.5 moles of H$_{2}$ as the initial amounts of reactants and with the assumption that the equilibrium mixture is an ideal gas, show that
\begin{displaymath}
\varepsilon_{e} = 1 \pm \left( 1 + 1.299 K \frc{P}{P^{\circ}}\right)^{-0.5}
\end{displaymath}

%%%
%%% Johannes T0902
%%%
\item\label{T0902} Assuming that all species and their mixtures are ideal gases, derive an equation for the Gibbs energy as a function of the reaction coordinate for the reaction below at 1000K,
\begin{displaymath}
H_{2} + CO_{2} \Longleftrightarrow H_{2}O + CO
\end{displaymath}
Given $\Delta G_{f}^{\circ}$ $\left(\text{J.mol}^{-1}\right)$ at 1000K: (a) H$_{2}$O: -192420, (b) CO: -200240, (c) CO$_{2}$: -395790 and (d) H$_{2}$: 0.

%%%
%%% SM&VN 13.14
%%%
\item\label{T06} Ethylbenzene is produced through hydrogenation of styrene at 923.15 K and atmospheric pressure,
     \begin{displaymath}
          C_{6}H_{5}CH:CH_{2} (g) + H_{2} (g) \Longleftrightarrow C_{6}H_{5}CH_{2}CH_{3} (g).
     \end{displaymath}
     Initially, there are 1.5 moles of H$_{2}$ and 1 mol of styrene i the system. Calculate the equilibrium mol fraction assuming an ideal gas mixture, given,
         \begin{displaymath}
            C_{p} = a + bT + cT^{2} + dT^{3}
         \end{displaymath}
         with $[a]=\text{J.mol}^{-1}\text{.K}^{-1}$, $[b]=\text{J.mol}^{-1}\text{.K}^{-2}$, $[c]=\text{J.mol}^{-1}\text{.K}^{-3}$ and $[d]=\text{J.mol}^{-1}\text{.K}^{-4}$, and 
         \begin{center}
            \begin{tabular}{c|c c c c c c}
                \hline
                {\it Species}    & $a$      &  $b\times 10^{-2}$  & $c\times 10^{-5}$ & $d\times 10^{-9}$  & $\Delta H_{f,298}^{\circ}$ & $\Delta G_{f,298}^{\circ}$ \\
                \hline
                Styrene          & -24.971  &   60.059           & -38.285         &  92.176          & 147.36                                                & 213.95                \\
                Hydrogen         &  29.088  &    0.192           &   0.400         &  -0.870          &   0.00                                                &   0.00                \\
                Ethylbenzene     & -35.138  &   66.674           & -41.854         & 100.209          &  29.92                                                & 130.89                \\
                \hline
            \end{tabular}
         \end{center}
         with $\left[\Delta H_{f,298}^{\circ}\right]=\left[\Delta G_{f,298}^{\circ}\right] = \text{ kJ.mol}^{-1}$.

%{
%\includepdf[pages=-,fitpaper, angle=0]{./HuntSelect.pdf}
%}
\end{enumerate} 

\end{document}
