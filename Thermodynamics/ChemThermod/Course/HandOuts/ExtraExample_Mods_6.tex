% Aberdeen style guide should be followed when using this
% layout. Their template powerpoint slide is used to extract the
% Aberdeen color and logo but is otherwise ignored (it has little or
% no formatting in it anyway).
%
% http://www.abdn.ac.uk/documents/style-guide.pdf

%%%%%%%%%%%%%%%%%%%% Document Class Settings %%%%%%%%%%%%%%%%%%%%%%%%%
% Pick if you want slides, or draft slides (no animations)
%%%%%%%%%%%%%%%%%%%%%%%%%%%%%%%%%%%%%%%%%%%%%%%%%%%%%%%%%%%%%%%%%%%%%%
%Normal document mode%
\documentclass[10pt,compress]{beamer}
%Draft or handout mode
%\documentclass[10pt,compress,handout]{beamer}
%\documentclass[10pt,compress,handout,ignorenonframetext]{beamer}

\renewcommand{\insertframenumber}{\theframenumber}
\renewcommand{\theframenumber}{\thesection-\arabic{framenumber}}
\renewcommand{\thesubsectionslide}{\thesection-\arabic{framenumber}}
\setbeamertemplate{headline}[text line]{This is frame: \insertframenumber}
\AtBeginSection{\setcounter{framenumber}{0}}


%%%%%%%%%%%%%%%%%%%% General Document settings %%%%%%%%%%%%%%%%%%%%%%%
% These settings must be set for each presentation
%%%%%%%%%%%%%%%%%%%%%%%%%%%%%%%%%%%%%%%%%%%%%%%%%%%%%%%%%%%%%%%%%%%%%%
\newcommand{\shortname}{jefferson.gomes@abdn.ac.uk}
\newcommand{\fullname}{Dr Jeff Gomes}
\institute{School of Engineering}
\newcommand{\emailaddress}{}%jefferson.gomes@abdn.ac.uk}
\newcommand{\logoimage}{../../FigBanner/UoAHorizBanner}
\title{Chemical Thermodynamics (EX3029)}
\subtitle{Module 6: Chemical Reaction Equilibrium}
\date[ ]{ }

%%%%%%%%%%%%%%%%%%%% Template settings %%%%%%%%%%%%%%%%%%%%%%%%%%%%%%%
% You shouldn't have to change below this line, unless you want to.
%%%%%%%%%%%%%%%%%%%%%%%%%%%%%%%%%%%%%%%%%%%%%%%%%%%%%%%%%%%%%%%%%%%%%%
\usecolortheme{whale}
\useoutertheme{infolines}

% Use the fading effect for items that are covered on the current
% slide.
\beamertemplatetransparentcovered

% We abuse the author command to place all of the slide information on
% the title page.
\author[\shortname]{%
  \fullname\\\ttfamily{\emailaddress}
}


%At the start of every section, put a slide indicating the contents of the current section.
\AtBeginSection[] {
  \begin{frame}
    \frametitle{Section Outline}
    \tableofcontents[currentsection]
  \end{frame}
}

% Allow the inclusion of movies into the Presentation! At present,
% only the Okular program is capable of playing the movies *IN* the
% presentation.
\usepackage{multimedia}
\usepackage{animate}

%% Handsout -- comment out the lines below to create handstout with 4 slides in a page with space for comments
\usepackage{handoutWithNotes}

\mode<handout>
{
\usepackage{pgf,pgfpages}

\pgfpagesdeclarelayout{2 on 1 boxed with notes}
{
\edef\pgfpageoptionheight{\the\paperheight} 
\edef\pgfpageoptionwidth{\the\paperwidth}
\edef\pgfpageoptionborder{0pt}
}
{
\setkeys{pgfpagesuselayoutoption}{landscape}
\pgfpagesphysicalpageoptions
    {%
        logical pages=4,%
        physical height=\pgfpageoptionheight,%
        physical width=\pgfpageoptionwidth,%
        last logical shipout=2%
    } 
\pgfpageslogicalpageoptions{1}
    {%
    border code=\pgfsetlinewidth{1pt}\pgfstroke,%
    scale=1,
    center=\pgfpoint{.25\pgfphysicalwidth}{.75\pgfphysicalheight}%
    }%
\pgfpageslogicalpageoptions{2}
    {%
    border code=\pgfsetlinewidth{1pt}\pgfstroke,%
    scale=1,
    center=\pgfpoint{.25\pgfphysicalwidth}{.25\pgfphysicalheight}%
    }%
\pgfpageslogicalpageoptions{3}
    {%
    border shrink=\pgfpageoptionborder,%
    resized width=.7\pgfphysicalwidth,%
    resized height=.5\pgfphysicalheight,%
    center=\pgfpoint{.75\pgfphysicalwidth}{.29\pgfphysicalheight},%
    copy from=3
    }%
\pgfpageslogicalpageoptions{4}
    {%
    border shrink=\pgfpageoptionborder,%
    resized width=.7\pgfphysicalwidth,%
    resized height=.5\pgfphysicalheight,%
    center=\pgfpoint{.75\pgfphysicalwidth}{.79\pgfphysicalheight},%
    copy from=4
    }%

\AtBeginDocument
    {
    \newbox\notesbox
    \setbox\notesbox=\vbox
        {
            \hsize=\paperwidth
            \vskip-1in\hskip-1in\vbox
            {
                \vskip1cm
                Notes\vskip1cm
                        \hrule width\paperwidth\vskip1cm
                    \hrule width\paperwidth\vskip1cm
                        \hrule width\paperwidth\vskip1cm
                    \hrule width\paperwidth\vskip1cm
                        \hrule width\paperwidth\vskip1cm
                    \hrule width\paperwidth\vskip1cm
                    \hrule width\paperwidth\vskip1cm
                    \hrule width\paperwidth\vskip1cm
                        \hrule width\paperwidth
            }
        }
        \pgfpagesshipoutlogicalpage{3}\copy\notesbox
        \pgfpagesshipoutlogicalpage{4}\copy\notesbox
    }
}
}

%\pgfpagesuselayout{2 on 1 boxed with notes}[letterpaper,border shrink=5mm]
%\pgfpagesuselayout{2 on 1 boxed with notes}[letterpaper,border shrink=5mm]


%%%%%%%%%% Chemical Reactions %%%%%%%%%%%%%%%%

\usepackage[T1]{fontenc}
\usepackage[utf8]{inputenc}
\usepackage{lmodern}
\usepackage[version=3]{mhchem}
\makeatletter
\newcounter{reaction}
%%% >> for article <<
%\renewcommand\thereaction{C\,\arabic{reaction}}
%%% << for article <<
%%% >> for report and book >>
%\renewcommand\thereaction{C\,\thechapter.\arabic{reaction}}
%\@addtoreset{reaction}{chapter}
%%% << for report and book <<
\newcommand\reactiontag{\refstepcounter{reaction}\tag{\thereaction}}
\newcommand\reaction@[2][]{\begin{equation}\ce{#2}%
\ifx\@empty#1\@empty\else\label{#1}\fi%
\reactiontag\end{equation}}
\newcommand\reaction@nonumber[1]{\begin{equation*}\ce{#1}%
\end{equation*}}
\newcommand\reaction{\@ifstar{\reaction@nonumber}{\reaction@}}
\makeatother

%%%%%%%%%%%%%%%%%%%%%%%%%%%%%%%%%%%%%%%%%%%%%%


%%%%% Color settings
\usepackage{color}
%% The background color for code listings (i.e. example programs)
\definecolor{lbcolor}{rgb}{0.9,0.9,0.9}%
\definecolor{UoARed}{rgb}{0.64706, 0.0, 0.12941}
\definecolor{UoALight}{rgb}{0.85, 0.85, 0.85}
\definecolor{UoALighter}{rgb}{0.92, 0.92, 0.92}
\setbeamercolor{structure}{fg=UoARed} % General background and higlight color
\setbeamercolor{frametitle}{bg=black} % General color
\setbeamercolor{frametitle right}{bg=black} % General color
\setbeamercolor{block body}{bg=UoALighter} % For blocks
\setbeamercolor{structure}{bg=UoALight} % For blocks
% Rounded boxes for blocks
\setbeamertemplate{blocks}[rounded]

%%%%% Font settings
% Aberdeen requires the use of Arial in slides. We can use the
% Helvetica font as its widely available like so
% \usepackage{helvet}
% \renewcommand{\familydefault}{\sfdefault}
% But beamer already uses a sans font, so we will stick with that.

% The size of the font used for the code listings.
\newcommand{\goodsize}{\fontsize{6}{7}\selectfont}

% Extra math packages, symbols and colors. If you're using Latex you
% must be using it for formatting the math!
\usepackage{amscd,amssymb} \usepackage{amsfonts}
\usepackage[mathscr]{eucal} \usepackage{mathrsfs}
\usepackage{latexsym} \usepackage{amsmath} \usepackage{bm}
\usepackage{amsthm} \usepackage{textcomp} \usepackage{eurosym}
% This package provides \cancel{a} and \cancelto{a}{b} to "cancel"
% expressions in math.
\usepackage{cancel}

\usepackage{comment} 

% Get rid of font warnings as modern LaTaX installations have scalable
% fonts
\usepackage{type1cm} 

%\usepackage{enumitem} % continuous numbering throughout enumerate commands

% For exact placement of images/text on the cover page
\usepackage[absolute]{textpos}
\setlength{\TPHorizModule}{1mm}%sets the textpos unit
\setlength{\TPVertModule}{\TPHorizModule} 

% Source code formatting package
\usepackage{listings}%
\lstset{ backgroundcolor=\color{lbcolor}, tabsize=4,
  numberstyle=\tiny, rulecolor=, language=C++, basicstyle=\goodsize,
  upquote=true, aboveskip={1.5\baselineskip}, columns=fixed,
  showstringspaces=false, extendedchars=true, breaklines=false,
  prebreak = \raisebox{0ex}[0ex][0ex]{\ensuremath{\hookleftarrow}},
  frame=single, showtabs=false, showspaces=false,
  showstringspaces=false, identifierstyle=\ttfamily,
  keywordstyle=\color[rgb]{0,0,1},
  commentstyle=\color[rgb]{0.133,0.545,0.133},
  stringstyle=\color[rgb]{0.627,0.126,0.941}}

% Allows the inclusion of other PDF's into the final PDF. Great for
% attaching tutorial sheets etc.
\usepackage{pdfpages}
\setbeamercolor{background canvas}{bg=}  

% Remove foot note horizontal rules, they occupy too much space on the slide
\renewcommand{\footnoterule}{}

% Force the driver to fix the colors on PDF's which include mixed
% colorspaces and transparency.
\pdfpageattr {/Group << /S /Transparency /I true /CS /DeviceRGB>>}

% Include a graphics, reserve space for it but
% show it on the next frame.
% Parameters:
% #1 Which slide you want it on
% #2 Previous slides
% #3 Options to \includegraphics (optional)
% #4 Name of graphic
\newcommand{\reserveandshow}[4]{%
\phantom{\includegraphics<#2|handout:0>[#3]{#4}}%
\includegraphics<#1>[#3]{#4}%
}

\newcommand{\frc}{\displaystyle\frac}
\newcommand{\red}{\textcolor{red}}
\newcommand{\blue}{\textcolor{blue}}
\newcommand{\green}{\textcolor{green}}
\newcommand{\purple}{\textcolor{purple}}
 
\begin{document}


%%%%%%%%%%%%%%%%%%%% The Presentation Proper %%%%%%%%%%%%%%%%%%%%%%%%%
% Fill below this line with \begin{frame} commands! It's best to
% always add the fragile option incase you're going to use the
% verbatim environment.
%%%%%%%%%%%%%%%%%%%%%%%%%%%%%%%%%%%%%%%%%%%%%%%%%%%%%%%%%%%%%%%%%%%%%%

%%%
%%%
%%% Slides
%%%
\begin{frame}
 \frametitle{Example 1}\scriptsize
Five moles of hydrogen, two moles of CO and 1.5 moles of CH$_{3}$OH vapour are combined in a closed system methanol synthesis reactor at 500 K and 1 MPa. Develop expressions for the mole fractions of the species in terms of the reaction coordinate. The components are known to reacto with the following stoichiometry:
  \begin{displaymath}
      2 H_{2} (g) + CO (g) \Longleftrightarrow CH_{3}OH (g).
  \end{displaymath}

\visible<1->{{\bf Solution} }
\begin{enumerate}
      \item<2-> We need to develop expressions for each species in the gaseous form. In the equilibrium, the molar stoichiometric coefficient and the initial number of moles are:
         \visible<2->{\begin{displaymath}
            \nu = \sum\limits_{i}\nu_{i}= (-2)+(-1)+(+1)= -2 \;\;\text{ and }\;\; n_{0} = \sum\limits_{i}n_{i,0}= 5 + 2 + 1.5 = 8.5,
         \end{displaymath} 
         respectively.}
      \item<3-> The mole fraction of each species is given by
         \visible<3->{\begin{displaymath}
           y_{i} = \frc{n_{i}}{n} = \frc{n_{i,0}+\nu_{i}\epsilon}{n_{0}+\nu\epsilon}.
         \end{displaymath}
         with the total number of moles in equilibrium given by $n=8.5-2\epsilon$.}
         \begin{displaymath}
           \visible<4->{y_{\text{H}_{2}} = \frc{5-2\epsilon}{8.5-2\epsilon},}\;\;\; \visible<5->{y_{\text{CO}} = \frc{2-\epsilon}{8.5-2\epsilon}\;\;\text{ and }}\;\; \visible<6->{y_{\text{CH}_{3}\text{OH}}=\frc{1.5+\epsilon}{8.5-2\epsilon}}
         \end{displaymath}
   \end{enumerate}



\end{frame}


%%%
%%%
%%% Slides
%%%
\begin{frame}
 \frametitle{Example 2}\scriptsize 
Ethylene is produced from the decomposition of ethane,
       \begin{displaymath}
          C_{2}H_{6} (g) \Longleftrightarrow C_{2}H_{4} (g) + H_{2} (g) 
       \end{displaymath} 
       Determine the equilibrium composition at 1000$^{\circ}$C and 1 bar. Assume that, initially, there is 1 mol of ethane. Given,
       \begin{center}
           \begin{tabular}{|c c c c|}
           \hline
                                        &  C$_{2}$H$_{6}$ (g) & C$_{2}$H$_{4}$ (g) &  H$_{2}$ (g)  \\ 
           \hline
             $\Delta G^{0}_{\text{f,298}}$  &  -32.84            &  68.15           & 0.0           \\
                   (kJ/mol)             &                    &                  &               \\
           \hline
             $\Delta H^{0}_{\text{f,298}}$  &  -84.68            &  52.26           & 0.0           \\
                   (kJ/mol)             &                    &                  &               \\
           \hline 
           \end{tabular}
       \end{center}
       where $\Delta G^{0}_{\text{f,298}}$ and $\Delta H^{0}_{\text{f,298}}$ are the standard state Gibbs energy and enthalpy of formation. 

\visible<2->{{\bf Solution:} }
\begin{enumerate}\scriptsize 
      \item<2->{ First we need to determine the standard state Gibbs energy of reaction,}
         \visible<2->{\begin{eqnarray}
             \Delta G^{0}_{\text{r,298}} &=& \sum\limits_{i}\nu_{i}\left(\Delta G^{0}_{\text{f,298}}\right)_{i} \nonumber \\
                                     &=& (+1).(68.15)+ (1).(0.0) + (-1).(-32.84) = 100.99\; \text{kJ/mol} \nonumber
         \end{eqnarray}}
      \item<3-> The equilibrium constant can be calculated from,
         \visible<3->{\begin{displaymath}
            K = \exp\left(-\frc{\Delta G^{0}_{\text{r,298}}}{RT}\right) = exp\left(-\frc{100.99\times 10^{3}}{(8.314).(298.15)}\right) = 2.0246\times 10^{-18}
         \end{displaymath}}
   \end{enumerate} 

\end{frame}


%%%
%%%
%%% Slides
%%%
\begin{frame}
 \frametitle{Example 2}
\begin{enumerate}\scriptsize \setcounter{enumi}{2}
      \item<1-> The Van't Hoff equation can be used to calculate the equilibrium constant at temperature $T$,
         \visible<1->{\begin{displaymath}
            \frc{\partial\left(\ln K\right)}{\partial T} = \frc{\Delta H^{0}_{\text{r}}}{RT^{2}}
         \end{displaymath}}
      \item<2-> Before we can integrate the Van't Hoff equation, we first need to determine the standard heat of reaction, $\Delta H^{0}_{\text{r}}$,
         \visible<2->{\begin{eqnarray}
            \Delta H^{0}_{\text{r}} &=& \sum\limits_{i} \nu_{i}\left(\Delta H^{0}_{\text{f,298}}\right)_{i} \nonumber \\
                                 &=& (+1).(52.26) + (+1).(0.0) + (-1).(-84.68) = 136.94\;\text{kJ/mol}\nonumber
         \end{eqnarray}}

      \item<3-> Now integrating the Van't Hoff equation from 298.15 K to 1273.15 K,
         \visible<3->{\begin{eqnarray}
            \ln\frc{K_{1273}}{K_{298}} &=& -\frc{\Delta H^{0}_{\text{r}}}{R}\left(\frc{1}{T}-\frc{1}{298.15}\right) \nonumber \\
            \ln\frc{K_{1273}}{2.0246\times 10^{-18}} &=& -\frc{136.94\times 10^{3}}{8.314}\left(\frc{1}{1273.15}-\frc{1}{298.15}\right) \nonumber \\
            K_{1273} &=& 4.7859 \nonumber
         \end{eqnarray}}

\end{enumerate} 
\end{frame}

%%%
%%%
%%% Slides
%%%
\begin{frame}
 \frametitle{Example 2}
\begin{enumerate}\scriptsize \setcounter{enumi}{5}
      \item<1-> The equilibrium constant $K$ can also be expressed as a function of the components' activities,
         \visible<1->{\begin{displaymath}
            K = \frc{a_{\text{C}_{2}\text{H}_{4}}a_{\text{H}_{2}}}{a_{\text{C}_{2}\text{H}_{6}}} = \frc{\left(\frc{\overline{f}_{\text{C}_{2}\text{H}_{4}}}{f^{0}_{\text{C}_{2}\text{H}_{4}}}\right)\left(\frc{\overline{f}_{\text{H}_{2}}}{f^{0}_{\text{H}_{2}}}\right)}{\left(\frc{\overline{f}_{\text{C}_{2}\text{H}_{6}}}{f^{0}_{\text{C}_{2}\text{H}_{6}}}\right)}
         \end{displaymath}}
         \visible<2->{Assuming ideal gas behaviour, $\overline{f}_{i}=P_{i}$, $f_{i}^{0}=P^{0}_{\text{C}_{2}\text{H}_{6}}=P^{0}_{\text{C}_{2}\text{H}_{4}}=P^{0}_{\text{H}_{2}}=$ 1 bar. Thus,
         \begin{displaymath}
            K = \frc{a_{\text{C}_{2}\text{H}_{4}}a_{\text{H}_{2}}}{a_{\text{C}_{2}\text{H}_{6}}} = \frc{\left(\frc{\overline{f}_{\text{C}_{2}\text{H}_{4}}}{f^{0}_{\text{C}_{2}\text{H}_{4}}}\right)\left(\frc{\overline{f}_{\text{H}_{2}}}{f^{0}_{\text{H}_{2}}}\right)}{\left(\frc{\overline{f}_{\text{C}_{2}\text{H}_{6}}}{f^{0}_{\text{C}_{2}\text{H}_{6}}}\right)} = \frc{\left(\frc{P_{\text{C}_{2}\text{H}_{4}}}{P^{0}_{\text{C}_{2}\text{H}_{4}}}\right)\left(\frc{P_{\text{H}_{2}}}{P^{0}_{\text{H}_{2}}}\right)}{\left(\frc{P_{\text{C}_{2}\text{H}_{6}}}{P^{0}_{\text{C}_{2}\text{H}_{6}}}\right)} = \frc{\left(\frc{y_{\text{C}_{2}\text{H}_{4}}P}{1\text{ bar}}\right)\left(\frc{y_{\text{H}_{2}}P}{1\text{ bar}}\right)}{\left(\frc{y_{\text{C}_{2}\text{H}_{6}}P}{1\text{ bar}}\right)}          
         \end{displaymath}}
         \visible<3->{Since $P=$ 1 bar,
         \begin{displaymath}
            K = \frc{\left(\frc{y_{\text{C}_{2}\text{H}_{4}}P}{1\text{ bar}}\right)\left(\frc{y_{\text{H}_{2}}P}{1\text{ bar}}\right)}{\left(\frc{y_{\text{C}_{2}\text{H}_{6}}P}{1\text{ bar}}\right)} = \frc{y_{\text{C}_{2}\text{H}_{4}} y_{\text{H}_{2}}}{y_{\text{C}_{2}\text{H}_{6}}} = 4.7859
         \end{displaymath}} 


\end{enumerate} 
\end{frame}

%%%
%%%
%%% Slides
%%%
\begin{frame}
 \frametitle{Example 2}
\begin{enumerate}\scriptsize \setcounter{enumi}{6}
      \item<1-> The composition of each species in equilibrium is given by,
         \begin{displaymath}
           y_{i} = \frc{n_{i}}{n} = \frc{n_{i,0}+\nu_{i}\epsilon}{n_{0}+\nu\epsilon}.
         \end{displaymath}
         with 
         \begin{displaymath}
            \nu = \sum\limits_{i}\nu_{i}= (-1)+(+1)+(+1)= 1 \;\;\text{ and }\;\; n_{0} = \sum\limits_{i}n_{i,0}= 1 + 0 + 0 = 1,
         \end{displaymath}
      \item<2-> Thus,
         \visible<2->{\begin{displaymath}
             y_{\text{C}_{2}\text{H}_{6}} = \frc{1-\epsilon}{1+\epsilon},\;\;y_{\text{C}_{2}\text{H}_{4}} = \frc{\epsilon}{1+\epsilon}\;\text{ and }\;y_{\text{H}_{2}} = \frc{\epsilon}{1+\epsilon}
         \end{displaymath}}
      \item<3-> Replacing the compositions in the expression for $K$,
         \begin{eqnarray}
            K  &=& \frc{y_{\text{C}_{2}\text{H}_{4}} y_{\text{H}_{2}}}{y_{\text{C}_{2}\text{H}_{6}}} = 4.7859 \nonumber \\
               &&  \frc{\left(\frc{\epsilon}{1+\epsilon}\right)^{2}}{\left(\frc{1-\epsilon}{1+\epsilon}\right)} =4.7859 \nonumber \\
            &&\epsilon = 0.9095 \nonumber
         \end{eqnarray}
        The equilibrium concentration of C$_{2}$H$_{4}$(g) is $y_{\text{C}_{2}\text{H}_{4}} =$ 0.4763.

\end{enumerate} 

\end{frame}

%%%
%%%
%%% Slides
%%%
\begin{frame}
 \frametitle{Example 4}
The experimental value of the partial molar volume $\left(\text{cm}^{3}/\text{gmol}\right)$ of a aqueous solution of K$_{2}$SO$_{4}$ is given by
                \begin{displaymath}
                   \overline{V}_{A} = 32.280 + 18.216 m^{1/2},
                \end{displaymath} 
 where $m$ is the molality (= number of moles per kg of water) of the K$_{2}$SO$_{4}$ . Use the Gibbs-Duhem equation to derive an equation for the partial molar volume of water in the solution. Plot $\overline{V}_{i}\;\;\times m$, with $0.0\leq m\leq 0.1$. The molar volume of pure water at 298.15 K is 18.079 cm$^{3}$/gmol and the molar mass of pure water is 18 kg/kgmol.

\end{frame}


\end{document}
 
