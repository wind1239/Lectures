
\documentclass[calculator,refrigeranttables,steamtables,datasheet]{exam}
%\usepackage[colorlinks=true,urlcolor=rltblue,citecolor=blue]{hyperref}
% The full list of class options are
% calculator : Allows approved calculator use.
% datasheet : Adds a note that data sheet are attached to the exam.
% handbook : Allows the use of the engineering handbook.
% resit : Adds the resit markings to the paper.
% sample : Adds conspicuous SAMPLE markings to the paper
% solutions : Uses the contents of \solution commands (and \solmarks) to generate a solution file

\coursecode{EG5066}%
\coursetitle{Energy Technologies: Current Issues and Future Directions}%
\examtime{09.00--12.00}%
\examdate{01}{01}{2014}%
\examformat{Candidates must attempt 3 out of 4 questions.}

\newcommand{\frc}{\displaystyle\frac}

\begin{document}

%%%
%%% Question 01
%%%
\begin{question}

\begin{enumerate}[(a)]
\item Boiling water reactors (BWR) and pressurised water reactors (PWR) are the most common nuclear reactor types used to produce electricity worldwide. Describe the main differences between them.
\begin{flushright}
{\bf [05 Marks]}
\end{flushright}
\medskip

{\it The BWR has a {\bf single loop of primary coolant}, i.e., the water boiled in the reactor is the same steam that is passed through the turbines, cooled, then recirculated back into the reactor. The PWR contains {\bf 2 coolant loops}: the first water loop remains in the reactor with the water being boiled through fission and is then used to create steam (in a steam generator) from a secondary loop.  The secondary loop moves the turbines, and is then cooled and sent back to the steam generator. }

\medskip


\item Describe the stages of the uranium cycle, i.e., from mining to disposal. 
\begin{flushright}
{\bf [05 Marks]}
\end{flushright}

\medskip
{\it
\begin{itemize}
\item Uranium mining and milling: uranium ore is extracted from mines and separated from other metals;
\item Uranium conversion: the uranium is chemically purified and converted into uranium hexafluoride (UF$_{6}$);
\item Uranium enrichment: natural uranium ($\sim$0.7$\%$ of U$^{235}$) is converted into enriched uranium with 3-5$\%$ of U$^{235}$;
\item Fuel fabrication: enriched uranium is converted into uranium dioxide and allocated in the nuclear fuels;
\item Reactor and services: the U$^{235}$ fuel is loaded into the reactor and during the operation (i.e., nuclear fission), heat is produced (whilst the uranium is decayed) and converted into electricity;
\item Interiam storage of spent nuclear fuel (SNF): the fuel assembly containing spent fuel is stored in the reactor (in pools/ponds, i.e., cooled with circulate water) for a few years before being transferred into an interim storage (for a few decades) to reduce radioactivity and decay-heat before permanent disposal;
\item Waste disposal: final destination of the SNF.  

\end{itemize}
}
\medskip

\item CO$_{2}$ produced in fossil fuel-based power plants can be captured, transported and stored in deep geological formations. After captured, CO$_{2}$ is compressed until reach supercritical state and transported via pipelines. What are the main technological challenges for long distance transport?
\begin{flushright}
{\bf [05 Marks]}
\end{flushright}
\medskip

{\it 
Transporting CO$_{\text{2}}$ via pipelines requires that fluid be kept during all path in a single phase stage -- either as gas or liquid phases. It is necessary to undertake a strict monitoring/controlling of the pressure drop conditions through the pipeline with intermediate pumps or compressors. In either phase, CO$_{\text{2}}$ will need to be free of water (to avoid formation of gas hydrates) and contaminants (e.g., SO$_{\text{x}}$ can react with water to produce H$_{\text{2}}$S that can potentially act as a corrosion agent in the pipeline metal). For short distances, pressure conditions may be ensured without the use of pumps or compressors, but by substantially increasing the inlet pressure.  However this would require extra energy for the compression process and the pipeline wall would need to be reengineered to sustain larger pressure. 
}

\medskip

\item Four main mechanisms are assumed to be responsible for holding the CO$_{2}$ within the pores of the underground geological formations: (i) physical trapping, (ii) dissolution trapping; (iii) mineral trapping; (iv) capillary trapping. Explain these mechanisms.
\begin{flushright}
{\bf [05 Marks]}
\end{flushright}
\medskip

{\it
\begin{itemize}
\item Physical trapping (also known as structural and stratigraphic trapping): CO$_{\text{2}}$ is contained below $\lq$impermeable' or low-permeability rocks (i.e., caprock sealing integrity); 
\item Dissolution trapping: CO$_{\text{2}}$ is dissolved in brine -- the resulting solution is denser and slowly sink through the storage aquifer;
\item Mineral trapping: Insoluble carbonates and bicarbonates $\left(\text{CO}_{\text{3}}^{\text{-2}},\text{ HCO}_{\text{3}}^{\text{-}}\right)$ are formed and precipitated by the reaction of CO$_{\text{2}}$ and the surrounding rocks;
\item Capillary trapping: CO$_{\text{2}}$ can be trapped as micro-bubbles in the pore space.
\end{itemize}
}

\medskip
\end{enumerate}


\vfill

\paperend
\end{document}
