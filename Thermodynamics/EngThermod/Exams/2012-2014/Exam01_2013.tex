
\documentclass[12pt,twoside]{report}

\usepackage{amsfonts,amsmath,amssymb,stmaryrd,indentfirst}
\usepackage{epsfig,graphicx,times,psfrag}
\usepackage{natbib,enumerate}
%\usepackage{fancyhdr} %%%%
%\pagestyle{fancy}%%%%
\def\newblock{\hskip .11em plus .33em minus .07em}

\setlength\textwidth      {16.5cm}
\setlength\textheight     {22.0cm}
\setlength\oddsidemargin  {-0.3cm}
\setlength\evensidemargin {-0.3cm}

\setlength\headheight{0in} 
\setlength\topmargin{0.cm}
\setlength\headsep{1.cm}
\setlength\footskip{1.cm}
\setlength\parskip{0pt}

%%%%%%%%%%%%%%%%%%%%%%%%%%%%%%%%%%%%%%%%%%%
%%%%%%                              %%%%%%%
%%%%%%      NOTATION SECTION        %%%%%%%
%%%%%%                              %%%%%%%
%%%%%%%%%%%%%%%%%%%%%%%%%%%%%%%%%%%%%%%%%%%

% Text abbreviations.
\newcommand{\ie}{{\em{i.e., }}}
\newcommand{\eg}{{\em{e.g., }}}
\newcommand{\cf}{{\em{cf., }}}
\newcommand{\wrt}{with respect to}
\newcommand{\lhs}{left hand side}
\newcommand{\rhs}{right hand side}
% Commands definining mathematical notation.

% This is for quantities which are physically vectors.
\renewcommand{\vec}[1]{{\mbox{\boldmath$#1$}}}
% Physical rank 2 tensors
\newcommand{\tensor}[1]{\overline{\overline{#1}}}
% This is for vectors formed of the value of a quantity at each node.
\newcommand{\dvec}[1]{\underline{#1}}
% This is for matrices in the discrete system.
\newcommand{\mat}[1]{\mathrm{#1}}


\DeclareMathOperator{\sgn}{sgn}
\newtheorem{thm}{Theorem}[section]
\newtheorem{lemma}[thm]{Lemma}

%\newcommand\qed{\hfill\mbox{$\Box$}}
\newcommand{\re}{{\mathrm{I}\hspace{-0.2em}\mathrm{R}}}
\newcommand{\inner}[2]{\langle#1,#2\rangle}
\renewcommand\leq{\leqslant}
\renewcommand\geq{\geqslant}
\renewcommand\le{\leqslant}
\renewcommand\ge{\geqslant}
\renewcommand\epsilon{\varepsilon}
\newcommand\eps{\varepsilon}
\renewcommand\phi{\varphi}
\newcommand{\bmF}{\vec{F}}
\newcommand{\bmphi}{\vec{\phi}}
\newcommand{\bmn}{\vec{n}}
\newcommand{\bmns}{{\textrm{\scriptsize{\boldmath $n$}}}}
\newcommand{\bmi}{\vec{i}}
\newcommand{\bmj}{\vec{j}}
\newcommand{\bmk}{\vec{k}}
\newcommand{\bmx}{\vec{x}}
\newcommand{\bmu}{\vec{u}}
\newcommand{\bmv}{\vec{v}}
\newcommand{\bmr}{\vec{r}}
\newcommand{\bma}{\vec{a}}
\newcommand{\bmg}{\vec{g}}
\newcommand{\bmU}{\vec{U}}
\newcommand{\bmI}{\vec{I}}
\newcommand{\bmq}{\vec{q}}
\newcommand{\bmT}{\vec{T}}
\newcommand{\bmM}{\vec{M}}
\newcommand{\bmtau}{\vec{\tau}}
\newcommand{\bmOmega}{\vec{\Omega}}
\newcommand{\pp}{\partial}
\newcommand{\kaptens}{\tensor{\kappa}}
\newcommand{\tautens}{\tensor{\tau}}
\newcommand{\sigtens}{\tensor{\sigma}}
\newcommand{\etens}{\tensor{\dot\epsilon}}
\newcommand{\ktens}{\tensor{k}}
\newcommand{\half}{{\textstyle \frac{1}{2}}}
\newcommand{\tote}{E}
\newcommand{\inte}{e}
\newcommand{\strt}{\dot\epsilon}
\newcommand{\modu}{|\bmu|}
% Derivatives
\renewcommand{\d}{\mathrm{d}}
\newcommand{\D}{\mathrm{D}}
\newcommand{\ddx}[2][x]{\frac{\d#2}{\d#1}}
\newcommand{\ddxx}[2][x]{\frac{\d^2#2}{\d#1^2}}
\newcommand{\ddt}[2][t]{\frac{\d#2}{\d#1}}
\newcommand{\ddtt}[2][t]{\frac{\d^2#2}{\d#1^2}}
\newcommand{\ppx}[2][x]{\frac{\partial#2}{\partial#1}}
\newcommand{\ppxx}[2][x]{\frac{\partial^2#2}{\partial#1^2}}
\newcommand{\ppt}[2][t]{\frac{\partial#2}{\partial#1}}
\newcommand{\pptt}[2][t]{\frac{\partial^2#2}{\partial#1^2}}
\newcommand{\DDx}[2][x]{\frac{\D#2}{\D#1}}
\newcommand{\DDxx}[2][x]{\frac{\D^2#2}{\D#1^2}}
\newcommand{\DDt}[2][t]{\frac{\D#2}{\D#1}}
\newcommand{\DDtt}[2][t]{\frac{\D^2#2}{\D#1^2}}
% Norms
\newcommand{\Ltwo}{\ensuremath{L_2} }
% Basis functions
\newcommand{\Qone}{\ensuremath{Q_1} }
\newcommand{\Qtwo}{\ensuremath{Q_2} }
\newcommand{\Qthree}{\ensuremath{Q_3} }
\newcommand{\QN}{\ensuremath{Q_N} }
\newcommand{\Pzero}{\ensuremath{P_0} }
\newcommand{\Pone}{\ensuremath{P_1} }
\newcommand{\Ptwo}{\ensuremath{P_2} }
\newcommand{\Pthree}{\ensuremath{P_3} }
\newcommand{\PN}{\ensuremath{P_N} }
\newcommand{\Poo}{\ensuremath{P_1P_1} }
\newcommand{\PoDGPt}{\ensuremath{P_{-1}P_2} }

\newcommand{\metric}{\tensor{M}}
\newcommand{\configureflag}[1]{\texttt{#1}}

% Units
\newcommand{\m}[1][]{\unit[#1]{m}}
\newcommand{\km}[1][]{\unit[#1]{km}}
\newcommand{\s}[1][]{\unit[#1]{s}}
\newcommand{\invs}[1][]{\unit[#1]{s}\ensuremath{^{-1}}}
\newcommand{\ms}[1][]{\unit[#1]{m\ensuremath{\,}s\ensuremath{^{-1}}}}
\newcommand{\mss}[1][]{\unit[#1]{m\ensuremath{\,}s\ensuremath{^{-2}}}}
\newcommand{\K}[1][]{\unit[#1]{K}}
\newcommand{\PSU}[1][]{\unit[#1]{PSU}}
\newcommand{\Pa}[1][]{\unit[#1]{Pa}}
\newcommand{\kg}[1][]{\unit[#1]{kg}}
\newcommand{\rads}[1][]{\unit[#1]{rad\ensuremath{\,}s\ensuremath{^{-1}}}}
\newcommand{\kgmm}[1][]{\unit[#1]{kg\ensuremath{\,}m\ensuremath{^{-2}}}}
\newcommand{\kgmmm}[1][]{\unit[#1]{kg\ensuremath{\,}m\ensuremath{^{-3}}}}
\newcommand{\Nmm}[1][]{\unit[#1]{N\ensuremath{\,}m\ensuremath{^{-2}}}}

% Dimensionless numbers
\newcommand{\dimensionless}[1]{\mathrm{#1}}
\renewcommand{\Re}{\dimensionless{Re}}
\newcommand{\Ro}{\dimensionless{Ro}}
\newcommand{\Fr}{\dimensionless{Fr}}
\newcommand{\Bu}{\dimensionless{Bu}}
\newcommand{\Ri}{\dimensionless{Ri}}
\renewcommand{\Pr}{\dimensionless{Pr}}
\newcommand{\Pe}{\dimensionless{Pe}}
\newcommand{\Ek}{\dimensionless{Ek}}
\newcommand{\Gr}{\dimensionless{Gr}}
\newcommand{\Ra}{\dimensionless{Ra}}
\newcommand{\Sh}{\dimensionless{Sh}}
\newcommand{\Sc}{\dimensionless{Sc}}


% Journals
\newcommand{\IJHMT}{{\it International Journal of Heat and Mass Transfer}}
\newcommand{\NED}{{\it Nuclear Engineering and Design}}
\newcommand{\ICHMT}{{\it International Communications in Heat and Mass Transfer}}
\newcommand{\NET}{{\it Nuclear Engineering and Technology}}
\newcommand{\HT}{{\it Heat Transfer}}   
\newcommand{\IJHT}{{\it International Journal for Heat Transfer}}

\newcommand{\frc}{\displaystyle\frac}

%%%%%%%%%%%%%%%%%%%%%%%%%%%%%%%%%%%%%%%%%%%
%%%%%%                              %%%%%%%
%%%%%% END OF THE NOTATION SECTION  %%%%%%%
%%%%%%                              %%%%%%%
%%%%%%%%%%%%%%%%%%%%%%%%%%%%%%%%%%%%%%%%%%%


% Cause numbering of subsubsections. 
%\setcounter{secnumdepth}{8}
%\setcounter{tocdepth}{8}

\setcounter{secnumdepth}{4}%
\setcounter{tocdepth}{4}%

\DeclareMathAlphabet{\mathpzc}{OT1}{pzc}{m}{it}

\begin{document}


\setcounter{page}{1}

\vfill

\pagebreak


\begin{description}

%%%
%%% Module 01
%%%
\item [Question 1:] In the secondary cooling circuit of a nuclear power plant, the steam generator (boiler / reheater) is connected to two turbines operating as a reheat Rankine cycle (Fig. \ref{exam_mod01_rankinecycle}). Primary superheated steam is at 40 bar and 370$^{\text{o}}$C, with reheat to 7 bar and 370$^{\text{o}}$C. The isentropic efficiencies of the first $\left(\eta_{\text{T1}}\right)$ and second $\left(\eta_{\text{T2}}\right)$ turbines and boiler feed pump $\left(\eta_{\text{P}}\right)$ are 84$\%$, 80$\%$ and 61$\%$ respectively.


\begin{figure}[h]
\begin{center}
\includegraphics[width=13.cm,clip]{./Pics/Exam_Reheat_Rankine_Cycle}
\caption{ Reheat Rankine cycle with 2 turbines.}
\label{exam_mod01_rankinecycle}
\end{center}
\end{figure}


\begin{enumerate}[(a)]
\item (10 Marks) In the Table below, determine {\it (a)-(j)}.\\
%\begin{table}[h]
%\label{exam1_table1}
\begin{center}
\begin{tabular} {||c | c c c c c || }
\hline\hline
{\bf Stage} & {\bf P}    & {\bf T}        & {\bf State}    & {\bf H}             & {\bf S}                  \\
            & {\bf (bar)}& {\bf ($^{o}$C)} &               & {\bf (kJ.kg$^{-1}$)} & {\bf (kJ.(kg.K)$^{-1}$)} \\
\hline\hline
 {\bf 1 }   & 40         & 370            &   superheated  & {\bf (a)}           & {\bf (b)}                 \\
            &            &                &   steam        &                     &                           \\
 {\bf 2 }   &  --        &  --            &     {\bf (c)}  & --                  &   --                      \\
 {\bf 3 }   & 7          & 370            &   superheated  & {\bf (d)}           & {\bf (e)}                 \\
            &            &                &   steam        &                     &                           \\
 {\bf 4 }   & 0.10       & --             &     --         & --                   & --                      \\
 {\bf 5 }   & 0.10       & --             &   {\bf (f)}    & {\bf (g)}           & {\bf (h)}                 \\
 {\bf 6 }   & 40         & --             &   {\bf (i)}    & {\bf (j)}           & --                       \\

\hline\hline
\end{tabular}
\end{center}
%\caption{Thermodynamic table of the reheat Rankine cycle.}
%\end{table}


\item (5 Marks) Calculate the thermal efficiency $\left(\eta_{\text{Thermal}}\right)$ of the reheat Rankine cycle with 2 turbines. $\eta_{\text{Thermal}}$ is expressed as,
\begin{displaymath}
\eta_{\text{Thermal}} = \frc{ \left(H_{1}-H_{2s}\right)\eta_{\text{T1}} + \left(H_{3}-H_{4s}\right)\eta_{\text{T2}} - V_{5}\left(P_{6}-P_{5}\right)\eta_{\text{P}}^{-1}} {\left(H_{1}-H_{6}\right)+\left(H_{3}-H_{2}\right)}
\end{displaymath}
\item (5 Marks) Sketch the {\it T-S} diagram for this cycle.

\end{enumerate}

%%%
%%% Module 02
%%%
\item [Question 2:] \mbox{}
\begin{enumerate}[(a)]
\item (8 Marks) In France, 421 billion kWh of electricity were made from nuclear fuels in 2011.  If an equivalent amount had been raised from natural gas, what would have been the carbon footprint? \\
{\it Heat of combustion of methane = 889 kJ.mol$^{-1}$ \\
Atomic weights/gmol$^{-1}$: C: 12 \;\; H: 1}
\item (2 Marks) Give an example, in qualitative terms, of how a chemical and nuclear explosion can have equivalent blasts if quantities in the former are much larger than in the latter.
\item (8 Marks) Coke, of calorific value is 25 MJ.kg$^{-1}$, is used to make heat at 300 MW. It is desired to reduce the carbon footprint by 10$\%$ by blending the coke with citrus peel of calorific value 7 MJ.kg$^{-1}$ whilst maintaining a heat production rate of 30MW. At what ratios by weight will coke and citrus peel have to be blended?
\item (2 Marks) Explain how in the supply of biomass for fuel use forest sustainability is ensured. 
\end{enumerate} 


%%%
%%% Module 03
%%%
\item [Question 3:] \mbox{}

\begin{enumerate}[(a)]
%
\item (4 Marks) A horizontally mounted turbine is housed between circular inlet and outlet pipes of circumference 1 m and 0.6 m, respectively. Assume gas satisfying the steady flow energy conservation
\begin{displaymath}
\frc{ \dot{Q} -\dot{W}_{s}}{\dot{m}} = \left( h_{2}+ \frc{u_{2}^{2}}{2}\right) - \left(h_{1} + \frc{u_{1}^{2}}{2}\right),
\end{displaymath}
flows through the turbine at a steady rate of 4 kg/s. At the inlet the fluid has an enthalpy of 70 kJ and a velocity of 30 m/s, while at the outlet the fluid has an enthalpy of 40 kJ. If the gas does work on the turbine at a rate of 30 kW and transfers heat to the surroundings at a rate of 15 kW, then find the change in gas density between the inlet and the outlet.

%
\item (2 Marks) For gas flow along a duct whose length is parameterized by $x$ and has slowly-varying cross-sectional area  $A(x)$, use equations corresponding to mass and energy conservation to show that
\begin{displaymath}
\frc{dV}{V} + \frc{d h}{u^{2}} - \frc{dA}{A}=0,
\end{displaymath}
where the specific volume is denoted $V$, the specific enthalpy $h$, and fluid velocity $u$.
\medskip

\item (4 Marks) Define the speed of sound and the Mach number in a gas. Give equations that are appropriate for calculating these quantities in an isentropic gas.
\medskip

\item (3 Marks) For an isentropic process show that changes in specific volume are related to changes in pressure through
\begin{displaymath}
dV=-\frc{V^{2}}{c^{2}}dp
\end{displaymath}
and explain how changes in enthalpy are related to changes in pressure. 
\medskip

\item (5 Marks) Hence, for isentropic flow along a duct, show that
\begin{displaymath}
\frc{1}{A\left(1-Ma^{2}\right)}\frc{dA}{dx} = \frc{1}{\rho M a^{2}}\frc{d\rho}{dx}
\end{displaymath}
\medskip

\item (2 Marks) Explain with reasoning how the gas density $\left(\rho\right)$, changes for flow along a supersonic diffuser.
%
\end{enumerate}

%%%
%%% Module 04 (Rajput 14.19)
%%%
\item [Question 4:] A refrigerator operating with Freon-12 as a refrigerant fluid produces a cooling effect of 20 kJ/s (Fig. \ref{exam_refrig1}). The refrigerator operates on a simple cycle with pressure limits of 1.509 bar and 9.607 bar. The vapour leaves the evaporator dry saturated and there is no undercooling.  Assume that the compressor operaters at 300 rpm and has a clearance volume of 3$\%$ of stroke volume.  For the compressor assume that the expansion is described by $PV^{1.13}$ = constant. 
\begin{enumerate}[(a)]
\item (10 Marks) Determine the power required by the machine (W).
\item (10 Marks) Calculate the piston displacement of the compressor $\left(\text{m}^{3}\right)$. 
\end{enumerate}

\medskip
Given:
\begin{center}
\begin{tabular}{|c c| c c c c c c| }
\hline
$T$             & $P_{s}$  & $V_{g}$  & $H_{f}$  & $H_{g}$   &  $S_{f}$   &  $S_{g}$   & {\it Specific Heat} \\
($^{\text{o}}$C)  & (bar)   & $\left(\text{m}^{3}/\text{kg}\right)$ & (kJ/kg) & (kJ/kg) & (kJ/(kg.K)) &  (kJ/(kg.K)) &   (kJ/(kg.K)) \\
\hline
-20   & 1.509 & 0.1088 & 17.8 & 178.61 & 0.073 & 0.7082 & -- \\
40    & 9.607 & --     & 74.53 & 203.05 & 0.2716 & 0.682 & 0.747 \\
\hline
\end{tabular}
\end{center}

and the volumetric efficiency,

\begin{displaymath}
\eta_{\text{vol}} = 1 + C - C\left(\frc{P_{d}}{P_{s}}\right)^{1/n} 
\end{displaymath}

\begin{figure}[h]
\begin{center}
\includegraphics[width=10.cm,clip]{./Pics/Exam_Refrigeration1}
\caption{ Refrigeration cycle, $Ts$ diagram  -- Question 4}
\label{exam_refrig1}
\end{center}
\end{figure}

\clearpage

%%%
%%% Module 05
%%%
\item [Question 5:]\mbox{}
\begin{enumerate}[(a)]
\item (4 Marks) Define the specific humidity $\omega$. Assuming both dry air and water vapour behave like ideal gases with specific gas constants $R_{a}=0.2871$ kJ/(kg.K) and $R_{v}=0.4615$ kJ/(kg.K), respectively, show that
\begin{displaymath}
\omega = \frc{0.622 p_{v}}{p-p_{v}}
\end{displaymath}
where $p$ is the absolute pressure and $p_{v}$ is the partial pressure of water vapour. 
\medskip

\item (2 Marks) If the saturation pressure of water is denoted $p_{g}$, and relative humidity $\varphi$, then show that
\begin{displaymath}
\omega = \frc{0.622 \varphi p_{g}}{p-\varphi p_{g}}
\end{displaymath}
\medskip

\item An air-conditioning system takes in outdoor air at 12$^{\text{o}}$C and 25 percent relative humidity at a steady rate of 40 m$^{3}$/min and then conditions it to 24$^{\text{o}}$C and 55 percent relative humidity. This heating and humidification takes place in two distinct steady processes. Firstly the outdoor air is heated to 20$^{\text{o}}$C in a heating section, and secondly the air is humidified by the injection of hot steam in a humidifying section. Assuming both stages take place at a constant pressure of 100 kPa, determine:
\begin{enumerate}[(i)]
\item (3 Marks) the partial pressures of water vapour and dry air, and the specific humidity at the inlet;  
\item (6 Marks) the rate heat is supplied in the heating section;  
\item (5 Marks) the mass flow rate of the steam required in the humidifying section.
\end{enumerate}
\medskip
\end{enumerate}
You may assume that the specific heat of dry air is independent of temperature and has the value $C_{p}=1.005$ kJ/(kg.K). The saturation pressure of water is 1.4028 kPa at 12$^{\text{o}}$C, and 2.9858 kPa at 24$^{\text{o}}$C. The enthalpy of saturated water vapour is 2523 kJ/kg at 12$^{\text{o}}$C, and 2537 kJ/kg at 20$^{\text{o}}$C.


\end{description}

%\begin{Large}
%\include{Module_Questions}
%% Aberdeen style guide should be followed when using this
% layout. Their template powerpoint slide is used to extract the
% Aberdeen color and logo but is otherwise ignored (it has little or
% no formatting in it anyway).
%
% http://www.abdn.ac.uk/documents/style-guide.pdf

%%%%%%%%%%%%%%%%%%%% Document Class Settings %%%%%%%%%%%%%%%%%%%%%%%%%
% Pick if you want slides, or draft slides (no animations)
%%%%%%%%%%%%%%%%%%%%%%%%%%%%%%%%%%%%%%%%%%%%%%%%%%%%%%%%%%%%%%%%%%%%%%
%Normal document mode%
\documentclass[10pt,compress]{beamer}
%Draft or handout mode
%\documentclass[10pt,compress,handout]{beamer}
%\documentclass[10pt,compress,handout,ignorenonframetext]{beamer}

%%%%%%%%%%%%%%%%%%%% General Document settings %%%%%%%%%%%%%%%%%%%%%%%
% These settings must be set for each presentation
%%%%%%%%%%%%%%%%%%%%%%%%%%%%%%%%%%%%%%%%%%%%%%%%%%%%%%%%%%%%%%%%%%%%%%
\newcommand{\shortname}{jefferson.gomes@abdn.ac.uk}
\newcommand{\fullname}{Dr Jeff Gomes}
\institute{School of Engineering}
\newcommand{\emailaddress}{}%jefferson.gomes@abdn.ac.uk}
\newcommand{\logoimage}{../../FigBanner/UoAHorizBanner}
\title{Chemical Thermodynamics (EX3029)}
\subtitle{Module 3: Thermodynamic Properties of Pure Fluids}
\date[ ]{ }

%%%%%%%%%%%%%%%%%%%% Template settings %%%%%%%%%%%%%%%%%%%%%%%%%%%%%%%
% You shouldn't have to change below this line, unless you want to.
%%%%%%%%%%%%%%%%%%%%%%%%%%%%%%%%%%%%%%%%%%%%%%%%%%%%%%%%%%%%%%%%%%%%%%
\usecolortheme{whale}
\useoutertheme{infolines}

% Use the fading effect for items that are covered on the current
% slide.
\beamertemplatetransparentcovered

% We abuse the author command to place all of the slide information on
% the title page.
\author[\shortname]{%
  \fullname\\\ttfamily{\emailaddress}
}


%At the start of every section, put a slide indicating the contents of the current section.
\AtBeginSection[] {
  \begin{frame}
    \frametitle{Section Outline}
    \tableofcontents[currentsection]
  \end{frame}
}

% Allow the inclusion of movies into the Presentation! At present,
% only the Okular program is capable of playing the movies *IN* the
% presentation.
\usepackage{multimedia}
\usepackage{animate}

%% Handsout -- comment out the lines below to create handstout with 4 slides in a page with space for comments
\usepackage{handoutWithNotes}

\mode<handout>
{
\usepackage{pgf,pgfpages}

\pgfpagesdeclarelayout{2 on 1 boxed with notes}
{
\edef\pgfpageoptionheight{\the\paperheight} 
\edef\pgfpageoptionwidth{\the\paperwidth}
\edef\pgfpageoptionborder{0pt}
}
{
\setkeys{pgfpagesuselayoutoption}{landscape}
\pgfpagesphysicalpageoptions
    {%
        logical pages=4,%
        physical height=\pgfpageoptionheight,%
        physical width=\pgfpageoptionwidth,%
        last logical shipout=2%
    } 
\pgfpageslogicalpageoptions{1}
    {%
    border code=\pgfsetlinewidth{1pt}\pgfstroke,%
    scale=1,
    center=\pgfpoint{.25\pgfphysicalwidth}{.75\pgfphysicalheight}%
    }%
\pgfpageslogicalpageoptions{2}
    {%
    border code=\pgfsetlinewidth{1pt}\pgfstroke,%
    scale=1,
    center=\pgfpoint{.25\pgfphysicalwidth}{.25\pgfphysicalheight}%
    }%
\pgfpageslogicalpageoptions{3}
    {%
    border shrink=\pgfpageoptionborder,%
    resized width=.7\pgfphysicalwidth,%
    resized height=.5\pgfphysicalheight,%
    center=\pgfpoint{.75\pgfphysicalwidth}{.29\pgfphysicalheight},%
    copy from=3
    }%
\pgfpageslogicalpageoptions{4}
    {%
    border shrink=\pgfpageoptionborder,%
    resized width=.7\pgfphysicalwidth,%
    resized height=.5\pgfphysicalheight,%
    center=\pgfpoint{.75\pgfphysicalwidth}{.79\pgfphysicalheight},%
    copy from=4
    }%

\AtBeginDocument
    {
    \newbox\notesbox
    \setbox\notesbox=\vbox
        {
            \hsize=\paperwidth
            \vskip-1in\hskip-1in\vbox
            {
                \vskip1cm
                Notes\vskip1cm
                        \hrule width\paperwidth\vskip1cm
                    \hrule width\paperwidth\vskip1cm
                        \hrule width\paperwidth\vskip1cm
                    \hrule width\paperwidth\vskip1cm
                        \hrule width\paperwidth\vskip1cm
                    \hrule width\paperwidth\vskip1cm
                    \hrule width\paperwidth\vskip1cm
                    \hrule width\paperwidth\vskip1cm
                        \hrule width\paperwidth
            }
        }
        \pgfpagesshipoutlogicalpage{3}\copy\notesbox
        \pgfpagesshipoutlogicalpage{4}\copy\notesbox
    }
}
}

%\pgfpagesuselayout{2 on 1 boxed with notes}[letterpaper,border shrink=5mm]

%%%%% Color settings
\usepackage{color}
%% The background color for code listings (i.e. example programs)
\definecolor{lbcolor}{rgb}{0.9,0.9,0.9}%
\definecolor{UoARed}{rgb}{0.64706, 0.0, 0.12941}
\definecolor{UoALight}{rgb}{0.85, 0.85, 0.85}
\definecolor{UoALighter}{rgb}{0.92, 0.92, 0.92}
\setbeamercolor{structure}{fg=UoARed} % General background and higlight color
\setbeamercolor{frametitle}{bg=black} % General color
\setbeamercolor{frametitle right}{bg=black} % General color
\setbeamercolor{block body}{bg=UoALighter} % For blocks
\setbeamercolor{structure}{bg=UoALight} % For blocks
% Rounded boxes for blocks
\setbeamertemplate{blocks}[rounded]

%%%%% Font settings
% Aberdeen requires the use of Arial in slides. We can use the
% Helvetica font as its widely available like so
% \usepackage{helvet}
% \renewcommand{\familydefault}{\sfdefault}
% But beamer already uses a sans font, so we will stick with that.

% The size of the font used for the code listings.
\newcommand{\goodsize}{\fontsize{6}{7}\selectfont}

% Extra math packages, symbols and colors. If you're using Latex you
% must be using it for formatting the math!
\usepackage{amscd,amssymb} \usepackage{amsfonts}
\usepackage[mathscr]{eucal} \usepackage{mathrsfs}
\usepackage{latexsym} \usepackage{amsmath} \usepackage{bm}
\usepackage{amsthm} \usepackage{textcomp} \usepackage{eurosym}
% This package provides \cancel{a} and \cancelto{a}{b} to "cancel"
% expressions in math.
\usepackage{cancel}

\usepackage{comment} 

% Get rid of font warnings as modern LaTaX installations have scalable
% fonts
\usepackage{type1cm} 

%\usepackage{enumitem} % continuous numbering throughout enumerate commands

% For exact placement of images/text on the cover page
\usepackage[absolute]{textpos}
\setlength{\TPHorizModule}{1mm}%sets the textpos unit
\setlength{\TPVertModule}{\TPHorizModule} 

% Source code formatting package
\usepackage{listings}%
\lstset{ backgroundcolor=\color{lbcolor}, tabsize=4,
  numberstyle=\tiny, rulecolor=, language=C++, basicstyle=\goodsize,
  upquote=true, aboveskip={1.5\baselineskip}, columns=fixed,
  showstringspaces=false, extendedchars=true, breaklines=false,
  prebreak = \raisebox{0ex}[0ex][0ex]{\ensuremath{\hookleftarrow}},
  frame=single, showtabs=false, showspaces=false,
  showstringspaces=false, identifierstyle=\ttfamily,
  keywordstyle=\color[rgb]{0,0,1},
  commentstyle=\color[rgb]{0.133,0.545,0.133},
  stringstyle=\color[rgb]{0.627,0.126,0.941}}

% Allows the inclusion of other PDF's into the final PDF. Great for
% attaching tutorial sheets etc.
\usepackage{pdfpages}
\setbeamercolor{background canvas}{bg=}  

% Remove foot note horizontal rules, they occupy too much space on the slide
\renewcommand{\footnoterule}{}

% Force the driver to fix the colors on PDF's which include mixed
% colorspaces and transparency.
\pdfpageattr {/Group << /S /Transparency /I true /CS /DeviceRGB>>}

% Include a graphics, reserve space for it but
% show it on the next frame.
% Parameters:
% #1 Which slide you want it on
% #2 Previous slides
% #3 Options to \includegraphics (optional)
% #4 Name of graphic
\newcommand{\reserveandshow}[4]{%
\phantom{\includegraphics<#2|handout:0>[#3]{#4}}%
\includegraphics<#1>[#3]{#4}%
}

\newcommand{\frc}{\displaystyle\frac}
\newcommand{\red}{\textcolor{red}}
\newcommand{\blue}{\textcolor{blue}}
\newcommand{\green}{\textcolor{green}}
\newcommand{\purple}{\textcolor{purple}}
\newcommand{\eg}{{\it e.g., }}
\newcommand{\ie}{{\it i.e., }}
\newcommand{\wrt}{{\it wrt }}
\newcommand{\Partial}[3][error]{\left(\frc{\partial #1}{\partial #2}\right)_{#3}}
\newcommand{\mfr}[3][error]{#1_{#2}^{\left(#3\right)}} 
\newcommand{\summation}[3][error]{\sum\limits_{#2}^{#3}#1}
 
\begin{document}

% Title page layout
\begin{frame}
  \titlepage
  \vfill%
  \begin{center}
    \includegraphics[clip,width=0.8\textwidth]{\logoimage}
  \end{center}
\end{frame}

% Table of contents
\frame{ \frametitle{Slides Outline}
  \tableofcontents
}


%%%%%%%%%%%%%%%%%%%% The Presentation Proper %%%%%%%%%%%%%%%%%%%%%%%%%
% Fill below this line with \begin{frame} commands! It's best to
% always add the fragile option incase you're going to use the
% verbatim environment.
%%%%%%%%%%%%%%%%%%%%%%%%%%%%%%%%%%%%%%%%%%%%%%%%%%%%%%%%%%%%%%%%%%%%%%


%%%
%%% SECTION
%%%
%\section{General Remarks}

%%%
%%% Slides
%%%
\begin{frame}
 \frametitle{Aims and Objectives}
   \begin{enumerate}
     \item<1-> In Modules 1-2, we learnt:
       \begin{enumerate}
         \item<1-> the laws of Thermodynamics and how they describe thermal equilibrium of species in closed and opened systems;
         \item<1-> how to calculate thermodynamics properties -- internal energy, enthalpy and entropy for pure chemicallspecies;
         \item<1-> PVT behaviour of pure species in equilibrium and;
         \item<1-> Equations of state.
       \end{enumerate} 
     \item<2-> This Module focuses on 
         \begin{enumerate}
           \item<2-> Thermodynamic properties of pure fluids;
           \item<2-> Introduce two remaining thermodynamic properties: Gibbs and Helmholtz free energies;
           \item<2-> Maxwell relations.
         \end{enumerate}
   \end{enumerate}

\end{frame}


%%%
%%% SECTION
%%%
%\section{Bibliography}
\begin{frame}
 \frametitle{Suggested References}
  Literature relevant for this module:
  \begin{enumerate}[(a)]
   \item\label{SVN_Book} J.M. Smith, H.C. Van Ness, M.M. Abbott, $\lq$Introduction to Chemical Engineering Thermodynamics', 6$^{th}$ Edition: Chapter 6;
   \item Y.A. Cengel, M.A. Boles, $\lq$Thermodynamics -- An Engineering Approach', 5$^{th}$ Edition: Chapter 12.1-4; 
   %\item M.J. Moran, H.N. Saphiro, D.D. Boettner, M.B. Bailey, $\lq$Principles of Engineering Thermodynamics', 7$^{th}$ Edition: Chapters 3;
   %\item C. Borgnakke, R.E. Sonntag,$\lq$Fundamentals of Thermodynamics',8$^{th}$ Edition: Chapter 2.
   \item S.I. Sandler, $\lq$Chemical, Biochemical and Engineering Thermodynamics', 4$^{th}$ Edition: Chapter 6.
  \end{enumerate}
\end{frame}


%%%
%%% SECTION
%%%
\section{Property Relations for Homogeneous Phases}

%%%
%%% SUBSECTION
%%%
\subsection{Thermodynamic Potentials} 


%%%
%%% Slide
%%%
%\scriptsize
\begin{frame}
  \frametitle{New State Functions}
   \visible<1->{\begin{block}{Enthalpy}
      \begin{displaymath}
         H = U + P V
      \end{displaymath}
   \end{block}
   }
   \visible<2->{\begin{block}{Gibbs Free Energy}
      \begin{displaymath}
         G = H - T S
      \end{displaymath}
   \end{block}
   }
   \visible<3->{\begin{block}{Helmholtz Free Energy}
      \begin{displaymath}
         A = U - T S
      \end{displaymath}
   \end{block}
   }

\end{frame}
\normalsize

%%%
%%% SUBSECTION
%%%
\subsection{Maxwell's Relations}
%%%
%%% Slide
%%%
%\scriptsize
\begin{frame}
  \frametitle{Maxwell's Relations}\label{Module03:Section:MaxwellRelation}
   \visible<1->{\begin{block}{For 1 mol of homogeneous fluid at $T=$ constant}
      \begin{center}
        \begin{tabular}{l c  l}
           $dU = T dS - PdV$  &  \hspace{1cm} & $dH = TdS + VdP$ \\
           $dA = -PdV - SdT$  &  \hspace{1cm} & $dG = VdP - SdT$ 
        \end{tabular}
      \end{center}
   \end{block}
   }
   \visible<2->{\begin{block}{Maxwell's Equations}
      \begin{center}
        \begin{tabular}{l c  l}
           $\left(\frc{\partial T}{\partial V}\right)_{S} = -\left(\frc{\partial P}{\partial S}\right)_{v}$  &  \hspace{1cm} & $\left(\frc{\partial T}{\partial P}\right)_{S} =  \left(\frc{\partial V}{\partial S}\right)_{P}$  \\
           $\left(\frc{\partial P}{\partial T}\right)_{V} =  \left(\frc{\partial S}{\partial V}\right)_{T}$  &  \hspace{1cm} & $\left(\frc{\partial V}{\partial T}\right)_{\red{P}} = -\left(\frc{\partial S}{\partial P}\right)_{T}$ 
        \end{tabular}
      \end{center}
   \end{block}
   }

\end{frame}
\normalsize


%%%
%%% SUBSECTION
%%%
\subsection{Thermodynamic Potentials: Dependence on $T$ and $P$}

%%%
%%% Slide
%%%
%\scriptsize
\begin{frame}
  \frametitle{Internal Energy, Enthalpy and Entropy}
   \visible<1->{\begin{block}{$H$ and $S$ as functions of $T$ and $P$}
      \begin{center}
        \begin{tabular}{l c  l}
           $dH = C_{p}dT + \left[ V - T\left(\frc{\partial V}{\partial T}\right)_{P}\right]dP$ &  \hspace{1cm} & $dS=C_{p}\frc{dT}{T} - \left(\frc{\partial V}{\partial T}\right)_{P}dP$
        \end{tabular}
      \end{center}
   \end{block}
   }
   \visible<2->{\begin{block}{Ideal Gas}
      \begin{center}
        \begin{tabular}{l c  l}
           $dH = C_{p}dT$  &  \hspace{1cm} & $dS=C_{p}\frc{dT}{T} - R\frc{dP}{P}$  
        \end{tabular}
      \end{center}
   \end{block}
   }
   \visible<3->{\begin{block}{Liquid}
      \begin{center}
        \begin{tabular}{l c  l}
           $dH = C_{p}dT + \left(1-\beta\right)VdP$  &  \hspace{1cm} & $dS = C_{p}\frc{dT}{T} - \beta V{dP}$  
        \end{tabular}
      \end{center}
   \end{block}
   }
   \visible<4->{\begin{block}{$U$ and $S$ as functions of $T$ and $V$}
      \begin{center}
        \begin{tabular}{l c  l}
           $dU = C_{v}dT + \left[T\left(\frc{\partial P}{\partial T}\right)_{V} - P\right]dV$  &  \hspace{1cm} & $dS = C_{v}\frc{dT}{T} - \left(\frc{\partial P}{\partial T}\right)_{V}dV$  
        \end{tabular}
      \end{center}
   \end{block}
   }

\end{frame}
\normalsize


%%%
%%% Slide
%%%
%\scriptsize
\begin{frame}
  \frametitle{Gibbs Free Energy as Differential Operator}
     \begin{enumerate}[(a)]
        \item<1-> Gibbs Energy as generating function:
              \begin{displaymath}
                 \visible<1->{dG = VdP - SdT} \visible<2->{\Longrightarrow d\left(\frc{G}{RT}\right) = \frc{1}{RT}dG - \frc{G}{RT^{2}}dT} \nonumber 
              \end{displaymath}

        \item<3-> After substitution $\&$ algebraic reduction:
              \begin{displaymath}
                  \visible<3->{\frc{V}{RT} = \left\{ \frc{\partial\left(\frc{G}{RT}\right)}{\partial P}\right\}_{T} \hspace{1cm}\text{and}\hspace{1cm} \frc{H}{RT} = -T\left\{ \frc{\partial\left(\frc{G}{RT}\right)}{\partial T}\right\}_{P}}
              \end{displaymath}

        \item<4-> \textcolor{blue}{The Gibbs free energy when expressed as $G\left(T,P\right)$ operates as a generating function for other thermodynamic properties, and implicitly represents complete property information.}
     \end{enumerate}

\end{frame}
\normalsize

%%%
%%% SECTION
%%%
\section{Residual Properties}

\subsection{General Approach}
%%%
%%% Slide
%%%
%\scriptsize
\begin{frame}
  \frametitle{Residual Properties: General Approach}
     \begin{enumerate}[(a)]
        \item<1-> \textcolor{blue}{Residual Gibbs energy} is defined as:
              \visible<1->{\begin{displaymath}
                 \textcolor{red}{G^{R} = G - G^{ig}}
              \end{displaymath}}
            where \textcolor{blue}{$G$} is the actual Gibbs energy and \textcolor{blue}{$G^{id}$} is the correspondent thermodynamic function for ideal gas at same $T$ and $P$;
        \item<2-> This leads to,
              \visible<2->{\begin{displaymath}
                 d\left(\frc{G^{R}}{RT}\right) =\frc{V^{R}}{RT}dP - \frc{H^{R}}{RT^{2}}dT
              \end{displaymath}}
             
              \visible<3->{with,\begin{displaymath}
                 \frc{V^{R}}{RT} = \left\{ \frc{\partial\left(\frc{G^{R}}{RT}\right)}{\partial P}\right\}_{T} \hspace{0.5cm}\text{ and }\hspace{0.5cm} \frc{H^{R}}{RT} = -T\left\{ \frc{\partial\left(\frc{G^{R}}{RT}\right)}{\partial T}\right\}_{P}
              \end{displaymath}}
     \end{enumerate}

\end{frame}
\normalsize


%%%
%%% SUBSECTION
%%%
\subsection{Equations of State}
%%%
%%% Slide
%%%
%\scriptsize
\begin{frame}
  \frametitle{Residual Properties by EOS}
       Alternative approach to numerical integration: analytical solution by EOS:
            \begin{enumerate}[(a)]
                \item<1-> Virial EOS: \textcolor{blue}{$Z = 1 + \frc{BP}{RT}$}, 
                   \visible<2->{leading to \begin{displaymath}
                      \frc{G^{R}}{RT} = \displaystyle\int\limits_{0}^{\rho}\left(Z - 1\right)\frc{d\rho}{\rho} + Z - 1 - \ln Z \hspace{1cm} \frc{H^{R}}{RT} = -T \displaystyle\int\limits_{0}^{\rho}\left(\frc{\partial Z}{\partial T}\right)_{\rho}\frc{d\rho}{\rho} + Z - 1
                   \end{displaymath}
                   \begin{displaymath}
                      \frc{S^{R}}{RT} = \ln Z - T \displaystyle\int\limits_{0}^{\rho} \left(\frc{\partial Z}{\partial T}\right)_{\rho} \frc{d\rho}{\rho} - \displaystyle\int\limits_{0}^{\rho}\left(Z - 1\right)\frc{d\rho}{\rho} 
                   \end{displaymath}
}
             \end{enumerate}
\end{frame}
\normalsize



%%%
%%% Slide
%%%
%\scriptsize
\begin{frame}
  \frametitle{Residual Properties by EOS}
            \begin{enumerate}[(a)]\setcounter{enumi}{1}
               \item<1-> Cubic EOS: \textcolor{blue}{$P = \frc{RT}{V - b} - \frc{a\left(T\right)}{\left(V + \epsilon b\right)\left(V + \sigma b\right)}$}
                   \visible<2->{leading to\begin{displaymath}
                      \frc{G^{R}}{RT} = Z - 1 - \ln\left(Z - \beta\right) - q\mathcal{I} \hspace{1cm} \frc{H^{R}}{RT} = Z - 1 + \left[\frc{d\ln\alpha\left(T_{r}\right)}{d\ln T_{r}} - 1\right]q\mathcal{I}
                   \end{displaymath}
                   \begin{displaymath}
                       \frc{S^{R}}{R} = \ln\left(Z - \beta \right) + \frc{d\ln\alpha\left(T_{r}\right)}{d\ln T_{r}}q\mathcal{I} 
                   \end{displaymath}
                   with
                   \begin{displaymath}
                      \mathcal{I} \equiv \displaystyle\int\limits_{0}^{P} \frc{d\left(\rho b\right)}{\left(1 + \epsilon\rho b\right)\left(1 + \sigma\rho b\right)}\;\;\;\;\left(T\text{ const.}\right)
                   \end{displaymath}
                   See Section 6.3 of Smith, Van Ness $\&$ Abbott -- \textcolor{blue}{Reference (\ref{SVN_Book})}.
}
             \end{enumerate}

\end{frame}
\normalsize





%%%
%%% SECTION
%%%
\section{Two-Phase Systems}

\subsection{General Remarks}
%%%
%%% Slide
%%%
%\scriptsize
\begin{frame}
  \frametitle{General Remarks}
     \begin{enumerate}[(a)]
         \item<1-> \textcolor{blue}{Phase transition}: many extensive properties change abruptly during phase transition at given $P$ and $T$: specific volume, internal energy, enthalpy and entropy;
         \item<2-> \textcolor{blue}{Exception:} \textcolor{red}{molar Gibbs energy};
         \item<3-> For 2 phases \textcolor{blue}{$\alpha$} and \textcolor{blue}{$\beta$} of pure species at equilibrium,
            \visible<3->{\begin{block}{\textcolor{blue}{Equilibrium $\Rightarrow$ Equality of Gibbs energy}}
                     \begin{displaymath}
                        \textcolor{red}{G^{\alpha} = G^{\beta}} 
                     \end{displaymath}
            \end{block}}
         \item<4-> Clapeyron equation:
            \visible<4->{\begin{displaymath}
              \frc{d P^{sat}}{d T} = \frc{\Delta H^{lv}}{T \Delta V^{lv}}
            \end{displaymath}}
     \end{enumerate}

\end{frame}
\normalsize


%%%
%%% Slide
%%%
%\scriptsize
\begin{frame}
  \frametitle{General Remarks}
     \begin{enumerate}[(a)]\setcounter{enumi}{4}
         \item<1-> Temperature dependence of vapour pressure \blue{$\Rightarrow$} Empirical approaches for practical applications:
         \begin{enumerate}[(i)]
            \item<2-> Simplest case:
                \visible<2->{\begin{displaymath}
                   \ln P^{sat} = A - \frc{B}{T}
                \end{displaymath}}
            \item<3-> Antoine Equation:
                \visible<3->{\begin{displaymath}
                   \ln P^{sat} = A - \frc{B}{T+C}
                \end{displaymath}}
            \item<4-> Wagner Equation \blue{(more accurate)}:
                \visible<4->{\begin{displaymath}
                   \ln P_{r}^{sat} = \frc{A\tau + B\tau^{1.5} + C\tau^{3} + D\tau^{6}}{1-\tau}\;\;\;\text{ with }\;\;\; \tau = 1 - T_{r}
                \end{displaymath}}
         \end{enumerate}
     \end{enumerate}

\end{frame}
\normalsize


%%%
%%% SUBSECTION 
%%%
\subsection{Liquid/Vapour Systems}
%%%
%%% Slide
%%%
%\scriptsize
\begin{frame}
  \frametitle{Liquid/Vapour Systems}
     \begin{enumerate}[(a)]
         \item<1-> Systems with \blue{saturated vapour} and \blue{saturated liquid} in \red{equilibrium};
         \item<2-> \blue{Mass/Energy Balance} for any extensive property:
            \visible<3->{\begin{displaymath} 
                nV = n^{(l)}V^{(l)} + n^{(v)}V^{(v)} \;\;\; \Leftrightarrow \;\;\; V = x^{(l)}V^{(l)} + x^{(v)}V^{(v)}
             \end{displaymath}
             where $x^{(j)}$ is the \blue{molar/mass fraction} of phase \blue{j = l, v} $\rightarrow$  $x^{(l)} + x^{(v)} = 1$. 
            }
         \item<4-> The mass/molar volume fraction of vapour, \blue{$x^{(v)}$}, is also called \blue{vapour quality}. 
     \end{enumerate}

\end{frame}
\normalsize


%%%
%%%
%%% SUBSECTION 
%%%
\subsection{Thermodynamic Diagrams}

%%%
%%% Slide
%%%
%\scriptsize
\begin{frame}
  \frametitle{Pressure $\times$ Enthalpy ({\it PH}) Diagram}
      \begin{figure}%
        \begin{center}
          \includegraphics[width=1\columnwidth,clip]{./../Pics/LnP_H_Diagram}
        \end{center}
      \end{figure}
\end{frame}
\normalsize


%%%
%%% Slide
%%%
%\scriptsize
\begin{frame}
  \frametitle{Temperature $\times$ Entropy ({\it TS}) Diagram}
      \begin{figure}%
        \begin{center}
          \includegraphics[width=1\columnwidth,clip]{./../Pics/T_S_Diagram}
        \end{center}
      \end{figure}
\end{frame}
\normalsize

%%%
%%% Slide
%%%
%\scriptsize
\begin{frame}
  \frametitle{Enthalpy $\times$ Entropy ({\it Moiller}) Diagram}
      \begin{figure}%
        \begin{center} 
          \includegraphics[width=1\columnwidth,clip]{./../Pics/MoillerDiagram}
        \end{center}
      \end{figure}
\end{frame}
\normalsize


\section{Summary}

%%%
%%% Slide
%%%
%\scriptsize
\begin{frame}
 \frametitle{Summary}
   \begin{enumerate}[(i)]
     \item New thermodynamic potential properties: Gibbs and Helmholtz free energies;
     \item Introduction of Maxwell's relations and applications;
     \item Internal energy, enthalpy, entropy Gibbs and Helmholtz energies described as functions of pressure, volume and temperature (PVT);
     \item Introduction of residual properties and applications;
     \item Two-phase systems.
   \end{enumerate}
\end{frame}

\section{Examples}

%%%
%%% Slide
%%%
%\scriptsize
\begin{frame}
   \frametitle{Example 1}%[label=Module03:Example01]
    \blue{A block of copper of 1 kg undertakes a reversible compression from 0.1 MPa to 100 MPa at constant temperature of 15$^{\circ}$C. Calculate:}
    \begin{enumerate}[a)]
       \item \blue{Work done on the copper block during the process;}
       \item \blue{Change in entropy {\it per} kg of copper;}
       \item \blue{Heat transfer and;}
       \item \blue{Change of internal energy {\it per} kg.}
    \end{enumerate}
    \blue{Given, }
    \begin{itemize}
       \item \blue{Volume expansivity coefficient: $\beta = 5\times 10^{-5}$ K$^{-1}$;}
       \item \blue{Isothermal compressibility coefficient: $\kappa = 8.6\times 10^{-12}$ m$^{2}$.N$^{-1}$;}
       \item \blue{specific volume: $v=1.14\times 10^{-4}$ m$^{3}$.kg$^{-1}$.}
    \end{itemize} 

    \noindent{\bf Solution:} 

    \visible<2->{{\bf(a)} The work done during the compression,
                \begin{displaymath}
                   w = -\int P dv,
                \end{displaymath}
                where $v$ is the specific volume.}\visible<3->{ $\kappa$ was defined in Module 2 as,
                \begin{displaymath}
                   \kappa = \frc{1}{v}\left(\frc{\partial v}{\partial P}\right)_{T}\;\Longrightarrow \; v\kappa dP = - dv \;\;\text{ (with constant T)}
                \end{displaymath}}
                For isothermal processes
                \begin{displaymath}
                   w = -\int P dv = - \int P\left(-v\kappa dP\right) = \frc{v}{2}\kappa\left(P_{2}^{2}-P_{1}^{2}\right) = 4.90 \frc{\text{J}}{\text{kg}}
                \end{displaymath}

\end{frame}

%%%
%%% Slide
%%%
%\scriptsize
\begin{frame}
   \frametitle{Example 1}

    \visible<1->{For isothermal processes
                \begin{displaymath}
                   w = -\int P dv = - \int P\left(-v\kappa dP\right) = \frc{v}{2}\kappa\left(P_{2}^{2}-P_{1}^{2}\right) = 4.90 \frc{\text{J}}{\text{kg}}
                \end{displaymath}}
    
   \visible<2->{{\bf (b)} $ds$ = ? (specific entropy).}

   \visible<3->{From the Maxwell relations, 
                \begin{displaymath}
                   -\left(\frc{\partial s}{\partial P}\right)_{T} = \left(\frc{\partial v}{\partial T}\right)_{P},
                \end{displaymath}} 

   \visible<4->{and from the definition of $\beta$,
                \begin{eqnarray}
                    && \beta = \frc{1}{v}\left(\frc{\partial v}{\partial T}\right)_{P} \;\;\Longrightarrow\;\; -\left(\frc{\partial s}{\partial P}\right)_{T} = \beta v \nonumber \\
                    && ds = -\beta v dP \;\;\Longrightarrow ds = s_{2}-s_{1} = -\beta v \left(P_{2}-P_{1}\right) = -0.5694 \frc{\text{J}}{\text{kg.K}} \nonumber
                \end{eqnarray}}

\end{frame}

%%%
%%% Slide
%%%
%\scriptsize
\begin{frame}
   \frametitle{Example 1}

    \visible<1->{{\bf (c)} The heat transferred in such reversible isothermal process is
                \begin{displaymath}
                   dq = Tds \;\;\Longrightarrow q = T\left(s_{2}-s_{1}\right) = -164.07 \frc{\text{J}}{\text{kg}}.
                \end{displaymath}}

    \visible<1->{{\bf (d)} The specific internal energy,
                \begin{eqnarray}
                   &&  du = q + w \nonumber \\
                   && \left(u_{2}-u_{1}\right) = \underbrace{-164.07}_{\text{heat removed from the system}} + \overbrace{4.90}^{\text{work given to the system}} = -159.17 \frc{\text{J}}{\text{kg}}. \nonumber
                \end{eqnarray}}

\end{frame}



%%%
%%% Slide
%%%
%\scriptsize
\begin{frame}
   \frametitle{Example 2}
    \blue{Demonstrate that the derivative of molar volume \wrt temperature at constant pressure is}
     \begin{displaymath}
         \blue{\Partial[V]{T}{P} = -\frc{\Partial[P]{T}{V}}{\Partial[P]{V}{T}},}
     \end{displaymath}
     \blue{and obtain an expression for $\Partial[V]{T}{P}$ for the van der Waals EOS.} \\

   \blue{ {\bf Hint:} You should start the proof from the total differential of a continuous function $f(a,b)$,}
     \begin{displaymath}
         \blue{df = \Partial[f]{a}{b}da + \Partial[f]{b}{a}db.}
     \end{displaymath}

\end{frame}


%%%
%%% Slide
%%%
%\scriptsize
\begin{frame}
   \frametitle{Example 2}

    \noindent{\bf Solution:} 
    \visible<1->{The total differential of a generic continuous function $f(a,b)$ is
       \begin{displaymath}
           df = \Partial[f]{a}{b}da + \Partial[f]{b}{a}db.
       \end{displaymath}
       where (from the given thermodynamic function) $f=P$, $a=T$ and $b=V$, \ie}

    \visible<2->{\begin{displaymath}
          dP = \Partial[P]{T}{V}dT + \Partial[P]{V}{T}dV.
       \end{displaymath}}

    \visible<3->{However we want a differential expression in which $P$ is constant, therefore} \visible<4->{$\mathbf{dP = 0}$,}
       \begin{eqnarray}
           \visible<5->{0 &=& \Partial[P]{T}{V}dT + \Partial[P]{V}{T}dV \;\;\;\text{ at } P \text{ constant},}\nonumber \\
           \visible<6->{\Partial[V]{T}{P} &=& -\frc{\Partial[P]{T}{V}}{\Partial[P]{V}{T}}.} \nonumber
       \end{eqnarray}

\end{frame}

%%%
%%% Slide
%%%
%\scriptsize
\begin{frame}
   \frametitle{Example 2}

    \visible<1->{Now we want to obtain a differential expression for the vdW-EOS,}
    \visible<2->{\begin{displaymath}
          P = \frc{RT}{V-b} - \frc{a}{V^{2}},
       \end{displaymath}
       where $V$ is the molar volume and $a$ and $b$ are constants that {\it depends only on critical properties}, $P_{c}$ and $T_{c}$.} 

    \visible<3->{Due to the non-linearity of this EOS, obtaining $\Partial[V]{T}{P}$ from a direct differentiation would be difficult. However, now we know an expression that cn help us,}

    \visible<4->{\begin{displaymath}
         \Partial[V]{T}{P} = -\frc{\Partial[P]{T}{V}}{\Partial[P]{V}{T}} = -\frc{\frc{R}{V-b}}{-\frc{RT}{\left(V-b\right)^{2}+\frc{2a}{V^{3}}}}
     \end{displaymath} }

\end{frame}


%%%
%%% Slide
%%%
%\scriptsize
\begin{frame}
   \frametitle{Example 3}
    \blue{Derive an expression for enthalpy change of a gas during an isothermal process assuming using the following EOS: $P\left(V-b\right)=RT$}

    \noindent\visible<2->{{\bf Solution:} We have seen that enthalpy change is given by (see Slide 8 or Eqn. 36a from the {\it Notes})
    \begin{displaymath}
       dH = C_{p}dT + \left[V - T\Partial[V]{T}{P}\right]dP.
    \end{displaymath}}

    \visible<3->{We can rearrange the EOS and obtain $\Partial[V]{T}{P}$, (or, for a more complex EOS, we could use the procedure of Exmaple 2 to obtain this differential)}
    \begin{eqnarray}
       \visible<3->{&& P\left(V-b\right)=RT \;\;\;\rightarrow\;\;\; V = \frc{RT}{P} + b \;\;\;\rightarrow\;\;\; \Partial[V]{T}{P} = \frc{R}{P}\;\;\text{ thus, }} \nonumber \\
       \visible<3->{&& dH = C_{p}dT + \left(V - \frc{RT}{P}\right)dP = \blue{C_{p}dT + bdP}.} \nonumber 
    \end{eqnarray}

\end{frame}

%%%
%%% Slide
%%%
%\scriptsize
\begin{frame}
   \frametitle{Example 4}
    \blue{The Antoine equation constants for toluene are $A=14.01415$, $B=3106.46$ K and $C=-53.15$ K (for pressure given in kPa). At 1.01325$\times$10$^{5}$ Pa, calculate:}
        \begin{enumerate}[(a)]
           \item \blue{the boiling temperature and;}
           \item \blue{the enthalpy of vaporisation at these conditions.}
        \end{enumerate}

    \noindent\visible<2->{{\bf Solution:} }
       \begin{enumerate}[a)]
%
           \item<2->Boiling temperature can be calculated from the Antoine equation,
               \begin{eqnarray}
                   \visible<3->{&& \ln{P^{\text{sat}}} = A - \frc{B}{T+C}} \nonumber \\
                   \visible<4->{&& T = \frc{B}{A-\ln{P^{\text{sat}}}} - C = \blue{383.77 \text{ K}}}  \nonumber          
               \end{eqnarray}
%
           \item<5-> The enthalpy of vaporisation, $\Delta H^{\text{fg}}$, can be obtained from the Clausius-Clapeyron equation,
               \begin{eqnarray}
                   && \visible<5->{\frc{d}{dT} \left(\ln{P^{\text{sat}}}\right) = \frc{\Delta H^{\text{fg}}}{RT^{2}}} \nonumber \\
                   && \visible<6->{\frc{B}{\left(T+C\right)^{2}} =  \frc{\Delta H^{\text{fg}}}{RT^{2}} \;\;\Longrightarrow \Delta H^{\text{fg}} = \blue{34.7984 \text{ kJ.mol}^{-1}}.} \nonumber
               \end{eqnarray}
%
        \end{enumerate}

\end{frame}

%%%
%%% Slide
%%%
%\scriptsize
\begin{frame}
   \frametitle{Example 5}
    \blue{Steam (dry and saturated) is supplied by the boiler at 15 bar and the condenser inlet pressure is 0.4 bar. Calculate the Rankine efficiency of the cycle. Neglect the pump work, assume the enthalpy of fluid leaving the pump is 317.58 kJ.kg$^{-1}$.}
    \noindent\visible<2->{{\bf Solution:} For this problem, let's assume the same numbering of Fig. 7a in the Lecture-Notes.}\visible<3->{ At 15 bar, dry and saturated $\left(\ie\; x_{1}=1\right)$ steam} \visible<4->{has the following properties (from saturated table),}
           \visible<4->{\begin{eqnarray}
             T_{1} &=& T_{\text{sat}} = 198.3^{\circ}\text{C},\nonumber \\
             h_{1} &=& h_{\text{g}} = 2792.2\; \text{kJ.kg}^{-1} \nonumber \\
             s_{1} &=& s_{\text{g}} = 6.4448\; \text{kJ.(kg.K)}^{-1} \nonumber
          \end{eqnarray} }
          \visible<5->{In the condenser, $P_{2}=0.4$ bar,
          \begin{eqnarray}
              T_{2} &=& T_{\text{sat}} = 75.87^{\circ}\text{C}, \nonumber \\
              h_{\text{g}2} &=& 2636.8\;\text{kJ.kg}^{-1},\;\;\; h_{\text{f}2} = 317.58\;\text{kJ.kg}^{-1},  \nonumber \\
              s_{\text{g}2} &=& 7.6700 \;\text{kJ.(kg.K)}^{-1},\;\;\; s_{\text{f}2} = 1.0259\;\text{kJ.(kg.K)}^{-1}. \nonumber  
          \end{eqnarray}}
\end{frame}

%%%
%%% Slide
%%%
%\scriptsize
\begin{frame}
   \frametitle{Example 5}
         \visible<1-> {$h_{2}$ and $s_{2}$ depend on the knowledge of how vaporised the water is, in other words, we need to determine the quality of the steam, $x_{1}$ through},
         \begin{displaymath}
              \visible<2->{M = \mfr[M]{}{L} + \mfr[x]{}{V}\Delta\mfr[M]{}{LV}}
              \visible<3->{ \begin{cases}
                      h_{2} = h_{\text{f}2} + x_{2}\left(h_{\text{g}2} - h_{\text{f}2}\right), \\
                      s_{2} = s_{\text{f}2} + x_{2}\left(s_{\text{g}2} - s_{\text{f}2}\right).  
              \end{cases}}
         \end{displaymath}

         \visible<4-> {As we know that water is expanded isentropically in the turbine, \ie \blue{$s_{1}=s_{2}$},}
         \begin{eqnarray}
             && \visible<5-> {s_{2} = s_{\text{f}2} + x_{2}\left(s_{\text{g}2} - s_{\text{f}2}\right) = s_{1} = 6.4448}  \nonumber \\
             && \visible<6-> {x_{2} = 0.8156 \;\;(81.56\% \text{ of vapour})}\nonumber
         \end{eqnarray}

         \visible<7-> {Thus replacing in 
             \begin{displaymath}
                h_{2} = h_{\text{f}2} + x_{2}\left(h_{\text{g}2} - h_{\text{f}2}\right) = 2209.14\text{ kJ.kg}^{-1}.
             \end{displaymath}}

         \visible<8-> {The Rankine efficiency is given by}
             \begin{displaymath}
                 \visible<8-> {\eta_{\text{Rankine}} = \frc{\text{Adiabatic or Isentropic Heat Drop}}{\text{Heat Supplied}} =} \visible<9-> {\frc{\left|h_{1}-h_{2}\right|}{h_{1}-h_{\text{f}4}} = 0.2356\;\;\;\rightarrow \;\;\; 23.56\%}
             \end{displaymath}
\end{frame}



\end{document}
 

%
% Aberdeen style guide should be followed when using this
% layout. Their template powerpoint slide is used to extract the
% Aberdeen color and logo but is otherwise ignored (it has little or
% no formatting in it anyway).   
% 
% http://www.abdn.ac.uk/documents/style-guide.pdf

%%%%%%%%%%%%%%%%%%%% Document Class Settings %%%%%%%%%%%%%%%%%%%%%%%%%
% Pick if you want slides, or draft slides (no animations)
%%%%%%%%%%%%%%%%%%%%%%%%%%%%%%%%%%%%%%%%%%%%%%%%%%%%%%%%%%%%%%%%%%%%%%
%Normal document mode% 
\documentclass[10pt,compress,unknownkeysallowed]{beamer}
%Draft or handout mode 
%\documentclass[10pt,compress,handout,unknownkeysallowed]{beamer}
%\documentclass[10pt,compress,handout,ignorenonframetext,unknownkeysallowed]{beamer}


%%%%%%%%%%%%%%%%%%%% General Document settings %%%%%%%%%%%%%%%%%%%%%%%
% These settings must be set for each presentation
%%%%%%%%%%%%%%%%%%%%%%%%%%%%%%%%%%%%%%%%%%%%%%%%%%%%%%%%%%%%%%%%%%%%%%
\newcommand{\shortname}{jefferson.gomes@abdn.ac.uk}
\newcommand{\fullname}{Dr Jeff Gomes}
\institute{School of Engineering}
\newcommand{\emailaddress}{}%jefferson.gomes@abdn.ac.uk}
\newcommand{\logoimage}{../../FigBanner/UoAHorizBanner}
\title{Chemical Thermodynamics (EX3029)}
\subtitle{Module 4: Solution Thermodynamics}
\date[ ]{ }



%%%%%%%%%%%%%%%%%%%%%%%%%%%%%%%%%%%%%%%%%%%%%%%%%%%%%%%%%%%%%%%%%%%%%%%%%%%%%%%
% BABEL and LANGUAGES %%%%%%%%%%%%%%%%%%%%%%%%%%%%%%%%%%%%%%%%%%%%%%%%%%%%%%%%%
%%%%%%%%%%%%%%%%%%%%%%%%%%%%%%%%%%%%%%%%%%%%%%%%%%%%%%%%%%%%%%%%%%%%%%%%%%%%%%%
% \usepackage{listings}                   % it is a source code printer for LATEX
                                          % \lstset{language=Python}
                                          % \lstinputlisting{source.py}   % command used to pretty-print stand alone files
\usepackage[english]{babel}               % [french, frenchb, english, ]
    % http://forum.mathematex.net/latex-f6/les-puces-avec-babel-t4256.html
    % http://www.grappa.univ-lille3.fr/FAQ-LaTeX/11.1.html


%%%%%%%%%%%%%%%%%%%%%%%%%%%%%%%%%%%%%%%%%%%%%%%%%%%%%%%%%%%%%%%%%%%%%%%%%%%%%%%
% FONTS and ENCODING %%%%%%%%%%%%%%%%%%%%%%%%%%%%%%%%%%%%%%%%%%%%%%%%%%%%%%%%%%
%%%%%%%%%%%%%%%%%%%%%%%%%%%%%%%%%%%%%%%%%%%%%%%%%%%%%%%%%%%%%%%%%%%%%%%%%%%%%%%
%
% See:
% http://tex.stackexchange.com/questions/59702/suggest-a-nice-font-family-for-my-basic-latex-template-text-and-math-i-am
%

\usepackage{lmodern}        % Latin Modern family of fonts. Very much like Computer Modern, but with many more glyphs 
                            % (e.g., for characters with accents, glyphs, cedillas, etc)
\usepackage[T1]{fontenc}    % fontenc is oriented to output, that is, what fonts to use for printing characters. 
                            % http://tex.stackexchange.com/questions/44694/fontenc-vs-inputenc 
                            % http://tex.stackexchange.com/questions/664/why-should-i-use-usepackaget1fontenc

% Change some fonts or the whole font family (i.e. serif, sans serif, monospace, and 'math')
    % \usepackage[varg, cmintegrals, cmbraces, ]{newtxtext,newtxmath}  % Other options: libertine, uprightGreek (U.S.) or slantedGreek (ISO), etc...
     \usepackage{tgtermes}                                            % Only serif ("TeX-Gyre" text)
    % \usepackage{kpfonts}                                             % "Kepler" fonts
    % \usepackage{mathpazo}                                            % Based on Hermann Zapf's Palatino font
    % \usepackage{txfonts}                                             % More than a decade old
    % \usepackage{pslatex}                                             % Obsolete?
    %  - \usepackage{mathptmx}
    %  - \usepackage[scaled=.90]{helvet}
    %  - \usepackage{courier}

% \usepackage{textcomp}     % required for special glyphs
% \usepackage{bm}           % load after all math to give access to bold math
\usepackage[utf8]{inputenc} % inputenc allows the user to input accented characters directly from the keyboard; 
                            % utf8x : much broader but less compatible ; latin1 : old?
                            % http://tex.stackexchange.com/questions/44694/fontenc-vs-inputenc

% See:
% http://tex.stackexchange.com/questions/59626/nicely-force-66-characters-per-line
%
% pslatex is a very obsolete package and that its descendant mathptmx is rather inadequate for serious typesetting involving math.
% If you don't need mathematics, other choices based on (Linotype) Times Roman are
%  - tgtermes
%  - newtxtext (based on txfonts, but with corrected metrics) (with its companion math package newtxmath)
%
%
% See:
% http://www.latex-community.org/forum/viewtopic.php?f=8&t=6637
%
% (times, helvet, courier)
% pslatex and txfonts produce (almost) same resutls.
% pslatex supposedly obsolete
% txfonts supposedly up-to-date
%
%
% See:
% ftp://ftp.rrzn.uni-hannover.de/pub/mirror/tex-archive/info/l2tabu/english/l2tabuen.pdf
% or 
% ftp://ftp.dante.de/tex-archive/info/l2tabu/english/l2tabuen.pdf
% in
% 2.3.3 pslatex.sty
%
% pslatex uses a Courier font scaled too narrowly.
% Its main disadvantage is that it does not work with T1 and TS1 encodings.
% So replace:
% \usepackage{pslatex} or \usepackage{txfonts}
% by all three:
% - \usepackage{mathptmx}
% - \usepackage[scaled=.90]{helvet}
% - \usepackage{courier}
%
%
% See:
% http://xpt.sourceforge.net/techdocs/language/latex/latex32-LaTeXAndFonts/single/
% or http://thirteen-01.stat.iastate.edu/wiki/LaTeXFonts
% or http://www.tex.ac.uk/tex-archive/info/beginlatex/html/chapter8.html
%
% When changing fonts, you can change all of the default fonts at once with the following commands:
% 
% Command     Changes the defaults to
% 
% times       Times, Helvetica, Courier
% pslatex     same as Times, but uses a specially narrowed Courier. This is preferred over Times because of the way it handles Courier.
% newcent     New Century Schoolbook, Avant Garde, Courier
% palatino    Palatino, Helevetica, Courier
% palatcm     changes the Roman to Palatino only, but uses CM mathematics
% kpfonts     "Kepler" fonts. A very nicely evolved set of fonts also based originally on Palatino, but with many special features.
%
%
% See:
% http://tex.stackexchange.com/questions/59702/suggest-a-nice-font-family-for-my-basic-latex-template-text-and-math-i-am
%
% There are, of course, many other font packages, most of which provide "only" text-mode fonts.
% Among these are the "TeX-Gyre" font families: 
%  - Termes (a Times Roman clone), 
%  - Pagella (a Palatino clone), and 
%  - Schola (a Century Schoolbook clone); 
% one would load the packages tgtermes, tgpagella, and tgschola, respectively, to access these fonts.
% However, as these are text fonts, you still need to choose a suitable math font.
% 
% Still another possibility you may want to look into is the Linux Libertine font family, to be loaded via the libertine-legacy package.
% If you like this text font and wish to employ the newtxmath package, be sure to load the newtxmath package with the libertine option set;
% doing so will set up a special set of math-mode fonts that harmonizes well with the libertine text fonts.
% 
%
% See also:
% http://tex.stackexchange.com/questions/56876/times-new-roman-fonts-and-maths-without-mathptmx
%
%
% For a comparison, in:
% /home/christophe/Personal/Truc_Et_Astuce_Informatik/LaTeX/comparison_font_types/,
% see: 
% computer.pdf  lmodern.pdf  pslatex.pdf  test_font_type.pdf  three_replacements.pdf  txfonts.pdf
%


%%%%%%%%%%%%%%%%%%%%%%%%%%%%%%%%%%%%%%%%%%%%%%%%%%%%%%%%%%%%%%%%%%%%%%%%%%%%%%%
% AMS MATH %%%%%%%%%%%%%%%%%%%%%%%%%%%%%%%%%%%%%%%%%%%%%%%%%%%%%%%%%%%%%%%%%%%%
%%%%%%%%%%%%%%%%%%%%%%%%%%%%%%%%%%%%%%%%%%%%%%%%%%%%%%%%%%%%%%%%%%%%%%%%%%%%%%%
% \usepackage{amsmath}      % loads amstext, amsbsy, amsopn but not amssymb
                            % equation stuff (eqref, subequations, equation, align, gather, flalign, multline, alignat, split...)
% \usepackage{amsfonts}     % may be redundant with amsmath
% \usepackage{amssymb}      % may be redundant with amsmath
% \numberwithin{equation}{section}  % reset equation counters at start of each "section" and prefix numbers by section number
% \numberwithin{figure}{section}    % reset figure   counters at start of each "section" and prefix numbers by section number


%%%%%%%%%%%%%%%%%%%%%%%%%%%%%%%%%%%%%%%%%%%%%%%%%%%%%%%%%%%%%%%%%%%%%%%%%%%%%%%
% LAY OUT %%%%%%%%%%%%%%%%%%%%%%%%%%%%%%%%%%%%%%%%%%%%%%%%%%%%%%%%%%%%%%%%%%%%%
%%%%%%%%%%%%%%%%%%%%%%%%%%%%%%%%%%%%%%%%%%%%%%%%%%%%%%%%%%%%%%%%%%%%%%%%%%%%%%%
%
% See:
% http://tex.stackexchange.com/questions/59626/nicely-force-66-characters-per-line
% (must be after pslatex, tgterms, etc...)
%
% a) (but works mostly for a4paper, and changes top and bottom margin too...)
% \usepackage[DIV=calc]{typearea}
%
% or
%
% b) (but you have to choose the value and the margin ratio depending on the class...)
% \newlength{\alphabet}
% \settowidth{\alphabet}{\normalfont abcdefghijklmnopqrstuvwxyz}
% \usepackage{geometry}
% \geometry{%
% textwidth=2.5\alphabet,% (Note: 2.5 * 26 = 65)
% hmarginratio={2:3}}    % (Problem: geometry uses 2:3 as default for twoside and 1:1 for oneside,
%                        % independently of what the class thinks about the margins)

% \usepackage{layout}       % use \layout in the tex file to see the values
% \usepackage{layouts}      % it extends the functionality of layout, allowing you to do much, much more
                            % some commands: \pagelayout, \pagevalues, \pagedesign, ...
% \usepackage[cm]{fullpage} % set 'default' full page
% \usepackage{geometry}     % very customizable margins. Under some (rare) circumstances, should be loaded after hyperref
% \usepackage{anysize}      % \marginsize{left}{right}{top}{bottom}
% \usepackage{pdflscape}    % include landscape layout pages (automatically rotate pages in pdf file for easier reading)
% \usepackage{multicol}     % for multi column environment
\usepackage{lipsum}         % to fill in with arbitrary text
\widowpenalty = 4000        % help suppress widows,  default = 4,000 (?), from 0 to 10 000 (from 300 to 1 000 recommended, 10 000 not recommended)
\clubpenalty  = 4000        % help suppress orphans, default = 4,000 (?), from 0 to 10 000 (from 300 to 1 000 recommended, 10 000 not recommended)
\usepackage[final, babel]{microtype} % many good lay-out/justification effects, see:
                                     % texblog.net/latex-archive/layout/pdflatex-microtype/


%%%%%%%%%%%%%%%%%%%%%%%%%%%%%%%%%%%%%%%%%%%%%%%%%%%%%%%%%%%%%%%%%%%%%%%%%%%%%%%
% EMBED FILEs %%%%%%%%%%%%%%%%%%%%%%%%%%%%%%%%%%%%%%%%%%%%%%%%%%%%%%%%%%%%%%%%%
%%%%%%%%%%%%%%%%%%%%%%%%%%%%%%%%%%%%%%%%%%%%%%%%%%%%%%%%%%%%%%%%%%%%%%%%%%%%%%%
\usepackage{embedfile}    % embed (attach) any files (eg tex source) to a PDF document.
                          % Currently only supported driver is pdfTEX >= 1.30 in PDF mode
%\embedfile{to_post.tex}


%%%%%%%%%%%%%%%%%%%%%%%%%%%%%%%%%%%%%%%%%%%%%%%%%%%%%%%%%%%%%%%%%%%%%%%%%%%%%%%
% EASY EDITS %%%%%%%%%%%%%%%%%%%%%%%%%%%%%%%%%%%%%%%%%%%%%%%%%%%%%%%%%%%%%%%%%%
%%%%%%%%%%%%%%%%%%%%%%%%%%%%%%%%%%%%%%%%%%%%%%%%%%%%%%%%%%%%%%%%%%%%%%%%%%%%%%%
\usepackage{ifdraft}        % ask for selective behavior depending on the draft option (used for waterdraftmark, not draftmark)
% \usepackage{comment}      % provide new {comment} environment: all text inside the environment is ignored.
% \usepackage{fixme}        % allow nice comment / warning system, displayed in draft mode in right margin ; % [status=draft]
% \usepackage{lineno}       % number all lines in left margin if activated with \linenumbers
% \linenumbers


%%%%%%%%%%%%%%%%%%%%%%%%%%%%%%%%%%%%%%%%%%%%%%%%%%%%%%%%%%%%%%%%%%%%%%%%%%%%%%%
% GRAPHICX %%%%%%%%%%%%%%%%%%%%%%%%%%%%%%%%%%%%%%%%%%%%%%%%%%%%%%%%%%%%%%%%%%%%
%%%%%%%%%%%%%%%%%%%%%%%%%%%%%%%%%%%%%%%%%%%%%%%%%%%%%%%%%%%%%%%%%%%%%%%%%%%%%%%
% \usepackage[final]{graphicx} % options = [final]  = all graphics displayed, regardless of draft option in class
                               % options = [pdftex] = necessary (?) if import PDF files
                               % no option : when importing ps- and eps-files (?)
% \graphicspath{{../images/}}  % tell LaTeX where to look for images
% \DeclareGraphicsExtensions{.pdf, .PDF, .jpg, .JPG, .jpeg, .JPEG, .png, .PNG, .bmp, .BMP, .eps, .ps}
\usepackage{float}                      % Improved interface for floating objects ; add [H] option


%%%%%%%%%%%%%%%%%%%%%%%%%%%%%%%%%%%%%%%%%%%%%%%%%%%%%%%%%%%%%%%%%%%%%%%%%%%%%%%
% FILIGREE %%%%%%%%%%%%%%%%%%%%%%%%%%%%%%%%%%%%%%%%%%%%%%%%%%%%%%%%%%%%%%%%%%%%
%%%%%%%%%%%%%%%%%%%%%%%%%%%%%%%%%%%%%%%%%%%%%%%%%%%%%%%%%%%%%%%%%%%%%%%%%%%%%%%
% draftmark : newer and better package but not on Phil's computers,
% in particular, draftmark has a "ifdraft" option included...
%
\ifdraft{
\usepackage{draftwatermark} % add watermark ("draft", "confidential"...)
                            % option: [firstpage] (insert on only the first page)
\SetWatermarkText{COPY~---~DRAFT}
\SetWatermarkAngle{55}
\SetWatermarkScale{6.0}
\SetWatermarkLightness{0.85}
\SetWatermarkFontSize{12 pt}
}{}


\renewcommand{\insertframenumber}{\theframenumber}
\renewcommand{\theframenumber}{\thesection-\arabic{framenumber}}
\renewcommand{\thesubsectionslide}{\thesection-\arabic{framenumber}}
\setbeamertemplate{headline}[text line]{This is frame: \insertframenumber}
\AtBeginSection{\setcounter{framenumber}{0}}


%%%%%%%%%%%%%%%%%%%% Template settings %%%%%%%%%%%%%%%%%%%%%%%%%%%%%%%
% You shouldn't have to change below this line, unless you want to.
%%%%%%%%%%%%%%%%%%%%%%%%%%%%%%%%%%%%%%%%%%%%%%%%%%%%%%%%%%%%%%%%%%%%%%
\usecolortheme{whale}
\useoutertheme{infolines}

% Use the fading effect for items that are covered on the current
% slide.
\beamertemplatetransparentcovered

% We abuse the author command to place all of the slide information on
% the title page.
\author[\shortname]{%
  \fullname\\\ttfamily{\emailaddress}
}


%At the start of every section, put a slide indicating the contents of the current section.
\AtBeginSection[] {
  \begin{frame}
    \frametitle{Section Outline}
    \tableofcontents[currentsection]
  \end{frame}
}

% Allow the inclusion of movies into the Presentation! At present,
% only the Okular program is capable of playing the movies *IN* the
% presentation.
\usepackage{multimedia}
\usepackage{animate}

%% Handsout -- comment out the lines below to create handstout with 4 slides in a page with space for comments
\usepackage{handoutWithNotes}

\mode<handout>
{
\usepackage{pgf,pgfpages}

\pgfpagesdeclarelayout{2 on 1 boxed with notes}
{
\edef\pgfpageoptionheight{\the\paperheight} 
\edef\pgfpageoptionwidth{\the\paperwidth}
\edef\pgfpageoptionborder{0pt}
}
{
\setkeys{pgfpagesuselayoutoption}{landscape}
\pgfpagesphysicalpageoptions
    {%
        logical pages=4,%
        physical height=\pgfpageoptionheight,%
        physical width=\pgfpageoptionwidth,%
        last logical shipout=2%
    } 
\pgfpageslogicalpageoptions{1}
    {%
    border code=\pgfsetlinewidth{1pt}\pgfstroke,%
    scale=1,
    center=\pgfpoint{.25\pgfphysicalwidth}{.75\pgfphysicalheight}%
    }%
\pgfpageslogicalpageoptions{2}
    {%
    border code=\pgfsetlinewidth{1pt}\pgfstroke,%
    scale=1,
    center=\pgfpoint{.25\pgfphysicalwidth}{.25\pgfphysicalheight}%
    }%
\pgfpageslogicalpageoptions{3}
    {%
    border shrink=\pgfpageoptionborder,%
    resized width=.7\pgfphysicalwidth,%
    resized height=.5\pgfphysicalheight,%
    center=\pgfpoint{.75\pgfphysicalwidth}{.29\pgfphysicalheight},%
    copy from=3
    }%
\pgfpageslogicalpageoptions{4}
    {%
    border shrink=\pgfpageoptionborder,%
    resized width=.7\pgfphysicalwidth,%
    resized height=.5\pgfphysicalheight,%
    center=\pgfpoint{.75\pgfphysicalwidth}{.79\pgfphysicalheight},%
    copy from=4
    }%

\AtBeginDocument
    {
    \newbox\notesbox
    \setbox\notesbox=\vbox
        {
            \hsize=\paperwidth
            \vskip-1in\hskip-1in\vbox
            {
                \vskip1cm
                Notes\vskip1cm
                        \hrule width\paperwidth\vskip1cm
                    \hrule width\paperwidth\vskip1cm
                        \hrule width\paperwidth\vskip1cm
                    \hrule width\paperwidth\vskip1cm
                        \hrule width\paperwidth\vskip1cm
                    \hrule width\paperwidth\vskip1cm
                    \hrule width\paperwidth\vskip1cm
                    \hrule width\paperwidth\vskip1cm
                        \hrule width\paperwidth
            }
        }
        \pgfpagesshipoutlogicalpage{3}\copy\notesbox
        \pgfpagesshipoutlogicalpage{4}\copy\notesbox
    }
}
}

%\pgfpagesuselayout{2 on 1 boxed with notes}[letterpaper,border shrink=5mm]
%\pgfpagesuselayout{2 on 1 boxed with notes}[letterpaper,border shrink=5mm]


%%%%%%%%%% Chemical Reactions %%%%%%%%%%%%%%%%

\usepackage[T1]{fontenc}
\usepackage[utf8]{inputenc}
\usepackage{lmodern}
\usepackage[version=3]{mhchem}
\makeatletter
\newcounter{reaction}
%%% >> for article <<
%\renewcommand\thereaction{C\,\arabic{reaction}}
%%% << for article <<
%%% >> for report and book >>
%\renewcommand\thereaction{C\,\thechapter.\arabic{reaction}}
%\@addtoreset{reaction}{chapter}
%%% << for report and book <<
\newcommand\reactiontag{\refstepcounter{reaction}\tag{\thereaction}}
\newcommand\reaction@[2][]{\begin{equation}\ce{#2}%
\ifx\@empty#1\@empty\else\label{#1}\fi%
\reactiontag\end{equation}}
\newcommand\reaction@nonumber[1]{\begin{equation*}\ce{#1}%
\end{equation*}}
\newcommand\reaction{\@ifstar{\reaction@nonumber}{\reaction@}}
\makeatother

%%%%%%%%%%%%%%%%%%%%%%%%%%%%%%%%%%%%%%%%%%%%%%


%%%%% Color settings
\usepackage{color}
%% The background color for code listings (i.e. example programs)
\definecolor{lbcolor}{rgb}{0.9,0.9,0.9}%
\definecolor{UoARed}{rgb}{0.64706, 0.0, 0.12941}
\definecolor{UoALight}{rgb}{0.85, 0.85, 0.85}
\definecolor{UoALighter}{rgb}{0.92, 0.92, 0.92}
\setbeamercolor{structure}{fg=UoARed} % General background and higlight color
\setbeamercolor{frametitle}{bg=black} % General color
\setbeamercolor{frametitle right}{bg=black} % General color
\setbeamercolor{block body}{bg=UoALighter} % For blocks
\setbeamercolor{structure}{bg=UoALight} % For blocks
% Rounded boxes for blocks
\setbeamertemplate{blocks}[rounded]

%%%%% Font settings
% Aberdeen requires the use of Arial in slides. We can use the
% Helvetica font as its widely available like so
% \usepackage{helvet}
% \renewcommand{\familydefault}{\sfdefault}
% But beamer already uses a sans font, so we will stick with that.

% The size of the font used for the code listings.
\newcommand{\goodsize}{\fontsize{6}{7}\selectfont}

% Extra math packages, symbols and colors. If you're using Latex you
% must be using it for formatting the math!
\usepackage{amscd,amssymb} \usepackage{amsfonts}
\usepackage[mathscr]{eucal} \usepackage{mathrsfs}
\usepackage{latexsym} \usepackage{amsmath} \usepackage{bm}
\usepackage{amsthm} \usepackage{textcomp} \usepackage{eurosym}
% This package provides \cancel{a} and \cancelto{a}{b} to "cancel"
% expressions in math.
\usepackage{cancel}

\usepackage{comment} 

% Get rid of font warnings as modern LaTaX installations have scalable
% fonts
\usepackage{type1cm} 

%\usepackage{enumitem} % continuous numbering throughout enumerate commands

% For exact placement of images/text on the cover page
\usepackage[absolute]{textpos}
\setlength{\TPHorizModule}{1mm}%sets the textpos unit
\setlength{\TPVertModule}{\TPHorizModule} 

% Source code formatting package
\usepackage{listings}%
\lstset{ backgroundcolor=\color{lbcolor}, tabsize=4,
  numberstyle=\tiny, rulecolor=, language=C++, basicstyle=\goodsize,
  upquote=true, aboveskip={1.5\baselineskip}, columns=fixed,
  showstringspaces=false, extendedchars=true, breaklines=false,
  prebreak = \raisebox{0ex}[0ex][0ex]{\ensuremath{\hookleftarrow}},
  frame=single, showtabs=false, showspaces=false,
  showstringspaces=false, identifierstyle=\ttfamily,
  keywordstyle=\color[rgb]{0,0,1},
  commentstyle=\color[rgb]{0.133,0.545,0.133},
  stringstyle=\color[rgb]{0.627,0.126,0.941}}

% Allows the inclusion of other PDF's into the final PDF. Great for
% attaching tutorial sheets etc.
\usepackage{pdfpages}
\setbeamercolor{background canvas}{bg=}  

% Remove foot note horizontal rules, they occupy too much space on the slide
\renewcommand{\footnoterule}{}

% Force the driver to fix the colors on PDF's which include mixed
% colorspaces and transparency.
\pdfpageattr {/Group << /S /Transparency /I true /CS /DeviceRGB>>}

% Include a graphics, reserve space for it but
% show it on the next frame.
% Parameters:
% #1 Which slide you want it on
% #2 Previous slides
% #3 Options to \includegraphics (optional)
% #4 Name of graphic
\newcommand{\reserveandshow}[4]{%
\phantom{\includegraphics<#2|handout:0>[#3]{#4}}%
\includegraphics<#1>[#3]{#4}%
}

\newcommand{\frc}{\displaystyle\frac}
\newcommand{\red}{\textcolor{red}}
\newcommand{\blue}{\textcolor{blue}}
\newcommand{\green}{\textcolor{green}}
\newcommand{\purple}{\textcolor{purple}}
\newcommand{\eg}{{\it e.g., }}
\newcommand{\ie}{{\it i.e., }}
\newcommand{\wrt}{{\it wrt }}
\newcommand{\rhs}{{\it r.h.s. }}
\newcommand{\Partial}[3][error]{\left(\frc{\partial #1}{\partial #2}\right)_{#3}}
\newcommand{\mfr}[3][error]{#1_{#2}^{\left(#3\right)}} 
\newcommand{\summation}[3][error]{\sum\limits_{#2}^{#3}#1}

 
\begin{document}

% Title page layout
\begin{frame}
  \titlepage
  \vfill%
  \begin{center}
    \includegraphics[clip,width=0.8\textwidth]{\logoimage}
  \end{center}
\end{frame}

% Table of contents
\frame{ \frametitle{Slides Outline}
  \tableofcontents
}


%%%%%%%%%%%%%%%%%%%% The Presentation Proper %%%%%%%%%%%%%%%%%%%%%%%%%
% Fill below this line with \begin{frame} commands! It's best to
% always add the fragile option incase you're going to use the
% verbatim environment.
%%%%%%%%%%%%%%%%%%%%%%%%%%%%%%%%%%%%%%%%%%%%%%%%%%%%%%%%%%%%%%%%%%%%%%


%%%
%%% SECTION
%%%
\section{Learning Objectives}

%%%
%%% Slides
%%%
\begin{frame}
 \frametitle{Learning Objectives}
   \begin{enumerate}
     \item<1-> Define fugacity, chemical potential and activity; 
     \item<1-> Discuss chemical potential as a critical element for chemical equilibrium;
     \item<1-> Identify the requirements for an ideal solution;       
     \item<1-> Apply activity models to Raoult's law calculations;
     \item<1-> Derive expressions for activity coefficient based on excess Gibbs energy;
     \item<1-> Describe ideal gas mixtures;
     %\item<1-> Apply Henry's law to estimare fugacities of dilure components;
     \item<1-> Estimate fugacity coefficient through residual and excess properties.
   \end{enumerate}

\end{frame}

%%%
%%% SECTION
%%%
\section{Bibliography}
\begin{frame}
 \frametitle{Suggested References}
  Literature relevant for this module:
  \begin{enumerate}[(i)]
   \item \blue{Chapter 8 of Lecture Notes};
   \item\label{SVN_Book} J.M. Smith, H.C. Van Ness, M.M. Abbott, $\lq$Introduction to Chemical Engineering Thermodynamics', 6$^{th}$ Edition: Chapters 11-12;
   \item\label{Sandle_Book} S.I. Sandler, $\lq$Chemical, Biochemical and Engineering Thermodynamics', 4$^{th}$ Edition: Chapters 9-10;
   \item H. Devoe, $\lq$Thermodynamics and Chemistry, Pearson Education. 2$^{\text{nd}}$ Edition: Chapters 11-12.
  \end{enumerate}
\end{frame}



%%%
%%% SECTION
%%%
\section{Introduction}

%%%
%%% Slides
%%%
\begin{frame}
 \frametitle{Aims and Objectives}
    \begin{enumerate}
        \item<1-> In previous Modules we have focused primarily on thermodynamic systems comprising pure substances and mixtures in VLE;
        \item<1-> In this module, five new thermodynamic functions will be introduced: chemical potential, fugacity, activity, activity coefficient and fugacity coefficient;
        \item<1-> These functions help describe the non-ideality of solutions in the same way as the compressibility factor, $Z$, helped with gases;
        \item<1-> This module will focus on the thermodynamic description of solutions and how these functions can be used to determine the properties of ideal and real liquid solutions. 
   \end{enumerate}
\end{frame}


%%%
%%% SECTION
%%%
\section{Introducing New Thermodynamic Functions}

%%%
%%% SUBSECTION
%%%
\subsection{Partial Molar Properties}

%%%
%%% Slide
%%%
%\scriptsize
\begin{frame}
  \frametitle{Partial Molar Properties: Definition}
  \begin{enumerate}%\setcounter{enumi}{8}
    \item<1-> In \blue{multi-component systems}, the total value of any {\bf extensive property, $M^{t}$} $\left(M\equiv V,U, H, S, A, G\right)$, is not \underline{only} dependent on $T$ and $P$, but also on the \blue{number of moles of each species} present in the system;
    \item<2-> We can thus write it in functional form as
      \begin{displaymath}
        M^{t}=nM = M\left(T,P,n_{1}, n_{2}, \cdots n_{\mathcal{C}}\right), \hspace{.5cm} \text{ with } n = \summation[n_{i}]{i=1}{\mathcal{C}},
      \end{displaymath}
      where $\mathcal{C}$ is the total number of chemical species in the system, $n_{i}$ is the number of moles of chemical species $i$ and $n$ is the total number of moles;
  \end{enumerate} 
\end{frame}
\normalsize
%%%
%%% Slide
%%%
%\scriptsize
\begin{frame}
  \frametitle{Partial Molar Properties: Definition}
  \begin{enumerate}\setcounter{enumi}{2}
    \item<1-> The total derivative of $M^{t}$,
        \visible<1->{\begin{displaymath}%\label{totalderivative}
           d M^{t} = d\left(n M\right) = \left[\frc{\partial\left(nM\right)}{\partial P}\right]_{T,n}dP + \left[\frc{\partial\left(nM\right)}{\partial T}\right]_{P,n}dT +\textcolor{blue}{\left[\frc{\partial\left(nM\right)}{\partial n_{i}}\right]_{T,P,n_{j}}} dn_{i}
        \end{displaymath}}
        \visible<2->{\begin{block}{\begin{center}\normalsize{Partial Molar Properties }\end{center}}
            The last term in the \rhs is called \red{\it{partial molar property}, $\overline{M}_{i}$},
               \begin{displaymath}
                  \red{\overline{M}_{i}=\left[\frc{\partial\left(nM\right)}{\partial n_{i}}\right]_{T,P,n_{j}}}
               \end{displaymath}
        \end{block}}
    \item<2-> \red{Partial molar property} represents the change of total property \blue{$M^{t}=nM$} of a mixture resulting from addition of an infinitesimal amount of species $i$ to a finite amount of solution (at constant $T$ and $P$);
  \end{enumerate}
\end{frame}
\normalsize

%%%
%%% Slide
%%%
%\scriptsize
\begin{frame}
  \frametitle{Partial Molar Properties: Definition}
  \begin{enumerate}\setcounter{enumi}{4}
    \item<1-> In general, \red{\underline{partial molar properties}} of a chemical species differs from its \blue{\underline{molar properties}} in a pure state in the mixture at the same $T$ and $P$; %in a pure state in the mixture (or solution) at the same $T$ and $P$; 
    \item<2-> This is because in a pure state, the molecules interact with its own species, however;
    \item<3-> In a mixture (\ie solution) it may be subjected to different inter-molecular interactions with other (dissimilar) molecules. 
  \end{enumerate}
\end{frame}
\normalsize

%%%
%%% SUBSECTION
%%%
\subsection{Chemical Potential}  

%%%
%%% Slide
%%%
%\scriptsize
\begin{frame}
  \frametitle{Chemical Potential: Extending Fundamental Relations to Multi-Component Systems} 
  \begin{enumerate}
    \item<1-> The \blue{Fundamental Thermodynamic Relation} for total Gibbs energy \blue{(Module 2)} for closed system stated:
       \visible<1->{\begin{displaymath}
         d\left(n G\right) = \left(n V\right)dP - \left(n S\right)dT
       \end{displaymath}}
     \item<2-> In other words, \blue{Gibbs energy} may be defined as a function of \red{T} and \red{P}. However, let's extend such definition to a \underline{\it single phase multi-component} system, \ie \blue{$G=G(T,P,n)$}:
       \visible<2->{\begin{displaymath}
         d\left(n G\right) = \left[\frc{\partial \left(n G\right)}{\partial P}\right]_{T,n}dP + \left[\frc{\partial \left(n G\right)}{\partial T}\right]_{P,n}dT + \blue{\left.\sum\limits_{i}\left[\frc{\partial \left(n G\right)}{\partial n_{i}}\right]_{P,T,n_{j}}dn_{i}\right.}
       \end{displaymath}}
  \end{enumerate}
\end{frame}
\normalsize
  

%%%
%%% Slide
%%%
%\scriptsize
\begin{frame}
  \frametitle{Chemical Potential: Extending Fundamental Relations to Multi-Component Systems}
        \visible<1->{\begin{block}{\begin{center}\normalsize{Chemical Potential }\end{center}}
            The last term in the \rhs -- \blue{partial molar Gibbs free energy}, is called the \red{chemical potential $\left(\mu_{i}\right)$} of species {\it i}:
               \begin{displaymath}
                  \red{\mu_{i} = \left[\frc{\partial\left(n G\right)}{\partial n_{i}}\right]_{P,T,n_{j}} } 
               \end{displaymath}
        \end{block}} 
        \begin{enumerate}\setcounter{enumi}{2}  
           \item<1-> Therefore, for single phase systems with variable composition,
              \visible<1->{\begin{displaymath}
                d\left(n G\right) = \left(n V\right)dP - \left(n S\right)dT + \sum\limits_{i}\mu_{i} dn_{i};
              \end{displaymath}}
        \end{enumerate}
\end{frame}
\normalsize

%%%
%%% Slide
%%%
%\scriptsize
\begin{frame}
  \frametitle{Chemical Potential: Extending Fundamental Relations to Multi-Component Systems}
        \begin{enumerate}\setcounter{enumi}{3}  
           \item<1-> From the last module, the \blue{stability criteria for phase equilibria} was expressed as a function of the Gibbs energy if the system is held at constant $T$, $P$ and number of moles (see Table 7.1 of Lecture Notes):
              \visible<1->{\begin{displaymath}
                 d\left(n G\right)^{k} = \left(n V\right)^{k}dP - \left(n S\right)^{k}dT + \sum\limits_{i}\mu_{i}^{k} dn_{i}^{k}  \;\;\;\text{ with } k = 1,2,\cdots,\mathcal{P}
              \end{displaymath}
               where $k$ represents an arbitrary phase. Therefore,}
           
           \visible<2->{\begin{block}{\begin{center}\normalsize{Equilibrium Criteria for Multiphase and Multi-Component Systems }\end{center}}
               \blue{Multiphase systems at constant $T$ and $P$ conditions are in equilibrium when the chemical potential of each species $\left(\mu_{i}\right)$ is the same at {\bf all phases},}
                 \begin{displaymath}
                    \red{\mu_{i}^{1} = \mu_{i}^{2} = \cdots = \mu_{i}^{\mathcal{P}}}
                  \end{displaymath} 
               \end{block}}
               
        \end{enumerate}
\end{frame}
\normalsize


%%%
%%% SUBSECTION
%%%
\subsection{Fugacity}  

%%%
%%% Slide
%%%
%\scriptsize
\begin{frame}
  \frametitle{Fugacity: Definition for Pure Fluids}
        \begin{enumerate}%\setcounter{enumi}{3}  
           \item<1-> From the previous slides, we have expressed the {\it Gibbs free energy} as a function of $T$, $P$ and $n$, \ie $G=G(T,P,n)$, and defined the \blue{chemical potential},
              \begin{displaymath}
                dG = \Partial[G]{P}{T,n}dP + \Partial[G]{T}{P,n}dT + \underbrace{\Partial[G]{n}{T,P}}_{\blue{\mu}}dn.
              \end{displaymath}
              The above expression describes the dependency of $G$ on $T$, $P$ and $n$ for \blue{pure fluids};
           \item<2-> For an arbitrary number of components, the \blue{chemical potential} (\ie last term of the rhs) becomes,
             \begin{displaymath}
               \mu_{i} = \overline{G}_{i}=\Partial[G]{n_{i}}{T,P,n_{j} \left(n_{j}\ne n_{i}\right)}, \hspace{0.5cm} \forall i,j=\left\{1,2,\cdots,\mathcal{C}\right\}
               \end{displaymath}
        \end{enumerate}
\end{frame}
\normalsize


%%%
%%% Slide
%%%
%\scriptsize
\begin{frame}
  \frametitle{Fugacity: Definition for Pure Fluids}
        \begin{enumerate}\setcounter{enumi}{2}  
           \item<1-> For \blue{pure fluids}, the {\it chemical potential} can be approximated to
             \begin{displaymath}
               \mu = \frc{G}{n} = \overline{g},
             \end{displaymath}
             where \blue{$\overline{g}$} is the \blue{molar Gibbs free energy};
             
           \item<2-> From the \blue{Maxwell derived relations} (Module 2), a relationship between \blue{G}, \blue{P}, \blue{T} and \blue{V} is (see Eqn. 6.2i from the Lecture Notes)
             \begin{displaymath}
                \Partial[G]{P}{T} = V \visible<3->{= \Partial[n\mu]{P}{T}}
             \end{displaymath}
             \visible<4->{\begin{displaymath}
                \blue{\Partial[\mu]{P}{T} = \overline{v}}, 
             \end{displaymath}
             where $\overline{v}$ is the molar volume;}
               
        \end{enumerate}
\end{frame}
\normalsize


%%%
%%% Slide
%%%
%\scriptsize
\begin{frame}
  \frametitle{Fugacity: Definition for Pure Fluids}
        \begin{enumerate}\setcounter{enumi}{4}  
           \item<1-> For an ideal gas, this expression becomes
              \begin{displaymath}
                 \Partial[\mu^{\text{ig}}]{P}{T} = \frc{RT}{P}, 
              \end{displaymath}
              
           \item<2-> Integrating from initial to final states,
              \begin{displaymath}
                \mu^{\text{ig}} = RT\ln{P} + C(T), 
              \end{displaymath}
              where $C(T)$ is an integration constant, and
              \begin{displaymath}
                \text{at limiting conditions, \ie }
                \visible<3->{\begin{cases}
                  P\rightarrow 0, & \\
                  P\rightarrow \infty &
                \end{cases}}
                \visible<4->{\Longrightarrow -\infty < \mu < \infty,}
              \end{displaymath}               
        \end{enumerate}
\end{frame}
\normalsize

%%%
%%% Slide
%%%
%\scriptsize
\begin{frame}
  \frametitle{Fugacity: Definition for Pure Fluids}
        \begin{enumerate}\setcounter{enumi}{6}  
           \item<1-> However, for a fluid to be considered to behave as an \blue{ideal gas}, {\it pressure} must be relatively low;
           \item<2-> Therefore, the previous expression can also be used to represent the \red{chemical potential of real fluids},
           
             \visible<2->{\begin{block}{\begin{center}\normalsize{ Fugacity of Real Fluids (Devoe {\it et al.})}\end{center}}
                 \begin{displaymath}
                   \blue{\mu = RT\ln{f} + C(T).}
                 \end{displaymath}
                 \red{``Fugacity is the pressure that the ideal gas (\ie the gas with no intermolecular forces) would need to have in order for its chemical potential at the given temperature to be the same as the chemical potential of the real gas''},
                    \begin{displaymath}
                        \begin{cases}
                            \mu^{\text{ig}} = RT \ln{P} + C(T), & \\
                            \mu = RT \ln{f} + C(T). & 
                        \end{cases}
                    \end{displaymath}
                    For ideal gases \blue{$f=P$}, but fugacity for real gases is the equivalent to a pressure that results in $\mu^{\text{ig}}=\mu$.

               \end{block}}
               
        \end{enumerate}
\end{frame}
\normalsize


%%%
%%% Slide
%%%
%\scriptsize
\begin{frame}
  \frametitle{Fugacity: Definition for Pure Fluids}
        \begin{enumerate}\setcounter{enumi}{8}  
           \item<1-> \red{Fugacity} has the same units of pressure and is often referred as {\it fictitious or effective pressure}.
           \item<2-> As pressure tends to zero ($P\rightarrow 0$), the fluid behaves as an ideal gas; 
           \item<3-> Therefore, the \red{fugacity} of a \blue{pure component} is equal to the pressure in the limiting \red{zero pressure}, \ie
               \visible<4->{\begin{block}{\begin{center}\normalsize{ Fugacity at the limiting case for a pure fluid}\end{center}}
                 \begin{displaymath}
                    \blue{ \lim\limits_{P\rightarrow 0}\frc{f}{P} = 1}
                 \end{displaymath}
               \end{block}}
        \end{enumerate}
\end{frame}
\normalsize

%%%
%%% Slide
%%%
%\scriptsize
\begin{frame}
  \frametitle{Fugacity: Lewis-Randall Rule for Mixtures}
        \begin{enumerate} 
           \item<1-> Using the definition of chemical potential and extending it to fugacity using $\mu=RT\ln{f}+C(T)$,
              \visible<2->{
                 \begin{displaymath}
                    \Partial[\mu]{P}{T} = \overline{v} \hspace{.5cm} \Longrightarrow \hspace{.5cm} RT\Partial[\left(\ln{f}\right)]{P}{T} = \overline{v} \hspace{1cm} \text{at constant temperature}
                 \end{displaymath}}
           
           \item<3-> In a mixture of $\mathcal{C}$ components, we can rewrite this equation as
              \visible<3->{
                 \begin{displaymath}
                    \begin{cases}
                       RT\Partial[\left(\ln{f_{i}}\right)]{P}{T,n} = \overline{v_{i}}, &          \\
                         &                                                                       \\
                       RT\Partial[\left(\ln{\overline{f}_{i}}\right)]{P}{T,n} = \overline{V_{i}}, & 
                    \end{cases}
                 \end{displaymath}}
                 \visible<4->{ 
                 \begin{displaymath}
                    \text{where}\begin{cases}
                         f_{i}: & \text{fugacity of pure component } i; \\
                         \overline{f}_{i}: & \text{fugacity of component } i \text{ in the mixture};  \\
                         \overline{v}_{i}: & \text{molar volume of of pure component } i;  \\
                         \overline{V}_{i}: & \text{molar volume of component } i \text{ in the mixture};  \\
                    \end{cases}
                 \end{displaymath}}
               
        \end{enumerate}
\end{frame}
\normalsize

%%%
%%% Slide
%%%
%\scriptsize
\begin{frame}
  \frametitle{Fugacity: Lewis-Randall Rule for Mixtures}
        \begin{enumerate}\setcounter{enumi}{2}  
           \item<1-> Subtracting the second equation from the first and integrating from state $P'$ to $P$, 
              \visible<2->{\begin{displaymath}
                  RT \left\{\left.\ln{\left(\frc{\overline{f}_{i}}{f_{i}}\right)}\right|_{P'}^{P}\right\} = \int\limits_{P'}^{P}\left(\overline{V}_{i} - \overline{v}_{i}\right)dP, \hspace{0.5cm} \text{at constant }T
               \end{displaymath}}

           \item<3-> In the limiting case, in which $P'\rightarrow 0$,
              \visible<3->{\begin{displaymath}
                  RT \left\{ \ln{ \left(\frc{\overline{f}_{i}}{f_{i}}\right) } - \lim\limits_{P'\rightarrow 0} \ln{ \left(\frc{\overline{f}_{i}}{f_{i}}\right) } \right\} = \int\limits_{P'}^{P}\left(\overline{V}_{i} - \overline{v}_{i}\right)dP,
               \end{displaymath}}

           \item<4-> For the second term in the brackets at the left-hand side,
                 \begin{displaymath}
                    \visible<4->{\begin{cases}
                       f_{i} \rightarrow P', & \\
                                     & \\
                       \overline{f}_{i} \rightarrow y_{i}P', &
                    \end{cases}}
                    \text{ as } P'\rightarrow 0  \hspace{0.1cm}\left(\text{where } y_{i}P' \text{ is obtained from Dalton's law}\right),
                 \end{displaymath}
               
        \end{enumerate}
\end{frame}
\normalsize 

%%%
%%% Slide
%%%
%\scriptsize
\begin{frame}
  \frametitle{Fugacity: Lewis-Randall Rule for Mixtures}
        \begin{enumerate}\setcounter{enumi}{4}  
           
           \item<1-> (Cont.) resulting in 
              \begin{displaymath}
                 \lim\limits_{P'\rightarrow 0} \ln{ \left(\frc{\overline{f}_{i}}{f_{i}}\right) } \rightarrow \ln{\left(\frc{y_{i}P'}{P'}\right)} = \ln{y_{i}},
              \end{displaymath}
              leading to,
                 \visible<2->{\begin{block}{\begin{center}\normalsize{ Relationship between Fugacities of a Pure Component and a Component in the Mixture}\end{center}}
                       \begin{displaymath}
                           \blue{RT\ln{\left(\frc{\overline{f}_{i}}{y_{i}f_{i}}\right)} = \int\limits_{0}^{P}\left(\overline{V}_{i} - \overline{v}_{i}\right)dP, \hspace{0.5cm} i=1, 2, \cdots, \mathcal{C}.}
                       \end{displaymath}
                       The term in brackets (left-hand side) represents the ratio between partial pressures between real and ideal fluids in a mixture.
                 \end{block}}
               
        \end{enumerate}
\end{frame}
\normalsize

%%%
%%% Slide
%%%
%\scriptsize
\begin{frame}
  \frametitle{Fugacity: Lewis-Randall Rule for Mixtures}

  \begin{block}{\begin{center}\normalsize{ Lewis-Randall Rule for Mixtures }\end{center}}
        Assuming \blue{an ideal mixture},
        \begin{enumerate}\setcounter{enumi}{5}  
           \item<1-> The \blue{the partial pressure of species $i$ is the vapour phase fugacity of component $i$ in the mixture}, \ie
                 \begin{displaymath}
                    \blue{\mfr[\overline{f}]{i}{V} = y_{i}P = P_{i}}
                 \end{displaymath}
                 
           \item<2-> And the liquid phase fugacity of component $i$ in the mixture is,
                 \begin{displaymath}
                    \red{\mfr[\overline{f}]{i}{L} = x_{i}\mfr[f]{i}{L}}
                 \end{displaymath}
        \end{enumerate}
        \visible<3->{The later expression is known as \red{Lewis-Randall rule} \blue{(or Raoult's law for fugacity)}. It shows that the fugacity of each species in an \blue{ideal solution} is proportional to its mole fraction; the 'proportionality term' is the fugacity of pure species i in the same physical state as the solution and at the same T and P.}
  \end{block}
  
\end{frame}
\normalsize

%%%
%%% SECTION
%%%
\section{Activity Coefficient}

%%%
%%% SUBSECTION
%%%
\subsection{Ideal Solutions}

%%%
%%% Slide
%%%
%\scriptsize
\begin{frame}
  \frametitle{Ideal Solutions}
        \begin{enumerate}%\setcounter{enumi}{3}  
           \item<1-> In Module 1, the concept of \blue{ideal gas} was introduced as {\it gaseous solution with no intra-molecular interactions};
           \item<2-> We can extend this idea to \blue{liquid solutions}. For a \red{solution} \blue{(or mixture)} to be considered \red{ideal}: 
             \begin{enumerate}[a)]
                \item<3-> Chemical species have similar molecular size and structure;
                \item<4-> Raoult's law (and the Raoult's law for fugacity) must be obeyed over the entire range of {\bf composition};
                \item<5-> $\Delta H_{\text{mix}}=0$, \ie no release or absorption of heat during dissolution (in other words, neither exothermic nor endothermic dissolution);
                \item<6-> $\Delta V_{\text{mix}}=0$, \ie total volume of the solution is equal to the sum of the molar volumes of all components;
                \item<7-> Total pressure is the sum of all partial pressures (\ie Dalton's law is obeyed), \ie
                  \begin{displaymath}
                    P = \summation[P_{i}]{i=1}{\mathcal{C}} = \summation[P_{i}^{\text{sat}}x_{i}]{i=1}{\mathcal{C}}.
                  \end{displaymath}
             \end{enumerate}  
        \end{enumerate}
\end{frame}
\normalsize

%%%
%%% Slide
%%%
%\scriptsize
\begin{frame}
  \frametitle{Ideal Solutions}
        \begin{enumerate}\setcounter{enumi}{2}  
           \item<1-> Example: Liquid solutions of benzene and toluene obey Raoult's law very closely, whereas water and methanol (although soluble in all proportions) form liquid solutions that deviate considerably from Raoult's law;
           \item<2-> The most commonly encountered situation for solutions of organic liquids is that each component deviates from Raoult's law behavior by having a higher fugacity than predicted by the Lewis-Randall rule (\ie positive deviation from Raoult's law).    
        \end{enumerate}
\end{frame}
\normalsize

%%%
%%% Slide
%%%
%\scriptsize
\begin{frame}
  \frametitle{Ideal and Non-Ideal Solutions: Deviations from the Raoult's Law ($xy$ phase diagrams)}
  \vbox{
     \hbox{\visible<1->{\includegraphics[width=5.5cm,height=4.cm,clip]{./../Pics/VLE_xy_DiagramIdeal}} \hspace{1cm}
           \visible<2->{\includegraphics[width=5.5cm,height=4.cm,clip]{./../Pics/VLE_xy_DiagramNonIdeal1}}}
  \vspace{-0.2cm}
  \hbox{\hspace{4cm}
        \visible<3->{\includegraphics[width=5.5cm,height=4.cm,clip]{./../Pics/VLE_xy_DiagramNonIdeal2}}}
  }
\end{frame}
\normalsize


%%%
%%% SUBSECTION
%%%
\subsection{Activity}
%%%
%%% Slide
%%%
%\scriptsize
\begin{frame}
  \frametitle{Activity: An Assessment Tool of Ideality of Mixtures}
        \begin{enumerate}%\setcounter{enumi}{3}  
           \item<1-> Let's consider a liquid mixture with
               \begin{displaymath}
                  \mu_{i} = RT\ln{\overline{f}_{i}} + C_{i}(T), \text{ with } i=1, 2, \cdots, \mathcal{C}.
               \end{displaymath}
           \item<2-> Also, let's consider the same fluid $i$ pure at a \blue{reference-state}, \ie at $T$ and reference-state pressure, $P_{\text{ref}}$. If we subtract the chemical potential of $i$ at the reference-state $\left(\mu_{i}^{\circ}\right)$ from the mixture,
               \begin{displaymath}
                  \mu_{i} - \mu_{i}^{\circ} = RT\ln{\left(\frc{\overline{f}_{i}}{f_{i}^{\circ}}\right)} , \text{ with } i=1, 2, \cdots, \mathcal{C},
               \end{displaymath}
               where $f_{i}^{\circ}$ is the fugacity of component $i$ pure at \blue{reference-state} conditions.
           
           \visible<3->{\begin{block}{\begin{center}\normalsize{Activity }\end{center}}
               The term in brackets is defined as the \red{activity $\left(a_{i}\right)$} of component $i$,
                    \begin{displaymath} 
                          a_{i} = \frc{\overline{f}_{i}}{f_{i}^{\circ}},
                    \end{displaymath}
                    which measures {\it deviation} from ideal behaviour of liquid solutions. 
               \end{block}}
               
        \end{enumerate}
\end{frame}
\normalsize


%%%
%%% SUBSECTION
%%%
\subsection{Activity Coefficient}

%%%
%%% Slide
%%%
%\scriptsize
\begin{frame}
  \frametitle{Activity Coefficient}
        \begin{enumerate}%\setcounter{enumi}{2}  
           \item<1-> Since the chemical potential was defined as the molar Gibbs free energy, at the \blue{reference-state} $\mu_{i}^{\circ} = \overline{g}_{i}^{\circ}$, therefore
              \begin{displaymath}
                 \mu_{i} = \overline{g}_{i}^{\circ} + RT\ln{a_{i}},
              \end{displaymath}
              where $\overline{g}_{i}^{\circ}$ is the molat Gibbs free energy of pure component $i$ at \blue{reference-state}. This is often tabulated for a number of chemical species and can be found in any ChemEng handbooks;
           
           \visible<2->{\begin{block}{\begin{center}\normalsize{ Activity Coefficient}\end{center}}
               For an \blue{ideal solution}, we can apply the \blue{Lewis-Randall rule}, 
                  \begin{displaymath}
                     a_{i} = \frc{\overline{f}_{i}}{f_{i}^{\circ}} = \frc{y_{i}f_{i}}{f_{i}^{\circ}} \;\; \Longrightarrow \;\; \frc{f_{i}}{f_{i}^{\circ}} = \frc{a_{i}}{y_{i}} = \gamma_{i}
                  \end{displaymath}
                  and define the \red{activity coefficient $\left(\gamma_{i}\right)$}. The \blue{activity coefficient} is an {\it adjustment} parameter that \blue{relates the actual behaviour to the ideal behaviour of a solution at the same $T$ and $P$}. In a broad way, $\gamma$ operarates in a similar way as the compressibility factor ($Z$) for gases (Module 1).
                    
               \end{block}}
               
        \end{enumerate}
\end{frame}
\normalsize

%%%
%%% Slide
%%% 
%\scriptsize
\begin{frame}
  \frametitle{Activity Coefficient}
       \begin{block}{\begin{center}\normalsize{ Activity Coefficient}\end{center}}
            \begin{enumerate}\setcounter{enumi}{1}  
               \item<1-> For solutions
                  \begin{displaymath}
                     \gamma_{i} = \frc{\overline{f}_{i}}{x_{i}f_{i}}
                  \end{displaymath}
               \item<2-> For VLE at low pressures:
                  \begin{eqnarray}
                      \red{\gamma_{i}} &=& \frc{\overline{f}_{i}}{x_{i}f_{i}} = \frc{y_{i}P}{x_{i}f_{i}} \nonumber \\
                                        &=&  \red{\frc{y_{i}P}{x_{i}P_{i}^{\text{sat}}}}, \nonumber
                  \end{eqnarray}
                  this expression is the \red{modified Raoult's law}, defined in Module 3.
             \end{enumerate}
        \end{block}
         
\end{frame}
\normalsize


%%%
%%% Slide
%%%
%\scriptsize 
\begin{frame}
 \frametitle{Example 8.1 (pg 138)}\label{Ex8_1}
    \blue{A process stream contains light species 1 and heavy species 2. A relatively pure liquid stream containing mostly 2 is obtained through a single-stage liquid/vapour separator. Specifications on the equilibrium composition are: $x_{1}$ = 0.002 and $y_{1}$ = 0.950. Assuming that the modified Raoult's law applies, 
\begin{displaymath}
  y_{i} P = x_{i}\gamma_{i}P_{i}^{\text{sat}}
\end{displaymath} 
Determine $T$ and $P$ for the separator. Given the activity coefficients for the liquid phase,
\begin{displaymath}
\ln\gamma_{1} = 0.93x_{2}^{2} \;\;\;\;\;\text{ and }\;\;\;\;\;\ln\gamma_{2}=0.93x_{1}^{2}
\end{displaymath}
\begin{displaymath}
\ln P_{i}^{\text{sat}} = A_{i} - \frc{B_{i}}{T}\;\;\;\text{with [P] = bar and [T] = K}
\end{displaymath} 
$A_{1}$ =10.08, $B_{1}$ = 2572.0 K, $A_{2}$ = 11.63 and $B_{2}$ = 6254.0 K.}

\end{frame}



%%%
%%% Slide
%%%
%\scriptsize 
\begin{frame}
 \frametitle{Example 8.2 (pg 139)}\label{Ex8_1}
    \blue{For the acetone (Ket) / methanol (MetOH) system, a vapour mixture of $z_{\text{Ket}}$ = 0.25 and $z_{\text{MetOH}}$ = 0.75 is cooled to temperature $T$ in the two-phase region and flows into a separation chamber at a pressure of 1 bar. If the composition of the liquid product is $x_{\text{Ket}}$ = 0.175, calculate $T$  and $y_{\text{Ket}}$. For liquid mixture, assume that
\begin{displaymath}
\ln\gamma_{1} = 0.64x_{2}^{2} \;\;\;\;\;\text{ and }\;\;\;\;\;\ln\gamma_{2}=0.64x_{1}^{2}
\end{displaymath}
For the Antoine equation, 
\begin{displaymath}
\ln P_{i}^{\text{sat}} = A_{i} - \frc{B_{i}}{T + C} \;\;\left(\text{ [P] = kPa and [T] = }^{\circ}\text{C}\right)
\end{displaymath}
$A_{\text{Ket}}$ = 14.3145, $B_{\text{Ket}}$ = 2756.22$^{\circ}$C, $C_{\text{Ket}}$ = 228.060$^{\circ}$C, $A_{\text{MetOH}}$ = 16.5785, $B_{\text{MetOH}}$ = 3638.27$^{\circ}$C, $C_{\text{MetOH}}$ = 239.50$^{\circ}$C.}

\end{frame}



%%%
%%% SUBSECTION
%%5
\section{Ideal Gas Mixture Model}

%%%
%%% Slide
%%%
%\scriptsize
\begin{frame}
  \frametitle{Ideal Gas Mixture Model}
  \begin{enumerate}
    \item<1->Useful model:
        \begin{enumerate}
          \item<1->Has a molecular basis;
          \item<1-> Approximates reality in well-defined limit of zero pressure;
          \item<1-> is analytically simple.
        \end{enumerate}
    \item<2-> Partial pressure:
        \visible<2->{\begin{displaymath}
            P_{i} = \frc{y_{i}R T}{V^{\text{ig}}} = y_{i}P\;\;\;\;\;\left(i=1,2,\cdots,\mathcal{C}\right)
        \end{displaymath} }
  \end{enumerate}
  \visible<3->{\begin{block}{\begin{center}\normalsize{ Gibbs Theorem}\end{center} }
                  \textcolor{blue}{$\lq$A partial molar property of a constituent species in an ideal-gas mixture is equal to the corresponding molar property of the species as a pure ideal gas at the mixture $T$ but at a $P$ equal to its partial pressure in the mixture.'\begin{displaymath}
            \overline{M}^{\text{igm}}_{i}\left(T,P\right) = M^{\text{ig}}_{i}\left(T,P_{i}\right)
          \end{displaymath}}
               \end{block}}
\end{frame}
\normalsize


%%%
%%% Slide
%%%
%\scriptsize
\begin{frame}
  \frametitle{Ideal Gas Mixture Model}
  \begin{enumerate}\setcounter{enumi}{2}
      \item<1->Properties of ideal-gas mixtures:
          \visible<1->{\begin{eqnarray}
             \blue{\overline{H}^{\text{igm}} = \sum\limits_{i} y_{i}\overline{h}_{i}^{\text{ig}}}, &&\blue{\overline{S}^{\text{igm}} = \sum\limits_{i} y_{i}\overline{s}_{i}^{\text{ig}}-R\sum\limits_{i}y_{i}\ln y_{i}}\;\;\; \nonumber \\
             && \blue{\overline{G}^{\text{igm}}=\sum\limits_{i}y_{i}\overline{g}_{i}^{\text{ig}} + RT\sum\limits_{i}y_{i}\ln y_{i}}\nonumber
          \end{eqnarray}}
      \item<2-> Defining the \red{isothermal molar property change of mixing, $\Delta M_{\text{mix}}$} \blue{$\left(\text{or }\Delta_{\text{mix}}M\right)$} as
           \begin{displaymath}
              \Delta M_{\text{mix}}(T,P) =  M(T,P) - \summation[y_{i}M_{i}(T,P)]{i}{} = \summation[y_{i}\left(\overline{M}_{i}-M_{i}\right)]{i}{},
           \end{displaymath}
      \item<3-> Thus,
           \begin{displaymath}
              \red{\Delta_{\text{mix}} \overline{s}^{\text{igm}} = -R\summation[y_{i}\ln{y_{i}}]{i}{}, \hspace{1cm} \Delta_{\text{mix}} \overline{g}^{\text{igm}} = RT\summation[y_{i}\ln{y_{i}}]{i}{}}
           \end{displaymath}


  \end{enumerate}
  \begin{center}
      \visible<3->{\blue{See Table 8.1 from Lecture Notes for a full compiled list of ideal gas mixture properties.}}
  \end{center}
\end{frame}
\normalsize


%%%
%%% SECTION
%%% 
\section{Solutions Theory}

%%%
%%% SUBSECTION
%%%
\subsection{Gibbs-Duhem Relations}

%\scriptsize
\begin{frame}
  \frametitle{Gibbs-Duhem Relations}
  \begin{enumerate}%\setcounter{enumi}{8}
    \item<1->For the total property $M\;(\equiv V,U,H,S,A,G)$:
      \visible<1->{\begin{displaymath}
          n M = \sum\limits_{i} n_{i}\overline{M}_{i} \Longrightarrow M = \sum\limits_{i}x_{i}\overline{M}_{i}, \text{ where }\;\overline{M}_{i}=\Partial[\left(nM\right)]{n_{i}}{P,T,n_{j} \left(n_{i} \ne n_{j}\right)  }
      \end{displaymath}
      is the partial molar property of species $i$ in solution.}
    \item<2->General expression for $dM$:
      \visible<2->{\begin{displaymath}
          dM = \sum\limits_{i}x_{i}d\overline{M}_{i} + \sum\limits_{i}\overline{M}_{i}dx_{i}
      \end{displaymath}}
  \end{enumerate}
  \visible<3->{\begin{block}{\begin{center}\normalsize{Gibbs-Duhem Equation }\end{center}}
               \begin{displaymath}
                   \red{\left(\frc{\partial M}{\partial P}\right)_{T,x} dP + \left(\frc{\partial M}{\partial T}\right)_{P,x}dT - \sum\limits_{i}x_{i}d\overline{M}_{i} = 0}
                \end{displaymath}
               and at $T$ and $P$ constant:
               \begin{displaymath}
                  \red{\sum\limits_{i}x_{i}d\overline{M}_{i} = 0}
               \end{displaymath}
             \end{block}}
\end{frame}
\normalsize

%%%
%%% Slide
%%%
%\scriptsize
\begin{frame}
  \frametitle{Partial Properties in Binary Solutions}
    \begin{block}{\begin{center}\normalsize{Applications of the Gibbs-Duhem Equation }\end{center}}
      \begin{enumerate}%\setcounter{enumi}{8}
         \item<1->Partial properties are readily calculated directly from an expression of the solution property as a function of composition at constant $T$ and $P$:
            \visible<1->{\begin{displaymath}
                            \red{\overline{M}_{1} = M + x_{2}\frc{d M}{dx_{1}} \;\;\;\text{ and } \;\;\; \overline{M}_{2}=M-x_{1}\frc{d M}{dx_{1}}}
                         \end{displaymath}}
         \item<2->Or in derivative format:
            \visible<2->{\begin{displaymath}
                            \blue{x_{1}\frc{d\overline{M}_{1}}{dx_{1}} + x_{2}\frc{d\overline{M}_{2}}{dx_{1}} = 0\;\;\;\text{ and }\;\;\; \frc{d\overline{M}_{1}}{dx_{1}} = -\frc{x_{2}}{x_{1}}\frc{d\overline{M}_{2}}{dx_{1}}}
                         \end{displaymath}}
      \end{enumerate}
   \end{block}
\end{frame}
\normalsize




%%%
%%% FUGACITY
%%%
\subsection{Fugacity Coefficient}

%%%
%%% Slide
%%%
%\scriptsize
\begin{frame}
  \frametitle{Fugacity Coefficient for Pure Species}
  \begin{enumerate}%\setcounter{enumi}{2}
      \item<1-> The \red{fugacity coefficient $\left(\phi\right)$} is defined as \blue{$\phi\equiv\frc{f}{P}$}, and for ideal gases, \red{$\phi = 1$},
          \visible<2->{\begin{displaymath}
              \begin{cases}
                 \text{ideal gas:} & G^{\text{ig}} = RT\ln{P} + C(T) \\
                 \text{real gas:} & G = RT\ln{f} + C(T) \\
              \end{cases}
          \end{displaymath}}
  
      \item<3-> Now, we can re-define the \blue{residual Gibbs free energy} $\left(\text{from Module 2: } M^{R} = M - M^{\text{ig}}\right)$,
          \visible<3->{\begin{displaymath}
              \blue{ G^{\text{R}} = G-G^{\text{ig}} = RT\ln\frc{f}{P} = RT\ln\phi}
          \end{displaymath}}
  \end{enumerate}
\end{frame}
\normalsize

%%%
%%% Slide
%%%
%\scriptsize
\begin{frame}
  \frametitle{Fugacity Coefficient for Pure Species}
  
      \visible<1->{\begin{block}{\begin{center}\normalsize{Fugacity and Fugacity Coefficient during VLE of Pure Species}\end{center}}
               During \blue{phase transition} from {\it saturated liquid} to {\it saturated vapour}, $dG=0$ (Module 2), \ie
               \begin{displaymath}
                   \mfr[G]{}{L} - \mfr[G]{}{V} = 0 = RT\ln{\frc{\mfr[f]{}{V}}{\mfr[f]{}{L}}} \;\Longrightarrow\; \blue{\mfr[f]{}{L}=\mfr[f]{}{V}=\mfr[f]{}{sat}}\;\Longrightarrow\; \red{\mfr[\phi]{}{L}=\mfr[\phi]{}{V}=\mfr[\phi]{}{sat}}
               \end{displaymath}
               \blue{The vapour and liquid phases of pure chemical species are in equilibrium when both phases have the same temperature, pressure and} \red{fugacity}.
             \end{block}} 
\end{frame}
\normalsize


%%%
%%% Slide
%%%
%\scriptsize
\begin{frame}
  \frametitle{Fugacity and Fugacity Coefficient of Species in Solution}
  \begin{enumerate}%\setcounter{enumi}{2}
      \item<1-> For species $i$ in a mixture of \blue{real gases} or \blue{liquids (solution)}:
          \visible<1->{\begin{displaymath}
             \mu_{i} = RT\ln\overline{f}_{i} + C_{i}\left(T\right)
          \end{displaymath}
          where $\overline{f}_{i}$ represents the fugacity of component $i$ in the mixture.} 
      \item<2-> Therefore, for an arbitrary number of phases $\left(\mathcal{P}=\alpha,\beta,\cdots,\pi\right)$ in equilibrium:
          \visible<2->{\begin{displaymath}
             \overline{f}_{i}^{\alpha} = \overline{f}_{i}^{\beta} = \cdots = \overline{f}_{i}^{\pi} \;\;\;\text{ with } i = 1,2,3,\cdots,\mathcal{C} 
          \end{displaymath}}
          \visible<3->{\blue{Multiple phases at the same $T$ and $P$ are in equilibrium when the fugacity of each component (i.e., chemical species) is the same in {\bf all phases}. }}
      \item<4-> \blue{Fugacity coefficient} for species in solution: \blue{$\overline{\phi}_{i}=\frc{\overline{f}_{i}}{y_{i}P}$}.
  \end{enumerate}
\end{frame}
\normalsize

%%%
%%% SUBSECTION
%%%
\subsection{Ideal Solution Model}

%%%
%%% Slide
%%%
%\scriptsize
\begin{frame}
  \frametitle{Ideal Solution Model}
  \begin{enumerate}%\setcounter{enumi}{2}
      \item<1->\blue{A solution is ideal (id) when:} 
          \visible<1->{\begin{displaymath}
             \blue{\mu_{i}^{\text{id}} = \overline{G}_{i}^{\text{id}} = G_{i}\left(T,P\right) + RT\ln x_{i}}
          \end{displaymath}}
      \item<2-> Total values of properties:
          \visible<2->{\begin{eqnarray}
             G^{\text{id}} = \sum\limits_{i}x_{i}G_{i} + RT\sum\limits_{i}x_{i}\ln x_{i}, && S^{\text{id}} = \sum\limits_{i}x_{i}S_{i}-R\sum\limits_{i}x_{i}\ln x_{i} \nonumber \\
              V^{\text{id}} = \sum\limits_{i}x_{i}V_{i}   && H^{\text{id}} = \sum\limits_{i}x_{i}H_{i} \nonumber
          \end{eqnarray}}
  \end{enumerate}
\end{frame}
\normalsize


%%%
%%% SUBSECTION
%%%
\subsection{Excess Properties}

%%%
%%% Slide
%%%
%\scriptsize
\begin{frame}
  \frametitle{Excess Properties}
  \begin{enumerate}%\setcounter{enumi}{2}
      \item<1-> In Module 3, \red{excess properties} were defined for any extensive thermodynamic property $M$ (= $V$, $U$, $H$, $G$,$S$ etc) as
          \visible<1->{\begin{displaymath}
             M^{\text{E}} = M - M^{\text{id}}
          \end{displaymath}
          where $M$ and $M^{\text{id}}$ are the actual and ideal solution properties.}
      \item<2-> Excess properties are often complex non-linear functions of the composition, $T$ and $P$, and are usually obtained from experiments.
      \item<3-> Excess properties have \blue{no meaning} for pure species, whereas {\it residual properties} exist for both \red{pure species} and \red{mixtures}.
  \end{enumerate}
\end{frame}
\normalsize


%%%
%%% Slide
%%%
%\scriptsize
\begin{frame}[label={ExcessProperty}]
  \frametitle{Excess Properties}
  \begin{enumerate}\setcounter{enumi}{4}
      \item<1->Fundamental excess property relation
          \visible<1->{\begin{displaymath}
             d\left(\frc{\red{n} G^{\text{E}}}{RT}\right) = \frc{nV^{\text{E}}}{RT}dP - \frc{nH^{\text{E}}}{RT^{2}}dT + \sum\limits_{i}\frc{\overline{G}_{i}^{\text{E}}}{RT}dn_{i}
          \end{displaymath}}
      \item<2->\label{activitycoefficient} Relation with \blue{activity coefficient, $\gamma_{i}$}:
           \visible<2->{\begin{displaymath}
              \gamma_{i} = \frc{\overline{f}_{i}}{x_{i}f_{i}} \hspace{1cm}\red{\Longrightarrow}\hspace{1cm} \overline{G}_{i}^{\text{E}}=RT\ln\gamma_{i}
           \end{displaymath}}
      \item<3-> Effects of $P$ and $T$:
           \visible<3->{\begin{displaymath}
              \left(\frc{\partial\ln\gamma_{i}}{\partial P}\right)_{T,x}=\frc{\overline{V}^{\text{E}}_{i}}{RT}, \hspace{1cm} \left(\frc{\partial\ln\gamma_{i}}{\partial T}\right)_{P,x}=-\frc{\overline{H}^{\text{E}}_{i}}{RT^{2}}
           \end{displaymath}}
      \item<4-> Gibbs-Duhem equations
           \visible<4->{\begin{displaymath}
              \frc{G^{\text{E}}}{RT} = \sum\limits_{i}x_{i}\ln\gamma_{i}, \hspace{1cm} \left[\sum\limits_{i}x_{i}d\left(\ln\gamma_{i}\right)\right]_{T,P}=0
           \end{displaymath}}
  \end{enumerate}
\end{frame}
\normalsize

%%%
%%% Slide
%%%
%\scriptsize
\begin{frame}
  \frametitle{Excess Properties}
  \begin{enumerate}\setcounter{enumi}{8}
      \item<1-> Excess properties can be determined via
         \begin{enumerate}
            \item<1-> $G^{\text{E}}$ from VLE data;
            \item<1-> $H^{\text{E}}$ from mixing experiments;
            \item<1-> $S^{\text{E}}$ from $S^{\text{E}}=\frc{H^{\text{E}}-G^{\text{E}}}{T}$
         \end{enumerate}
      \item<2-> All excess properties become zero as either species approaches purity $\left(\text{i.e.,} x_{i}\rightarrow 1\right)$.
      \item<3-> Excess properties approach zero for ideal solutions, but thermodynamic properties might still change upon mixing
         \visible<3->{\begin{eqnarray}
             \Delta G^{\text{id}} = RT \sum\limits_{i} x_{i}\ln x_{i} && \Delta S^{\text{id}} = -R \sum\limits_{i} x_{i}\ln x_{i} \nonumber \\
             \Delta V^{\text{id}} = 0 && \Delta H^{\text{id}} = 0 \nonumber
         \end{eqnarray}}
  \end{enumerate}
\end{frame}
\normalsize

%%%
%%% Slide 
%%%
\scriptsize
\begin{frame} 
  \frametitle{Excess Properties}
     \begin{center}
       \begin{figure}
         \includegraphics[width=9.cm, height=6.5cm,clip]{../Pics/ExcessProperties_Plot}
          \caption{\scriptsize Excess properties at 50$^{\circ}$C for the following binary liquid systems: (a) chloroform / n-heptane,(b) acetone / methanol, (c) acetone / chloroform, (d) ethanol / n-heptane, (e) ethanol / chloroform and (f) ethanol / water (Extracted from {\it Smith, Van Ness and Abott}). }
       \end{figure}
     \end{center}
\end{frame}
\normalsize


%%%
%%% SUBSECTION
%%%
\subsection{Activity Coefficient Models}

%%%
%%% Slide
%%%
%\scriptsize
\begin{frame}
  \frametitle{Activity Coefficient Models}
  \begin{enumerate}%\setcounter{enumi}{8}
      \item<1-> In several industrial and environmental applications EOS can not accurately predict the thermodynamic behaviour of solutions. In these cases, we can estimate the excess Gibbs energy \blue{$G^{\text{E}}$} (see Slide~\ref{ExcessProperty}, item~\ref{activitycoefficient}) by first calculating the activity coefficient, $\gamma_{i}$;
      \item<2-> Activity coefficient models are often more accurate (than traditional EOS) when strong intermolecular interactions are present;
      \item<3-> Models commonly used coefficient activity models in \blue{fluid simulators}:
          \begin{enumerate}
             \item<3-> Margules;
             \item<3-> Van Laar;
             \item<3-> Wilson;
             \item<3-> Non-Random-Two-Liquids (NRTL);
             \item<3-> UNIversal QUAsi Chemical (UNIQUAC).     
          \end{enumerate}
      \item<4-> Basic Equations:
          \visible<4->{\begin{displaymath}
            \gamma_{i} = \frc{\overline{f}_{i}}{x_{i}f_{i}}=\frc{\overline{f}_{i}}{\overline{f}_{i}^{\text{id}}},\hspace{1cm} \frc{G^{\text{E}}}{RT}=\xi\left(\gamma_{i}\right)
          \end{displaymath}
           where $\xi\left(\gamma_{i}\right)$ is a function of $\gamma_{i}$. Bear in mind that $\gamma_{i}$ is a function of composition, $x_{i}$. }
  \end{enumerate}
\end{frame}
\normalsize


%%%
%%% Slide
%%%
%\scriptsize
\begin{frame}
  \frametitle{Activity Coefficient Models}
  \begin{enumerate}\setcounter{enumi}{4}
      \item<1-> 2-parameter models for binary systems:
        \begin{enumerate}
          \item<1-> Mergules equations;
             \visible<1->{\begin{displaymath}
                \ln\gamma_{1}=x_{2}^{2}\left[A_{12}+2\left(A_{21}-A_{12}\right)x_{1}\right] \hspace{1cm} \ln\gamma_{2}=x_{1}^{2}\left[A_{21}+2\left(A_{12}-A_{21}\right)x_{2}\right]
             \end{displaymath}}
          \item<2-> Van Laar equations:
             \visible<2->{\begin{displaymath}
                \ln\gamma_{1}= B_{12}\left(1+\frc{B_{12}x_{1}}{A_{21}x_{2}}\right)^{-2}  \hspace{1cm} \ln\gamma_{2}= B_{21}\left(1+\frc{B_{21}x_{1}}{A_{12}x_{2}}\right)^{-2}
             \end{displaymath}}
          \item<3-> Wilson equations:
             \visible<3->{\begin{eqnarray}
                && \frc{G^{\text{E}}}{RT} = x_{1}\ln\left(x_{1}+x_{2}C_{12}\right) - x_{2}\ln\left(x_{2}+x_{1}C_{21}\right) \nonumber \\
                && \ln\gamma_{1}= -\ln\left(x_{1}+x_{2}C_{12}\right) + x_{2}\left(\frc{C_{12}}{x_{1}+x_{2}C_{12}}-\frc{C_{21}}{x_{2}+x_{1}C_{21}}\right)\nonumber \\
                && \ln\gamma_{2}= -\ln\left(x_{2}+x_{2}C_{21}\right) + x_{2}\left(\frc{C_{12}}{x_{1}+x_{2}C_{12}}-\frc{C_{21}}{x_{2}+x_{1}C_{21}}\right)\nonumber 
             \end{eqnarray}}
        \end{enumerate} 
  \end{enumerate}
\end{frame}
\normalsize


%%%
%%% Slide
%%%
%\scriptsize
\begin{frame}
  \frametitle{Activity Coefficient Models}
  \begin{enumerate}\setcounter{enumi}{5}
      \item<1-> NRTL and UNIQUAC are 3-parameter models that incorporate binary attraction/repulsion parameters of multiple chemical species;
       \item<2-> UNIFAC (UNIQUAC Functional-group Activity Coefficient) is a group contribution model that assign specific thermodynamic properties to functional group species, e.g., \red{$\cdot$CH$_{3}$}, \red{:CH$_{2}$}, \red{:COH}, \red{$\cdot$OH}, etc.
  \end{enumerate}
\end{frame}
\normalsize


%%%
%%% SUMMARY
%%%
\section{Summary}

%%%
%%% Slide
%%%
%\scriptsize
\begin{frame}
 \frametitle{Summary}
   \begin{enumerate}[(i)]
      \item Introduction to concept of partial molar properties;
      \item Chemical potential as a molar Gibbs free energy surrogate;
      \item Definition of fugacity, activity, activity coefficient, fugacity coefficient for pure species and mixtures;
      \item Requirements for Ideal solutions;
      \item Criteria for phase equilibrium;
      \item Brief description of activity coefficient models currently used in simulators.
   \end{enumerate}
\end{frame}


%%%
%%% EXAMPLES
%%%
%\begin{comment}

%%%
%%%  SECTION
%%%
\section{Examples}

%%%
%%% Slide
%%%
%\scriptsize 
\begin{frame}
 \frametitle{Example 1}\label{Ex1}
    \blue{At 25$^{\circ}$C and atmospheric pressure, the volume change of mixing of a binary liquid mixture of species 1 and 2 is given by,}
       \begin{displaymath}
          \blue{\Delta V = x_{1}x_{2}\left(45x_{1}+25x_{2}\right)\;\;\;\;\text{ with }\;\;\;\left[\Delta V\right] = \text{cm}^{3}.\text{mol}^{-1}}
       \end{displaymath}
       \blue{with molar volumes of $V_{1} = 110\;\text{cm}^{3}.\text{mol}^{-1}$ and $V_{2} = 90\;\text{cm}^{3}.\text{mol}^{-1}$. Determine the partial molar volumes, $\overline{V}_{1}$ and $\overline{V}_{2}$, in a mixture containing 40$\%$-mol of species 1.}

\end{frame}

%%%
%%% Slide
%%%
%\scriptsize 
\begin{frame}
 \frametitle{Example 2}\label{Ex2}
    \blue{A process stream contains light species 1 and heavy species 2. A relatively pure liquid stream containing mostly 2 is obtained through a single-stage liquid/vapour separator. Specifications on the equilibrium composition are: $x_{1}$ = 0.002 and $y_{1}$ = 0.950. Assuming that the modified Raoult's law applies, }
\begin{displaymath}
  \blue{y_{i} P = x_{i}\gamma_{i}P_{i}^{\text{sat}}}
\end{displaymath} 
\blue{Determine $T$ and $P$ for the separator. Given the activity coefficients for the liquid phase,}
\begin{displaymath}
   \blue{\ln\gamma_{1} = 0.93x_{2}^{2} \;\;\;\;\;\text{ and }\;\;\;\;\;\ln\gamma_{2}=0.93x_{1}^{2}}
\end{displaymath}
\begin{displaymath}
    \blue{\ln P_{i}^{\text{sat}} = A_{i} - \frc{B_{i}}{T}\;\;\;\text{with [P] = bar and [T] = K}}
\end{displaymath} 
\blue{$A_{1}$ =10.08, $B_{1}$ = 2572.0 K, $A_{2}$ = 11.63 and $B_{2}$ = 6254.0 K.}

\end{frame}

%%%
%%% Slide
%%%
%\scriptsize 
\begin{frame}
 \frametitle{Example 3}\label{Ex3}
    \blue{The experimental value of the partial molar volume $\left(\text{cm}^{3}\text{.mol}^{-1}\right)$ of a aqueous solution of K$_{2}$SO$_{4}$ is given by}
                \begin{displaymath}
                   \blue{\overline{V}_{A} = 32.280 + 18.216 m^{1/2},}
                \end{displaymath} 
 \blue{where $m$ is the molality (= number of moles per kg of water) of the K$_{2}$SO$_{4}$ . Use the Gibbs-Duhem equation to derive an equation for the partial molar volume of water in the solution. Plot $\overline{V}_{i}\;\;\times m$, with $0.0\leq m\leq 0.1$. The molar volume of pure water at 298.15 K is 18.079 cm$^{3}$.mol$^{-1}$ and the molar mass of pure water is 18 g.mol$^{-1}$.}

\end{frame}

%\end{comment}  
\end{document}
 

%% Aberdeen style guide should be followed when using this
% layout. Their template powerpoint slide is used to extract the
% Aberdeen color and logo but is otherwise ignored (it has little or
% no formatting in it anyway).
%
% http://www.abdn.ac.uk/documents/style-guide.pdf

%%%%%%%%%%%%%%%%%%%% Document Class Settings %%%%%%%%%%%%%%%%%%%%%%%%%
% Pick if you want slides, or draft slides (no animations)
%%%%%%%%%%%%%%%%%%%%%%%%%%%%%%%%%%%%%%%%%%%%%%%%%%%%%%%%%%%%%%%%%%%%%%
%Normal document mode%
\documentclass[10pt,compress]{beamer}
%Draft or handout mode
%\documentclass[10pt,compress,handout]{beamer}
%\documentclass[10pt,compress,handout,ignorenonframetext]{beamer}

%%%%%%%%%%%%%%%%%%%% General Document settings %%%%%%%%%%%%%%%%%%%%%%%
% These settings must be set for each presentation
%%%%%%%%%%%%%%%%%%%%%%%%%%%%%%%%%%%%%%%%%%%%%%%%%%%%%%%%%%%%%%%%%%%%%%
\newcommand{\shortname}{jefferson.gomes@abdn.ac.uk}
\newcommand{\fullname}{Dr Jeff Gomes}
\institute{School of Engineering}
\newcommand{\emailaddress}{}%jefferson.gomes@abdn.ac.uk}
\newcommand{\logoimage}{../../FigBanner/UoAHorizBanner}
\title{Chemical Thermodynamics (EG3029)}
\subtitle{Module 5: Vapour-Liquid Equilibrium of Mixtures}
\date[2014-15]{2014-15}

%%%%%%%%%%%%%%%%%%%% Template settings %%%%%%%%%%%%%%%%%%%%%%%%%%%%%%%
% You shouldn't have to change below this line, unless you want to.
%%%%%%%%%%%%%%%%%%%%%%%%%%%%%%%%%%%%%%%%%%%%%%%%%%%%%%%%%%%%%%%%%%%%%%
\usecolortheme{whale}
\useoutertheme{infolines}

% Use the fading effect for items that are covered on the current
% slide.
\beamertemplatetransparentcovered

% We abuse the author command to place all of the slide information on
% the title page.
\author[\shortname]{%
  \fullname\\\ttfamily{\emailaddress}
}


%At the start of every section, put a slide indicating the contents of the current section.
%\AtBeginSection[] {
%  \begin{frame}
%    \frametitle{Section Outline}
%    \tableofcontents[currentsection]
%  \end{frame}
%}

% Allow the inclusion of movies into the Presentation! At present,
% only the Okular program is capable of playing the movies *IN* the
% presentation.
\usepackage{multimedia}
\usepackage{animate}

%% Handsout -- comment out the lines below to create handstout with 4 slides in a page with space for comments
\usepackage{handoutWithNotes}

\mode<handout>
{
\usepackage{pgf,pgfpages}

\pgfpagesdeclarelayout{2 on 1 boxed with notes}
{
\edef\pgfpageoptionheight{\the\paperheight} 
\edef\pgfpageoptionwidth{\the\paperwidth}
\edef\pgfpageoptionborder{0pt}
}
{
\setkeys{pgfpagesuselayoutoption}{landscape}
\pgfpagesphysicalpageoptions
    {%
        logical pages=4,%
        physical height=\pgfpageoptionheight,%
        physical width=\pgfpageoptionwidth,%
        last logical shipout=2%
    } 
\pgfpageslogicalpageoptions{1}
    {%
    border code=\pgfsetlinewidth{1pt}\pgfstroke,%
    scale=1,
    center=\pgfpoint{.25\pgfphysicalwidth}{.75\pgfphysicalheight}%
    }%
\pgfpageslogicalpageoptions{2}
    {%
    border code=\pgfsetlinewidth{1pt}\pgfstroke,%
    scale=1,
    center=\pgfpoint{.25\pgfphysicalwidth}{.25\pgfphysicalheight}%
    }%
\pgfpageslogicalpageoptions{3}
    {%
    border shrink=\pgfpageoptionborder,%
    resized width=.7\pgfphysicalwidth,%
    resized height=.5\pgfphysicalheight,%
    center=\pgfpoint{.75\pgfphysicalwidth}{.29\pgfphysicalheight},%
    copy from=3
    }%
\pgfpageslogicalpageoptions{4}
    {%
    border shrink=\pgfpageoptionborder,%
    resized width=.7\pgfphysicalwidth,%
    resized height=.5\pgfphysicalheight,%
    center=\pgfpoint{.75\pgfphysicalwidth}{.79\pgfphysicalheight},%
    copy from=4
    }%

\AtBeginDocument
    {
    \newbox\notesbox
    \setbox\notesbox=\vbox
        {
            \hsize=\paperwidth
            \vskip-1in\hskip-1in\vbox
            {
                \vskip1cm
                Notes\vskip1cm
                        \hrule width\paperwidth\vskip1cm
                    \hrule width\paperwidth\vskip1cm
                        \hrule width\paperwidth\vskip1cm
                    \hrule width\paperwidth\vskip1cm
                        \hrule width\paperwidth\vskip1cm
                    \hrule width\paperwidth\vskip1cm
                    \hrule width\paperwidth\vskip1cm
                    \hrule width\paperwidth\vskip1cm
                        \hrule width\paperwidth
            }
        }
        \pgfpagesshipoutlogicalpage{3}\copy\notesbox
        \pgfpagesshipoutlogicalpage{4}\copy\notesbox
    }
}
}

%\pgfpagesuselayout{2 on 1 boxed with notes}[letterpaper,border shrink=5mm]

%%%%% Color settings
\usepackage{color}
%% The background color for code listings (i.e. example programs)
\definecolor{lbcolor}{rgb}{0.9,0.9,0.9}%
\definecolor{UoARed}{rgb}{0.64706, 0.0, 0.12941}
\definecolor{UoALight}{rgb}{0.85, 0.85, 0.85}
\definecolor{UoALighter}{rgb}{0.92, 0.92, 0.92}
\setbeamercolor{structure}{fg=UoARed} % General background and higlight color
\setbeamercolor{frametitle}{bg=black} % General color
\setbeamercolor{frametitle right}{bg=black} % General color
\setbeamercolor{block body}{bg=UoALighter} % For blocks
\setbeamercolor{structure}{bg=UoALight} % For blocks
% Rounded boxes for blocks
\setbeamertemplate{blocks}[rounded]

%%%%% Font settings
% Aberdeen requires the use of Arial in slides. We can use the
% Helvetica font as its widely available like so
% \usepackage{helvet}
% \renewcommand{\familydefault}{\sfdefault}
% But beamer already uses a sans font, so we will stick with that.

% The size of the font used for the code listings.
\newcommand{\goodsize}{\fontsize{6}{7}\selectfont}

% Extra math packages, symbols and colors. If you're using Latex you
% must be using it for formatting the math!
\usepackage{amscd,amssymb} \usepackage{amsfonts}
\usepackage[mathscr]{eucal} \usepackage{mathrsfs}
\usepackage{latexsym} \usepackage{amsmath} \usepackage{bm}
\usepackage{amsthm} \usepackage{textcomp} \usepackage{eurosym}
% This package provides \cancel{a} and \cancelto{a}{b} to "cancel"
% expressions in math.
\usepackage{cancel}

\usepackage{comment} 

% Get rid of font warnings as modern LaTaX installations have scalable
% fonts
\usepackage{type1cm} 

%\usepackage{enumitem} % continuous numbering throughout enumerate commands

% For exact placement of images/text on the cover page
\usepackage[absolute]{textpos}
\setlength{\TPHorizModule}{1mm}%sets the textpos unit
\setlength{\TPVertModule}{\TPHorizModule} 

% Source code formatting package
\usepackage{listings}%
\lstset{ backgroundcolor=\color{lbcolor}, tabsize=4,
  numberstyle=\tiny, rulecolor=, language=C++, basicstyle=\goodsize,
  upquote=true, aboveskip={1.5\baselineskip}, columns=fixed,
  showstringspaces=false, extendedchars=true, breaklines=false,
  prebreak = \raisebox{0ex}[0ex][0ex]{\ensuremath{\hookleftarrow}},
  frame=single, showtabs=false, showspaces=false,
  showstringspaces=false, identifierstyle=\ttfamily,
  keywordstyle=\color[rgb]{0,0,1},
  commentstyle=\color[rgb]{0.133,0.545,0.133},
  stringstyle=\color[rgb]{0.627,0.126,0.941}}

% Allows the inclusion of other PDF's into the final PDF. Great for
% attaching tutorial sheets etc.
\usepackage{pdfpages}
\setbeamercolor{background canvas}{bg=}  

% Remove foot note horizontal rules, they occupy too much space on the slide
\renewcommand{\footnoterule}{}

% Force the driver to fix the colors on PDF's which include mixed
% colorspaces and transparency.
\pdfpageattr {/Group << /S /Transparency /I true /CS /DeviceRGB>>}

% Include a graphics, reserve space for it but
% show it on the next frame.
% Parameters:
% #1 Which slide you want it on
% #2 Previous slides
% #3 Options to \includegraphics (optional)
% #4 Name of graphic
\newcommand{\reserveandshow}[4]{%
\phantom{\includegraphics<#2|handout:0>[#3]{#4}}%
\includegraphics<#1>[#3]{#4}%
}

\newcommand{\frc}{\displaystyle\frac}
\newcommand{\red}{\textcolor{red}}
\newcommand{\blue}{\textcolor{blue}}
\newcommand{\green}{\textcolor{green}}
\newcommand{\purple}{\textcolor{purple}}
 
\begin{document}

% Title page layout
\begin{frame}
  \titlepage
  \vfill%
  \begin{center}
    \includegraphics[clip,width=0.8\textwidth]{\logoimage}
  \end{center}
\end{frame}

% Table of contents
\frame{ \frametitle{Slides Outline}
  \tableofcontents
}


%%%%%%%%%%%%%%%%%%%% The Presentation Proper %%%%%%%%%%%%%%%%%%%%%%%%%
% Fill below this line with \begin{frame} commands! It's best to
% always add the fragile option incase you're going to use the
% verbatim environment.
%%%%%%%%%%%%%%%%%%%%%%%%%%%%%%%%%%%%%%%%%%%%%%%%%%%%%%%%%%%%%%%%%%%%%%

%%%
%%% SECTION
%%%
\section{General Remarks}

%%%
%%% Slides
%%%
\begin{frame}
 \frametitle{Aims and Objectives}
   \begin{enumerate}
     \item<1-> In Modules 3-4, we focused on 
       \begin{enumerate}
         \item<1-> Evaluating partial derivatives of thermodynamic variables and derive fundamental thermodynamic relations;
         \item<1-> Obtaining volumetric EOS parameters from critical properties;
         \item<1-> Studying thermodynamic behaviour of pure substances in arbitrary number of phases (multiphase).
       \end{enumerate} 
     \item<2-> However, in most industrial processes more than one chemical specie is present within multiple phases in thermodynamic equilibrium. 
     \item<3-> Our focus in this Module is to study equilibrium of arbitrary number of chemical species at vapour and liquid phases.
     \item<4-> This Module focuses on 
         \begin{enumerate}
           \item<4-> Partial molar properties; 
           \item<4-> Vapour-liquid equilibrium (VLE) relations and the main diagrams ($X-y$, $T-x-y$, $P-x-y$); 
           %\item<4-> Concept of activity, fugacity, chemical potential (and its relationship with the free Gibbs energy).
         \end{enumerate}
   \end{enumerate}
\end{frame}


%%%
%%% SECTION
%%%
\subsection{Bibliography}
\begin{frame}
 \frametitle{Suggested References}
  Literature relevant for this module:
  \begin{enumerate}[(i)]
   \item\label{SVN_Book} J.M. Smith, H.C. Van Ness, M.M. Abbott, $\lq$Introduction to Chemical Engineering Thermodynamics', 6$^{th}$ Edition: Chapter 10;
   \item Y.V.C. Rao, $\lq$Chemical Engineering Thermodynamics',4$^{th}$ Edition: Chapters 10 and 12.
   \item\label{Sandle_Book} S.I. Sandler, $\lq$Chemical, Biochemical and Engineering Thermodynamics', 4$^{th}$ Edition: Chapter 10.
  \end{enumerate}
\end{frame}


%%%
%%% SUBSECTION
%%%
\subsection{Introduction} 

%%%
%%% Slide
%%%
%\scriptsize
\begin{frame}
  \frametitle{Introduction}
  \begin{enumerate}
    \item<1-> In most technical processes there are no pure substances but mixture of various species in
        \begin{enumerate}
          \item<1-> industrial reactors and pressure vessels $\&$ tanks;
          \item<1-> separation process facilities (e.g., distillation, absorption, extraction etc);
          \item<1-> subsurface transport processes (e.g., groundwater pollutant contamination, hydrocarbon exploration etc).     
        \end{enumerate}
    \item<2-> It is necessary \textcolor{blue}{to identify how the concentration of species will change during the process};
    \item<3-> During \textcolor{blue}{phase equilibrium} we need to be able to effectively quantify variations in the thermodynamic properties \textcolor{blue}{in particular, concentration}. This quantitative analysis is crucial in the design of processes and equipment;
    \item<4-> \textcolor{blue}{Equilibria} of industrial interest:
        \begin{enumerate}
          \item<5-> \textcolor{blue}{Vapour-Liquid (VLE)}: distillation;
          \item<5-> Liquid-Liquid (LLE): extraction;
          \item<5-> Solid-vapour (SVE): particle generation;
          \item<5-> Liquid-solid (LSE): cristallisation
        \end{enumerate}
    \item<5-> {\it In this Module we will only consider non-reacting systems.}
  \end{enumerate}
\end{frame}
\normalsize


%%%
%%% Slide
%%%
%\scriptsize
\begin{frame}
  \frametitle{Representing Composition}
  \begin{enumerate}
    \item<1-> Mass $\left(w_{i}\right)$ and mole $\left(x_{i}\right)$ fraction:
         \visible<1->{\begin{displaymath}
            w_{i} = \frc{m_{i}}{m} = \frc{\dot{m}_{i}}{\dot{m}} \;\;\;\text{ and }\;\;\; x_{i} = \frc{n_{i}}{n} = \frc{\dot{n}_{i}}{\dot{n}}
         \end{displaymath}
         where $\left[\dot{\gamma}\right]$ represents flow rate quantities.}
    \item<2-> Molar concentration:
         \visible<2->{\begin{displaymath}
            C_{i} = \frc{n_{i}}{V} 
         \end{displaymath}
          also called \textcolor{blue}{molarity}.}
    \item<2-> Molar concentration in flow process:
         \visible<2->{\begin{displaymath}
            C_{i} = \frc{\dot{n}_{i}}{\dot{q}} 
         \end{displaymath}
         where $\dot{q}$ is the volumetric flow rate.}
    \item<3-> (Average) Molar mass of mixtures:
         \visible<3->{\begin{displaymath}
            \overline{MW} = \sum\limits_{i} \left(x_{i} \; MW_{i}\right) 
         \end{displaymath}
         where $MW_{i}$ is the molecular weight of species $i$.}
  \end{enumerate}
\end{frame}
\normalsize


\subsection{Excess Properties}
%%%
%%% Slide
%%%
%\scriptsize
\begin{frame}
  \frametitle{Excess Properties}
  \visible<1->{\begin{block}{Excess Mixing Properties (Sandler)}
$\lq$\textcolor{blue}{Excess mixing property} is the change in the extensive thermodynamic property $M$ that occurs on mixing at constant pressure and temperature in addition to that which would occur if an ideal mixture were formed.'
  \end{block}}
  \visible<1->{\begin{displaymath}
     M^{\text{E}} = M - M^{\text{id}}
  \end{displaymath}
  $M$ is the actual property of the mixture, and $M^{\text{id}}$ is the ideal solution property.}\\
  \visible<2->{This accounts for extensive properties of the individual components, i.e.,
     \begin{displaymath}
     M^{\text{id}} = \sum\limits_{i} x_{i} M_{i}
  \end{displaymath}}
  \medskip\noindent
  \visible<3->{As we saw in the Lab-work for \textcolor{blue}{excess volume} for two components --  water (1) and DMSO (2):
      \begin{displaymath}
         \textcolor{blue}{V^{\text{E}}} = V - V^{\text{id}} = V - \sum\limits_{i} x_{i} V_{i} = \textcolor{blue}{V - \left(x_{1}V_{1} + x_{2}V_{2}\right)}
      \end{displaymath}} 
\end{frame}
\normalsize

%%%
%%% SECTION
%%%
\section{Equilibrium Conditions}

\subsection{Stability Test}
%%%
%%% Slide
%%%
%\scriptsize
\begin{frame}
  \frametitle{Equilibrium Conditions and Stability Criteria}
     \begin{center}
         \begin{tabular}{l l l c}
         \hline\hline
            {\bf System}         &   {\bf Constraint}    &    {\bf Equilibrium}            & {\bf Stability } \\
                                 &                       &    {\bf Criterion}              & {\bf Criterion} \\ 
         \hline
            Isolated, adiabatic  &  $U$ and $V$          &      $S$ = maximum               &    d$^{2}$S$<$ 0    \\
            fixed-boundary system&   constant            &      $dS$ = 0                    &                     \\
         \hline
            Isothermal closed    &   $T$ and $V$         &      $A$ = minimum               &                   \\
            system with fixed    &   constant            &      $dA$ = 0                    &  d$^{2}$A $>$ 0    \\
            boundaries           &                       &                                  &                    \\
         \hline
            Isothermal, isobaric &  $T$ and $P$          &    $G$ = minimum                 &  d$^{2}$G $>$ 0    \\
            closed system        &   constant            &    $dG$ = 0                      &                    \\
         \hline
            Isothermal, isobaric &  $T$, $P$ and $N$    &     $G$ = minimum                 &   \\
            open system moving  with& constant          &     $dG$ = 0                      & d$^{2}$G $>$ 0    \\
           the fluid velocity    &                      &                                   &                   \\
         \hline\hline
         \end{tabular}
     \end{center}
   \end{frame}
\normalsize

%%%
%%% SUBSECTION
%%%
\subsection{Phase Rule}
%%%
%%% Slide
%%%
\begin{frame}
  \frametitle{Phase Rule and Duhem's Theorem}
  \begin{enumerate}
      \item<1-> From Module 1, we defined that a system may be characterised by $T$, $P$ and $\left(\mathcal{C}-1\right)$ mole fractions for each phase $\mathcal{P}$ (where $\mathcal{C}$ is the number of chemical components);
      \item<1-> However, the masses of the phases {\bf are not} $\lq$phase-rule' variables as they do not affect the intensive state properties of the system;
      \item<2-> Thus, in order to determine a system we need to know \textcolor{blue}{$2 + \left(\mathcal{C}-1\right)\mathcal{P}$} variables;
      \item<3-> At equilibrium, $\mu_{i}^{\alpha} = \mu_{i}^{\beta} = \mu_{i}^{\gamma} = \cdots = \mu_{i}^{\mathcal{P}}$ for all $\mathcal{C}$ components;
      \item<4->These relations provide \textcolor{blue}{$\left(\mathcal{P}-1\right)\mathcal{C}$} equations;
      \item<5-> The difference is the \textcolor{blue}{degree of freedom ($\Psi$)} of the system, leads to
         \visible<5->{\begin{displaymath}
             \Psi = \left[2 + \left(\mathcal{C}-1\right)\mathcal{P}\right] - \left[\left(\mathcal{P}-1\right)\mathcal{C}\right] %\textcolor{blue}{= 2 + \mathcal{C} - \mathcal{P}} 
         \end{displaymath}}  
  \end{enumerate}
  \visible<6->{\begin{block}{\begin{center}Gibbs Phase Rule\end{center}}
     \begin{displaymath}
         \textcolor{blue}{\Psi = 2 + \mathcal{C} - \mathcal{P}}
     \end{displaymath} 
  \end{block}}
\end{frame}


%%%
%%% Slide
%%%
\begin{frame}
  \frametitle{Phase Rule and Duhem's Theorem}
  \begin{enumerate}\setcounter{enumi}{6}
      \item<1-> If the system is closed and formed from specified amounts of each component, we can thus write:
         \visible<1->{\begin{tabular}{l l}
            Equilibrium equations for chemical potentials: & $\left(\mathcal{P}-1\right)\mathcal{C}$ \\
            Material balance for each component:           & $\mathcal{C}$ \\
            \hline
            {\bf Total:}                                   & $\mathbf{\mathcal{P}\mathcal{C}}$ {\bf equations}
         \end{tabular}}  
      \item<2-> The system is characterised by
         \visible<1->{\begin{tabular}{l l}
            $T$, $P$ and $\left(\mathcal{C}\right)$ mole fractions for each phase: & $2 + \left(\mathcal{C}-1\right)\mathcal{P}$ \\
            Masses of each phase:                                           & $\mathcal{P}$ \\
            \hline
            {\bf Total:}                                   & $\mathbf{2 + \mathcal{P}\mathcal{C}}$ {\bf variables}
         \end{tabular}}  
      \item<3-> Therefore, in order to determine the equilibrium state we just need $\left[2 + \mathcal{P}\mathcal{C}\right] - \mathcal{P}\mathcal{C} =$ \textcolor{blue}{2 variables}.
  \end{enumerate}
  \visible<4->{\begin{block}{\begin{center}Duhem's Theorem \end{center}}
         \textcolor{blue}{$\lq$For any closed system of known composition, the equilibrium state is determined when any {\bf two independent variables} are fixed.'}
  \end{block}}
\end{frame}


%%%
%%% SECTION
%%%
\section{Vapour-Liquid Equilibrium (VLE)}

\subsection{Qualitative Behaviour}
%%%
%%% Slide
%%%
%\scriptsize
\begin{frame}
  \frametitle{$P$-$T$-$xy$ Diagram}
  \begin{columns}
     \begin{column}[l]{0.3\linewidth}
        \begin{enumerate}
            \item<1-> System with $\mathcal{C} =2$ (hexane and triethylamine, extracted from Sandler, 2006);
            \item<2-> From the phase rule: $\Psi = 4 -\mathcal{P}$;
                \begin{enumerate}
                    \item<3-> For $\mathcal{P}=1$ $\Longrightarrow$ $\Psi=3$, i.e., $P$, $T$ and {\bf one} mole fraction;
                    \item<4-> For $\mathcal{P}=2$ $\Longrightarrow$ $\Psi=2$ $\leftrightarrow$ define surfaces;
                \end{enumerate}
            \item<5-> \textcolor{blue}{Surfaces} represent equilibrium states: saturated vapour and liquid.
        \end{enumerate}
     \end{column}
     \begin{column}[l]{0.7\linewidth}
        \hbox{\hspace{-1cm}
           \includegraphics[width=9.7cm,clip]{./Pics/PTxy_diagram}}
     \end{column}
  \end{columns}
\end{frame}
\normalsize

%%%
%%% Slide
%%%
%\scriptsize
\begin{frame}
  \frametitle{$P$-$T$-$xy$ Diagram}
     \begin{center}
       \includegraphics[width=8.cm,clip]{./Pics/PTxy_diagram2}
     \end{center}
\end{frame}
\normalsize


%%%
%%% Slide
%%%
%\scriptsize
\begin{frame}
  \frametitle{$P$-$T$-$xy$ Diagram}
     \begin{center}
       \includegraphics[width=8.cm,clip]{./Pics/PTxy_diagram3}
     \end{center}
\end{frame}
\normalsize


%%%
%%% Slide
%%%
%\scriptsize
\begin{frame}
  \frametitle{$P-T$ Diagrams at fixed composition}
  \begin{columns}
     \begin{column}[l]{0.5\linewidth}
        \includegraphics[width=\linewidth,clip]{./Pics/VLE_PT_diagram1}
     \end{column}
     \begin{column}[l]{0.5\linewidth}
        \includegraphics[width=\linewidth,clip]{./Pics/VLE_PT_diagram2}
     \end{column}
  \end{columns}
  \begin{center}
     Solid line: saturated liquid (bubble line); \\
     Dashed line: saturated vapour (dew line) 
  \end{center}
\end{frame}
\normalsize


%%%
%%% Slide
%%%
\scriptsize
\begin{frame}
  \frametitle{$Pxy$ Diagrams}
  \vspace{-0.8cm}
  \begin{columns}
     \begin{column}[l]{0.5\linewidth}
        \visible<1->{\hbox{\hspace{-1.7cm}\includegraphics[width=1.6\linewidth,clip]{./Pics/VLE_Pxy_Diagram1}}}
        %Upper curve: liquid phase composition $x$;\\
        %Lower curve: vapour phase composition $y$;
     \end{column}
     \begin{column}[l]{0.5\linewidth}
       \visible<2->{\hbox{\hspace{-1.7cm}\includegraphics[width=1.6\linewidth,clip]{./Pics/VLE_Pxy_Diagram2}}}
     \end{column}
  \end{columns}
         \visible<1->{Upper and lower curves represent liquid ($x$) and vapour phases compositions.}\\
         \visible<2->{$P_{H}$ and $P_{L}$ are the vapour pressure of pure heavy and light components at temperature $T$, respectively.}
\end{frame}
\normalsize


%%%
%%% Slide
%%%
%\scriptsize
\begin{frame}
  \frametitle{$Pxy$ Diagrams}
  \begin{columns}
     \begin{column}[l]{0.5\linewidth}
       \begin{enumerate}
          \item<1-> For a given \textcolor{blue}{initial (feed) composition $z$} (at constant $T$), if pressure is raised to $P_{\text{DP}}$, the first droplet of liquid is formed. \textcolor{blue}{$P_{\text{DB}}$} is called {\bf \textcolor{blue}{dew point}};
          \item<2-> The liquid droplets are much leaner in the light component and the composition is ${\bf \textcolor{blue}{x_{\text{DB}}}}$;
          \item<3-> As the pressure is further increased, more liquid is formed. At \textcolor{blue}{P$_{1}$}, liquid and vapour are in equilibrium with composition $\textcolor{blue}{x_{1}}$ and $\textcolor{blue}{y_{1}}$;
          \item<4-> When the pressure reaches \textcolor{blue}{P$_{\text{BP}}$}, the last bubble of vapour condenses -- {\bf \textcolor{blue}{bubble point}}.  
       \end{enumerate}
     \end{column}
     \begin{column}[l]{0.5\linewidth}
        \visible<1->{\hbox{\hspace{-1.7cm}\includegraphics[width=1.6\linewidth,clip]{./Pics/VLE_Pxy_Diagram3}}}
     \end{column}
  \end{columns}
\end{frame}
\normalsize


%%%
%%% Slide
%%%
%\scriptsize
\begin{frame}
  \frametitle{$xy$ Diagrams}
  \begin{columns}
     \begin{column}[l]{0.4\linewidth}
       \begin{enumerate}
          \item<1-> The $xy$ diagram for a binary system correlates the compositions of the liquid and vapour phases in equilibrium;
          \item<1-> They are designed assuming either constant $P$ or $T$ $\Longrightarrow$ though most industrial applications are isobaric.
       \end{enumerate}
     \end{column}
     \begin{column}[l]{0.6\linewidth}
        \visible<1->{\hbox{\hspace{-.0cm}\includegraphics[width=1.\linewidth,clip]{./Pics/VLE_xy_Diagram1}}}
     \end{column}
  \end{columns}
\end{frame}
\normalsize


%%%
%%% Slide
%%%
%\scriptsize
\begin{frame}
  \frametitle{$xy$ Diagrams: Ideal and Non-Ideal Solutions}
  \vbox{
     \hbox{\visible<1->{\includegraphics[width=5.5cm,height=4.cm,clip]{./Pics/VLE_xy_DiagramIdeal}} \hspace{1cm}
           \visible<2->{\includegraphics[width=5.5cm,height=4.cm,clip]{./Pics/VLE_xy_DiagramNonIdeal1}}}
  \vspace{-0.2cm}
  \hbox{\hspace{4cm}
        \visible<3->{\includegraphics[width=5.5cm,height=4.cm,clip]{./Pics/VLE_xy_DiagramNonIdeal2}}}
  }
\end{frame}
\normalsize


%%%
%%% SUBSECTION
%%%
\subsection{Simple Models}

%%%
%%% Slide
%%%
\begin{frame}
  \frametitle{General Remarks}
  \begin{enumerate}
      \item<1-> \textcolor{blue}{VLE models} aim for mathematical descriptions of the PVT behaviour of mixtures at equilibrium conditions:
         \begin{enumerate}
            \item<1-> Prediction of \textcolor{blue}{vapour and liquid compositions} at given $T$ and $P$;
            \item<1-> Basis for process modelling.
         \end{enumerate}
      \item<2-> Two simple models:
         \begin{enumerate}
            \item<2-> \textcolor{blue}{Raoult's law};
            \item<2-> \textcolor{blue}{Henry's law}.
         \end{enumerate}
  \end{enumerate}
\end{frame}

%%%
%%% Slide
%%%
\begin{frame}
  \frametitle{Raoult's Law}
  \begin{enumerate}
      \item<1-> Main assumptions:
         \begin{enumerate}
             \item<1-> \textcolor{blue}{vapour phase} behaves as an {\bf ideal gas} $\Longrightarrow$ low to moderate pressures;
             \item<1-> \textcolor{blue}{liquid phase} is an {\bf ideal solution}.
         \end{enumerate}
         \visible<2->{\begin{block}{Raoult's Law}
              \begin{displaymath}
                  \textcolor{blue}{ P_{i}= y_{i}P = x_{i} P_{i}^{\text{sat}}\left(T\right)}\;\;\;\;\textcolor{blue}{\forall i\in\left\{1,2,\cdots,\mathcal{C}\right\}}
              \end{displaymath}
         \end{block}
         where $P_{i}=y_{i}P$ is the {\bf partial pressure} of species {\it i} in the vapour phase.}
      \item<2-> This equation indicates that the partial pressure of a component in an ideal solution is equal to the product of the species mole fraction and its vapour pressure (of the pure component);
      \item<3-> Two major constraints:
         \begin{enumerate}
             \item<2-> Mass balance of both phases: $\sum\limits_{i=1}^{\mathcal{C}} x_{i} = 1$ and $\sum\limits_{i=1}^{\mathcal{C}} y_{i} = 1$
             \item<2-> {\bf Ideal solution:} \textcolor{blue}{$x_{i}\rightarrow 1$}. Also species {\bf must} be chemically similar (i.e., size, same chemical nature, e.g., isomers such as ortho-, meta-, and para-xylene) 
         \end{enumerate}
  \end{enumerate}
\end{frame}


%%%
%%% Slide
%%%
\begin{frame}
  \frametitle{Raoult's Law: Ideal gas mixture properties}
  \begin{enumerate}
      \item<1-> Dalton's Law: \textcolor{blue}{$P=\sum\limits_{i=1}^{\mathcal{C}}P_{i}=\sum\limits_{i=1}^{\mathcal{C}}y_{i}P$}
      \item<2-> Amagat's Law: \textcolor{blue}{$V^{t}=\sum\limits_{i=1}^{\mathcal{C}}V_{i}^{t} = \sum\limits_{i=1}^{\mathcal{C}} y_{i}V^{t}$}
      \item<3-> Kay's rule: pseudo-critical pressure \textcolor{blue}{$\left(T_{c}^{t}=\sum\limits_{i=1}^{\mathcal{C}}y_{i}T_{c,i}\right)$} and temperature \textcolor{blue}{$\left(T_{c}^{t}=\sum\limits_{i=1}^{\mathcal{C}}y_{i}T_{c,i}\right)$}. 
  \end{enumerate}
\end{frame}


%%%
%%% Slide
%%%
%\scriptsize
\begin{frame} % Sandler Example 10.1.1(page 498)
  \frametitle{Example 1:} 
    Assuming a mixture of n-pentane $\left(nC_{5}\right)$ and n-heptane $\left(nC_{7}\right)$ is ideal, draw vapour-liquid equilibrium diagrams for this mixtures at:
    \begin{enumerate}
        \item Constant temperature of 50$^{\circ}$C;
        \item Constant pressure of 1.013 bar.
    \end{enumerate}
    Given Antoine relation, $\ln P_{i}^{\text{sat}} = A_{i} - \frc{B_{i}}{RT}$, with $A_{nC_{5}}=10.422$, $A_{nC_{7}}=11.431$, $B_{nC_{5}}=26799$ and $B_{nC_{7}}=35200$. [$P$] = bar and [$T$] = K
\end{frame}
\normalsize


%%%
%%% Slide
%%%
\begin{frame}
  \frametitle{Raoult's Law: Dew and Bubble Point Calculations}
  \begin{description}
      \item[Dew Point:]<2->: Calculate, 
           \begin{enumerate}
               \item<2-> $\textcolor{blue}{\bf x_{i}}$ and \textcolor{blue}{$P$}, given $y_{i}$ and $T$;
               \item<2-> $\textcolor{blue}{\bf x_{i}}$ and \textcolor{blue}{$T$}, given $y_{i}$ and $P$;
           \end{enumerate}
      \item[Bubble Point:]<3-> Calculate, 
           \begin{enumerate}
               \item<3-> $\textcolor{blue}{\bf y_{i}}$ and \textcolor{blue}{$P$}, given $x_{i}$ and $T$;
               \item<3-> $\textcolor{blue}{\bf y_{i}}$ and \textcolor{blue}{$T$}, given $x_{i}$ and $P$.
           \end{enumerate}
  \end{description}
\end{frame}

%%%
%%% Slide
%%%
%\scriptsize
\begin{frame} % Sandler Example 10.1.2(page 501)
  \frametitle{Example 2:}
    \begin{enumerate}
        \item Estimate the bubble and dew point temperatures of a 25 mol-$\%$ n-pentane $\left(nC_{5}\right)$, 45 mol-$\%$ n-hexane $\left(nC_{6}\right)$ and 30 mol-$\%$ n-heptane $\left(nC_{7}\right)$ mixture at 1.013 bar. 
        \item  Estimate the bubble and dew point pressures of this mixture at 73$^{\circ}$C.
    \end{enumerate}
    Given the Antonie equation:
    \begin{displaymath}
       \ln P_{i}^{\text{sat}} = A_{i} - \frc{B_{i}}{RT}
    \end{displaymath}

    with
    \begin{center}
       \begin{tabular}{l l l} 
          $A_{nC_{5}}=10.422$ & $A_{nC_{6}}=10.456$ & $A_{nC_{7}}=11.431$ \\
          $B_{nC_{5}}=26799$  & $B_{nC_{6}}=29676$  & $B_{nC_{7}}=35200$.
       \end{tabular}
    \end{center}
\end{frame}
\normalsize

%%%
%%% Slide
%%%
\begin{frame}
  \frametitle{Generalised Relation for VLE in Ideal Mixtures}
  \begin{enumerate}
      \item<1-> Raoult's law is a particular case of a more general relation involving \textcolor{blue}{vapour-liquid mixtures} (that we will see with more details on Module 6),
           \visible<2->{\begin{displaymath}
              f_{i}^{\left(L\right)}\left(T,P,\underline{x}\right) = f_{i}^{\left(V\right)}\left(T,P,\underline{y}\right)
           \end{displaymath}
           where $\underline{x}$ and $\underline{y}$ represent the array of compositions in the liquid and vapour phases, respectively.}
      \item<3-> For low-pressure VLE,
           \visible<3->{\begin{displaymath}
               \textcolor{blue}{y_{i}P=x_{i}\gamma_{i}P_{i}^{\text{sat}}}
           \end{displaymath}}
      \item<4-> The activity  coefficient of species $i$, $\gamma_{i}$, is a thermodynamic property that indicates $\lq$how much a solution will deviate from the ideal solution';
      \item<4-> For an ideal solution, $\gamma_{i}=1$, and the equation above becomes the Raoult's law.  
  \end{enumerate}
\end{frame}

%%%
%%% Slide
%%%
\begin{frame}
  \frametitle{Henry's Law}
  \begin{enumerate}
      \item<1-> Raoult's law requires vapour pressure data $\left(\text{i.e., }P_{i}^{\text{sat}}\right)$, thus:
        \begin{enumerate}
            \item<1-> It is not applicable if $T\leq T_{c,i}$;
            \item<1-> Does not take into account gasses dissolved in the liquid.
        \end{enumerate}
      \item<2-> Henry's law:
         \visible<2->{\begin{displaymath}
             \textcolor{blue}{y_{i}P = x_{i}\mathcal{H}_{i}}\;\;\;\;\forall i\in\left\{1,2,\cdots,\mathcal{C}\right\}
         \end{displaymath}}
  \end{enumerate}
  \visible<2->{
 \begin{table}
  \begin{center}
    \begin{tabular}{l r || l r }
      \hline
       {\bf Gas}    &  ${\bf \mathcal{H}\text{ (bar)}}$ & {\bf Gas}    &  ${\bf \mathcal{H}\text{ (bar)}}$ \\
      \hline
         Acetylene  &   1350                            & He           &  126600 \\
         Air        &   72950                           & H$_{2}$      &  71600  \\
         CO$_{2}$    & 1670                              & H$_{2}$S     & 550 \\
         CO         &  54600                            &  CH$_{4}$    &  41850 \\
         C$_{2}$H$_{6}$ & 30600                          &  N$_{2}$     & 87650  \\
         Ethylene  & 11550                              & O$_{2}$      & 44380 \\
      \hline
    \end{tabular}
    \caption{Henry's constant for gases dissolved in water at 25$^{\circ}$C.}
  \end{center}
\end{table}}
\end{frame}


%%%
%%% Slide
%%%
\begin{frame}
  \frametitle{K-Value Correlations}
  \begin{columns}
     \begin{column}[l]{0.5\linewidth}
        \begin{enumerate}
           \item<1-> Equilibrium ratio, $K_{i}$
               \visible<1->{\begin{displaymath}
                  K_{i} = \frc{y_{i}}{x_{i}}
               \end{displaymath}}
           \item<2-> With Raoult's law:
               \visible<2->{\begin{displaymath}
                  K_{i} = \frc{P_{i}^{\text{sat}}}{P}
               \end{displaymath}}
           \item<3-> With modified Raoult's law:
               \visible<3->{\begin{displaymath}
                  K_{i} = \frc{\gamma_{i}P_{i}^{\text{sat}}}{P}
               \end{displaymath}}
        \end{enumerate}
     \end{column}
     \begin{column}[l]{0.5\linewidth}
        \begin{center}
            \hspace{-.8cm}\includegraphics[width=6.1cm,clip]{./Pics/02_07_fig_02.png}
        \end{center}
     \end{column}
   \end{columns}
\end{frame}


%%%
%%% Slide
%%%
\begin{frame}
  \frametitle{Flash Calculations}
  \begin{columns}
     \begin{column}[c]{0.5\linewidth}
       \visible<3->{\hspace{-1cm}\includegraphics[width=1.3\linewidth,clip]{./Pics/Flash_VLE}}
     \end{column}
     \begin{column}[l]{0.5\linewidth}
        \begin{enumerate}
           \item<1-> If a fluid mixture is at liquid state with pressure larger than the associated bubble point pressure $\left(\text{i.e., }P \leq P_{\text{BP}}\right)$;
           \item<2-> Part of the fluid evaporates \textcolor{blue}{(or {\it flashes})} when the pressure is reduced leading to a two-phase system -- vapour and liquid in equilibrium;
            \item<3-> This phenomenon is often called \textcolor{blue}{\it flash} and is the most important application of VLE;
            \item<4-> Thus, a feeding stream $F$ with overall composition $z_{i}$ is separated into vapour ($V$) and liquid ($L$) streams with compositions $y_{i}$ and $x_{i}$, respectively. 
        \end{enumerate}
     \end{column}
   \end{columns}
\end{frame}

%%%
%%% Slide
%%%
%\scriptsize
\begin{frame} % Sandler Example 10.1.4(page 504)
  \frametitle{Example 3:}
        A liquid mixture of 25 mol-$\%$ n-pentane $\left(nC_{5}\right)$, 45 mol-$\%$ n-hexane $\left(nC_{6}\right)$ and 30 mol-$\%$ n-heptane $\left(nC_{7}\right)$ initially at 69$^{\circ}$C and a high pressure, is partially vaporised by isothermically lowering the pressure to  1.013 bar. Calculate the relative amounts of vapour and liquid in equilibrium and compositions.
\end{frame}
\normalsize


\section{Summary}

%%%
%%% Slide
%%%
%\scriptsize
\begin{frame}
 \frametitle{Summary}
   \begin{enumerate}[(i)]
     \item Stability conditions/criteria;
     \item Phase rule and Duhem's theorem;
     \item Qualitative analysis of VLE -- efficient use of graphical representation of $P$, $T$ and composition of species;
     \item Introduction to generalised relation for VLE in non-ideal mixtures;
     \item Raoult and Henry's laws; 
     \item Industrial application for VLE: Flash problem.
   \end{enumerate}
\end{frame}


\end{document}
 



\end{document}
