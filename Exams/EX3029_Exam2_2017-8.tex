
\documentclass[calculator,allquestions,datasheet,solutions]{exam_newMarcus2}
%\documentclass[calculator,allquestions,datasheet,Pens]{exam_newMarcus2}

% The full list of class options are
% calculator : Allows approved calculator use.
% datasheet : Adds a note that data sheet are attached to the exam.
% handbook : Allows the use of the engineering handbook.
% resit : Adds the resit markings to the paper.
% sample : Adds conspicuous SAMPLE markings to the paper
% solutions : Uses the contents of \solution commands (and \solmarks) to generate a solution file

\usepackage{pdfpages}  
\usepackage{lscape,comment} 
 
\coursecode{EX3029}%% 
\coursetitle{Chemical Thermodynamics}
 
\examtime{00.00--00.00}%
\examdate{15}{12}{2017}% 
\examformat{Attempt ALL questions. \\ Each question is worth 20 marks.}

% Other symbols
\newcommand{\frc}{\displaystyle\frac}
\newcommand{\br}[1]{\!\left( #1 \right)}
\newcommand{\abs}[1]{\left| #1 \right|}
\newcommand{\fracd}[2]{\frac{\mathrm{d} #1}{\mathrm{d} #2}}
\newcommand{\fracp}[2]{\frac{\partial #1}{\partial #2}}
\renewcommand{\d}[1]{\mathrm{d} #1 } 
\newcommand{\Ma}{\mathrm{M\!a}} 
\newcommand{\eg}{{\it e.g., }}
\newcommand{\ie}{{\it i.e., }}
\newcommand{\wrt}{{\it wrt }}
\newcommand{\Partial}[3][error]{\left(\frc{\partial #1}{\partial #2}\right)_{#3}}
\newcommand{\mfr}[3][error]{#1_{#2}^{\left(#3\right)}} 
\newcommand{\summation}[3][error]{\sum\limits_{#2}^{#3}#1}


\begin{document}


%%%
%%% Question 01 
%%%
\begin{question}
   An engineering consultancy company is contracted to design a separation process for a mixture of petroleum naphtha and fertilisers by-products. The mixture will be separated by a series of distillation and crystallisation processes. After the first distillation, lighter and heavier streams are stored in two pressure vessels:
         \begin{itemize}
            \item {\bf Vessel 1:} 45 mol-$\%$ of n-hexane, 30 mol-$\%$ of n-heptane and 25 mol-$\%$ of i-octane;
            \item {\bf Vessel 2:} 1.70 mol-$\%$ of n-hexane  and 98.30 mol-$\%$ of chlorobenzene.
         \end{itemize} 
         Vessels 1 and 2 are kept at 1.5 and 4.5 bar, respectively. In order to design the second set of distillations, 
         \begin{enumerate}[a)]
            \item  Estimate the bubble and dew temperatures for Vessel 1. Also calculate the compositions at bubble and dew points;~\marks{14}
                \solution{\begin{enumerate}[i)]
                            \item Bubble point:
                                 \begin{eqnarray}
                                    \sum\limits_{i=1}^{n} y_{i} &=& \sum\limits_{i=1}^{n} \frc{x_{i}P_{i}^{\text{sat}}}{P} = 1 \nonumber \\
                                     &=& \frc{x_{C6}P_{C6}^{\text{sat}}}{P} + \frc{x_{C7}P_{C7}^{\text{sat}}}{P} + \frc{x_{C8}P_{C8}^{\text{sat}}}{P} = 1 \nonumber \\
                                     &=& x_{C6}P_{C6} + x_{C7}P_{C7} + x_{C8}P_{C8} = P, \nonumber 
                                  \end{eqnarray}
                                  with the Antoine relation,
                                  \begin{displaymath}
                                    \ln{P^{\text{sat}}} = A - \frc{B}{T+C},
                                  \end{displaymath}
                                  with $P$ 150 kPa. Solving this non-linear equation (using a calculator) leads to the bubble point temperature $\Longrightarrow$ $T_{\text{bubble}}$ = 95.68$^{\circ}$C~\solmarks{2/8}.  Vapour phase composition is obtained from:
                      \begin{displaymath}
                         y_{i} = \frc{x_{i}P_{i}^{\text{sat}}}{P}.
                      \end{displaymath}
                      Leading to y = [ 0.6603, 0.1870, 0.1527 ]. ~\solmarks{2/8}
%
               \item Dew point:
                    \begin{eqnarray}
                        \sum\limits_{i=1}^{n} x_{i} &=& \sum\limits_{i=1}^{n} \frc{y_{i}P}{P_{i}^{\text{sat}}} = 1 \nonumber \\
                                                 &=& \frc{y_{C6}P}{P_{C6}^{\text{sat}}} + \frc{y_{C7}P}{P_{C7}^{\text{sat}}} + \frc{y_{C8}P}{P_{C8}^{\text{sat}}} = 1 \nonumber 
                    \end{eqnarray}
                    Solving this non-linear equation (using a calculator) leads to the dew point temperature $\Longrightarrow$ $T_{\text{dew}}$ = 102.10$^{\circ}$C~\solmarks{2/8}.  Liquid phase composition is obtained from:
                      \begin{displaymath}
                         x_{i} = \frc{y_{i}P}{P_{i}^{\text{sat}}}.
                      \end{displaymath}
                      Leading to x = [ 0.2599, 0.3989, 0.3412 .] ~\solmarks{2/8}
                          \end{enumerate}
                }
            %\item  Estimate the bubble and dew temperatures for Vessel 2. Also calculate the compositions at bubble and dew points;~\marks{7}
            \item  Sketch the $T-xy$ diagram for the mixture in vessel 1, indicating bubble and dew point temperature and compositionss.~\marks{6}
         \end{enumerate}

         Sauration pressure, $P^{\text{sat}}$, can be obtained from the Antoine equation,
          \begin{displaymath}
            \ln{P^{\text{sat}}} = A - \frc{B}{T+C}
          \end{displaymath}
          where $\left[P^{\text{sat}}\right]$ = kPa and $[T]$ = $^{\circ}$C, with coefficients given in Table~\ref{Practical1:Table2}.
           

\begin{table}[h]
\begin{center}
\begin{tabular}{||c | c c c ||} 
\hline\hline
                           & {\bf A}    &  {\bf B}    & {\bf C}    \\
\hline
{\bf n-hexane}             & 13.8193    & 2696.04     & 224.317    \\  
{\bf n-heptane}            & 13.8622    & 2910.26     & 216.432    \\  
{\bf i-octane}             & 13.6703    & 2896.31     & 220.767    \\ 
{\bf chlorobenze}           & 13.8635    & 3174.78     & 211.700    \\  
\hline\hline
\end{tabular}
\caption{Constants for the Antoine equation for vapour pressure.}
\label{Practical1:Table2}
\end{center}
\end{table}

\end{question}

\clearpage


%%%
%%% Question 02
%%%
\begin{question}
  \begin{enumerate}[a)]
    \item u
  \end{enumerate}
\end{question}

\clearpage



%%%
%%% Question 03
%%%
\begin{question}
  \begin{enumerate}[(a)]
    \item u
  \end{enumerate}
\end{question}

\clearpage


%%%
%%% Question 04
%%%
\begin{question}
  \begin{enumerate}[a)]
    \item u
  \end{enumerate}
\end{question}

\clearpage

%%%
%%% QUESTION 05
%%%
\begin{question}
  \begin{enumerate}[a)]
    \item u
  \end{enumerate}
\end{question}


\vfill
\paperend
 


\vfill 



%\begin{comment}
{
  \includepdf[pages=-,fitpaper]{./Pics/EquationsList}
}
%\end{comment}



\end{document}
