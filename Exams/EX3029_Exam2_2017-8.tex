
\documentclass[calculator,allquestions,datasheet,solutions]{exam_newMarcus2}
%\documentclass[calculator,allquestions,datasheet,Pens]{exam_newMarcus2}

% The full list of class options are
% calculator : Allows approved calculator use.
% datasheet : Adds a note that data sheet are attached to the exam.
% handbook : Allows the use of the engineering handbook.
% resit : Adds the resit markings to the paper.
% sample : Adds conspicuous SAMPLE markings to the paper
% solutions : Uses the contents of \solution commands (and \solmarks) to generate a solution file

\usepackage{pdfpages}  
\usepackage{lscape,comment} 
 
\coursecode{EX3029}%% 
\coursetitle{Chemical Thermodynamics}
 
\examtime{00.00--00.00}%
\examdate{15}{12}{2017}% 
\examformat{Attempt ALL questions. \\ Each question is worth 20 marks.}

% Other symbols
\newcommand{\frc}{\displaystyle\frac}
\newcommand{\br}[1]{\!\left( #1 \right)}
\newcommand{\abs}[1]{\left| #1 \right|}
\newcommand{\fracd}[2]{\frac{\mathrm{d} #1}{\mathrm{d} #2}}
\newcommand{\fracp}[2]{\frac{\partial #1}{\partial #2}}
\renewcommand{\d}[1]{\mathrm{d} #1 } 
\newcommand{\Ma}{\mathrm{M\!a}} 
\newcommand{\eg}{{\it e.g., }}
\newcommand{\ie}{{\it i.e., }}
\newcommand{\wrt}{{\it wrt }}
\newcommand{\Partial}[3][error]{\left(\frc{\partial #1}{\partial #2}\right)_{#3}}
\newcommand{\mfr}[3][error]{#1_{#2}^{\left(#3\right)}} 
\newcommand{\summation}[3][error]{\sum\limits_{#2}^{#3}#1}


\begin{document}


%%%
%%% Question 01 
%%%
\begin{question}
   An engineering consultancy company is contracted to design a separation process for a mixture of petroleum naphtha and fertilisers by-products. The mixture will be separated by a series of distillation and crystallisation processes. After the first distillation, lighter and heavier streams are stored in two pressure vessels:
         \begin{itemize}
            \item {\bf Vessel 1:} 45 mol-$\%$ of n-hexane, 30 mol-$\%$ of n-heptane and 25 mol-$\%$ of i-octane;
            \item {\bf Vessel 2:} 20 mol-$\%$ of n-hexane  and 80 mol-$\%$ of chlorobenzene.
         \end{itemize} 
         Vessels 1 and 2 are kept at 1.5 and 1.0 bar, respectively. In order to design the second set of distillations, 
         \begin{enumerate}[a)]
            \item  Estimate the bubble and dew temperatures for Vessel 1. Also calculate the compositions at bubble and dew points;~\marks{8}
                \solution{\begin{enumerate}[i)]
                            \item Bubble point:
                                 \begin{eqnarray}
                                    \sum\limits_{i=1}^{n} y_{i} &=& \sum\limits_{i=1}^{n} \frc{x_{i}P_{i}^{\text{sat}}}{P} = 1 \nonumber \\
                                     &=& \frc{x_{C6}P_{C6}^{\text{sat}}}{P} + \frc{x_{C7}P_{C7}^{\text{sat}}}{P} + \frc{x_{C8}P_{C8}^{\text{sat}}}{P} = 1 \nonumber \\
                                     &=& x_{C6}P_{C6} + x_{C7}P_{C7} + x_{C8}P_{C8} = P, \nonumber 
                                  \end{eqnarray}
                                  with the Antoine relation,
                                  \begin{displaymath}
                                    \ln{P^{\text{sat}}} = A - \frc{B}{T+C},
                                  \end{displaymath}
                                  with $P$ 150 kPa. Solving this non-linear equation (using a calculator) leads to the bubble point temperature $\Longrightarrow$ $T_{\text{bubble}}$ = 95.68$^{\circ}$C~\solmarks{2/8}.  Vapour phase composition is obtained from:
                      \begin{displaymath}
                         y_{i} = \frc{x_{i}P_{i}^{\text{sat}}}{P}.
                      \end{displaymath}
                      Leading to y = [ 0.6603, 0.1870, 0.1527 ]. ~\solmarks{2/8}
%
               \item Dew point:
                    \begin{eqnarray}
                        \sum\limits_{i=1}^{n} x_{i} &=& \sum\limits_{i=1}^{n} \frc{y_{i}P}{P_{i}^{\text{sat}}} = 1 \nonumber \\
                                                 &=& \frc{y_{C6}P}{P_{C6}^{\text{sat}}} + \frc{y_{C7}P}{P_{C7}^{\text{sat}}} + \frc{y_{C8}P}{P_{C8}^{\text{sat}}} = 1 \nonumber 
                    \end{eqnarray}
                    Solving this non-linear equation (using a calculator) leads to the dew point temperature $\Longrightarrow$ $T_{\text{dew}}$ = 102.10$^{\circ}$C~\solmarks{2/8}.  Liquid phase composition is obtained from:
                      \begin{displaymath}
                         x_{i} = \frc{y_{i}P}{P_{i}^{\text{sat}}}.
                      \end{displaymath}
                      Leading to x = [ 0.2599, 0.3989, 0.3412 ] ~\solmarks{2/8}
                          \end{enumerate}
                }
            \item  Estimate the bubble and dew temperatures for Vessel 2. Also calculate the compositions at bubble and dew points;~\marks{6}
                \solution{\begin{enumerate}[i)]
                            \item Bubble point:
                                 \begin{eqnarray}
                                    \sum\limits_{i=1}^{n} y_{i} &=& \sum\limits_{i=1}^{n} \frc{x_{i}P_{i}^{\text{sat}}}{P} = 1 \nonumber \\
                                     &=& \frc{x_{C6}P_{C6}^{\text{sat}}}{P} + \frc{x_{ClBz}P_{ClBz}^{\text{sat}}}{P} = 1 \nonumber \\
                                     &=& x_{C6}P_{C6} + x_{ClBz}P_{C8} = P, \nonumber 
                                  \end{eqnarray}
                                  with the Antoine relation,
                                  \begin{displaymath}
                                    \ln{P^{\text{sat}}} = A - \frc{B}{T+C},
                                  \end{displaymath}
                                  with $P$ 100 kPa. Solving this non-linear equation (using a calculator) leads to the bubble point temperature $\Longrightarrow$ $T_{\text{bubble}}$ = 107.63$^{\circ}$C~\solmarks{2/6}.  Vapour phase composition is obtained from:
                      \begin{displaymath}
                         y_{i} = \frc{x_{i}P_{i}^{\text{sat}}}{P}.
                      \end{displaymath}
                      Leading to y = [ 0.5962, 0.4038 ]. ~\solmarks{2/6}
%
               \item Dew point:
                    \begin{eqnarray}
                        \sum\limits_{i=1}^{n} x_{i} &=& \sum\limits_{i=1}^{n} \frc{y_{i}P}{P_{i}^{\text{sat}}} = 1 \nonumber \\
                                                 &=& \frc{y_{C6}P}{P_{C6}^{\text{sat}}} + \frc{y_{ClBz}P}{P_{ClBz}^{\text{sat}}} = 1 \nonumber 
                    \end{eqnarray}
                    Solving this non-linear equation (using a calculator) leads to the dew point temperature $\Longrightarrow$ $T_{\text{dew}}$ = 124.78$^{\circ}$C~\solmarks{2/8}.  Liquid phase composition is obtained from:
                      \begin{displaymath}
                         x_{i} = \frc{y_{i}P}{P_{i}^{\text{sat}}}.
                      \end{displaymath}
                      Leading to x = [0.0450, 0.9559] ~\solmarks{2/6}
                          \end{enumerate}
                }
              \item  Sketch the $T-xy$ diagram for the mixture in vessel 2, indicating saturation, bubble and dew point temperatures and compositionss.~\marks{6}
                \solution{From Antoine relation,
                                  \begin{displaymath}
                                    \ln{P^{\text{sat}}} = A - \frc{B}{T+C},
                                  \end{displaymath}
                                  with $P=P^{\text{sat}}=$ 100 kPa (\ie pure component), we can obtain the $T^{\text{sat}}$ for both components,~\solmarks{2/6}
                                  \begin{displaymath}
                                    \begin{cases}
                                      T^{\text{sat}}_{C6}= 68.28^{\circ}\text{C}, &\\
                                      T^{\text{sat}}_{ClBz}= 131.21^{\circ}\text{C}, &\\                                      
                                    \end{cases}
                                  \end{displaymath}
                                  \begin{itemize}
                                     \item Correct shape of the $T-xy$ diagram;~\solmarks{1/6}
                                     \item Correct allocation of compositions;~\solmarks{1/6}
                                     \item Correct of bubble and dew temperatures;~\solmarks{2/6}
                                  \end{itemize}

      \begin{figure}[h]
        \begin{center}
          \includegraphics[width=.8\linewidth,clip]{./Pics/VLE_Txy_Diagram}
          \caption{VLE for binary mixture: T-xy diagram at constant pressure.}\label{Figure:Fig1}
        \end{center}
        \end{figure} 
                  }
         \end{enumerate}

         Sauration pressure, $P^{\text{sat}}$, can be obtained from the Antoine equation,
          \begin{displaymath}
            \ln{P^{\text{sat}}} = A - \frc{B}{T+C}
          \end{displaymath}
          where $\left[P^{\text{sat}}\right]$ = kPa and $[T]$ = $^{\circ}$C, with coefficients given in Table~\ref{Practical1:Table2}.
           

\begin{table}[h]
\begin{center}
\begin{tabular}{||c | c c c ||} 
\hline\hline
                           & {\bf A}    &  {\bf B}    & {\bf C}    \\
\hline
{\bf n-hexane}             & 13.8193    & 2696.04     & 224.317    \\  
{\bf n-heptane}            & 13.8622    & 2910.26     & 216.432    \\  
{\bf i-octane}             & 13.6703    & 2896.31     & 220.767    \\ 
{\bf chlorobenze}           & 13.8635    & 3174.78     & 211.700    \\  
\hline\hline
\end{tabular}
\caption{Constants for the Antoine equation for vapour pressure.}
\label{Practical1:Table2}
\end{center}
\end{table}

\end{question}

\clearpage


%%%
%%% Question 02
%%%
\begin{question}
  
%
The methanol steam reforming reaction for hydrogen generation is given by the following chemical reaction,
           \begin{displaymath}
               H_{2}O\text{ (g)} + CH_{3}OH\text{ (g)}  \Leftrightarrow CO_{2}\text{ (g)} + 3H_{2}\text{ (g)},
           \end{displaymath}
with the thermodynamic data at 25$^{\circ}$C,
          \begin{center}
             \begin{tabular}{ c | c c c c }
              \hline
                            &  $H_{2}O\text{ (g)}$ & $CH_{3}OH\text{ (g)}$ & $CO_{2}\text{ (g)}$  & $H_{2}\text{ (g)}$ \\
              \hline
                 $\Delta G^{\circ}_{f,298}$ $\left(\text{kJ mol}^{-1}\right)$ & -228.57 & -161.96 & -394.36 & 0.0 \\
                 $\Delta H^{\circ}_{f,298}$ $\left(\text{kJ mol}^{-1}\right)$ & -241.82 & -200.66 & -393.51 & 0.0 \\
             \end{tabular}
          \end{center}
          where $G^{\circ}_{f,298}$ and $H^{\circ}_{f,298}$ are the standard molar free Gibbs energy and enthalpy of formation, respectively. Determine:

\begin{enumerate}[(a)]
%
   \item The equilibrium constant, $K_{\text{eq}}$, at 25$^{\circ}$C.~\marks{7}
%======================
         \solution{ The equilibrium constant at 25$^{\circ}$C is given by
          \begin{displaymath}
              K_{\text{eq},298} = \exp{\left[-\frc{\Delta G^{\circ}_{\text{mix},298}}{R T}\right]}.
          \end{displaymath}
          The first step is to calculate the standard free Gibbs energy change of the mixture, $\Delta G^{\circ}_{\text{mix},298}$,\solmarks{3/7}
          \begin{eqnarray}
             \Delta G^{\circ}_{\text{mix},298} &=& \left(\Delta G^{\circ}_{f, 298}\right)_{CO_{2}} + 3\left(\Delta G^{\circ}_{f, 298}\right)_{H_{2}} - \left(\Delta G^{\circ}_{f, 298}\right)_{H_{2}O} - \left(\Delta G^{\circ}_{f, 298}\right)_{CH_{3}OH} \nonumber \\
                                                         &=& -3.83 \text{kJ.mol}^{-1} \nonumber
          \end{eqnarray}
          The equilibrium constant can then be calculated,~\solmarks{4/7}
          \begin{displaymath} 
              K_{\text{eq},298} = \exp{\left[-\frc{\Delta G^{\circ}_{\text{mix},298}}{R T} \right]} = \exp{\left[-\frc{3830}{8.314 \times 298.15 } \right]}=4.6884
          \end{displaymath}
}
      
%
   \item The equilibrium constant, $K_{\text{eq}}$, at 60$^{\circ}$C.~\marks{13}
%======================
         \solution{ The equilibrium constant of the chemical reaction can be expressed as a function of the temperature through the Van't Hoff relation: 
          \begin{displaymath}
               \frc{\d}{\d T} \ln{K_{\text{eq}}} = \frc{\Delta H^{\circ}_{\text{mix},298}}{RT^{2}},
          \end{displaymath}
          As data is available at 25$^{\circ}$C (= 298.15 K) and we need it at 60$^{\circ}$C, we can integrate the Van't Hoff equation assuming constant $\Delta H^{\circ}_{298}$,~\solmarks{5/13}
          \begin{eqnarray}
             \Delta H^{\circ}_{\text{mix},298} = \Delta H^{\circ}_{f, 298} &=& \sum \nu_{i}\left(\Delta H^{\circ}_{f, 298}\right)_{i} \nonumber \\
                                                         &=& \left(\Delta H^{\circ}_{f, 298}\right)_{CO_{2}} + 3\left(\Delta H^{\circ}_{f, 298}\right)_{H_{2}} - \left(\Delta H^{\circ}_{f, 298}\right)_{H_{2}O} - \left(\Delta H^{\circ}_{f, 298}\right)_{CH_{3}OH} \nonumber \\
                                                         &=& 48.97 \text{kJ.mol}^{-1} \nonumber
          \end{eqnarray}
          The equilibrium constant at 25$^{\circ}$C is given by
          \begin{displaymath} 
              K_{\text{eq},298} = \exp{\left[-\frc{\Delta G^{\circ}_{\text{mix},298}}{R T} \right]},
          \end{displaymath}
          therefore, we first need to obtain $\Delta G^{\circ}_{\text{mix},298}$, %solmarks{2/13}
          \begin{eqnarray}
             \Delta G^{\circ}_{\text{mix},298} &=& \left(\Delta G^{\circ}_{f, 298}\right)_{CO_{2}} + 3\left(\Delta G^{\circ}_{f, 298}\right)_{H_{2}} - \left(\Delta G^{\circ}_{f, 298}\right)_{H_{2}O} - \left(\Delta G^{\circ}_{f, 298}\right)_{CH_{3}OH} \nonumber \\
                                                         &=& -3.83 \text{kJ.mol}^{-1} \nonumber
          \end{eqnarray}
          Thus,%\solmarks{2/13}
          \begin{displaymath} 
              K_{\text{eq},298} = \exp{\left[-\frc{\Delta G^{\circ}_{\text{mix},298}}{R T} \right]} = \exp{\left[-\frc{3830}{8.314 \times 298.15 } \right]}=4.6884
          \end{displaymath}
           Now, we can proceed with the integration of the Van't Hoff equation,~\solmarks{8/13}
          \begin{displaymath} 
              \frc{\d}{\d T} \left(\ln{K_{\text{eq}}}\right) = \frc{\Delta H^{\circ}_{\text{mix},298}}{RT^{2}} \Longrightarrow \int\limits_{K_{\text{eq}}^{298.15\text{K}}}^{K_{\text{eq}}^{333.15\text{K}}} \d\left(\ln{K_{\text{eq}}}\right) = \frc{\Delta H^{\circ}_{\text{mix},298}}{R}\int\limits_{298.15\text{K}}^{333.15\text{K}}\frc{1}{T^{2}}\d T
          \end{displaymath}
          \begin{eqnarray}
              \ln{\frc{K_{\text{eq}}^{333.15\text{K}}}{K_{\text{eq}}^{298.15\text{K}}}} &=& -\frc{\Delta H^{\circ}_{\text{mix},298}}{R} \left.\frc{1}{T}\right|_{298.15\text{K}}^{333.15\text{K}} \nonumber \\
             K_{\text{eq}}^{333.15\text{K}} &=&  4.6884 \exp{ \left[-\frc{48970}{8.314}\left(\frc{1}{333.15} - \frc{1}{298.15}\right)\right] } = 37.3580 \nonumber 
           \end{eqnarray}

}
%
\end{enumerate}
        For this problem the equilibrium constant at 25$^{\circ}$C is given by
          \begin{displaymath}
              K_{\text{eq},298} = \exp{\left[-\frc{\Delta G^{\circ}_{\text{mix},298}}{R T}\right]}
          \end{displaymath}
          where $\Delta G^{\circ}_{\text{mix},298}$ is the standard free Gibbs energy change of the mixture. Also the Van't Hoff equation is
          \begin{displaymath}
               \frc{\d}{\d T} \ln{K_{\text{eq}}} = \frc{\Delta H^{\circ}_{\text{mix},298}}{RT^{2}},
          \end{displaymath}
          where $\Delta H^{\circ}_{\text{mix},298}$ is the standard enthalpy change of the mixture and $R\left[=\text{8.314 J}\left(\text{mol K}\right)^{-1}\right]$ is the molar gas constant.
%
\end{question}

\clearpage



%%%
%%% Question 03
%%%
\begin{question}

%
   \begin{enumerate}[(a)]
%
     \item Two litres of an anti-freezing solution is needed for a cooling process. The solution is prepared by mixing 30$\%$-mol of methanol in water. What are the volumes of pure methanol and water at 25$^{\circ}$C necessary to prepare solution? Partial molar volumes $\left(\overline{V}\right)$ for methanol and water in a 30$\%$-mol of methanol solution and their pure species molar volumes $\left(V\right)$, both at 25$^{\circ}$C are:~\marks{8}
       \begin{center}
          \begin{tabular}{ l l l }
                             & $\overline{V}_{i} \left(\text{cm}^{3}.\text{mol}^{-1}\right)$ & $V_{i} \left(\text{cm}^{3}.\text{mol}^{-1}\right)$ \\
               Methanol (1)  & 38.6320                                                     & 40.7270  \\
               Water (2)     & 17.7650                                                     & 18.0680 
          \end{tabular}
       \end{center}
%======================
         \solution{ The molar volume of the 30$\%$-mol of methanol solution is given by,~\solmarks{2/8}
                  \begin{eqnarray}
                     V &=& \frc{V^{T}}{n_{T}} = \frc{\sum\limits_{i=1}^{2}n_{i}\overline{V}_{i}}{n_{T}} = \sum\limits_{i=1}^{2}x_{i}\overline{V}_{i} = x_{1}\overline{V}_{1} + x_{2}\overline{V}_{2} \nonumber \\
                       &=& (0.3)(38.6320) + (0.7)(17.7650) = 24.0251 \text{ cm}^{3}.\text{mol}^{-1} \nonumber
                  \end{eqnarray}
             The total number of moles are:~\solmarks{2/8}
                  \begin{displaymath}
                      n_{T} = \frc{V^{T}}{V} = \frc{2000}{24.0251} = 83.2463 \text{ mol}
                  \end{displaymath}
             The volume of pure methanol and water for the solution are:~\solmarks{4/8}
                  \begin{eqnarray}
                       && V^{\text{pure}}_{1}= x_{1}n_{T}V_{1} = 1017.1116\text{ cm}^{3}\nonumber \\
                       && V^{\text{pure}}_{2}= x_{2}n_{T}V_{2} = 1052.8759\text{ cm}^{3}\nonumber 
                  \end{eqnarray}
}
%
      \item In generating expressions from $G^{E}/RT$ from VLE data, a convenient approach is to plot values of $G^{E}/\left(x_{1}x_{2}RT\right)$ {\it vs} $x_{1}$ and fitting results with an appropriate function. Consider if such data were fit by the expression,
   \begin{displaymath}
       \frc{G^{E}}{x_{1}x_{2} R T} = A + B x_{1}^{2}.
   \end{displaymath}
         From the expression $G^{E}/\left(x_{1}x_{2}RT\right)$, provide equations for the activity coefficient, $\ln{\gamma_{i}}$, as a function of $A$, $B$, $x_{1}$ and $x_{2}$, given~\marks{12}
       \begin{displaymath}
               \ln{\gamma_{i}} = \frc{\overline{G}_{i}^{E}}{RT}.
       \end{displaymath} 
%======================
         \solution{ For a binary mixture, the partial molar property, $\overline{M}$ can be defined as,~\solmarks{2/12}
             \begin{displaymath}
                  \overline{M}_{1} = M + x_{2}\frc{\d M}{\d x_{1}} \hspace{2cm} \text{ and } \hspace{2cm} \overline{M}_{2} = M - x_{1}\frc{\d M}{\d x_{1}}
             \end{displaymath}
             thus with $\overline{M}_{i}=\frc{\overline{G}^{E}_{i}}{RT}$,~\solmarks{3/12}
             \begin{displaymath}
                 \ln{\gamma_{1}} = \frc{\overline{G}^{E}_{1}}{RT} = \frc{G^{E}}{RT} + x_{2}\frc{\d \left(G^{E}/RT\right)}{\d x_{1}} \;\text{ and }\; \ln{\gamma_{2}} = \frc{\overline{G}^{E}_{2}}{RT} = \frc{G^{E}}{RT} - x_{1}\frc{\d \left(G^{E}/RT\right)}{\d x_{1}}.
             \end{displaymath}
             Replacing $x_{2}=1-x_{1}$ in $\frc{G^{E}}{R T} = x_{1}x_{2} \left(A + B x_{1}^{2}\right)$ and differentiating $G^{E}/(RT)$ {\it w.r.t.} $x_{1}$,~\solmarks{5/12}
             \begin{displaymath}
                 \frc{\d \left(G^{E}/RT\right)}{\d x_{1}} = \left(1-2x_{1}\right)A + \left(3-4x_{1}\right)Bx_{1}^{2}
             \end{displaymath}
             The activity coefficient $\left(\text{as a function of } A, B, x_{1} \text{ and } x_{2}\right)$ is then given by,~\solmarks{2/12}
             \begin{eqnarray}
                  && \ln{\gamma_{1}} = x_{1}x_{2}\left(A+Bx_{1}^{2}\right) + x_{2}\left[\left(1-2x_{1}\right)A + \left(3-4x_{1}\right)Bx_{1}\right] \nonumber \\
                  && \ln{\gamma_{2}} = x_{1}x_{2}\left(A+Bx_{1}^{2}\right) - x_{1}\left[\left(1-2x_{1}\right)A + \left(3-4x_{1}\right)Bx_{1}\right] \nonumber 
             \end{eqnarray}
             Alternatively, one could also further develop/simplify the equations above considering $x_{2}=1-x_{1}$.
                
}
%
   \end{enumerate}
%
\end{question}

\clearpage


%%%
%%% Question 04
%%%
\begin{question}
  \begin{enumerate}[a)]
    
%
\item Derive the Maxwell relations below from the fundamental thermodynamic equations.~\marks{11}
\begin{eqnarray}
 \left(\frac{\partial T}{\partial V}\right)_{S} = -\left(\frc{\partial P}{\partial S}\right)_{V}; && 
 \left(\frc{\partial T}{\partial P}\right)_{S} = \left(\frac{\partial V}{\partial S}\right)_{P}; \nonumber \\
 \left(\frc{\partial P}{\partial T}\right)_{V} = \left(\frac{\partial S}{\partial V}\right)_{T}; &&%
  \left(\frac{\partial V}{\partial T}\right)_{P} = -\left(\frc{\partial S}{\partial P}\right)_{T} \nonumber 
\end{eqnarray}
%==========================
%
\solution{First, let's assume a functional $f=f\left(a,b\right)$ and rewrite it as a function of the variables $a$ and $b$,~\solmarks{1/11}
\begin{displaymath}
df = \left(\frc{\partial f}{\partial a}\right)_{b}da + \left(\frc{\partial f}{\partial b}\right)_{a}db
\end{displaymath}
If we define $M=\left(\frc{\partial f}{\partial a}\right)_{b}$ and $N=\left(\frc{\partial f}{\partial b}\right)_{a}$, the equation above becomes~\solmarks{1/11}
\begin{equation}
{\bf df = M da + N db}\label{eqn1}
\end{equation}
Now, if we differentiate $M$ and $N$ with respect to $b$ and $a$, respectively,~\solmarks{1/11}
\begin{displaymath}
\left(\frc{\partial M}{\partial b}\right)_{a} = \frc{\partial^{2} f}{\partial a\partial b}\;\;\text{ and }\;\;\left(\frc{\partial N}{\partial a}\right)_{b} = \frc{\partial^{2} f}{\partial b\partial a}
\end{displaymath}
If the functional $f$ is continuous and differentiable over all domain,~\solmarks{1/11}
\begin{equation}\label{eqn2}
\frc{\partial^{2} f}{\partial a\partial b} = \frc{\partial^{2} f}{\partial b\partial a} \Longrightarrow {\bf \left(\frc{\partial M}{\partial b}\right)_{a} = \left(\frc{\partial N}{\partial a}\right)_{b} }
\end{equation}
The fundamental thermodynamic relations,~\solmarks{1/11} 
\begin{eqnarray}
&& dU = - PdV + TdS \nonumber \\ 
&& dH =   TdS + VdP \nonumber \\
&& dA = - PdV - SdT \nonumber \\
&& dG = - VdP - SdT \nonumber
\end{eqnarray}
have similar shape as Eqn.~\ref{eqn1}, where, for example, in the first relation: ~\solmarks{1/11} 
   \begin{displaymath}
     U = f,\; M=-P,\; N=T,\; \d V=\d a\;\text{ and }\; \d S=\d b. 
   \end{displaymath}
Using relation~\ref{eqn2},~\solmarks{2/11}
   \begin{displaymath}
         -\left(\frc{\partial P}{\partial S}\right)_{V}=\left(\frc{\partial T}{\partial V}\right)_{S}.
   \end{displaymath}
Applying the same to the remaining relations we obtain:~\solmarks{3/11}
\begin{displaymath}
 \left(\frc{\partial T}{\partial P}\right)_{S} = \left(\frac{\partial V}{\partial S}\right)_{P},\; \left(\frc{\partial P}{\partial T}\right)_{V} = \left(\frac{\partial S}{\partial V}\right)_{T}, \text{ and }\; \left(\frac{\partial V}{\partial T}\right)_{P} = -\left(\frc{\partial S}{\partial P}\right)_{T} 
\end{displaymath}
}
%
\item Using the Maxwell relations above, evaluate $\left(\frc{\partial S}{\partial V}\right)_{T}$ for water vapour at 240$^{\circ}$C and molar volume of 0.0258 m$^{3}$ mol$^{-1}$ through the Redlich-Kwong equation of state,
\begin{displaymath}
P = \frc{RT}{V-b} - \frc{a}{V\left(V+b\right)T^{1/2}}
\end{displaymath}
with 
$R$ = 8.314$\times$ 10$^{-5}\;\text{bar m}^{3} \left(\text{mol K}\right)^{-1}$, $a$ = 142.59 $\times$ 10$^{-6}\;\text{bar m}^{6} \left(\text{mol K}\right)^{-2}$ and $b$ = 0.0211 $\times$ 10$^{-3}\text{m}^{3}\;\text{mol}^{-1}$.~\marks{9}


%\frc{\text{bar.m}^{3}}{\text{mol.K}}$, $a$ = 142.59 $\times$ 10$^{-6}\; \frc{\text{bar.m}^{6}}{\left(\text{mol.K}\right)^{2}}$ and $b$ = 0.0211 $\times$ 10$^{-3}\frc{\text{m}^{3}}{\text{mol}}$.~\marks{9}
%==========================
\solution{The Maxwell relation ~\solmarks{2/9}
   \begin{displaymath}
      \left(\frc{\partial P}{\partial T}\right)_{V} = \left(\frc{\partial S}{\partial V}\right)_{T}
   \end{displaymath}
    allows to determine $\left(\frc{\partial S}{\partial V}\right)_{T}$ from the PVT relationship in the RK EOS. Thus,~\solmarks{4/9}
\begin{displaymath}
{\bf \left(\frc{\partial P}{\partial T}\right)_{V} = \frc{R}{V-b} + \frc{a}{2 V\left( V + b \right)T^{\frac{3}{2}}}}
\end{displaymath}
Now substituting the variables by their values~\solmarks{3/9}
\begin{displaymath}
  \left(\frc{\partial P}{\partial T}\right)_{V} = \left(\frc{\partial S}{\partial V}\right)_{T} = 3.2342\times 10^{-3} \text{bar.K}^{-1} = 3.2342\times 10^{-1}\frc{\text{kJ}}{\text{m}^{3}.\text{K}}
\end{displaymath}

}
  \end{enumerate}
\end{question}

\clearpage

%%%
%%% QUESTION 05
%%%
\begin{question}
   Scientists in China are investigating the feasibility of novel enhanced geothermal system (EGS) technology to produce power and heat in the Gonghe basin in the northwestern province of Qinghai. Their initial well test demonstrated that at a depth of 3705 meters, the geological formation temperature is of 235$^{\circ}$C.  They predicted that 5.6805 kg.s$^{-1}$ of brine $\left(m_{\text{br}}\right)$ can be extracted from the reservoir and, at the top of the well is at 15 bar and 200$^{\circ}$C. In their preliminary plant design (Fig.~\ref{Figure:Fig1}), 2 turbines would produce 3.5 MW of power, whilst a condenser (fed with cold water, A, at 20$^{\circ}$C) would extract 9.75 MW of heat from the brine before re-injection into the geothermal reservoir. 
      \begin{figure}[h]
        \begin{center}
          \includegraphics[width=.8\linewidth,clip]{./Pics/Exam2_PowerPorousMedia.pdf}
          \caption{Schematics of geothermal source exploration for power and heat production.}\label{Figure:Fig1}
        \end{center}
      \end{figure}
      
      \begin{table}[h]
        \begin{center}
          \begin{tabular}{||c|c|c|c|c|c||}
            \hline\hline
                {\bf Stream} & {\bf Pressure} & {\bf Temperature} & {\bf Fluid} & {\bf Specific enthalpy} & {\bf Specific entropy} \\
                             & {\bf (bar)}    & {\bf $\left(^{\circ}\text{C}\right)$}& &  {\bf $\left(\text{kJ.kg}^{-1}\right)$} & {\bf $\left(\text{kJ.kg}^{-1}\text{.K}^{-1}\right)$} \\  
            \hline\hline
                {\bf 1}      &   15           &   200              & (i)        &  (ii)                   &   (iii)                  \\
            \hline
                {\bf 2}      &   7            &   180              & dry vapour &   (iv)                  &   (v)                     \\ 
            \hline
                {\bf 3}      &   5            &   --               & (vi)       &   (vii)                 &    (viii)                 \\
            \hline
                {\bf 4}      &   --           &   (ix)             & --        &    (x)                   &    --                     \\
            \hline
                {\bf 5}      &   4.50         &   147.90           & --        &    (xi)                  &    --                     \\
            \hline
                {\bf 6}      &   25.00        &   --               & (xii)      &    (xiii)               &    --                     \\
            \hline
                {\bf A}      &   --           &   20               & Cold water  &   --                   &    --                     \\
            \hline
                {\bf B}      &   --           &   60               & Hot water  &   --                   &    --                     \\
            \hline
                {\bf C}      &   --           &   --               & Liquid residue&   --                   &    --                     \\
            \hline
                {\bf D}      &   --           &   --               & Cold water  &   --                   &    --                     \\
            \hline\hline
          \end{tabular}
           \caption{Thermal-physical​ conditions​ ​of​ all​ streams​ ​from​ ​Fig.​~\ref{Figure:Fig1}}\label{Table:Tab1}
        \end{center}
      \end{table}

       Before re-injection, contaminants are removed from the brine stream at a rate of 0.1 kg.s​$^{-1}$ (stream C) and, in order to replenish the geothermal reservoir, 10 kg.s​$^{-1}$ is added into the brine stream (D). Your task is to check if the predicted power and heat production are correct, therefore: 
  \begin{enumerate}[a)]
     \item Calculate conditions (i-xiii) from Table~\ref{Table:Tab1};~\marks{13}
         \solution{Following the streams:
            \begin{enumerate}[1)]
               \item At $P_{1}=15$ bar and $T_{1}=200^{\circ}$C, brine is at {\it superheated vapour} state {\bf (i)}\solmarks{1/13} with $h_{1}=2796.8\text{ kJ.kg}^{-1}$ {\bf (ii)}\solmarks{1/13} and $s_{1}=6.4546\text{ kJ.kg}^{-1}$ {\bf (iii)}\solmarks{1/13};
               \item At $P_{2}=7$ bar, brine is a dry vapour with $h_{2}=2763.5\text{ kJ.kg}^{-1}$ {\bf (iv)}\solmarks{1/13} and $s_{2}=6.7080\text{ kJ.kg}^{-1}$ {\bf (v)}\solmarks{1/13};
               \item At $P_{3}=5$ bar, an isentropic expansion of the fluid occurs in Turbine 2, \ie $s_{3}=s_{2}=6.7080\text{ kJ.kg}^{-1}$ {\bf (viii)}\solmarks{1/13}. At such pressure, the saturation table gives:
                 \begin{displaymath}
                   \begin{cases}
                      h_{f} = 640.23\text{ kJ.kg}^{-1}, & s_{f} = 1.8607\text{ kJ.(kg.K)}^{-1}, \\
                      h_{g} = 2748.7\text{ kJ.kg}^{-1}, & s_{g} = 6.8212\text{ kJ.(kg.K)}^{-1}.
                   \end{cases}
                 \end{displaymath}
                 As $s_{3}<s_{g}$, the fluid is {\it wet vapour} {\bf (vi)}~\solmarks{1/13}. In order to calculate the enthalpy of this stream, we first need​​ to obtain the quality​ of the vapour through,\solmarks{1/13}
                 \begin{displaymath}
                   \begin{cases}
                     x_{3} = \frc{s_{3}-s_{f}}{s_{g}-s_{f}} = 0.9772, & \\
                     x_{3} = \frc{h_{3}-h_{f}}{h_{g}-h_{f}} & \Longrightarrow h_{3} = 2700.626 \text{ kJ.kg}^{-1}\textbf{ (vii)}
                   \end{cases}
                 \end{displaymath}
               \item The condenser removes heat from the brine stream with no pressure drop, therefore $P_{4}=P_{3}=5$ bar with $T_{4}=T_{\text{sat}}=151.9^{\circ}$C {\bf (ix)}~\solmarks{1/13}
                 

            \end{enumerate}
         }
%%%%%%%%%%%%%%%%%%%%%%       
     \item Are the predictions of power and heat correct? If not, calculate the power produced by the Turbines and heat extracted at the Condenser;~\marks{3}
     \item Sketch the temperature $\times$ specific entropy $\left(Ts\right)$ of the process involving streams (1)-(4), indicating:~\marks{4}
       \begin{itemize}
          \item entropies and temperatures;
          \item liquid saturated line;
          \item vapour saturated line;
          \item critical pressure;
          \item isobars.
       \end{itemize}  
  \end{enumerate}
  In order to solve this problem, assume that brine has the same thermodynamic properties as water with heat capacity of 4.18 kJ.(kg.K)$^{-1}$. Also, assume that an ideal isentropic expansion occurs at Turbine 2.
\end{question}

\clearpage

\vfill
\paperend
 


\vfill 



%\begin{comment}
{
  \includepdf[pages=-,fitpaper]{./Pics/EquationsList}
}
%\end{comment}



\end{document}
