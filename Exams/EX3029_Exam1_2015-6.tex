
\documentclass[calculator,steamtables,datasheet,solutions]{exam}
%\documentclass[calculator,fluidstables,allquestions,datasheet]{exam}

% The full list of class options are
% calculator : Allows approved calculator use.
% datasheet : Adds a note that data sheet are attached to the exam.
% handbook : Allows the use of the engineering handbook.
% resit : Adds the resit markings to the paper.
% sample : Adds conspicuous SAMPLE markings to the paper
% solutions : Uses the contents of \solution commands (and \solmarks) to generate a solution file

\usepackage{pdfpages} 
\usepackage{lscape,comment}
 
\coursecode{EX3029}%%

\examtime{09.00--12.00}%
\examdate{03}{03}{2015}% 
\examformat{Candidates must attempt \textit{all} questions.}

\newcommand{\frc}{\displaystyle\frac}
\newcommand{\br}[1]{\!\left( #1 \right)}
\newcommand{\abs}[1]{\left| #1 \right|}
\newcommand{\fracd}[2]{\frac{\mathrm{d} #1}{\mathrm{d} #2}}
\newcommand{\fracp}[2]{\frac{\partial #1}{\partial #2}}
\renewcommand{\d}[1]{\mathrm{d} #1 } 
\newcommand{\Ma}{\mathrm{M\!a}} 



\begin{document}

%%%
%%% Question 01
%%%
\begin{question}
%
\begin{enumerate}[(a)]
%%% Johannes T3Q4
\item Assuming $S = S\left(P,V\right)$ and taking into consideration that,
\begin{displaymath}
\left(\frc{\partial S}{\partial T}\right)_{V} = \frc{C_{V}}{T}\;\;\;\text{ and }\;\;\; \left(\frc{\partial S}{\partial T}\right)_{P} = \frc{C_{P}}{T}
\end{displaymath}
Prove that 
\begin{displaymath}
\d S = \frc{C_{V}}{T}\left(\frc{\partial T}{\partial P}\right)_{V}\d P + \frc{C_{P}}{T}\left(\frc{\partial T}{\partial V}\right)_{P}\d V
\end{displaymath}~\marks{8}
%
\solution{As entropy is expressed as a function of pressure and molar volume, we can write it in differenctial form as,~\solmarks{2/8}
\begin{displaymath}
  dS = \left(\frc{\partial S}{\partial P}\right)_{V}dP + \left(\frc{\partial S}{\partial V}\right)_{p}dV 
\end{displaymath}
We can rewrite this equation as~\solmarks{3/8}
\begin{displaymath}
dS = \left(\frc{\partial S}{\partial T}\right)_{V}\left(\frc{\partial T}{\partial P}\right)_{V}dP + \left(\frc{\partial S}{\partial T}\right)_{p}\left(\frc{\partial T}{\partial V}\right)_{p}dV
\end{displaymath}
As $\left(\frc{\partial S}{\partial T}\right)_{V}=\frc{C_{v}}{T}$ and $\left(\frc{\partial S}{\partial T}\right)_{P}=\frc{C_{P}}{T}$, ~\solmarks{3/8}
\begin{displaymath}
dS = \frc{C_{v}}{T}\left(\frc{\partial T}{\partial P}\right)_{V}dP + \frc{C_{P}}{T}\left(\frc{\partial T}{\partial V}\right)_{p}dV
\end{displaymath}
}

%%%
%%% Johannes Lecture Example
%%%
\item\label{LectExample} In the esterification reaction of acetic acid with ethanol at 100$^{\circ}$C,
\begin{displaymath}
CH_{3}COOH + C_{2}H_{5}OH  \Longleftrightarrow  CH_{3}COOC_{2}H_{5} + H_{2}O,
\end{displaymath}
calculate the mass fraction of ethyl acetate given that initially there was 1 mole of acetic acid and ethanol. The reaction enthalpy and Gibbs energy at standard state (25$^{\circ}$C and 1 atm) are $\Delta H_{298}^{\circ}=-3640$ J and $\Delta G_{298}^{\circ}=-4650$ J. Given the van't Hoff equation,~\marks{12}
\begin{displaymath}
\frc{d}{dT}\left(\ln{K}\right) = - \frc{\Delta H_{298}^{o}}{RT^{2}}
\end{displaymath}

%==========================
\solution{ Initially we have 1 mol of acetic acid (HAc) and 1 mol of ethanol (EtOH). We can calculate the mole fractions for all species as a function of the reaction coordinate, $\epsilon$~\solmarks{4/12}
\begin{eqnarray}
 x_{\text{EtAc}} &=& \frc{\epsilon}{2+0.\epsilon} = \frc{\epsilon}{2} = x_{H_{2}O}  \nonumber \\
 x_{\text{HAc}} &=& \frc{1-\epsilon}{2} = x_{\text{EtOH}} \nonumber
\end{eqnarray}
Assuming ideal solution,~\solmarks{2/12}
\begin{displaymath}
  \prod\limits_{i=1}^{c=4} x_{i}^{\nu} = K = x_{\text{HAc}}^{-1}\;x_{\text{EtOH}}^{-1}\;x_{\text{EtAc}}\;x_{H_{2}O} \Longrightarrow K = \frc{\epsilon^{2}}{\left(1-\epsilon\right)^{2}}
\end{displaymath} 
Thus, by calculating $K$, we can obtain $\epsilon$ and then $x_{\text{EtAc}}$. $K$ (temperature-dependent) can be obtained from the Gibbs free energy,~\solmarks{2/12}
\begin{displaymath}
   \ln{K_{298}} = -\frc{\Delta G_{298}^{o}}{RT} = \frc{4650.0}{8.314\times 298.15} = 1.8759
\end{displaymath}
Now, in order to calculate $K$ at 373.15 K, we can integrate the van't Hoff equation from 298.15 K to 373.15 K~\solmarks{2/12}
\begin{displaymath}
\int\limits_{K_{298}}^{K_{373}} d\left(\ln{K}\right) = - \int\limits_{298.15}^{373.15}\frc{\Delta H_{298}^{o}}{RT^{2}} dT \Longrightarrow K_{373} = 4.8586
\end{displaymath}
Now we can calculate the reaction coordinate,
\begin{displaymath}
   K_{373} = \frc{\epsilon^{2}}{\left(1-\epsilon\right)^{2}} \Longrightarrow \epsilon= 0.6879
\end{displaymath} 
Thus ~\solmarks{2/12}
\begin{displaymath}
x_{\text{EtAc}} = \frc{\epsilon}{2} = 0.3440
\end{displaymath}
%==========================
 }
%
\end{enumerate}
%
\end{question}
\clearpage


%%%
%%% Question 02
%%%
\begin{question}
\begin{enumerate}[(a)]
% LectureNotes_Nguyen (pg 89)
\item Show that the van der Waals equation of state (vdW EOS),
\begin{displaymath}
  P = \frc{RT}{V-b} - \frc{a}{V^{2}}
\end{displaymath} 
can be expressed as a cubic polynomial equation in $Z$ (compressibility coefficient),
\begin{displaymath}
Z^{3} -(1+B)Z^{2} +AZ -AB = 0 
\end{displaymath}
with $B=bP/(RT)$ and $A=aP/(RT)^{2}$.~\marks{7}

%==========================
\solution{We can rearrange the vdW EOS,~\solmarks{3/7}
\begin{eqnarray}
 && P = \frc{RT}{V-b} - \frc{a}{V^{2}} \Longrightarrow \frc{PV}{RT} = \frc{V}{V-b} - \frc{a}{RTV} \nonumber \\
 && Z = \frc{1}{1-\frc{b}{V}} - \frc{a}{RTV}\;\;\text{ since } V=\frc{ZRT}{P}  \nonumber \\
&& Z = \left(1-\frc{bP}{ZRT}\right)^{-1} - \frc{aP}{Z\left(RT\right)^{2}} \nonumber
\end{eqnarray}
Defining $B=bP/(RT)$ and $A=aP/(RT)^{2}$~\solmarks{2/7} and replacing in the above expression leads to,~\solmarks{2/7}
\begin{displaymath}
Z^{3} -(1+B)Z^{2} +AZ -AB = 0 
\end{displaymath}
}
%==========================

\item Calculate the fugacity of CO$_{2}$ at 310 K and 1.4$\times$10$^{6}$ Pa using the van der Waals equation of state (EOS), with $a=$ 0.3658 Pa.m$^{6}$/mol$^{2}$, $b=$ 4.286$\times$10$^{-5}$ m$^{3}$/mol. Given,~\marks{13}
\begin{displaymath}
\ln{\left(\frc{f}{P}\right)} = -\ln{\left(1-\frc{b}{V}\right)} - \frc{a}{RTV} - \ln{Z} +\left(Z-1\right).
\end{displaymath}

%====================
\solution{Solving the cubic polynomial in $Z$, with $B=bP/(RT)$ and $A=aP/(RT)^{2}$,~\solmarks{5/13}
\begin{eqnarray}
Z^{3} -(1+B)Z^{2} +AZ -AB = 0 & \Longrightarrow& A = 7.7095\times 10^{-2}\; ;\; B = 2.3281\times 10^{-2} \nonumber \\ 
 &\Longrightarrow& Z = 0.9436 \nonumber
\end{eqnarray}
Now for the fugacity equation:~\solmarks{8/13}
\begin{eqnarray}
&& \ln{\left(\frc{f}{P}\right)} = -\ln{\left(1-\frc{b}{V}\right)} - \frc{a}{RTV} - \ln{Z} +\left(Z-1\right) \nonumber \\
&&  \ln{\left(\frc{f}{P}\right)} = -\ln{\left(1-\frc{B}{Z}\right)} - \frc{A}{Z} - \ln{Z} +\left(Z-1\right) \Longrightarrow f = 1.26\times 10^{6}\;\text{Pa}\nonumber
\end{eqnarray}

}
%====================
%
\end{enumerate}
%
\end{question}

\clearpage

%%%
%%% Question 03
%%%
\begin{question}
%
%%%
%%% Jeff Solved Example 3 ==> Sandler Example 10.1.4 (page 504)
%%%
 An ideal liquid mixture of 25 mol-$\%$ n-pentane $\left(nC_{5}\right)$, 45 mol-$\%$ n-hexane $\left(nC_{6}\right)$ and 30 mol-$\%$ n-heptane $\left(nC_{7}\right)$, initially at 69$^{\circ}$C and high pressure, is partially vaporised by isothermically lowering the pressure to 1.013 bar. Calculate the relative amounts of vapour and liquid in equilibrium and their compositions.~\marks{20}
%======================
\solution{ From the Antoine equation, we can calculate the saturation pressure of the species $P_{\text{C}_{5}}^{sat}=$ 2.721 bar, $P_{\text{C}_{6}}^{sat}=$ 1.024 bar, $P_{\text{C}_{7}}^{sat}=$ 0.389 bar.~\solmarks{3/20} Assuming ideal solution,
\begin{displaymath}
\frc{y_{i}}{x_{i}} = K_{i} = \frc{P_{i}^{sat}}{P}
\end{displaymath}
Thus $K_{\text{C}_{5}}=$ 2.6861, $K_{\text{C}_{6}}=$ 1.0109 and $K_{\text{C}_{7}}=$ 0.3840~\solmarks{3/20}, therefore~\solmarks{2/20}
\begin{eqnarray}
&& y_{\text{C}_{5}} = x_{\text{C}_{5}}K_{\text{C}_{5}},\;\; y_{\text{C}_{6}} = x_{\text{C}_{6}}K_{\text{C}_{6}},\;\; y_{\text{C}_{7}} = x_{\text{C}_{7}}K_{\text{C}_{7}} \nonumber \\
&& \sum\limits_{i=1}^{3}x_{i} = x_{\text{C}_{5}} + x_{\text{C}_{6}} + x_{\text{C}_{7}} = 1  \nonumber \\
&& \sum\limits_{i=1}^{3}y_{i} = y_{\text{C}_{5}} + y_{\text{C}_{6}} + y_{\text{C}_{7}} = 1  = K_{\text{C}_{5}}x_{\text{C}_{5}} + K_{\text{C}_{6}}x_{\text{C}_{6}} + K_{\text{C}_{7}}x_{\text{C}_{7}} \nonumber
\end{eqnarray}
The mass balance is,~\solmarks{1/20}
\begin{eqnarray}
&& L + V = 1 \nonumber \\
&& x_{i}L + y_{i}V = z_{i} \nonumber 
\end{eqnarray}
with $z_{i}=\left(0.25\;\;0.45\;\;0.30\right)^{T}$. Rearranging this set of equations lead to a non-linear expression in $L$,~\solmarks{3/20}
\begin{displaymath}
\frc{0.25}{\left(1-K_{\text{C}_{5}}\right)L+K_{\text{C}_{5}}} + \frc{0.45}{\left(1-K_{\text{C}_{6}}\right)L+K_{\text{C}_{6}}} +  \frc{0.30}{\left(1-K_{\text{C}_{7}}\right)L+K_{\text{C}_{7}}} = 1 
\end{displaymath}
Solving this equation leads to $L=0.5748$ and $V=0.4252$~\solmarks{2/20}. Calculating the molar fractions of the species:~\solmarks{6/20}
\begin{center}
\begin{tabular}{c c c c}
\hline
                 & {\bf n-C$_{5}$} &  {\bf n-C$_{6}$} &  {\bf n-C$_{7}$} \\
\hline
  {\bf x$_{i}$}   & 0.1456         &  0.4479         & 0.4065    \\
  {\bf y$_{i}$}   &  0.3911        &  0.4528         & 0.1561    \\
\hline
\end{tabular} 
\end{center}

}
%======================

For this problem, use 
\begin{displaymath}
   \ln P_{i}^{\text{sat}} = A_{i} - \frc{B_{i}}{RT}
\end{displaymath} 
with [P] = bar and [T] = K, and
    \begin{center}
       \begin{tabular}{l l l} 
          $A_{nC_{5}}=10.422$ & $A_{nC_{6}}=10.456$ & $A_{nC_{7}}=11.431$ \\
          $B_{nC_{5}}=26799$  & $B_{nC_{6}}=29676$  & $B_{nC_{7}}=35200$  
       \end{tabular}
    \end{center}
%
\end{question}

\clearpage

%%%
%%% Question 04
%%%
\begin{question}
%
\begin{enumerate}[(a)]

%%%
%%% Johannes Problem 2 (Tutorial 5)
%%%
\item\label{Tut05P2} A process stream contains light species 1 and heavy species 2. A relatively pure liquid stream containing mostly 2 is obtained through a single-stage liquid/vapour separator. Specifications on the equilibrium composition are: $x_{1}$ = 0.002 and $y_{1}$ = 0.950. Assuming that the modified Raoult's law applies, 
\begin{displaymath}
  y_{i} P = x_{i}\gamma_{i}P_{i}^{\text{sat}}
\end{displaymath} 
Determine $T$ and $P$ for the separator. Given the activity coefficients for the liquid phase,
\begin{displaymath}
\ln\gamma_{1} = 0.93x_{2}^{2} \;\;\;\;\;\text{ and }\;\;\;\;\;\ln\gamma_{2}=0.93x_{1}^{2},
\end{displaymath}
\begin{displaymath}
\ln P^{\text{sat}} = A - \frc{B}{T}\;\;\;\text{with [P] = bar and [T] = K},
\end{displaymath} 
with $A_{1}$ =10.08, $B_{1}$ = 2572.0, $A_{2}$ = 11.63 and $B_{2}=6254.0$.~\marks{13}

%===================
\solution{Given,
\begin{eqnarray}
&& x_{1} = 0.002 \;\;\Longrightarrow \;\; x_{2} = 0.998 \nonumber \\
&& y_{1} = 0.950 \;\;\Longrightarrow \;\; y_{2} = 0.050 \nonumber
\end{eqnarray}
Calculating the activity coefficient,~\solmarks{2/13},
\begin{eqnarray}
&& \ln{\gamma_{1}} = 0.93 x_{2}^{2} \;\;\Longrightarrow \;\; \gamma_{1} = 2.5251 \nonumber \\
&& \ln{\gamma_{2}} = 0.93 x_{1}^{2} \;\;\Longrightarrow \;\; \gamma_{2} = 1.0000 \nonumber
\end{eqnarray}
The modified Raoult's law,~\solmarks{4/13}
\begin{eqnarray}
&& y_{i}P=x_{i}\gamma_{1}P_{i}^{sat} \;\; \Longrightarrow \;\; P=\frc{x_{i}\gamma_{i}P_{i}^{sat}}{y_{i}} \nonumber \\
&& \frc{P_{1}^{sat}}{P_{2}^{sat}} = \frc{x_{2}\gamma_{2}y_{1}}{x_{1}\gamma_{1}y_{2}} = 3754.7028 = \frc{\exp{\left(A_{1}-\frc{B_{1}}{T}\right)}}{\exp{\left(A_{2}-\frc{B_{2}}{T}\right)}}\nonumber
\end{eqnarray}
Solving this equation results in $T=376.45$ K~\solmarks{3/13}. The pressure can now be obtained,~\solmarks{4/13}
\begin{displaymath}
  P=\frc{x_{i}\gamma_{1}P_{1}^{sat}}{y_{1}} = 0.1368\text{ bar}
\end{displaymath}
}
%==================

% Nguyen (page 106, Ex 5.1-1)
\item Determine the temperature and composition of the first bubble created from a saturated liquid mixture of benzene and toluene containing 45 mole percent of benzene at 200 kPa. Benzene and toluene mixtures may be considered as ideal. Given
\begin{displaymath}
\ln{P^{sat}} = A - \frc{B}{T+C}\;\;\text{ with [P] = kPa and  [T] = K}.
\end{displaymath}
And~\marks{7}
\begin{center}
\begin{tabular}{ c  c  c  c }
\hline
           & ${\bf A}$  &  ${\bf B}$  & ${\bf C}$ \\
\hline
Benzene    & 14.1603    &  2948.78    & -44.5633 \\
Toluene    & 14.2514    &  3242.38    & -47.1806 \\
\hline
\end{tabular}
\end{center}

%=================
\solution{From Raoult's law, 
\begin[displaymath}
y_{i} = \frc{x_{i}P_{i}^{sat}}{P}
\end{displaymath}
with benzene (1) and toluene (2),~\solmarks{1/7}
\begin{displaymath}
P = x_{1}P_{1}^{sat} + x_{2}P_{2}^{sat}
\end{displaymath}



}
%=================

\end{enumerate}

\end{question}



\clearpage

%%%
%%% Question 06
%%%
\begin{question}
%
\begin{enumerate}[(a)]

%%% Johannes T0602
%%%
\item\label{T0602} What is the change in entropy when 700 litres of CO$_{2}$ and 300 litres of N$_{2}$, each at 1 bar and 25$^{\circ}$C blend to form a gas mixture at the same conditions? Assume ideal gases, and given~\marks{6}
\begin{displaymath}
\Delta S = - nR\sum\limits_{i=1}^{n}y_{i}\ln{y_{i}}
\end{displaymath}


%%%
%%% SM &VN 11.13
%%%
\item\label{Ex2} The molar volume $\left(\text{in cm}^{3}\text{.gmol}^{-1}\right)$ of a binary liquid mixture at $T$ and $P$ is given by:
\begin{displaymath}
V = 120 x_{1} + 70 x_{2} + \left( 15x_{1} + 8x_{2}\right)x_{1}x_{2} 
\end{displaymath}
\begin{enumerate}
\item\label{first} Find expressions for the partial molar volumes of species 1 and 2 at $T$ and $P$.
\item Show that when these expressions are combined in accord with $M=\sum\limits_{i}x_{i}\overline{M}_{i}$, the given equation for $V$ is recovered;
\item Show that these expressions satisfy the Gibbs-Duhem equation, $\sum\limits_{i}x_{i}d\overline{M}_{i}=0$;
\item Show that 
   \begin{displaymath}
        \left(\frc{d\overline{V}_{1}}{dx_{1}}\right)_{x_{1}=1} = \left(\frc{d\overline{V}_{2}}{dx_{1}}\right)_{x_{1}=0} = 0
   \end{displaymath}
 \item Plot values of $V$, $\overline{V}_{1}$ and $\overline{V}_{2}$ calculated by the given equation for $V$ and by the equations developed in (a) {\it vs} $x_{1}$. Label points $\overline{V}_{1}^{\infty}$ and $\overline{V}_{2}^{\infty}$ and show their values.~\marks{14}
\end{enumerate}
\end{question}


\clearpage

%%%
%%% Question 07
%%%
\begin{question}
%
\begin{enumerate}[(a)]
% Nguyen pg.5-25
\item Calculate the bubble point pressur and vapour composition for a liquid mixture of 41.2 mol$\%$ ethanol (1) and n-hexane (2) at 331K. Given:
\begin{displaymath}
\ln{\gamma_{1}} = \frac{A}{\left[1+\left(Ax_{1} / Bx_{2}\right)\right]^{2}} \;\;\text{ and }\;\; \ln{\gamma_{2}} = \frac{B}{\left[1+\left(Bx_{2} / Ax_{1}\right)\right]^{2}}
\end{displaymath}
with $A=2.409$ and $B=1.970$. Also,
\begin{displaymath}
\ln{P_{1}^{\text{sat}}} = 16.1952 - \frac{3423.53}{T-55.7152} \;\;\text{ and }\;\; \ln{P_{2}^{\text{sat}}} = 14.0568 - \frac{1825.42}{T-42.7089}
\end{displaymath}
with $P_{i}^{\text{sat}}$ in $kPa$ and $T$ in $K$ 


\end{enumerate}
\end{question}

\clearpage

%%%
%%% Question 08
%%%
\begin{question}

%%%
%%% Problem 8.3 (Power Lectures Notes)
%%%
Given saturated ammonia $\left(\text{NH}_{3}\right)$ vapour at $P_{1} = 200 kPa$ compressed by a piston to $P_{2} = 1.6 MPa$ in a reversible adiabatic process,
\begin{enumerate}
\item Find the work done per unit mass; 
\item Sketch the $T-s$ and $P-v$ diagrams.
\end{enumerate}
Given the following critical conditions for ammonia: T$_{c}$ = 132.4$^{\circ}$C and P$_{c}$ = 112.8 bar.


\end{question}




\end{document}
