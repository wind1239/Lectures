
\documentclass[calculator,datasheet,handbook,solutions]{exam}
%\documentclass[calculator,datasheet,handbook]{exam}
% The full list of class options are
% calculator : Allows approved calculator use.
% datasheet : Adds a note that data sheet are attached to the exam.
% handbook : Allows the use of the engineering handbook.
% resit : Adds the resit markings to the paper.
% sample : Adds conspicuous SAMPLE markings to the paper
% solutions : Uses the contents of \solution commands (and \solmarks) to generate a solution file

\usepackage{array}
\usepackage{multirow}
\usepackage{pdfpages}
\usepackage[hidelinks]{hyperref}

\examtime{2~pm -- 5~pm}%
\examdate{13}{12}{2017}%
\examformat{Candidates should attempt \textit{all} questions.}

\newtoggle{3030paper}

%This changes the mode from EX3030 to EM40JN
\toggletrue{3030paper}
%\togglefalse{3030paper}

\iftoggle{3030paper}{
  \coursecode{EX3030}%
  \coursetitle{Heat, Mass, \& Momentum Transfer}%
}{
  \coursecode{EM40JN}%
  \coursetitle{Heat and Momentum Transfer}%
}

\begin{document}
%%%%%%%%%%%%%%%%%%%%%%%%%%%%%%%%%%%%%%%%%%%%%%%%%%%%%%%%%%%%%%%%%%%%%%%%%%%%%%%% 
%%%%%%%%%%%%%%%%%%%%%%%%%%%%%%%%%%%%%%%%%%%%%%%%%%%%%%%%%%%%%%%%%%%%%%%%%%%%%%%%

%%%
%%%  December Diet: Q01
%%%
\begin{question}
  \begin{enumerate}
    \item Consider a large aluminium plate of thickness 0.12 m with initial uniform temperature of 85$^{\circ}$C. Suddenly, the temperature of one of the faces is lowered to 20$^{\circ}$C, while the other face is perfectly insulated. Assuming that the plate can be modelled as a 1D problem, the following thermal energy conservative equation can be used,
    \begin{displaymath}
      \rho C_{p}\frac{\partial T}{\partial t} = \kappa \frac{\partial^{2} T}{\partial x^{2}} + \mathcal{S},
    \end{displaymath}
    where $\rho$, $C_{p}$, $t$ are density, heat capacity and time, respectively. $\kappa$ and $\mathcal{S}$ are thermal conductivity coefficient and source term.  Using the finite difference method (FDM) with spatial $\left(\Delta  x\right)$ and temporal $\left(\Delta t\right)$ increments of 0.03 m and 300 s, respectively, 
  \begin{enumerate}[a)]
     \item Determine the number of nodes necessary to discretise the problem;~\marks{2}
       \solution{For a length $L=0.12$ m and spatial increment $\Delta  x = 30\times 10^{-3}$ m,~\solmarks{2/2}
         \begin{displaymath}
           \Delta x = \frac{L}{M-1} \;\;\Longrightarrow\;\; M = 5\text{ nodes}
          \end{displaymath}

         }
     \item Describe the boundary and initial conditions for this problem;~\marks{5} 
       \solution{ For 5 nodes, $i=0,\cdots,4$
         \begin{enumerate}[(i)]
            \item Boundary conditions:
              \begin{itemize}
                 \item Dirichlet BC at $x = L =$ 0.12 m;~\solmarks{1/5}
                     \begin{displaymath}
                        T\left(x=L,t\right) = T_{4}^{j} = 20^{\circ}\text{C},\;\;\;\left(j=0,1,\cdots\right)
                     \end{displaymath}
                 \item Adiabatic plane at $x=0$ can be treated as a symmetric plane, i.e.,~\solmarks{3/5}
                     \begin{displaymath}
                        T_{i+1}^{j} = T_{i-1}^{j} \;\;\text{ for } i=0\;\;\Longrightarrow\;\; T_{1}^{j}=T_{-1}^{j}
                     \end{displaymath}
              \end{itemize}
            \item Initial condition: the plate has uniform temperature of 85$^{\circ}$C, i.e.,~\solmarks{1/5}
              \begin{displaymath}
                T\left(x<L,t=0s\right) = T_{i}^{0} = 85^{\circ}\text{C},\;\;\;\left(i=0,1,2,3\right)
              \end{displaymath}
         \end{enumerate}
         }
     \item Determine the temperature distribution of the plate at $t = 10$ minutes using the FDM,~\marks{16}
       \solution{ The discretised thermal energy equation,
          \begin{displaymath}
             \rho C_{p}\frac{T_{i}^{j+1}-T_{i}^{j+1}}{\Delta t} = \kappa \frac{T_{i+1}^{j}-2T_{i}^{j}+T_{i-1}^{j}}{\left(\Delta x\right)^{2}} + \mathcal{S}_{i}^{j},
          \end{displaymath}
          can be re-arranged to~\solmarks{1/16}
          \begin{displaymath}
            \frac{\rho C_{p}}{\kappa} \frac{\left(\Delta x\right)^{2}}{\Delta t} \left(T_{i}^{j+1}-T_{i}^{j+1}\right) =  \left(T_{i+1}^{j}-2T_{i}^{j}+T_{i-1}^{j}\right) + \frac{\left(\Delta x\right)^{2}}{\kappa}\mathcal{S}_{i}^{j}.
          \end{displaymath}
          Defining the mesh Fourier number,
          \begin{displaymath}
            \tau = \frac{\alpha \Delta t}{\left(\Delta x\right)^{2}}
          \end{displaymath}
          and the thermal diffusivity, $\left(\alpha=\kappa\rho^{-1}C_{p}^{-1}\right)$,
          \begin{displaymath}
            T_{i}^{j+1} = T_{i}^{j} + \tau\left(T_{i+1}^{j} - 2T_{i}^{j}+T_{i-1}^{j}\right) + \frac{\tau\left(\Delta x\right)^{2}}{\kappa}\mathcal{S}_{i}^{j}.
          \end{displaymath}
          For this problem, the source term is null, $\mathcal{S}_{i}^{j}=0$,~\solmarks{2/16}
          \begin{displaymath}
            T_{i}^{j+1} = T_{i}^{j} + \tau\left(T_{i+1}^{j} - 2T_{i}^{j}+T_{i-1}^{j}\right).
          \end{displaymath}
          The first step to solve FDM problems is to check if the stability criteria is met, i.e., $\tau \le 0.5$~\solmarks{1/16}
          \begin{displaymath}
            \tau = \frac{\alpha \Delta t}{\left(\Delta x\right)^{2}} = 0.5 \;\;\text{ with }\;\;\Delta t = 300\text{ s}.
          \end{displaymath}
          Boundary and initial conditions are:~\solmarks{2/16}
          \begin{displaymath}
            \begin{cases}
               \text{BC1: } T_{4}^{j} = 20, & j = 0,1,\cdots \\
               \text{BC2: } T_{i}^{j} = T_{-1}^{j}, & j = 0,1,\cdots \\
               \text{IC: } T_{i}^{0} = 85, & i = 0,\cdots, 3 
            \end{cases}
          \end{displaymath}
          Now solving the problem for $j=0,1,2$
          \begin{description}
            \item[j=0] $\rightarrow\;\;T_{i}^{j+1}=T_{i}^{1}$ $\left(t=300 s\right)$:\solmarks{5/16}
              \begin{displaymath}
                \begin{cases}
                   i=0, & T_{0}^{1} = T_{0}^{0} + \tau\left(T_{1}^{0}-2T_{0}^{0}+T_{-1}^{0}\right) = 85 \\
                   i=1, & T_{1}^{1} = T_{1}^{0} + \tau\left(T_{2}^{0}-2T_{1}^{0}+T_{0}^{0}\right) = 85 \\
                   i=2, & T_{2}^{1} = T_{2}^{0} + \tau\left(T_{3}^{0}-2T_{2}^{0}+T_{1}^{0}\right) = 85 \\
                   i=3, & T_{3}^{1} = T_{3}^{0} + \tau\left(T_{4}^{0}-2T_{3}^{0}+T_{2}^{0}\right) = 52.50 \\
                   i=4, & T_{4}^{1} = 20 \\
                \end{cases}
              \end{displaymath}
              
            \item[j=1] $\rightarrow\;\;T_{i}^{j+1}=T_{i}^{2}$ $\left(t=600 s\right)$:\solmarks{5/16}
              \begin{displaymath}
                \begin{cases}
                   i=0, & T_{0}^{2} = T_{0}^{1} + \tau\left(T_{1}^{1}-2T_{0}^{1}+T_{-1}^{1}\right) = 85 \\
                   i=1, & T_{1}^{2} = T_{1}^{1} + \tau\left(T_{2}^{1}-2T_{1}^{1}+T_{0}^{1}\right) = 85 \\
                   i=2, & T_{2}^{2} = T_{2}^{1} + \tau\left(T_{3}^{1}-2T_{2}^{1}+T_{1}^{1}\right) = 68.75 \\
                   i=3, & T_{3}^{2} = T_{3}^{1} + \tau\left(T_{4}^{1}-2T_{3}^{1}+T_{2}^{1}\right) = 52.50 \\
                   i=4, & T_{4}^{2} = 20 \\
                \end{cases}
              \end{displaymath}
          \end{description}
          
         }
  \end{enumerate}
    Thermal diffusivity $\left(\alpha=\kappa\rho^{-1}C_{p}^{-1}\right)$ of the plate is 1.5$\times$10$^{-6}$ m$^{2}$.s$^{-1}$. The discretised form of the thermal energy equation is
    \begin{displaymath}
      \rho C_{p}\frac{T_{i}^{j+1}-T_{i}^{j+1}}{\Delta t} = \kappa \frac{T_{i+1}^{j}-2T_{i}^{j}+T_{i-1}^{j}}{\left(\Delta x\right)^{2}} + \mathcal{S}_{i}^{j},
    \end{displaymath}
    where $i$ and $j$ are spatial and temporal indices.

   \item A double-pipe (shell-and-tube) heat exchanger is constructed of a stainless steel ($\kappa=$ 15.1 W/(m.$^{\circ}$C) inner tube of inner diameter D$_{i}=$ 1.5 cm and outer diameter D$_{o}=$ 1.9 cm and an outer shell of inner diameter 3.2 cm. The convective heat transfer coefficient is h$_{i}=$ 800 W/(m$^{2}.^{\circ}$C) on the inner surface of the tube and h$_{o}=$ 1200 W/(m$^{2}.^{\circ}$C) on the outer surface. For a fouling factor R$_{f,i}=$ 0.0004 m$^{2}.^{\circ}$C/W on the tube side and R$_{f,o}=$ 0.0001 m$^{2}.^{\circ}$C/W on the shell side, determine:%%% Slide (Cengel Example 13.1)
\begin{enumerate}
   \item The thermal resistance of the heat exchanger per unit length $\left(\text{in }^{\circ}\text{C/W}\right)$ and; \marks{3}
     \solution{ The total thermal resistance, $R$, of the heat exchanger through the pipes per unit length is\solmarks{1/3}
       \begin{displaymath}
           R = \frac{1}{h_{i}A_{i}} + \frac{R_{fi}}{A_{i}} + \frac{\ln{\left(\frac{D_{0}}{D_{i}}\right)}}{2\pi\kappa L} + \frac{R_{f0}}{A_{0}} + \frac{1}{h_{0}A_{0}},
       \end{displaymath}
     where $A_{i}=0.04712$ m$^{2}$ and $A_{0}=0.5969$ m$^{2}$, resulting in $\mathbf{R = 5.3145\times 10^{-2\circ}}${\bf C.W}$^{-1}$~\solmarks{2/3}}
       
   \item The overall heat transfer coefficients, U$_{i}$ and U$_{o}$ $\left(\text{in W/}\left(\text{m}^{2}.^{\circ}\text{C}\right)\right)$ based on the inner and outer surface areas of the tube, respectively.\marks{7}
     \solution{The overall heat transfer coefficient based on the inner and the outter surface areas of the tube per length are~\solmarks{1/7}
       \begin{displaymath}
         R = \frac{1}{UA} = \frac{1}{U_{i}A_{i}} = \frac{1}{U_{0}A_{0}},
       \end{displaymath}
       thus for $U_{i}$\solmarks{3/7}
       \begin{displaymath}
         U_{i} = \frac{1}{R A_{i}} = 399.3303 \frac{\text{W}}{\text{m}^{2}.^{\circ}\text{C}}
       \end{displaymath}
       and for $U_{0}$\solmarks{3/7}
       \begin{displaymath}
         U_{0} = \frac{1}{R A_{0}} = 315.2461 \frac{\text{W}}{\text{m}^{2}.^{\circ}\text{C}}
       \end{displaymath}
       }
\end{enumerate}


  \end{enumerate}
  
\end{question}

\pagebreak
%%%
%%%  December Diet: Q02
%%%
  
\begin{question}

  \begin{enumerate}[1.]
    \item A new material is to be developed for bearing balls in a new rolling-element bearing. For annealing (heat treatment) each bearing ball, a sphere of radius $r_{o} =$ 5 mm, is heated in a furnace until it reaches to the equilibrium temperature of the furnace at 400$^{\circ}$C. Then, it is suddenly removed from the furnace and subjected to a two-step cooling process.
       \begin{description}
           \item[Stage 1:] Cooling in an air flow of 20$^{\circ}$C for a period of time $t_{\text{air}}$ until the center temperature reaches 335$^{\circ}$C. For this situation, the convective heat transfer coefficient of air is assumed constant and equal to $h =$ 10 W/(m$^{2}$.K). After the sphere has reached this specific temperature, the second step is initiated. 
           \item[Stage 2:] Cooling in a well-stirred water bath at 20$^{\circ}$C, with a convective heat transfer coefficient of water $h =$ 6000 W/(m$^{2}$.K). 
       \end{description}
       The thermophysical properties of the material are $\rho =$ 3000 kg/m$^{3}$, $\kappa =$ 20 W/(m.K), $C_{p} =$ 1000 J/(kg.K). Determine:
        \begin{enumerate}%[(a)]
           \item The time $t_{\text{air}}$ required for {\it Stage 1} of the annealing process to be completed;\marks{5}
             \solution{ The first step is to check if the lumped-capacitance method can be used:\solmarks{1/5}
               \begin{displaymath}
                 \mathbf{Bi =} \frac{h L_{c}}{\kappa} = \mathbf{8.33\times 10^{-4}}
               \end{displaymath}
               As $\mathbf{Bi < 0.1}$~\solmarks{1/5}, the lumped-capacitance method can be used in this stage, i.e., temperature changes uniformily throughout the sphere,\solmarks{3/5}
                 \begin{eqnarray}
                   && \frac{T(t)-T_{\infty}}{T_{0}-T_{\infty}} = \exp{\left[-\frac{h}{L_{c}\rho C_{p}}t\right]} \nonumber \\
                   && \frac{335-20}{400-20} = \exp{\left[-\frac{10 t}{\frac{5\times 10^{-3}}{3}\times 3000 \times 1000}\right]} \;\;\Rightarrow \;\; \mathbf{t = 93.80 \text{ s}} \nonumber
                 \end{eqnarray}     
             }
           \item The time $t_{\text{water}}$ required for {\it Stage 2} of the annealing process during which the center of the sphere cools from 335$^{\circ}$C (the condition at the completion of {\it Stage 1}) to 50$^{\circ}$C.\marks{6}
             \solution{ Checking if the lumped-capacitance method can be used:
                \begin{displaymath}
                   \mathbf{Bi = \frac{h L_{c}}{\kappa} = 0.5 > 0.1},
                \end{displaymath}
                therefore the {\bf lumped-capacitance method can not be used}~\solmarks{1/6}. In order to use the Tables for the analytical solution, we need to calculate the $Bi$ number based on $r_{0}$, 
                \begin{displaymath}
                   Bi = \frac{hr_{0}}{\kappa}=1.5
                \end{displaymath}
                From the Table with coefficients for the one-term approximate solution of 1D transient conduction in spheres, we can obtain $\mathbf{\lambda_{1} = 1.7998}$ and $\mathbf{A_{1} = 1.3763}$\solmarks{2/6} to be applied into
                \begin{displaymath}
                   \theta_{\text{centre}} = \frac{T_{0}-T_{\infty}}{T_{i}-T_{\infty}} = \frac{50-20}{335-20}=A_{1}\exp{\left[-\lambda_{1}^{2}\tau\right]},
                \end{displaymath}
                where $\tau=\alpha t.r_{0}^{-2}$ with $\mathbf{\alpha=\kappa.\left(\rho C_{p}\right)^{-1} = 6.67\times 10^{-6} \text{ m}^{2}.\text{s}^{-1}}$.\solmarks{1/6}
          
                Solving the equation above results in $\mathbf{\tau = 0.8245}$~\solmarks{1/6}, and $t$ is obtained from the Fourier number,~\solmarks{1/6}
                  \begin{displaymath}
                    \mathbf{\tau =} \frac{\alpha t}{r_{0}^{2}} \;\;\Rightarrow \;\; \mathbf{t = 3.09}\text{ {\bf s}}
                  \end{displaymath}        
             }
        \end{enumerate}
        
    \item Water at the rate of 68 kg/min is heated from 35 to 75$^{\circ}$C by an oil having a specific heat of 1.9 kJ/(kg.$^{\circ}$C). The fluids are used in a counterflow double-pipe HE, and the oil enters the exchanger at 110$^{\circ}$C and leaves at 75$^{\circ}$C. The overall heat-transfer coefficient is 320 W/(m$^{2}.^{\circ}$C). Given heat capacity of water (at constant pressure) of 4.18 kJ/(kg.$^{\circ}$C),
\begin{enumerate}[(a)]
   \item Calculate the HE area;\marks{5}
     \solution{The total heat transfer can be obtained from~\solmarks{1/5}
       \begin{displaymath}
         Q = \dot{m}_{w}C_{p,w}\Delta T_{w} = 189.49\text{ kJ.s}^{-1}.
       \end{displaymath}
       And the heat exchange surface area can be calculated from
       \begin{displaymath}
         Q = U A \Delta T_{lm},
       \end{displaymath}
       where $U= 320$ W.m$^{-2}.^{\circ}$C$^{-1}$ and~\solmarks{2/5}
       \begin{displaymath}
         \Delta T_{lm} = \frac{\left(T_{h,in}-T_{c,out}\right)-\left(T_{h,out}-T_{c,in}\right)}{\ln{\frac{T_{h,in}-T_{c,out}}{T_{h,out}-T_{c,in}}}} = 37.44^{\circ}\text{C},
       \end{displaymath}
       thus~\solmarks{2/5}
       \begin{eqnarray}
         Q &=& U A \Delta T_{lm} =  189.49\text{ kJ.s}^{-1} \nonumber \\
         A &=& 15.82\text{ m}^{2} \nonumber
       \end{eqnarray}       
     }
   \item Now assume that the heat exchanger is a shell-and-tube with water making one shell pass and the oil making two tube passes. Calculate the area of the new heat exchanger. Assume that the overall heat-transfer coefficient remains the same.\marks{4}
     \solution{For cross-flow and multi-pass shell-and-tubes heat exchangers,
       \begin{displaymath}
         Q = U A F \Delta T_{lm}. 
       \end{displaymath}
       From Fig. 10.8 with,~\solmarks{2/4}
       \begin{displaymath}
         \begin{cases}
           R = \frac{T_{c,in}-T_{c,out}}{T_{h,out}-T_{h,in}} = 1.1429, & \\
           P = \frac{T_{h,out}-T_{h,in}}{T_{c,in}-T_{h,in}} = 0.4667, & \\
         \end{cases}
       \end{displaymath}
       leads to $F\sim 0.8$ and~\solmarks{2/4}
       \begin{displaymath}
         Q = U A F \Delta T_{lm} \;\;\Longrightarrow\;\; A = 19.77\text{ m}^{2}
       \end{displaymath}
     }
\end{enumerate}

  \end{enumerate}
  
\end{question}

\pagebreak
%%%
%%%  Resit Diet: Q01
%%%
\begin{question}
  \begin{enumerate}
      \item  Hot oil is to be cooled in a double-tube counter-flow heat exchanger. The copper inner tubes have diameter of 2 cm and negligible thickness. The inner diameter of the outer tube (shell) is 3 cm. Water flows through the tube at a rate of 0.5 kg.s$^{-1}$, and the oil through the shell at a rate of 0.8 kg.s$^{-1}$. Taking the average temperatures of the water and the oil to be 45$^{\circ}$C and 80$^{\circ}$C, respectively, determine the overall heat transfer coefficient of this heat exchanger. Given, 
         \begin{enumerate} 
             \item Water at 45$^{\circ}$C: $\rho=990$ kg.m$^{-3}$, $\kappa=0.637$ W.(m.K)$^{-1}$, $Pr=3.91$, $\nu=\mu/\rho=0.602\times 10^{-6}$ m$^{2}$.s$^{-1}$; 
             \item Oil at 80$^{\circ}$C: $\rho=852$ kg.m$^{-3}$, $\kappa=0.138$ W.(m.K)$^{-1}$, $Pr=490$, $\nu=37.5\times 10^{-6}$ m$^{2}$.s$^{-1}$.
         \end{enumerate}
         The overall heat transfer coefficient can be expressed as,
           \begin{displaymath}
             U^{-1} = h_{i}^{-1} + h_{0}^{-1}
           \end{displaymath}
         The inner convective heat transfer coefficient, $h_{i}$, can be obtained from
         \begin{displaymath}
                Nu = \frac{h_{i}D_{h}}{\kappa} =
                     \begin{cases}
                         4.36  & \text{(for laminar flows),} \\
                         0.023 Re^{0.8} Pr^{0.4} & \text{(for turbulent flows),} \\
                     \end{cases}
         \end{displaymath}
         where $D_{h}$ is the hydraulic diameter. The outter convective heat transfer coefficient, $h_{0}$ is 75.2 W.$\left(\text{m}^{2}.\text{K}\right)$.\marks{17}
         \solution{In order to solve this problem, we need to calculate $h_{i}$ through the dimensionless Nusselt number with $D_{h} = D_{i} = 2\times 10^{-2}$ m,
         \begin{displaymath}
                Nu = \frac{h_{i}D_{h}}{\kappa} =
                     \begin{cases}
                         4.36  & \text{(for laminar flows),} \\
                         0.023 Re^{0.8} Pr^{0.4} & \text{(for turbulent flows).} \\
                     \end{cases}
         \end{displaymath}
         However, we first need to assess the flow regime in the inner tube through the dimensionless Reynolds number
         \begin{displaymath}
           Re_{D} = \frac{\rho v D}{\mu} = \frac{v D}{\nu}
         \end{displaymath}
         The flow velocity, $v$, can be obtained from the mass flow rate, diameter of the tube and density of the fluid,~\solmarks{4/17}
         \begin{displaymath}
           \dot{m}_{w} = v\rho A \;\;\;\Longrightarrow\;\;\; v = 1.6077\text{ m.s}^{-1},
         \end{displaymath}
         with $Re_{D}$~\solmarks{2/17}
         \begin{displaymath}
           Re_{D} = \frac{v D}{\nu} = 53411.96
         \end{displaymath}
         As $Re >>> 4000$, the water flow is turbulent and we can use~\solmarks{6/17} 
         \begin{eqnarray}
           Nu &=& 0.023 Re^{0.8} Pr^{0.4} = 240.2754 \nonumber \\
           Nu &=& \frac{h_{i}D_{h}}{\kappa} = 240.2754 \;\;\Longrightarrow \;\; h_{i} = 7652.77 \text{ W.}\left(\text{m}^{2}.\text{K}\right)^{-1} \nonumber
         \end{eqnarray}
         Now, calculating $U$~\solmarks{5/17} 
           \begin{displaymath}
             U^{-1} = h_{i}^{-1} + h_{0}^{-1} \;\;\;\Longrightarrow\;\; U =74.47 \text{ W.}\left(\text{m}^{2}.\text{K}\right)^{-1}
           \end{displaymath}
            
         }
         
      \item A long rod of 60 mm diameter and thermophysical properties $\rho=$ 8000 kg/m$^{3}$, $C_{p}=$ 500 J/(kg.K), and $k=$ 50 W/(m.K) is initially at a uniform temperature and is heated in a forced convection furnace maintained at 750 K. The convection coefficient is estimated to be 1000 W/(m$^{2}$.K). Calculate the centerline temperature of the rod when the surface temperature is 550 K.\marks{16}
   \solution{ Assuming that the Fourier number for this problem is larger than 0.2, we can use the 1D analytical solution for the transient conductive equation,\solmarks{3/16}
        \begin{displaymath}
           \theta_{\text{cyl}} = \frac{T(r,t)-T_{\infty}}{T_{i}-T_{\infty}}= A_{1}e^{-\lambda_{1}^{2}\tau}\mathbf{J_{0}}\left(\frac{\lambda_{1}r}{r_{0}}\right)
        \end{displaymath}
and at the centre of the geometry,\solmarks{3/16}
        \begin{displaymath}
           \theta_{\text{cyl}} = \frac{T(0,t)-T_{\infty}}{T_{i}-T_{\infty}}= A_{1}e^{-\lambda_{1}^{2}\tau}
        \end{displaymath}
We need to obtain $T(0,t)$ at time $t=t_{i}$ such that $T\left(r_{0},t_{i}\right) =$ 500 K. Merging both equations,\solmarks{3/16}
        \begin{displaymath}
           \frac{T(r_{0},t_{i})-T_{\infty}}{T_{i}-T_{\infty}} = \frac{T(0,t_{i})-T_{\infty}}{T_{i}-T_{\infty}}\mathbf{J_{0}}\left(\frac{\lambda_{1}r_{0}}{r_{0}}\right)
        \end{displaymath}
In order to solve this expression, $\mathbf{J_{0}}\left(\frac{\lambda_{1}r}{r_{0}}\right)$ and $\lambda_{1}$ need to be obtained from the Table of coefficients for the approximate solution of the transient 1D heat conduction based on,\solmarks{4/16}
       \begin{displaymath}
           Bi = \frac{h r_{0}}{\kappa} = 0.60 \;\;\Rightarrow \;\; \lambda_{1} = 1.0184\;\;\text{ and }\;\; \mathbf{J_{0}}\left(\frac{\lambda_{1}r_{0}}{r_{0}}\right) = \mathbf{J_{0}}\left(\lambda_{1}\right) = 0.7571
       \end{displaymath} 
Replacing $\mathbf{J_{0}}$ in the merged equation, \solmarks{3/16}
       \begin{displaymath}
           (550-750) = \left[T\left(0,t_{i}\right)-750\right] \times 0.7571 \;\;\Rightarrow T\left(0,t_{i}\right) = 485.83\text{ K}
       \end{displaymath} 
    
}

  
  \end{enumerate}
\end{question}
 
\pagebreak
%%%
%%%  Resit Diet: Q02
%%%
\begin{question}
  \begin{enumerate}
  \item In a counterflow double-pipe heat exchanger, water $\left(C_{p}= 4.18\text{ kJ.kg}^{-1}.^{\circ}\text{C}\right)$ at 35$^{\circ}$C is heated by oil $\left(C_{p}= 1.9\text{ kJ.kg}^{-1}.^{\circ}\text{C}\right)$. The mass flow rate of the water stream is 40 kg.min$^{-1}$ and 170.97 kg.min$^{-1}$ of oil enters the heat exchanger at 110$^{\circ}$C. The overall heat-transfer coefficient is 320 W.$\left(\text{m}^{2}.^{\circ}\text{C}\right)^{-1}$. Calculate:
    \begin{enumerate}[(a)]
      \item Exit water and oil temperatures;\marks{5}
          \solution{The energy balance for both fluid streams (oil and water), where $T_{oil,in}=T_{h,in}=110^{\circ}$C and $T_{w,in}=T_{c,in}=35^{\circ}$C, ~\solmarks{1/5}
            \begin{eqnarray}
              &&Q_{w} + Q_{oil} = 0 \nonumber \\
              &&\dot{m}_{w}C_{p,w}\left(T_{c,out}-T_{c,in}\right) = -\dot{m}_{o}C_{p,o}\left(T_{h,out}-T_{h,in}\right) \nonumber \\
              &&\frac{T_{c,out}-T_{c,in}}{T_{h,in}-T_{h,out}} =  \frac{\dot{m}_{o}C_{p,o}}{\dot{m}_{w}C_{p,w}} \;\;\Longrightarrow \;\; T_{c,out} = T_{c,in} + \frac{\dot{m}_{o}C_{p,o}}{\dot{m}_{w}C_{p,w}} {T_{h,in}-T_{h,out}}, \nonumber
            \end{eqnarray}
            $T_{c,out}=T_{w,out}$ can be obtain as a function of $T_{h,out}$. However, with $Q=U A \Delta T_{lm}$,~\solmarks{1/5}
            \begin{displaymath}
              \dot{m}_{w}C_{p,w}\left(T_{c,out}-T_{c,in}\right) = U A \frac{\left(T_{h,in}-T_{c,out}\right)-\left(T_{h,out}-T_{c,in}\right)}{\ln{\frac{T_{h,in}-T_{c,out}}{T_{h,out}-T_{c,in}}}}
            \end{displaymath}
            and substituting $T_{c,out}$ in this expression we obtain $T_{h,out}=79.70^{\circ}$C\solmarks{1/5}. Now,\solmarks{2/5}
            \begin{displaymath}
              \frac{T_{c,out}-T_{c,in}}{T_{h,in}-T_{h,out}} = \frac{\dot{m}_{o}C_{p,o}}{\dot{m}_{w}C_{p,w}} = 1.9427 \;\;\Longrightarrow \;\;T_{c,out}=T_{w,out} = 93.86^{\circ}\text{C}
             \end{displaymath}
          }
        \item Total heat transfer (in kW);\marks{1}
          \solution{ The heat transferred between streams can be obtained from~\solmarks{1/1}
            \begin{displaymath}
              Q = \dot{m}_{w}C_{p,w}\left(T_{c,out}-T_{c,in}\right) = 164023.2\text{ J.s}^{-1} = 164.02\text{ kW}
            \end{displaymath}
            }
    \end{enumerate}

    \item Consider three consecutive nodes $n-1$, $n$, $n+1$ in a plane wall. Using the finite difference form of the first derivative at the midpoints, show that the finite difference form of  the second derivative can be expressed as,\marks{4}
\begin{displaymath}
  \frac{T_{n-1}-2T_{n}+T_{n+1}}{\Delta x^{2}} = \left.\frac{\partial^{2} T}{\partial x^{2}}\right|_{N}.
\end{displaymath}
    {\bf Hint:} You should start the demonstration from the 1D expansion in Taylor series of a continuous and real function $f(x)$ about a point $x=a$,
    \begin{displaymath}
       f(x) = f(a) + f^{'}(a)\left(x-a\right) + \frac{f^{''}(a)}{2!}\left(x-a\right)^{2} + \frac{f^{'''}(a)}{3!}\left(x-a\right)^{3} + \cdots + \frac{f^{n}(a)}{n!}\left(x-a\right)^{n}
    \end{displaymath}
    \solution{In a 1D plane wall divided in $M$ nodes with an internal node $N$. Spatial increment is given by $\Delta x$. The expansion in Taylor series around node $N$ truncating in the second derivative leads to:\solmarks{3/4}
      \begin{displaymath}
        \begin{cases}
             T_{N-1} = T_{N} - \Delta x \left(\frac{dT}{dx}\right)_{N} + \frac{\left(\Delta x\right)^{2}}{2!} \left(\frac{d^{2}T}{dx^{2}}\right)_{N} + \mathcal{O}\left[\left(\Delta x\right)^{3}\right], & \\
             T_{N+1} = T_{N} + \Delta x \left(\frac{dT}{dx}\right)_{N} + \frac{\left(\Delta x\right)^{2}}{2!} \left(\frac{d^{2}T}{dx^{2}}\right)_{N} + \mathcal{O}\left[\left(\Delta x\right)^{3}\right], & 
        \end{cases}
      \end{displaymath}
      Summing these expressions leads to\solmarks{1/4}
\begin{displaymath}
  \frac{T_{n-1}-2T_{n}+T_{n+1}}{\Delta x^{2}} = \left.\frac{\partial^{2} T}{\partial x^{2}}\right|_{N}.
\end{displaymath}
    }

  \item Heavy oil at 150$^{\circ}$C is pumped into a storage tank before being transported to distillation columns. The tank is insulated with a layer of polyisocyanurate of 10 cm thick. The tank's wall is at 150$^{\circ}$C and the initial temperature of the insulation layer is 20$^{\circ}$C. Assuming that the tank wall-insulation layer-environment system can be modelled as a 1-D finite difference method (FDM) problem,  calculate the temperature profile of the insulated layer, $T\left(\underline{x},t\right)$, at $t = 5$ seconds with $\underline{x}=\left[0.0, 2.5, 5.0, 7.5, 10.0\right]$. The outer layer of the insulation material is subjected to the environment with temperature of 15$^{\circ}$C and convective heat transfer coefficient of 10 W$\left(\text{m}^{2}.^{\circ}\text{C}\right)^{-1}$. Given for polyisocyanurate insulation layer:
    \begin{itemize}
      \item Conductive heat transfer coefficient: 5.40 W.$\left(\text{m.}^{\circ}\text{C}\right)^{-1}$;
      \item Heat capacity at constant pressure: 0.1 kJ.$\left(\text{kg.}^{\circ}\text{C}\right)^{-1}$;
      \item Density: 550 kg.m$^{-3}$.
    \end{itemize}
    The discretised thermal energy equation is
\begin{displaymath}
    T_{i}^{j+1} = T_{i}^{j} + \alpha\frac{\Delta t}{\left(\Delta x\right)^{2}}\left(T_{i+1}^{j}-2T_{i}^{j}+T_{i-1}^{j}\right)
\end{displaymath}
where $\alpha=\kappa\left(\rho C_{p}\right)^{-1}$ is the thermal diffusivity, $\Delta x$ and $\Delta t$ are the spatial-interval and time-step size, respectively. $i$ and $j$ are the spatial- and time-indices. For this problem, use $\Delta t = 5$ seconds.~\marks{10}
%=============
\solution{ The thermal diffusivity,~\solmarks{1/10}
\begin{displaymath}
     \alpha = \frac{\kappa}{\rho C_{p}} = \left(5.40\frac{W}{m.^{\circ}C}\right)\left(\frac{1}{550}\frac{m^{3}}{kg}\right)\left(\frac{1}{0.1\times 10^{3}}\frac{kg.^{\circ}C}{J}\right) = {\bf 9.8182\times 10^{-5} \frac{m^{2}}{s}}, 
\end{displaymath}
and the Fourier number,~\solmarks{1/10}
\begin{displaymath}
  \tau = \frac{\alpha \Delta t}{\left(\Delta x\right)^{2}} = \left(9.8182\times 10^{-5} \frac{m^{2}}{s}\right)\left( 5 s\right)\left(\frac{1}{2.5\times 10^{-2}}\frac{1}{m}\right)^{2} = {\bf 0.7855}.
\end{displaymath}
Boundary and initial conditions are:
\begin{itemize}
    \item Dirichlet BC at node $i=0\;\left(x = 0.0\text{cm}\right)$ : {\bf $T_{0}^{0} = T_{0}^{1} = T_{0}^{2} = \cdots = $ 150$^{\circ}$C};%~\solmarks{1/10}
    \item Newmann BC at node $i=4\;\left(x = 10.0\text{cm}\right)$: %~\solmarks{2/10}
          \begin{eqnarray}
             && - \kappa\frac{\partial T}{\partial x} = h\left( T - T_{\infty}\right) \Longrightarrow - \kappa\frac{T_{i+1}^{j}-T_{i-1}^{j}}{2\Delta x} = h\left( T_{i}^{j}- T_{\infty}\right) \nonumber \\
             && \mathbf{T_{i+1}^{j} = T_{i-1}^{j} - \frac{2 h \Delta x }{\kappa}\left(T_{i}^{j}-T_{\infty}\right)}, \nonumber
          \end{eqnarray}
    \item Initial conditions: {\bf $T_{1}^{0}=T_{2}^{0}=T_{3}^{0}=T_{4}^{0}=$ 20$^{\circ}$C}.%~\solmarks{1/10}
\end{itemize}

The discretised thermal equation,~\solmarks{2/10}
\begin{eqnarray}
    && T_{i}^{j+1} = T_{i}^{j} + \alpha\frac{\Delta t}{\left(\Delta x\right)^{2}}\left(T_{i+1}^{j}-2T_{i}^{j}+T_{i-1}^{j}\right)\;\;\text{ with }\;\;\tau = \frac{\alpha \Delta t}{\left(\Delta x\right)^{2}} \nonumber \\
    && \mathbf{T_{i}^{j+1} = \left( 1- 2\tau \right)T_{i}^{j} + \tau\left(T_{i+1}^{j}+T_{i-1}^{j}\right)} \nonumber
\end{eqnarray}

Thus for $j=0$:%~\solmarks{2/10}
         \begin{eqnarray}
             \mathbf{i=1}   &&  \mathbf{T_{1}^{1} = \left(1-2\tau\right)T_{1}^{0} + \tau\left(T_{2}^{0}+T_{0}^{0}\right)} \nonumber \\
               \vdots &&   \hspace{3cm}                \vdots \nonumber \\
             \mathbf{i=4}   &&  \mathbf{T_{4}^{1} = \left(1-2\tau\right)T_{4}^{0} + \tau\left(T_{5}^{0}+T_{3}^{0}\right)} \nonumber 
         \end{eqnarray} 
         with ghost-cell, $T_{5}^{0}$ defined through the Newmann BC:~\solmarks{2/10}
         \begin{displaymath}
             \mathbf{T_{5}^{0} = T_{3}^{0} - \frac{2 h\Delta x}{\kappa}\left(T_{4}^{0}-T_{\infty}\right)}
         \end{displaymath}

Thus:~\solmarks{4/10}
\begin{center}
\begin{tabular}{c | c c c c c }
{\bf t (s)}  &  $T_{0}$  &  $T_{1}$  &  $T_{2}$  &  $T_{3}$  &  $T_{4}$  \\
\hline
0.0          &  150.00  &  20.00    &  20.00   &  20.00   &  20.00  \\
{\bf 5.0 }         &  {\bf 150.00 }  &  {\bf 122.12 }   &  {\bf 20.00 }   &  {\bf 20.00 }   &  {\bf 20.00 }
\end{tabular}
\end{center}
}
      
  \end{enumerate}
\end{question}
 

\pagebreak
%%%%%%%%%%%%%%%%%%%%%%%%%%%%%%%%%%%%%%%%%%%%%%%%%%%%%%%%%%%%%%%%%%%%%%%%%%%%%%%%%%%%%%%%%%%%%%%%%%%%%%%%%%%%%% 
%%%%%%%%%%%%%%%%%%%%%%%%%%%%%%%%%%%%%%%%%%%%%%%%%%%%%%%%%%%%%%%%%%%%%%%%%%%%%%%%%%%%%%%%%%%%%%%%%%%%%%%%%%%%%% 
%\begin{datasheet}
{\bf General balance equations:}
\begin{align}
  \frac{\partial\rho}{\partial t} &= -\nabla \cdot \rho\,\bm{v}\label{eq:continuity} & \text{(Mass/Continuity)}\\
  \frac{\partial C_A}{\partial t} &= -\nabla\cdot\bm{N}_{A} + \sigma_A& \text{(Species)}\\
  \rho \frac{\partial\bm{v}}{\partial t} &= -\rho\,\bm{v}\cdot\nabla
  \bm{v}
  - \nabla\cdot\bm{\tau} - \nabla\,p + \rho\,\bm{g}\label{eq:momentumbalance} & \text{(Momentum)}\\
  \rho\,C_p\frac{\partial T}{\partial t} &= -\rho\,C_p\,\bm{v}
  \cdot\nabla\,T - \nabla\cdot\bm{q} - \bm{\tau}:\nabla\,\bm{v} -
  p\,\nabla\cdot\bm{v} +\sigma_{energy}\label{eq:energybalance} &
  \text{(Heat/Energy)}
\end{align}

In Cartesian coordinate systems, $\nabla$ can be treated as a vector of
derivatives. In curvelinear coordinate systems, the directions
$\hat{\bm{r}}$, $\hat{\bm{\theta}}$, and $\hat{\bm{\phi}}$ depend on
the position. For convenience in these systems, look-up tables are
provided for common terms involving $\nabla$.

\vspace{\baselineskip}
{\bf Cartesian coordinates} (with index notation examples)\\
where $s$ is a scalar, $\bm{v}$ is a vector, and $\bm{\tau}$ is a tensor.
\begin{align*}
  \nabla s = \nabla_i s &= \left[\frac{\partial\,s}{\partial x},\,
  \frac{\partial\,s}{\partial y},\, \frac{\partial\,s}{\partial z}\right]
  \\
  \nabla^2 s = \nabla_i\nabla_i s &=\frac{\partial^2\,s}{\partial x^2} +
  \frac{\partial^2\,s}{\partial y^2}+ \frac{\partial^2\,s}{\partial z^2}
  \\
  \nabla\cdot\bm{v} =\nabla_i v_i &= \frac{\partial\,v_x}{\partial x} +
  \frac{\partial\,v_y}{\partial y}+ \frac{\partial\,v_z}{\partial z}
  \\
  \nabla \cdot \bm{\tau} &= \nabla_i\, \tau_{ij}
  \\
  \left[\nabla \cdot \bm{\tau}\right]_x &= \frac{\partial\,\tau_{xx}}{\partial x} +
  \frac{\partial\, \tau_{yx}}{\partial y} + \frac{\partial\, \tau_{zx}}{\partial z}
  \\
  \left[\nabla \cdot \bm{\tau}\right]_y &= \frac{\partial\,\tau_{xy}}{\partial x} +
  \frac{\partial\, \tau_{yy}}{\partial y} + \frac{\partial\, \tau_{zy}}{\partial z}
  \\
  \left[\nabla \cdot \bm{\tau}\right]_z &= \frac{\partial\,\tau_{xz}}{\partial x} +
  \frac{\partial\, \tau_{yz}}{\partial y} + \frac{\partial\, \tau_{zz}}{\partial z}
  \\
  \bm{v}\cdot \nabla \bm{v} &= v_i\,\nabla_i\,v_j
  \\
  \left[\bm{v}\cdot \nabla \bm{v}\right]_x &= v_x\frac{\partial\,v_x}{\partial x} + v_y\frac{\partial\,v_x}{\partial y} +v_z\frac{\partial\,v_x}{\partial z}
  \\
  \left[\bm{v}\cdot \nabla \bm{v}\right]_y &= v_x\frac{\partial\,v_y}{\partial x} + v_y\frac{\partial\,v_y}{\partial y} +v_z\frac{\partial\,v_y}{\partial z}
  \\
  \left[\bm{v}\cdot \nabla \bm{v}\right]_z &= v_x\frac{\partial\,v_z}{\partial x} + v_y\frac{\partial\,v_z}{\partial y} +v_z\frac{\partial\,v_z}{\partial z}
\end{align*}

\pagebreak
{\bf Cylindrical coordinates}\\
where $s$ is a scalar, $\bm{v}$ is a vector, and $\bm{\tau}$ is a
tensor. All expressions involving $\bm{\tau}$ are for symmetrical
$\bm{\tau}$ only.
\begin{align*}
  \nabla s &= \left[\frac{\partial\,s}{\partial r},\,
    \frac{1}{r}\frac{\partial\,s}{\partial \theta},\,
    \frac{\partial\,s}{\partial z}\right]
  \\
  \nabla^2 s &=\frac{1}{r}\frac{\partial}{\partial
    r}\left(r\frac{\partial s}{\partial r}\right) +
  \frac{1}{r^2}\frac{\partial^2\,s}{\partial \theta^2}+
  \frac{\partial^2\,s}{\partial z^2}
  \\
  \nabla\cdot\bm{v} &= \frac{1}{r}\frac{\partial}{\partial r}\left(r\,
    v_r\right) + \frac{1}{r}\frac{\partial\,v_\theta}{\partial
    \theta}+ \frac{\partial\,v_z}{\partial z}
  \\
  \left[\nabla \cdot \bm{\tau}\right]_r &=
  \frac{1}{r}\frac{\partial}{\partial r}\left(r\,\tau_{rr}\right) +
  \frac{1}{r}\frac{\partial\, \tau_{r\theta}}{\partial \theta} -
  \frac{1}{r} \tau_{\theta\theta} + \frac{\partial\,
    \tau_{rz}}{\partial z}
  \\
  \left[\nabla \cdot \bm{\tau}\right]_\theta &=
  \frac{1}{r}\frac{\partial \tau_{\theta\theta}}{\partial \theta} +
  \frac{\partial\, \tau_{r\theta}}{\partial r} + \frac{2}{r}
  \tau_{r\theta} + \frac{\partial\, \tau_{\theta z}}{\partial z}
  \\
  \left[\nabla \cdot \bm{\tau}\right]_z &= \frac{1}{r}\frac{\partial
  }{\partial r}\left(r\,\tau_{rz}\right) + \frac{1}{r}\frac{\partial
    \tau_{\theta z}}{\partial\theta} + \frac{\partial\, \tau_{z
      z}}{\partial z}
  \\
  \left[\bm{v}\cdot \nabla \bm{v}\right]_r &= v_r \frac{\partial
    v_r}{\partial r} + \frac{v_\theta}{r}\frac{\partial v_r}{\partial
    \theta} - \frac{v_\theta^2}{r}+v_z\frac{\partial v_r}{\partial z}
  \\
  \left[\bm{v}\cdot \nabla \bm{v}\right]_\theta &=
  v_r \frac{\partial v_\theta}{\partial r} + \frac{v_\theta}{r}\frac{\partial v_\theta}{\partial \theta} + \frac{v_r\,v_\theta}{r} + v_z \frac{\partial v_\theta}{\partial z}
  \\
  \left[\bm{v}\cdot \nabla \bm{v}\right]_z &=
  v_r \frac{\partial v_z}{\partial r} + \frac{v_\theta}{r}\frac{\partial v_z}{\partial \theta}+v_z \frac{\partial v_z}{\partial z}
\end{align*}
{\bf Spherical coordinates}\\
where $s$ is a scalar, $\bm{v}$ is a vector, and $\bm{\tau}$ is a
tensor. All expressions involving $\bm{\tau}$ are for symmetrical
$\bm{\tau}$ only.
\begin{align*}
  \nabla s &= \left[\frac{\partial\,s}{\partial r},\,
    \frac{1}{r}\frac{\partial\,s}{\partial \theta},\,
    \frac{1}{r\,\sin\theta}\frac{\partial\,s}{\partial \phi}\right]
  \\
  \nabla^2 s &=\frac{1}{r^2}\frac{\partial}{\partial r}\left(r^2
    \frac{\partial s}{\partial r}\right) +
  \frac{1}{r^2\sin\theta}\frac{\partial}{\partial
    \theta}\left(\sin\theta \frac{\partial s}{\partial \theta}\right)+
  \frac{1}{r^2 \sin^2\theta}\frac{\partial^2\,s}{\partial \phi^2}
  \\
  \nabla\cdot\bm{v} &= \frac{1}{r^2}\frac{\partial}{\partial
    r}\left(r^2\, v_r\right) +
  \frac{1}{r\sin\theta}\frac{\partial}{\partial
    \theta}\left(v_\theta\sin\theta\right)+
  \frac{1}{r\sin\theta}\frac{\partial\,v_\phi}{\partial \phi}
  \\
  \left[\nabla \cdot \bm{\tau}\right]_r &=
  \frac{1}{r^2}\frac{\partial}{\partial r}\left(r^2\,\tau_{rr}\right)
  + \frac{1}{r\sin\theta}\frac{\partial}{\partial
    \theta}\left(\tau_{r\theta}\sin\theta\right)
  +\frac{1}{r\sin\theta}\frac{\partial\, \tau_{r\phi}}{\partial \phi}
  - \frac{\tau_{\theta\theta}+\tau_{\phi\phi}}{r}
  \\
  \left[\nabla \cdot \bm{\tau}\right]_\theta &=
  \frac{1}{r^2}\frac{\partial}{\partial
    r}\left(r^2\,\tau_{r\theta}\right) +
  \frac{1}{r\sin\theta}\frac{\partial}{\partial
    \theta}\left(\tau_{\theta\theta}\sin\theta\right)
  +\frac{1}{r\sin\theta}\frac{\partial\, \tau_{\theta\phi}}{\partial
    \phi} + \frac{\tau_{r\theta}}{r} -
  \frac{\cot\theta}{r}\tau_{\phi\phi}
  \\
  \left[\nabla \cdot \bm{\tau}\right]_\phi &=
  \frac{1}{r^2}\frac{\partial}{\partial
    r}\left(r^2\,\tau_{r\phi}\right) + \frac{1}{r}\frac{\partial
    \tau_{\theta\phi}}{\partial \theta}
  +\frac{1}{r\sin\theta}\frac{\partial\, \tau_{\phi\phi}}{\partial
    \phi} + \frac{\tau_{r\theta}}{r} +
  \frac{2\cot\theta}{r}\tau_{\theta\phi}
  \\
  \left[\bm{v}\cdot \nabla \bm{v}\right]_r &= v_r \frac{\partial
    v_r}{\partial r} + \frac{v_\theta}{r}\frac{\partial v_r}{\partial
    \theta} + \frac{v_\phi}{r\sin\theta}\frac{\partial v_r}{\partial
    \phi}-\frac{v_\theta^2+v_\phi^2}{r}
  \\
  \left[\bm{v}\cdot \nabla \bm{v}\right]_\theta &= v_r \frac{\partial
    v_\theta}{\partial r} + \frac{v_\theta}{r}\frac{\partial v_\theta}{\partial
    \theta} + \frac{v_\phi}{r\sin\theta}\frac{\partial v_\theta}{\partial
    \phi}+\frac{v_r\,v_\theta -v_\phi^2\cot\theta}{r}
  \\
  \left[\bm{v}\cdot \nabla \bm{v}\right]_\phi &= v_r \frac{\partial
    v_\phi}{\partial r} + \frac{v_\theta}{r}\frac{\partial v_\phi}{\partial
    \theta} + \frac{v_\phi}{r\sin\theta}\frac{\partial v_\phi}{\partial
    \phi}+\frac{v_r\,v_\phi +v_\theta\,v_\phi\cot\theta}{r}
\end{align*}
\begin{table}[h!]\label{tab:constitutive}
  {\hspace{-25pt}\setlength{\extrarowheight}{5pt}%
    \begin{tabular}{|r|c|r|c|r|c|}
      \hline
      \multicolumn{2}{|c|}{Rectangular} & \multicolumn{2}{c|}{Cylindrical} & \multicolumn{2}{c|}{Spherical}
      \\\hline\hline
      $q_x$ & $-k\frac{\partial T}{\partial x}$ &
      $q_r$ & $-k\frac{\partial T}{\partial r}$ &
      $q_r$ & $-k\frac{\partial T}{\partial r}$
      \\[5pt]\hline
      $q_y$ & $-k\frac{\partial T}{\partial y}$ &
      $q_\theta$ & $-k\frac{1}{r}\frac{\partial T}{\partial \theta}$ &
      $q_\theta$ & $-k\frac{1}{r}\frac{\partial T}{\partial \theta}$ 
      \\[5pt]\hline
      $q_z$ & $-k\frac{\partial T}{\partial z}$ &
      $q_z$ & $-k\frac{\partial T}{\partial z}$ & 
      $q_\phi$ & $-k\frac{1}{r\,\sin\theta}\frac{\partial T}{\partial \phi}$
      \\[5pt]\hline
      $\tau_{xx}$
      &$-2\,\mu\frac{\partial v_x}{\partial x} + \mu^B \,\nabla\cdot\bm{v}$
      & $\tau_{rr}$ & $-2\,\mu\frac{\partial v_r}{\partial r} + \mu^B \,\nabla\cdot\bm{v}$
      & $\tau_{rr}$ & $ -2\,\mu\frac{\partial v_r}{\partial r} + \mu^B \,\nabla\cdot\bm{v}$
      \\[5pt]\hline
      $\tau_{yy}$ & $ -2\,\mu\frac{\partial v_y}{\partial y} + \mu^B \,\nabla\cdot\bm{v}$
      & $\tau_{\theta\theta}$ & $-2\,\mu\left(\frac{1}{r}\frac{\partial v_\theta}{\partial\theta}+\frac{v_r}{r}\right) + \mu^B \,\nabla\cdot\bm{v}$
      & $\tau_{\theta\theta}$ & $-2\,\mu\left(\frac{1}{r}\frac{\partial v_\theta}{\partial\theta}+\frac{v_r}{r}\right) + \mu^B \,\nabla\cdot\bm{v}$
      \\[5pt]\hline
      \multirow{2}{*}{$\tau_{zz}$} & \multirow{2}{*}{$-2\,\mu\frac{\partial v_z}{\partial z} + \mu^B \,\nabla\cdot\bm{v}$}
      & \multirow{2}{*}{$\tau_{zz}$} & \multirow{2}{*}{$-2\,\mu\frac{\partial v_z}{\partial z} + \mu^B \,\nabla\cdot\bm{v}$}
      & \multirow{2}{*}{$\tau_{\phi\phi}$} & $-2\,\mu\left(\frac{1}{r\sin\theta}\frac{\partial v_\phi}{\partial \phi} + \frac{v_r+v_\theta\cot\theta}{r}\right)$
      \\
      & & & & & \hfill $+ \mu^B \,\nabla\cdot\bm{v}$
      \\[5pt]\hline
      $\tau_{xy}$ & $-\mu\left(\frac{\partial v_x}{\partial y} +\frac{\partial v_y}{\partial x}\right)$
      & $\tau_{r\theta}$ & $-\mu\left(r\frac{\partial}{\partial r}\left(\frac{v_\theta}{r}\right) +\frac{1}{r}\frac{\partial v_r}{\partial \theta}\right)$
      & $\tau_{r\theta}$ & $-\mu\left(r\frac{\partial}{\partial r}\left(\frac{v_\theta}{r}\right) +\frac{1}{r}\frac{\partial v_r}{\partial \theta}\right)$
      \\[5pt]\hline
      $\tau_{yz}$ & $-\mu\left(\frac{\partial v_y}{\partial z} +\frac{\partial v_z}{\partial y}\right)$
      & $\tau_{\theta z}$ & $-\mu\left(\frac{1}{r}\frac{\partial v_z}{\partial \theta} +\frac{\partial v_\theta}{\partial z}\right)$
      & $\tau_{\theta \phi}$ & $-\mu\left(\frac{\sin\theta}{r}\frac{\partial}{\partial \theta}\left(\frac{v_\phi}{\sin\theta}\right) +\frac{1}{r\sin\theta}\frac{\partial v_\theta}{\partial \phi}\right)$
      \\[5pt]\hline
      $\tau_{xz}$ & $-\mu\left(\frac{\partial v_x}{\partial z} +\frac{\partial v_z}{\partial x}\right)$
      & $\tau_{zr}$ & $-\mu\left(\frac{\partial v_r}{\partial z} +\frac{\partial v_z}{\partial r}\right)$
      & $\tau_{\phi r}$ & $-\mu\left(\frac{1}{r\sin\theta}\frac{\partial v_r}{\partial \phi} +r\frac{\partial}{\partial r}\left(\frac{v_\phi}{r}\right)\right)$
      \\[5pt]\hline
      % \multicolumn{2}{|c|}{$\nabla\cdot\bm{v} = \frac{\partial v_x}{\partial x}+\frac{\partial v_y}{\partial y} +\frac{\partial v_z}{\partial z}$}
      % & \multicolumn{2}{c|}{$\nabla\cdot\bm{v} = \frac{1}{r}\frac{\partial r\,v_r}{\partial r}+\frac{1}{r}\frac{\partial v_\theta}{\partial \theta} +\frac{\partial v_z}{\partial z}$}
      % & \multicolumn{2}{c|}{$\nabla\cdot\bm{v} = \frac{1}{r^2}\frac{\partial r^2\,v_r}{\partial r}+\frac{1}{r\sin\theta}\frac{\partial v_\theta\sin\theta}{\partial \theta} +\frac{1}{r\sin\theta}\frac{\partial v_\phi}{\partial \phi}$}
      % \\[5pt]\hline
    \end{tabular}
  }
  \caption{Fourier's law for the heat flux and Newton's law for the stress in several coordinate systems. Please remember that the stress is symmetric, so $\tau_{ij}=\tau_{ji}$.}
\end{table}

\newpage
{\bf Viscous models:}\\*
Power-Law Fluid:
\begin{align}\label{eq:powerlawfluid}
  \left|\tau_{xy}\right| &= k \left|\frac{\partial v_x}{\partial y}\right|^{n}
\end{align}
Bingham-Plastic Fluid:
\begin{align*}
  \frac{\partial v_x}{\partial y} = \begin{cases}
    -\mu^{-1}\left(\tau_{xy}-\tau_0)\right) & \text{if
      $\tau_{xy} > \tau_0$}
    \\
    0 & \text{if $\tau_{xy} \leq \tau_0$}
  \end{cases}
\end{align*}
{\bf Dimensionless Numbers}\\*
\begin{align}
  \text{Re}&=\frac{\rho\,\left\langle v\right\rangle\,D}{\mu}
  &
  \text{Re}_{H}&=\frac{\rho\,\left\langle v\right\rangle\,D_H}{\mu}
  &
  \text{Re}_{MR}&=-\frac{16\,L\,\rho\left\langle
      v\right\rangle^2}{R\,\Delta p}\label{eq:Reynolds}
\end{align}
The hydraulic diameter is defined as $D_H=4\,A / P_w$.

{\bf Single phase pressure drop calculations in pipes:}\\*
Darcy-Weisbach equation:
\begin{align}
  \frac{\Delta p}{L} = -\frac{C_f\,\rho\left\langle v\right\rangle^2}{R}
\end{align}
where $C_f=16/Re$ for laminar Newtonian flow.  For turbulent flow of
Newtonian fluids in smooth pipes, we have the Blasius correlation:
\begin{align*}
  C_f&=0.079\,\text{Re}^{-1/4} & \text{for $2.5\times10^3<\text{Re}<10^5$ and smooth pipes.}
\end{align*}
Otherwise, you may refer to the Moody diagram.
\begin{figure}[h]
  \includegraphics[clip,width=\linewidth]{figures/Moody_diagram}
\end{figure}

Laminar Power-Law fluid:
\begin{align*}
  \dot{V} = \frac{n\,\pi\,R^3}{3\,n+1} \left(\frac{R}{2\,k}\right)^{\frac{1}{n}} \left(-\frac{\Delta p}{L}\right)^{\frac{1}{n}}
\end{align*}
{\bf Two-Phase Flow:}\\*
Lockhart-Martinelli parameter:
\begin{align*}
  X^2=\frac{\Delta p_{liq.-only}}{\Delta p_{gas-only}}
\end{align*}
Pressure drop calculation:
\begin{align*}
  \Delta p_{two-phase} = \Phi^2_{liq.}\,\Delta p_{liq.-only} = \Phi^2_{gas}\,\Delta p_{gas-only}
\end{align*}
Chisholm's relation:
\begin{align*}
  \Phi^2_{gas} &= 1 + c\,X + X^2 &\\
  \Phi^2_{liq.} &= 1 + \frac{c}{X}+\frac{1}{X^2} &
  c &= \begin{cases}
    20& \text{turbulent liquid \& turbulent gas}\\
    12& \text{laminar liquid \& turbulent gas}\\
    10& \text{turbulent liquid \& laminar gas}\\
    5& \text{laminar liquid \& laminar gas}
  \end{cases}
\end{align*}
Farooqi and Richardson expression for liquid hold-up in co-current
flows of Newtonian fluids and air in horizontal pipes:
\begin{align*}
  h &=
  \begin{cases}
    0.186+0.0191\,X & 1 < X < 5\\
    0.143\,X^{0.42} & 5 < X < 50\\
    1/\left(0.97 + 19/X\right) & 50 < X < 500
  \end{cases}
\end{align*}

\begin{figure}[p!]%
  \begin{center}%
    \includegraphics[width=0.65\textwidth,clip]{figures/Hewitt_Taylor}
  \end{center}
  \caption{Hewitt-Taylor flow pattern map for multiphase flows in
    vertical pipes.}
  \begin{center}%
    \includegraphics[width=0.65\textwidth,clip]{figures/Chhabra_richardson_flow_map}
  \end{center}
  \caption{Chhabra and Richardson flow pattern map for horizontal pipes.}
\end{figure}

\pagebreak[0]
{\bf Heat Transfer:}\\*
Stefan-Boltzmann constant $\sigma=5.6703\times10^{-8}$~W/m${}^2$~K${}^4$

{\bf Heat Transfer Dimensionless numbers:}
\begin{align*}
  \text{Nu} &= \frac{h\,L}{k} 
  &
  \text{Pr} &= \frac{\mu\,C_p}{k}
  &
  \text{Gr} &= \frac{g\,\beta
    \left(T_w - T_\infty\right)\,L^3}{\nu^2}
\end{align*}

{\bf Resistances}
\begin{align*}
  Q &= U_T\,A_T\,\Delta T = R_T^{-1}\,\Delta T & Q_{rad.}= \sigma\,\varepsilon\,A\left(T_\infty^4-T_w^4\right) = h_{rad.}\,A\left(T_\infty-T_w\right)
\end{align*}
\begin{center}
  \begin{tabular}{|r|c|c|c|c|}\hline
    & \multicolumn{3}{c|}{Conduction Shell Resistances} & Radiation\\
    & Rect. & Cyl. & Sph. & \\\hline
    & & & & \\
    $R$ & $\displaystyle\frac{X}{k\,A}$ & $\displaystyle\frac{\ln\left(R_{outer}/R_{inner}\right)}{2\,\pi\,L\,k}$ & $\displaystyle\frac{R_{inner}^{-1} - R_{outer}^{-1}}{4\,\pi\,k}$ & $\left[A\,\varepsilon\,\sigma\left(T_\infty^2+T_w^2\right)\left(T_\infty + T_w\right)\right]^{-1}$
    \\[5pt]\hline
  \end{tabular}
\end{center}

{\bf Natural Convection}
\begin{table}[h!]
  \begin{center}
    \begin{tabular}{|p{3cm}|p{3cm}|p{3cm}|}\hline
      $\text{Ra}=\text{Gr}\,\text{Pr}$ & $C$ & $m$\\\hline
      $< 10^{4}$ & 1.36 & 1/5\\
      $10^{4}$--$10^{9}$ & 0.59 & 1/4 \\
      $>10^{9}$ & 0.13 & 1/3\\\hline
    \end{tabular}
    \caption{\label{tab:conv}Natural convection coefficients for isothermal vertical
      plates in the empirical relation $\text{Nu}\approx C\left(\text{Gr}\,\text{Pr}\right)^m$.
      % The values are taken from Table~7-1 in ``Heat Transfer'' by J.\
      % P.\ Holman.
    }
  \end{center}
\end{table}

For isothermal vertical cylinders, the above expressions for
isothermal vertical plates may be used but must be scaled by a factor,
$F$:
\begin{align*}
  F = \begin{cases}
    1 & \text{for }\left(D/H\right) < 35\,\text{Gr}^{-1/4}_H
    \\
    1.3\left[H\,D^{-1}\,\text{Gr}_D^{-1}\right]^{1/4}+1 & \text{for }\left(D/H\right) \ge 35\,\text{Gr}^{-1/4}_H
  \end{cases}
\end{align*}
where $D$ is the diameter and $H$ is the height of the cylinder. The
subscript on $\text{Gr}$ indicates which length is to be used as the
critical length to calculate the Grashof number.

Churchill and Chu expression for natural convection from a
horizontal pipe:
\begin{align*}
  \text{Nu}^{1/2} &= 0.6 + 0.387
  \left\{\frac{\text{Gr}\,\text{Pr}}{\left[1 + \left(0.559 /
          \text{Pr}\right)^{9/16}\right]^{16/9}}\right\}^{1/6} &
  \text{for $10^{-5}<\text{Gr}\,\text{Pr}<10^{12}$}
\end{align*}%

{\bf Forced Convection:}\\*
Laminar flows:
\begin{align*}
  \text{Nu} \approx 0.332\,\text{Re}^{1/2}\,\text{Pr}^{1/3}
\end{align*}
Well-Developed turbulent flows in smooth pipes:
\begin{align*}
  \text{Nu} \approx \frac{(C_f/2)
    \text{Re}\,\text{Pr}}{1.07+12.7(C_f/2)^{1/2}\left(\text{Pr}^{2/3}
      -1\right)}\left(\frac{\mu_b}{\mu_w}\right)^{0.14}
\end{align*}

{\bf Boiling:}\\*
Forster-Zuber pool-boiling coefficient:
\begin{align*}
  h_{nb}=0.00122\frac{k_L^{0.79}\, C_{p,L}^{0.45}\, \rho_L^{0.49}}{\gamma^{0.5}\,\mu_L^{0.29}\,h_{fg}^{0.24}\,\rho_G^{0.24}}\left(T_w - T_{sat}\right)^{0.24}\left(p_w-p_{sat}\right)^{0.75}
\end{align*}

Mostinski correlations: 
\begin{align*}
  h_{nb} &= 0.104\,p_c^{0.69}\,q^{0.7}\left[1.8\left(\frac{p}{p_c}\right)^{0.17}+4\left(\frac{p}{p_c}\right)^{1.2}+10\left(\frac{p}{p_c}\right)^{10}\right]\\
  q_c &=
  3.67\times10^4\,p_c\left(\frac{p}{p_c}\right)^{0.35}\left[1-\frac{p}{p_c}\right]^{0.9}
\end{align*}
({\bf Note}: for the Mostinski correlations, the pressures are in units of bar)\\*
{\bf Condensing:}\\*
Horizontal pipes
\begin{align*}
  h = 0.72
  \left(\frac{k^3\,\rho^2\,g_x\,E_{latent}}{D\,\mu\,\left(T_w-T_\infty\right)}\right)^{1/4}
\end{align*}

{\bf Lumped capacitance method:}\\*
\begin{align*}
  \text{Bi} &= \frac{h\,L_c}{\kappa} & \\
  L_c &= 
  V/A & \text{for $\text{Bi}<0.1$}
\end{align*}

{\bf 1-D Transient Heat Conduction:}\\*
\begin{align*}
   Fo = \frac{\alpha \Delta t}{\left(\Delta x\right)^{2}}
\end{align*}


\begin{align*}
   \theta_{\text{wall}} = \frac{T(x,t)-T_{\infty}}{T_{i}-T_{\infty}}= A_{1}e^{-\lambda_{1}^{2}\tau}\cos{\left(\frac{\lambda_{1}x}{L}\right)},\;\;\;\;\; \theta_{\text{cyl}} = \frac{T(r,t)-T_{\infty}}{T_{i}-T_{\infty}}= A_{1}e^{-\lambda_{1}^{2}\tau}\mathbf{J_{0}}\left(\frac{\lambda_{1}r}{r_{0}}\right)
\end{align*}

\begin{align*}
   \theta_{\text{sph}} = \frac{T(r,t)-T_{\infty}}{T_{i}-T_{\infty}}= A_{1}e^{-\lambda_{1}^{2}\tau}\frac{\sin{\left(\frac{\lambda_{1}r}{r_{0}}\right)}}{\frac{\lambda_{1}r}{r_{0}}}
\end{align*}


\begin{align*}
   \left(\frac{\mathcal{Q}}{\mathcal{Q}_{\text{max}}}\right)_{\text{wall}} = 1 - \theta_{0,\text{wall}}\frac{\sin{\lambda_{1}}}{\lambda_{1}},\;\;\left(\frac{\mathcal{Q}}{\mathcal{Q}_{\text{max}}}\right)_{\text{cyl}} = 1- 2\theta_{0,\text{cyl}}\frac{\mathbf{J_{1}}}{\lambda_{1}}
\end{align*}

\begin{align*}
    \left(\frac{\mathcal{Q}}{\mathcal{Q}_{\text{max}}}\right)_{\text{sph}} = 1 - 3\theta_{0,\text{sph}}\frac{\sin{\lambda_{1}}-\lambda_{1}\cos{\lambda_{1}}}{\lambda_{1}^{3}}
\end{align*}

\begin{figure}[h!]%
  \begin{center}%
    \includegraphics[width=0.7\textwidth,clip]{figures/BaselFunctionTable}
  \end{center}
  \caption{Coefficients for the 1D transient equations.}
\end{figure}


\pagebreak[0]

{\bf Overall Heat Transfer Coefficient:}\\*
\begin{align*}
  \dot{\mathcal{Q}} =  \frac{\Delta T}{\mathcal{R}} = U A \Delta T  = U_{i}A_{i}\Delta T = U_{o}A_{o}\Delta T
\end{align*}

\begin{align*}
  \mathcal{R} = R_{i} + R_{\text{wall}} + R_{o} = \frac{1}{h_{i}A_{i}} + \frac{\ln{D_{o}/D_{i}}}{2\pi \kappa L} + \frac{1}{h_{o}A_{o}} 
\end{align*}

{\bf Fouling Factor:}*
\begin{align*}
   \mathcal{R} = \frac{1}{h_{i}A_{i}} + \frac{R_{\text{f,i}}}{A_{i}} + R_{\text{wall}} + \frac{R_{\text{f,o}}}{A_{o}} + \frac{1}{h_{o}A_{o}}
\end{align*}

{\bf LMTD Method:}\\*
\begin{align*}
   \dot{\mathcal{Q}} = U A_{s} \Delta T_{\text{lm}}\;\text{ with }\;  \Delta T_{\text{lm}} = \frac{\left(T_{\text{hot,out}} - T_{\text{cold,out}}\right)-\left(T_{\text{hot,in}} - T_{\text{cold,in}}\right)}{\ln{\left(\frac{T_{\text{hot,out}} - T_{\text{cold,out}}}{T_{\text{hot,in}} - T_{\text{cold,in}}}\right)}}.
\end{align*}

\begin{figure}[h!]%
  \begin{center}%
    \includegraphics[width=0.8\textwidth,clip]{figures/NTU}
  \end{center}
  \caption{NTU relations}
\end{figure}


\pagebreak[0]
{\bf Diffusion Dimensionless Numbers}\\*
\begin{align*}
  \text{Sc} &= \frac{\mu}{\rho\,D_{AB}} & \text{Le} &= \frac{k}{\rho\,C_p\,D_{AB}}
\end{align*}
\pagebreak[0]
{\bf Diffusion}\\*
General expression for the flux:
\begin{align*}
  \bm{N}_{A} = \bm{J}_A + x_A \sum_B \bm{N}_B
\end{align*}
Fick's law:
\begin{align*}
  \bm{J}_{A} = - D_{AB}\,\nabla C_A
\end{align*}
Stefan's law:
\begin{align*}
  N_{s,r} = -D \frac{c}{1-x}\frac{\partial x}{\partial r}
\end{align*}
{\bf Misc}\\*
\begin{align*}
  P\,V &= n\,R\,T & 
  R&\approx8.314598~\text{J K}^{-1}\text{ mol}^{-1} 
\end{align*}
\end{datasheet}

% \includepdf{BaselFunctionTable.pdf}
\paperend
\end{document}
