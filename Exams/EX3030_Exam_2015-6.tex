
%\documentclass[calculator,steamtables,datasheet,solutions]{exam_newMarcus}
\documentclass[calculator,fluidstables,allquestions,datasheet]{exam}

% The full list of class options are
% calculator : Allows approved calculator use.
% datasheet : Adds a note that data sheet are attached to the exam.
% handbook : Allows the use of the engineering handbook.
% resit : Adds the resit markings to the paper.
% sample : Adds conspicuous SAMPLE markings to the paper
% solutions : Uses the contents of \solution commands (and \solmarks) to generate a solution file

\usepackage{pdfpages} 
%\usepackage{lscape,comment}
 
\coursecode{EX3029}%%

\examtime{09.00--12.00}%
\examdate{03}{12}{2015}% 
\examformat{Candidates must attempt \textit{all} questions.}

\newcommand{\frc}{\displaystyle\frac}
\newcommand{\br}[1]{\!\left( #1 \right)}
\newcommand{\abs}[1]{\left| #1 \right|}
\newcommand{\fracd}[2]{\frac{\mathrm{d} #1}{\mathrm{d} #2}}
\newcommand{\fracp}[2]{\frac{\partial #1}{\partial #2}}
\renewcommand{\d}[1]{\mathrm{d} #1 } 
\newcommand{\Ma}{\mathrm{M\!a}} 

 

\begin{document}

%%%
%%% Question 01 
%%%
\begin{question}
%
\begin{enumerate}
%
% (Holman 10.23)
\item\label{HE_Exam1} An engineer is designing a new system in which 230 kg/h of water is heated from 35$^{\circ}$C to 93$^{\circ}$C by oil initially at 175$^{\circ}$C. The mass flow rate of oil is also 230 kg/h. Two double-pipe heat exchangers are available:
\begin{itemize}
   \item Heat Exchanger 1: Overall heat transfer coefficient $\left(U_{1}\right)$ of 570 W/$\left(\text{m}^{2}.^{\circ}\text{C}\right)$ with superficial area $\left(A_{1}\right)$ of 0.47 m$^{2}$, and;
   \item Heat Exchanger 2: $U_{2}$ = 370 W/$\left(\text{m}^{2}.^{\circ}\text{C}\right)$ and $A_{2}$ = of 0.94 m$^{2}$.
\end{itemize}
Which exchanger should be used? Why?
Given C$_{p,\text{oil}}$ = 2.10 kJ/$\left(\text{kg.}^{\circ}\text{C}\right)$ and C$_{p,\text{water}}$ = 4.18 kJ/$\left(\text{kg.}^{\circ}\text{C}\right)$
%
\solution{hjhj} 

%Holman (4.40)
\item\label{Transient_Exam1b} A plate of stainless steel (18$\%$ Cr, 8$\%$ Ni) has a thickness of 3.0 cm and is initially uniform in temperature at 500$^{\circ}$C. The plate is suddenly exposed to a convection environment on both sides at 40$^{\circ}$C with h = 150 W/$\left(\text{m}^{2}.^{\circ}\text{C}\right)$. Calculate the times for the centre and face temperatures to reach 120$^{\circ}$C. Given: (a) thermal conductivity coefficient: 16.3 W/$\left(\text{m}.^{\circ}\text{C}\right)$; thermal diffusivity: 0.44$\times$10$^{-5}$ m$^{2}$/s.


% Holman (4.21)
%\item\label{Transient_Exam1} A thick concrete wall having a uniform temperature of 54$^{\circ}$C is suddenly subjected to an airstream at 10$^{\circ}$C. The convective heat transfer coefficient is 10 W/$\left(\text{m}^{2}.^{\circ}\text{C}\right)$. Calculate the temperature in the concrete slab at a depth of 7 cm after 30 min. Given the conductive heat transfer coefficient of concrete $\left(\kappa_{\text{concrete}}\right)$ of 1.37 W/$\left(\text{m.}^{\circ}\text{C}\right)$.

% Jeff
\item\label{Transient_Exam2_FDM} Heavy oil at 150$^{\circ}$C is pumped into a storage tank before being transported to distillation columns. The tank is insulated with a layer of polyisocyanurate of 10 cm thick. The tank's wall is at 150$^{\circ}$C and the initial temperature of the insulation layer is 20$^{\circ}$C. Assuming that the problem can be considered as 1-D, calculate the temperature profile of the insulated layer, $T\left(\underline{x},t\right)$, at $t = 2$ seconds with $\underline{x}=\left[0.0, 2.5, 5.0, 7.5, 10.0\right]$. The outer layer of the insulation material is subjected to the environment with temperature of 15$^{\circ}$C and convective heat transfer coefficient of 10 W/$\left(\text{m}^{2}.^{\circ}\text{C}\right)$. Given for polyisocyanurate insulation layer: (a) conductive heat transfer coefficient: 0.054 W/$\left(\text{m.}^{\circ}\text{C}\right)$; (b) heat capacity at constant pressure: 1 kJ/$\left(\text{kg.}^{\circ}\text{C}\right)$; (c) density: 550 kg/m$^{3}$.

% Holman (10.36)
\item\label{HE_Exam2} A shell-and-tube heat exchanger has condensing steam at 100$^{\circ}$C in the shell side with one shell pass. Two tube passes are used with air in the tubes entering at 10$^{\circ}$C. The total surface area of the heat exchanger is 30 m$^{2}$ and the overall heat transfer coefficient may be taken as 150 W/$\left(\text{m}^{2}.^{\circ}\text{C}\right)$. If the effectiveness of the heat exchanger is 85$\%$, calculate the total heat transfer rate?


\end{enumerate}

\end{question}


\end{document}
