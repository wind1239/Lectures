\documentclass[14pt,twoside]{report}

\usepackage{amsfonts,amsmath,amssymb,stmaryrd,indentfirst}
\usepackage{epsfig,graphicx,times,psfrag}
\usepackage{natbib}
\usepackage{pdfpages,enumitem}

%%% Rotating Page
\usepackage{pdflscape}
\usepackage{afterpage}
%\usepackage{capt-of}% or use the larger `caption` package
%\usepackage{lipsum}% dummy text

\newcommand\blankpage{%
    \null
    \thispagestyle{empty}%
    \addtocounter{page}{-1}%
    \newpage}

\usepackage[pdftex,bookmarks,colorlinks=true,urlcolor=blue,citecolor=blue]{hyperref}

%\usepackage{fancyhdr} %%%%
%\pagestyle{fancy}%%%%
\pagestyle{empty}
\def\newblock{\hskip .11em plus .33em minus .07em}

\setlength\textwidth      {16.5cm}
\setlength\textheight     {22.0cm}
\setlength\oddsidemargin  {-0.3cm}
\setlength\evensidemargin {-0.3cm}

\setlength\headheight{0in} 
\setlength\topmargin{0.cm}
\setlength\headsep{1.cm}
\setlength\footskip{1.cm}
\setlength\parskip{0pt}

\newcommand{\frc}{\displaystyle\frac}
\newcommand{\ie}{{\it i.e.,}}
\newcommand{\eg}{{\it e.g.,}}
\newcommand{\wrt}{{\it wrt }}

%%%
%%% space between lines
%%%
\renewcommand {\baselinestretch}{1.5}

\begin{document}



%%%%%% 
%%%%%%
%%%%%%


%\lipsum % Text before
\afterpage{%
    \clearpage% Flush earlier floats (otherwise order might not be correct)
    \thispagestyle{empty}% empty page style (?)
    \begin{landscape}% Landscape page
        \centering % Center table


\Huge{MEng/BEng Study Assessment -- Winter Report (EG4013/EG4014)}\\
\huge{(Review + Feedback)}\\
\huge{January 2018}
%\end{center}
\normalsize

\bigskip

%\begin{center}
\begin{tabular}{||c| c c c |c| c||}
\hline\hline
                           & {\bf Presentation and Style} & {\bf Technical Content}                   & {\bf Critical Reasonning} & {\bf Discipline} & {\bf Averaged}   \\
                           &  {\bf $\left(\%\right)$}     & {\bf and Merit $\left(\%\right)$}         & {\bf $\left(\%\right)$}   &                  & {\bf CGS Mark}   \\
\hline
Joseph J. Moyes           &                              &                                           &                           &   MEng-ChemEng   &                  \\
Ehsan F. Zanjani           &        71                    &               50                          &         --                &   BEng-PetEng    &   56.30 (13/C2)  \\
Motshidisi Mmadintsi       &                              &                                           &                           &   MEng-ChemEng   &                  \\
Sean D. Clark              &                              &                                           &                           &   MEng-ChemEng   &                  \\
Andrew Robertson           &        67                    &               65                          &         68                &   MEng-ChemEng   &   66.30 (16/B2)  \\
\hline\hline
\end{tabular}
%\end{center}
    \end{landscape}
    \clearpage% Flush page
}



%\lipsum % Text after

\vfill

\clearpage


%%%%%% 
%%%%%%
%%%%%%


\noindent{\bfseries\large EG4014 -- BEng Thesis \hfill May, 2018}

\bigskip

\begin{center}
  {\Large Review of the BEng Thesis `Numerical Simulation of CO$_{2}$-EOR in Heterogeneous Reservoir Simulations' by Ehsan F. Zanjani}
\end{center}


%\noindent
%    {\Large Comments from Marker 1:}
    
The dissertation aims to investigate CO$_{2}$-EOR processes and in particular the impact of viscous flow instabilities in oil recovery. Mr Zanjani undertook a limited literature review of some EOR-related technologies and performed numerical experiments using an open-source flow simulator to assess viscous fingering during oil recovery. The dissertation encompasses two main subject areas within Petroleum Engineering: fluid dynamics through porous media, production technology and geophysics.

The manuscript is relatively well-written with a small number of typos and unrevised sentences. Few sentences are confusing and disconnected with no clear objectives and inter-connectivities. Most of all, the dissertation is mostly well-structured with clear division and linkages between chapters, sections and paragraphs, leading to a smooth reading.

Although fluid mechanics (\ie flow instabilities, methods and algorithms for flow dynamics) and geophysics (\ie heterogeneity) were the main focus of the thesis, they were partially explained and analysed. Numerical simulations were performed to investigate the impact of viscous flow instabilities in oil recovery. Reasoning for the simulations were not explained, and simulation conditions (\ie boundary and initial conditions) were incomplete. Overall, analysis of results were superficial with little linkage with the theory. A few specific comments,
\begin{enumerate}
%
\item The main aim of {\it Abstracts} is to briefly describe the work undertaken by the author (\ie the thesis' content). In general {\it Abstracts} are divided in 4 parts: (i) motivation, (ii) aims and objectives, (iii) summary of the main procedures / techniques / technologies (optional) and (iv) main findings (or summary of the work). The current {\it Abstract} encompass (i-ii) and (iv).
%
\item The main {\it Introduction} section usually has the same (but more in-depth and descriptive) four parts of the {\it Abstract} and a brief summary of the remaining of the work. In addition, it is \underline{always} expected a few clear statements -re main background (thus recent innovations related to the main topic), initial literature review and, most of all, technological / scientific gaps in the current understanding. Also, it is expected a {\it summary of the remaining sections} at the end of the {\it Introduction}.  Current {\it Introduction} covered only (i) and (ii) above. `Literature review` was part of the second chapter.
%
\item The aim of Literature Review Sections/Chapters is to introduce the main fundamental (i.e., theoretical) aspects of the work and to assess (with critical analysis) previous academic and/or industrial developments on the main subject areas. In summary, literature review should focus on 3-5 subjects of the project main topic and give an overview of past and current work ({\it state-of-the-art}) on them. Most of all, the section/chapter should highlight current gaps in specific knowledge. In this thesis, literature review, embedded in Chapters 2 was extremely limited and covered few topics on oil recovery (overview of past and current practices in industry), CO$_{2}$-EOR (general technology) and flow instabilities (basic theory).
%
\item Some references have missing fields. Regardless of the chosen citation style (e.g., ACS, AIP, AMS, IEEE, AIAA, etc) any reference {\bf must} contain the following fields: 
\begin{enumerate}
\item For journal papers: Authors, Paper Tittle, Journal Name, Volume, Pages, Year of publication;
\item For books: Authors, Book Tittle, Publisher, Year or Edition;
\item For book chapters: Authors, Chapter Tittle, Book Tittle, Editors, Publisher, Year or Edition;
\item For conference papers: Authors, Paper Tittle, Conference Tittle, Place (Country and/or City) where the conference was held, Year of the conference;
\item For reports, private communications and Lecture Notes: Authors, Tittle, Place issued (Country and/or City and Institution where the document was originated), Year;
\item For PhD Thesis and MSc Dissertations: Author, Tittle, Institution (University and Department/School), Year.
\end{enumerate}  
Thus, for example:
\begin{enumerate}[label={[\arabic*]}]
\item P.L. Houtekamer and L. Mitchell, `Data Assimilation Using an Ensemble Kalman Filter Technique', {\it Monthly Weather Review}, 126:796-811, 1998.
\item K. Pruess, `Numerical Modelling of Gas Migration at a Proposed Repository for Low and Intermediate Level Nuclear Wastes', Technical Report LBL-25413, Lawrence Berkeley Laboratory, Berkeley (USA), 1990.
\item K. Aziz, A. Settari, {\it Fundamentals of Reservoir Simulation}, Elsevier Applied Science Publishers, New York (USA), 1986.
\item R.B. Lowrie, `Compact higher-Order Numerical Methods for Hyperbolic Conservation Laws', PhD Thesis, Department of Aerospace Engineering and Scientific Computing, University of Michigan (USA), 1996.
\end{enumerate}
%
\item Poor description and analysis of viscous flow instabilities and the relationship with CO$_{2}$-EOR -- main topic of the work;
%
\item Simulations were poorly described and analysis of results were superficial (\ie not backed by theory listed in previous chapters);
%
\item Numbering of equations were not consistent.
% 
\end{enumerate}
In summary, the work undertook by the Mr Zanjani was extensive and complex but he managed to produce a good dissertation. The topic is very relevant for petroleum and environmental sectors, and each sub-topic has been the focus of several academic- and industrial-based studies worldwide with clear cross-fertilisation with physics (thermodynamics, fluid mechanics etc) and petroleum engineering (recovery technologies, thermal sciences, optimisation etc). The student demonstrated that he had a good understanding of the main technologies involved in this project.

\clearpage




%%%%%% 
%%%%%%
%%%%%%


\noindent{\bfseries\large EG4014 -- MEng Thesis \hfill May, 2018}

\bigskip

\begin{center}
  {\Large Review of the MEng Thesis `Separation of Biofuel Fermentation Products: Carbon Materials' by Andrew Robertson}
\end{center}
    
The dissertation investigates synthesis and separation (via adsorption mechanisms) of bio-(n-butanol). Mr Robertson undertook a limited literature review of few aspects related to bio-(n-butanol) syntesis via fermentation (and industrial production), separation processes and data analysis. This supported experiments on n-butanol syntesis and adsorption separation that were reported in Chapters 4-6. The dissertation encompasses three main subject areas within Chemical Engineering: organic chemistry synthesis, microbiology and biochemistry and surface physics (\ie adsorption mechanisms).

The manuscript is relatively well-written with a small number of typos and unrevised sentences. Few sentences are confusing and disconnected with no clear objectives and inter-connectivities. Most of all, the dissertation is mostly well-structured with clear division and linkages between chapters, sections and paragraphs, leading to a smooth reading.

Although microbiology, biochemistry and surface physics (\ie mechanisms and methods for production and separation of organic solutions) were the main focus of the thesis, they were partially explained and analysed. Experiments were performed to investigate production and conversion of n-butanol, however reasoning for the experiments and analysis of results were limited with little linkage with the theory. A few specific comments,
\begin{enumerate}
%
\item The main aim of {\it Abstracts} is to briefly describe the work undertaken by the author (\ie the thesis' content). In general {\it Abstracts} are divided in 4 parts: (i) motivation, (ii) aims and objectives, (iii) summary of the main procedures / techniques / technologies (optional) and (iv) main findings (or summary of the work). The current {\it Abstract} encompass (i, iii-iv) and (ii) (partially).
%
\item The main {\it Introduction} section usually has the same (but more in-depth and descriptive) four parts of the {\it Abstract} and a brief summary of the remaining of the work. In addition, it is \underline{always} expected a few clear statements -re main background (thus recent innovations related to the main topic), initial literature review and, most of all, technological / scientific gaps in the current understanding. Also, it is expected a {\it summary of the remaining sections} at the end of the {\it Introduction}.  Current {\it Introduction} covered only (i) and (ii) above. `Literature review` was part of the second chapter.
%
\item Dissertations and thesis are always divided into chapters $\rightarrow$ sections, whereas reports are divided into sections. Chapters must always start in a new page.
%
\item The aim of Literature Review Sections/Chapters is to introduce the main fundamental (i.e., theoretical) aspects of the work and to assess (with critical analysis) previous academic and/or industrial developments on the main subject areas. In summary, literature review should focus on 3-5 subjects of the project main topic and give an overview of past and current work ({\it state-of-the-art}) on them. Most of all, the section/chapter should highlight current gaps in specific knowledge. In this thesis, literature review, embedded in Chapters 2 was extremely limited and covered few topics on bio-(n-butanol) syntesis via fermentation, separation processes and data analysis.
%
\item Some references have missing fields. Regardless of the chosen citation style (e.g., ACS, AIP, AMS, IEEE, AIAA, etc) any reference {\bf must} contain the following fields: 
\begin{enumerate}
\item For journal papers: Authors, Paper Tittle, Journal Name, Volume, Pages, Year of publication;
\item For books: Authors, Book Tittle, Publisher, Year or Edition;
\item For book chapters: Authors, Chapter Tittle, Book Tittle, Editors, Publisher, Year or Edition;
\item For conference papers: Authors, Paper Tittle, Conference Tittle, Place (Country and/or City) where the conference was held, Year of the conference;
\item For reports, private communications and Lecture Notes: Authors, Tittle, Place issued (Country and/or City and Institution where the document was originated), Year;
\item For PhD Thesis and MSc Dissertations: Author, Tittle, Institution (University and Department/School), Year.
\end{enumerate}  
Thus, for example:
\begin{enumerate}[label={[\arabic*]}]
\item P.L. Houtekamer and L. Mitchell, `Data Assimilation Using an Ensemble Kalman Filter Technique', {\it Monthly Weather Review}, 126:796-811, 1998.
\item K. Pruess, `Numerical Modelling of Gas Migration at a Proposed Repository for Low and Intermediate Level Nuclear Wastes', Technical Report LBL-25413, Lawrence Berkeley Laboratory, Berkeley (USA), 1990.
\item K. Aziz, A. Settari, {\it Fundamentals of Reservoir Simulation}, Elsevier Applied Science Publishers, New York (USA), 1986.
\item R.B. Lowrie, `Compact higher-Order Numerical Methods for Hyperbolic Conservation Laws', PhD Thesis, Department of Aerospace Engineering and Scientific Computing, University of Michigan (USA), 1996.
\end{enumerate}
%
\end{enumerate}
In summary, the work undertook by the Mr Robertson was extensive and complex but he managed to produce a very good dissertation. The topic is very relevant for chemistry and biotechnology sectors, and each sub-topic has been the focus of several academic- and industrial-based studies worldwide with clear cross-fertilisation with physics (thermodynamics, fluid mechanics, surface science etc) and biochemical engineering (bio-synthesis, separation processes, biochemistry etc). The student demonstrated that he had a very good understanding of the main technologies involved in this project.

\clearpage




%%%%%% 
%%%%%%
%%%%%%


\noindent{\bfseries\large Disserta\c{c}\~ao de MSc \hfill Abril 2018}

\bigskip

\begin{center}
  {\Large Revis\~ao da Disserta\c{c}\~ao de MSc `Din\^amica de Fluidos Computacional na Modelagem Termohidr\'aulica Real\'{\i}stica de Sistemas Nucleares de Temperatura muito Elevada em Regimes com Perda de Refrigerante' -- U.G. Moreira}
\end{center}

A disserta\c{c}\~ao de mestrado do candidado U.G. Moreira investiga o escoamento monof\'asico em reatores HTR-10 de bola (`pebble bed reactors') em regime transiente sob situa\c{c}\~oes de acidentes. A disserta\c{c}\~ao \'e bem estruturada e muito bem escrita com poucos erros de digita\c{c}\~ao e com apenas algumas senten\c{c}as que precisariam de revis\~ao (veja c\'opia anexada). Resumos sobre alguns dos principais t\'opicos em c\'alculo de reatores foram feitos pelo candidato, entre eles: (a) `evolu\c{c}\~ao de reatores nucleares' e situa\c{c}\~oes de acidentes; (b) equa\c{c}\~oes de escoamento de fluidos e; (c)  m\'etodos de solu\c{c}\~ao num\'erica.

A disserta\c{c}\~ao \'e dividida em 3 partes: (a) introdu\c{c}\~ao e revis\~ao bibliogr\'afica; (b) defini\c{c}\~ao do problema num\'erico e; (c) an\'alise de resultados. A parte (a) \'e bem definida e cobre os principais aspectos do trabalho. Defini\c{c}\~ao do problema num\'erico \'e um pouco superficial, \ie n\~ao se dedica especificamente a nenhum dos temas principais do trabalho de pesquisa (\eg mecanismos de transfer\^encia de calor, impactos de modelos de turbulencia na solu\c{c}\~ao do problema, etc).
\begin{enumerate}
   \item O principal objetivo do `Resumo/Abstract' \'e uma descri\c{c}\~ao sucinta do trabalho de pesquisa. Em geral, `Abstracts' s\~ao divididos em 4 partes: (i) motiva\c{c}\~ao para o trabalho; (ii) objetivo espec\'{\i}fico do trabalho; (iii) resumo dos principais m\'etodos e/ou t\'ecnicas usadas no trabalho de pesquisa (opcional) e; (iv) sum\'ario das conclus\~oes principais. O `Abstract' da disserta\c{c}\~ao cobre (i), (iii) e (iv) (em parte);
   \item Objetivo Espec\'{\i}fico e Objetivos Gerais: Trabalhos cient\'{\i}ficos precisam ter um alvo (`target ou aim') claro e conciso. No terceiro par\'agrafo da pagina 3, o objetivo \'e extremamente gen\'erico e precisa ser modificado, \eg impacto da taxa de escoamento do fluido refrigerante no perfil de temperatura no reator de bolas (PBR);
   \item Mecanismos de transfer\^encia de calor: O trabalho \'e eminentemente sobre processos de transfer\^encia de calor em PBR. Seria adequado discutir estes processos (condu\c{c}\~ao, convec\c{c}\~ao e radia\c{c}\~ao t\'ermica);
   \item Modelos de turbul\^encia: Grande parte da din\^amica associada a transfer\^encia de calor \'e devido ao escoamanto turbulento entre as esferas. Est\'a claro que o assunto foi alvo de outros estudos (dentro do grupo de pesquisa), mas ainda assim necess\'ario a inclus\~ao dos mesmos (\eg como resumo);
   \item Avalia\c{c}\~ao de solu\c{c}\~oes independentes da malha foi abordado rapidamente no in\'{\i}cio do Cap\'{\i}tulo 5. Seria interessante incluir esta metodologia -- deixando claro que esta ana\'alise foi feita com as vari\'aveis fundamentais (\ie velocidades e press\~ao);
   \item Sobre as refer\^encias bibliogr\'aficas  e cita\c{c}\~oes:
      \begin{enumerate}
        \item Cita\c{c}\~oes no corpo do texto precisam ser consistentes, \ie ou todas mai\'usculas ou todas min\'usculas;
        \item Est\~ao faltando `campos obrigat\'orios na grande parte das refer\^encias e, em v\'arias delas, n\~ao h\'a distin\c{c}\~ao entre artigos em jornais e/ou confer\^encias, relat\'orios (internos ou externos), livros, cap\'{\i}tulo de livros etc. Independentemente da escolha do estilo de cita\c{c}\~oes (\eg ACS, AIP, AMS, IEEE, AIAA, etc) a se\c{c}\~ao de refer\^encias bibliogr\'aficas {\bf deve} conter os seguintes campos: 
                      \begin{enumerate}
                         \item Artigos em jornais: autores, t\'{\i}tulo do artigo, nome do jornal, volume, p\'aginaas, ano de publica\c{c}\~ao;
                         \item Livros: autores, t\'{\i}tulo do livro, `publisher`, ano de publica\c{c}\~ao ou edi\c{c}\~ao;
                         \item Cap\'{\i}tulo de livros: autores, t\'{\i}tulo do cap\'{\i}tulo, t\'{\i}tulo do livro, editores, `publisher', ano de publica\c{c}\~ao ou edi\c{c}\~ao;
                         \item Artigos em confer\^encias: autores, t\'{\i}tulo do artigo, nome da confer\^encia, local (pa\'{\i}s e/ou cidade), ano em que a confer\^encia foi realizada;
                         \item Relat\'orios, Comunica\c{c}\~oes privativas e Notas de Aulas: autores, t\'{\i}tulo do documento, local (pa\'{\i}s e/ou cidade e institui\c{c}\~ao) onde o documento foi originado, ano de emiss\~ao do documento; 
                         \item Teses de Doutorado e Disserta\c{c}\~oes de Mestrado: autores, t\'{\i}tulo do trabalho, institui\c{c}\~ao onde o trabalho foi produzido, ano da defesa.
                      \end{enumerate}  
                      Assim, por exemplo:
                         \begin{enumerate}[label={[\arabic*]}]
                            \item P.L. Houtekamer and L. Mitchell, `Data Assimilation Using an Ensemble Kalman Filter Technique', {\it Monthly Weather Review}, 126:796-811, 1998.
                            \item K. Pruess, `Numerical Modelling of Gas Migration at a Proposed Repository for Low and Intermediate Level Nuclear Wastes', Technical Report LBL-25413, Lawrence Berkeley Laboratory, Berkeley (USA), 1990.
                            \item K. Aziz, A. Settari, {\it Fundamentals of Reservoir Simulation}, Elsevier Applied Science Publishers, New York (USA), 1986.
                            \item R.B. Lowrie, `Compact Higher-Order Numerical Methods for Hyperbolic Conservation Laws', PhD Thesis, Department of Aerospace Engineering and Scientific Computing, University of Michigan (USA), 1996.
                         \end{enumerate}
%
                \end{enumerate}

                \end{enumerate} 

\clearpage


%%%%%% 
%%%%%%
%%%%%%


%\lipsum % Text before
\afterpage{%
    \clearpage% Flush earlier floats (otherwise order might not be correct)
    \thispagestyle{empty}% empty page style (?)
    \begin{landscape}% Landscape page
        \centering % Center table


\Huge{MEng/BEng Study Assessment -- Winter Report (EG4013/EG4014)}\\
\huge{(Review + Feedback)}\\
\huge{January 2018}
%\end{center}
\normalsize

\bigskip

%\begin{center}
\begin{tabular}{||c| c c c |c| c||}
\hline\hline
                           & {\bf Presentation and Style} & {\bf Technical Content} & {\bf Technical Merit} & {\bf Discipline} & {\bf Averaged}  \\
                           &                              &                         &                       &                  & {\bf CGS Mark}  \\
\hline
Joseph J. Moyesn           &         20.50                &         20.33           &       20.80           &   MEng-ChemEng   &      21 (A2)    \\
Ehsan F. Zanjani           &         13.00                &         11.00           &        8.20           &   BEng-PetEng    &      10 (D2)    \\
Motshidisi Mmadintsi       &         13.00                &         11.33           &       14.60           &   MEng-ChemEng   &      14 (C1)    \\
Sean D. Clark              &         19.50                &         18.33           &       18.40           &   MEng-ChemEng   &      18 (A5)    \\
\hline\hline
\end{tabular}
%\end{center}
    \end{landscape}
    \clearpage% Flush page
}



%\lipsum % Text after

\vfill

\clearpage

%%%%%% 
%%%%%%
%%%%%%


\noindent{\bfseries\large EG4013/14 -- B/MEng Winter Report \hfill January, 2018}

\bigskip

\begin{center}
  {\Large Review of the B/MEng Winter Report `Technoeconomic Comparison and Feasibility Study of Power Plants for Electrical Power Generation from Enhanced Geothermal Systems in Great Britain' by Joseph J. Moyes}
\end{center}


The report describes Mr Moyes's winter research on coupled thermodynamics and optimisation methods and technologies for geothermal process integration, and in particular his initial investigation on energy and exergy analysis and applications with economic feasibility of thermal systems. A comprehensive overview of the main topics on geothermal and economics engineering was undertaken by Mr Moyes including, (a) thermodynamic cycles; (b) exergy-energy analysis, and ; (c) general optimisation for engineering systems.  


\begin{enumerate}
%
    \item {\bf Presentation and Style of Writing:}
                \begin{enumerate}
                   \item The report is well-written with few typos and unrevised sentences. Most of all, the report has a very coherent and logical structured and linkages between sections, leading to a smooth reading;
                   \item References are consistent and well designed. A few references need to be revised as there is no indication of their nature (article, book chapter, internal report, personal notes etc); 
                   %\item Figures, Tables and Equations must be referenced and fully explained in the main text. They can not be 'floating'.
                   \item Authors \underline{must} avoid use {\it colloquial (informal / personal)} writing. Also, try to avoid long sentences.
                   %\item Dissertations and thesis are always divided into chapters $\rightarrow$ sections, whereas reports are divided into sections.
                \end{enumerate}
%
    \item {\bf Technical Contents:}
                \begin{enumerate}
                   \item The main aim of Abstracts is to briefly describe the work undertaken by the author. In general Abstracts are divided in 4 parts: (i) motivation, (ii) main objectives, (iii) summary of the main procedures / techniques / technologies (optional) and (iv) main findings. The current Abstract encompasses all 4 parts;
                   \item The Introduction section/chapter usually has the same (but more in-depth and descriptive) four parts of the Abstract and a brief summary of the remaining of the work. In addition, it is always expected a few clear statements -re main background (thus recent innovations related to the main topic), initial literature review and, most of all, technological / scientific gaps in the current understanding. Also, it is expected a summary of the remaining sections at the end of the Introduction. In this report, the Introduction section covered all four expected parts. `Literature review` was mostly covered in the remaining of the report; 
                   \item The aim of Literature Review Sections is to introduce the main fundamental (i.e., theoretical) aspects of the work and to assess (with critical analysis) previous academic and/or industrial developments on the main subject areas. In summary, literature review should focus on 3-5 subjects of the project main topic and give an overview of past and current work (state-of-the-art) on them. Most of all, the section should highlight current gaps in specific knowledge;
                     
                   The literature review undertaken in this report was very thorough {\it wrt} geothermal systems, exergy balance and  technoeconomics analys, but limited on the energy integration part;
                   \item The work plan is appropriate for the project and seems realistic;
                \end{enumerate}
%
    \item {\bf Technical Merit:}

                    Mr Moyes undertook an excellent literature review (with critical analysis) of the topics for his project and designed a demanding but fair program of work for the Spring. Specific comments can be seen in the annotated copy in attachment.
%
\end{enumerate}

\clearpage


%%%%%% 
%%%%%%
%%%%%%


\noindent{\bfseries\large EG4013/14 -- B/MEng Winter Report \hfill January, 2018}

\bigskip

\begin{center}
  {\Large Review of the B/MEng Winter Report `Optimisation of Energy Systems in Smart Cities' by Sean D. Clark}
\end{center}

The report describes Mr Clark's winter research on energy utilisation and optimisation for sustainable communities -- {\it Smart Citites}. Mr Clark undertook an insightful preliminary investigation on energy sustainability towards smart cities focusing on engineering and city planning strategies to ensure optimal energy savings in non-centralised power production systems. A comprehensive overview of the main topics on optimal design and utilisation of urban energy systems was undertaken by Mr Clark including, (a) power thermodynamic cycles and (b) sustainable environments.  


\begin{enumerate}
%
    \item {\bf Presentation and Style of Writing:}
                \begin{enumerate}
                   \item The report is well-written with few typos and unrevised sentences. Most of all, the report has a very coherent and logical structured and linkages between sections, leading to a smooth reading;
                   \item References are consistent and well designed. A couple of references need to be revised as there is no indication of their nature (article, book chapter, internal report, personal notes etc); 
                   %\item Figures, Tables and Equations must be referenced and fully explained in the main text. They can not be 'floating'.
                   \item Authors \underline{must} avoid use {\it colloquial (informal / personal)} writing. Also, try to avoid long sentences.
                   %\item Dissertations and thesis are always divided into chapters $\rightarrow$ sections, whereas reports are divided into sections.
                \end{enumerate}
%
    \item {\bf Technical Contents:}
                \begin{enumerate}
                   \item The main aim of Abstracts is to briefly describe the work undertaken by the author. In general Abstracts are divided in 4 parts: (i) motivation, (ii) main objectives, (iii) summary of the main procedures / techniques / technologies (optional) and (iv) main findings. The current Abstract encompasses (i) and (iii) (in part);
                   \item The Introduction section/chapter usually has the same (but more in-depth and descriptive) four parts of the Abstract and a brief summary of the remaining of the work. In addition, it is always expected a few clear statements -re main background (thus recent innovations related to the main topic), initial literature review and, most of all, technological / scientific gaps in the current understanding. Also, it is expected a summary of the remaining sections at the end of the Introduction. In this report, the Introduction section covered all four expected parts. `Literature review` was mostly covered in the remaining of the report; 
                   \item The aim of Literature Review Section(s) is to introduce the main fundamental (i.e., theoretical) aspects of the work and to assess (with critical analysis) previous academic and/or industrial developments on the main subject areas. In summary, literature review should focus on 3-5 subjects of the project main topic and give an overview of past and current work (state-of-the-art) on them. Most of all, the section should highlight current gaps in specific knowledge;
                     
                   The literature review undertaken in this report was very thorough {\it wrt} thermodynamic cycles and energy aspects of smart cities, but limited on the energy integration part;
                   \item The work plan is appropriate for the project and seems realistic;
                \end{enumerate}
%
    \item {\bf Technical Merit:}

                    Mr Clark undertook an excellent literature review (with critical analysis) of the topics for his project and designed a demanding but fair program of work for the Spring.  Specific comments can be seen in the annotated copy in attachment.
%
\end{enumerate}

\clearpage


\noindent{\bfseries\large EG4013/14 -- B-/M-Eng Winter Report \hfill January, 2018}

\bigskip

\begin{center}
  {\Large Review of the B/MEng Winter Report `Implementation of PSO Algorithm for Global Minimum Thermodynamic Problems' by M. Mmadintsi}
\end{center}

The report describes Ms Mmadintsi's winter research on global evolutionary optimisation algorithms for chemical engineering applications, and in particular her initial investigation of the performance of PSO and CS algorithms on standard benchmark optimisation problems. A brief overview of the main topics on numerical optimisation relevant to thermodynamic problems was undertaken by Ms Mmadintsi including (a) types of optimisation methods (deterministic and stochastic); (b) general terminology used on nonlinear optimisation field, and; (c) short review of the implementation of standard PSO, CS and SA algorithms.   

\begin{enumerate}
%
    \item {\bf Presentation and Style of Writing:}
               \begin{enumerate}
                   \item The report lacks a coherent structure as expected in a report with: (a) Abstract; (b) Introduction; (c) Literature Review; (d) Detailed of techniques/technologies/procedures relevant for the dissertation; (e) Detailed set of actions for the Spring, and; (f) Conclusions and general remarks;
                   \item The report is relatively well-written with a number of typos and unrevised sentences. Several sentences and paragraphs are confusing and disconnected with no clear objectives and inter-connectivities. Most of all, the reports lacks a coherent structured and linkages between sections and the project's aims and objectives. This leads to an unsmooth reading at times;
                   \item Referencing is very limited -- 5 references for a winter report is definitely (much) below the expected threshold. Also, the references are not linked in the main text;
                   \item Figures (and Tables) must be referenced and fully explained in the main text. They can not be 'floating'. The same also applies to Equations;
                   \item Equations must be numbered;  
                   \item Authors \underline{must} avoid use {\it colloquial (informal / personal)} writing (\eg 'we'). Also, try to avoid long sentences.
                \end{enumerate}
%
    \item {\bf Technical Contents:}
                \begin{enumerate}
                   \item The main aim of Abstracts is to briefly describe the work undertaken by the author. In general Abstracts are divided in 4 parts: (i) motivation, (ii) main objectives, (iii) summary of the main procedures / techniques / technologies (optional) and (iv) main findings. The current report does not contain an Abstract;
                   \item The Introduction section/chapter usually has the same (but more in-depth and descriptive) four parts of the Abstract and a brief summary of the remaining of the work. In addition, it is always expected a few clear statements -re main background (thus recent innovations related to the main topic), initial literature review and, most of all, technological / scientific gaps in the current understanding. Also, it is expected a summary of the remaining sections at the end of the Introduction. In this report, the Introduction section covered only (i)-(iii). `Literature review` was the focus of Section 2;
                   \item The aim of Literature Review Sections/Chapters is to introduce the main fundamental (i.e., theoretical) aspects of the work and to assess (with critical analysis) previous academic and/or industrial developments on the main subject areas. In summary, literature review should focus on 3-5 subjects of the project main topic and give an overview of past and current work (state-of-the-art) on them. Most of all, the section/chapter should highlight current gaps in specific knowledge.
                     
                     The literature review undertaken in this report was broad, covering from general optimisation theory and types of optimisation methods to few details -re computational implementation of PSO, CS and SA algorithms. Most of all, it lacked a coherent structure of the main science and technologies that will be relevant to the project;
                   \item Work plan (Gantt chart) lacks rigour on specific activities that will be undertaken by the student during the Spring session;
                   \item There is no conclusion section in the report.
                \end{enumerate}
%
    \item {\bf Technical Merit:}
                \begin{enumerate}
                   \item Although the report demonstrated Ms Mmadintsi appreciation of the main issues \wrt to optimisation and had summarised the main techniques that will be used during the Spring, it lacked a coherent structure of analysis;
                   \item  Ms Mmadintsi demonstrated a good level of understanding of the relevant concepts for the project;
                   \item  Specific comments can be seen in the annotated copy in attachment.
                \end{enumerate}
%
\end{enumerate}

\medskip

\clearpage


%%%%%% 
%%%%%%
%%%%%%


\noindent{\bfseries\large EG4013/14 -- B-/M-Eng Winter Report \hfill January, 2018}

\bigskip

\begin{center}
  {\Large Review of the B/MEng Winter Report `Numerical Simulation of CO$_{2}$-EOR in Heterogeneous Reservoir Simulations' by Ehsan F. Zanjani}
\end{center}

The report describes Mr Zanjani's winter research on CO$_{2}$-EOR processes, and in particular his initial investigation on the elements \wrt hydrocarbon recovery. The report is very superficial and lacks depth on the main elements of his future (Spring) research. 

\begin{enumerate}
%
    \item {\bf Presentation and Style of Writing:}
                \begin{enumerate}
                   \item The report lacks a coherent structure as expected in a report with: (a) Abstract; (b) Introduction; (c) Literature Review; (d) Detailed of techniques/technologies/procedures relevant for the dissertation; (e) Detailed set of actions for the Spring, and; (f) Conclusions and general remarks;  
                   \item The report is relatively well-written with a large number of typos and unrevised sentences. Several sentences are confusing and disconnected with no clear objectives and inter-connectivities. Most of all, the paper lacks a coherent structured and linkages between sections resulting in an unsmooth reading at times;
                   \item Finally, the report lacks a clear statement (or section) of aims and objectives of either the report or the project; it is therefore unclear how all sections connect towards a goal (project's aim) and how the Gantt chart may help to achieve it;
                   \item Several references have missing fields and no clear distinction between articles, conference proceedings, reports (internal or external), book chapters, books, communications (internal or external) etc.  Regardless of the chosen citation style (e.g., ACS, AIP, AMS, IEEE, AIAA, etc) any reference section {\bf must} contain the following fields: 
                      \begin{enumerate}
                         \item For journal papers: Authors, Paper Title, Journal Name, Volume, Pages, Year of publication;
                         \item For books: Authors, Book Title, Publisher, Year or Edition;
                         \item For book chapters: Authors, Chapter Title, Book Tittle, Editors, Publisher, Year or Edition;
                         \item For conference papers: Authors, Paper Title, Conference Tittle, Place (Country and/or City) where the conference was held, Year of the conference;
                         \item For reports, private communications and Lecture Notes: Authors, Title, Place issued (Country and/or City and Institution where the document was originated), Year;
                         \item For PhD Thesis and MSc Dissertations: Author, Title, Institution (University and Department/School), Year.
                      \end{enumerate}  
                      Thus, for example:
                         \begin{enumerate}[label={[\arabic*]}]
                            \item P.L. Houtekamer and L. Mitchell, `Data Assimilation Using an Ensemble Kalman Filter Technique', {\it Monthly Weather Review}, 126:796-811, 1998.
                            \item K. Pruess, `Numerical Modelling of Gas Migration at a Proposed Repository for Low and Intermediate Level Nuclear Wastes', Technical Report LBL-25413, Lawrence Berkeley Laboratory, Berkeley (USA), 1990.
                            \item K. Aziz, A. Settari, {\it Fundamentals of Reservoir Simulation}, Elsevier Applied Science Publishers, New York (USA), 1986.
                            \item R.B. Lowrie, `Compact Higher-Order Numerical Methods for Hyperbolic Conservation Laws', PhD Thesis, Department of Aerospace Engineering and Scientific Computing, University of Michigan (USA), 1996.
                         \end{enumerate}
                   \item Equations must be numbered -- numbered should be placed in brackets at the rhs of the equation;
                   \item Dissertations and thesis are always divided into chapters $\rightarrow$ sections, whereas reports are divided into sections;
                  \item Authors \underline{must} avoid use {\it colloquial (informal / personal)} writing. Also, try to avoid long sentences;
                \end{enumerate}
%
    \item {\bf Technical Contents:}
                \begin{enumerate}
                   \item The main aim of Abstracts is to briefly describe the work undertaken by the author. In general Abstracts are divided in 4 parts: (i) motivation, (ii) main objectives, (iii) summary of the main procedures / techniques / technologies (optional) and (iv) main findings. The current Abstract encompasses (i) and (iii);
                   \item The Introduction section/chapter usually has the same (but more in-depth and descriptive) four parts of the Abstract and a brief summary of the remaining of the work. In addition, it is always expected a few clear statements -re main background (thus recent innovations related to the main topic), initial literature review and, most of all, technological / scientific gaps in the current understanding. Also, it is expected a summary of the remaining sections at the end of the Introduction. In this report, the Introduction section covered only (i) -- Motivation for the work. `Literature review` was spread over Sections 1 to 3; 
                   \item The aim of Literature Review Sections/Chapters is to introduce the main fundamental (i.e., theoretical) aspects of the work and to assess (with critical analysis) previous academic and/or industrial developments on the main subject areas. In summary, literature review should focus on 3-5 subjects (with depth) of the project main topic and {\bf give an overview of past and current work (state-of-the-art) on them}. Most of all, the section/chapter should highlight current gaps in specific knowledge.
                     
                     The literature review undertaken in this report was very limited and lacked a coherent structure, and covered very few topics (with {\bf no} depth) on methodologies currently used by industry to assess effectiveness and efficiency of CO$_{2}$-EOR;
                   \item The report does not state aims and objectives of either the project or the report;
                   \item Work plan (Gantt chart) lacks rigour on specific activities that will be undertaken by the student during the Spring session, and milestones designed to achieve the project's aims;
                   \item There is no conclusion section in the report.
                \end{enumerate}
%
    \item {\bf Technical Merit:}
                \begin{enumerate}
                   \item Literature review was very limited and did not cover most of the critical points for Mr Zanjani's work over the Spring. Also, 8 references for a winter report is definitely (much) below the expected threshold;
                  \item Also, the references are not linked in the main text;
                   \item  Mr Zanjani demonstrated a basic level of understanding of the relevant concepts for the project. 
                \end{enumerate}
%
\end{enumerate}

\medskip

\clearpage

%%%%%%%
%%%%%%%
%%%%%%%

%\lipsum % Text before
\afterpage{%
    \clearpage% Flush earlier floats (otherwise order might not be correct)
    \thispagestyle{empty}% empty page style (?)
    \begin{landscape}% Landscape page
        \centering % Center table


\Huge{MSc Study Assessment (EG5909/5911)}\\
\huge{(Review + Feedback)}\\
\huge{September 2017}
%\end{center}
\normalsize

\bigskip

%\begin{center}
\begin{tabular}{||c| c c c |c| c||}
\hline\hline
                           & {\bf Presentation and Style} & {\bf Technical Merit  } & {\bf Critical Reasoning} & {\bf Discipline} & {\bf Averaged}  \\
                           &                              &                         &                       &                     & {\bf CGS Mark}  \\
\hline
Stefan Wilms               &             13.00            &           14.00         &          15.00           &   PetEng         &    14.10 (C1)  \\
Ahmed Hassan               &             19.00            &           15.00         &          13.00           &   PetEng         &    15.20 (B3)   \\
Afam Ejidike               &             20.00            &           19.00         &          18.00           &   Renewable      &    19.90 (A4)   \\
Tan Heng Kwang             &             11.00            &           15.00         &          14.00           &   PetEng         &    13.90 (C1)   \\
\hline\hline
\end{tabular}
%\end{center}
    \end{landscape}
    \clearpage% Flush page
}

%\lipsum % Text after
\vfill
\clearpage


%%%%%% 
%%%%%%
%%%%%%
\noindent{\bfseries\large EG5911 -- MSc Thesis \hfill Sept, 2017}

\bigskip

\begin{center}
  {\Large Review of the MSc Thesis `Effect of Salinity and Polymer Concentration on Polymer Flooding with Partially Hydrolysed Polyacrylamide Polymers' by S. Wilms}
\end{center}
\begin{enumerate}%A.]
\item Presentation and Style
    \begin{enumerate}%i)]
       \item Assessor 1: ``The dissertation investigates polymer flooding for EOR. The dissertation is relatively well-written with a number of typos and unrevised sentences. Few sentences are confusing and disconnected with no clear objectives and inter-connectivities. Chapters and sections are not numbered, making the reading slightly more difficult. Most figures have very low quality and with poor colour schemes and symbols. In general, referencing is fine with a few missing fields in citations.''
       \item Assessor 2: ``The dissertation is neat and well presented. Main sections are generally about the review of polymer properties but lack sufficient details about heterogeneous reservoir as stated in the title of the project. Tables are well presented and correctly referenced. The standard of English is acceptable. There are some grammatical errors but they generally do not inhibit clarity and understanding.''
    \end{enumerate}
%  
   \item Technical Content and Merit
    \begin{enumerate}%i)]
       \item Assessor 1: ``Abstract contains the main expected elements (i.e., motivation, summary of technologies and findings) but a clear statement of the main aim and objective(s) of the work is missing. In fact, there is no clear statement of the aim of the work at any place in the report. Introduction section was very confusing and not longer than the Abstract. The students undertook an excellent literature review of the main topics of the dissertation with a good and relatively detailed description of physical and chemical mechanisms involving polymer flooding. Several numerical simulations were performed for the work, however they were poorly explained, with missing information {\it wrt} boundary $\&$ initial conditions, SPE-10 referencing, mesh grid analysis, equations (i.e., model to be solved) $\&$ symbols etc. Also, the reasons for using and (mostly) modifications of the SPE-10 benchmark were not fully explained which led to poor results. Conclusions and recommendations were very limited in scope and analysis.''
       \item Assessor 2: ``The introduction states the background and methodology but does not adequately summarise the sweep efficiency result from wettability. Chapter 2 (Introduction) should be more of a summary and critical review of literature. In terms of mechanism of polymer flooding efficiency, the important part of formation property which greatly influence the sweep efficiency is missing in the dissertation. The concept of relative permeability relate to the wettability is wrongly stated in the thesis. The methodology of viscosity ratio used in simulation is appropriate and detailed but needs to be a more logical progression, needs linking heterogeneity of reservoir rock property to further explain the consequence and impact on oil recovery. The work lacks depth and does not cover of the main issues for polymer flooding. The methodology does not employ any particularly new or novel techniques. The conclusions are supported by the project results but should be more concise. Recommendations are clear but not entirely relevant.''
    \end{enumerate}
%  
   \item Evidence of Critical Reasoning
    \begin{enumerate}%i)]
       \item Assessor 1: ``The candidate undertook an excellent literature review of the main topics of the dissertation with a good and relatively detailed description and analysis of physical and chemical mechanisms involving polymer flooding. He managed to link a difficult theoretical background (chemical and physical properties of polymers and surface physics) and a relatively new technology. Choice of research methodology (numerical simulation configuration and setups) was not fully clear and effectively did not achieved the main aim of the work (though this was not clearly stated). Results were analysed based on theoretical background introduced in the literature review chapters/sections with good insights.''

       \item Assessor 2: ``The literature review contains little critical analysis. The student listed a few Polymer stability and other factors affecting polymer viscosity. The most important part to decide if the polymer flooding is effective would be the wettability of rock, and unfortunately this is not logically presented. The concept of relative permeability relates to the wettability is wrong in the thesis.''
    \end{enumerate}
%  
   \item Overall Performance
    \begin{enumerate}%i)]
       \item Assessor 1: ``In general, the dissertation was interesting to read and offered a good insight on current challenges in polymer flooding. Results and conclusions were aligned with literature. Discussions of the results were clear, though lacked depth. In summary, the work undertook by the candidate was extensive and complex but he managed to produce a good dissertation.''
       \item Assessor 2: ``The student did a basic work for literature review of polymers used for chemical EOR, but lacked deep understanding the multiphase flow and relative permeability analysis.''
    \end{enumerate}
%  
\end{enumerate}

\vfill
\clearpage
%%%%%
%%%%%
%%%%%


\noindent{\bfseries\large EG5911 -- MSc Thesis \hfill Sept, 2017}

\bigskip

\begin{center}
  {\Large Review of the MSc Thesis `Impact of Heterogeneity on Productivity during Waterflooding' by Ahmed Hassan}
\end{center}
\begin{enumerate}%A.]
\item Presentation and Style
    \begin{enumerate}%i)]
       \item Assessor 1: ''Dissertation is neat and well written with a small number of typos and unrevised sentences. Figures, tables and equations are well presented and correctly referenced, though a few figures are 'floating'. Bibliography is spotless.''
       \item Assessor 2: ''Generally neat looking and well written thesis with a clear structure. Good use of figures in Chapter 2 to illustrate concepts. Line plots in results chapter are mostly screen captures from software, these could have been clearer if they were made in an external program. Referencing style appropriate.''
    \end{enumerate}
%  
   \item Technical Content and Merit
    \begin{enumerate}%i)]
       \item Assessor 1: ``Abstract contains the main expected elements (i.e., motivation, summary of technologies, aims and objectives and main findings). Introduction is just an extended version of the Abstract (with the same elements) stating the relevance and aims for the work. Literature review is superficial and limited to a few well-established definitions in PetEng (i.e., porosity, permeability, capillary forces, heterogeneity etc). Methodology leading to numerical simulations is appropriate and detailed with logical progression, although it is not clear the aim of the proposed 'homogeneous model'. Simulations' configurations were clearly stated (though a few boundary and initial conditions are still missing) with a good discussion for the choices. A number of plots from the numerical simulations were presented, but they lack in-depth analysis of the results based on either theoretical reasoning or literature results. Conclusions and suggestions for future work were appropriate and based on the results.``
       \item Assessor 2: ``Abstract captures motivation, main method and results adequately. Main aim and objectives not clearly defined in Chapter 1. Background theory clearly explained in Chapter 2 also to non-experts, demonstrating a good understanding of the key processes and concepts. Petrel description was rather limited, but the Eclipse model set-up was clearly described including the governing equations. However, it was unclear how the relative permeabilities were established in the model. Analysis of results appropriate, but it do not answer which type of heterogeneity mostly affected the production rate. Therefore the conclusions remain rather qualitative. A more systematic approach would have been interesting and probably lead to more quantitative results.''
    \end{enumerate}
%  
   \item Evidence of Critical Reasoning
    \begin{enumerate}%i)]
       \item Assessor 1: ``Literature review was superficial and contains no critical analysis. It was based on common knowledge of PetEng subjects with small links with the overall project objective. Research methodology was not clearly explained, i.e., reasons for choosing the 3 simulations and a potential way to validate numerical solutions. A number of plots from the numerical simulations were presented, but they lack in-depth analysis of the results based on either theoretical reasoning or literature results.''
       \item Assessor 2: ``Background theory was well explained with a good use of examples. Current industry practices were not really discussed and the wider implications of the results . To what extend is heterogeneity accounted for in industry modeling? To what extend some of the processes described in the theory were included in the numerical models was unclear, this could have demonstrated to what extend the model choice was appropriate.''
    \end{enumerate}
%  
  \item Overall Performance \\
    In general, the dissertation was interesting to read and offered a good insight on the impact of reservoir heterogeneity on oil recovery. Results and conclusion were aligned with the literature. Discussion of the results were clear and logical. In summary, the work undertook by the candidate was extensive but he managed to produce a very good dissertation.
%  
\end{enumerate}

\vfill
\clearpage
%%%%%
%%%%%
%%%%%

\noindent{\bfseries\large EG5909 -- MSc Thesis \hfill Sept, 2017}

\bigskip

\begin{center}
  {\Large Review of the MSc Thesis `Geothermal Cooling System Design Study for the Tropics' by Afam Anthony Ejidike}
\end{center}
\begin{enumerate}%A.]
\item Presentation and Style
    \begin{enumerate}%i)]
       \item Assessor 1: ``The dissertation is neat and well-written with all relevant chapters and sections in a logical order to progression of ideas. There are a few typos and unrevised sentences. Fonts of equations are slightly smaller than the remaining text. Figures and Tables are of very good
quality and well referenced. Part of the bibliography is not correct with missing key-fields (e.g., page numbers). My only concern is the written style, calculations' chapters (Chapters 3-4) were written as a technical report rather than a MSc dissertation style, however it made the logical progression of candidate's work much clearer.''
       \item Assessor 2: ``The thesis is logically structured, clearly written and contains extensive appendixes outlining systems design and theoretical background. Projects objectives are well defined. English is clear; plots and tables are informative and well presented . Equations are presented very well, all parameters are properly defined after each equation. Overall presentation of the thesis is very good. There are a few minor deficiencies: notation list should have been included; the thesis should have been formatted to avoid empty spaces on some pages; number of decimal points in calculations and plots should have been consistent.''
    \end{enumerate}
%  
   \item Technical Content and Merit
    \begin{enumerate}%i)]
       \item Assessor 1: ``Abstract contains all expected elements (motivation, main technologies and findings) although a clear statement of the main aims and objectives is missing. Introduction is well-designed with clear focus on motivation for the work and aims $\&$ objectives. Literature review (Chapter 2) is relatively extensive and covers the main subjects of the dissertation (in particular well-established heat transfer model/mechanism for EAHE). Methodology leading to heat transfer calculations and numerical simulations for all the designs is well explained and with appropriate logical progression. Simulations configuration are clearly stated in Tables and Figures with good discussion of assumptions and choices made. Analysis of the results are with appropriate depth and are mostly based on observation of results rather than the literature. Conclusions and suggestions for future work were appropriate and based on the results.''

       \item Assessor 2: ``The student has carried out a Geothermal Cooling System Design Study for the tropics. The thesis represents a good combination of literature review, systems design, investment appraisal and interpretation of results . Although it is not clear from the text how deeply the author understands the background equations, his data analysis, interpretation , and conclusions show that he has a very good level of knowledge of the methods used, simulation procedure, and investment appraisal analysis.''
    \end{enumerate}
%  
   \item Evidence of Critical Reasoning
    \begin{enumerate}%i)]
       \item Assessor 1: ``Literature review was slightly superficial and based on a very few number of references, although with consistent critical analysis of the material. Research methodology was appropriate with a clear appreciation and understanding of importance, limitations and strengths of the subject and methods. Analysis of results were based on calculations and simulation data with linkage with theoretical background.''
       \item Assessor 2: ``The literature review, description of the design methodology, and discussions shows that the author has a good ability to outline the concepts and approaches in a reasonable and critical way.''
    \end{enumerate}
%  
   \item Overall Performance
    \begin{enumerate}%i)]
       \item Assessor 1: ``In general the dissertation was interesting to read and offered a good insight on the design of geothermal facility. Results and conclusion were aligned with the literature. Discussion of the results were clear and logical. In summary, the work undertook by the candidate was extensive but he managed to produce an excellent dissertation.''
       \item Assessor 2: ``Overall the thesis is at MSc level that is distinction.''
    \end{enumerate}
%  
\end{enumerate}

\vfill
\clearpage
%%%%%
%%%%%
%%%%%

\noindent{\bfseries\large EG5911 -- MSc Thesis \hfill Sept, 2017}

\bigskip

\begin{center}
  {\Large Review of the MSc Thesis `Upscaling Techniques in Heterogeneous Formation for EOR' by Tan Heng Kwang }
\end{center}
\begin{enumerate}%A.]
\item Presentation and Style
    \begin{enumerate}%i)]
       \item Assessor 1: ``Overall, the dissertation is neat but with a large number of typos and unrevised sentences. Several sentences are confusing and disconnected with no clear objectives and inter-connectivities. Numbering of a few equations are inconsistent and referencing of them are wrong throughout the dissertation. Figures and Tables are of very good quality and well referenced, although a few Tables are 'floating'. Bibliography and most of citations are wrong with missing key-fields.''
       \item Assessor 2: ``Overall the presentation of the dissertation is good, the quality of English is good and each paragraph works well to construct the dissertation. Also, there are some typos that needs to be corrected and a relatively large number of error in equations and references''
    \end{enumerate}
%  
   \item Technical Content and Merit
    \begin{enumerate}%i)]
       \item Assessor 1: ''Abstract contains the main expected elements (motivation, main technologies and findings) although a clear statement of the main aims and objectives is missing. Introduction is relatively well-designed with clear focus on motivation for the work and aims $\&$ objectives. Literature review (Chapters 2 and 3) is relatively extensive but lacks depth as most of its content is based on common PetEng subjects, i.e., state-of-the art wrt upcaling methods is missing. Methodology leading to numerical simulations is appropriate with logical progression. Simulations configuration are clearly stated in Tables (though a few boundary and initial conditions are missing) with good discussion of choices made. Analysis of the results lacks depth and are based only on observation of results rather than the theory (i.e., literature review). Conclusions and suggestions for future work were appropriate and based on the results.''
       \item Assessor 2: ``Overall the technical content of this dissertation is good at MSc level. The Abstract is excellent, which covers the importance, methodology and findings of this work. Several single phase permeability upscaling methodologies have been proposed, but not addressed  well how this have been doe. Specially, for the flow based upscaling, more details of the method should have been presented. It seems that all the work were done with SLB software. The results is discussed in detail and the conclusions are supported by the findings in the dissertation.''
    \end{enumerate}
%  
   \item Evidence of Critical Reasoning
    \begin{enumerate}%i)]
       \item Assessor 1: ``Literature review was slightly superficial and contains no critical analysis. It was mostly based on common knowledge of PetEng subjects (in particular Chapter 2). Research methodology was appropriate with a clear appreciation and understanding of importance, limitations and strengths of the subject and methods. Analysis of results lack in-depth discussion and were based only on observation of simulation data with no linkage with theoretical background.''
       \item Assessor 2: ``The results were discussed in details in term of accuracy between simulation results from fine and coarse-scale models, but the limitation of the proposed approach and alternative approaches were not discussed.''
    \end{enumerate}
%  
  \item Overall Performance
    In general the dissertation was interesting to read and offered a good insight on upscaling methods in reservoir engineering and simulation. Results and conclusion were aligned with the literature. Discussion of the results were clear and logical. In summary, the work undertook by the candidate was extensive but he managed to produce a very good dissertation.
%  
\end{enumerate}

\vfill
\clearpage
%%%%%
%%%%%
%%%%%

%\lipsum % Text before
\afterpage{%
    \clearpage% Flush earlier floats (otherwise order might not be correct)
    \thispagestyle{empty}% empty page style (?)
    \begin{landscape}% Landscape page
        \centering % Center table

 
\Huge{MEng/BEng Study Assessment -- Dissertation)}\\
\huge{(Review + Feedback)}\\
\huge{May 2017}
%\end{center}
\normalsize

\bigskip

  
%\begin{center}
\begin{tabular}{||c| c c c |c| c||}
\hline\hline
                           & {\bf Presentation and Style} & {\bf Technical Merit  } & {\bf Critical Reasoning} & {\bf Discipline} & {\bf Averaged}  \\
                           &                              &                         &                       &                     & {\bf CGS Mark}  \\
\hline
Ahmed Nassar               &             17.10            &           33.60         &          --              &   BEng-MechEng   &    50.70 (C3)  \\
Ross M. Keightley          &             16.40            &           40.50         &          24.30           &   MEng-ChemEng   &    81.20 (A2)   \\
Hasan Kayani               &             20.40            &           47.83         &          --              &   BEng-PetEng    &    68.23 (B1)   \\
Briony Macaulay-Donn       &             15.40            &           34.00         &          18.30           &   MEng ChemEng   &    67.70 (B1)   \\
Greig Adams                &             16.00            &           38.50         &          21.60           &   MEng ChemEng   &    76.10 (A4)   \\
Patrick Kelsey             &             16.00            &           40.00         &          21.75           &   MEng ChemEng   &    77.75 (A3)   \\
Beth Wisely                &             16.40            &           41.00         &          21.38           &   MEng ChemEng   &    78.78 (A3)   \\

\hline\hline
\end{tabular}
%\end{center}
    \end{landscape}
    \clearpage% Flush page
}



%\lipsum % Text after
\vfill
\clearpage


%%%%%% 
%%%%%%
%%%%%%
\noindent{\bfseries\large EG4013 -- MEng Thesis \hfill May, 2017}

\bigskip

\begin{center}
  {\Large Review of the MEng Thesis `Optimisation of Heat Integration using Particle Swarms' by Ross Keightley}
\end{center}
\noindent
{\Large Comments from Marker 1:}

The dissertation investigates thermal energy integration through a stochastic optimisation algorithm, 'particle swarm optimisation' (PSO). Mr Keightley undertook a comprehensive literature review of the latest developments on PSO algorithm family (including performance and algorithm optimisation) and on current methods for plant's energy integration. He developed his own PSO Matlab code followed by cross-code / cross-model validation (benchmark) using industry-standard test cases and applied the model to a synthetic thermal energy system. The dissertation encompass three main subject areas within Chemical Engineering: computational optimisation, thermal engineering and modelling $\&$ design of processes.

The manuscript is well-written with a small number of typos and unrevised sentences. Few sentences are confusing and disconnected with no clear objectives and inter-connectivities. Most of all, the dissertation is very well-structured with clear division and linkages between chapters, sections and paragraphs, leading to an easy, smooth and engaging reading. A few general comments,
\begin{enumerate}
%
\item The main aim of {\it Abstracts} is to briefly describe the work undertaken by the author (i.e., the thesis' content). In general {\it Abstracts} are divided in 4 parts: (i) motivation, (ii) aims and objectives, (iii) summary of the main procedures / techniques / technologies (optional) and (iv) main findings (or summary of the work). The current {\it Abstract} encompass all the above.
%
\item The main {\it Introduction} section usually has the same (but more in-depth and descriptive) four parts of the {\it Abstract} and a brief summary of the remaining of the work. In addition, it is \underline{always} expected a few clear statements -re main background (thus recent innovations related to the main topic), initial literature review and, most of all, technological / scientific gaps in the current understanding. Also, it is expected a {\it summary of the remaining sections} at the end of the {\it Introduction}.  Current {\it Introduction} covered all the above. `Literature review` was the focus of Chapter 2.
%
\item The aim of Literature Review Sections/Chapters is to introduce the main fundamental (i.e., theoretical) aspects of the work and to assess (with critical analysis) previous academic and/or industrial developments on the main subject areas. In summary, literature review should focus on 3-5 subjects of the project main topic and give an overview of past and current work (state-of-the-art) on them. Most of all, the section/chapter should highlight current gaps in specific knowledge. In this thesis, literature review (Chapters 2) for PSO was very comprehensive, whereas for energy integration was relatively limited. 
%
\item Reference 2 is incomplete.
%
\item Overall, it was good conclusion but recommendations (\ie future work) was very limited.
% 
\end{enumerate}
In summary, the work undertook by the Mr Keightley was extensive and complex but he managed to produce an excellent dissertation. The topic is very relevant for the energy and environmental sectors, and each sub-topic has been the focus of several academic- and industrial-based studies worldwide with clear cross-fertilisation with chemical engineering (process design, thermal engineering, fluid mechanics etc) and mathematics (computational optimisation, numerical methods etc). The student demonstrated that he had an excellent understanding of the main technologies involved in this project.


\bigskip
\noindent
{\Large Comments from Marker 2:}

'Abstract is concise yet complete. Introduction is well structured, complete and engaging to read.

Literature review gets right to the point focusing on PSO. It is however a little difficult to take in at the start for someone who does not have knowledge of the technique already. It might have been worth giving the intelligent layman a fighting chance with a slightly slower paced lead in. The subject matter is quite complex but the author seems to have a good knowledge and understanding. Literature review is excellent with strong evidence of critical thinking. 

Software design and benchmarking sections are very good. I like the fact that an overview and conclusion is present at start and finish of each chapter.

The heat exchanger optimisation feels somewhat of an anti-climax after what came before but results are well presented and discussion. It might have been interesting to compare findings with those from pinch analysis to demonstrate the added value of this approach as the saving presented are versus the do-nothing case which is unlikely to ever be the case. It would have also been informative to show the HEN annotated with temperatures as it is difficult for the reader to get a feel for assumptions around temperature approach. 

Document is well written with good use of English. The style used is quite relax and informal but nonetheless capable of conveying complex methods with clarity. Watch use of US spellings.

Figures are clear with suitable titles and are numbered. Draws from an appropriate number and variety of sources. Well referenced with appropriate in text citations. Figures are referenced when appropriate.

Overall an excellent piece of work.'

\clearpage
\noindent
{\Large Thesis Conduct Assessment:}

The dissertation investigates thermal energy integration through a stochastic optimisation algorithm, “particle swarm Optimisation” (PSO). Mr Keightley undertook a comprehensive literature review of the latest developments on PSO algorithm family (including performance and algorithm optimisation) and on current methods for plant’s energy integration. During the two half-terms, he was very engaged with the project and demonstrated a strong interest on the main subject areas (optimisation and energy integration for process design). He attended all scheduled meetings, always bringing ideas, results and suggestions for the project. During this period, it was clear that he dealt with the project with great professionalism leading the project to the end. Overall, Mr Keightley produced an excellent dissertation.




\vfill
\clearpage
%%%%%
%%%%%
%%%%%



\noindent{\bfseries\large EG4014 -- BEng Thesis \hfill May, 2017}

\bigskip

\begin{center}
  {\Large Review of the BEng Thesis `Study of MultiScale Waterflooding Mechanisms in Heterogeneous Reservoir Simulations' by Hasan Kayani}
\end{center}

\noindent
{\Large Comments from Marker 1:}

The dissertation aims to investigate formation and growth of viscous fingering during waterflooding in heterogeneous geological formations. Mr Kayani undertook a comprehensive literature review of some oil recovery related-technologies. He also performed numerical experiments using Fluidity model software to investigate relationship between viscosity ratio and fingering in a synthetic 2D formation. The dissertation encompass two main subject areas within Petroleum Engineering: reservoir engineering and reservoir simulation.

The manuscript is well-written with a small number of typos and unrevised sentences. Few sentences are confusing and disconnected with no clear objectives and inter-connectivities. Most of all, the dissertation is well-structured with clear division and linkages between chapters, sections and paragraphs, leading to a smooth and engaging reading. A few general comments,
\begin{enumerate}
%
\item The main aim of {\it Abstracts} is to briefly describe the work undertaken by the author (i.e., the thesis' content). In general {\it Abstracts} are divided in 4 parts: (i) motivation, (ii) aims and objectives, (iii) summary of the main procedures / techniques / technologies (optional) and (iv) main findings (or summary of the work). The current {\it Abstract} encompass (i)-(ii) and (iv).
%
\item The main {\it Introduction} section usually has the same (but more in-depth and descriptive) four parts of the {\it Abstract} and a brief summary of the remaining of the work. In addition, it is \underline{always} expected a few clear statements -re main background (thus recent innovations related to the main topic), initial literature review and, most of all, technological / scientific gaps in the current understanding. Also, it is expected a {\it summary of the remaining sections} at the end of the {\it Introduction}.  Current {\it Introduction} covered all the above. `Literature review` was the focus of Chapter 2.
%
\item The aim of Literature Review Sections/Chapters is to introduce the main fundamental (i.e., theoretical) aspects of the work and to assess (with critical analysis) previous academic and/or industrial developments on the main subject areas. In summary, literature review should focus on 3-5 subjects of the project main topic and give an overview of past and current work (state-of-the-art) on them. Most of all, the section/chapter should highlight current gaps in specific knowledge. In this thesis, literature review, Chapters 2, covered most of the relevant topics for the dissertation.
%
\item Legends containing the colour scheme of all figures are missing. 
%
\item Chapters should be placed in new pages.
%
\item Some references are repeated, and a few of them have missing fields. Regardless of the chosen citation style (e.g., ACS, AIP, AMS, IEEE, AIAA, etc) any reference {\bf must} contain the following fields: 
\begin{enumerate}
\item For journal papers: Authors, Paper Tittle, Journal Name, Volume, Pages, Year of publication;
\item For books: Authors, Book Tittle, Publisher, Year or Edition;
\item For book chapters: Authors, Chapter Tittle, Book Tittle, Editors, Publisher, Year or Edition;
\item For conference papers: Authors, Paper Tittle, Conference Tittle, Place (Country and/or City) where the conference was held, Year of the conference;
\item For reports, private communications and Lecture Notes: Authors, Tittle, Place issued (Country and/or City and Institution where the document was originated), Year;
\item For PhD Thesis and MSc Dissertations: Author, Tittle, Institution (University and Department/School), Year.
\end{enumerate}  
Thus, for example:
\begin{enumerate}[label={[\arabic*]}]
\item P.L. Houtekamer and L. Mitchell, `Data Assimilation Using an Ensemble Kalman Filter Technique', {\it Monthly Weather Review}, 126:796-811, 1998.
\item K. Pruess, `Numerical Modelling of Gas Migration at a Proposed Repository for Low and Intermediate Level Nuclear Wastes', Technical Report LBL-25413, Lawrence Berkeley Laboratory, Berkeley (USA), 1990.
\item K. Aziz, A. Settari, {\it Fundamentals of Reservoir Simulation}, Elsevier Applied Science Publishers, New York (USA), 1986.
\item R.B. Lowrie, `Compact higher-Order Numerical Methods for Hyperbolic Conservation Laws', PhD Thesis, Department of Aerospace Engineering and Scientific Computing, University of Michigan (USA), 1996.
\end{enumerate}
%
\item Description of the numerical experiments lacks important information, \eg mesh resolution, permeability distribution function etc
% 
\end{enumerate}
In summary, the work undertook by the Mr Kayani was extensive and complex but he managed to produce a very good dissertation. The topic is very relevant for the energy sectors, and each sub-topic has been the focus of several academic- and industrial-based studies worldwide with clear cross-fertilisation with physics (thermodynamics, fluid mechanics etc) and petroleum engineering (reservoir engineering $\&$ simulation etc). The student demonstrated that he had a good understanding of the main technologies involved in this project.



\bigskip
\noindent
{\Large Comments from Marker 2:}

'Overall look is reasonable as the thesis is presented in an organised way with somehow acceptable flow. However, it lacks relevant literature review and instead it includes huge part of the text with generic information about enhanced oil recovery processes rather than focusing on a specific topic of this study.

There are grammatical issues and typos through the whole text. But overall it’s a satisfactory presentation.

Student showed a good level of understaffing of the relevant concepts. However, the methodology he used is not clear and there are some issue with the assumptions and technical results. As a results it contains flaws in logic, and arguments are poorly presented where it is not easy to satisfy the objectives based on them.'


\bigskip
\noindent
{\Large Thesis Conduct Assessment:}

The dissertation aims to investigate formation and growth of viscous fingering during waterflooding in heterogeneous geological formations. Mr Kayani undertook a comprehensive literature review of some oil recovery related-technologies. He also performed numerical experiments using Fluidity model software to investigate relationship between viscosity ratio and fingering in a synthetic 2D formation.  During most of the two half-sessions, Mr Kayani was not very engaged with the project and missed a few scheduled meetings. By the end of the project, he became more engaged and started demonstrating increasing interest (in particular on the main topic of the project, viscous flow instabilities) and worked independently. He ended up making an active contribution to the discussion of the technical content.  Overall, Mr Kayani produced a good dissertation.

\vfill
\clearpage

%%%%%% 
%%%%%%
%%%%%%



\noindent{\bfseries\large EG4013 -- MEng Thesis \hfill May, 2017}

\bigskip

\begin{center}
  {\Large Review of the MEng Thesis `Separation of Biofuel Fermentation Products: Process Analysis' by Beth Wisely}
\end{center}
The dissertation investigates energy and conversion efficiency of n-butanol production via fermentation process. Ms Wisely undertook a comprehensive literature review of the main production and separation technologies involving fermentation from feed-stock and separation from ABE flow streams. She also performed simulated design experiments of the key-processes to determine the efficiency, conversion and energy budget. The dissertation encompass four main subject areas within Chemical Engineering: thermal engineering, modelling $\&$ design of processes, separation engineering and biochemistry engineering.

The manuscript is well-written with a small number of typos and unrevised sentences. Few sentences are confusing and disconnected with no clear objectives and inter-connectivities. Most of all, the dissertation is well-structured with clear division and linkages between chapters, sections and paragraphs, leading to an easy and smooth reading. A few general comments,
\begin{enumerate}
%
\item The main aim of {\it Abstracts} is to briefly describe the work undertaken by the author (i.e., the thesis' content). In general {\it Abstracts} are divided in 4 parts: (i) motivation, (ii) aims and objectives, (iii) summary of the main procedures / techniques / technologies (optional) and (iv) main findings (or summary of the work). The current {\it Abstract} encompass all the above, although (ii) was/were not clearly stated.
%
\item The main {\it Introduction} section usually has the same (but more in-depth and descriptive) four parts of the {\it Abstract} and a brief summary of the remaining of the work. In addition, it is \underline{always} expected a few clear statements -re main background (thus recent innovations related to the main topic), initial literature review and, most of all, technological / scientific gaps in the current understanding. Also, it is expected a {\it summary of the remaining sections} at the end of the {\it Introduction}.  Current {\it Introduction} covered (i) and (ii) only. `Literature review` was part of Chapters 2-5.
%
\item The aim of Literature Review Sections/Chapters is to introduce the main fundamental (i.e., theoretical) aspects of the work and to assess (with critical analysis) previous academic and/or industrial developments on the main subject areas. In summary, literature review should focus on 3-5 subjects of the project main topic and give an overview of past and current work (state-of-the-art) on them. Most of all, the section/chapter should highlight current gaps in specific knowledge. In this thesis, literature review, embedded in Chapters 2-5, was very comprehensive and covered most of current technologies for production of butanol. 
%
\item A few figures are 'floating', \ie with no reference in the main text (\eg Fig 4.2 and 5.1).
% 
\end{enumerate}
In summary, the work undertook by the Ms Wisely was extensive and complex but she managed to produce an excellent dissertation. The topic is very relevant for the chemical, energy and environmental sectors, and each sub-topic has been the focus of several academic- and industrial-based studies worldwide with clear cross-fertilisation with chemical engineering (chemical reaction and kinetics, fluid phase equilibria, process design, micro-biology etc), physics (thermodynamics, fluid mechanics etc) and mathematics (computational optimisation, numerical methods etc). The student demonstrated that she had an excellent understanding of the main technologies involved in this project.


\vfill
\clearpage
%%%%%
%%%%%
%%%%%


\noindent{\bfseries\large EG4013 -- MEng Thesis \hfill May, 2017}

\bigskip 

\begin{center}
  {\Large Review of the MEng Thesis `CCSU: Process Analysis, Simulation and Design' by Patrick Kelsey}
\end{center}
The dissertation investigates energy efficiency of carbon capture (CC) process units via the design of distinct technologies focused on gas and coal energy sources. Mr Kelsey undertook a comprehensive literature review of the main carbon capture technologies. He also performed simulated design experiments of key CC processes to determine the energy budget. The dissertation encompass two main subject areas within Chemical Engineering: thermal engineering, modelling $\&$ design of processes.

The manuscript is well-written with a small number of typos and unrevised sentences. Few sentences are confusing and disconnected with no clear objectives and inter-connectivities. Most of all, the dissertation is well-structured with clear division and linkages between chapters, sections and paragraphs, leading to an easy, smooth and engaging reading. A few general comments,
\begin{enumerate}
%
\item The main aim of {\it Abstracts} is to briefly describe the work undertaken by the author (i.e., the thesis' content). In general {\it Abstracts} are divided in 4 parts: (i) motivation, (ii) aims and objectives, (iii) summary of the main procedures / techniques / technologies (optional) and (iv) main findings (or summary of the work). The current {\it Abstract} encompass (i), (iii)-(iv), aim(s) for the work was/were not clearly stated.
%
\item The main {\it Introduction} section usually has the same (but more in-depth and descriptive) four parts of the {\it Abstract} and a brief summary of the remaining of the work. In addition, it is \underline{always} expected a few clear statements -re main background (thus recent innovations related to the main topic), initial literature review and, most of all, technological / scientific gaps in the current understanding. Also, it is expected a {\it summary of the remaining sections} at the end of the {\it Introduction}.  Current {\it Introduction} covered (i) and (ii) only. `Literature review` was part of Chapters 4-7.
%
\item The aim of Literature Review Sections/Chapters is to introduce the main fundamental (i.e., theoretical) aspects of the work and to assess (with critical analysis) previous academic and/or industrial developments on the main subject areas. In summary, literature review should focus on 3-5 subjects of the project main topic and give an overview of past and current work (state-of-the-art) on them. Most of all, the section/chapter should highlight current gaps in specific knowledge. In this thesis, literature review, embedded in Chapters 4-7, was very comprehensive and covered most of today's CC-related technologies used worldwide to capture and separate fluids during the combustion processes. 
%
\item Chapters should be placed in new pages.
%
\item Some references are repeated, and a few of them have missing fields. Regardless of the chosen citation style (e.g., ACS, AIP, AMS, IEEE, AIAA, etc) any reference {\bf must} contain the following fields: 
\begin{enumerate}
\item For journal papers: Authors, Paper Tittle, Journal Name, Volume, Pages, Year of publication;
\item For books: Authors, Book Tittle, Publisher, Year or Edition;
\item For book chapters: Authors, Chapter Tittle, Book Tittle, Editors, Publisher, Year or Edition;
\item For conference papers: Authors, Paper Tittle, Conference Tittle, Place (Country and/or City) where the conference was held, Year of the conference;
\item For reports, private communications and Lecture Notes: Authors, Tittle, Place issued (Country and/or City and Institution where the document was originated), Year;
\item For PhD Thesis and MSc Dissertations: Author, Tittle, Institution (University and Department/School), Year.
\end{enumerate}  
Thus, for example:
\begin{enumerate}[label={[\arabic*]}]
\item P.L. Houtekamer and L. Mitchell, `Data Assimilation Using an Ensemble Kalman Filter Technique', {\it Monthly Weather Review}, 126:796-811, 1998.
\item K. Pruess, `Numerical Modelling of Gas Migration at a Proposed Repository for Low and Intermediate Level Nuclear Wastes', Technical Report LBL-25413, Lawrence Berkeley Laboratory, Berkeley (USA), 1990.
\item K. Aziz, A. Settari, {\it Fundamentals of Reservoir Simulation}, Elsevier Applied Science Publishers, New York (USA), 1986.
\item R.B. Lowrie, `Compact higher-Order Numerical Methods for Hyperbolic Conservation Laws', PhD Thesis, Department of Aerospace Engineering and Scientific Computing, University of Michigan (USA), 1996.
\end{enumerate}
%
\item Some design decisions were taken without given any reasoning, \eg choice of EOS, choice of turbine discharge pressure as a primary parameter for system thermal efficiency etc.
% 
\end{enumerate}
In summary, the work undertook by the Mr Kelsey was extensive and complex but he managed to produce an excellent dissertation. The topic is very relevant for the chemical, energy and environmental sectors, and each sub-topic has been the focus of several academic- and industrial-based studies worldwide with clear cross-fertilisation with chemical engineering (chemical reaction and kinetics, fluid phase equilibria, process design etc), physics (thermodynamics, fluid mechanics etc) and mathematics (computational optimisation, numerical methods etc). The student demonstrated that he had an excellent understanding of the main technologies involved in this project.


\vfill
\clearpage
%%%%%
%%%%%
%%%%%

\noindent{\bfseries\large EG4014 -- BEng Thesis \hfill May, 2017}

\bigskip

\begin{center}
  {\Large Review of the BEng Thesis `Assessing and Optimising CHP Performance in Smart Cities' by Ahmed Nassar}
\end{center}

\noindent
{\Large Comments from Marker 1:}

The dissertation aims to investigate the performance of combined heat and power (CHP) stations in district heating. Mr Nassar undertook a limited literature review of some CHP-related technologies. He also performed numerical experiments using ASPEN software to assess power outputs in a hypothetical CHP. The dissertation encompasses two main subject areas within Mechanical Engineering: thermal power generation and system optimisation.

The manuscript is relatively well-written with a small number of typos and unrevised sentences. Few sentences are confusing and disconnected with no clear objectives and inter-connectivities. Most of all, the dissertation is well-structured with clear division and linkages between chapters, sections and paragraphs, leading to a smooth reading.

Although, thermal engineering (\ie thermo-fluid dynamics) and optimisation technologies (methods and algorithms) were focus on the thesis, were partially explained and analysed. A CHP plant design was the focus of the third chapter. Source of data was missing and no real explanation was given for some of the assumptions used in the simulation (\eg, last sentence of the first paragraph of page 29). Also, in the final design (Fig. 11), there is no information of system set up and data streams (\ie mass flow rates, temperatures, etc). Finally, some of the calculations are not correct.

A few general comments,
\begin{enumerate}
%
\item The main aim of {\it Abstracts} is to briefly describe the work undertaken by the author (i.e., the thesis' content). In general {\it Abstracts} are divided in 4 parts: (i) motivation, (ii) aims and objectives, (iii) summary of the main procedures / techniques / technologies (optional) and (iv) main findings (or summary of the work). The current {\it Abstract} encompass all the above.
%
\item The main {\it Introduction} section usually has the same (but more in-depth and descriptive) four parts of the {\it Abstract} and a brief summary of the remaining of the work. In addition, it is \underline{always} expected a few clear statements -re main background (thus recent innovations related to the main topic), initial literature review and, most of all, technological / scientific gaps in the current understanding. Also, it is expected a {\it summary of the remaining sections} at the end of the {\it Introduction}.  Current {\it Introduction} covered all the above. `Literature review` was the focus of Chapter 2.
%
\item The aim of Literature Review Sections/Chapters is to introduce the main fundamental (i.e., theoretical) aspects of the work and to assess (with critical analysis) previous academic and/or industrial developments on the main subject areas. In summary, literature review should focus on 3-5 subjects of the project main topic and give an overview of past and current work (state-of-the-art) on them. Most of all, the section/chapter should highlight current gaps in specific knowledge. In this thesis, literature review, embedded in Chapters 2, was very limited and covered just a few topics on methodologies and theory currently used by academics (and/or industry) to assess performance and productivity of CHP. Optimisation methods and technologies applied to CHP plant design and operation were poorly covered.
%
\item Figures and Tables {\bf must} be referenced in the main text -- they can not just $\lq$float around' (\eg Fig. 5). Also, figure/table captions should be self-contained, i.e., with a good description of the figure/table highlighting the most relevant aspects/information that the author wants to convene. Also, a few figures were not correctly numbered in the main text.
% 
\end{enumerate}
In summary, the work undertook by the Mr Nassar was extensive and complex but he managed to produce a good dissertation. The topic is very relevant for mechanical, energy and environmental sectors, and each sub-topic has been the focus of several academic- and industrial-based studies worldwide with clear cross-fertilisation with physics (thermodynamics, fluid mechanics etc) and mechanical engineering (thermal systems, optimisation etc). The student demonstrated that he had a good understanding of the main technologies involved in this project.


\bigskip
\noindent
{\Large Comments from Marker 2:}

'Thesis is reasonably well-prepared and organised.

Quality of the graphical content and referencing shall be improved, as well as the consistency of the text style.

The analysis of the results shows some understanding of the problem background (e.g., the costs of the CHP plant construction are not considered at all).

Overall, I believe that the thesis is a solid 2.2 attempt.'

\clearpage
\noindent
{\Large Thesis Conduct Assessment:}

The dissertation aims to investigate the performance of combined heat and power (CHP) stations in district heating. Mr Nassar undertook a limited literature review on the concept of smart cities and on some CHP related technologies. He also performed numerical experiments using ASPEN software to assess power outputs in a hypothetical CHP supplying thermal energy and power to confined community in central London. During the two half-sessions, Mr Nassar was not very engaged with the project and missed / rescheduled a few meetings. By the end of the project, he became slightly more engaged, bringing some topics and ideas to discussion and working mostly independently. Overall, Mr Nassar produced a good dissertation.

\vfill
\clearpage

%%%%%% 
%%%%%%
%%%%%%


\noindent{\bfseries\large EG4013 -- MEng Thesis \hfill May, 2017}

\bigskip

\begin{center}
  {\Large Review of the MEng Thesis `Alcohol Formation over Alkaline Promoted Fischer Tropsch Catalysts' by Greig Andrew Adams}
\end{center}
The dissertation aims to assess seletivity and conversion of syngas to alcohol through Fischer-Tropsch reactions. Mr Adams undertook a comprehensive literature review of the main FT technologies involving mechanisms and surface chemistry. He also performed experiments to determine selectivity and effectiveness of a set of supported metal catalysts. The dissertation encompass two main subject areas within Chemical Engineering: surface chemistry (i.e., catalysis) and organic reaction mechanisms.

The manuscript is well-written with a small number of typos and unrevised sentences. Few sentences are confusing and disconnected with no clear objectives and inter-connectivities. Most of all, the dissertation is well-structured with clear division and linkages between chapters, sections and paragraphs, leading to an easy and smooth reading.

Reaction mechanisms and catalysts processes were partially explained with coherent analysis backed by theory. A few general comments,
\begin{enumerate}
%
\item The main aim of {\it Abstracts} is to briefly describe the work undertaken by the author (i.e., the thesis' content). In general {\it Abstracts} are divided in 4 parts: (i) motivation, (ii) aims and objectives, (iii) summary of the main procedures / techniques / technologies (optional) and (iv) main findings (or summary of the work). The current {\it Abstract} encompass (ii)-(iv), motivation for the work was not clearly stated.
%
\item The main {\it Introduction} section usually has the same (but more in-depth and descriptive) four parts of the {\it Abstract} and a brief summary of the remaining of the work. In addition, it is \underline{always} expected a few clear statements -re main background (thus recent innovations related to the main topic), initial literature review and, most of all, technological / scientific gaps in the current understanding. Also, it is expected a {\it summary of the remaining sections} at the end of the {\it Introduction}.  Current {\it Introduction} covered (i) and (ii) only. `Literature review` was part of the third chapter.
%
\item The aim of Literature Review Sections/Chapters is to introduce the main fundamental (i.e., theoretical) aspects of the work and to assess (with critical analysis) previous academic and/or industrial developments on the main subject areas. In summary, literature review should focus on 3-5 subjects of the project main topic and give an overview of past and current work (state-of-the-art) on them. Most of all, the section/chapter should highlight current gaps in specific knowledge. In this thesis, literature review, embedded in Chapters 2 (`Background Theory') and 3 (`Catalyst Analysis'), was limited and covered few topics on methodologies and theory currently used by academics (and/or industry) to assess properties, effectiveness and efficiency of metal supported catalysts in F-T processes.
% 
\end{enumerate}
In summary, the work undertook by the Mr Adams was extensive and complex but he managed to produce an excellent dissertation. The topic is very relevant for the chemical, energy and environmental sectors, and each sub-topic has been the focus of several academic- and industrial-based studies worldwide with clear cross-fertilisation with chemistry (surface chemistry, catalysis etc) and chemical engineering (chemical kinetics, reactor design etc). The student demonstrated that he had an excellent understanding of the main technologies involved in this project.


\vfill
\clearpage
%%%%%
%%%%%
%%%%%

\noindent{\bfseries\large EG4013 -- MEng Thesis \hfill May, 2017}

\bigskip

\begin{center}
  {\Large Review of the MEng Thesis `Emerging Ionic Liquid Technologies for Carbon Capture from Large Point Sources' by Briony Macaulay-Donn}
\end{center}
The dissertation aims to assess thermophysical properties of ionic liquids that can be used in carbon capture processes.  Ms Macaulay-Donn undertook a comprehensive literature review of the main technologies involving IL and carbon capture processes. She also performed experiments to determine thermo-physical properties on CO$_{2}$ + IL solutions. The dissertation encompass two main subject areas within Chemical Engineering: environmental technology and thermo-physical properties.

The manuscript is well-written with a small number of typos and unrevised sentences. Few sentences are confusing and disconnected with no clear objectives and inter-connectivities. Most of all, the dissertation is well-structured with clear division and linkages between chapters, sections and paragraphs, leading to an easy and smooth reading.  A few general comments,
\begin{enumerate}
%
\item The main aim of {\it Abstracts} is to briefly describe the work undertaken by the author (i.e., the thesis' content). In general {\it Abstracts} are divided in 4 parts: (i) motivation, (ii) aims and objectives, (iii) summary of the main procedures / techniques / technologies (optional) and (iv) main findings (or summary of the work). The current {\it Abstract} encompass (i), (iii)-(iv), aims and objectives of the work were not clearly stated.
%
\item The main {\it Introduction} section usually has the same (but more in-depth and descriptive) four parts of the {\it Abstract} and a brief summary of the remaining of the work. In addition, it is \underline{always} expected a few clear statements -re main background (thus recent innovations related to the main topic), initial literature review and, most of all, technological / scientific gaps in the current understanding. Also, it is expected a {\it summary of the remaining sections} at the end of the {\it Introduction}.  Current {\it Introduction} covered all the above. `Literature review` was part of the third chapter.
%
\item The aim of Literature Review Sections/Chapters is to introduce the main fundamental (i.e., theoretical) aspects of the work and to assess (with critical analysis) previous academic and/or industrial developments on the main subject areas. In summary, literature review should focus on 3-5 subjects of the project main topic and give an overview of past and current work (state-of-the-art) on them. Most of all, the section/chapter should highlight current gaps in specific knowledge. In this thesis, literature review, embedded in Chapters 2 (`Background Theory') and 4 (`Results and discussion'), was limited and covered few topics on methodologies and theory currently used by academics (and/or industry) to assess properties, effectiveness and efficiency of CO$_{2}$ + IL systems.
%
\item Some references are repeated, and a few of them have missing fields. Regardless of the chosen citation style (e.g., ACS, AIP, AMS, IEEE, AIAA, etc) any reference {\bf must} contain the following fields: 
\begin{enumerate}
\item For journal papers: Authors, Paper Tittle, Journal Name, Volume, Pages, Year of publication;
\item For books: Authors, Book Tittle, Publisher, Year or Edition;
\item For book chapters: Authors, Chapter Tittle, Book Tittle, Editors, Publisher, Year or Edition;
\item For conference papers: Authors, Paper Tittle, Conference Tittle, Place (Country and/or City) where the conference was held, Year of the conference;
\item For reports, private communications and Lecture Notes: Authors, Tittle, Place issued (Country and/or City and Institution where the document was originated), Year;
\item For PhD Thesis and MSc Dissertations: Author, Tittle, Institution (University and Department/School), Year.
\end{enumerate}  
Thus, for example:
\begin{enumerate}[label={[\arabic*]}]
\item P.L. Houtekamer and L. Mitchell, `Data Assimilation Using an Ensemble Kalman Filter Technique', {\it Monthly Weather Review}, 126:796-811, 1998.
\item K. Pruess, `Numerical Modelling of Gas Migration at a Proposed Repository for Low and Intermediate Level Nuclear Wastes', Technical Report LBL-25413, Lawrence Berkeley Laboratory, Berkeley (USA), 1990.
\item K. Aziz, A. Settari, {\it Fundamentals of Reservoir Simulation}, Elsevier Applied Science Publishers, New York (USA), 1986.
\item R.B. Lowrie, `Compact higher-Order Numerical Methods for Hyperbolic Conservation Laws', PhD Thesis, Department of Aerospace Engineering and Scientific Computing, University of Michigan (USA), 1996.
\end{enumerate}
%
\item Colour legends are missing in a few figures (e.g., 4.7-12).  
%
\item Chapters should be placed in new pages.
%
\item Numbering of figures and tables were not consistent.
%
\item In general, there was good analysis and reasoning of the results backed by theory.
% 
\end{enumerate}
In summary, the work undertook by the Ms Macaulay-Donn was extensive and complex but she managed to produce a very good dissertation. The topic is very relevant for the chemical, energy and environmental sectors, and each sub-topic has been the focus of several academic- and industrial-based studies worldwide with clear cross-fertilisation with physics (thermodynamics, material science etc) and chemical engineering (flow simulators, reactor design, flow separators etc). The student demonstrated that she had a very good understanding of the main technologies involved in this project.

\medskip

\clearpage

%%%%%%%
%%%%%%%
%%%%%%%

%\lipsum % Text before
\afterpage{%
    \clearpage% Flush earlier floats (otherwise order might not be correct)
    \thispagestyle{empty}% empty page style (?)
    \begin{landscape}% Landscape page
        \centering % Center table


\Huge{MEng/BEng Study Assessment -- Winter Report (EG4013/EG4014)}\\
\huge{(Review + Feedback)}\\
\huge{January 2017}
%\end{center}
\normalsize

\bigskip

%\begin{center}
\begin{tabular}{||c| c c c |c| c||}
\hline\hline
                           & {\bf Presentation and Style} & {\bf Technical Content} & {\bf Technical Merit} & {\bf Discipline} & {\bf Averaged}  \\
                           &                              &                         &                       &                  & {\bf CGS Mark}  \\
\hline
Ahmed Nassar               &         12.25                &         11.33           &        9.50           &   BEng-MechEng   &      11 (D1)    \\
Ross M. Keightley          &         20.50                &         19.33           &       20.50           &   MEng-ChemEng   &      20 (A3)    \\
Hasan Kayani               &         14.50                &         14.33           &       13.00           &   BEng-PetEng    &      14 (C1)    \\
\hline\hline
\end{tabular}
%\end{center}
    \end{landscape}
    \clearpage% Flush page
}



%\lipsum % Text after

\vfill

\clearpage

%%%%%% 
%%%%%%
%%%%%%


\noindent{\bfseries\large EG4013/14 -- B-/M-Eng Winter Report \hfill January, 2017}

\bigskip

\begin{center}
  {\Large Review of the BEng Winter Report `Assessing $\&$ Optimising (C)CHP Performance in Smart Cities' by Ahmed Nassar}
\end{center}

The report describes Mr Nassar's winter research on CHP and CCHP systems, and in particular his initial investigation on energy assessment and optimisation for 'smart cities'. An overview of some set of criteria currently used in energy/power assessment was conducted by Mr Nassar including, (a) simplified energy/mass balance, (b) parameterised coupled energy consumption and contaminant emissions models, (c) general exergy analysis and (d) simplified unconstraint optimisation technologies. The report is relatively well-written with a large number of typos and unrevised sentences. Several sentences are confusing and disconnected with no clear objectives and inter-connectivities. Most of all, the paper lacks a coherent structured and linkages between sections resulting in an unsmooth reading at times. Finally, the report lacks a clear statement (or section) of aims and objectives of either the report or the project; it is therefore unclear how all sections connect towards a goal (project's aim) and how the Gantt chart may help to achieve it.

\begin{enumerate}
%
    \item {\bf Presentation and Style of Writing:}
                \begin{enumerate}
                   \item Numbered sections would greatly improve quality of the report;
                   \item Several references have missing fields and no clear distinction between articles, conference proceedings, reports (internal or external), book chapters, books, communications (internal or external) etc.  A few {\it references} used in the report are incomplete and/or wrong. Regardless of the chosen citation style (e.g., ACS, AIP, AMS, IEEE, AIAA, etc) any reference section {\bf must} contain the following fields: 
                      \begin{enumerate}
                         \item For journal papers: Authors, Paper Title, Journal Name, Volume, Pages, Year of publication;
                         \item For books: Authors, Book Title, Publisher, Year or Edition;
                         \item For book chapters: Authors, Chapter Title, Book Tittle, Editors, Publisher, Year or Edition;
                         \item For conference papers: Authors, Paper Title, Conference Tittle, Place (Country and/or City) where the conference was held, Year of the conference;
                         \item For reports, private communications and Lecture Notes: Authors, Title, Place issued (Country and/or City and Institution where the document was originated), Year;
                         \item For PhD Thesis and MSc Dissertations: Author, Title, Institution (University and Department/School), Year.
                      \end{enumerate}  
                      Thus, for example:
                         \begin{enumerate}[label={[\arabic*]}]
                            \item P.L. Houtekamer and L. Mitchell, `Data Assimilation Using an Ensemble Kalman Filter Technique', {\it Monthly Weather Review}, 126:796-811, 1998.
                            \item K. Pruess, `Numerical Modelling of Gas Migration at a Proposed Repository for Low and Intermediate Level Nuclear Wastes', Technical Report LBL-25413, Lawrence Berkeley Laboratory, Berkeley (USA), 1990.
                            \item K. Aziz, A. Settari, {\it Fundamentals of Reservoir Simulation}, Elsevier Applied Science Publishers, New York (USA), 1986.
                            \item R.B. Lowrie, `Compact Higher-Order Numerical Methods for Hyperbolic Conservation Laws', PhD Thesis, Department of Aerospace Engineering and Scientific Computing, University of Michigan (USA), 1996.
                         \end{enumerate}
                       \item Figures (and Tables) must be referenced and fully explained in the main text. They can not be 'floating'. The same also applies to Equations.
                         \item Dissertations and thesis are always divided into chapters $\rightarrow$ sections, whereas reports are divided into sections.
                       \item Authors \underline{must} avoid use {\it colloquial (informal / personal)} writing. Also, try to avoid long sentences.
                \end{enumerate}
%
    \item {\bf Technical Contents:}
                \begin{enumerate}
                   \item The main aim of Abstracts is to briefly describe the work undertaken by the author. In general Abstracts are divided in 4 parts: (i) motivation, (ii) main objectives, (iii) summary of the main procedures / techniques / technologies (optional) and (iv) main findings. The current Abstract encompasses (i) and, at a limited extent (ii).
                   \item The Introduction section/chapter usually has the same (but more in-depth and descriptive) four parts of the Abstract and a brief summary of the remaining of the work. In addition, it is always expected a few clear statements -re main background (thus recent innovations related to the main topic), initial literature review and, most of all, technological / scientific gaps in the current understanding. Also, it is expected a summary of the remaining sections at the end of the Introduction. In this report, the Introduction section covered only (i) -- Motivation for the work. `Literature review` was the title of a following section but it was partially undertaken throughout the remaining oh the report. 
                   \item The aim of Literature Review Sections/Chapters is to introduce the main fundamental (i.e., theoretical) aspects of the work and to assess (with critical analysis) previous academic and/or industrial developments on the main subject areas. In summary, literature review should focus on 3-5 subjects of the project main topic and give an overview of past and current work (state-of-the-art) on them. Most of all, the section/chapter should highlight current gaps in specific knowledge.
                     
                     The literature review undertaken in this report was very limited and lacked a coherent structure, and covered very few topics on methodologies currently used by academics (and industry) to assess effectiveness and efficiency of power systems.
                   \item The report does not state aims and objectives of either the project or the report.
                   \item Work plan (Gantt chart) lacks rigour on specific activities that will be undertaken by the student, and milestones designed to achieve the project's aims.
                   \item There is no conclusion section in the report.
                \end{enumerate}
%
    \item {\bf Technical Merit:}
                \begin{enumerate}
                   \item Although the literature review was very limited with superficial critical analysis, Mr Nassar used a good number of papers to investigate the work undertaken by several authors on energy systems assessment;
                   \item  Mr Nassar demonstrated a basic level of understanding of the relevant concepts for the project. 
                \end{enumerate}
%
\end{enumerate}

\medskip

\clearpage

%%%%%% 
%%%%%%
%%%%%%


\noindent{\bfseries\large EG4013/14 -- B-/M-Eng Winter Report \hfill January, 2017}

\bigskip

\begin{center}
  {\Large Review of the MEng Winter Report `Particle Swarm Optimisation' by Ross M. Keightley}
\end{center}

The report describes Mr Keightley's winter research on heuristics optimisation methods for thermal process integration, and in particular his initial investigation on particle swarm algorithms and their applications to heat integration problems. A brief overview of the main topics on global/local optimisation and energy analysis in thermal processes was undertaken by Mr Keightley including, (a) main versions of the algorithm; (b) convergence rate analysis; (c) main terminology and (d) few methods (i.e., submodels) currently in heat integration problems.  


\begin{enumerate}
%
    \item {\bf Presentation and Style of Writing:}
                \begin{enumerate}
                   \item The report is well-written with few typos and unrevised sentences. Most of all, the report has a very coherent and logical structured and linkages between sections, leading to a smooth reading.
                   \item References are consistent and well designed. 
                   \item Figures, Tables and Equations must be referenced and fully explained in the main text. They can not be 'floating'.
                   \item Authors \underline{must} avoid use {\it colloquial (informal / personal)} writing. Also, try to avoid long sentences.
                   \item Dissertations and thesis are always divided into chapters $\rightarrow$ sections, whereas reports are divided into sections.
                \end{enumerate}
%
    \item {\bf Technical Contents:}
                \begin{enumerate}
                   \item The main aim of Abstracts is to briefly describe the work undertaken by the author. In general Abstracts are divided in 4 parts: (i) motivation, (ii) main objectives, (iii) summary of the main procedures / techniques / technologies (optional) and (iv) main findings. The current Abstract encompasses (ii) and, partially, (i) and (v).
                   \item The Introduction section/chapter usually has the same (but more in-depth and descriptive) four parts of the Abstract and a brief summary of the remaining of the work. In addition, it is always expected a few clear statements -re main background (thus recent innovations related to the main topic), initial literature review and, most of all, technological / scientific gaps in the current understanding. Also, it is expected a summary of the remaining sections at the end of the Introduction. In this report, the Introduction section covered (i,ii,iv). `Literature review` was the focus of the second section of the report. 
                   \item The aim of Literature Review Sections/Chapters is to introduce the main fundamental (i.e., theoretical) aspects of the work and to assess (with critical analysis) previous academic and/or industrial developments on the main subject areas. In summary, literature review should focus on 3-5 subjects of the project main topic and give an overview of past and current work (state-of-the-art) on them. Most of all, the section/chapter should highlight current gaps in specific knowledge.
                     
                   The literature review undertaken in this report was very thorough {\it wrt} PSO, but very limited on the energy integration part.
                   \item The work plan is appropriate for the project and seems realistic.
                \end{enumerate}
%
    \item {\bf Technical Merit:}

                    Mr Keightley undertook an excellent literature review (with critical analysis) of the topics for his project and designed a demanding but fair program of work for the Fall. 
%
\end{enumerate}

\clearpage

%%%%%% 
%%%%%%
%%%%%%


\noindent{\bfseries\large EG4013/14 -- B-/M-Eng Winter Report \hfill January, 2017}

\bigskip

\begin{center}
  {\Large Review of the BEng Winter Report `Study of Multi-scale Waterflooding Mechanisms in Heterogeneous Reservoir Simulation' by H. Kayani}
\end{center}

The report describes Mr Kayani's winter research on fundamentals of reservoir engineering and simulation, and in particular his initial investigation on viscous fluid instabilities in waterflooding processes for oil and gas exploration. A brief overview of the main topics on reservoir engineering and simulation relevant to waterflooding was undertaken by Mr Kayani  including, (a) main terminology, (b) main recovery techniques and reservoir morphological properties, and (d) fundamentals of fluid instabilities.   

\begin{enumerate}
%
    \item {\bf Presentation and Style of Writing:}
                \begin{enumerate}
                   \item The report is relatively well-written with a number of typos and unrevised sentences. Several sentences and paragraphs are confusing and disconnected with no clear objectives and inter-connectivities. Most of all, the reports lacks a coherent structured and linkages between sections and the project's aims and objectives. This leads to an unsmooth reading at times.
                   \item Several references have missing fields and no clear distinction between articles, conference proceedings, reports (internal or external), book chapters, books, communications (internal or external) etc.  A few {\it references} used in the report are incomplete and/or wrong. Regardless of the chosen citation style (e.g., ACS, AIP, AMS, IEEE, AIAA, etc) any reference section {\bf must} contain the following fields: 
                      \begin{enumerate}
                         \item For journal papers: Authors, Paper Title, Journal Name, Volume, Pages, Year of publication;
                         \item For books: Authors, Book Title, Publisher, Year or Edition;
                         \item For book chapters: Authors, Chapter Title, Book Tittle, Editors, Publisher, Year or Edition;
                         \item For conference papers: Authors, Paper Title, Conference Tittle, Place (Country and/or City) where the conference was held, Year of the conference;
                         \item For reports, private communications and Lecture Notes: Authors, Title, Place issued (Country and/or City and Institution where the document was originated), Year;
                         \item For PhD Thesis and MSc Dissertations: Author, Title, Institution (University and Department/School), Year.
                      \end{enumerate}  
                      Thus, for example:
                         \begin{enumerate}[label={[\arabic*]}]
                            \item P.L. Houtekamer and L. Mitchell, `Data Assimilation Using an Ensemble Kalman Filter Technique', {\it Monthly Weather Review}, 126:796-811, 1998.
                            \item K. Pruess, `Numerical Modelling of Gas Migration at a Proposed Repository for Low and Intermediate Level Nuclear Wastes', Technical Report LBL-25413, Lawrence Berkeley Laboratory, Berkeley (USA), 1990.
                            \item K. Aziz, A. Settari, {\it Fundamentals of Reservoir Simulation}, Elsevier Applied Science Publishers, New York (USA), 1986.
                            \item R.B. Lowrie, `Compact Higher-Order Numerical Methods for Hyperbolic Conservation Laws', PhD Thesis, Department of Aerospace Engineering and Scientific Computing, University of Michigan (USA), 1996.
                         \end{enumerate}
                       Additionally, citations {\bf must} be uniform throughout the report.
                       \item Figures (and Tables) must be referenced and fully explained in the main text. They can not be 'floating'. The same also applies to Equations.
                         \item Dissertations and thesis are always divided into chapters $\rightarrow$ sections, whereas reports are divided into sections.
                       \item Authors \underline{must} avoid use {\it colloquial (informal / personal)} writing. Also, try to avoid long sentences.
                \end{enumerate}
%
    \item {\bf Technical Contents:}
                \begin{enumerate}
                   \item The main aim of Abstracts is to briefly describe the work undertaken by the author. In general Abstracts are divided in 4 parts: (i) motivation, (ii) main objectives, (iii) summary of the main procedures / techniques / technologies (optional) and (iv) main findings. The current Abstract encompasses (i) and (iv).
                   \item The Introduction section/chapter usually has the same (but more in-depth and descriptive) four parts of the Abstract and a brief summary of the remaining of the work. In addition, it is always expected a few clear statements -re main background (thus recent innovations related to the main topic), initial literature review and, most of all, technological / scientific gaps in the current understanding. Also, it is expected a summary of the remaining sections at the end of the Introduction. In this report, the Introduction section covered only (i) and (ii). `Literature review` was the focus of Section 2.
                   \item The aim of Literature Review Sections/Chapters is to introduce the main fundamental (i.e., theoretical) aspects of the work and to assess (with critical analysis) previous academic and/or industrial developments on the main subject areas. In summary, literature review should focus on 3-5 subjects of the project main topic and give an overview of past and current work (state-of-the-art) on them. Most of all, the section/chapter should highlight current gaps in specific knowledge.
                     
                     The literature review undertaken in this report was limited to definitions of some of the technologies used in hydrocarbon recovery with little assimilation and / or critical analysis of the material. Most of all, it lacked a coherent structure of the main science and technologies that will be relevant to the project.
                   \item Work plan (Gantt chart) lacks rigour on specific activities that will be undertaken by the student.
                   \item There is no conclusion section in the report.
                \end{enumerate}
%
    \item {\bf Technical Merit:}
                \begin{enumerate}
                   \item Although the literature review was very limited with superficial critical analysis, Mr Kayani used a good number of papers to investigate the work undertaken by several authors on reservoir engineering;
                   \item  Mr Kayani demonstrated a good level of understanding of the relevant concepts for the project. 
                \end{enumerate}
%
\end{enumerate}





\medskip

\clearpage


%%%%%%%
%%%%%%%
%%%%%%%

%\lipsum % Text before
\afterpage{%
    \clearpage% Flush earlier floats (otherwise order might not be correct)
    \thispagestyle{empty}% empty page style (?)
    \begin{landscape}% Landscape page
        \centering % Center table

%\begin{center}
\Huge{MSc Dissertations}\\
\huge{(Review + Feedback)}\\
\huge{May-December 2016}
%\end{center}
\normalsize



\bigskip

%\begin{center}
\begin{tabular}{||c| c c c |c| c||}
\hline\hline
                           & {\bf Presentation and Style of} & {\bf Technical Content and}           & {\bf Evidence of Critical} & {\bf Discipline} & {\bf Averaged}  \\
                           & {\bf Writing (20$\%$)}          & {\bf Merit of Dissertation (50$\%$)}  & {\bf Reasoning (30$\%$)}   &                  & {\bf CGS Mark}  \\
\hline
Andrew Batson              &         15 (B3)                 &            14 (C1)                    &        9 (D3)              &   Subsea (DL)    &   12.70 (C2)    \\
Kenneth Uttih              &         21 (A2)                 &            19 (A4)                    &       17 (B1)              &   Subsea (DL)    &   18.80 (A4)    \\
Benjamin Hogan             &         21 (A2)                 &            19 (A4)                    &       20 (A3)              &   Subsea (DL)    &   19.70 (A3)    \\
\hline\hline
\end{tabular}
%\end{center}
    \end{landscape}
    \clearpage% Flush page
}

%\lipsum % Text after



\clearpage
%%%%%
%%%%%
%%%%%

\noindent{\bfseries\large MSc in Subsea Engineering \hfill December, 2016}

\bigskip

\begin{center}
{\Large Review of the MSc Dissertation `Subsea Condition $\&$ Performance Monitoring: A Focus on Subsea Control System Hydraulic Fluid Leakage Detection' by Kenneth Uttih}
\end{center}

\medskip

\begin{enumerate}
%
    \item {\bf Presentation and Style of Writing:}
                \begin{enumerate}
                   \item The manuscript is well written with a small number of typos and unrevised sentences. Few sentences are confusing and disconnected with no clear objectives and inter-connectivities. Overall the dissertation is well-structured with clear division and linkage between chapters and sections, leading to a smooth reading.
                   \item A few references follow different standards with missing fields and no clear distinction between articles, conference proceedings, reports (internal or external), internal communications, book chapters, books, communications (internal or external) etc. 
                \end{enumerate}
%
    \item {\bf Technical Content and Merit of the Dissertation:}
                \begin{enumerate}
                   \item The main aim of Abstracts is to briefly describe the work undertaken by the author. In general Abstracts are divided in 4 parts: (i) motivation, (ii) main objectives, (iii) summary of the main procedures / techniques / technologies (optional) and (iv) main findings. The current Abstract encompasses all of them.
                   \item The Introduction chapter usually has the same (but more in-depth and descriptive) four parts of the Abstract and a brief summary of the remaining of the work. In addition, it is always expected a few clear statements -re main background (thus recent innovations related to the main topic), initial literature review and, most of all, technological / scientific gaps in the current understanding. Also, it is expected a summary of the remaining sections at the end of the Introduction. In this dissertation, the Introduction section covered all the above. `Literature review` is the focus of `Chapters 2-4'. 
                   \item The aim of Literature Review Sections/Chapters is to introduce the main fundamental (i.e., theoretical) aspects of the work and to assess (with critical analysis) previous academic and/or industrial developments on the main subject areas. In summary, literature review should focus on 3-5 subjects of the dissertation main topic and give an overview of past and current work (state-of-the-art) on them. Most of all, the section/chapter should highlight current gaps in specific knowledge. Literature review was based on both industry experience (through industry internal reports and communications, and manuals) and recent innovations (through scientific journals and industry magazines) on assessing different strategies for monitoring leakage in subsea structures. The descriptions of equipment, processes and strategies to detect leakage were insightful and detailed.
                   \item Detailed description of a case-scenario and strategies to both identify failure that led to leakage and to fix the problem were done in a clear and logical way.
                   \item In-depth analysis of the subject with imaginative cross-fertilisation with current available technologies to tackle the problem (i.e., detection and monitoring of leakages of hydrocarbons and hydraulic fluid), though it is not clear if other more established technologies were considered and fully assessed.
                   \item Conclusion section is well-written and clearly linked with aims and objectives though not directly connected with analysed case-scenario. Recommendations for future work are clear connected with the subject and demonstrated a good understanding of the challenges in the field. 
                \end{enumerate}
%
    \item {\bf Evidence of Critical Reasoning:}  
                \begin{enumerate}
                   \item The dissertation aims to develop a methodology to monitor and detect leakage of hydraulic fluid in subsea installations. The student undertook a comprehensive literature review (though with limited critical analysis) on some of the main aspects involving technologies that have been employed to monitor and detect leakage of hydrocarbons in pipelines and extended them to hydraulic fluids. 
                   \item The student also proposed strategies for continuous assessment of installations (also based in existing methodologies employed by O$\&$G sector).  
                   \item The student performed an in-depth analysis of a case-scenario and analysed alternatives to current methodologies using experiences from other (more established) fields. The methodology was clear and logical and fully linked to the aims and objectives of the dissertation. 
                \end{enumerate}
%
    \item {\bf Student's Overall Performance:} In general, the dissertation was very interesting to read and offered a good insight on current challenges in subsea hydraulic systems and the procedure to design leakage detection protocols. Results and conclusions were entirely aligned with aims and objectives. Discussions of issues and decision wrt to design were clear and logical. The student demonstrated that he had an excellent understanding of the main fundamental physics and technologies for this project.
%
\end{enumerate}
\clearpage
%%%%%
%%%%%
%%%%%

\noindent{\bfseries\large MSc in Subsea Engineering \hfill December, 2016}

\bigskip

\begin{center}
{\Large Review of the MSc Dissertation `Development and Qualification of a Prototype Subsea Hydraulic Coupler' by Benjamin Hogan}
\end{center}

\medskip

\begin{enumerate}
%
    \item {\bf Presentation and Style of Writing:}
                \begin{enumerate}
                   \item The manuscript is well written with a small number of typos and unrevised sentences. Overall the dissertation is well-structured with clear division and linkage between chapters and sections, leading to a smooth reading.
                   \item Dissertations and thesis are always divided into chapters rather than sections (commonly used in reports), where each chapter is allocated in a new page.
                   \item A few references follow different standards with missing fields and no clear distinction between articles, conference proceedings, reports (internal or external), internal communications, book chapters, books, communications (internal or external) etc.  A few references used in the manuscript are incomplete.
                   \item All tables (and a few figures) are 'floating' with no referencing in the main text.
                \end{enumerate}
%
    \item {\bf Technical Content and Merit of the Dissertation:}
                \begin{enumerate}
                   \item The main aim of Abstracts is to briefly describe the work undertaken by the author. In general Abstracts are divided in 4 parts: (i) motivation, (ii) main objectives, (iii) summary of the main procedures / techniques / technologies (optional) and (iv) main findings. The current Abstract encompasses i-iii. It is important to notice that the dissertation followed the structure of an industrial report, with an extended abstract, named 'executive report' with the list of tasks undertaken during the project.

                   \item The Introduction chapter usually has the same (but more in-depth and descriptive) four parts of the Abstract and a brief summary of the remaining of the work. In addition, it is always expected a few clear statements -re main background (thus recent innovations related to the main topic), initial literature review and, most of all, technological / scientific gaps in the current understanding. Also, it is expected a summary of the remaining sections at the end of the Introduction. In this dissertation, the Introduction chapter covered i and ii.  Potential lacks of knowledge on the field were not covered in this chapter. 
                   \item The aim of Literature Review Sections/Chapters is to introduce the main fundamental (i.e., theoretical) aspects of the work and to assess (with critical analysis) previous academic and/or industrial developments on the main subject areas. In summary, literature review should focus on 3-5 subjects of the dissertation main topic and give an overview of past and current work (state-of-the-art) on them. Most of all, the section/chapter should highlight current gaps in specific knowledge. Literature review was based on industry experience (through industry internal reports and communications, and manuals) on developing and designing a coupler. The descriptions of equipment and processes were insightful and detailed.
                   \item Detailed description of the design procedure with clear and logical progression to solution and in-depth analysis of the results obtained through hand-calculation and numerical simulations (CFD and solid FEA).
                   \item Conclusion section is well-written and clearly linked with aims and objectives. Recommendations for future work are clear connected with the subject and demonstrated a good understanding of the challenges in the field. 
                \end{enumerate}
%
    \item {\bf Evidence of Critical Reasoning:} 

                   The dissertation aims to develop a new design for a subsea hydraulic coupler. The student undertook an extensive literature review on main mechanical aspects involving subsea hydraulic installations with an in-depth analysis of process.  The new design was assessed according to current industry standards and each stage of the design process was fully explored.
%
    \item {\bf Student's Overall Performance:} In general, the dissertation was very interesting to read and offered a good insight on current challenges in subsea hydraulic systems and the procedure to design specific equipment. Results and conclusions were entirely aligned with aims and objectives. Discussions of issues and decision wrt to design were clear and logical. The student demonstrated that he had an excellent understanding of the main fundamental physics and technologies for this project.
%
\end{enumerate}
\clearpage

%%%%%
%%%%%
%%%%%

\noindent{\bfseries\large MSc in Subsea Engineering \hfill December, 2016}

\bigskip

\begin{center}
{\Large Review of the MSc Dissertation `Novel and Cost Effective Bundle Decommissioning' by Andrew Batson}
\end{center}

\medskip

\begin{enumerate}
%
    \item {\bf Presentation and Style of Writing:}
                \begin{enumerate}
                   \item The manuscript is well written with a small number of typos and unrevised sentences. Overall the dissertation is well-structured with clear division and linkage between `chapters' and sections, leading to a smooth reading.
                   \item Dissertations and thesis are always divided into chapters rather than sections (commonly used in reports), where each chapter is allocated in a new page.
                   \item All References follow different standards with missing fields and no clear distinction between articles, conference proceedings, reports (internal or external), internal communications, book chapters, books, communications (internal or external) etc.  A few references used in the manuscript are incomplete.
                   \item Equations must be explained in full and all terms used must be defined afterwards as part of the main text. In addition, equations must have the same font size as the main text. Also, Equation 1 does not seem to be correct.
                   \item Quality of some figures are very poor. Nearly all figures are referenced in the main text with wrong numbering. Also, mostly all figures and Appendices are 'floating' with no explanation/description/referencing in the main text.
                \end{enumerate}
%
    \item {\bf Technical Content and Merit of the Dissertation:}
                \begin{enumerate}
                   \item The main aim of Abstracts is to briefly describe the work undertaken by the author. In general Abstracts are divided in 4 parts: (i) motivation, (ii) main objectives, (iii) summary of the main procedures / techniques / technologies (optional) and (iv) main findings. The current Abstract encompasses i, iv and ii (in part).
                   \item The Introduction chapter usually has the same (but more in-depth and descriptive) four parts of the Abstract and a brief summary of the remaining of the work. In addition, it is always expected a few clear statements -re main background (thus recent innovations related to the main topic), initial literature review and, most of all, technological / scientific gaps in the current understanding. Also, it is expected a summary of the remaining sections at the end of the Introduction. In this dissertation, the Introduction chapter covered i and ii.  Potential lacks of knowledge on the field were not covered neither in this chapter nor elsewhere. 
                   \item The aim of Literature Review Sections/Chapters is to introduce the main fundamental (i.e., theoretical) aspects of the work and to assess (with critical analysis) previous academic and/or industrial developments on the main subject areas. In summary, literature review should focus on 3-5 subjects of the dissertation main topic and give an overview of past and current work (state-of-the-art) on them. Most of all, the section/chapter should highlight current gaps in specific knowledge. Literature review was very limited (as expected for such new field) and based mostly on industry internal reports and communications, and manuals. The descriptions of equipment and processes, although insightful were very superficial and lacked critical analysis.
                   \item Although the `chapters and sections' were very well-linked (leading to a smooth reading), most of the work (`Chapters' 2 to 7) focused on the description of the current subsea infra-structure, (two) potential alternatives for decommissioning and legal/environmental/safety issues. However, general description and analysis of the problem and solutions were very superficial.
                   \item A simulation of stress in bundles were chosen as a test-case (although it is not clear why this particular case-scenario was chosen). There was explanation of the equations to be solved, methods used and the analysis of the results were very superficial.
                   \item Conclusion section is well-written and based on `Chapters' 2-7, and it is clearly linked with aims and objectives. Recommendations for future work are clear connected with the subject and demonstrated a basic understanding of the challenges in the field. 
                \end{enumerate}
%
    \item {\bf Evidence of Critical Reasoning:}
                \begin{enumerate}
                   \item The dissertation aims to produce an overview of methods and technologies for decommissioning of bundles. The student undertook a limited and superficial literature review on some of the main aspects involving mechanical stress in pipelines, technologies for placement and removal of bundles, and it lacked critical analysis on the literature.  
                   \item The student performed numerical simulations using STAAD Pro V8i to investigate stress in bundles during 'cut and recover removal technique'. The methodology of simulation and assumption were not explained, and the analysis was superficial and not really linked with the main aim of the work. 
                   \item In summary, design of numerical experiments and analysis of the results indicated a limited understanding of the fundamentals and current available technologies for the project. 
                \end{enumerate}
%
    \item {\bf Student's Overall Performance:} In general, the dissertation was very interesting to read and offered a good insight on current challenges in decommissioning of bundles. Results and conclusions were not completely aligned with aims and objectives. Also, there is a clear lack of discussion on the main objective: critical analysis of technologies to be used in decommissioning of bundles. As this dissertation was supposed to be a `review' work, it would be expected a full investigation of a few topics on decommissioning of bundles, instead of an overview of the whole process with no detailed analysis of any step. 

The topic is very relevant for energy sector and each section has been the focus of several academic- and industrial-based studies worldwide with clear cross-fertilisation with physics (fluid mechanics thermodynamics etc), petroleum engineering (e.g., reservoir modelling and simulation etc) and computer science (e.g., software engineering, parallel processing, etc). The student demonstrated that he had a limited understanding of the main fundamental physics and technologies for this project.
%
\end{enumerate}
\clearpage


%%%%%%%
%%%%%%%
%%%%%%%

%\lipsum % Text before
\afterpage{%
    \clearpage% Flush earlier floats (otherwise order might not be correct)
    \thispagestyle{empty}% empty page style (?)
    \begin{landscape}% Landscape page
        \centering % Center table

%\begin{center}
\Huge{MSc Dissertations}\\
\huge{(Review + Feedback)}\\
\huge{May-August 2016}
%\end{center}
\normalsize


\bigskip

%\begin{center}
\begin{tabular}{||c| c c c |c| c||}
\hline\hline
                           & {\bf Presentation and Style of} & {\bf Technical Content and}           & {\bf Evidence of Critical} & {\bf Discipline} & {\bf Averaged}  \\
                           & {\bf Writing (20$\%$)}          & {\bf Merit of Dissertation (50$\%$)}  & {\bf Reasoning (30$\%$)}   &                  & {\bf CGS Mark}  \\
\hline
Luis H. Hernandez          &         04                      &           09                          &          05                &   Petroleum      &    E2 (06.80)   \\
Yuanda  Tanrismet          &         17                      &           15                          &          18                &   Petroleum      &    B2 (16.30)   \\
Raushan Sabiryanov         &         13                      &           11                          &          10                &   Petroleum      &    D1 (11.10)   \\
Roberto Giannitelli        &         11                      &           15                          &          15                &   Petroleum      &    C1 (14.20)   \\
Chinedu Ubachukwu          &         18                      &           19                          &          20                &   Petroleum      &    A4 (19.10)   \\
\hline\hline
\end{tabular}
%\end{center}
    \end{landscape}
    \clearpage% Flush page
}

%\lipsum % Text after

\clearpage

\vfill 


%%%%%
%%%%%
%%%%%

\noindent{\bfseries\large MSc in Petroleum Engineering\hfill August, 2016}

\bigskip

\begin{center}
{\Large Review of the MSc Dissertation `Modelling of Reservoir Rocks with Single Phase Flow using Lattice Boltzmann Method' by Luis H. Hernandez}
\end{center}

\medskip

The dissertation aims to develop a methodology to predict reservoir rock properties using Lattice Boltzmann method (LBM). The manuscript summarises the methods and techniques developed for LBM and in particular the formulation used in the Palabos software. Mr Hernandez undertook a rather limited literature review on these technologies.

The manuscript is poorly written with a large number of typos and unrevised sentences. Most of sentences are very confusing and disconnected with no clear objectives leading to a very difficult reading. Main concepts and definitions, within `Introduction' and `Literature Review' chapters were superficially explained (e.g., first sentence of Secion 2.4.5 wrt boundary condition), demonstrating a clear lack of understanding of the main aspects of the subject. A few extra comments:
\begin{enumerate}
\item The main aim of {\it Abstracts} is to briefly describe the work undertaken by the author. In general {\it Abstracts} are divided in 4 parts: (i) motivation, (ii) main objectives, (iii) summary of the main procedures / techniques / technologies (optional) and (iv) main findings. The current {\it Abstract} encompasses ii (in part), iii and iv.
%
\item The {\it Introduction} chapter usually has the same (but more in-depth and descriptive) four parts of the {\it Abstract} and a brief summary of the remaining of the work. In addition, it is always expected a few clear statements -re main background (thus recent innovations related to the main topic), initial literature review and, most of all, technological / scientific gaps in the current understanding. Also, it is expected a summary of the remaining sections at the end of the {\it Introduction}. In this dissertation, the {\it Introduction} section covered most of above, although aims and objectives were not clearly stated. Literature review is the focus of `Chapter 2' with limited critical analysis of the work undertaken by other authors. 
%
\item Equations \underline{must} be explained in full and all terms used must be defined afterwards as part of the main text (e.g., Eqns. 2.2, 2.5, etc). Some terms were assigned with different symbols throughout the work with no explanation (e.g., velocity as $c$, $u$ and $v$). In addition, equations must have the same font size as the main text.
%
\item All {\it References} follow different standards with missing fields and no clear distinction between articles, conference proceedings, reports (internal or external), book chapters, books, communications (internal or external) etc.  A few {\it references} used in the manuscript are incomplete and/or wrong. Regardless of the chosen citation style (e.g., ACS, AIP, AMS, IEEE, AIAA, etc) any reference {\bf must} contain the following fields: 
\begin{enumerate}
\item For journal papers: Authors, Paper Tittle, Journal Name, Volume, Pages, Year of publication;
\item For books: Authors, Book Tittle, Publisher, Year or Edition;
\item For book chapters: Authors, Chapter Tittle, Book Tittle, Editors, Publisher, Year or Edition;
\item For conference papers: Authors, Paper Tittle, Conference Tittle, Place (Country and/or City) where the conference was held, Year of the conference;
\item For reports,  private communications and Lecture Notes: Authors, Tittle, Place issued (Country and/or City and Institution where the document was originated), Year;
\item For PhD Thesis and MSc Dissertations: Author, Tittle, Institution (University and Department/School), Year.
\end{enumerate}  
Thus, for example:
\begin{enumerate}[label={[\arabic*]}]
\item P.L. Houtekamer and L. Mitchell, $\lq$Data Assimilation Using an Ensemble Kalman Filter Technique', {\it Monthly Weather Review}, 126:796-811, 1998.
\item K. Pruess, $\lq$Numerical Modelling of Gas Migration at a Proposed Repository for Low and Intermediate Level Nuclear Wastes', Technical Report LBL-25413, Lawrence Berkeley Laboratory, Berkeley (USA), 1990.
\item K. Aziz, A. Settari, {\it Fundamentals of Reservoir Simulation}, Elsevier Applied Science Publishers, New York (USA), 1986.
\item R.B. Lowrie, $\lq$Compact higher-Order Numerical Methods for Hyperbolic Conservation Laws', PhD Thesis, Department of Aerospace Engineering and Scientific Computing, University of Michigan (USA), 1996.
\end{enumerate} 
%
\item Quality of some figures are very poor. Also, most figures (and a few tables) are 'floating' with no explanation/description/referencing in the main text.
%
\item Use of {\it colloquial (informal / personal)} writing should be avoided. 
%
\item Description of the numerical experiment was very confusing and limited.
%
\item Analysis of numerical results lacks theoretical depth and reasoning -- in particular comments of Fig. 4.12 (and associated text) where velocities reached 4.5 km/s. Also, in page 49, it is not clear if the student refers to number of time-steps, time-step size or number of (non-linear) iterations. 
%
\item Also, conclusions in Chapter 4 does not seem to be based on the plots.   
% 
\end{enumerate}

In general, the dissertation was very superficial and difficult to follow. Results and conclusions were not clearly aligned with aims and objectives. Also, there is a clear lack of discussion on the main objective: sensitivity analysis of parameters (velocity, tortuosity and number of either time-steps or iterations) in LBM simulation to assess permeability. Several crucial concepts are either not defined or not correct.

The topic is very relevant for energy sector and each section has been the focus of several academic- and industrial-based studies worldwide with clear cross-fertilisation with physics (fluid mechanics etc), geology $\&$ geophysics (e.g., lithography, petrology, geochemistry, etc) and computer science (e.g., software engineering, algorithms, parallel processing, etc). The student demonstrated that he had a limited understanding of the main fundamental physics and technologies for this project.

\clearpage

%%%%%% 
%%%%%%
%%%%%%


\noindent{\bfseries\large MSc in Petroleum Engineering\hfill August, 2016}

\bigskip

\begin{center}
{\Large Review of the MSc Dissertation `Assessment of Completion Depth and Directional Drilling Efficiency for the Thermal Performance of a Geothermal Hot Dry Rock Reservoir' by Roberto Giannitelli}
\end{center}

\medskip

The dissertation aims to investigate drilling design strategies for reliable and sustainable heat extraction in hot dry rock reservoir systems. Mr Giannitelli undertook a rather superficial literature review of the main aspects of geothermal technology. He also performed a preliminary analysis of directional and depth drilling to obtain successful (or economic viable) energy production. The dissertation encompasses three main subject areas within Petroleum Engineering: thermal recovery, drilling design and hydraulic fracturing.

The manuscript is relatively well-written with a small number of typos and unrevised sentences. Few sentences are confusing and disconnected with no clear objectives and inter-connectivities. Most of all, the dissertation is well-structured with clear division and linkages between chapters, sections and paragraphs, leading to an easy and smooth reading.  A few comments,

\begin{enumerate}
\item The main aim of {\it Abstracts} is to briefly describe the work undertaken by the author. In general {\it Abstracts} are divided in 4 parts: (i) motivation, (ii) main objectives, (iii) summary of the main procedures / techniques / technologies (optional) and (iv) main findings. The current {\it Abstract} encompasses ii (in part) and iv.
%
\item The {\it Introduction} chapter usually has the same (but more in-depth and descriptive) four parts of the {\it Abstract} and a brief summary of the remaining of the work. In addition, it is always expected a few clear statements -re main background (thus recent innovations related to the main topic), initial literature review and, most of all, technological / scientific gaps in the current understanding. Also, it is expected a summary of the remaining sections at the end of the {\it Introduction}. In this dissertation, the {\it Introduction} section covered i, ii (in part), iii and iv. It is important to notice that aims and objectives were not clearly stated. 
%
\item The aim of {\it Literature Review} Sections/Chapters is to introduce the main fundamental (i.e., theoretical) aspects of the work and to assess (with critical analysis) previous academic and/or industrial developments on the main subject areas. In summary, literature review should focus on 3-5 subjects of the dissertation main topic and give an overview of past and current work (state-of-the-art) on them. Most of all, the section/chapter should highlight current gaps in specific knowledge. Literature review is the focus of `Chapter 2' with limited description and critical analysis of the work undertaken by other authors. 
%
\item Equations \underline{must} be explained in full and all terms used must be defined afterwards as part of the main text. In addition, equations must have the same font size as the main text.
%
\item All {\it References} follow different standards with missing fields and no clear distinction between articles, conference proceedings, reports (internal or external), book chapters, books, communications (internal or external) etc.  A few {\it references} used in the manuscript are incomplete and/or wrong. Regardless of the chosen citation style (e.g., ACS, AIP, AMS, IEEE, AIAA, etc) any reference {\bf must} contain the following fields: 
\begin{enumerate}
\item For journal papers: Authors, Paper Tittle, Journal Name, Volume, Pages, Year of publication;
\item For books: Authors, Book Tittle, Publisher, Year or Edition;
\item For book chapters: Authors, Chapter Tittle, Book Tittle, Editors, Publisher, Year or Edition;
\item For conference papers: Authors, Paper Tittle, Conference Tittle, Place (Country and/or City) where the conference was held, Year of the conference;
\item For reports,  private communications and Lecture Notes: Authors, Tittle, Place issued (Country and/or City and Institution where the document was originated), Year;
\item For PhD Thesis and MSc Dissertations: Author, Tittle, Institution (University and Department/School), Year.
\end{enumerate}  
Thus, for example:
\begin{enumerate}[label={[\arabic*]}]
\item P.L. Houtekamer and L. Mitchell, $\lq$Data Assimilation Using an Ensemble Kalman Filter Technique', {\it Monthly Weather Review}, 126:796-811, 1998.
\item K. Pruess, $\lq$Numerical Modelling of Gas Migration at a Proposed Repository for Low and Intermediate Level Nuclear Wastes', Technical Report LBL-25413, Lawrence Berkeley Laboratory, Berkeley (USA), 1990.
\item K. Aziz, A. Settari, {\it Fundamentals of Reservoir Simulation}, Elsevier Applied Science Publishers, New York (USA), 1986.
\item R.B. Lowrie, $\lq$Compact higher-Order Numerical Methods for Hyperbolic Conservation Laws', PhD Thesis, Department of Aerospace Engineering and Scientific Computing, University of Michigan (USA), 1996.
\end{enumerate} 
%
\item All Figures were wrongly referenced in the main text.
%
\item Use of {\it colloquial (informal / personal)} writing should be avoided. 
%
\item Analysis of the results (Figs. 10-17) are limited and based mostly on observation of the numerical simulations instead of fundamentals and/or literature. For example, what is the reason for the sudden decrease in temperatures in Figs 11-12? Depths of wells do not seem to be the (only) reason. Also conclusions on the directional drilling (Figs. 15-17) lack theoretical background (based on the literature review) that may support critical reasoning.   
% 
\end{enumerate}

In general, the dissertation was interesting to read and offered a good insight on cross-fertilisation of petroleum and geothermal engineering technologies. Results and conclusions were clearly aligned with aims and objectives with consistent discussion of the main technological aspects. 

The topic is very relevant for energy sector and each section has been the focus of several academic- and industrial-based studies worldwide with clear cross-fertilisation with physics (fluid mechanics, thermodynamics etc), geology $\&$ geophysics (e.g., lithography, petrology, geochemistry etc) and computer science (e.g., software engineering, reservoir simulation, parallel processing etc). The student demonstrated that he had a good understanding of the main fundamental physics and technologies for this project.


\clearpage
%%%%%% 
%%%%%%
%%%%%%


\noindent{\bfseries\large MSc in Petroleum Engineering\hfill August, 2016}

\bigskip

\begin{center}
{\Large Review of the MSc Dissertation `Flow and Heat Extraction Performance Simulation of Vertical Wells and Horizontal Wells to Optimise CO$_{2}$ EGS in HDR Geothermal' by Yuanda Yanrismet}
\end{center}

\medskip

The dissertation aims to investigate the impact of well design strategies and underburden heat flux for heat extraction in hot dry rock reservoir systems. Mr Yanrismet undertook a comprehensive literature review of the main aspects of geothermal energy technologies. He also performed a CFD analysis of heat and mass fluxes in a synthetic reservoir with distinct well design. The dissertation encompasses three main subject areas within Petroleum Engineering: thermal recovery, drilling design and hydraulic fracturing.

The manuscript is relatively well-written with a small number of typos and unrevised sentences. Few sentences are confusing and disconnected. Most of all, the dissertation is well-structured with clear division and linkages between chapters, sections and paragraphs, leading to an easy and smooth reading.  A few comments,

\begin{enumerate}
\item The main aim of {\it Abstracts} is to briefly describe the work undertaken by the author. In general {\it Abstracts} are divided in 4 parts: (i) motivation, (ii) main objectives, (iii) summary of the main procedures / techniques / technologies (optional) and (iv) main findings. The current {\it Abstract} encompasses i (in part) and iv.
%
\item The {\it Introduction} chapter usually has the same (but more in-depth and descriptive) four parts of the {\it Abstract} and a brief summary of the remaining of the work. In addition, it is always expected a few clear statements -re main background (thus recent innovations related to the main topic), initial literature review and, most of all, technological / scientific gaps in the current understanding. Also, it is expected a summary of the remaining sections at the end of the {\it Introduction}. In this dissertation, the {\it Introduction} section covered i, ii and iii (in part). It is important to notice that potential lacks of knowledge on the field were clearly highlighted. 
%
\item The aim of {\it Literature Review} Sections/Chapters is to introduce the main fundamental (i.e., theoretical) aspects of the work and to assess (with critical analysis) previous academic and/or industrial developments on the main subject areas. In summary, literature review should focus on 3-5 subjects of the dissertation main topic and give an overview of past and current work (state-of-the-art) on them. Most of all, the section/chapter should highlight current gaps in specific knowledge. Literature review is the focus of `Chapter 2' (although it was also spread through Chapters 3 and 4) with a comprehensive description and critical analysis of the work undertaken by other authors. 
%
\item Equations \underline{must} be explained in full and all terms used must be defined afterwards as part of the main text. In addition, equations must have the same font size as the main text.
%
\item Some {\it References} follow different standards with missing fields and no clear distinction between articles, conference proceedings, reports (internal or external), book chapters, books, communications (internal or external) etc.  Regardless of the chosen citation style (e.g., ACS, AIP, AMS, IEEE, AIAA, etc) any reference {\bf must} contain the following fields: 
\begin{enumerate}
\item For journal papers: Authors, Paper Tittle, Journal Name, Volume, Pages, Year of publication;
\item For books: Authors, Book Tittle, Publisher, Year or Edition;
\item For book chapters: Authors, Chapter Tittle, Book Tittle, Editors, Publisher, Year or Edition;
\item For conference papers: Authors, Paper Tittle, Conference Tittle, Place (Country and/or City) where the conference was held, Year of the conference;
\item For reports,  private communications and Lecture Notes: Authors, Tittle, Place issued (Country and/or City and Institution where the document was originated), Year;
\item For PhD Thesis and MSc Dissertations: Author, Tittle, Institution (University and Department/School), Year.
\end{enumerate}  
Thus, for example:
\begin{enumerate}[label={[\arabic*]}]
\item P.L. Houtekamer and L. Mitchell, $\lq$Data Assimilation Using an Ensemble Kalman Filter Technique', {\it Monthly Weather Review}, 126:796-811, 1998.
\item K. Pruess, $\lq$Numerical Modelling of Gas Migration at a Proposed Repository for Low and Intermediate Level Nuclear Wastes', Technical Report LBL-25413, Lawrence Berkeley Laboratory, Berkeley (USA), 1990.
\item K. Aziz, A. Settari, {\it Fundamentals of Reservoir Simulation}, Elsevier Applied Science Publishers, New York (USA), 1986.
\item R.B. Lowrie, $\lq$Compact higher-Order Numerical Methods for Hyperbolic Conservation Laws', PhD Thesis, Department of Aerospace Engineering and Scientific Computing, University of Michigan (USA), 1996.
\end{enumerate} 
%
\item A few Figures were wrongly referenced in the main text. Also, some of them (e.g., Figs. 2.4, 2.5, E1, F1) are 'floating' with no explanation/description/referencing in the main text.
%
\item Use of {\it colloquial (informal / personal)} writing should be avoided. 
%
\item Literature review on the flow mechanics and well drilling (Sections 2.1-2.2) are very superficial with nearly no explanation of definitions, concepts, assumptions and equations.
%
\item Very good analysis of the results with consistent use of existing references and fundamentals.
% 
\end{enumerate}

In general, the dissertation was very interesting to read and offered a good insight on cross-fertilisation of petroleum and geothermal engineering technologies. Results and conclusions were clearly aligned with aims and objectives with consistent discussion of fundamentals and main technological aspects. 

The topic is very relevant for energy sector and each section has been the focus of several academic- and industrial-based studies worldwide with clear cross-fertilisation with physics (fluid mechanics, thermodynamics etc), geology $\&$ geophysics (e.g., lithography, petrology, geochemistry etc) and computer science (e.g., software engineering, reservoir simulation, parallel processing etc). The student demonstrated that he had a very good understanding of the main fundamental physics and technologies for this project.


\clearpage
%%%%%
%%%%%
%%%%%

\noindent{\bfseries\large MSc in Petroleum Engineering\hfill August, 2016}

\bigskip

\begin{center}
{\Large Review of the MSc Dissertation `Near-well Upscalling Techniques for Waterflooding' by Raushan Sabiryanov}
\end{center}

\medskip

The dissertation aims to investigate upscalling methods current used by industry to represent near-well regions and to assess dual coarsening-homogeneisation techniques implemented in ECLIPSE. Mr Sabiryanov undertook a comprehensive literature review on some of the main aspects involving upscalling methods and technologies. The dissertation encompass three main subject areas within Petroleum Engineering: reservoir modelling and simulation, computational geophysics and upscalling techniques.

The manuscript is relatively written with a number of typos and unrevised sentences. A number of sentences are very confusing and disconnected with no clear objectives but overall the dissertation is well-structured with clear division and linkage between chapters and sections, leading to a smooth reading. A few comments:
\begin{enumerate}
\item The main aim of {\it Abstracts} is to briefly describe the work undertaken by the author. In general {\it Abstracts} are divided in 4 parts: (i) motivation, (ii) main objectives, (iii) summary of the main procedures / techniques / technologies (optional) and (iv) main findings. The current {\it Abstract} encompasses i, ii and iii (in part).
%
\item The {\it Introduction} chapter usually has the same (but more in-depth and descriptive) four parts of the {\it Abstract} and a brief summary of the remaining of the work. In addition, it is always expected a few clear statements -re main background (thus recent innovations related to the main topic), initial literature review and, most of all, technological / scientific gaps in the current understanding. Also, it is expected a summary of the remaining sections at the end of the {\it Introduction}. In this dissertation, the {\it Introduction} section covered i and ii.  Potential lacks of knowledge on the field were not covered neither in this chapter nor elsewhere. 
%
\item The aim of {\it Literature Review} Sections/Chapters is to introduce the main fundamental (i.e., theoretical) aspects of the work and to assess (with critical analysis) previous academic and/or industrial developments on the main subject areas. In summary, literature review should focus on 3-5 subjects of the dissertation main topic and give an overview of past and current work (state-of-the-art) on them. Most of all, the section/chapter should highlight current gaps in specific knowledge. Literature review is the focus of `Chapter 2' with a comprehensive description (although with no critical analysis) of the work undertaken by other authors. Section 2.8 was supposed to describe (and discuss) upscalling methods within ECLIPSE used in Chapter 3, however the section focused on show a partial upscalling template that can be conducted with the software , with no real relationship with the remaining of the chapter.
%
\item Equations \underline{must} be explained in full and all terms used must be defined afterwards as part of the main text. Some terms were assigned with different symbols throughout the work with no explanation (e.g., $K$ and $k$ for permeability, $\phi$ and $\varphi$ for porosity etc). In addition, equations must have the same font size as the main text.
%
\item All {\it References} follow different standards with missing fields and no clear distinction between articles, conference proceedings, reports (internal or external), book chapters, books, communications (internal or external) etc.  A few {\it references} used in the manuscript are incomplete and/or wrong. Regardless of the chosen citation style (e.g., ACS, AIP, AMS, IEEE, AIAA, etc) any reference {\bf must} contain the following fields: 
\begin{enumerate}
\item For journal papers: Authors, Paper Tittle, Journal Name, Volume, Pages, Year of publication;
\item For books: Authors, Book Tittle, Publisher, Year or Edition;
\item For book chapters: Authors, Chapter Tittle, Book Tittle, Editors, Publisher, Year or Edition;
\item For conference papers: Authors, Paper Tittle, Conference Tittle, Place (Country and/or City) where the conference was held, Year of the conference;
\item For reports,  private communications and Lecture Notes: Authors, Tittle, Place issued (Country and/or City and Institution where the document was originated), Year;
\item For PhD Thesis and MSc Dissertations: Author, Tittle, Institution (University and Department/School), Year.
\end{enumerate}  
Thus, for example:
\begin{enumerate}[label={[\arabic*]}]
\item P.L. Houtekamer and L. Mitchell, $\lq$Data Assimilation Using an Ensemble Kalman Filter Technique', {\it Monthly Weather Review}, 126:796-811, 1998.
\item K. Pruess, $\lq$Numerical Modelling of Gas Migration at a Proposed Repository for Low and Intermediate Level Nuclear Wastes', Technical Report LBL-25413, Lawrence Berkeley Laboratory, Berkeley (USA), 1990.
\item K. Aziz, A. Settari, {\it Fundamentals of Reservoir Simulation}, Elsevier Applied Science Publishers, New York (USA), 1986.
\item R.B. Lowrie, $\lq$Compact higher-Order Numerical Methods for Hyperbolic Conservation Laws', PhD Thesis, Department of Aerospace Engineering and Scientific Computing, University of Michigan (USA), 1996.
\end{enumerate} 
%
\item Quality of some figures are very poor. Also, some figures (and  tables) are 'floating' with no explanation/description/referencing in the main text.
%
\item Use of {\it colloquial (informal / personal)} writing should be avoided. 
%
\item Description of the numerical experiments was very confusing and it was not clear the relation with the aims and objectives.
%
\item Analysis of numerical results lacks theoretical depth and reasoning. In particular, Section 3.3 is very confusing and mostly describes the ECLIPSE's workflow towards upscalling. 
%
\item Conclusion section is well-written and based on results shown in Chapter 3, although not clearly linked with aims and objectives. Recommendations for future work are clear connected with the subject and demonstrated a basic understanding of the challenges in the field. 
% 
\end{enumerate}

In general, the dissertation was very interesting to read and offered a good insight on current challenges in reservoir simulation and upscalling techniques. Results and conclusions were not completely aligned with aims and objectives. Also, there is a clear lack of discussion on the main objective: sensitivity analysis of flow dynamics parameters during permeability upscalling. A few crucial concepts are either not defined or not correct.

The topic is very relevant for energy sector and each section has been the focus of several academic- and industrial-based studies worldwide with clear cross-fertilisation with physics (fluid mechanics thermodynamics etc), geology $\&$ geophysics (e.g., lithography, petrology, geochemistry, etc), petroleum engineering (e.g., reservoir modelling and simulation etc) and computer science (e.g., software engineering, parallel processing, etc). The student demonstrated that he had a limited understanding of the main fundamental physics and technologies for this project.

\clearpage


%%%%%
%%%%%
%%%%%

\noindent{\bfseries\large MSc in Petroleum Engineering\hfill August, 2016}

\bigskip

\begin{center}
{\Large Review of the MSc Dissertation `Reservoir Discretisation Study: Absolute Permeability Upscalling Techniques' by Chinedu Ubachukwu}
\end{center}

\medskip

The dissertation aims to investigate upscalling methods used by industry to represent heterogeneous oil and gas reservoirs and to assess several coarsening-homogeneisation techniques in the industry-standard SPE10 benchmark. Mr Ubachukwu undertook a comprehensive literature review on the main aspects involving reservoir engineering and upscalling methods and technologies. The dissertation encompass three main subject areas within Petroleum Engineering: reservoir modelling and simulation, computational geophysics and property homogenisation techniques.

The manuscript is well-written with a small number of typos and unrevised sentences. It is well-structured with clear division and linkage between chapters and sections, leading to a smooth and engaging reading. A few comments:
\begin{enumerate}
\item The main aim of {\it Abstracts} is to briefly describe the work undertaken by the author. In general {\it Abstracts} are divided in 4 parts: (i) motivation, (ii) main objectives, (iii) summary of the main procedures / techniques / technologies (optional) and (iv) main findings. The current {\it Abstract} encompasses i, ii (in part) and iv.
%
\item The {\it Introduction} chapter usually has the same (but more in-depth and descriptive) four parts of the {\it Abstract} and a brief summary of the remaining of the work. In addition, it is always expected a few clear statements -re main background (thus recent innovations related to the main topic), initial literature review and, most of all, technological / scientific gaps in the current understanding. Also, it is expected a summary of the remaining sections at the end of the {\it Introduction}. In this dissertation, the {\it Introduction} section covered all four parts.  Potential lacks of knowledge on the field were not covered in chapter but it is highlighted in Chapters 2- 4. 
%
\item The aim of {\it Literature Review} (LR) Sections/Chapters is to introduce the main fundamental (i.e., theoretical) aspects of the work and to assess (with critical analysis) previous academic and/or industrial developments on the main subject areas. In summary, LR should focus on 3-5 subjects of the dissertation main topic and give an overview of past and current work (state-of-the-art) on them. Most of all, the section/chapter should highlight current gaps in specific knowledge. Literature review is the focus of `Chapter 2' (although it is also spread over Chapters 3 and 4) with a comprehensive description (and critical analysis) of the work undertaken by other authors. 
%
\item A few {\it References} follow different standards with missing fields and no clear distinction between articles, conference proceedings, reports (internal or external), book chapters, books, communications (internal or external) etc.  
%
\item Quality of some figures is poor (i.e., low resolution). Also, some figures and tables (Fig, 4,23, A2-5) are 'floating' with no explanation/description/referencing in the main text.
%
\item Some symbols were used (in figures) without prior definition (e.g., Fig 2.6 K$_{\text{rel}}$, S$_{wc}$ etc).
%
\item Analysis of numerical results were strongly based on theoretical concepts, literature and reasoning. 
%
\item Conclusion section is well-written and based on results shown in Chapter 4 with clear link with stated aims and objectives. Recommendations for future work are clear connected with the subject and demonstrated an excellent understanding of the challenges in the field. 
% 
\end{enumerate}

In general, the dissertation was very interesting to read and offered a good insight on current challenges in reservoir simulation and upscalling techniques. Results and conclusions were completely aligned with stated aims and objectives. Excellent critical analysis of both the major development on the field and numerical results. 

The topic is very relevant for energy sector and each section has been the focus of several academic- and industrial-based studies worldwide with clear cross-fertilisation with physics (fluid mechanics thermodynamics etc), geology $\&$ geophysics (e.g., lithography, petrology, geochemistry, etc), petroleum engineering (e.g., reservoir modelling and simulation etc) and computer science (e.g., software engineering, parallel processing, etc). The student demonstrated that he had an excellent understanding of the main fundamental physics and technologies for this project.

\clearpage 


\noindent{\bfseries\large MSc in Petroleum Engineering\hfill August, 2016}

\bigskip

\begin{center}
{\Large Review of the MSc Dissertation `Assessment of Completion Depth and Directional Drilling Efficiency for the Thermal Performance of a Geothermal Hot Dry Rock Reservoir' by Roberto Giannitelli}
\end{center}

\medskip

\begin{enumerate}
%
    \item {\bf Presentation and Style of Writing:}
        \begin{description}
            \item[Examiner 1:] Comments:
                   \begin{enumerate}
                      \item The manuscript is relatively well-written with a small number of typos and unrevised sentences. Few sentences are confusing and disconnected with no clear objectives and inter-connectivities. Most of all, the dissertation is well-structured with clear division and linkages between chapters, sections and paragraphs, leading to an easy and smooth reading.
                      \item Equations must be explained in full and all terms used must be defined afterwards as part of the main text. In addition, equations must have the same font size as the main text.
                      \item All References follow different standards with missing fields and no clear distinction between articles, conference proceedings, reports (internal or external), book chapters, books, communications (internal or external) etc.  A few references used in the manuscript are incomplete and/or wrong.
                      \item All Figures were wrongly referenced in the main text.
                      \item Use of colloquial (informal / personal) writing should be avoided.	
                    \end{enumerate}
            \item[Examiner 2:] The thesis is presented well. Plots and diagrams are neat and tidy. The thesis structure is good, however, the thesis appears to be short. It only contains 4 chapters. Literature review chapter is short (4 pages long). The font size of mathematical symbols are too large compared to font size of the main text. A notation list should have been included. Abstract should have included more details, e.g. what software was used to run simulations. 
        \end{description}
%
    \item {\bf Technical Content and Merit of the Dissertation:}
        \begin{description}
            \item[Examiner 1:] Comments:
                 \begin{enumerate}
                     \item The main aim of Abstracts is to briefly describe the work undertaken by the author. In general Abstracts are divided in 4 parts: (i) motivation, (ii) main objectives, (iii) summary of the main procedures / techniques / technologies (optional) and (iv) main findings. The current  Abstract encompasses ii (in part) and iv.
                     \item The Introduction chapter usually has the same (but more in-depth and descriptive) four parts of the  Abstract and a brief summary of the remaining of the work. In addition, it is always expected a few clear statements -re main background (thus recent innovations related to the main topic), initial literature review and, most of all, technological / scientific gaps in the current understanding. Also, it is expected a summary of the remaining sections at the end of the Introduction. In this dissertation, the Introduction section covered i, ii (in part), iii and iv. It is important to notice that aims and objectives were not clearly stated. 
                      \item The aim of  Literature Review (LR) Sections/Chapters is to introduce the main fundamental (i.e., theoretical) aspects of the work and to assess (with critical analysis) previous academic and/or industrial developments on the main subject areas. In summary, LR should focus on 3-5 subjects of the dissertation main topic and give an overview of past and current work (state-of-the-art) on them. Most of all, LR should highlight current gaps in specific knowledge. Literature review is the focus of `Chapter 2' with limited description and critical analysis of the work undertaken by other authors. 
                      \item Description of the numerical experiments were slightly confusing. Analysis of the results and conclusions were rather limited and based on observation of the plots with minimal integration with fundamentals.
                \end{enumerate}
            \item[Examiner 2:] The student conducted a study related to drilling and reservoir enhancement plan evaluation for a typical geothermal HDR (hot dry rock) reservoir with the main focus on performances of specific drilling decisions. A software package TOUGH2 that allows to execute multiphase and multi-dimensional flow simulation was used to study fluid flows through porous media. Several simulations have been carried out to assess (1) effects of the vertical position between injection and production wells and (2) the influence of a non-vertical completion section on the productivity of these wells.  
        \end{description} 
%
    \item {\bf Evidence of Critical Reasoning:}
        \begin{description}
            \item[Examiner 1:] Comments:
               \begin{enumerate}
                     \item The dissertation aims to investigate drilling design strategies for reliable and sustainable heat extraction in hot dry rock reservoir systems. The student undertook a rather superficial literature review of the main aspects of geothermal technology. 
                     \item The student also performed a preliminary analysis of directional and depth drilling to obtain successful (or economic viable) energy production. Design of experiments and awareness of technology limitations were clearly assessed and stated.
                     \item Overall, design of the experiment and analysis of the results indicated a basic understanding of the fundamentals but with limited knowledge of advanced material relevant for the work (i.e., thermal recovery, drilling design and hydraulic fracturing).
                \end{enumerate}

            \item[Examiner 2:] Results are presented and discussed well. However, it would have been beneficial if simulation results were compared with the results available in literature. Conclusions are also presented well and summarized focusing on productivity, generalization and sustainability issues of the hot dry rock technology. Although, the thesis would have been improved if the conclusion chapter was linked with both project objectives and literature review chapters. 
 
 
        \end{description}
%
    \item {\bf Student's Overall Performance:}
        \begin{description}
            \item[Examiner 1:]I n general, the dissertation was interesting to read and offered a good insight on cross-fertilisation of petroleum and geothermal engineering technologies. Results and conclusions were clearly aligned with aims and objectives with consistent discussion of the main technological aspects. 

The topic is very relevant for energy sector and each section has been the focus of several academic- and industrial-based studies worldwide with clear cross-fertilisation with physics (fluid mechanics, thermodynamics etc), geology $\&$ geophysics (e.g., lithography, petrology, geochemistry etc) and computer science (e.g., software engineering, reservoir simulation, parallel processing etc). The student demonstrated that he had a good understanding of the main fundamental physics and technologies for this project.

            \item[Examiner 2:]Overall, the thesis is at MSc level that is a commendation.
        \end{description}
%
\end{enumerate}

\clearpage
%%%%%% 
%%%%%%
%%%%%%

\noindent{\bfseries\large MSc in Petroleum Engineering\hfill August, 2016}

\bigskip

\begin{center}
{\Large Review of the MSc Dissertation `Flow and Heat Extraction Performance Simulation of Vertical Wells and Horizontal Wells to Optimise CO$_{2}$ EGS in HDR Geothermal' by Yuanda Yanrismet}
\end{center}

\medskip

\begin{enumerate}
%
    \item {\bf Presentation and Style of Writing:}
        \begin{description}
            \item[Examiner 1:] Comments:
                \begin{enumerate}
                   \item The manuscript is relatively well-written with a small number of typos and unrevised sentences. Few sentences are confusing and disconnected with no inter-connectivity. Most of all, the dissertation is well-structured with clear division and linkages between chapters, sections and paragraphs, leading to an easy and smooth reading.
                   \item Equations must be explained in full and all terms used must be defined afterwards as part of the main text. In addition, equations must have the same font size as the main text.
                   \item Some References follow different standards with missing fields and no clear distinction between articles, conference proceedings, reports (internal or external), book chapters, books, communications (internal or external) etc.
                   \item A few Figures were wrongly referenced in the main text. Also, some of them (e.g., Figs. 2.4, 2.5, E1, F1) are 'floating' with no explanation / description / referencing in the main text.
                   \item Use of  colloquial (informal / personal) writing should be avoided. 
                \end{enumerate}
            \item[Examiner 2:] Good overview of the research topic. Satisfactory presetation. However, there are some typos and inconsistency in the thesis. Some signs of presentation information in a new light or drawing stands together in a new framework.
        \end{description}
%
    \item {\bf Technical Content and Merit of the Dissertation:}
        \begin{description}
            \item[Examiner 1:] Comments,
                \begin{enumerate}
                   \item The main aim of Abstracts is to briefly describe the work undertaken by the author. In general Abstracts are divided in 4 parts: (i) motivation, (ii) main objectives, (iii) summary of the main procedures / techniques / technologies (optional) and (iv) main findings. The current Abstract encompasses i (in part) and iv.
                   \item The Introduction chapter usually has the same (but more in-depth and descriptive) four parts of the Abstract and a brief summary of the remaining of the work. In addition, it is always expected a few clear statements -re main background (thus recent innovations related to the main topic), initial literature review and, most of all, technological / scientific gaps in the current understanding. Also, it is expected a summary of the remaining sections at the end of the Introduction. In this dissertation, the Introduction section covered i, ii and iii (in part). It is important to notice that potential lacks of knowledge on the field were clearly highlighted. 
                   \item The aim of  Literature Review Sections/Chapters is to introduce the main fundamental (i.e., theoretical) aspects of the work and to assess (with critical analysis) previous academic and/or industrial developments on the main subject areas. In summary, literature review should focus on 3-5 subjects of the dissertation main topic and give an overview of past and current work (state-of-the-art) on them. Most of all, the section/chapter should highlight current gaps in specific knowledge. Literature review is the focus of `Chapter 2' (although it was also spread through Chapters 3 and 4) with a comprehensive description and critical analysis of the work undertaken by other authors.
                    \item Very good analysis of the results with consistent use of existing references and fundamentals.
                \end{enumerate}
            \item[Examiner 2:] Basic clear level of understanding of the relevant concepts. However, I think it is repeated piece of work without evidence of further appreciation of subject. Lacking illustrative originality. Develops and analyses the core issues by the assessment to a limited extent but shows understanding of relevant material.
        \end{description}
%
    \item {\bf Evidence of Critical Reasoning:}
        \begin{description}
            \item[Examiner 1:] Comments,
                \begin{enumerate}
                   \item The dissertation aims to investigate the impact of well design strategies and underburden heat flux for heat extraction in hot dry rock reservoir systems. The student undertook a comprehensive literature review of the main aspects of geothermal energy technologies. 
                   \item He also performed a CFD analysis of heat and mass fluxes in a synthetic reservoir with distinct well design. 
                   \item In summary, design of numerical experiments and analysis of the results indicated a strong understanding of the fundamentals and the current available technologies relevant for the project (e.g., thermal recovery and well design). 
                \end{enumerate}
            \item[Examiner 2:] Arguments are well constructed but do not develop sufficiently some significant issues. Not enough evidence for critical reasoning.
        \end{description}
%
    \item {\bf Student's Overall Performance:}
        \begin{description}
            \item[Examiner 1:] In general, the dissertation was interesting to read and offered a good insight on cross-fertilisation of petroleum and geothermal engineering technologies. Results and conclusions were clearly aligned with aims and objectives with consistent discussion of the main technological aspects. 

The topic is very relevant for energy sector and each section has been the focus of several academic- and industrial-based studies worldwide with clear cross-fertilisation with physics (fluid mechanics, thermodynamics etc), geology $\&$ geophysics (e.g., lithography, petrology, geochemistry etc) and computer science (e.g., software engineering, reservoir simulation, parallel processing etc). The student demonstrated that he had a very good understanding of the main fundamental physics and technologies for this project.
            \item[Examiner 2:] The student is between upper and lower second class.
        \end{description}
%
\end{enumerate}

\clearpage
%%%%%
%%%%%
%%%%%

\noindent{\bfseries\large MSc in Petroleum Engineering\hfill August, 2016}

\bigskip

\begin{center}
{\Large Review of the MSc Dissertation `Near-well Upscalling Techniques for Waterflooding' by Raushan Sabiryanov}
\end{center}

\medskip

\begin{enumerate}
%
    \item {\bf Presentation and Style of Writing:}
        \begin{description}
            \item[Examiner 1:] Comments,
                \begin{enumerate}
                   \item The manuscript is relatively written with a number of typos and unrevised sentences. A number of sentences are very confusing and disconnected with no clear objectives but overall the dissertation is well-structured with clear division and linkage between chapters and sections, leading to a smooth reading.
                   \item All References follow different standards with missing fields and no clear distinction between articles, conference proceedings, reports (internal or external), book chapters, books, communications (internal or external) etc.  A few references used in the manuscript are incomplete and/or wrong.
                   \item Equations must be explained in full and all terms used must be defined afterwards as part of the main text. Some terms were assigned with different symbols throughout the work with no explanation. In addition, equations must have the same font size as the main text.
                   \item Quality of some figures are very poor. Also, some figures (and  tables) are 'floating' with no explanation/description/referencing in the main text.
                   \item Use ofcolloquial (informal / personal) writing should be avoided
                \end{enumerate}
            \item[Examiner 2:] Generally well written, however some figures appear with little  to no discussion in text. Appendices are of questionable use, being merely unexplained tables of numbers or cryptic output file listing that add little to the document. In some places, results are embedded in paragraphs or first discussed in conclusions. For the most part, well structured.
        \end{description}
%
    \item {\bf Technical Content and Merit of the Dissertation:}
        \begin{description}
            \item[Examiner 1:] Comments,
                \begin{enumerate}
                   \item The main aim of Abstracts is to briefly describe the work undertaken by the author. In general Abstracts are divided in 4 parts: (i) motivation, (ii) main objectives, (iii) summary of the main procedures / techniques / technologies (optional) and (iv) main findings. The current Abstract encompasses i, ii and iii (in part).
                   \item The Introduction chapter usually has the same (but more in-depth and descriptive) four parts of the Abstract and a brief summary of the remaining of the work. In addition, it is always expected a few clear statements -re main background (thus recent innovations related to the main topic), initial literature review and, most of all, technological / scientific gaps in the current understanding. Also, it is expected a summary of the remaining sections at the end of the Introduction. In this dissertation, the Introduction chapter covered i and ii.  Potential lacks of knowledge on the field were not covered neither in this chapter nor elsewhere. 
                   \item The aim of Literature Review Sections/Chapters is to introduce the main fundamental (i.e., theoretical) aspects of the work and to assess (with critical analysis) previous academic and/or industrial developments on the main subject areas. In summary, literature review should focus on 3-5 subjects of the dissertation main topic and give an overview of past and current work (state-of-the-art) on them. Most of all, the section/chapter should highlight current gaps in specific knowledge. Literature review is the focus of `Chapter 2' with a comprehensive description (although with no critical analysis) of the work undertaken by other authors. Section 2.8 was supposed to describe (and discuss) upscalling methods within ECLIPSE used in Chapter 3, however the section focused on show a partial upscalling template that can be conducted with the software, with no real relationship with the remaining of the chapter.
                   \item Description of the numerical experiments was very confusing and it was not clear its relationship with the stated aims and objectives.
                   \item Analysis of numerical results lacks theoretical depth and reasoning. In particular, Section 3.3 is very confusing and mostly describes the ECLIPSE's workflow towards upscalling instead of an analysis of the results based on the methodology/technology in software (that should have been fully described elsewhere).
                   \item Conclusion section is well-written and based on results shown in Chapter 3, although not clearly linked with aims and objectives. Recommendations for future work are clear connected with the subject and demonstrated a basic understanding of the challenges in the field. 
                \end{enumerate}
            \item[Examiner 2:] Introduction leaves room for some confusion as to what the aims/objectives are, but the information is there. Would benefit from more explicit outlining of aims and objectives. Review of work in the area appears detailed and in-depth. Results are fairly well explored, but may nor have been analysed as well as they could be. Some discussion brought into conclusion that should be covered in the discussions section of the thesis. Elements of the data not discussed or brought into the conclusion (such as discrepancy after upscaling models and how serious this discrepancy would be to industry).
        \end{description}
%
    \item {\bf Evidence of Critical Reasoning:}
        \begin{description}
            \item[Examiner 1:] Comments,
                \begin{enumerate}
                   \item The dissertation aims to investigate upscaling methods current used by industry to represent near-well regions and to assess dual coarsening-homogeneisation techniques implemented in ECLIPSE. The student undertook a comprehensive literature review on some of the main aspects involving upscaling methods and technologies. Although it lacked critical analysis on the literature. The dissertation encompass three main subject areas within Petroleum Engineering: reservoir modelling and simulation, computational geophysics and upscalling techniques.
                   \item The student performed numerical simulations using ECLIPSE software to investigate the impact of coarsen grids and upscalling permeability fields in flow dynamics. Overall critical analysis based on these simulations were very superficial and mostly not based on either literature or fundamentals.
                   \item In summary, design of numerical experiments and analysis of the results indicated a limited understanding of the fundamentals and current available technologies for the project. 
                \end{enumerate}
            \item[Examiner 2:] Literature review definitely relevant to aims and objectives. Conclusions drawn seem relevant and demonstrate understanding. Implications of results perhaps not discussed to the level they should be.
        \end{description}
%
    \item {\bf Student's Overall Performance:}
        \begin{description}
            \item[Examiner 1:] In general, the dissertation was very interesting to read and offered a good insight on current challenges in reservoir simulation and upscalling techniques. Results and conclusions were not completely aligned with aims and objectives. Also, there is a clear lack of discussion on the main objective: sensitivity analysis of flow dynamics parameters during permeability upscalling. A few crucial concepts are either not defined or not correct.

The topic is very relevant for energy sector and each section has been the focus of several academic- and industrial-based studies worldwide with clear cross-fertilisation with physics (fluid mechanics thermodynamics etc), geology $\&$ geophysics (e.g., lithography, petrology, geochemistry, etc), petroleum engineering (e.g., reservoir modelling and simulation etc) and computer science (e.g., software engineering, parallel processing, etc). The student demonstrated that he had a good but limited understanding of the main fundamental physics and technologies for this project.

            \item[Examiner 2:] Reasonable thesis, if poorly structured in some places. Not as in-depth and analysis as there could be, but definitely evidence of understanding and reasoning.
        \end{description}
%
\end{enumerate}
\clearpage


%%%%%
%%%%%
%%%%%

\noindent{\bfseries\large MSc in Petroleum Engineering\hfill August, 2016}

\bigskip

\begin{center}
{\Large Review of the MSc Dissertation `Reservoir Discretisation Study: Absolute Permeability Upscalling Techniques' by Chinedu Ubachukwu}
\end{center}

\medskip

\begin{enumerate}
%
    \item {\bf Presentation and Style of Writing:}
        \begin{description}
            \item[Examiner 1:] Comments,
                \begin{enumerate}
                   \item The manuscript is well-written with a small number of typos and unrevised sentences. It is well-structured with clear division and linkage between chapters and sections, leading to a smooth and engaging reading.
                   \item A few References follow different standards with missing fields and no clear distinction between articles, conference proceedings, reports (internal or external), book chapters, books, communications (internal or external) etc. 
                   \item Quality of some figures is poor (i.e., low resolution). Also, some figures and tables (Fig, 4,23, A2-5) are 'floating' with no explanation/description/referencing in the main text.
                   \item Some symbols were used (in figures) without prior definition (e.g., in Fig. 2.6: Krel, Swc, Krwro etc).
                \end{enumerate}
            \item[Examiner 2:] Overall the presentation of this dissertation is excellent, each paragraph works well to construct the dissertation. The quality of English is also very good, although some minor misuse of words.
        \end{description}
%
    \item {\bf Technical Content and Merit of the Dissertation:}
        \begin{description}
            \item[Examiner 1:] Comments,
                \begin{enumerate}
                   \item The main aim of Abstracts is to briefly describe the work undertaken by the author. In general Abstracts are divided in 4 parts: (i) motivation, (ii) main objectives, (iii) summary of the main procedures / techniques / technologies (optional) and (iv) main findings. The current Abstract encompasses i, ii (in part) and iv.
                   \item The Introduction chapter usually has the same (but more in-depth and descriptive) four parts of the Abstract and a brief summary of the remaining of the work. In addition, it is always expected a few clear statements -re main background (thus recent innovations related to the main topic), initial literature review and, most of all, technological / scientific gaps in the current understanding. Also, it is expected a summary of the remaining sections at the end of the Introduction. In this dissertation, the Introduction section covered all four parts.  Potential lacks of knowledge on the field were not covered in chapter but it is highlighted in Chapters 2- 4. 
                   \item The aim of  Literature Review (LR) Sections/Chapters is to introduce the main fundamental (i.e., theoretical) aspects of the work and to assess (with critical analysis) previous academic and/or industrial developments on the main subject areas. In summary, LR should focus on 3-5 subjects of the dissertation main topic and give an overview of past and current work (state-of-the-art) on them. Most of all, the section/chapter should highlight current gaps in specific knowledge. Literature review is the focus of `Chapter 2' (although it is also spread over Chapters 3 and 4) with a comprehensive description and critical analysis of the work undertaken by other authors.
                   \item Analysis of numerical results were strongly based on theoretical concepts, literature and reasoning.
                \end{enumerate}
            \item[Examiner 2:] Overall, the contents of this dissertation is very good at MSc level. The abstract clearly summarised the main contribution of the dissertation, and the introduction gave a short overview of the presentation. The technical contents of the dissertation is very good for a 12 weeks project, although lack the novelty. The presented single phase upscaling is well-documented in the textbook, and the multiphase upscaling technique is not well presented here in the dissertation.
        \end{description}
%
    \item {\bf Evidence of Critical Reasoning:}
        \begin{description}
            \item[Examiner 1:] Comments,
                \begin{enumerate}
                   \item The dissertation aims to investigate upscalling methods used by industry to represent heterogeneous oil and gas reservoirs and to assess several coarsening-homogeneisation techniques in the industry-standard SPE10 benchmark. The student undertook a comprehensive literature review on the main aspects involving reservoir engineering and upscalling methods and technologies. The dissertation encompass three main subject areas within Petroleum Engineering: reservoir modelling and simulation, computational geophysics and property homogenisation techniques.
                   \item He designed and performed a large number of numerical simulations to support his investigations. Results were gathered and critically analysed based on current literature.
                   \item In summary, design of numerical experiments, data manipulation and analysis of results indicated a strong understanding of the fundamentals and the current available technologies relevant for the project (upscalling techniques for reservoir simulation).
                \end{enumerate}
            \item[Examiner 2:] The literature review is good, but lack of evidence of critically analyse and interpret work of other authors. The results are well presented, and the conclusions were drawn based on the findings of the work, while lack comparison with that from other authors.
        \end{description}
%
    \item {\bf Student's Overall Performance:}
        \begin{description}
            \item[Examiner 1:] In general, the dissertation was very interesting to read and offered a good insight on current challenges in reservoir simulation and upscalling techniques. Results and conclusions were completely aligned with stated aims and objectives. Excellent critical analysis of both the major development on the field and numerical results. 

The topic is very relevant for energy sector and each section has been the focus of several academic- and industrial-based studies worldwide with clear cross-fertilisation with physics (fluid mechanics thermodynamics etc), geology $\&$ geophysics (e.g., lithography, petrology, geochemistry, etc), petroleum engineering (e.g., reservoir modelling and simulation etc) and computer science (e.g., software engineering, parallel processing, etc). The student demonstrated that he had an excellent understanding of the main fundamental physics and technologies for this project.
            \item[Examiner 2:] Overall, based on my judgement, this dissertation is very good at MSc level.
        \end{description}
%
\end{enumerate}


%%%%%%%
%%%%%%%
%%%%%%%

%\lipsum % Text before
\afterpage{%
    \clearpage% Flush earlier floats (otherwise order might not be correct)
    \thispagestyle{empty}% empty page style (?)
    \begin{landscape}% Landscape page
        \centering % Center table


\Huge{BEng and MEng Study Assessment -- Thesis (EG4013)}\\
\huge{(Review + Feedback)}\\
\huge{May 2016}
%\end{center}
\normalsize



\bigskip

\begin{center}
\begin{tabular}{l c c c | c }
\hline
                   & {\bf Presentation $\&$}    & {\bf Technical Content $\&$}  & {\bf Critical}            &   {\bf Total}         \\
                   & {\bf Style} (20/30$\%$)    & {\bf Merit} (50/70$\%$)       & {\bf Reasoning} (30$\%$)  &  {\bf ($\%$/SGS)}      \\

Nahian Chowdhury   &             54             &           58                  &           53              &        55.70 / 13 (C2) \\
Allan W. MacLeod   &             70             &           74                  &           76              &        73.80 / 19 (A4) \\
Molly Gray         &             75             &           67                  &           56              &        65.30 / 16 (B2) \\
Rasul Aliyev       &             74             &           58                  &           --              &        62.80 / 16 (B2) \\
Craig Barbour      &             57             &           48                  &           --              &        50.35 / 12 (C3) \\
\hline
\end{tabular}
\end{center}


    \end{landscape}
    \clearpage% Flush page
}

%\lipsum % Text after

\vfill

\clearpage


%%%%%% 
%%%%%%
%%%%%%

\noindent{\bfseries\large EG4515 -- MEng Thesis \hfill May, 2016}

\bigskip

\begin{center}
  {\Large Review of the MEng Thesis `Optimization of Energy Systems in Smart Cities' by Nahian Chowdhury}
\end{center}

\begin{description}
    \item[Examiner 1]: The dissertation aims to study energy systems in a 'smart cities' framework, focusing on strategies for local power generation and its integration with national power grid.  In particular, the manuscript investigates tri-generation systems (\ie electricity, heating and cooling) based on biomass source. Ms Chowdhury undertook a comprehensive review of design and optimisation of power generation systems, and also performed an energy system design involving `Trichy City' in India as a case-study. This involved specific design (and in a limited way optimisation) of fuel cells (as main electricity output), organic Rankine cycle for the heat waste from the fuel cell (derived from biomass gasification) and refrigeration absorption cycle. The design was performed using Aspen Hysis. The dissertation encompasses three main subject areas within the main topic (power systems): engineering thermodynamics, environmental technologies and process system optimisation. 

The manuscript is relatively well-written with a number of typos and unrevised sentences. Several sentences are confusing and disconnected with no clear objectives and inter-connectivities. Most of all, the dissertation is relatively well-structured with clear division and linkages between chapters, sections and paragraphs, leading to an easy and smooth reading.  A few general comments,
\begin{enumerate}
%
\item The main aim of {\it Abstracts} is to briefly describe the work undertaken by the author (i.e., the thesis' content). In general {\it Abstracts} are divided in 4 parts: (i) motivation, (ii) aims and objectives, (iii) summary of the main procedures / techniques / technologies (optional) and (iv) main findings (or summary of the work). The current {\it Abstract} encompass (i) and (iv), and partially (iii).
%
\item The main {\it Introduction} section usually has the same (but more in-depth and descriptive) four parts of the {\it Abstract} and a brief summary of the remaining of the work. In addition, it is \underline{always} expected a few clear statements -re main background (thus recent innovations related to the main topic), initial literature review and, most of all, technological / scientific gaps in the current understanding. Current {\it Introduction} covered (i,ii) above and the summary of the chapters. Literature review is spread over the remaining chapters with minimal critical analysis of the work undertaken by several authors. 
%
\item Although the dissertation is focused on energy systems in `smart cities', this was not explicitly defined in any part of the text.
%
\item Numbering of a few sections were not consistent (\eg Section 4 and 4.1 in page 21).
%
\item Extensive use of literature throughout the dissertation, however description and analysis were very limited.
%
\item A critical part of the dissertation was the tri-generation system design. The quality of all process design fluxograms (Hysis) were very poor and can be read. 
%
\item All equations used must be explained in full.
%
\item In the report of any design problem, all relevant variables need to be presented. In the design introduced in Chapter 4, the initial composition of the syngas stream (output from biomass gasification) was described in Table 8, however there is no mention of the mass flow rate of this stream. This information is crucial for the whole analysis. Mass flow rates in all cycles, temperature and pressure gradients, thermo-physical properties of the fluids used  should be contained in the dissertation. 
%
\item Different font sizes and sources were used throughout the report.
%
\item Several figures/tables are $\lq$floating' with no explanation/description/analysis in the main text.
%
\item Vapour-based turbines, instead of gas-based turbines, are used in ORC. In (organic) Rankine cycles vapour-based turbines are used, therefore the explanation/reasoning in pages 35-37 is wrong. This refers to gas-power cycle where combustion (fuel + O$_{2}$) takes place, which is not the case on vapour-power cycle (\ie Rankine cycles).  
%
\item The {\it References} have a few missing fields and no clear distinction between articles, conference proceedings, reports (internal or external), book chapters, books, communications (internal or external) etc.  A few {\it references} used in the report are incomplete and/or wrong. Regardless of the chosen citation style (e.g., ACS, AIP, AMS, IEEE, AIAA, etc) any reference {\bf must} contain the following fields: 
\begin{enumerate}
\item For journal papers: Authors, Paper Tittle, Journal Name, Volume, Pages, Year of publication;
\item For books: Authors, Book Tittle, Publisher, Year or Edition;
\item For book chapters: Authors, Chapter Tittle, Book Tittle, Editors, Publisher, Year or Edition;
\item For conference papers: Authors, Paper Tittle, Conference Tittle, Place (Country and/or City) where the conference was held, Year of the conference;
\item For reports, private communications and Lecture Notes: Authors, Tittle, Place issued (Country and/or City and Institution where the document was originated), Year;
\item For PhD Thesis and MSc Dissertations: Author, Tittle, Institution (University and Department/School), Year.
\end{enumerate}  
Thus, for example:
\begin{enumerate}[label={[\arabic*]}]
\item P.L. Houtekamer and L. Mitchell, `Data Assimilation Using an Ensemble Kalman Filter Technique', {\it Monthly Weather Review}, 126:796-811, 1998.
\item K. Pruess, `Numerical Modelling of Gas Migration at a Proposed Repository for Low and Intermediate Level Nuclear Wastes', Technical Report LBL-25413, Lawrence Berkeley Laboratory, Berkeley (USA), 1990.
\item K. Aziz, A. Settari, {\it Fundamentals of Reservoir Simulation}, Elsevier Applied Science Publishers, New York (USA), 1986.
\item R.B. Lowrie, `Compact higher-Order Numerical Methods for Hyperbolic Conservation Laws', PhD Thesis, Department of Aerospace Engineering and Scientific Computing, University of Michigan (USA), 1996.
\end{enumerate}
% 
\end{enumerate}
In summary, the work undertook by the Ms Chowdhury was extensive and complex but she managed to produce a good dissertation. The topic is very relevant for the chemical, energy and environmental sectors, and each sub-topic has been the focus of several academic- and industrial-based studies worldwide with clear cross-fertilisation with physics (thermodynamics, fluid mechanics, material science etc), mathematics (optimisation, differential equations etc) and chemical engineering (flow simulators, reactor design etc). The student demonstrated that she had a fundamental understanding of the main technologies involved in this project.

\item[Examiner 2:] `The thesis begins well. The written language is good and the introduction is motivating, however it rapidly goes downhill. All technical work is missing or avoided. Basic `pop' science is repeated over and over, with no detail on the real operating principles behind the technology discussed. The actual system studies is not described. There are serious technical errors which demonstrate the student has no understanding of the models used, or the results which are obtained (which are completely wrong). Finally there is little critical reasoning, even in what topics are presented. The report is too long for the content presented and could be considerably shortened.'
%
\item[Comments -re Conduct:] Ms Chowdhury demonstrated a live interest in the beginning of the project. However, her engagement with the contents and design problem strongly oscillated throughout the two half-terms. Several meetings were re-scheduled with no good reason and her progress was very slow/limited. 

\end{description}
\clearpage


%%%%%% 
%%%%%%
%%%%%%

\noindent{\bfseries\large EG4515 -- MEng Thesis \hfill May, 2016}

\bigskip

\begin{center}
  {\Large Review of the MEng Thesis `Investigation Into Reservoir Modelling and Effects of Varying Permeability and Mobility Ratio on Water Flooding' by Allan W. MacLeod}
\end{center}

\begin{description}
\item[Examiner 1]: The dissertation aims to study geo-morphological parameters that controls Darcy fluid flows in heterogeneous porous media. In particular, the manuscript investigates fluid instabilities in water-flooding framework. Mr MacLeod undertook a comprehensive review of the state-of-the art of reservoir engineering, modelling and simulation. He also performed water-flooding numerical simulations on 1D and 2D systems to investigate: (a) impact of empirical parameters, often used in reservoir simulation hand-calculation, on simplified 1D configurations (through Matlab codes implemented by the himself) and (b) sensitivity analysis of permeability and porosity in a synthetic 2D domain with prescribed geometry, boundary conditions and morphology. The dissertation encompasses three main subject areas within the main topic (reservoir simulation): multi-fluid instabilities, multi-scale analysis and technologies for oil recovery. 

The manuscript is relatively well-written with a small number of typos and unrevised sentences. A few sentences are confusing and disconnected with no clear objectives and inter-connectivities. Most of all, the dissertation is very well-structured with clear division and linkages between chapters, sections and paragraphs, leading to an easy and smooth reading.  A few general comments,
\begin{enumerate}
%
\item The main aim of {\it Abstracts} is to briefly describe the work undertaken by the author (i.e., the thesis' content). In general {\it Abstracts} are divided in 4 parts: (i) motivation, (ii) aims and objectives, (iii) summary of the main procedures / techniques / technologies (optional) and (iv) main findings (or summary of the work). The current {\it Abstract} encompass all of them.
%
\item The main {\it Introduction} section usually has the same (but more in-depth and descriptive) four parts of the {\it Abstract} and a brief summary of the remaining of the work. In addition, it is \underline{always} expected a few clear statements -re main background (thus recent innovations related to the main topic), initial literature review and, most of all, technological / scientific gaps in the current understanding. Current {\it Introduction} covered (i-iii) above but did not summarise the chapters of the dissertation. Literature review is spread over the remaining chapters with comprehensive critical analysis of the work undertaken by several authors. 
%
\item Equation 1.2 does not seem to be correct, last term should be $\Delta p$, rather than $\partial p$. The later should refer to a partial differential equation, and the former to an algebraic equation.
%
\item A few terms were used before being defined, \eg FVF (Table 1.2), $\phi$ (porosity, firstly used in Table 2.1 but not defined in the current manuscript), {\it scf/stb} (page 12), etc.
%
\item Equations (just as Figures and Tables) need to be explained in full. Meaning of Eqn. 2.1 (Darcy Equation) was not explained.
%
\item `Welge Method' (Section 2.3) was used but not explained.
%
\item A few references are either incorrect and/or with missing fields (\eg 8, 13, 23, 27 etc).
% 
\end{enumerate}
In summary, the work undertook by the Mr MacLeod was extensive and complex but he managed to produce a very good dissertation. The topic is very relevant for the chemical, energy and environmental sectors, and each sub-topic has been the focus of several academic- and industrial-based studies worldwide with clear cross-fertilisation with physics (thermodynamics, fluid mechanics, material science etc), mathematics (optimisation, differential equations etc) and chemical/reservoir engineering (flow simulators, reactor/reservoir design etc). The student demonstrated that he had a good understanding of the main technologies involved in this project.

\item[Examiner 2] Comments
   \begin{description}
      \item[Presentation and Style of Thesis:] `Aesthetically excellent presentation. Appropriately sectioned. Generally well written but not free from errors. Graphical communication could have been improved. Good level of referencing.'
      \item[Technical Contents and Merit of Thesis:'] `Technical communication quite unclear at times. Lots of parameters introduced, concepts introduced, equations derived without appropriate technical discussion. Odd not to see more evidence of chemical engineering fundamentals.'
      \item[Evidence of Critical Reasoning:] `Discussion is good but doesn't clearly demonstrate command of necessary chemical engineering fundamentals (thermodynamics, phase behaviour, fluids etc).'
      \item[General Comments:] `A good thesis. Technical aspects difficult to follow as written but a strong performance in presentation and style aspects.'
   \end{description}
%
 \item[Comments -re Conduct:] Mr MacLeod is a clever student who took deep interest in the project since the beginning. He was determined, self-driven and enthusiastic through the length of this project. He attended all the meetings bringing fresh ideas, suggestions and actively discussing the relevant fundamentals and topics, although his interests were eventually very diverse. His engagement with the project was excellent. 

\end{description}
\clearpage

%%%%%% 
%%%%%%
%%%%%%

\noindent{\bfseries\large EG4515 -- MEng Thesis \hfill May, 2016}

\bigskip 

\begin{center}
  {\Large Review of the MEng Thesis `Exploring the Potential Recycling Technologies for Plastic Solid Waste (PSW)' by Molly Gray}
\end{center}
The dissertation aims to study technologies currently used in the recycling of PSW. Ms Gray undertook an extensive and comprehensive literature review of the main technologies used worldwide focusing on data of Scottish households. She also performed a preliminary analysis of suitable technologies for two hypothetical recycling companies. The dissertation encompass three main subject areas within Chemical Engineering: environmental technology, mechanical, thermal and chemical processes and management.

The manuscript is well-written with a small number of typos and unrevised sentences. Few sentences are confusing and disconnected with no clear objectives and inter-connectivities. Most of all, the dissertation is very well-structured with clear division and linkages between chapters, sections and paragraphs, leading to an easy and smooth reading.  A few general comments,
\begin{enumerate}
%
\item The main aim of {\it Abstracts} is to briefly describe the work undertaken by the author (i.e., the thesis' content). In general {\it Abstracts} are divided in 4 parts: (i) motivation, (ii) aims and objectives, (iii) summary of the main procedures / techniques / technologies (optional) and (iv) main findings (or summary of the work). The current {\it Abstract} encompass all of them.
%
\item The main {\it Introduction} section usually has the same (but more in-depth and descriptive) four parts of the {\it Abstract} and a brief summary of the remaining of the work. In addition, it is \underline{always} expected a few clear statements -re main background (thus recent innovations related to the main topic), initial literature review and, most of all, technological / scientific gaps in the current understanding. Also, it is expected a {\it summary of the remaining sections} at the end of the {\it Introduction}.  Current {\it Introduction} covered (i-iii) above but did not summarise the chapters of the dissertation.
%
\item Dissertations and thesis are always divided into chapters rather than sections (commonly used in reports), where each chapter is allocated in a new page. Also, numbering in a few sections were not consistent (e.g., Chapter 5).
%
\item One of the main aims of the project was to perform a case study of recycling strategies for two companies -- {\it A} and {\it B}. Available technologies for each stage and plastic types were (not exhaustively) listed and assessed. However, the criteria (technical/engineering, economical and environmental) used for the case studies were not clear and the analysis (Chapters 4 and 5) was superficial.     
% 
\end{enumerate}

The topic is very relevant for the chemical and energy sector, and each sub-topic has been the focus of several academic- and industrial-based studies worldwide with clear cross-fertilisation with physics (thermodynamics, fluid mechanics, surface chemistry, material science etc), chemistry (kinetics, catalysis, organic synthesis) and chemical engineering (kinetics engineering, reactor and process design etc). The student demonstrated that she had a sound understanding of the main technologies involved in this project.


%%%%%%%
%%%%%%%
%%%%%%%

\clearpage

\noindent{\bfseries\large EG4515 -- BEng Thesis \hfill May, 2016}

\bigskip

\begin{center}
  {\Large Review of the BEng Thesis `Optimising a Fixed Bed Reactor to Maximise Octane Rating from Heptane Hydroisomerisation' by Rasul Aliyev}
\end{center}
The dissertation aims to investigate hydro-isomerisation processes of alkanes. Mr Aliyev undertook a comprehensive but limited literature review of the main topics involved in the project, \ie isomerisation, heterogeneous catalysis (kinetics) and reactor design. He also performed conversion experiments to assess catalyst activity and selectivity and reactor's performance. The dissertation encompasses three main subject areas within the main topic: organic chemistry mechanisms, chemical kinetics and catalysis.

The manuscript is relatively well-written with a small number of typos and unrevised sentences. However, it lacks an in-depth description of the underlying science and technology covered by the project, \ie organic chemistry reaction mechanisms, surface chemistry thermodynamics, reactor bed design etc. A few general comments,
\begin{enumerate}
%
\item The main aim of {\it Abstracts} is to briefly describe the work undertaken by the author (i.e., the thesis' content). In general {\it Abstracts} are divided in 4 parts: (i) motivation, (ii) aims and objectives, (iii) summary of the main procedures / techniques / technologies (optional) and (iv) main findings (or summary of the work). The current {\it Abstract} encompass all of them.
%
\item The main {\it Introduction} section usually has the same (but more in-depth and descriptive) four parts of the {\it Abstract} and a brief summary of the remaining of the work. In addition, it is \underline{always} expected a few clear statements -re main background (thus recent innovations related to the main topic), initial literature review and, most of all, technological / scientific gaps in the current understanding. Also, it is expected a {\it summary of the remaining sections} at the end of the {\it Introduction}.  Current {\it Introduction} covered all above, and the literature review is allocated in a separated chapter.
%
\item Most figures (and a few table) are $\lq$floating' with no clear explanation/description in the main text.   
%
\item A few terms were used before being defined/explained (\eg RON); 
%
\item Avoid using {\it colloquial (informal / personal)} writing.
% 
\end{enumerate}

The topic is very relevant for the chemical and energy sector, and each sub-topic has been the focus of several academic- and industrial-based studies worldwide with clear cross-fertilisation with physics (thermodynamics, fluid mechanics, surface chemistry, material science etc), chemistry (kinetics, catalysis, organic synthesis) and chemical engineering (kinetics engineering, reactor design etc). The student demonstrated that he had a sound understanding of the main technologies involved in this project.


%%%%%%%
%%%%%%%
%%%%%%%

\clearpage

%%%%%% 
%%%%%%
%%%%%%


\noindent{\bfseries\large EG4515 -- BEng Thesis \hfill May, 2016}

\bigskip

\begin{center}
  {\Large Review of the BEng Thesis `Simulating BTX Separation using the process of Liquid-Liquid Extraction on Aspen HYSYS' by Craig Barbour}
\end{center}
The dissertation aims to investigate BTX separation from naphta solution through simulation of liquid-liquid extraction process. Mr Barbour undertook a limited review of liquid-liquid extraction process technology and fluids using for extraction. He also performed a limited sensitivity analysis on the extraction simulation using sulfolene as solvent. The dissertation encompasses two main subject are within the main topic (fluid-fluid separation): process simulation and extraction process.

The manuscript is relatively well-written with a small number of typos and unrevised sentences. Several sentences are confusing and disconnected with no clear objectives and inter-connectivities.  Most of all, the dissertation lacks clear statements -re aims and objectives, it is not clear if the aim is to design a separation process, analysis of extraction parameters or simulation and analysis of simplified processes. Project specification (page 2) were not achieved. A few general comments,
\begin{enumerate}
%
\item The main aim of {\it Abstracts} is to briefly describe the work undertaken by the author (i.e., the thesis' content). In general {\it Abstracts} are divided in 4 parts: (i) motivation, (ii) aims and objectives, (iii) summary of the main procedures / techniques / technologies (optional) and (iv) main findings (or summary of the work). The current {\it Abstract} encompass only (ii), (iii) and (iv) (partially).
%
\item The main {\it Introduction} section usually has the same (but more in-depth and descriptive) four parts of the {\it Abstract} and a brief summary of the remaining of the work. In addition, it is \underline{always} expected a few clear statements -re main background (thus recent innovations related to the main topic), initial literature review and, most of all, technological / scientific gaps in the current understanding. Also, it is expected a {\it summary of the remaining sections} at the end of the {\it Introduction}.  Current {\it Introduction} covered (i) and a (short) summary of the report, however it lacked explaining/summarising the main state-of-the-art aspects of the subject area.
%
\item The {\it References} have a few missing fields and no clear distinction between articles, conference proceedings, reports (internal or external), book chapters, books, communications (internal or external) etc.  Regardless of the chosen citation style (e.g., ACS, AIP, AMS, IEEE, AIAA, etc) any reference {\bf must} contain the following fields: 
\begin{enumerate}
\item For journal papers: Authors, Paper Tittle, Journal Name, Volume, Pages, Year of publication;
\item For books: Authors, Book Tittle, Publisher, Year or Edition;
\item For book chapters: Authors, Chapter Tittle, Book Tittle, Editors, Publisher, Year or Edition;
\item For conference papers: Authors, Paper Tittle, Conference Tittle, Place (Country and/or City) where the conference was held, Year of the conference;
\item For reports, private communications and Lecture Notes: Authors, Tittle, Place issued (Country and/or City and Institution where the document was originated), Year;
\item For PhD Thesis and MSc Dissertations: Author, Tittle, Institution (University and Department/School), Year.
\end{enumerate}  
Thus, for example:
\begin{enumerate}[label={[\arabic*]}]
\item P.L. Houtekamer and L. Mitchell, `Data Assimilation Using an Ensemble Kalman Filter Technique', {\it Monthly Weather Review}, 126:796-811, 1998. {\it Journal paper.}
\item K. Pruess, `Numerical Modelling of Gas Migration at a Proposed Repository for Low and Intermediate Level Nuclear Wastes', Technical Report LBL-25413, Lawrence Berkeley Laboratory, Berkeley (USA), 1990. {\it Technical report.}
\item K. Aziz, A. Settari, {\it Fundamentals of Reservoir Simulation}, Elsevier Applied Science Publishers, New York (USA), 1986. {\it Book.}
\item R.B. Lowrie, `Compact higher-Order Numerical Methods for Hyperbolic Conservation Laws', PhD Thesis, Department of Aerospace Engineering and Scientific Computing, University of Michigan (USA), 1996. {\it PhD thesis.}
\item L.R. Glicksman, W.K. Lord, `Prediction of Bubble Growth in Bubble Chains' in {\it Fluidization} (Eds: J.R. Grace and J.M. Matsen), Springer, 1980. {\it Chapter of a book}.
\end{enumerate}
%
\item A few terms were used before being defined/explained (e.g., BTX, 2M-2-Pentene etc); 
%
\item The aim of the thesis is to `provide background knowledge and the theory behind the process of liquid-liquid extraction'. However no chemical/physical theory -re LLE is presented;
%
\item Limited description of the design performed -- no model or data were presented;
%
\item Discussion of the results are very poor with little explanation of the main physical/chemical and engineering concepts.
%
\item Avoid using {\it colloquial (informal / personal)} writing.
% 
\end{enumerate}

The topic is very relevant for the chemical sector and each section has been the focus of several academic- and industrial-based studies worldwide with clear cross-fertilisation with physics (thermodynamics, fluid mechanics, shock-physics etc) and engineering (e.g., mechanical, chemical, petroleum, civil, etc). The student demonstrated that he had a limited knowledge and understanding of the main sciences and technologies for this project.



%%%%%%%
%%%%%%%
%%%%%%%

%\lipsum % Text before
\afterpage{%
    \clearpage% Flush earlier floats (otherwise order might not be correct)
    \thispagestyle{empty}% empty page style (?)
    \begin{landscape}% Landscape page
        \centering % Center table


\Huge{MEng Study Assessment -- Winter Report (EG4013)}\\
\huge{(Review + Feedback)}\\
\huge{January 2016}
%\end{center}
\normalsize



\bigskip

\begin{center}
\begin{tabular}{l c c }
\hline
Allan W. MacLeod   & 16  & B2 \\
Nahian Chowdhury   & 14  & C1 \\
\hline
\end{tabular}
\end{center}


    \end{landscape}
    \clearpage% Flush page
}

%\lipsum % Text after

\vfill

\clearpage

%%%%%% 
%%%%%%
%%%%%%


\noindent{\bfseries\large EG4013 -- MEng Winter Report \hfill February, 2016}

\bigskip

\begin{center}
  {\Large Review of the MEng Winter Report `Reservoir Waterflooding' by Allan W MacLeod}
\end{center}

The report describes Mr MacLeod's winter research on fundamentals of reservoir engineering and reservoir simulation, and in particular his initial investigation on viscous fluid instabilities in waterflooding processes for oil and gas exploration. A brief overview of the main topics on reservoir engineering and simulation relevant to EOR was undertaken by Mr MacLeod including, (a) typical workflow, (b) main terminology, (c) EOR techniques and reservoir morphological properties, and (d) fundamental equation for reservoir simulation.  

The report is relatively well-written with a number of typos and unrevised sentences. Several sentences are confusing and disconnected with no clear objectives and inter-connectivities. Most of all, the paper lacks a coherent structured and linkages between sections. This leads to an unsmooth reading at times. Section numbering could have been greatly improved. A few general comments,
\begin{enumerate}
%
\item The main aim of {\it Abstracts} is to briefly describe the work undertaken by the author (i.e., the report's content). In general {\it Abstracts} are divided in 4 parts: (i) motivation, (ii) aims and objectives, (iii) summary of the main procedures / techniques / technologies (optional) and (iv) main findings. The current {\it Abstract} encompass only (i) and (iv).
%
\item The main {\it Introduction} section usually has the same (but more in-depth and descriptive) four parts of the {\it Abstract} and a brief summary of the remaining of the work. In addition, it is \underline{always} expected a few clear statements -re main background (thus recent innovations related to the main topic), initial literature review and, most of all, technological / scientific gaps in the current understanding. Also, it is expected a {\it summary of the remaining sections} at the end of the {\it Introduction}.  Current {\it Introduction} covered (i-ii) and a (short) summary of the report, however it lacked explaining/summarising the main state-of-the-art aspects of the subject area.
%
\item The {\it References} have a few missing fields and no clear distinction between articles, conference proceedings, reports (internal or external), book chapters, books, communications (internal or external) etc.  A few {\it references} used in the report are incomplete and/or wrong. Regardless of the chosen citation style (e.g., ACS, AIP, AMS, IEEE, AIAA, etc) any reference {\bf must} contain the following fields: 
\begin{enumerate}
\item For journal papers: Authors, Paper Tittle, Journal Name, Volume, Pages, Year of publication;
\item For books: Authors, Book Tittle, Publisher, Year or Edition;
\item For book chapters: Authors, Chapter Tittle, Book Tittle, Editors, Publisher, Year or Edition;
\item For conference papers: Authors, Paper Tittle, Conference Tittle, Place (Country and/or City) where the conference was held, Year of the conference;
\item For reports, private communications and Lecture Notes: Authors, Tittle, Place issued (Country and/or City and Institution where the document was originated), Year;
\item For PhD Thesis and MSc Dissertations: Author, Tittle, Institution (University and Department/School), Year.
\end{enumerate}  
Thus, for example:
\begin{enumerate}[label={[\arabic*]}]
\item P.L. Houtekamer and L. Mitchell, `Data Assimilation Using an Ensemble Kalman Filter Technique', {\it Monthly Weather Review}, 126:796-811, 1998.
\item K. Pruess, `Numerical Modelling of Gas Migration at a Proposed Repository for Low and Intermediate Level Nuclear Wastes', Technical Report LBL-25413, Lawrence Berkeley Laboratory, Berkeley (USA), 1990.
\item K. Aziz, A. Settari, {\it Fundamentals of Reservoir Simulation}, Elsevier Applied Science Publishers, New York (USA), 1986.
\item R.B. Lowrie, `Compact higher-Order Numerical Methods for Hyperbolic Conservation Laws', PhD Thesis, Department of Aerospace Engineering and Scientific Computing, University of Michigan (USA), 1996.
\end{enumerate}
%
\item A few terms were used before being defined/explained. And a few of them were not defined at all (e.g., Eqn. 1.10, $B_{ti}$ and $B_{t}$); 
%
\item Avoid using {\it colloquial (informal / personal)} writing;
%
\item Consistency of symbols/terms, e.g., $RB$ and $rb$, $K$ and $k$, etc.
% 
\end{enumerate}

The report is a good review of the fundamentals of reservoir engineering and simulation, although it does not cover the literature review of the main subject area: fluid instabilities (that leads to fingering phenomena) and rock wetability (surface physics). Overall, the research plan for the spring is realistic.


In the attached scanned document:
\begin{itemize}
\item {\bf PE:} Poor English;
\item {\bf SC:} Sentence(s) is/are very confusing and do(es) not make much/any sense;
\item {\bf CS:} Use of colloquial sentence(s);
\item {\bf DFS:} Different (a) font size and/or (b) font source and/or (c) line space;
\item {\bf PQ:} Poor quality of tables and/or figures and/or equations. 
\end{itemize}
\medskip

\clearpage

%%%%%% 
%%%%%%
%%%%%%


\noindent{\bfseries\large EG4013 -- MEng Winter Report \hfill February, 2016}

\bigskip

\begin{center}
  {\Large Review of the MEng Winter Report `Optimisation of Energy Systems in {\it Smart Cities}' by Nahian Chowdhury}
\end{center}

The report describes Ms Chowdhury's winter research on energy optimisation strategies for `smart cities'. Her work focused on co-/tri-generation system technologies and energy/exergy analysis. An overview of the main topics on co-/tri-generation systems relevant to sustainability was undertaken by Ms Chowdhury including, (a) main definitions and equipment, (b) basic thermodynamic analysis and (c) fundamentals of biomass-fuel cell tri-generation technologies.  The report is relatively well-written with a large number of typos and unrevised sentences. Several sentences are confusing and disconnected with no clear objectives and inter-connectivities. Section numbering is confusing leading to a relatively difficult and confusing reading. Most of all, the paper lacks a coherent structured and linkages between sections resulting in an unsmooth reading at times. A few general comments,
\begin{enumerate}
%
\item The main aim of {\it Abstracts} is to briefly describe the work undertaken by the author (i.e., the report's content). In general {\it Abstracts} are divided in 4 parts: (i) motivation, (ii) aims and objectives, (iii) summary of the main procedures / techniques / technologies (optional) and (iv) main findings (or summary of the work). The current {\it Abstract} encompass only (iv).
%
\item The main {\it Introduction} section usually has the same (but more in-depth and descriptive) four parts of the {\it Abstract} and a brief summary of the remaining of the work. In addition, it is \underline{always} expected a few clear statements -re main background (thus recent innovations related to the main topic), initial literature review and, most of all, technological / scientific gaps in the current understanding. Also, it is expected a {\it summary of the remaining sections} at the end of the {\it Introduction}.  Current {\it Introduction} covered (i-ii) and a (short) summary of the report, however it lacked explaining/summarising the main state-of-the-art aspects of the subject area.
%
\item The {\it References} have a few missing fields and no clear distinction between articles, conference proceedings, reports (internal or external), book chapters, books, communications (internal or external) etc.  A few {\it references} used in the report are incomplete and/or wrong. Regardless of the chosen citation style (e.g., ACS, AIP, AMS, IEEE, AIAA, etc) any reference {\bf must} contain the following fields: 
\begin{enumerate}
\item For journal papers: Authors, Paper Tittle, Journal Name, Volume, Pages, Year of publication;
\item For books: Authors, Book Tittle, Publisher, Year or Edition;
\item For book chapters: Authors, Chapter Tittle, Book Tittle, Editors, Publisher, Year or Edition;
\item For conference papers: Authors, Paper Tittle, Conference Tittle, Place (Country and/or City) where the conference was held, Year of the conference;
\item For reports, private communications and Lecture Notes: Authors, Tittle, Place issued (Country and/or City and Institution where the document was originated), Year;
\item For PhD Thesis and MSc Dissertations: Author, Tittle, Institution (University and Department/School), Year.
\end{enumerate}  
Thus, for example:
\begin{enumerate}[label={[\arabic*]}]
\item P.L. Houtekamer and L. Mitchell, `Data Assimilation Using an Ensemble Kalman Filter Technique', {\it Monthly Weather Review}, 126:796-811, 1998. {\it Journal paper.}
\item K. Pruess, `Numerical Modelling of Gas Migration at a Proposed Repository for Low and Intermediate Level Nuclear Wastes', Technical Report LBL-25413, Lawrence Berkeley Laboratory, Berkeley (USA), 1990. {\it Technical report.}
\item K. Aziz, A. Settari, {\it Fundamentals of Reservoir Simulation}, Elsevier Applied Science Publishers, New York (USA), 1986. {\it Book.}
\item R.B. Lowrie, `Compact higher-Order Numerical Methods for Hyperbolic Conservation Laws', PhD Thesis, Department of Aerospace Engineering and Scientific Computing, University of Michigan (USA), 1996. {\it PhD thesis.}
\item L.R. Glicksman, W.K. Lord, `Prediction of Bubble Growth in Bubble Chains' in {\it Fluidization} (Eds: J.R. Grace and J.M. Matsen), Springer, 1980. {\it Chapter of a book}.
\end{enumerate}
%
\item A few terms were used before being defined/explained (e.g., GHG, PAFC, VOC etc); 
%
\item Different font sizes and sources were used throughout the report;
%
\item Lack of consistency of the section numbering made the report difficult to read/understand;
%
\item Quality of Table 4 and Eqns.10-29 are very poor. The font size for some of them is too small and several equations are $\lq$floating' with no explanation and/or description of the terms in surrounding text;
%
%\item Avoid using {\it colloquial (informal / personal)} writing.
% 
\end{enumerate}

The report is a good review of the fundamentals of energy systems. Overall, the research plan for the spring is realistic. In the attached scanned document:
\begin{itemize}
\item {\bf PE:} Poor English;
\item {\bf SC:} Sentence(s) is/are very confusing and do(es) not make much/any sense;
\item {\bf CS:} Use of colloquial sentence(s);
\item {\bf DFS:} Different (a) font size and/or (b) font source and/or (c) line space;
\item {\bf PQ:} Poor quality of tables and/or figures and/or equations. 
\end{itemize}
\medskip

\clearpage




%%%%%%%
%%%%%%%
%%%%%%%

%\lipsum % Text before
\afterpage{%
    \clearpage% Flush earlier floats (otherwise order might not be correct)
    \thispagestyle{empty}% empty page style (?)
    \begin{landscape}% Landscape page
        \centering % Center table

\Huge{EG501V (Computational Fluid Dynamics) Continuous Assessment Reports}\\
\bigskip

\huge{(Brief Feedback)}\\
\bigskip

\huge{December 2015}
\normalsize

    \end{landscape}}

\clearpage
%\pagebreak


%%%%%
%%%%%
%%%%%

\noindent{\bfseries\large Computational Fluid Dynamics (EG501V) \hfill December, 2015}

\bigskip

\noindent{\Large Matthews, Benjamin (51122272)}

\medskip

  \begin{itemize}
%
     \item Missing conservative (continuity, momentum and thermal energy) and constitutive equations and explanations;
     \item Quality of few figures (and equations) are poor. Also, several figures are $\lq$floating' with no explanation/description in the main text;
     \item No explanation on turbulent models and sub-models, including $k-\epsilon$ and wall-treatment;
     \item Simulation set-up, i.e., initial and boundary conditions, numerical schemes and (sub-)models used, for all cases, in particular for the design problem, were not summarised and explained;
     \item Summary of FVM is not correct;
     \item Definition of 1$^{\text{st}}$- and 2$^{\text{nd}}$-order accurate schemes are not correct;
     \item Fig. 16 is not correct;
     \item Research and analysis of the simulation results are fairly poor.
%
  \end{itemize}%\afterpage{\blankpage}
\clearpage

%%%%%
%%%%%
%%%%%

\noindent{\bfseries\large Computational Fluid Dynamics (EG501V) \hfill December, 2015}

\bigskip

\noindent{\Large Maule, Rebecca (51122194)}

\medskip

  \begin{itemize}
%
     \item Good work;
     \item All conservative equations (except the thermal energy) are shown, but the associated symbols are missing;
     \item No explanation on turbulent models and sub-models, including $k-\epsilon$ and wall-treatment;% 
     \item Quality of figures are very good but legends were difficult to read;
     \item Definition of 1$^{\text{st}}$- and 2$^{\text{nd}}$-order accurate schemes are not correct;
     \item Thermal boundary conditions are over-determined (outflow and top free);
     \item Good general discussion of the results but with poor analysis based on the theoretical background.
%
  \end{itemize}%\afterpage{\blankpage}

\clearpage 



%%%%%
%%%%%
%%%%%

\noindent{\bfseries\large Computational Fluid Dynamics (EG501V) \hfill December, 2015}

\bigskip

\noindent{\Large McCabe, Olivia (51122181)}

\medskip

  \begin{itemize}
%
     \item Good Work;
     \item All conservative equations are shown, except the {\it thermal} energy equation. Also the associated symbols are missing;
     \item No captions in Figs and Tables;
     \item No explanation on FVM, turbulent models and sub-models, including $k-\epsilon$ and wall-treatment;% 
     \item Quality of figures are very good but legends were difficult to read;
     \item Definition of 1$^{\text{st}}$- and 2$^{\text{nd}}$-order accurate schemes are not correct;
     \item Thermal boundary conditions are missing;
     \item Good general discussion of the results but with poor analysis based on the theoretical background.
%
  \end{itemize}%\afterpage{\blankpage}

\clearpage 



%%%%%
%%%%%
%%%%%

\noindent{\bfseries\large Computational Fluid Dynamics (EG501V) \hfill December, 2015}

\bigskip

\noindent{\Large McClelland, David (51013260)}

\medskip

  \begin{itemize}
%
     \item Poor work with neither comments nor discussions;
     \item Continuity and momentum equations are shown but with limited explanation. Text of solving this equations are for transient flows;
     \item No explanation on thermal conservative equations, FVM, turbulent models and sub-models, including $k-\epsilon$ and wall-treatment;% 
     \item No mesh analysis. Indeed it is not clear if the simulations were performed with optimal mesh grids or just the grid supplied in the first task;
     \item Definition and analysis of 1$^{\text{st}}$- and 2$^{\text{nd}}$-order accurate schemes are missing;
     \item Thermal boundary conditions for Task 2.2 are missing;
     \item Several plots are not correct and missing.
%
  \end{itemize}%\afterpage{\blankpage}

\clearpage 


%%%%%
%%%%%
%%%%%

\noindent{\bfseries\large Computational Fluid Dynamics (EG501V) \hfill December, 2015}

\bigskip

\noindent{\Large McClenaghan, Philip (51122173)}

\medskip

  \begin{itemize}
%
     \item Good general work with good analysis and discussion;
     \item Poor quality of equations (copy and paste from other sources);
     \item No explanation on thermal conservative equations, FVM and turbulent wall-treatment;
     \item Few figure captions were used as part of the text;
     \item Design target wasn't achieved but the discussions were interesting and creative. 
%
  \end{itemize}%\afterpage{\blankpage}

\clearpage 



%%%%%
%%%%%
%%%%%

\noindent{\bfseries\large Computational Fluid Dynamics (EG501V) \hfill December, 2015}

\bigskip

\noindent{\Large McKiddie, Ben (51122199)}

\medskip

  \begin{itemize}
%
     \item Good general work with good analysis but poor discussion;
     \item Good quality of equations with reasonable discussion;
     \item No/poor explanation/discussion on $k-\epsilon$ turbulent model, wall-treatment, FVM and model solutions;
     \item No table captions and no legend for figures. Figure/table captions should be \underline{self-contained}, i.e., with a good description of the figure/table highlighting the most relevant aspects/information that the author wants to convene.
%
  \end{itemize}%\afterpage{\blankpage}

\clearpage 

\vspace{20cm}

\clearpage 


%%%%%
%%%%%
%%%%%

\noindent{\bfseries\large Computational Fluid Dynamics (EG501V) \hfill December, 2015}

\bigskip

\noindent{\Large Moir, Craig (51120039)}

\medskip

  \begin{itemize}
%
     \item Excellent work, analysis and discussion;
     \item In Eqn. 9, $y$ should be replaced by $\Delta y$ and the plot (Fig. 15) overestimated $y^{+}$; 
     \item No explanation/discussion on thermal energy equation and wall-treatment.
%
  \end{itemize}%\afterpage{\blankpage}

\clearpage 

%%%%%
%%%%%
%%%%%

\noindent{\bfseries\large Computational Fluid Dynamics (EG501V) \hfill December, 2015}

\bigskip

\noindent{\Large Moro, James (51119535)}

\medskip

  \begin{itemize}
%
     \item Good general work with good analysis but poor discussion;
     \item Good quality of equations with reasonable discussion;
     \item No/poor explanation/discussion on thermal energy equation, $k-\epsilon$ turbulent model, wall-treatment, FVM and model solutions;
     \item Overestimated $y^{+}$ (Fig. 5), possibly due to mesh resolution close to the wall.
%
  \end{itemize}%\afterpage{\blankpage}

\clearpage 



%%%%%
%%%%%
%%%%%

\noindent{\bfseries\large Computational Fluid Dynamics (EG501V) \hfill December, 2015}

\bigskip

\noindent{\Large Morrison, Lauren (51119745)}

\medskip

  \begin{itemize}
%
     \item Good general work with reasonable analysis but poor discussion;
     \item Poor quality of equations and no list of symbols;
     \item No/poor explanation/discussion on thermal energy equation, $k-\epsilon$ turbulent model, wall-treatment, FVM and model solutions;
     \item Overestimated $y^{+}$ (Fig. 15), possibly due to mesh resolution close to the wall;
     \item A few figures and tables are $\lq$floating' with no explanation/description in the main text.
%
  \end{itemize}%\afterpage{\blankpage}

\clearpage 




%%%%%
%%%%%
%%%%%

\noindent{\bfseries\large Computational Fluid Dynamics (EG501V) \hfill December, 2015}

\bigskip

\noindent{\Large Morton, Kenneth (50110145)}

\medskip

  \begin{itemize}
%
     \item Good general report work with good analysis and reasonable discussion;
     \item Poor quality of equations (copy and paste from other sources);
     \item No explanation on thermal conservative equations, FVM and turbulent wall-treatment;
     \item A few figures and tables are $\lq$floating' with no explanation/description in the main text;
     \item Overestimated $y^{+}$ (Fig. 15), possibly due to mesh resolution close to the wall.
%
  \end{itemize}%\afterpage{\blankpage}

\clearpage 



%%%%%
%%%%%
%%%%%

\noindent{\bfseries\large Computational Fluid Dynamics (EG501V) \hfill December, 2015}

\bigskip

\noindent{\Large Muir, Calum (511200094)}

\medskip

  \begin{itemize}
%
     \item Poor report with poor comments and discussions;
     \item Continuity and momentum equations are shown but with limited explanation. Text of solving this equations are also very poor;
     \item No explanation on thermal conservative equations, FVM, turbulent models and sub-models, including  wall-treatment;% 
     \item No mesh analysis. Indeed it is not clear if the simulations were performed with optimal mesh grids;
     \item Definition and analysis of 1$^{\text{st}}$- and 2$^{\text{nd}}$-order accurate schemes are missing;
     \item Thermal boundary conditions for Task 2.2 are missing;
     \item A few figures and tables are $\lq$floating' with no explanation/description in the main text;
     \item A few plots are not correct and/or missing;
     \item Overestimated $y^{+}$ (Fig. 18), possibly due to mesh resolution close to the wall.
%
  \end{itemize}%\afterpage{\blankpage}

\clearpage 

%%%%%
%%%%%
%%%%%

\noindent{\bfseries\large Computational Fluid Dynamics (EG501V) \hfill December, 2015}

\bigskip

\noindent{\Large Murray, Shannen (51122273)}

\medskip

  \begin{itemize}
%
     \item Excellent work, analysis and discussion;
     \item In Eqn. 4, interphase heat transfer is missing; 
     \item Poor/no explanation/discussion on thermal energy equation and wall-treatment.
%
  \end{itemize}%\afterpage{\blankpage}

\clearpage 

%%%%%
%%%%%
%%%%%

\noindent{\bfseries\large Computational Fluid Dynamics (EG501V) \hfill December, 2015}

\bigskip

\noindent{\Large North, James (51119720)}

\medskip

  \begin{itemize}
%
     \item Good general work with good analysis but poor discussion;
     \item Good quality of equations with reasonable discussion;
     \item No/poor explanation/discussion on turbulent model, wall-treatment, FVM and model solutions;
     \item Figure/table captions should be \underline{self-contained}, i.e., with a good description of the figure/table highlighting the most relevant aspects/information that the author wants to convene;
     \item Overestimated $y^{+}$ (Fig. 4.6), possibly due to mesh resolution close to the wall.
%
  \end{itemize}%\afterpage{\blankpage}

\clearpage 



%%%%%
%%%%%
%%%%%

\noindent{\bfseries\large Computational Fluid Dynamics (EG501V) \hfill December, 2015}

\bigskip

\noindent{\Large Paterson, Aidan (51122258)}

\medskip

  \begin{itemize}
%
     \item Excellent report, analysis and discussion;
     \item In Eqn. 9, $y$ should be replaced by $\Delta y$ and the plot (Fig. 15) overestimated $y^{+}$; 
     \item No explanation/discussion on thermal energy equation and wall-treatment;
     \item Overestimated $y^{+}$ (Fig. 16), possibly due to mesh resolution close to the wall;
%
  \end{itemize}%\afterpage{\blankpage}

\clearpage 

%%%%%
%%%%%
%%%%%

\noindent{\bfseries\large Computational Fluid Dynamics (EG501V) \hfill December, 2015}

\bigskip

\noindent{\Large Reid, Andrew (51122156)}

\medskip

  \begin{itemize}
%
     \item Reasonable report with poor analysis;
     \item No introduction, with the topics requested by the assignment, was given;
     \item Fig. 1.9 is not correct, and no discussion/analysis of the problem was added;
     \item Figs. 2.2-3,5-8 are not correct. Possible causes are mesh resolution, lhs BCs, etc;
     \item Definition and analysis of 1$^{\text{st}}$- and 2$^{\text{nd}}$-order accurate schemes are missing;
     \item Thermal boundary conditions for Task 2.2 are missing;
     \item Several plots are not correct and missing.
%
  \end{itemize}%\afterpage{\blankpage}

\clearpage 



%%%%%
%%%%%
%%%%%

\noindent{\bfseries\large Computational Fluid Dynamics (EG501V) \hfill December, 2015}

\bigskip

\noindent{\Large Ritchie, Calum (51119351)}

\medskip

  \begin{itemize}
%
     \item Excellent report with good analysis and discussion;
     \item Poor/No explanation/discussion on $k-\epsilon$ turbulence model and wall-treatment;
     \item Overestimated $y^{+}$ (Fig. 15), possibly due to mesh resolution close to the wall.
%
  \end{itemize}%\afterpage{\blankpage}

\clearpage 


%%%%%
%%%%%
%%%%%

\noindent{\bfseries\large Computational Fluid Dynamics (EG501V) \hfill December, 2015}

\bigskip

\noindent{\Large Ritchie, Kristoffer (51119794)}

\medskip

  \begin{itemize}
%
     \item Good report, but very superficial analysis and discussion;
     \item Overestimated $y^{+}$ (Fig. 16), possibly due to mesh resolution close to the wall;
     \item Missing conservative (continuity, momentum and thermal energy) and constitutive equations and explanations;
     \item No/Poor explanation on FVM, solution methods and wall-treatment;
     \item Definition of 1$^{\text{st}}$- and 2$^{\text{nd}}$-order accurate schemes are missing.
%
  \end{itemize}%\afterpage{\blankpage}

\clearpage 

%%%%%
%%%%%
%%%%%

\noindent{\bfseries\large Computational Fluid Dynamics (EG501V) \hfill December, 2015}

\bigskip

\noindent{\Large Scott, Callum (51121696)}

\medskip

  \begin{itemize}
%
     \item Good report with reasonable discussion;
     \item All conservative equations (except the thermal energy) are shown, but the associated symbols are missing;
     \item No explanation on turbulent models and sub-models, including $k-\epsilon$ and wall-treatment;% 
     \item Quality of figures are very good but captions are missing;
     \item Figures are $\lq$floating' with no explanation/description in the main text;
     \item Overestimated $y^{+}$ and wall shear, possibly due to mesh resolution close to the wall;
     \item Good general discussion of the results but with poor analysis based on the theoretical background.
%
  \end{itemize}%\afterpage{\blankpage}

\clearpage 



%%%%%
%%%%%
%%%%%

\noindent{\bfseries\large Computational Fluid Dynamics (EG501V) \hfill December, 2015}

\bigskip

\noindent{\Large Seeds, David (51122195)}

\medskip

  \begin{itemize}
%
     \item Reasonable report with poor analysis;
     \item No introduction, with the topics requested by the assignment, was given;
     \item Fig. 2.5 is not correct, and no discussion/analysis of the problem was added;
     \item Figs. 3.2-4,6-7, 4.1-2 are not correct. Possible causes are mesh resolution, lhs BCs, etc;
     \item Definition and analysis of 1$^{\text{st}}$- and 2$^{\text{nd}}$-order accurate schemes are missing;
     \item Thermal boundary conditions for Task 2.2 are missing;
     \item Several plots are not correct and missing.
%
  \end{itemize}%\afterpage{\blankpage}

\clearpage 



%%%%%
%%%%%
%%%%%

\noindent{\bfseries\large Computational Fluid Dynamics (EG501V) \hfill December, 2015}

\bigskip

\noindent{\Large Sked, David (51120143)}

\medskip

  \begin{itemize}
%
     \item Good report, but very superficial analysis and discussion;
     \item Overestimated $y^{+}$ (Fig. 15), possibly due to mesh resolution close to the wall;
     \item No/Poor explanation on thermal energy, $k-\epsilon$ equations, wall treatment, FVM and solution methods;
     \item Definition of 1$^{\text{st}}$- and 2$^{\text{nd}}$-order accurate schemes are missing.
     \item Poor analysis of the results and design's decisions based on the theoretical background.
%
  \end{itemize}%\afterpage{\blankpage}

\clearpage 

%%%%%
%%%%%
%%%%%

\noindent{\bfseries\large Computational Fluid Dynamics (EG501V) \hfill December, 2015}

\bigskip

\noindent{\Large Slater, Christopher (51122268)}

\medskip

  \begin{itemize}
%
     \item Poor report with poor comments and discussions;
     \item Poor/No explanation of thermal energy equation, wall treatment, FVM and solution methods;
     \item Most figures are $\lq$floating' with no explanation/description in the main text;
     \item Definition and analysis of 1$^{\text{st}}$- and 2$^{\text{nd}}$-order accurate schemes are missing;
     \item Simulation set-up, i.e., initial and boundary conditions, numerical schemes and (sub-)models used, for all cases, in particular for the design problem, were not summarised and explained;
     \item Overestimated $y^{+}$ (Fig. 17), possibly due to mesh resolution close to the wall;
     \item Research and analysis of the simulation results are fairly poor.
%
  \end{itemize}%\afterpage{\blankpage}

\clearpage 
%%%%%
%%%%%
%%%%%

\noindent{\bfseries\large Computational Fluid Dynamics (EG501V) \hfill December, 2015}

\bigskip

\noindent{\Large Soitu, Christian (51122143)}

\medskip

  \begin{itemize}
%
     \item Poor report with poor comments and discussions;
     \item Poor/No explanation of thermal energy equation, wall treatment, FVM and solution methods;
     \item Definition and analysis of 1$^{\text{st}}$- and 2$^{\text{nd}}$-order accurate schemes are missing;
     \item Simulation set-up, i.e., initial and boundary conditions, numerical schemes and (sub-)models used, for all cases, in particular for the design problem, were not summarised and explained;
     \item Overestimated $y^{+}$ (Fig. 4.4), possibly due to mesh resolution close to the wall;
     \item Research and analysis of the simulation results are fairly poor;
     \item Design criteria were not met.
%
  \end{itemize}%\afterpage{\blankpage}

\clearpage 

%%%%%
%%%%%
%%%%%

\noindent{\bfseries\large Computational Fluid Dynamics (EG501V) \hfill December, 2015}

\bigskip

\noindent{\Large Stuart, Gavin (51119344)}

\medskip

  \begin{itemize}
%
     \item Good overall report;
     \item Poor/No explanation of thermal energy equation, wall treatment, FVM and solution methods;
     \item Definition and analysis of 1$^{\text{st}}$- and 2$^{\text{nd}}$-order accurate schemes are missing;
     \item Simulation set-up, i.e., initial and boundary conditions, numerical schemes and (sub-)models used, for all cases, in particular for the design problem, were not summarised and explained;
     \item Overestimated $y^{+}$ (Fig. 17), possibly due to mesh resolution close to the wall;
     \item Thermal boundary conditions are missing;
     \item Good general discussion of the results but with poor analysis based on the theoretical background.
%
  \end{itemize}%\afterpage{\blankpage}

\clearpage 

%%%%%
%%%%%
%%%%%

\noindent{\bfseries\large Computational Fluid Dynamics (EG501V) \hfill December, 2015}

\bigskip

\noindent{\Large Taylor, Rudi (51119467)}

\medskip

  \begin{itemize}
%
     \item Good general report with reasonable analysis and discussion;
     \item No/Poor explanation of conservative equations and wall treatment used in the turbulence model;
     \item No/poor explanation/discussion on FVM and model solutions;
     \item Definition and analysis of 1$^{\text{st}}$- and 2$^{\text{nd}}$-order accurate schemes are missing;
     \item Overestimated $y^{+}$ (Fig. 16), possibly due to mesh resolution close to the wall;
     \item No plot of stream functions.
%
  \end{itemize}%\afterpage{\blankpage}

\clearpage 

%%%%%
%%%%%
%%%%%

\noindent{\bfseries\large Computational Fluid Dynamics (EG501V) \hfill December, 2015}

\bigskip

\noindent{\Large Theobald, Olivier (51124649)}

\medskip

  \begin{itemize}
%
     \item Excellent report with good analysis and discussion;
     \item Overestimated $y^{+}$ (Fig. 14), but excellent discussion on the reason for the problem;
     \item No/Poor discussion on FVM and wall treatment.
%
  \end{itemize}%\afterpage{\blankpage}

\clearpage 

%%%%%
%%%%%
%%%%%

\noindent{\bfseries\large Computational Fluid Dynamics (EG501V) \hfill December, 2015}

\bigskip

\noindent{\Large Watson, Thomas (51119801)}

\medskip

  \begin{itemize}
%
     \item Poor report with poor comments and discussions;
     \item Poor/No explanation of conservative equations, turbulent $k-\epsilon$ model, wall treatment, FVM and solution methods;
     \item No captions in Figs and Tables;
     \item Definition and analysis of 1$^{\text{st}}$- and 2$^{\text{nd}}$-order accurate schemes are missing;
     \item Simulation set-up, i.e., initial and boundary conditions, numerical schemes and (sub-)models used, for all cases, in particular for the design problem, were not explained;
     \item Overestimated $y^{+}$ (Fig. 17), possibly due to mesh resolution close to the wall;
     \item Research and analysis of the simulation results are fairly poor.
%
  \end{itemize}%\afterpage{\blankpage}

\clearpage 

%%%%%
%%%%%
%%%%%

\noindent{\bfseries\large Computational Fluid Dynamics (EG501V) \hfill December, 2015}

\bigskip

\noindent{\Large Watts, Glen Andrew (51123076)}

\medskip

  \begin{itemize}
%
     \item Poor report with poor comments and discussions;
     \item Poor/No explanation of thermal energy equations, wall treatment, FVM and solution methods;
     \item Definition and analysis of 1$^{\text{st}}$- and 2$^{\text{nd}}$-order accurate schemes are missing;
     \item Simulation set-up, i.e., initial and boundary conditions, numerical schemes and (sub-)models used, for all cases, in particular for the design problem, were not explained;
     \item Overestimated $y^{+}$ (Fig. 16), possibly due to mesh resolution close to the wall;
     \item Research and analysis of the simulation results are fairly poor.
%
  \end{itemize}%\afterpage{\blankpage}

\clearpage 


%%%%%
%%%%%
%%%%%

\noindent{\bfseries\large Computational Fluid Dynamics (EG501V) \hfill December, 2015}

\bigskip

\noindent{\Large Willmot, George (51120032)}

\medskip

  \begin{itemize}
%
     \item Good general report with good analysis but poor discussion;
     \item Good quality of equations with reasonable discussion;
     \item Poor/No explanation of thermal energy equations, wall treatment, FVM and solution methods;
     \item Overestimated $y^{+}$ (Fig. 21), possibly due to mesh resolution close to the wall.
%
  \end{itemize}%\afterpage{\blankpage}

\clearpage 


%%%%%
%%%%%
%%%%%

\noindent{\bfseries\large Computational Fluid Dynamics (EG501V) \hfill December, 2015}

\bigskip

\noindent{\Large Wilson, Stuart (51124654)}

\medskip

  \begin{itemize}
%
     \item Poor report with poor comments and discussions;
     \item Poor/No explanation of thermal energy equation, $k-\epsilon$ turbulence model, wall treatment, FVM and solution methods;
     \item No captions in Tables and in a few figures;
     \item Definition and analysis of 1$^{\text{st}}$- and 2$^{\text{nd}}$-order accurate schemes are missing;
     \item Simulation set-up, i.e., initial and boundary conditions, numerical schemes and (sub-)models used, for all cases, in particular for the design problem, were not explained;
     \item Overestimated $y^{+}$ (Fig. 15), possibly due to mesh resolution close to the wall;
     \item Also, Figs. 17 and 18 are not correct;
     \item Research and analysis of the simulation results are fairly poor.
%
  \end{itemize}%\afterpage{\blankpage}

\clearpage 


%%%%%
%%%%%
%%%%%

\noindent{\bfseries\large Computational Fluid Dynamics (EG501V) \hfill December, 2015}

\bigskip

\noindent{\Large Wright, Adam (51119529)}

\medskip

  \begin{itemize}
%
     \item Good general report work with reasonable discussion;
     \item No/Poor discussion/explanation of $k-\epsilon$ turbulence model, FVM, solution methods and wall-treatment;
     \item No captions in Figs and Tables;
     \item Overestimated $y^{+}$ (Fig. 15), possibly due to mesh resolution close to the wall;
     \item Good general discussion of the results but with poor analysis based on the theoretical background.
%
  \end{itemize}%\afterpage{\blankpage}

\clearpage 

%%%%%
%%%%%
%%%%%

\noindent{\bfseries\large Computational Fluid Dynamics (EG501V) \hfill December, 2015}

\bigskip

\noindent{\Large Wright, Linden (51122123)}

\medskip

  \begin{itemize}
%
     \item Good general report with reasonable analysis and discussion;
     \item No/Poor explanation of Thermal energy and $k-\epsilon$ turbulence equations, wall treatment, FVM and model solutions;
     \item Definition and analysis of 1$^{\text{st}}$- and 2$^{\text{nd}}$-order accurate schemes are missing;
     \item Overestimated $y^{+}$, possibly due to mesh resolution close to the wall;
     \item No captions in Figs and Tables;
     \item Good general discussion of the results but with poor analysis based on the theoretical background.
%
  \end{itemize}%\afterpage{\blankpage}

\clearpage 

%%%%%
%%%%%
%%%%%

\noindent{\bfseries\large Computational Fluid Dynamics (EG501V) \hfill December, 2015}

\bigskip

\noindent{\Large Young, Joseph (51119968)}

\medskip

  \begin{itemize}
%
     \item Excellent report with good analysis and discussion; 
     \item No/Poor discussion on wall treatment and solution methods
     \item Definition and analysis of 1$^{\text{st}}$- and 2$^{\text{nd}}$-order accurate schemes are missing;;
%
%
  \end{itemize}%\afterpage{\blankpage}

\clearpage 


%%%%%%%
%%%%%%%
%%%%%%%

%\lipsum % Text before
\afterpage{%
    \clearpage% Flush earlier floats (otherwise order might not be correct)
    \thispagestyle{empty}% empty page style (?)
    \begin{landscape}% Landscape page
        \centering % Center table

%\begin{center}
\Huge{MSc Dissertations}\\
\huge{(Review + Feedback)}\\
\huge{August-September 2015}
%\end{center}
\normalsize



\bigskip

%\begin{center}
\begin{tabular}{||c| c c c |c| c||}
\hline\hline
                           & {\bf Presentation and Style of} & {\bf Technical Content and}           & {\bf Evidence of Critical} & {\bf Discipline} & {\bf Averaged}  \\
                           & {\bf Writing (20$\%$)}          & {\bf Merit of Dissertation (50$\%$)}  & {\bf Reasoning (30$\%$)}   &                  & {\bf CGS Mark}  \\
\hline
Muftah M. Emhemed          &         14                      &           08                          &          08                &    O$\&$GE       &    D3 (09.20)   \\
Jonathan V. Hernandez      &         10                      &           10                          &          08                &    O$\&$GE       &    D3 (09.40)   \\
Gheve Mohan                &         14                      &           16                          &          15                &  Safety (Campus) & (13.05+ Conduct)\\
Toluwaleke Odunuga         &         18                      &           13                          &          09                &    Petroleum     &    C2 (12.80)   \\
Samuel Akosa               &         15                      &           18                          &          18                &    O$\&$GE       &    B1 (17.40)   \\
Anindya Santoso            &         17                      &           17                          &          15                &    Petroleum     &    B2 (16.40)   \\
Ankit S.K. Gundecha        &         21                      &           20                          &          20                &    O$\&$GE       &    A3 (20.20)   \\
Igbinevo E. Ikponmwosa     &         16                      &           19                          &          18                &    O$\&$GE       &    A5 (18.10)   \\
Ken Dong                   &         11                      &           12                          &          09                &    Petroleum     &    D3 (10.90)   \\
\hline\hline
\end{tabular}
%\end{center}
    \end{landscape}
    \clearpage% Flush page
}

%\lipsum % Text after

\clearpage

\vfill 


%%%%%
%%%%%
%%%%%

\noindent{\bfseries\large MSc in Subsea Engineering\hfill December, 2015}

\bigskip

\begin{center}
{\Large Review of the MSc Dissertation $\lq$Installation of a Reeled Bundle Feasibility Study' by Richard Burke}
\end{center}

\medskip

The dissertation describes a feasibility study of a proposed reeled bundle system in subsea facilities in the North Sea. Mr Burke reviewed the main elements in subsea pipeline framework and assess the technical feasibilty of a `novel' reeled bundle system, from manufacturing to deployment. Although he his work heavly relied on his (and his industrial supervisor) professional experience, he undertook a relatively extensive literature review on the main technologies.

The manuscript is well-written with a very small number of typos. The dissertation was interesting to read with enough content to enable discussion and extensive analysis.  A few extra comments:
\begin{enumerate}
%
\item The main aim of {\it Abstracts} is to briefly describe the work undertaken by the author. In general {\it Abstracts} are divided in 4 parts: (i) motivation, (ii) main objectives, (iii) summary of the main procedures / techniques / technologies (optional) and (iv) main findings. The current {\it Abstract} encompasses (i-ii) and (iv) and, in part, iii).
%
\item The main {\it Introduction} section usually has the same (but more in-depth and descriptive) four parts of the {\it Abstract} and a brief summary of the remaining of the work. In addition, it is always expected a few clear statements -re main background (thus recent innovations related to the main topic), initial literature review and, most of all, technological / scientific gaps in the current understanding. Also, it is expected a summary of the remaining sections at the end of the {\it Introduction}. In this dissertation, the {\it Introduction} section covered all the above. Literature review is the focus of the second chapter. 
%
\item Extensive analysis on the proposed design with main focus on hypothetical operation conditions.  
%% 
\end{enumerate}

The topic is very relevant for the energy sector. The student demonstrated that he had an excellent  understanding of the main technologies for this project.



\clearpage


%%%%%
%%%%%
%%%%%

\noindent{\bfseries\large MSc in Subsea Engineering\hfill December, 2015}

\bigskip

\begin{center}
{\Large Review of the MSc Dissertation $\lq$Comparative Study of SHO and PLET Terminated Steel Catenary Riser' by Achoyamen M. Ogbeifun}
\end{center}

\medskip

The dissertation investigates different strategies to connect steel catenary risers to subsea structures. The manuscript describes the different mechanisms for connection and safety operation and configuration criteria. Mr Ogbeifun undertook a limited literature review of the main technologies.

The manuscript is relatively well-written with a number of typos and unrevised sentences. Few sentences are very confusing and disconnected with no clear objectives. In general, the dissertation was interesting to read with enough content to enable discussion and analysis.  A few extra comments:
\begin{enumerate}
%
\item Dissertations and thesis are always divided into chapters rather than sections (commonly used in reports).
%
\item The main aim of {\it Abstracts} is to briefly describe the work undertaken by the author. In general {\it Abstracts} are divided in 4 parts: (i) motivation, (ii) main objectives, (iii) summary of the main procedures / techniques / technologies (optional) and (iv) main findings. The current {\it Abstract} encompasses (i) and (iv) and, in part, (ii) and (iii).
%
\item The main {\it Introduction} section usually has the same (but more in-depth and descriptive) four parts of the {\it Abstract} and a brief summary of the remaining of the work. In addition, it is always expected a few clear statements -re main background (thus recent innovations related to the main topic), initial literature review and, most of all, technological / scientific gaps in the current understanding. Also, it is expected a summary of the remaining sections at the end of the {\it Introduction}. In this dissertation, the {\it Introduction} section covered all the above. Literature review is the focus of `Section 2'. 
%
\item A few symbols/acronyms were used with no prior description. In particular `FLET', although crucial in the work was never defined.
%
\item Appendices are used to convey complementary (and not crucial) information of the main chapters and need to be referenced in the main text.
%
\item A few {\it References} follows different standards with missing fields and no clear distinction between articles, conference proceedings, reports (internal or external), book chapters, books, communications (internal or external) etc.  A few {\it references} used in the manuscript are incomplete and/or wrong. Regardless of the chosen citation style (e.g., ACS, AIP, AMS, IEEE, AIAA, etc) any reference {\bf must} contain the following fields: 
\begin{enumerate}
\item For journal papers: Authors, Paper Tittle, Journal Name, Volume, Pages, Year of publication;
\item For books: Authors, Book Tittle, Publisher, Year or Edition;
\item For book chapters: Authors, Chapter Tittle, Book Tittle, Editors, Publisher, Year or Edition;
\item For conference papers: Authors, Paper Tittle, Conference Tittle, Place (Country and/or City) where the conference was held, Year of the conference;
\item For reports,  private communications and Lecture Notes: Authors, Tittle, Place issued (Country and/or City and Institution where the document was originated), Year;
\item For PhD Thesis and MSc Dissertations: Author, Tittle, Institution (University and Department/School), Year.
\end{enumerate}  
Thus, for example:
\begin{enumerate}[label={[\arabic*]}]
\item P.L. Houtekamer and L. Mitchell, $\lq$Data Assimilation Using an Ensemble Kalman Filter Technique', {\it Monthly Weather Review}, 126:796-811, 1998.
\item K. Pruess, $\lq$Numerical Modelling of Gas Migration at a Proposed Repository for Low and Intermediate Level Nuclear Wastes', Technical Report LBL-25413, Lawrence Berkeley Laboratory, Berkeley (USA), 1990.
\item K. Aziz, A. Settari, {\it Fundamentals of Reservoir Simulation}, Elsevier Applied Science Publishers, New York (USA), 1986.
\item R.B. Lowrie, $\lq$Compact higher-Order Numerical Methods for Hyperbolic Conservation Laws', PhD Thesis, Department of Aerospace Engineering and Scientific Computing, University of Michigan (USA), 1996.
\end{enumerate} 
%
\item Equations \underline{must} be explained in full and all terms used must be defined afterwards as part of the main text.
%
\item A few figures and tables are $\lq$floating' with no explanation/description in the main text.
%% 
\end{enumerate}

The topic is very relevant for energy sector. The student demonstrated that he had a good understanding of the main technologies for this project.



\clearpage


%%%%%
%%%%%
%%%%%

\noindent{\bfseries\large MSc in Oil $\&$ Gas Engineering\hfill August, 2015}

\bigskip

\begin{center}
{\Large Review of the MSc Dissertation $\lq$Reservoir Simulation Overview with an Emphasis on History Matching' by Muftah Emhemed}
\end{center}

\medskip

The dissertation investigates reservoir simulation workflow with particular focus on the uncertainty associated with flow dynamics. The manuscript describes exploration, production and monitoring technologies used in the subsurface energy sector. Mr Emhemed undertook a rather limited literature review on these technologies.

The manuscript is relatively well-written with a small number of typos and unrevised sentences. Few sentences are very confusing and disconnected with no clear objectives. In general, the dissertation was interesting to read but with limited content to enable discussion and analysis. However, the analysis of the main topics -- history matching technologies,  was very superficial. A few extra comments:
\begin{enumerate}
\item The main aim of {\it Abstracts} is to briefly describe the work undertaken by the author. In general {\it Abstracts} are divided in 4 parts: (i) motivation, (ii) main objectives, (iii) summary of the main procedures / techniques / technologies (optional) and (iv) main findings. The current {\it Abstract} encompasses (i) and, (in part) (ii) and (iv).
%
\item The main {\it Introduction} section usually has the same (but more in-depth and descriptive) four parts of the {\it Abstract} and a brief summary of the remaining of the work. In addition, it is always expected a few clear statements -re main background (thus recent innovations related to the main topic), initial literature review and, most of all, technological / scientific gaps in the current understanding. Also, it is expected a summary of the remaining sections at the end of the {\it Introduction}. In this dissertation, the {\it Introduction} section focused only on the motivation and summary of the work. Literature review is spread over the remaining chapters with limited critical analysis of the work undertaken by several authors. 
%
\item The manuscript was very confusing with no real focus on the goals described in the {\it Abstract} -- must of all, the level of analysis of the main technical subjects was extremely superficial. 
%
\item Equations \underline{must} be explained in full and all terms used must be defined afterwards as part of the main text (e.g., Eqns. 2.1, 4.1-3).
%
\item A few {\it References} follows different standards with missing fields and no clear distinction between articles, conference proceedings, reports (internal or external), book chapters, books, communications (internal or external) etc.  A few {\it references} used in the manuscript are incomplete and/or wrong. Regardless of the chosen citation style (e.g., ACS, AIP, AMS, IEEE, AIAA, etc) any reference {\bf must} contain the following fields: 
\begin{enumerate}
\item For journal papers: Authors, Paper Tittle, Journal Name, Volume, Pages, Year of publication;
\item For books: Authors, Book Tittle, Publisher, Year or Edition;
\item For book chapters: Authors, Chapter Tittle, Book Tittle, Editors, Publisher, Year or Edition;
\item For conference papers: Authors, Paper Tittle, Conference Tittle, Place (Country and/or City) where the conference was held, Year of the conference;
\item For reports,  private communications and Lecture Notes: Authors, Tittle, Place issued (Country and/or City and Institution where the document was originated), Year;
\item For PhD Thesis and MSc Dissertations: Author, Tittle, Institution (University and Department/School), Year.
\end{enumerate}  
Thus, for example:
\begin{enumerate}[label={[\arabic*]}]
\item P.L. Houtekamer and L. Mitchell, $\lq$Data Assimilation Using an Ensemble Kalman Filter Technique', {\it Monthly Weather Review}, 126:796-811, 1998.
\item K. Pruess, $\lq$Numerical Modelling of Gas Migration at a Proposed Repository for Low and Intermediate Level Nuclear Wastes', Technical Report LBL-25413, Lawrence Berkeley Laboratory, Berkeley (USA), 1990.
\item K. Aziz, A. Settari, {\it Fundamentals of Reservoir Simulation}, Elsevier Applied Science Publishers, New York (USA), 1986.
\item R.B. Lowrie, $\lq$Compact higher-Order Numerical Methods for Hyperbolic Conservation Laws', PhD Thesis, Department of Aerospace Engineering and Scientific Computing, University of Michigan (USA), 1996.
\end{enumerate} 
% 
\end{enumerate}

In general, the dissertation was interesting to read but extremely superficial. There is a clear lack of discussion on the main objective: history matching and the topics within (i) manual and automatic HM and (ii) data assimilation. Also, several crucial concepts are either not defined or not correct, e.g., saturation in single phase systems (pages 7-8), OOIP/OGIP, OWIP (page 7), adjoint method (page 20), Kalman filtering (page 21), etc.


The topic is very relevant for energy and environmental science sectors and each section has been the focus of several academic- and industrial-based studies worldwide with clear cross-fertilisation with physics (thermodynamics, fluid mechanics, cloud physics, shock-physics etc), geology $\&$ geophysics (e.g., lithography, petrology, geochemistry, etc) and computer science (e.g., software engineering, algorithms, parallel processing, etc). The student demonstrated that he had a limited understanding of the main fundamental physics and technologies for this project.



\clearpage


%%%%%
%%%%%
%%%%%

\noindent{\bfseries\large MSc in Oil $\&$ Gas Engineering\hfill August, 2015}

\bigskip

\begin{center}
{\Large Review of the MSc Dissertation $\lq$Subsea Tie-Back Simulation' by Jonathan Hernandez}
\end{center}

\medskip

The dissertation investigates heat transfer parameters in subsea tie-back systems with emphasis on cross-model benchmark analysis (PipeSim and a Matlab script). The manuscript describes some important technologies and parameters that are often considered for thermo-fluid dynamics analysis of subsea systems. Mr Hernandez undertook a rather limited literature review on these technologies and parameters.

The manuscript is poorly written with a large number of typos and unrevised sentences. Several sentences are very confusing and disconnected with no clear objectives. However, in general, the dissertation was interesting to read but with limited content to enable discussion and analysis. The main topics -- heat transfer technologies in subsea tie-back systems,  were covered in a very superficial way. A few extra comments:
\begin{enumerate}
\item The main aim of {\it Abstracts} is to briefly describe the work undertaken by the author. In general {\it Abstracts} are divided in 4 parts: (i) motivation, (ii) main objectives, (iii) summary of the main procedures / techniques / technologies (optional) and (iv) main findings. The current {\it Abstract} encompasses (i) and (ii), and partially (iii) and (iv).
%
\item The main {\it Introduction} section usually has the same (but more in-depth and descriptive) four parts of the {\it Abstract} and a brief summary of the remaining of the work. In addition, it is always expected a few clear statements -re main background (thus recent innovations related to the main topic), initial literature review and, most of all, technological / scientific gaps in the current understanding. Also, it is expected a summary of the remaining sections at the end of the {\it Introduction}. In this dissertation, the {\it Introduction} section focused only on the motivation and main objective of the work. Literature review is the focus of Chapter 1, however the main ub-topics of the project were listed and the main equations and correlations were summarised with no relation (and comments) with prior work from other authors. 
%
\item The manuscript was very confusing and the level of analysis of the main technical subjects was extremely superficial. 
%
\item Equations \underline{must} be explained in full and all terms used must be defined afterwards as part of the main text. Also, the font size used in the equation must be the same as the main text.
%
\item Quality of figures are very poor (legends were difficult to read). Also, several figures and tables are $\lq$floating' with no explanation/description in the main text.
%
\item Units were wrongly listed, e.g., page 24 $\lq$49.9 pa per meter' should be read as $\lq$49.9 Pa/m', page 25 $\lq$W/c*m$\hat{ }$\;2' should be read as $\lq$W/($^{\circ}$C.m$^{2}$)'.  
%
\item A few {\it References} follows different standards with missing fields and no clear distinction between articles, conference proceedings, reports (internal or external), book chapters, books, communications (internal or external) etc.  A few {\it references} used in the manuscript are incomplete and/or wrong. Regardless of the chosen citation style (e.g., ACS, AIP, AMS, IEEE, AIAA, etc) any reference {\bf must} contain the following fields: 
\begin{enumerate}
\item For journal papers: Authors, Paper Tittle, Journal Name, Volume, Pages, Year of publication;
\item For books: Authors, Book Tittle, Publisher, Year or Edition;
\item For book chapters: Authors, Chapter Tittle, Book Tittle, Editors, Publisher, Year or Edition;
\item For conference papers: Authors, Paper Tittle, Conference Tittle, Place (Country and/or City) where the conference was held, Year of the conference;
\item For reports,  private communications and Lecture Notes: Authors, Tittle, Place issued (Country and/or City and Institution where the document was originated), Year;
\item For PhD Thesis and MSc Dissertations: Author, Tittle, Institution (University and Department/School), Year.
\end{enumerate}  
Thus, for example:
\begin{enumerate}[label={[\arabic*]}]
\item P.L. Houtekamer and L. Mitchell, $\lq$Data Assimilation Using an Ensemble Kalman Filter Technique', {\it Monthly Weather Review}, 126:796-811, 1998.
\item K. Pruess, $\lq$Numerical Modelling of Gas Migration at a Proposed Repository for Low and Intermediate Level Nuclear Wastes', Technical Report LBL-25413, Lawrence Berkeley Laboratory, Berkeley (USA), 1990.
\item K. Aziz, A. Settari, {\it Fundamentals of Reservoir Simulation}, Elsevier Applied Science Publishers, New York (USA), 1986.
\item R.B. Lowrie, $\lq$Compact higher-Order Numerical Methods for Hyperbolic Conservation Laws', PhD Thesis, Department of Aerospace Engineering and Scientific Computing, University of Michigan (USA), 1996.
\end{enumerate} 
%
\item A few symbols/acronyms were used with no prior description.
%
\item Also, Python language was mentioned in Section 2.2.1, but never used.
% 
\end{enumerate}

In general, the dissertation was interesting to read but extremely superficial. There is a clear lack of discussion on the main objective: heat transfer technologies in subsea tie-back. Also, several crucial concepts are either not defined or not correct.

The topic is very relevant for the energy sector and each section has been the focus of several academic- and industrial-based studies worldwide with clear cross-fertilisation with physics (thermodynamics, fluid mechanics, shock-physics etc) and computer science (e.g., software engineering, algorithms, parallel processing, etc). The student demonstrated that he had a limited understanding of the main fundamental physics and technologies for this project.



\clearpage


%%%%%
%%%%%
%%%%%

\noindent{\bfseries\large MSc in Safety Engineering\hfill August, 2015}

\bigskip

\begin{center}
{\Large Review of the MSc Dissertation $\lq$Design Guide for Passive Fire Protection of Piping Systems and Supports' by Gheve Mohan}
\end{center}

\medskip

The dissertation aims to develop safety design guidelines for PFP of pipe system. The manuscript describes common industry-standard techniques for safety installation and operation of piping systems. Mr Mohan undertook a literature review on these techniques and associated technologies $\&$ parameters.

The manuscript is poorly written with a large number of typos and unrevised sentences. Several sentences are very confusing and disconnected with no clear objectives. However, in general, the dissertation was interesting to read with enough content to enable discussion and analysis. The main topics -- PFP technologies for piping systems,  were covered in a limited way. A few extra comments:
\begin{enumerate}
\item The main aim of {\it Abstracts} is to briefly describe the work undertaken by the author. In general {\it Abstracts} are divided in 4 parts: (i) motivation, (ii) main objectives, (iii) summary of the main procedures / techniques / technologies (optional) and (iv) main findings. The current {\it Abstract} encompasses (i), (ii) and (iv) and partially (iii).
%
\item The main {\it Introduction} section usually has the same (but more in-depth and descriptive) four parts of the {\it Abstract} and a brief summary of the remaining of the work. In addition, it is always expected a few clear statements -re main background (thus recent innovations related to the main topic), initial literature review and, most of all, technological / scientific gaps in the current understanding. Also, it is expected a summary of the remaining sections at the end of the {\it Introduction}. In this dissertation, the {\it Introduction} section focused only on the motivation and main objective of the work. Literature review is the focus of Chapter 2, however the main sub-topics of the project were listed and the main equations and correlations were summarised with no relation (and comments) with prior work from other authors. 
%
\item Equations \underline{must} be explained in full and all terms used must be defined afterwards as part of the main text. Also, the font size used in the equation must be the same as the main text.  
%
\item Appendices are used to convey complementary (and not crucial) information of the main chapters and need to be referenced in the main text.
%
\item A few {\it References} follows different standards with missing fields and no clear distinction between articles, conference proceedings, reports (internal or external), book chapters, books, communications (internal or external) etc.  A few {\it references} used in the manuscript are incomplete and/or wrong. Regardless of the chosen citation style (e.g., ACS, AIP, AMS, IEEE, AIAA, etc) any reference {\bf must} contain the following fields: 
\begin{enumerate}
\item For journal papers: Authors, Paper Tittle, Journal Name, Volume, Pages, Year of publication;
\item For books: Authors, Book Tittle, Publisher, Year or Edition;
\item For book chapters: Authors, Chapter Tittle, Book Tittle, Editors, Publisher, Year or Edition;
\item For conference papers: Authors, Paper Tittle, Conference Tittle, Place (Country and/or City) where the conference was held, Year of the conference;
\item For reports,  private communications and Lecture Notes: Authors, Tittle, Place issued (Country and/or City and Institution where the document was originated), Year;
\item For PhD Thesis and MSc Dissertations: Author, Tittle, Institution (University and Department/School), Year.
\end{enumerate}  
Thus, for example:
\begin{enumerate}[label={[\arabic*]}]
\item P.L. Houtekamer and L. Mitchell, $\lq$Data Assimilation Using an Ensemble Kalman Filter Technique', {\it Monthly Weather Review}, 126:796-811, 1998.
\item K. Pruess, $\lq$Numerical Modelling of Gas Migration at a Proposed Repository for Low and Intermediate Level Nuclear Wastes', Technical Report LBL-25413, Lawrence Berkeley Laboratory, Berkeley (USA), 1990.
\item K. Aziz, A. Settari, {\it Fundamentals of Reservoir Simulation}, Elsevier Applied Science Publishers, New York (USA), 1986.
\item R.B. Lowrie, $\lq$Compact higher-Order Numerical Methods for Hyperbolic Conservation Laws', PhD Thesis, Department of Aerospace Engineering and Scientific Computing, University of Michigan (USA), 1996.
\end{enumerate} 
% 
\end{enumerate}

The topic is very relevant for the safety science sector and each section has been the focus of several academic- and industrial-based studies worldwide with clear cross-fertilisation with physics (thermodynamics, fluid mechanics, shock-physics etc) and engineering (e.g., mechanical, chemical, petroleum, civil, etc). The student demonstrated that he had a sound knowledge and understanding of the main sciences and technologies for this project.



\clearpage


%%%%%
%%%%%
%%%%%

\noindent{\bfseries\large MSc in Petroleum Engineering\hfill August, 2015}

\bigskip

\begin{center}
{\Large Review of the MSc Dissertation $\lq$Oil PVT Properties Calculator' by Toluwaleke Odunuga}
\end{center}

\medskip

The dissertation aims to develop an MS Excel spreadsheet that calculates derived PVT properties based on a number of experimental correlations. The manuscript describes a few industry-standard algebraic expressions used to predict fluid properties, in particular GOR and bubblepoint pressure. Mr Odunuga undertook a limited literature review on methods and technologies used to predict/calculated these derived PVT properties but with no `in depth' discussion of any of them.

The manuscript is relatively well written with a small number of typos and unrevised sentences. In general, the dissertation was interesting to read with enough content to enable discussion and analysis. A few extra comments:
\begin{enumerate}
\item The main aim of {\it Abstracts} is to briefly describe the work undertaken by the author. In general {\it Abstracts} are divided in 4 parts: (i) motivation, (ii) main objectives, (iii) summary of the main procedures / techniques / technologies (optional) and (iv) main findings. The current {\it Abstract} encompasses all of them except (i).
%
\item The main {\it Introduction} section usually has the same (but more in-depth and descriptive) four parts of the {\it Abstract} and a brief summary of the remaining of the work. In addition, it is always expected a few clear statements -re main background (thus recent innovations related to the main topic), initial literature review and, most of all, technological / scientific gaps in the current understanding. Also, it is expected a summary of the remaining sections at the end of the {\it Introduction}. In this dissertation, the {\it Introduction} section focused on (i) and (ii) above in addition to the summary of the work. Literature review is the focus of Chapter 2, and although Mr Odunuga successfully defined the main parameters that his work was exploring, there was a clear lack of explanation of how the expressions were derived and the range of use.
%
\item A few {\it References} follows different standards with missing fields and no clear distinction between articles, conference proceedings, reports (internal or external), book chapters, books, communications (internal or external) etc.  A few {\it references} used in the manuscript are incomplete and/or wrong. Regardless of the chosen citation style (e.g., ACS, AIP, AMS, IEEE, AIAA, etc) any reference {\bf must} contain the following fields: 
\begin{enumerate}
\item For journal papers: Authors, Paper Tittle, Journal Name, Volume, Pages, Year of publication;
\item For books: Authors, Book Tittle, Publisher, Year or Edition;
\item For book chapters: Authors, Chapter Tittle, Book Tittle, Editors, Publisher, Year or Edition;
\item For conference papers: Authors, Paper Tittle, Conference Tittle, Place (Country and/or City) where the conference was held, Year of the conference;
\item For reports,  private communications and Lecture Notes: Authors, Tittle, Place issued (Country and/or City and Institution where the document was originated), Year;
\item For PhD Thesis and MSc Dissertations: Author, Tittle, Institution (University and Department/School), Year.
\end{enumerate}  
Thus, for example:
\begin{enumerate}[label={[\arabic*]}]
\item P.L. Houtekamer and L. Mitchell, $\lq$Data Assimilation Using an Ensemble Kalman Filter Technique', {\it Monthly Weather Review}, 126:796-811, 1998.
\item K. Pruess, $\lq$Numerical Modelling of Gas Migration at a Proposed Repository for Low and Intermediate Level Nuclear Wastes', Technical Report LBL-25413, Lawrence Berkeley Laboratory, Berkeley (USA), 1990.
\item K. Aziz, A. Settari, {\it Fundamentals of Reservoir Simulation}, Elsevier Applied Science Publishers, New York (USA), 1986.
\item R.B. Lowrie, $\lq$Compact higher-Order Numerical Methods for Hyperbolic Conservation Laws', PhD Thesis, Department of Aerospace Engineering and Scientific Computing, University of Michigan (USA), 1996.
\end{enumerate} 
%
\item A few figures were of poor quality;
%
\item The aim of the work was to develop a computational tool to quickly calculate derived PVT properties of fluids based on industry-standard algebraic expressions. When one is developing a computational tool is expected a validation of the tool and an assessment of its range of validity. Plots (and tables) in Chapter 4 were obtained from initially synthetic data for all correlations described in Chapters 2 and 3. There was no discussion of the data and analysis of the actual range of validity. No further analysis/discussion were conducted for cross-code validation (Table 4.9). 
% 
\end{enumerate}

The topic is very relevant for the energy sector and each section has been the focus of several academic- and industrial-based studies worldwide with clear cross-fertilisation with physics (thermodynamics, fluid mechanics, shock-physics etc) and engineering (e.g., mechanical, chemical, petroleum, civil, etc). The student demonstrated that he had a good knowledge and understanding of the main sciences and technologies for this project.


\clearpage


%%%%%
%%%%%
%%%%%

\noindent{\bfseries\large MSc in Oil $\&$ Gas Engineering\hfill September, 2015}

\bigskip

\begin{center}
{\Large Review of the MSc Dissertation $\lq$Assessment and Modelling of Reservoir PVT Properties' by Samuel P. Akosa}
\end{center}

\medskip

The main aim of this dissertation is to assess methods and technologies currently used in the energy sector to predict PVT properties of hydrocarbons. The dissertation describes general mass and energy conservation formulation (fluid phase equilibria) for closed and open thermodynamic systems and the variety of reservoir PVT models (black oil and compositional formulations). Mr Akosa undertook an {\it in-depth} literature review on methods and technologies used to predict/calculated fluid phase equilibria in hydrocarbon systems.


The manuscript is relatively well written with a number of typos and unrevised sentences. Few sentences are very confusing and disconnected with no clear objectives. However, the dissertation was interesting to read with good content to enable discussion and analysis. The main topic -- fluid phase equilibria in reservoir fluids, was well covered, including main established fundamentals and novel technologies. A few extra comments:
\begin{enumerate}
\item The main aim of {\it Abstracts} is to briefly describe the work undertaken by the author. In general {\it Abstracts} are divided in 4 parts: (i) motivation, (ii) main objectives, (iii) summary of the main procedures / techniques / technologies (optional) and (iv) main findings. The current {\it Abstract} encompasses (i-ii) and (iv).
%
\item The main {\it Introduction} section usually has the same (but more in-depth and descriptive) four parts of the {\it Abstract} and a brief summary of the remaining of the work. In addition, it is always expected a few clear statements -re main background (thus recent innovations related to the main topic), initial literature review and, most of all, technological / scientific gaps in the current understanding. Also, it is expected a summary of the remaining sections at the end of the {\it Introduction}. In this dissertation, the {\it Introduction} section focused on (i) and (ii) above in addition to the summary of the work. Literature review is the focus of Chapters 2-3.
%
\item A few {\it References} missed critical fields and made no clear distinction between articles, conference proceedings, reports (internal or external), book chapters, books, communications (internal or external) etc.  
%
\item A few figures and tables are $\lq$floating' with no explanation/description in the main text (e.g., Table 4.3 and Fig. 4.1). Also Figure B2 and Table B3 were not either referred or explained in both, the main text and in the Appendix.
%
\item Chapter 4 and Appendix A briefly described the fluid phase equilibria models developed by Mr Akosa and the benchmark (cross-code validation) undertaken. Although the amount of simulated data was impressive, the discussion of the results were superficial and slightly confusing.  
%
\item Mr Akosa \underline{must} avoid use {\it colloquial (informal / personal)} writing. Also, try to avoid long (and confusing) sentences.  
% 
\end{enumerate}

The topic is very relevant for the energy sector and each section has been the focus of several academic- and industrial-based studies worldwide with clear cross-fertilisation with physics (thermodynamics, fluid mechanics, shock-physics etc) and engineering (e.g., mechanical, chemical, petroleum, civil, etc). The student demonstrated that he had a solid knowledge and understanding of the main sciences and technologies for this project.



\clearpage


%%%%%
%%%%%
%%%%%

\noindent{\bfseries\large MSc in Petroleum Engineering\hfill September, 2015}

\bigskip

\begin{center}
{\Large Review of the MSc Dissertation $\lq$Production Performance and Fluid Type, The Pembroke Field' by Anindya Santoso}
\end{center}

\medskip

The dissertation investigates the impact of heterogeneity in oil and gas production in the Pembroke field. The manuscript describes a systematic procedure to assess production performance in a hydrocarbon field with simplified upscaled heterogeneous and homogeneous porous matrix description. Ms Santoso undertook a literature review on the main techniques and associated technologies $\&$ parameters used to predict flow behaviour in porous media.

The manuscript is relatively well written with a number of typos and unrevised sentences. Few sentences are very confusing and disconnected with no clear objectives. However, the dissertation was interesting to read with good content to enable discussion and analysis. The main topic -- impact of heterogeneity in reservoir fluid flow, was well covered, including main established fundamentals and novel technologies. A few extra comments:
\begin{enumerate}
\item The main aim of {\it Abstracts} is to briefly describe the work undertaken by the author. In general {\it Abstracts} are divided in 4 parts: (i) motivation, (ii) main objectives, (iii) summary of the main procedures / techniques / technologies (optional) and (iv) main findings. The current {\it Abstract} encompasses (ii) and (iv).
%
\item The main {\it Introduction} section usually has the same (but more in-depth and descriptive) four parts of the {\it Abstract} and a brief summary of the remaining of the work. In addition, it is always expected a few clear statements -re main background (thus recent innovations related to the main topic), initial literature review and, most of all, technological / scientific gaps in the current understanding. Also, it is expected a summary of the remaining sections at the end of the {\it Introduction}. In this dissertation, the {\it Introduction} section focused only on the motivation and main objective of the work. Literature review is the focus of Chapter 2. 
%
\item Equations \underline{must} be explained in full and all terms used must be defined afterwards as part of the main text. Also, the font size used in the equation must be the same as the main text.  
%
\item A few {\it References} missed critical fields and made no clear distinction between articles, conference proceedings, reports (internal or external), book chapters, books, communications (internal or external) etc.  
%
\item Ms Santosa \underline{must} avoid use {\it colloquial (informal / personal)} writing. Also, try to avoid long (and confusing) sentences.  
%
\item Dissertations and thesis are always divided into chapters rather than sections (commonly used in reports).
%
\item The main focus of the work was the assessment of impact of heterogeneous (through permeability and porosity fields) in fluid production. However, the procedure for upscaling was neither explained (or described) nor discussed.     
% 
\end{enumerate}
The topic is very relevant for the energy sector and each section has been the focus of several academic- and industrial-based studies worldwide with clear cross-fertilisation with physics (thermodynamics, fluid mechanics, shock-physics etc) and engineering (e.g., mechanical, chemical, petroleum, civil, etc). The student demonstrated that she had a solid knowledge and understanding of the main sciences and technologies for this project.


\clearpage


%%%%%
%%%%%
%%%%%

\noindent{\bfseries\large MSc in Oil and Gas Engineering\hfill September, 2015}

\bigskip

\begin{center}
{\Large Review of the MSc Dissertation $\lq$Kinetic Modelling of Methane Gas Hydrates in Natural Gas Transmission Pipelines' by Ankit S.K. Gundecha}
\end{center}

\medskip

The dissertation investigates methane hydrate of natural gas formation and growth in pipelines. The manuscript describes kinetic models used in the description of consumption of methane from the natural gas during enclathration. Mr Gundecha undertook an `in-depth' literature review on the main science, engineering and associated technologies topics on hydrate formation $\&$ growth, mitigation $\&$ remediation.

The manuscript is well written with a small of typos and unrevised sentences. The dissertation was interesting to read with good connection between chapters and sections, the main topic -- kinetic models of methane hydrate was well covered, including main established fundamentals and novel technologies. A few extra comments:

\begin{enumerate}
\item The main aim of {\it Abstracts} is to briefly describe the work undertaken by the author. In general {\it Abstracts} are divided in 4 parts: (i) motivation, (ii) main objectives, (iii) summary of the main procedures / techniques / technologies (optional) and (iv) main findings. The current {\it Abstract} encompasses all of them.
%
\item The main {\it Introduction} section usually has the same (but more in-depth and descriptive) four parts of the {\it Abstract} and a brief summary of the remaining of the work. In addition, it is always expected a few clear statements -re main background (thus recent innovations related to the main topic), initial literature review and, most of all, technological / scientific gaps in the current understanding. Also, it is expected a summary of the remaining sections at the end of the {\it Introduction}. In this dissertation, the {\it Introduction} section focused only on the motivation and main objective of the work. Literature review is fully covered on Chapter 2. 
%
\item Font size used in the equation must be the same as the main text.  
%
\item A few {\it References} were placed far away from the first appearance.  
%
\item Appendices are used to convey complementary (and not crucial) information of the main chapters and need to be referenced in the main text..     
% 
\end{enumerate}
The topic is very relevant for the energy sector and each section has been the focus of several academic- and industrial-based studies worldwide with clear cross-fertilisation with physics (thermodynamics, fluid mechanics, shock-physics etc), chemistry (kinetics, surface chemistry, etc) and engineering (e.g., mechanical, chemical, petroleum, civil, etc). The student demonstrated that he had a solid knowledge and understanding of the main sciences and technologies for this project.


\clearpage


%%%%%
%%%%%
%%%%%

\noindent{\bfseries\large MSc in Petroleum Engineering\hfill September, 2015}

\bigskip

\begin{center}
{\Large Review of the MSc Dissertation $\lq$Formation and Stability of Natural Gas Clathrate Hydrates in Pipelines' by Igbinevbo Emmanuel Ikponmwosa}
\end{center}

\medskip

The dissertation describes current methods and technologies used to tackle the formation of NG hydrates in confined open systems. The manuscript investigates chemical clathrate structures, thermodynamic conditions for formation and stability of hydrates and a few generic industrial technologies to remediate pipeline blockages. The literature review undertook by Mr Ikponwosa was extensive wrt predictive technologies of hydrates' formation but superficial on technologies for mitigation and remediation. The performed numerical- and hand-calculations on composition and fugacities of a synthetic NGsolution stream to assess the potential formation of clathrates.

The manuscript is relatively well written with a number of typos and unrevised sentences. Few sentences are confusing and disconnected with no clear objectives. However, the dissertation was interesting to read with good connections between chapters and sections, leading to a self-contained work. The main topic -- thermodynamic models of NG clathrate hydrates, was well covered, including main established fundamentals and novel technologies. A few extra comments:
\begin{enumerate}
\item The main aim of {\it Abstracts} is to briefly describe the work undertaken by the author. In general {\it Abstracts} are divided in 4 parts: (i) motivation, (ii) main objectives, (iii) summary of the main procedures / techniques / technologies (optional) and (iv) main findings. The current {\it Abstract} encompasses all four.
%
\item The main {\it Introduction} section usually has the same (but more in-depth and descriptive) four parts of the {\it Abstract} and a brief summary of the remaining of the work. In addition, it is always expected a few clear statements -re main background (thus recent innovations related to the main topic), initial literature review and, most of all, technological / scientific gaps in the current understanding. Also, it is expected a summary of the remaining sections at the end of the {\it Introduction}. In this dissertation, the {\it Introduction} section covered all four parts, literature review is the focus of Chapter 2.  
%
\item A few {\it References} missed critical fields and made no clear distinction between articles, conference proceedings, reports (internal or external), book chapters, books, communications (internal or external) etc. Regardless of the chosen citation style (e.g., ACS, AIP, AMS, IEEE, AIAA, etc) any reference {\bf must} contain the following fields: 
\begin{enumerate}
\item For journal papers: Authors, Paper Tittle, Journal Name, Volume, Pages, Year of publication;
\item For books: Authors, Book Tittle, Publisher, Year or Edition;
\item For book chapters: Authors, Chapter Tittle, Book Tittle, Editors, Publisher, Year or Edition;
\item For conference papers: Authors, Paper Tittle, Conference Tittle, Place (Country and/or City) where the conference was held, Year of the conference;
\item For reports,  private communications and Lecture Notes: Authors, Tittle, Place issued (Country and/or City and Institution where the document was originated), Year;
\item For PhD Thesis and MSc Dissertations: Author, Tittle, Institution (University and Department/School), Year.
\end{enumerate}  
Thus, for example:
\begin{enumerate}[label={[\arabic*]}]
\item P.L. Houtekamer and L. Mitchell, $\lq$Data Assimilation Using an Ensemble Kalman Filter Technique', {\it Monthly Weather Review}, 126:796-811, 1998.
\item K. Pruess, $\lq$Numerical Modelling of Gas Migration at a Proposed Repository for Low and Intermediate Level Nuclear Wastes', Technical Report LBL-25413, Lawrence Berkeley Laboratory, Berkeley (USA), 1990.
\item K. Aziz, A. Settari, {\it Fundamentals of Reservoir Simulation}, Elsevier Applied Science Publishers, New York (USA), 1986.
\item R.B. Lowrie, $\lq$Compact higher-Order Numerical Methods for Hyperbolic Conservation Laws', PhD Thesis, Department of Aerospace Engineering and Scientific Computing, University of Michigan (USA), 1996.
\end{enumerate} 
%
\item A few figures were of poor quality;
%
\end{enumerate}
The topic is very relevant for the energy sector and each section has been the focus of several academic- and industrial-based studies worldwide with clear cross-fertilisation with physics (thermodynamics, fluid mechanics, shock-physics etc) and engineering (e.g., mechanical, chemical, petroleum, civil, etc). The student demonstrated that he had a solid knowledge and understanding of the main sciences and technologies for this project.


\clearpage
%%%%%
%%%%%
%%%%%

\noindent{\bfseries\large MSc in Petroleum Engineering\hfill September, 2015}

\bigskip

\begin{center}
{\Large Review of the MSc Dissertation $\lq$Thermodynamic Modelling of Well Fluids' by Ken Dong}
\end{center}

\medskip

The dissertation describes standard methods to calculate PVT behaviour of hydrocarbons found in O$\&$G reservoirs. The manuscript revises the main fundamentals and methods to calculate boiling, dew conditions. The literature review undertook by Mr Dong was extensive but with superficial analysis. Simulations were performed with both an Excel spreadsheet devloped by the Michigan State University and the commercial UniSim process simulator.

The manuscript is poorly written with a large number of typos and unrevised sentences. Several sentences are very confusing and disconnected with no clear objectives. However, in general, the dissertation was interesting to read but with limited content to enable discussion and analysis. The main topic -- vapour-liquid equilibrium calculations in reservoir systems, was covered in a very superficial way. A few extra comments:
\begin{enumerate}
\item The main aim of {\it Abstracts} is to briefly describe the work undertaken by the author. In general {\it Abstracts} are divided in 4 parts: (i) motivation, (ii) main objectives, (iii) summary of the main procedures / techniques / technologies (optional) and (iv) main findings. The current {\it Abstract} encompasses (ii) and partially (i) and (iii).
%
\item The main {\it Introduction} section usually has the same (but more in-depth and descriptive) four parts of the {\it Abstract} and a brief summary of the remaining of the work. In addition, it is always expected a few clear statements -re main background (thus recent innovations related to the main topic), initial literature review and, most of all, technological / scientific gaps in the current understanding. Also, it is expected a summary of the remaining sections at the end of the {\it Introduction}. In this dissertation, the {\it Introduction} section covered motivation and basic PVT properties background. 
%
\item Main theoretical background supported by (superficial) literature review were the focus of Chapter 2.  
%
\item A few {\it References} missed critical fields and made no clear distinction between articles, conference proceedings, reports (internal or external), book chapters, books, communications (internal or external) etc. Regardless of the chosen citation style (e.g., ACS, AIP, AMS, IEEE, AIAA, etc) any reference {\bf must} contain the following fields: 
\begin{enumerate}
\item For journal papers: Authors, Paper Tittle, Journal Name, Volume, Pages, Year of publication;
\item For books: Authors, Book Tittle, Publisher, Year or Edition;
\item For book chapters: Authors, Chapter Tittle, Book Tittle, Editors, Publisher, Year or Edition;
\item For conference papers: Authors, Paper Tittle, Conference Tittle, Place (Country and/or City) where the conference was held, Year of the conference;
\item For reports,  private communications and Lecture Notes: Authors, Tittle, Place issued (Country and/or City and Institution where the document was originated), Year;
\item For PhD Thesis and MSc Dissertations: Author, Tittle, Institution (University and Department/School), Year.
\end{enumerate}  
Thus, for example:
\begin{enumerate}[label={[\arabic*]}]
\item P.L. Houtekamer and L. Mitchell, $\lq$Data Assimilation Using an Ensemble Kalman Filter Technique', {\it Monthly Weather Review}, 126:796-811, 1998.
\item K. Pruess, $\lq$Numerical Modelling of Gas Migration at a Proposed Repository for Low and Intermediate Level Nuclear Wastes', Technical Report LBL-25413, Lawrence Berkeley Laboratory, Berkeley (USA), 1990.
\item K. Aziz, A. Settari, {\it Fundamentals of Reservoir Simulation}, Elsevier Applied Science Publishers, New York (USA), 1986.
\item R.B. Lowrie, $\lq$Compact higher-Order Numerical Methods for Hyperbolic Conservation Laws', PhD Thesis, Department of Aerospace Engineering and Scientific Computing, University of Michigan (USA), 1996.
\end{enumerate} 
%
\item Quality of few figures are very poor (legends were difficult to read). Also, several figures and tables are $\lq$floating' with no explanation/description in the main text.
%
\item Equations \underline{must} be explained in full and all terms used must be defined afterwards as part of the main text. Also, the font size used in the equation must be the same as the main text.  
%
\item The work is about well fluids but all simulations were performed with n-C$_{4}$ and n-C$_{7}$. Why did you choose these two fluids to represent well fluids?
%
\end{enumerate}
The topic is very relevant for the energy sector and each section has been the focus of several academic- and industrial-based studies worldwide with clear cross-fertilisation with physics (thermodynamics, fluid mechanics, shock-physics etc) and engineering (e.g., mechanical, chemical, petroleum, civil, etc). The student demonstrated that he had a basic knowledge and understanding of the main sciences and technologies for this project.


\clearpage



%%%%%%%
%%%%%%%
%%%%%%%

%\lipsum % Text before
\afterpage{%
    \clearpage% Flush earlier floats (otherwise order might not be correct)
    \thispagestyle{empty}% empty page style (?)
    \begin{landscape}% Landscape page
        \centering % Center table

%\begin{center}
\Huge{MEng Thesis Assessment  (EG4013)}\\
\huge{(Review + Feedback)}\\
\huge{May 2015}
%\end{center}
\normalsize


\vfill


    \end{landscape}
    \clearpage% Flush page
}

%\lipsum % Text after

\vfill

\clearpage




%%%%%% 
%%%%%%
%%%%%%


\noindent{\bfseries\large EG4515 -- MEng Thesis Assessment \hfill May, 2015}

\bigskip

\begin{center}
  {\Large Review of the MEng Thesis $\lq$Optimising Reactor Bed Configuration to Maximise Octane Rating from Heptane Hydroisomerisation' by Glen Andrew Watts}
\end{center}

The manuscript investigates the effectiveness of isomerisation processes with respect to conversion and selectivity for iso-heptane production. Mr Watts performed conversion experiments to assess catalyst activity and selectivity and reactor's performance. The dissertation encompasses three main subject areas within the main topic (organic conversion): organic chemistry mechanisms, chemical kinetics and catalysis.

The manuscript is relatively well-written with a small number of typos and unrevised sentences. Few sentences are confusing and disconnected with no clear objectives and inter-connectivities. Most of all, the paper lacks appropriate numbering of chapters and sections. This leads to a relatively difficult and confusing reading. A few general comments,

\begin{enumerate}
%
\item The main aim of {\it Abstracts} is to briefly describe the work undertaken by the author. In general {\it Abstracts} are divided in 4 parts: (i) motivation, (ii) main objectives, (iii) summary of the main procedures / techniques / technologies (optional) and (iv) main findings. The current {\it Abstract} encompass (i-ii) and (iv). 
%
\item The main {\it Introduction} section usually has the same (but more in-depth and descriptive) four parts of the {\it Abstract} and a brief summary of the remaining of the work. In addition, it is \underline{always} expected a few clear statements -re main background (thus recent innovations related to the main topic), initial literature review and, most of all, technological / scientific gaps in the current understanding. Also, it is expected a summary of the remaining sections at the end of the {\it Introduction}.  Current {\it Introduction} covered (in some extent) a few aspects of the above but lacked explain/summarise the main state-of-the-art aspects of the subject area. 
%
\item Dissertations and thesis are always divided into chapters rather than sections (commonly used in reports).
%
\item You \underline{must} avoid use {\it colloquial (informal / personal)} writing. Also, try to avoid long sentences. 
%
\item A few {\it References} follows different standards with missing fields and no clear distinction between articles, conference proceedings, reports (internal or external), book chapters, books, communications (internal or external) etc.  A few {\it references} used in the manuscript are incomplete and/or wrong. Regardless of the chosen citation style (e.g., ACS, AIP, AMS, IEEE, AIAA, etc) any reference {\bf must} contain the following fields: 
\begin{enumerate}
\item For journal papers: Authors, Paper Tittle, Journal Name, Volume, Pages, Year of publication;
\item For books: Authors, Book Tittle, Publisher, Year or Edition;
\item For book chapters: Authors, Chapter Tittle, Book Tittle, Editors, Publisher, Year or Edition;
\item For conference papers: Authors, Paper Tittle, Conference Tittle, Place (Country and/or City) where the conference was held, Year of the conference;
\item For reports,  private communications and Lecture Notes: Authors, Tittle, Place issued (Country and/or City and Institution where the document was originated), Year;
\item For PhD Thesis and MSc Dissertations: Author, Tittle, Institution (University and Department/School), Year.
\end{enumerate}  
Thus, for example:
\begin{enumerate}[label={[\arabic*]}]
\item P.L. Houtekamer and L. Mitchell, $\lq$Data Assimilation Using an Ensemble Kalman Filter Technique', {\it Monthly Weather Review}, 126:796-811, 1998.
\item K. Pruess, $\lq$Numerical Modelling of Gas Migration at a Proposed Repository for Low and Intermediate Level Nuclear Wastes', Technical Report LBL-25413, Lawrence Berkeley Laboratory, Berkeley (USA), 1990.
\item K. Aziz, A. Settari, {\it Fundamentals of Reservoir Simulation}, Elsevier Applied Science Publishers, New York (USA), 1986.
\item R.B. Lowrie, $\lq$Compact higher-Order Numerical Methods for Hyperbolic Conservation Laws', PhD Thesis, Department of Aerospace Engineering and Scientific Computing, University of Michigan (USA), 1996.
\end{enumerate} 
%
\item Quality of a few figures is poor. Also, figures and tables {\bf must} be referenced in the main text -- they can not just $\lq$float around'! In addition, figure/table captions should be self-contained, i.e., with a good description of the figure/table highlighting the most relevant aspects/information that the author wants to convene. 
% 
\item A few chemical symbols are wrong and need to be revised.
%
\item Page 21: How do you ensure that the set of reactions occur in that particular order
%
\end{enumerate}

The paper describes isomerisation technologies for the catalytic synthesis of iso-heptane. The topic is very relevant for the energy sector, and each sub-topic has been the focus of several academic- and industrial-based studies worldwide with clear cross-fertilisation with physics (thermodynamics, fluid mechanics, surface chemistry, material science etc), chemistry (kinetics, catalysis, organic synthesis) and chemical engineering (kinetics engineering, reactor design etc). The student demonstrated that he had a sound understanding of the main technologies involved in this project.

\clearpage

%%%%%% 
%%%%%%
%%%%%%


\noindent{\bfseries\large EG4515 -- MEng Thesis Assessment \hfill May, 2015}

\bigskip

\begin{center}
  {\Large Review of the MEng Thesis $\lq$Optimisation of Energy Systems in {\it Smart Cities}' by David Andrew}
\end{center}

\begin{description}
\item[Examiner 1:]

The manuscript investigates current technologies for integrated power, heating and cooling systems for efficient and GHG emissions-controlled processes. Mr Andrew reported a few aspects of {\it smart cities} concept and how this concept can benefit from state-of-the-art integrated power and heating generation. The dissertation encompasses two main subject areas within the main topic (power generation and usage in {\it smart cities}): thermodynamic cycles (Rankine power and refrigeration systems) and energy and exergy optimisation (co- and tri-generation and pinch technology).

The manuscript is well-written with a small number of typos and unrevised sentences. Few sentences are confusing and disconnected with no clear objectives and inter-connectivities. Most of all, the dissertation is very well-structured with clear division and linkages between chapters, sections and paragraphs, leading to an easy and smooth reading. A few general comments,
\begin{enumerate}
%
\item The main aim of {\it Abstracts} is to briefly describe the work undertaken by the author. In general {\it Abstracts} are divided in 4 parts: (i) motivation, (ii) main objectives, (iii) summary of the main procedures / techniques / technologies (optional) and (iv) main findings. The current {\it Abstract} encompass (i-ii) and (iv). 
%
\item The main {\it Introduction} section usually has the same (but more in-depth and descriptive) four parts of the {\it Abstract} and a brief summary of the remaining of the work. In addition, it is \underline{always} expected a few clear statements -re main background (thus recent innovations related to the main topic), initial literature review and, most of all, technological / scientific gaps in the current understanding. Also, it is expected a summary of the remaining sections at the end of the {\it Introduction}.  Current {\it Introduction} covered (in some extent) most of the above but lacked to link the main subject area with the topics studied (and covered) in the dissertation. Literature review is spread over the remaining chapters with good critical analysis of the work undertaken by several authors. 
%
\item You \underline{must} avoid use {\it colloquial (informal / personal)} writing. Also, try to avoid long sentences. 
%
\item A few {\it References} follows different standards with missing fields and no clear distinction between articles, conference proceedings, reports (internal or external), book chapters, books, communications (internal or external) etc.  A few {\it references} used in the manuscript are incomplete and/or wrong. Regardless of the chosen citation style (e.g., ACS, AIP, AMS, IEEE, AIAA, etc) any reference {\bf must} contain the following fields: 
\begin{enumerate}
\item For journal papers: Authors, Paper Tittle, Journal Name, Volume, Pages, Year of publication;
\item For books: Authors, Book Tittle, Publisher, Year or Edition;
\item For book chapters: Authors, Chapter Tittle, Book Tittle, Editors, Publisher, Year or Edition;
\item For conference papers: Authors, Paper Tittle, Conference Tittle, Place (Country and/or City) where the conference was held, Year of the conference;
\item For reports,  private communications and Lecture Notes: Authors, Tittle, Place issued (Country and/or City and Institution where the document was originated), Year;
\item For PhD Thesis and MSc Dissertations: Author, Tittle, Institution (University and Department/School), Year.
\end{enumerate}  
Thus, for example:
\begin{enumerate}[label={[\arabic*]}]
\item P.L. Houtekamer and L. Mitchell, $\lq$Data Assimilation Using an Ensemble Kalman Filter Technique', {\it Monthly Weather Review}, 126:796-811, 1998.
\item K. Pruess, $\lq$Numerical Modelling of Gas Migration at a Proposed Repository for Low and Intermediate Level Nuclear Wastes', Technical Report LBL-25413, Lawrence Berkeley Laboratory, Berkeley (USA), 1990.
\item K. Aziz, A. Settari, {\it Fundamentals of Reservoir Simulation}, Elsevier Applied Science Publishers, New York (USA), 1986.
\item R.B. Lowrie, $\lq$Compact higher-Order Numerical Methods for Hyperbolic Conservation Laws', PhD Thesis, Department of Aerospace Engineering and Scientific Computing, University of Michigan (USA), 1996.
\end{enumerate} 
The student should also avoid relying on unconsolidated web-pages.
%
\item Quality of a few figures is poor. Also, figures and tables {\bf must} be referenced in the main text -- they can not just $\lq$float around'! In addition, figure/table captions should be self-contained, i.e., with a good description of the figure/table highlighting the most relevant aspects/information that the author wants to convene. 
% 
\item Why did Mr Andrew choose helium as the best option for the co-generation system ? Has this fluid been used before?
%
\item Why did Mr Andrew choose to perform simulations with UniSim? Are there other simulation tools (commercial or open-source) available?
%
\item Modelling and simulation discussion were very superficial.
%
\item The conclusions (or discussions) are very superficial and not really linked with Chapters 6 and 7. 
%
\end{enumerate}

The topic is very relevant for the energy and environmental sectors, and each sub-topic has been the focus of several academic- and industrial-based studies worldwide with clear cross-fertilisation with physics (thermodynamics, fluid mechanics, material science etc), applied mathematics (optimisation, flow simulation) and mechanical/chemical/material engineering (design of compressors, boilers, turbines, reactors etc; process design and optimisation etc). The student demonstrated that he had a good understanding of the main technologies involved in this project.

\item[Examiner 2:] `{\it Presentation and style:} The thesis is relatively well written, but graphs could be much improved. {\it Technical Content:} Too much text on cycles (that should be briefly mentioned only), but the modelling and discussion were too superficial. {\it Critical Reasoning:} There is little discussion in the thesis regarding the results and main conclusions.' 

\end{description}

\clearpage

%%%%%%%
%%%%%%%
%%%%%%%

\noindent{\bfseries\large EG4515 -- MEng Thesis Assessment \hfill May, 2015}

\bigskip

\begin{center}
  {\Large Review of the MEng Thesis $\lq$Optimisation of Energy Systems in {\it Smart Cities}' by Stuart Don}
\end{center}
\begin{description}
\item[Examiner 1:] The manuscript investigates current technologies for integrated power, heating and cooling systems for efficient and GHG emissions-controlled processes. Mr Don outlined the main thermodynamic cycles (vapour-/gas-power and refrigeration cycles) and the main current technologies for (C)CHP. The dissertation encompasses two main subject areas within the main topic (energy conversion in {\it smart cities}): fundamentals of thermodynamic cycles and current experience in (C)CHPs in UK. 

The manuscript is relatively well-written with a small number of typos and unrevised sentences. Few sentences are confusing and disconnected with no clear objectives and inter-connectivities. Most of all, the dissertation is well-structured with clear division and linkages between chapters, sections and paragraphs, leading to an easy and smooth reading. A few general comments,
\begin{enumerate}
%
\item The main aim of {\it Abstracts} is to briefly describe the work undertaken by the author. In general {\it Abstracts} are divided in 4 parts: (i) motivation, (ii) main objectives, (iii) summary of the main procedures / techniques / technologies (optional) and (iv) main findings. The current {\it Abstract} encompass (i) (in part) and (ii). Also, {\it Abstracts} is not considered as a chapter.
%
\item The main {\it Introduction} section usually has the same (but more in-depth and descriptive) four parts of the {\it Abstract} and a brief summary of the remaining of the work. In addition, it is \underline{always} expected a few clear statements -re main background (thus recent innovations related to the main topic), initial literature review and, most of all, technological / scientific gaps in the current understanding. Also, it is expected a summary of the remaining sections at the end of the {\it Introduction}.  Current {\it Introduction} is very short and covered (in a limited way) only (i) and (ii). Literature review is spread over the remaining chapters with very little critical analysis of the work undertaken by several authors. 
%
\item You \underline{must} avoid use {\it colloquial (informal / personal)} writing. Also, try to avoid long sentences. 
%
\item Quality of a few figures is poor. Also, figures and tables {\bf must} be referenced in the main text -- they can not just $\lq$float around'! In addition, figure/table captions should be self-contained, i.e., with a good and brief description of the figure/table highlighting the most relevant aspects/information that the author wants to convene. 
% 
\item There is no clear linkage between the sections of Chapter 3. The case studies (Section 3.3) is poorly referenced with the previous Sections and the whole work.
%
\item There is no real study of CO$_{2}$ emissions and technologies used in tri-generation, although those are outlined as main objectives of the project.
%
\end{enumerate}

The topic is very relevant for the energy and environmental sectors, and each sub-topic has been the focus of several academic- and industrial-based studies worldwide with clear cross-fertilisation with physics (thermodynamics, fluid mechanics, material science etc), applied mathematics (optimisation, flow simulation) and mechanical/chemical/material engineering (design of compressors, boilers, turbines, reactors etc; process design and optimisation etc). The student demonstrated that he had a basic understanding of the main technologies involved in this project.

\item[Examiner 2:] `The thesis presents a review of traditional thermodynamic cycles and combined heating and power generation technologies. The cities of Sheffield and Aberdeen were analysed/reviewed in terms of heat and power generation schemes, aiming at the optimisation of energy systems. It can be seen from the outset that the presentation and style of thesis is not at its best, e.g. quality of figures, overall look and accuracy and proper referencing, etc. The novelty and contribution of the research were not clearly justified and demonstrated, where a critical literature review is missing. Detailed analysis, technical insight and “research/results” are not in clear and convincing evidence.'

\end{description}

\clearpage


%%%%%%%
%%%%%%%
%%%%%%%

\noindent{\bfseries\large EG4515 -- MEng Thesis Assessment \hfill May, 2015}

\bigskip

\begin{center}  
  {\Large Review of the MEng Thesis $\lq$A Feasibility Study of CO$_{2}$ Capture and Storage' by Michael Ewen}
\end{center}

\begin{description}

\item[Examiner 1:] The dissertation investigates currently available technologies for CO$_{2}$ capture, transport and storage for mitigating GHG emissions in concentrated flow streams (i.e., carbon-based power stations). Most of all, the manuscript focuses on assessing the energy costs of injection of CO$_{2}$ in geological formations. Mr Ewen outlined the main processes and technologies involving capture (pre-, post- and oxy-fuel combustion), transportation and storage (long-term geological disposal sites and CO$_{2}$-EOR). The student also developed a simplified model to assess the energy/exergy budget for a CO$_{2}$-EOR unit. The dissertation encompasses three main subject areas within the main topic (energy optimisation): fundamentals of fluid and heat flows (in confined domains and in porous media), flow processing and current worldwide experience in CCS. 

The manuscript is well-written with a small number of typos and unrevised sentences. Although the dissertation was not divided into chapters, it is well-structured with clear division and linkages between the sections, leading to an easy and smooth reading. A few general comments,
\begin{enumerate}
%
\item Dissertations and thesis are always divided into chapters rather than sections (commonly used in reports). In addition, numbering in Section 2 is incorrect.
%
\item The main aim of {\it Abstracts} is to briefly describe the work undertaken by the author. In general {\it Abstracts} are divided in 4 parts: (i) motivation, (ii) main objectives, (iii) summary of the main procedures / techniques / technologies (optional) and (iv) main findings. The current {\it Abstract} encompass (i, ii, iv) and (iii) (in part). 
%
\item The main {\it Introduction} section usually has the same (but more in-depth and descriptive) four parts of the {\it Abstract} and a brief summary of the remaining of the work. In addition, it is \underline{always} expected a few clear statements -re main background (thus recent innovations related to the main topic), initial literature review and, most of all, technological / scientific gaps in the current understanding. Also, it is expected a summary of the remaining sections at the end of the {\it Introduction}.  Current {\it Introduction} covered (in some extent) all of above. Literature review is spread over the remaining chapters with good (although limited) critical analysis of the work undertaken by several authors. 
%
\item You \underline{must} avoid use {\it colloquial (informal / personal)} writing. Also, try to avoid long sentences. 
%
\item Quality of a few figures is poor. Also, figures and tables {\bf must} be referenced in the main text -- they can not just $\lq$float around'! (e.g., table in page 23) In addition, figure/table captions should be self-contained, i.e., with a good and brief description of the figure/table highlighting the most relevant aspects/information that the author wants to convene. Finally, figures and tables extracted from third-parties need to have the citation in the captions. 
%
\item A few symbols/acronyms were used with no prior description. For example, {\it API} in Eqn. 7.1, $m$ and $\mu$ in Eqn. 8.1. Also, as most of the equations are obtained from empirical correlations, the units for each term need to be introduced.
%
\item Although one of the objectives of the work was to undertake combined exergy and energy analysis, the former was not conducted.
%
\item Overall conclusions of the work and suggested future work are missing.
%
\end{enumerate}

The topic is very relevant for the energy and environmental sectors, and each sub-topic has been the focus of several academic- and industrial-based studies worldwide with clear cross-fertilisation with physics (thermodynamics, fluid mechanics, material science etc), applied mathematics (optimisation, flow simulation) and mechanical/chemical/material engineering (design of compressors, boilers, turbines, reactors etc; process design and optimisation etc). The student demonstrated that he had a basic understanding of the main technologies involved in this project.

\item[Examiner 2:] `Student has shown a literature review about CCS but there is nor originality neither sign of critical thinking. Poor case study, and no explanation of the terms of the equations in general. Massive use of reference [6] (IPCC, 2005) and some graphs/diagrams are of poor quality. Too much information pasted from IPCC, 2005 and some aspects of the CCS project were not updated.'

\end{description}

\clearpage

%%%%%%%
%%%%%%%
%%%%%%%

\noindent{\bfseries\large EG4515 -- MEng Thesis Assessment \hfill May, 2015}

\bigskip

\begin{center}
  {\Large Review of the MEng Thesis $\lq$A Mechanistic Study of the Fischer-Tropsch Reaction' by Christopher Slater}
\end{center}

The dissertation investigates Co-Alumina catalytic mechanisms for F-T synthesis. Mr Slater outlined the main technologies involving F-T process. The dissertation encompasses two main subject areas within the main topic (F-T synthesis): surface catalyst chemistry and organic reaction mechanisms. 

The manuscript is very well-written with a small number of typos and unrevised sentences. Although the dissertation was not divided into chapters, it is well-structured with clear division and linkages between the sections, leading to an easy and smooth reading. A few general comments,
\begin{enumerate}
%
\item Dissertations and thesis are always divided into chapters rather than sections (commonly used in reports). 
%
\item My main concern is that there is no clear indication of the actual objectives neither in the {\it Abstract} nor in the {\it General Introduction} Sections.
%
\item The main aim of {\it Abstracts} is to briefly describe the work undertaken by the author. In general {\it Abstracts} are divided in 4 parts: (i) motivation, (ii) main objectives, (iii) summary of the main procedures / techniques / technologies (optional) and (iv) main findings. The current {\it Abstract} encompass (i) and (iv). 
%
\item The main {\it Introduction} section usually has the same (but more in-depth and descriptive) four parts of the {\it Abstract} and a brief summary of the remaining of the work. In addition, it is \underline{always} expected a few clear statements -re main background (thus recent innovations related to the main topic), initial literature review and, most of all, technological / scientific gaps in the current understanding. Also, it is expected a summary of the remaining sections at the end of the {\it Introduction}.  Current {\it Introduction} covered (in some extent) all of above, except (ii). Literature review is spread over the remaining chapters with good critical analysis of the work undertaken by several authors. 
%
\item You \underline{must} avoid use {\it colloquial (informal / personal)} writing. Also, try to avoid long sentences.
%
\end{enumerate}

The topic is very relevant for the energy and environmental sectors, and each sub-topic has been the focus of several academic- and industrial-based studies worldwide with clear cross-fertilisation with physics (thermodynamics, fluid mechanics, surface chemistry, material science etc), chemistry (kinetics, catalysis, organic synthesis) and chemical engineering (kinetics engineering, reactor design etc). The student demonstrated that he had a good understanding of the main technologies involved in this project.

\clearpage

%%%%%%%
%%%%%%%
%%%%%%%

\noindent{\bfseries\large EG4515 -- MEng Thesis Assessment \hfill May, 2015}

\bigskip

\begin{center}
  {\Large Review of the MEng Thesis $\lq$A Study of Fluid Analysis in Reservoir Simulation' by Kristoffer Ritchie}
\end{center}
\begin{description}
\item[Examiner 1:] The dissertation investigates current technologies for fluid analysis and PVT behaviour assessment in reservoir engineering. Mr Ritchie undertook a comprehensive review of fluid sample techniques and phase equilibrium analysis currently used in O$\&$G industry.  He also performed a simplified VLE analysis in a synthetic hydrocarbon mixture. The dissertation encompasses two main subject areas within the main topic (reservoir simulation): thermodynamic analysis (PVT behaviour of multi-component mixture) and sampling techniques. 

The manuscript is very well-written with a small number of typos and unrevised sentences. The dissertation is well-structured with clear division and linkages between chapter and sections, leading to an easy and smooth reading. A few general comments,
\begin{enumerate} 
%
\item The main aim of {\it Abstracts} is to briefly describe the work undertaken by the author. In general {\it Abstracts} are divided in 4 parts: (i) motivation, (ii) main objectives, (iii) summary of the main procedures / techniques / technologies (optional) and (iv) main findings. The current {\it Abstract} encompass all 4 parts. 
%
\item The main {\it Introduction} section usually has the same (but more in-depth and descriptive) four parts of the {\it Abstract} and a brief summary of the remaining of the work. In addition, it is \underline{always} expected a few clear statements -re main background (thus recent innovations related to the main topic), initial literature review and, most of all, technological / scientific gaps in the current understanding. Also, it is expected a summary of the remaining sections at the end of the {\it Introduction}.  Current {\it Introduction} covered all of the above. Literature review is spread over the remaining chapters with comprehensive critical analysis of the work undertaken by several authors. 
%
\item You \underline{must} avoid use {\it colloquial (informal / personal)} writing. Also, try to avoid long sentences.
%
\item A few terms were not defined, e.g., $\omega$ in Eqn. 5.
%
\item Chapter 4 summarises the use of EOS to represent vapour-liquid equilibrium (PVT behaviour) under reservoir conditions. Although PR EOS is well outlined, there is no explanation on how to simulate liquid (or the equilibrium) behaviour under this conditions. This would be the key-point on the thermodynamic analysis for reservoir simulation.   
%
\end{enumerate}

The topic is very relevant for the energy and environmental sectors, and each sub-topic has been the focus of several academic- and industrial-based studies worldwide with clear cross-fertilisation with physics (thermodynamics, fluid mechanics, material science etc), mathematics (optimisation, differential equations etc) and chemical/reservoir engineering (flow simulators, reactor/reservoir design etc). The student demonstrated that he had a good understanding of the main technologies involved in this project.

\item[Examiner 2:] `Thesis is generally well presented as regards sectioning and writing. Aspects of presentation and style which could have been significantly improved were graphical content and referencing.

Thesis presents a literature review of technical material which doesn't extend far beyond what has been covered in taught modules. The review is, however, correct and logically presented. 

Technical analysis consists if implementing multicomponent PR model for a fluid consisting of 4 hydrocarbons $\left(C_{1}-C_{4}\right)$. Within this, IUPAC STP hasn't been used correctly; Fig. 12 and the labelling thereof is puzzling. 

Some evidence of critical reasoning in discussing MCPR with SAFT/PC-SAFT.'

\end{description}

\clearpage


%%%%%%%
%%%%%%%
%%%%%%%
%%%%%%%
%%%%%%%
%%%%%%%

\noindent{\bfseries\large EG4515 -- MEng Thesis Assessment \hfill May, 2015}

\bigskip

\begin{center}
  {\Large Review of the MEng Thesis $\lq$Study of Multiscale Waterflooding Mechanism in Hetterogeneous Reservoir Simulations' by Giulia Marzetti}
\end{center}

\begin{description}

\item[Examiner 1]: The dissertation investigates fluid instabilities in a waterflooding framework. Ms Marzetti undertook a comprehensive review of multi-fluid viscous instabilities in heterogeneous porous media. She also performed numerical simulations on a synthetic 2D domain with prescribed geometry, boundary conditions and morphology. The dissertation encompasses three main subject areas within the main topic (reservoir simulation): multi-fluid instabilities, multi-scale analysis and technologies for oil recovery. 

The manuscript is well-written with a small number of typos and unrevised sentences. Few sentences are confusing and disconnected with no clear objectives and inter-connectivities. The dissertation is well-structured with clear division and linkages between chapter and sections, leading to an easy and smooth reading. A few general comments,
\begin{enumerate} 
%
\item The main aim of {\it Abstracts} is to briefly describe the work undertaken by the author. In general {\it Abstracts} are divided in 4 parts: (i) motivation, (ii) main objectives, (iii) summary of the main procedures / techniques / technologies (optional) and (iv) main findings. The current {\it Abstract} encompass all 4 parts (although (ii) is not very clear). 
%
\item The main {\it Introduction} section usually has the same (but more in-depth and descriptive) four parts of the {\it Abstract} and a brief summary of the remaining of the work. In addition, it is \underline{always} expected a few clear statements -re main background (thus recent innovations related to the main topic), initial literature review and, most of all, technological / scientific gaps in the current understanding. Also, it is expected a summary of the remaining sections at the end of the {\it Introduction}.  Current {\it Introduction} covered all of the above. Literature review is spread over the remaining chapters with comprehensive critical analysis of the work undertaken by several authors. 
%
\item You \underline{must} avoid use {\it colloquial (informal / personal)} writing. Also, try to avoid long sentences.
%
\item A few equations are not correct, e.g., first term of Eqn. 2. 
%
\item Figures and tables {\bf must} be referenced in the main text -- they can not just $\lq$float around', e.g., Figs. 1, 2, 3, etc! In addition, figure/table captions should be self-contained, i.e., with a good and brief description of the figure/table highlighting the most relevant aspects/information that the author wants to convene.  
%
\item References 30, 40, 41,45, 46 and 48 are incorrect and/or incomplete.
%
\end{enumerate}

The topic is very relevant for the energy and environmental sectors, and each sub-topic has been the focus of several academic- and industrial-based studies worldwide with clear cross-fertilisation with physics (thermodynamics, fluid mechanics, material science etc), mathematics (optimisation, differential equations etc) and chemical/reservoir engineering (flow simulators, reactor/reservoir design etc). The student demonstrated that she had a good understanding of the main technologies involved in this project.

\item[Examiner 2:] `The work is good and the student has shown good understanding of the topic. The thesis is well-written and shows good critical analysis. Referencing is appropriate. Maybe some simulations would have been done to clarify some issues, as the student discussed in the conclusions (e.g., effect of the matrix resolution of the grid).'

\end{description}


\clearpage


%%%%%%%
%%%%%%%
%%%%%%%


%\lipsum % Text before
\afterpage{%
    \clearpage% Flush earlier floats (otherwise order might not be correct)
    \thispagestyle{empty}% empty page style (?)
    \begin{landscape}% Landscape page
        \centering % Center table

%\begin{center}
\Huge{Engineering Thermodynamics -- Continuous Assessment Report (EG3521)}\\
\huge{March 2015}
%\end{center}
\normalsize


\vfill


    \end{landscape}
    \clearpage% Flush page
}

%\lipsum % Text after

\vfill

\clearpage

%%%%%% 
%%%%%%
%%%%%%
\noindent{\bfseries\large EG3521 -- Engineering Thermodynamics (Continuous Assessment)  \hfill April, 2015}

\bigskip

\begin{center}
  {\Large Comments on the {\it Continuous Assessment Activities} -- Group 01}
\end{center}

\begin{enumerate}

\item Report:
\begin{enumerate}
%
\item The main aim of {\it Abstracts} is to briefly describe the work undertaken by the author. In general {\it Abstracts} are divided in 4 parts: (i) motivation, (ii) main objectives, (iii) summary of the main procedures / techniques / technologies (optional) and (iv) main findings. 
%
\item The main {\it Introduction} section usually has the same (but more in-depth and descriptive) four parts of the {\it Abstract} and a brief summary of the remaining of the work. In addition, it is \underline{always} expected a few clear statements -re main background (thus recent innovations related to the main topic), initial literature review and, most of all, technological / scientific gaps in the current understanding. Also, it is expected a summary of the remaining sections at the end of the {\it Introduction}.
%
\item Figures taken from other sources should be referenced.
%
\item Full stops are required after equations.
%
\item The fonts used for variables in the text should match the fonts used in equations.
%
\item The are some mis-used apostrophes (although apostrophe should be avoided altogether in scientific writing).
%
\item Lack of context placing article in related literature.
%
\item Having read your paper it's not clear who the authors of the paper you looked at are.
%
\item Very limited list of references and not referenced in the main text.

%\item Quality of a few figures are very poor. The font size for some of them is too small and no captions are present in some of them. Numbering is also not consistent. Also, several figures are $\lq$floating' with no explanation/description in the main text.
%
\item Avoid using {\it colloquial (informal / personal)} writing.
%
\item Regardless of the chosen citation style (e.g., ACS, AIP, AMS, IEEE, AIAA, etc) any reference {\bf must} contain the following fields: 
\begin{enumerate}
\item For journal papers: Authors, Paper Tittle, Journal Name, Volume, Pages, Year of publication;
\item For books: Authors, Book Tittle, Publisher, Year or Edition;
\item For book chapters: Authors, Chapter Tittle, Book Tittle, Editors, Publisher, Year or Edition;
\item For conference papers: Authors, Paper Tittle, Conference Tittle, Place (Country and/or City) where the conference was held, Year of the conference;
\item For reports,  private communications and Lecture Notes: Authors, Tittle, Place issued (Country and/or City and Institution where the document was originated), Year;
\item For PhD Thesis and MSc Dissertations: Author, Tittle, Institution (University and Department/School), Year.
\end{enumerate}  
Thus, for example:
\begin{enumerate}[label={[\arabic*]}]
\item P.L. Houtekamer and L. Mitchell, $\lq$Data Assimilation Using an Ensemble Kalman Filter Technique', {\it Monthly Weather Review}, 126:796-811, 1998.
\item K. Pruess, $\lq$Numerical Modelling of Gas Migration at a Proposed Repository for Low and Intermediate Level Nuclear Wastes', Technical Report LBL-25413, Lawrence Berkeley Laboratory, Berkeley (USA), 1990.
\item K. Aziz, A. Settari, {\it Fundamentals of Reservoir Simulation}, Elsevier Applied Science Publishers, New York (USA), 1986.
\item R.B. Lowrie, $\lq$Compact higher-Order Numerical Methods for Hyperbolic Conservation Laws', PhD Thesis, Department of Aerospace Engineering and Scientific Computing, University of Michigan (USA), 1996.
\end{enumerate}
%
\end{enumerate}

\item Oral Presentation:
\begin{enumerate}
%
\item Do NOT read from notes and/or screen. Look at and interact with your audience.
%
\item Graphics used appropriately to illustrate technical concepts to a general audience.
%
\item Good description of flow chart and discussion of equations.
%
\item Nice clean slide design.
%
\item Speak at a little slower pace to allow the entire audience to keep up.
%
\item Cue cards are supposed to have words on them that will remind the speaker what they want to say. They are not to be read off of. This defeats their purpose.
%
\item Be more enthusiastic, try to burst with enthusiasm, if you are not, your audience will not be enthusiastic to listen to you.
%
\end{enumerate}

\end{enumerate}


\clearpage


%%%%%% 
%%%%%%
%%%%%%
\noindent{\bfseries\large EG3521 -- Engineering Thermodynamics (Continuous Assessment)  \hfill April, 2015}

\bigskip

\begin{center}
  {\Large Comments on the {\it Continuous Assessment Activities} -- Group 02}
\end{center}

\begin{enumerate}

\item Report:
\begin{enumerate}
%
\item The main aim of {\it Abstracts} is to briefly describe the work undertaken by the author. In general {\it Abstracts} are divided in 4 parts: (i) motivation, (ii) main objectives, (iii) summary of the main procedures / techniques / technologies (optional) and (iv) main findings. 
%
\item The main {\it Introduction} section usually has the same (but more in-depth and descriptive) four parts of the {\it Abstract} and a brief summary of the remaining of the work. In addition, it is \underline{always} expected a few clear statements -re main background (thus recent innovations related to the main topic), initial literature review and, most of all, technological / scientific gaps in the current understanding. Also, it is expected a summary of the remaining sections at the end of the {\it Introduction}.
%
\item Something can't be $\lq$particularly ... unique' (page 3).
%
\item Very limited technical content.
%
\item Conclusions in different font to rest of text.
%
\item No figures or equations.
%
\item Very limited reference list and not referenced in the main text.
%
%\item Quality of a few figures are very poor. The font size for some of them is too small and no captions are present in some of them. Numbering is also not consistent. Also, several figures are $\lq$floating' with no explanation/description in the main text.
%
\item Avoid using {\it colloquial (informal / personal)} writing.
%
\item Regardless of the chosen citation style (e.g., ACS, AIP, AMS, IEEE, AIAA, etc) any reference {\bf must} contain the following fields: 
\begin{enumerate}
\item For journal papers: Authors, Paper Tittle, Journal Name, Volume, Pages, Year of publication;
\item For books: Authors, Book Tittle, Publisher, Year or Edition;
\item For book chapters: Authors, Chapter Tittle, Book Tittle, Editors, Publisher, Year or Edition;
\item For conference papers: Authors, Paper Tittle, Conference Tittle, Place (Country and/or City) where the conference was held, Year of the conference;
\item For reports,  private communications and Lecture Notes: Authors, Tittle, Place issued (Country and/or City and Institution where the document was originated), Year;
\item For PhD Thesis and MSc Dissertations: Author, Tittle, Institution (University and Department/School), Year.
\end{enumerate}  
Thus, for example:
\begin{enumerate}[label={[\arabic*]}]
\item P.L. Houtekamer and L. Mitchell, $\lq$Data Assimilation Using an Ensemble Kalman Filter Technique', {\it Monthly Weather Review}, 126:796-811, 1998.
\item K. Pruess, $\lq$Numerical Modelling of Gas Migration at a Proposed Repository for Low and Intermediate Level Nuclear Wastes', Technical Report LBL-25413, Lawrence Berkeley Laboratory, Berkeley (USA), 1990.
\item K. Aziz, A. Settari, {\it Fundamentals of Reservoir Simulation}, Elsevier Applied Science Publishers, New York (USA), 1986.
\item R.B. Lowrie, $\lq$Compact higher-Order Numerical Methods for Hyperbolic Conservation Laws', PhD Thesis, Department of Aerospace Engineering and Scientific Computing, University of Michigan (USA), 1996.
\end{enumerate}
%
\end{enumerate}

\item Oral Presentation:
\begin{enumerate}
%
\item Good time keeping.
%
\item Consistent/Uniform slides.
%
\item Delivery stuttered and unsure, probably due to nerves. Some speakers were low toned.
%
\item Slides over packed with text. Too many words on a slide makes the audience lose interest.
%
\item Delivery required more confidence and authority.
%
\item Very limited technical content of thermodynamics and no appropriate graphics used.
%
\item Be more enthusiastic, try to burst with enthusiasm, if you are not, your audience will not be enthusiastic to listen to you.
%
\end{enumerate}

\end{enumerate}


\clearpage


%%%%%% 
%%%%%%
%%%%%%
\noindent{\bfseries\large EG3521 -- Engineering Thermodynamics (Continuous Assessment)  \hfill April, 2015}

\bigskip

\begin{center}
  {\Large Comments on the {\it Continuous Assessment Activities} -- Group 03}
\end{center}

\begin{enumerate}

\item Report:
\begin{enumerate}
%
\item The main aim of {\it Abstracts} is to briefly describe the work undertaken by the author. In general {\it Abstracts} are divided in 4 parts: (i) motivation, (ii) main objectives, (iii) summary of the main procedures / techniques / technologies (optional) and (iv) main findings. 
%
\item The main {\it Introduction} section usually has the same (but more in-depth and descriptive) four parts of the {\it Abstract} and a brief summary of the remaining of the work. In addition, it is \underline{always} expected a few clear statements -re main background (thus recent innovations related to the main topic), initial literature review and, most of all, technological / scientific gaps in the current understanding. Also, it is expected a summary of the remaining sections at the end of the {\it Introduction}.
%
\item It's not made clear who the authors of the original paper actually are.
%
\item After an equation you should have either a comma (if the sentence continues below the equation), or a full stop (if the equation ends a sentence). Where the text after an equation continues with $\lq$where...', then this should have a lower case $\lq$W'.
%
\item Avoid the tendency to write in short one and two sentence paragraphs.
%
\item Figures/tables taken from papers should be referenced in the figure caption.
%
\item Avoid the use of apostrophes in scientific writing - say the $\lq$platform for the tool', rather than the $\lq$tool's platform'.
%
\item Referencing in the text is good, a suitable number of references is included and the bibliographic accuracy is good.
%
%\item Quality of a few figures are very poor. The font size for some of them is too small and no captions are present in some of them. Numbering is also not consistent. Also, several figures are $\lq$floating' with no explanation/description in the main text.
%
\item Avoid using {\it colloquial (informal / personal)} writing.
%
\item Regardless of the chosen citation style (e.g., ACS, AIP, AMS, IEEE, AIAA, etc) any reference {\bf must} contain the following fields: 
\begin{enumerate}
\item For journal papers: Authors, Paper Tittle, Journal Name, Volume, Pages, Year of publication;
\item For books: Authors, Book Tittle, Publisher, Year or Edition;
\item For book chapters: Authors, Chapter Tittle, Book Tittle, Editors, Publisher, Year or Edition;
\item For conference papers: Authors, Paper Tittle, Conference Tittle, Place (Country and/or City) where the conference was held, Year of the conference;
\item For reports,  private communications and Lecture Notes: Authors, Tittle, Place issued (Country and/or City and Institution where the document was originated), Year;
\item For PhD Thesis and MSc Dissertations: Author, Tittle, Institution (University and Department/School), Year.
\end{enumerate}  
Thus, for example:
\begin{enumerate}[label={[\arabic*]}]
\item P.L. Houtekamer and L. Mitchell, $\lq$Data Assimilation Using an Ensemble Kalman Filter Technique', {\it Monthly Weather Review}, 126:796-811, 1998.
\item K. Pruess, $\lq$Numerical Modelling of Gas Migration at a Proposed Repository for Low and Intermediate Level Nuclear Wastes', Technical Report LBL-25413, Lawrence Berkeley Laboratory, Berkeley (USA), 1990.
\item K. Aziz, A. Settari, {\it Fundamentals of Reservoir Simulation}, Elsevier Applied Science Publishers, New York (USA), 1986.
\item R.B. Lowrie, $\lq$Compact higher-Order Numerical Methods for Hyperbolic Conservation Laws', PhD Thesis, Department of Aerospace Engineering and Scientific Computing, University of Michigan (USA), 1996.
\end{enumerate}
%
\end{enumerate}

\item Oral Presentation:
\begin{enumerate}
%
\item Good reference to Aberdeen systems. 
%
\item Nice, simple, but very clear slide design, although body font could be a few pts higher. 
%
\item Watch for spelling mistakes. Proof read many times before submitting.
%
\item Graphics used appropriately to illustrate technical concepts to a general audience. 
%
\item Be more enthusiastic, try to burst with enthusiasm, if you are not, your audience will not be enthusiastic to listen to you.
%
\end{enumerate}

\end{enumerate}


\clearpage



%%%%%% 
%%%%%%
%%%%%%
\noindent{\bfseries\large EG3521 -- Engineering Thermodynamics (Continuous Assessment)  \hfill April, 2015}

\bigskip

\begin{center}
  {\Large Comments on the {\it Continuous Assessment Activities} -- Group 05}
\end{center}

\begin{enumerate}

\item Report:
\begin{enumerate}
%
\item The main aim of {\it Abstracts} is to briefly describe the work undertaken by the author. In general {\it Abstracts} are divided in 4 parts: (i) motivation, (ii) main objectives, (iii) summary of the main procedures / techniques / technologies (optional) and (iv) main findings. 
%
\item The main {\it Introduction} section usually has the same (but more in-depth and descriptive) four parts of the {\it Abstract} and a brief summary of the remaining of the work. In addition, it is \underline{always} expected a few clear statements -re main background (thus recent innovations related to the main topic), initial literature review and, most of all, technological / scientific gaps in the current understanding. Also, it is expected a summary of the remaining sections at the end of the {\it Introduction}.
%
\item It's not made clear who the authors of the original paper actually are.
%
\item In a 14 page document use sections rather than chapters.
%
\item Figures taken from other sources should be referenced.
%
\item The text in figure 1 is too small to read.
%
\item The use of $\lq$analytic method' to describe experiments seems strange.
%
\item What does the underscore at the end of equation 1 represent?
%
\item After an equation you should have either a comma (if the sentence continues below the equation), or a full stop (if the equation ends a sentence). Where the text after an equation continues with $\lq$where...', then this should have a lower case $\lq$W'.
%
\item Formatting of $\lq$A-A0-ii0ni0' is poor.
%
\item A later paper by Christophe Coquelet is mentioned, but there is no reference in the bibliography.
%
\item No references included in the text and not enough references in the bibliography.
%
%\item Quality of a few figures are very poor. The font size for some of them is too small and no captions are present in some of them. Numbering is also not consistent. Also, several figures are $\lq$floating' with no explanation/description in the main text.
%
\item Avoid using {\it colloquial (informal / personal)} writing.
%
\item Regardless of the chosen citation style (e.g., ACS, AIP, AMS, IEEE, AIAA, etc) any reference {\bf must} contain the following fields: 
\begin{enumerate}
\item For journal papers: Authors, Paper Tittle, Journal Name, Volume, Pages, Year of publication;
\item For books: Authors, Book Tittle, Publisher, Year or Edition;
\item For book chapters: Authors, Chapter Tittle, Book Tittle, Editors, Publisher, Year or Edition;
\item For conference papers: Authors, Paper Tittle, Conference Tittle, Place (Country and/or City) where the conference was held, Year of the conference;
\item For reports,  private communications and Lecture Notes: Authors, Tittle, Place issued (Country and/or City and Institution where the document was originated), Year;
\item For PhD Thesis and MSc Dissertations: Author, Tittle, Institution (University and Department/School), Year.
\end{enumerate}  
Thus, for example:
\begin{enumerate}[label={[\arabic*]}]
\item P.L. Houtekamer and L. Mitchell, $\lq$Data Assimilation Using an Ensemble Kalman Filter Technique', {\it Monthly Weather Review}, 126:796-811, 1998.
\item K. Pruess, $\lq$Numerical Modelling of Gas Migration at a Proposed Repository for Low and Intermediate Level Nuclear Wastes', Technical Report LBL-25413, Lawrence Berkeley Laboratory, Berkeley (USA), 1990.
\item K. Aziz, A. Settari, {\it Fundamentals of Reservoir Simulation}, Elsevier Applied Science Publishers, New York (USA), 1986.
\item R.B. Lowrie, $\lq$Compact higher-Order Numerical Methods for Hyperbolic Conservation Laws', PhD Thesis, Department of Aerospace Engineering and Scientific Computing, University of Michigan (USA), 1996.
\end{enumerate}
%
\end{enumerate}

\item Oral Presentation:
\begin{enumerate}
%
\item Good time keeping.
%
\item Confident presentation but some speakers reading notes.
%
\item Graphics used appropriately to illustrate technical concepts to a general audience. 
%
\item Slides were clear and fit for purpose with a good level of technical content which is well described, although note that font size of heading should NOT be smaller than that of main text.
%
\item Be more enthusiastic, try to burst with enthusiasm, if you are not, your audience will not be enthusiastic to listen to you.
%
\item Well done overall.
%
\end{enumerate}

\end{enumerate}


\clearpage


%%%%%% 
%%%%%%
%%%%%%
\noindent{\bfseries\large EG3521 -- Engineering Thermodynamics (Continuous Assessment)  \hfill April, 2015}

\bigskip

\begin{center}
  {\Large Comments on the {\it Continuous Assessment Activities} -- Group 06}
\end{center}

\begin{enumerate}

\item Report:
\begin{enumerate}
%
\item The main aim of {\it Abstracts} is to briefly describe the work undertaken by the author. In general {\it Abstracts} are divided in 4 parts: (i) motivation, (ii) main objectives, (iii) summary of the main procedures / techniques / technologies (optional) and (iv) main findings. 
%
\item The main {\it Introduction} section usually has the same (but more in-depth and descriptive) four parts of the {\it Abstract} and a brief summary of the remaining of the work. In addition, it is \underline{always} expected a few clear statements -re main background (thus recent innovations related to the main topic), initial literature review and, most of all, technological / scientific gaps in the current understanding. Also, it is expected a summary of the remaining sections at the end of the {\it Introduction}.
%
\item It's not made clear who the authors of the original paper actually are.
%
\item There's a tendency to waffle slightly -- keep sentences to the point.
%
\item There is no need to write the Journal name in the main text -- stick to the author (date) reference.
%
\item Incorporate figures closer to where they appear in the text, rather than having them all at the end of the document.
%
\item Put the reference for inserted figures in the figure caption, not immediately after the figure reference.
%
\item The technical details and discussion of the model is very limited.
%
\item Good set of appropriate references, but make sure all journal article references include volume and page numbers for the articles.
%
%\item Quality of a few figures are very poor. The font size for some of them is too small and no captions are present in some of them. Numbering is also not consistent. Also, several figures are $\lq$floating' with no explanation/description in the main text.
%
\item Avoid using {\it colloquial (informal / personal)} writing.
%
\item Regardless of the chosen citation style (e.g., ACS, AIP, AMS, IEEE, AIAA, etc) any reference {\bf must} contain the following fields: 
\begin{enumerate}
\item For journal papers: Authors, Paper Tittle, Journal Name, Volume, Pages, Year of publication;
\item For books: Authors, Book Tittle, Publisher, Year or Edition;
\item For book chapters: Authors, Chapter Tittle, Book Tittle, Editors, Publisher, Year or Edition;
\item For conference papers: Authors, Paper Tittle, Conference Tittle, Place (Country and/or City) where the conference was held, Year of the conference;
\item For reports,  private communications and Lecture Notes: Authors, Tittle, Place issued (Country and/or City and Institution where the document was originated), Year;
\item For PhD Thesis and MSc Dissertations: Author, Tittle, Institution (University and Department/School), Year.
\end{enumerate}  
Thus, for example:
\begin{enumerate}[label={[\arabic*]}]
\item P.L. Houtekamer and L. Mitchell, $\lq$Data Assimilation Using an Ensemble Kalman Filter Technique', {\it Monthly Weather Review}, 126:796-811, 1998.
\item K. Pruess, $\lq$Numerical Modelling of Gas Migration at a Proposed Repository for Low and Intermediate Level Nuclear Wastes', Technical Report LBL-25413, Lawrence Berkeley Laboratory, Berkeley (USA), 1990.
\item K. Aziz, A. Settari, {\it Fundamentals of Reservoir Simulation}, Elsevier Applied Science Publishers, New York (USA), 1986.
\item R.B. Lowrie, $\lq$Compact higher-Order Numerical Methods for Hyperbolic Conservation Laws', PhD Thesis, Department of Aerospace Engineering and Scientific Computing, University of Michigan (USA), 1996.
\end{enumerate}
%
\end{enumerate}

\item Oral Presentation:
\begin{enumerate}
%
\item Graphics used appropriately to illustrate technical concepts to a general audience. 
%
\item Confident delivery although required more authority. Do not give the impression to your audience that you are not sure of what you are saying.
%
\item Nice clearly designed slides.
%
\item Be more enthusiastic, try to burst with enthusiasm, if you are not, your audience will not be enthusiastic to listen to you.
%
\end{enumerate}

\end{enumerate}


\clearpage



%%%%%% 
%%%%%%
%%%%%%
\noindent{\bfseries\large EG3521 -- Engineering Thermodynamics (Continuous Assessment)  \hfill April, 2015}

\bigskip

\begin{center}
  {\Large Comments on the {\it Continuous Assessment Activities} -- Group 07}
\end{center}

\begin{enumerate}

\item Report:
\begin{enumerate}
%
\item The main aim of {\it Abstracts} is to briefly describe the work undertaken by the author. In general {\it Abstracts} are divided in 4 parts: (i) motivation, (ii) main objectives, (iii) summary of the main procedures / techniques / technologies (optional) and (iv) main findings. 
%
\item The main {\it Introduction} section usually has the same (but more in-depth and descriptive) four parts of the {\it Abstract} and a brief summary of the remaining of the work. In addition, it is \underline{always} expected a few clear statements -re main background (thus recent innovations related to the main topic), initial literature review and, most of all, technological / scientific gaps in the current understanding. Also, it is expected a summary of the remaining sections at the end of the {\it Introduction}.
%
\item It's not made clear who the authors or what the title of the original paper actually are.
%
\item You should write out the equations yourself rather than insert them as graphics.
%
\item Equations should fit within the standard sentence structure, so there shouldn't be a full stop immediately before the equation. 
%
\item After an equation you should have either a comma (if the sentence continues below the equation), or a full stop (if the equation ends a sentence).
%
\item The meaning of variables used in equations isn't clear.
%
\item Skip phrases like $\lq$the reader finds out' and $\lq$the manuscript says'.
%
\item Equations in the text are referenced, but the equations are not numbered, so it's unclear as to what the references are referring.
%
\item There is a lot of historical context, but not a lot of thermodynamics.
%
\item The number of references is on the low side.
%
%\item Quality of a few figures are very poor. The font size for some of them is too small and no captions are present in some of them. Numbering is also not consistent. Also, several figures are $\lq$floating' with no explanation/description in the main text.
%
\item Avoid using {\it colloquial (informal / personal)} writing.
%
\item Regardless of the chosen citation style (e.g., ACS, AIP, AMS, IEEE, AIAA, etc) any reference {\bf must} contain the following fields: 
\begin{enumerate}
\item For journal papers: Authors, Paper Tittle, Journal Name, Volume, Pages, Year of publication;
\item For books: Authors, Book Tittle, Publisher, Year or Edition;
\item For book chapters: Authors, Chapter Tittle, Book Tittle, Editors, Publisher, Year or Edition;
\item For conference papers: Authors, Paper Tittle, Conference Tittle, Place (Country and/or City) where the conference was held, Year of the conference;
\item For reports,  private communications and Lecture Notes: Authors, Tittle, Place issued (Country and/or City and Institution where the document was originated), Year;
\item For PhD Thesis and MSc Dissertations: Author, Tittle, Institution (University and Department/School), Year.
\end{enumerate}  
Thus, for example:
\begin{enumerate}[label={[\arabic*]}]
\item P.L. Houtekamer and L. Mitchell, $\lq$Data Assimilation Using an Ensemble Kalman Filter Technique', {\it Monthly Weather Review}, 126:796-811, 1998.
\item K. Pruess, $\lq$Numerical Modelling of Gas Migration at a Proposed Repository for Low and Intermediate Level Nuclear Wastes', Technical Report LBL-25413, Lawrence Berkeley Laboratory, Berkeley (USA), 1990.
\item K. Aziz, A. Settari, {\it Fundamentals of Reservoir Simulation}, Elsevier Applied Science Publishers, New York (USA), 1986.
\item R.B. Lowrie, $\lq$Compact higher-Order Numerical Methods for Hyperbolic Conservation Laws', PhD Thesis, Department of Aerospace Engineering and Scientific Computing, University of Michigan (USA), 1996.
\end{enumerate}
%
\end{enumerate}

\item Oral Presentation:
\begin{enumerate}
%
\item Do NOT read from notes. Look at and interact with your audience.
%
\item Delivery gave the impression of not understanding and/or being unsure of technical content.
%
\item Graphics not used appropriately to illustrate technical concepts to a general audience. 
%
\item Very little thermodynamic content.
%
\item Delivery lacked confidence and authority.
%
\item It would be better if all or majority of group members participate.
%
\item Be more enthusiastic, try to burst with enthusiasm, if you are not, your audience will not be enthusiastic to listen to you.
%
\end{enumerate}

\end{enumerate}


\clearpage


%%%%%% 
%%%%%%
%%%%%%
\noindent{\bfseries\large EG3521 -- Engineering Thermodynamics (Continuous Assessment)  \hfill April, 2015}

\bigskip

\begin{center}
  {\Large Comments on the {\it Continuous Assessment Activities} -- Group 08}
\end{center}

\begin{enumerate}

\item Report:
\begin{enumerate}
%
\item The main aim of {\it Abstracts} is to briefly describe the work undertaken by the author. In general {\it Abstracts} are divided in 4 parts: (i) motivation, (ii) main objectives, (iii) summary of the main procedures / techniques / technologies (optional) and (iv) main findings. 
%
\item The main {\it Introduction} section usually has the same (but more in-depth and descriptive) four parts of the {\it Abstract} and a brief summary of the remaining of the work. In addition, it is \underline{always} expected a few clear statements -re main background (thus recent innovations related to the main topic), initial literature review and, most of all, technological / scientific gaps in the current understanding. Also, it is expected a summary of the remaining sections at the end of the {\it Introduction}.
%
\item It's not made clear who the authors of the original paper actually are.
%
\item There are formatting issues with the contents page.
%
\item Articles mentioned in the text: Kirkpatrick (1959) and Ramey (1962) should appear on the bibliography.
%
\item The curly d symbol should not be used to represent the gas density.
%
\item It's not made clear what the variables in the equations represent.
%
\item Keep the font size used in equations consistent.
%
\item The formatting is very messy and inconsistent, which makes the text difficult to follow. The document appears very hurriedly put together with insufficient care and attention.
%
\item Don't copy equations as figures, you should write them out yourself.
%
\item The references to unsteady heat transfers are wrong as the model is steady throughout.
%
\item The number of references is insufficient.
%
%\item Quality of a few figures are very poor. The font size for some of them is too small and no captions are present in some of them. Numbering is also not consistent. Also, several figures are $\lq$floating' with no explanation/description in the main text.
%
\item Avoid using {\it colloquial (informal / personal)} writing.
%
\item Regardless of the chosen citation style (e.g., ACS, AIP, AMS, IEEE, AIAA, etc) any reference {\bf must} contain the following fields: 
\begin{enumerate}
\item For journal papers: Authors, Paper Tittle, Journal Name, Volume, Pages, Year of publication;
\item For books: Authors, Book Tittle, Publisher, Year or Edition;
\item For book chapters: Authors, Chapter Tittle, Book Tittle, Editors, Publisher, Year or Edition;
\item For conference papers: Authors, Paper Tittle, Conference Tittle, Place (Country and/or City) where the conference was held, Year of the conference;
\item For reports,  private communications and Lecture Notes: Authors, Tittle, Place issued (Country and/or City and Institution where the document was originated), Year;
\item For PhD Thesis and MSc Dissertations: Author, Tittle, Institution (University and Department/School), Year.
\end{enumerate}  
Thus, for example:
\begin{enumerate}[label={[\arabic*]}]
\item P.L. Houtekamer and L. Mitchell, $\lq$Data Assimilation Using an Ensemble Kalman Filter Technique', {\it Monthly Weather Review}, 126:796-811, 1998.
\item K. Pruess, $\lq$Numerical Modelling of Gas Migration at a Proposed Repository for Low and Intermediate Level Nuclear Wastes', Technical Report LBL-25413, Lawrence Berkeley Laboratory, Berkeley (USA), 1990.
\item K. Aziz, A. Settari, {\it Fundamentals of Reservoir Simulation}, Elsevier Applied Science Publishers, New York (USA), 1986.
\item R.B. Lowrie, $\lq$Compact higher-Order Numerical Methods for Hyperbolic Conservation Laws', PhD Thesis, Department of Aerospace Engineering and Scientific Computing, University of Michigan (USA), 1996.
\end{enumerate}
%
\end{enumerate}

\item Oral Presentation:
\begin{enumerate}
%
\item Do NOT read from notes and/or screen. Look at and interact with your audience.
%
\item Delivery gave the impression of not understanding and/or being unsure of technical content.
%
\item Graphics not used appropriately to illustrate technical concepts to a general audience. 
%
\item Font too small. 
%
\item Too much text on slides.
%
\item Poor time keeping.
%
\item Equations were poorly explained.
%
\item Delivery lacked confidence and authority.
%
\item Be more enthusiastic, try to burst with enthusiasm, if you are not, your audience will not be enthusiastic to listen to you.
%
\end{enumerate}

\end{enumerate}


\clearpage






%%%%%% 
%%%%%%
%%%%%%
\noindent{\bfseries\large EG3521 -- Engineering Thermodynamics (Continuous Assessment)  \hfill April, 2015}

\bigskip

\begin{center}
  {\Large Comments on the {\it Continuous Assessment Activities} -- Darren Mills (Paper 08)}
\end{center}

\begin{enumerate}

\item Report:
\begin{enumerate}
%
\item The main aim of {\it Abstracts} is to briefly describe the work undertaken by the author. In general {\it Abstracts} are divided in 4 parts: (i) motivation, (ii) main objectives, (iii) summary of the main procedures / techniques / technologies (optional) and (iv) main findings. 
%
\item The main {\it Introduction} section usually has the same (but more in-depth and descriptive) four parts of the {\it Abstract} and a brief summary of the remaining of the work. In addition, it is \underline{always} expected a few clear statements -re main background (thus recent innovations related to the main topic), initial literature review and, most of all, technological / scientific gaps in the current understanding. Also, it is expected a summary of the remaining sections at the end of the {\it Introduction}.
%
\item It's not made clear who the authors of the original paper actually are.
%
\item It's not made clear what the variables in the equations represent.
%
\item Keep the font size used in equations consistent.
%
\item The formatting is very messy and inconsistent, which makes the text difficult to follow.
%
\item Report should be self-contained, i.e., it should not rely on further reading to the main manuscript. E.g., Figures are referred to the text.
%
\item Don't copy equations as figures, you should write them out yourself.
%
\item Poor discussion of the thermodynamic content.
%
%\item Quality of a few figures are very poor. The font size for some of them is too small and no captions are present in some of them. Numbering is also not consistent. Also, several figures are $\lq$floating' with no explanation/description in the main text.
%
\item Avoid using {\it colloquial (informal / personal)} writing.
%
\item Regardless of the chosen citation style (e.g., ACS, AIP, AMS, IEEE, AIAA, etc) any reference {\bf must} contain the following fields: 
\begin{enumerate}
\item For journal papers: Authors, Paper Tittle, Journal Name, Volume, Pages, Year of publication;
\item For books: Authors, Book Tittle, Publisher, Year or Edition;
\item For book chapters: Authors, Chapter Tittle, Book Tittle, Editors, Publisher, Year or Edition;
\item For conference papers: Authors, Paper Tittle, Conference Tittle, Place (Country and/or City) where the conference was held, Year of the conference;
\item For reports,  private communications and Lecture Notes: Authors, Tittle, Place issued (Country and/or City and Institution where the document was originated), Year;
\item For PhD Thesis and MSc Dissertations: Author, Tittle, Institution (University and Department/School), Year.
\end{enumerate}  
Thus, for example:
\begin{enumerate}[label={[\arabic*]}]
\item P.L. Houtekamer and L. Mitchell, $\lq$Data Assimilation Using an Ensemble Kalman Filter Technique', {\it Monthly Weather Review}, 126:796-811, 1998.
\item K. Pruess, $\lq$Numerical Modelling of Gas Migration at a Proposed Repository for Low and Intermediate Level Nuclear Wastes', Technical Report LBL-25413, Lawrence Berkeley Laboratory, Berkeley (USA), 1990.
\item K. Aziz, A. Settari, {\it Fundamentals of Reservoir Simulation}, Elsevier Applied Science Publishers, New York (USA), 1986.
\item R.B. Lowrie, $\lq$Compact higher-Order Numerical Methods for Hyperbolic Conservation Laws', PhD Thesis, Department of Aerospace Engineering and Scientific Computing, University of Michigan (USA), 1996.
\end{enumerate}
%
\end{enumerate}

\item Oral Presentation:
\begin{enumerate}
%
\item Lacks main objective of the work;
%
\item Graphics not used appropriately to illustrate technical concepts to a general audience. 
%
\item Equations and key-content were poorly explained.
%
\item Equations are copied as gifs with poor quality.
%
\item Conclusions/Discussions are very superficial.
%
\end{enumerate}

\end{enumerate}


\clearpage

%%%%%% 
%%%%%%
%%%%%%
\noindent{\bfseries\large EG3521 -- Engineering Thermodynamics (Continuous Assessment)  \hfill April, 2015}

\bigskip

\begin{center}
  {\Large Comments on the {\it Continuous Assessment Activities} -- Group 09}
\end{center}

\begin{enumerate}

\item Report:
\begin{enumerate}
%
\item The main aim of {\it Abstracts} is to briefly describe the work undertaken by the author. In general {\it Abstracts} are divided in 4 parts: (i) motivation, (ii) main objectives, (iii) summary of the main procedures / techniques / technologies (optional) and (iv) main findings. 
%
\item The main {\it Introduction} section usually has the same (but more in-depth and descriptive) four parts of the {\it Abstract} and a brief summary of the remaining of the work. In addition, it is \underline{always} expected a few clear statements -re main background (thus recent innovations related to the main topic), initial literature review and, most of all, technological / scientific gaps in the current understanding. Also, it is expected a summary of the remaining sections at the end of the {\it Introduction}.
%
\item Avoid apostrophes in scientific writing - say $\lq$the population of the world' rather than $\lq$the world's population'.
%
\item Don't mix author/date and number referencing styles.
%
\item How is a solar power station (presumably subject to night and day even in orbit), able to produce a "round the clock" source of energy?
%
\item Include equations in the main text, rather than in an appendix.
%
\item A diagram comparing the different cycles would be useful.
%
\item There are no figures in the report.
%
\item Take care with subscripts, particularly on specifc heats.
%
\item The population of Earth isn't growing EXPONENTIALLY.
%
\item The relevance section repeats material discussed previously.
%
\item It's not clear what all the variables in the equations in the appendix represent. Most of these equations should be included in the main text.
%
\item Journal articles don't need URL in bibliography - the journal name, volume number and page numbers are required however.
%
%\item Quality of a few figures are very poor. The font size for some of them is too small and no captions are present in some of them. Numbering is also not consistent. Also, several figures are $\lq$floating' with no explanation/description in the main text.
%
\item Avoid using {\it colloquial (informal / personal)} writing.
%
\item Regardless of the chosen citation style (e.g., ACS, AIP, AMS, IEEE, AIAA, etc) any reference {\bf must} contain the following fields: 
\begin{enumerate}
\item For journal papers: Authors, Paper Tittle, Journal Name, Volume, Pages, Year of publication;
\item For books: Authors, Book Tittle, Publisher, Year or Edition;
\item For book chapters: Authors, Chapter Tittle, Book Tittle, Editors, Publisher, Year or Edition;
\item For conference papers: Authors, Paper Tittle, Conference Tittle, Place (Country and/or City) where the conference was held, Year of the conference;
\item For reports,  private communications and Lecture Notes: Authors, Tittle, Place issued (Country and/or City and Institution where the document was originated), Year;
\item For PhD Thesis and MSc Dissertations: Author, Tittle, Institution (University and Department/School), Year.
\end{enumerate}  
Thus, for example:
\begin{enumerate}[label={[\arabic*]}]
\item P.L. Houtekamer and L. Mitchell, $\lq$Data Assimilation Using an Ensemble Kalman Filter Technique', {\it Monthly Weather Review}, 126:796-811, 1998.
\item K. Pruess, $\lq$Numerical Modelling of Gas Migration at a Proposed Repository for Low and Intermediate Level Nuclear Wastes', Technical Report LBL-25413, Lawrence Berkeley Laboratory, Berkeley (USA), 1990.
\item K. Aziz, A. Settari, {\it Fundamentals of Reservoir Simulation}, Elsevier Applied Science Publishers, New York (USA), 1986.
\item R.B. Lowrie, $\lq$Compact higher-Order Numerical Methods for Hyperbolic Conservation Laws', PhD Thesis, Department of Aerospace Engineering and Scientific Computing, University of Michigan (USA), 1996.
\end{enumerate}
%
\end{enumerate}

\item Oral Presentation:
\begin{enumerate}
%
\item Do NOT read from notes and/or screen. Look at and interact with your audience.
%
\item Good speaker change over. 
%
\item Neat and nice looking slides.
%
\item Font on figures too small.
%
\item Equations are quite small.
%
\item All figures must have a caption.
%
\item Delivery lacked authority.
%
\item Speak at a pace that will allow a general audience to follow.
%
\item Body font too small.
%
\item Graphics used appropriately to illustrate technical concepts to a general audience.
%
\item Be more enthusiastic, try to burst with enthusiasm, if you are not, your audience will not be enthusiastic to listen to you.
%
\item Be more careful when typing equations. (Particularly subscripts).
%
\end{enumerate}

\end{enumerate}


\clearpage


%%%%%% 
%%%%%%
%%%%%%
\noindent{\bfseries\large EG3521 -- Engineering Thermodynamics (Continuous Assessment)  \hfill April, 2015}

\bigskip

\begin{center}
  {\Large Comments on the {\it Continuous Assessment Activities} -- Group 10}
\end{center}

\begin{enumerate}

\item Report:
\begin{enumerate}
%
\item The main aim of {\it Abstracts} is to briefly describe the work undertaken by the author. In general {\it Abstracts} are divided in 4 parts: (i) motivation, (ii) main objectives, (iii) summary of the main procedures / techniques / technologies (optional) and (iv) main findings. 
%
\item The main {\it Introduction} section usually has the same (but more in-depth and descriptive) four parts of the {\it Abstract} and a brief summary of the remaining of the work. In addition, it is \underline{always} expected a few clear statements -re main background (thus recent innovations related to the main topic), initial literature review and, most of all, technological / scientific gaps in the current understanding. Also, it is expected a summary of the remaining sections at the end of the {\it Introduction}.
%
\item There should be an introduction to your paper without diving straight into the lit review.
%
\item Siviter {\it et al.} should be followed by the year the article was published (page 4).
%
\item The first section, described as a literature review, doesn't have references to other references.
%
\item Figures and tables taken from other sources should be referenced.
%
\item Avoid writing in short 1 and 2 sentence paragraphs.
%
\item Page 11 - Asia is not a country.
%
\item Inconsistent formating - change of font for discussion.
%
\item Very limited number of figures in the report.
%
\item Little detailed thermodynamics.
%
\item There are a good number of references in the bibliography, but these are not referenced in the main text.
%
\item Journal articles don't need URL in bibliography - the journal name, volume number and page numbers are required however.

%\item Quality of a few figures are very poor. The font size for some of them is too small and no captions are present in some of them. Numbering is also not consistent. Also, several figures are $\lq$floating' with no explanation/description in the main text.
%
\item Avoid using {\it colloquial (informal / personal)} writing.
%
\item Regardless of the chosen citation style (e.g., ACS, AIP, AMS, IEEE, AIAA, etc) any reference {\bf must} contain the following fields: 
\begin{enumerate}
\item For journal papers: Authors, Paper Tittle, Journal Name, Volume, Pages, Year of publication;
\item For books: Authors, Book Tittle, Publisher, Year or Edition;
\item For book chapters: Authors, Chapter Tittle, Book Tittle, Editors, Publisher, Year or Edition;
\item For conference papers: Authors, Paper Tittle, Conference Tittle, Place (Country and/or City) where the conference was held, Year of the conference;
\item For reports,  private communications and Lecture Notes: Authors, Tittle, Place issued (Country and/or City and Institution where the document was originated), Year;
\item For PhD Thesis and MSc Dissertations: Author, Tittle, Institution (University and Department/School), Year.
\end{enumerate}  
Thus, for example:
\begin{enumerate}[label={[\arabic*]}]
\item P.L. Houtekamer and L. Mitchell, $\lq$Data Assimilation Using an Ensemble Kalman Filter Technique', {\it Monthly Weather Review}, 126:796-811, 1998.
\item K. Pruess, $\lq$Numerical Modelling of Gas Migration at a Proposed Repository for Low and Intermediate Level Nuclear Wastes', Technical Report LBL-25413, Lawrence Berkeley Laboratory, Berkeley (USA), 1990.
\item K. Aziz, A. Settari, {\it Fundamentals of Reservoir Simulation}, Elsevier Applied Science Publishers, New York (USA), 1986.
\item R.B. Lowrie, $\lq$Compact higher-Order Numerical Methods for Hyperbolic Conservation Laws', PhD Thesis, Department of Aerospace Engineering and Scientific Computing, University of Michigan (USA), 1996.
\end{enumerate}
%
\end{enumerate}

\item Oral Presentation:
\begin{enumerate}
%
\item Do NOT read from notes and/or screen. Look at and interact with your audience.
%
\item Slides were clear, consistent, and fit for purpose with a good level of technical content which is well described.
%
\item Graphics used appropriately to illustrate technical concepts to a general audience. 
%
\item Poor time keeping.
%
\item Delivery lacked confidence and authority.
%
\item Delivery was stuttered and unsure, probably due to nerves. 
%
\item Point at the projection on the screen and not at the computer screen so your audience can tell what you are pointing out.
%
\item Be more enthusiastic, try to burst with enthusiasm, if you are not, your audience will not be enthusiastic to listen to you.
%
\end{enumerate}

\end{enumerate}


\clearpage


%%%%%% 
%%%%%%
%%%%%%
\noindent{\bfseries\large EG3521 -- Engineering Thermodynamics (Continuous Assessment)  \hfill April, 2015}

\bigskip

\begin{center}
  {\Large Comments on the {\it Continuous Assessment Activities} -- Group 11}
\end{center}

\begin{enumerate}

\item Report:
\begin{enumerate}
%
\item The main aim of {\it Abstracts} is to briefly describe the work undertaken by the author. In general {\it Abstracts} are divided in 4 parts: (i) motivation, (ii) main objectives, (iii) summary of the main procedures / techniques / technologies (optional) and (iv) main findings. 
%
\item The main {\it Introduction} section usually has the same (but more in-depth and descriptive) four parts of the {\it Abstract} and a brief summary of the remaining of the work. In addition, it is \underline{always} expected a few clear statements -re main background (thus recent innovations related to the main topic), initial literature review and, most of all, technological / scientific gaps in the current understanding. Also, it is expected a summary of the remaining sections at the end of the {\it Introduction}.
%
\item Nice looking report.
%
\item References by Szargut {\it et al.} and Kotas {\it et al.} should have dates.
%
\item Figures taken from other sources should have references.
%
\item Equations should be followed by full stops (where they end sentences) and commas (where the sentence continues after the equation). You shouldn't have a full stop immediately before an equation.
%
\item Re-caption figures so their number appears correctly in the text.
%
\item Some figures on top of each other.
%
\item Some paragraphs have a blank line before starting the next paragraph - other times not.
%
\item Are there really 1000 different uses for LPG? Do you have a reference for that?
%
\item Limited number of references, which doesn't include the paper you're looking at. 
%
\item References not included in the text.
%\item Quality of a few figures are very poor. The font size for some of them is too small and no captions are present in some of them. Numbering is also not consistent. Also, several figures are $\lq$floating' with no explanation/description in the main text.
%
\item Avoid using {\it colloquial (informal / personal)} writing.
%
\item Regardless of the chosen citation style (e.g., ACS, AIP, AMS, IEEE, AIAA, etc) any reference {\bf must} contain the following fields: 
\begin{enumerate}
\item For journal papers: Authors, Paper Tittle, Journal Name, Volume, Pages, Year of publication;
\item For books: Authors, Book Tittle, Publisher, Year or Edition;
\item For book chapters: Authors, Chapter Tittle, Book Tittle, Editors, Publisher, Year or Edition;
\item For conference papers: Authors, Paper Tittle, Conference Tittle, Place (Country and/or City) where the conference was held, Year of the conference;
\item For reports,  private communications and Lecture Notes: Authors, Tittle, Place issued (Country and/or City and Institution where the document was originated), Year;
\item For PhD Thesis and MSc Dissertations: Author, Tittle, Institution (University and Department/School), Year.
\end{enumerate}  
Thus, for example:
\begin{enumerate}[label={[\arabic*]}]
\item P.L. Houtekamer and L. Mitchell, $\lq$Data Assimilation Using an Ensemble Kalman Filter Technique', {\it Monthly Weather Review}, 126:796-811, 1998.
\item K. Pruess, $\lq$Numerical Modelling of Gas Migration at a Proposed Repository for Low and Intermediate Level Nuclear Wastes', Technical Report LBL-25413, Lawrence Berkeley Laboratory, Berkeley (USA), 1990.
\item K. Aziz, A. Settari, {\it Fundamentals of Reservoir Simulation}, Elsevier Applied Science Publishers, New York (USA), 1986.
\item R.B. Lowrie, $\lq$Compact higher-Order Numerical Methods for Hyperbolic Conservation Laws', PhD Thesis, Department of Aerospace Engineering and Scientific Computing, University of Michigan (USA), 1996.
\end{enumerate}
%
\end{enumerate}

\item Oral Presentation:
\begin{enumerate}
%
\item Do NOT read from notes and/or screen. Look at and interact with your audience.
%
\item Graphics used appropriately to illustrate technical concepts to a general audience. 
%
\item Good description of formulas.
%
\item Slide layout hard to follow and not very practical. May be unappealing to certain audience.
%
\item Delivery lacked confidence and authority.
%
\item Delivery gave the impression of not understanding and/or being unsure of technical content.
%
\item Be more enthusiastic, try to burst with enthusiasm, if you are not, your audience will not be enthusiastic to listen to you.
%
\end{enumerate}

\end{enumerate}


\clearpage


\clearpage


%%%%%% 
%%%%%%
%%%%%%
\noindent{\bfseries\large EG3521 -- Engineering Thermodynamics (Continuous Assessment)  \hfill April, 2015}

\bigskip

\begin{center}
  {\Large Comments on the {\it Continuous Assessment Activities} -- Group 12}
\end{center}

\begin{enumerate}

\item Report:
\begin{enumerate}
%
\item The main aim of {\it Abstracts} is to briefly describe the work undertaken by the author. In general {\it Abstracts} are divided in 4 parts: (i) motivation, (ii) main objectives, (iii) summary of the main procedures / techniques / technologies (optional) and (iv) main findings. 
%
\item The main {\it Introduction} section usually has the same (but more in-depth and descriptive) four parts of the {\it Abstract} and a brief summary of the remaining of the work. In addition, it is \underline{always} expected a few clear statements -re main background (thus recent innovations related to the main topic), initial literature review and, most of all, technological / scientific gaps in the current understanding. Also, it is expected a summary of the remaining sections at the end of the {\it Introduction}.
%
\item Beyond academics at Kobe University, it's not clear who the authors of the paper you looked at actually are.
%
\item References by Hisadome {\it et al.} and Shugishita {\it et al.} should have dates published.
%
\item Avoid the use of apostrophes in scientific writing - say the $\lq$population of the world' rather than the $\lq$world's population'.
%
\item The section numbering is strange (2.2) follows (2) and there's no (1).
%
\item Equations should be followed by full stops (where they end sentences) and commas (where the sentence continues after the equation). You shouldn't have a full stop immediately before an equation.
%
\item Figures copied from elsewhere should have a reference.
%
\item In configuration section $\lq$as shown in figure x'.
%
\item The text on some of the figures is too small to read.
%
\item A $\lq$summary of the graphs' introduced, but then doesn't say anything.
%
\item Limited number of references.

%\item Quality of a few figures are very poor. The font size for some of them is too small and no captions are present in some of them. Numbering is also not consistent. Also, several figures are $\lq$floating' with no explanation/description in the main text.
%
\item Avoid using {\it colloquial (informal / personal)} writing.
%
\item Regardless of the chosen citation style (e.g., ACS, AIP, AMS, IEEE, AIAA, etc) any reference {\bf must} contain the following fields: 
\begin{enumerate}
\item For journal papers: Authors, Paper Tittle, Journal Name, Volume, Pages, Year of publication;
\item For books: Authors, Book Tittle, Publisher, Year or Edition;
\item For book chapters: Authors, Chapter Tittle, Book Tittle, Editors, Publisher, Year or Edition;
\item For conference papers: Authors, Paper Tittle, Conference Tittle, Place (Country and/or City) where the conference was held, Year of the conference;
\item For reports,  private communications and Lecture Notes: Authors, Tittle, Place issued (Country and/or City and Institution where the document was originated), Year;
\item For PhD Thesis and MSc Dissertations: Author, Tittle, Institution (University and Department/School), Year.
\end{enumerate}  
Thus, for example:
\begin{enumerate}[label={[\arabic*]}]
\item P.L. Houtekamer and L. Mitchell, $\lq$Data Assimilation Using an Ensemble Kalman Filter Technique', {\it Monthly Weather Review}, 126:796-811, 1998.
\item K. Pruess, $\lq$Numerical Modelling of Gas Migration at a Proposed Repository for Low and Intermediate Level Nuclear Wastes', Technical Report LBL-25413, Lawrence Berkeley Laboratory, Berkeley (USA), 1990.
\item K. Aziz, A. Settari, {\it Fundamentals of Reservoir Simulation}, Elsevier Applied Science Publishers, New York (USA), 1986.
\item R.B. Lowrie, $\lq$Compact higher-Order Numerical Methods for Hyperbolic Conservation Laws', PhD Thesis, Department of Aerospace Engineering and Scientific Computing, University of Michigan (USA), 1996.
\end{enumerate}
%
\end{enumerate}

\item Oral Presentation:
\begin{enumerate}
%
\item Do NOT read from notes and/or screen. Look at and interact with your audience.
%
\item Do NOT put/mention something in slides/presentation that you do not fully understand and cannot fully explain.
%
\item Do NOT speak too quickly, try to stay at a pace that a general audience can follow.
%
\item Graphics used appropriately to illustrate technical concepts to a general audience. 
%
\item Cue cards are supposed to have words on them that will remind the speaker what they want to say. They are not to be read off of. This defeats their purpose.
%
\item Good referencing to North Sea oil sector.
%
\item Good description of flowchart.
%
\item Lots of equations but some poorly explained.
%
\item Be more enthusiastic, try to burst with enthusiasm, if you are not, your audience will not be enthusiastic to listen to you.
%
\end{enumerate}

\end{enumerate}


\clearpage


%%%%%% 
%%%%%%
%%%%%%
\noindent{\bfseries\large EG3521 -- Engineering Thermodynamics (Continuous Assessment)  \hfill April, 2015}

\bigskip

\begin{center}
  {\Large Comments on the {\it Continuous Assessment Activities} -- Group 13}
\end{center}

\begin{enumerate}

\item Report:
\begin{enumerate}
%
\item The main aim of {\it Abstracts} is to briefly describe the work undertaken by the author. In general {\it Abstracts} are divided in 4 parts: (i) motivation, (ii) main objectives, (iii) summary of the main procedures / techniques / technologies (optional) and (iv) main findings. 
%
\item The main {\it Introduction} section usually has the same (but more in-depth and descriptive) four parts of the {\it Abstract} and a brief summary of the remaining of the work. In addition, it is \underline{always} expected a few clear statements -re main background (thus recent innovations related to the main topic), initial literature review and, most of all, technological / scientific gaps in the current understanding. Also, it is expected a summary of the remaining sections at the end of the {\it Introduction}.
%
\item It's not made clear who the authors of the original paper actually are.
%
\item Figures taken from other sources should be referenced.
%
\item Figure captions should explain what is shown in the figure.
%
\item Numbers for equations should be parallel to equation and not on the line above.
%
\item Good number of references, but inconsistent font usage in bibliography.
%\item Quality of a few figures are very poor. The font size for some of them is too small and no captions are present in some of them. Numbering is also not consistent. Also, several figures are $\lq$floating' with no explanation/description in the main text.
%
\item Avoid using {\it colloquial (informal / personal)} writing.
%
\item Regardless of the chosen citation style (e.g., ACS, AIP, AMS, IEEE, AIAA, etc) any reference {\bf must} contain the following fields: 
\begin{enumerate}
\item For journal papers: Authors, Paper Tittle, Journal Name, Volume, Pages, Year of publication;
\item For books: Authors, Book Tittle, Publisher, Year or Edition;
\item For book chapters: Authors, Chapter Tittle, Book Tittle, Editors, Publisher, Year or Edition;
\item For conference papers: Authors, Paper Tittle, Conference Tittle, Place (Country and/or City) where the conference was held, Year of the conference;
\item For reports,  private communications and Lecture Notes: Authors, Tittle, Place issued (Country and/or City and Institution where the document was originated), Year;
\item For PhD Thesis and MSc Dissertations: Author, Tittle, Institution (University and Department/School), Year.
\end{enumerate}  
Thus, for example:
\begin{enumerate}[label={[\arabic*]}]
\item P.L. Houtekamer and L. Mitchell, $\lq$Data Assimilation Using an Ensemble Kalman Filter Technique', {\it Monthly Weather Review}, 126:796-811, 1998.
\item K. Pruess, $\lq$Numerical Modelling of Gas Migration at a Proposed Repository for Low and Intermediate Level Nuclear Wastes', Technical Report LBL-25413, Lawrence Berkeley Laboratory, Berkeley (USA), 1990.
\item K. Aziz, A. Settari, {\it Fundamentals of Reservoir Simulation}, Elsevier Applied Science Publishers, New York (USA), 1986.
\item R.B. Lowrie, $\lq$Compact higher-Order Numerical Methods for Hyperbolic Conservation Laws', PhD Thesis, Department of Aerospace Engineering and Scientific Computing, University of Michigan (USA), 1996.
\end{enumerate}
%
\end{enumerate}

\item Oral Presentation:
\begin{enumerate}
%
\item Do NOT read from notes and/or screen. Look at and interact with your audience.
%
\item Graphics used appropriately to illustrate technical concepts to a general audience. 
%
\item Good authority and confidence in delivery. 
%
\item All figures must have a caption.
%
\item Don’t overcrowd a slide with too many words. This makes it harder for the audience to follow and eventually cause them to lose interest.
%
\item Good attitude and enthusiasm.
%
\item Simple neat slide design.
%
\end{enumerate}

\end{enumerate}


\clearpage


\clearpage


%%%%%% 
%%%%%%
%%%%%%
\noindent{\bfseries\large EG3521 -- Engineering Thermodynamics (Continuous Assessment)  \hfill April, 2015}

\bigskip

\begin{center}
  {\Large Comments on the {\it Continuous Assessment Activities} -- Group 14}
\end{center}

\begin{enumerate}

\item Report:
\begin{enumerate}
%
\item The main aim of {\it Abstracts} is to briefly describe the work undertaken by the author. In general {\it Abstracts} are divided in 4 parts: (i) motivation, (ii) main objectives, (iii) summary of the main procedures / techniques / technologies (optional) and (iv) main findings. 
%
\item The main {\it Introduction} section usually has the same (but more in-depth and descriptive) four parts of the {\it Abstract} and a brief summary of the remaining of the work. In addition, it is \underline{always} expected a few clear statements -re main background (thus recent innovations related to the main topic), initial literature review and, most of all, technological / scientific gaps in the current understanding. Also, it is expected a summary of the remaining sections at the end of the {\it Introduction}.
%
\item It's not made clear who the authors of the original paper actually are.
%
\item Simple but effective report style.
%
\item Figures and tables should have numbers and captions.
%
\item References in the text should have the form Wang {\it et al.} (2012) and Tchanche {\it et al.} (2009).
%
\item Equations should be followed by full stops (where they end sentences) and commas (where the sentence continues after the equation). You shouldn't have a full stop immediately before an equation. If the first word after an equation is $\lq$where', then this is a continuation of the preceeding sentence and should have a lower case $\lq$w'.
%
\item A subscript {\it n} becomes a subscript {\it j} in the description of some of the variables.
%
\item Table linked to Tchanche {\it et al.} is too small to read.
%
\item Lots of good material related to other sources.
%
\item Degrees missing from Celsius units.
%
\item You don't need the URL for a journal article in the bibliography, just the journal name, volume number and page number.

%\item Quality of a few figures are very poor. The font size for some of them is too small and no captions are present in some of them. Numbering is also not consistent. Also, several figures are $\lq$floating' with no explanation/description in the main text.
%
\item Avoid using {\it colloquial (informal / personal)} writing.
%
\item Regardless of the chosen citation style (e.g., ACS, AIP, AMS, IEEE, AIAA, etc) any reference {\bf must} contain the following fields: 
\begin{enumerate}
\item For journal papers: Authors, Paper Tittle, Journal Name, Volume, Pages, Year of publication;
\item For books: Authors, Book Tittle, Publisher, Year or Edition;
\item For book chapters: Authors, Chapter Tittle, Book Tittle, Editors, Publisher, Year or Edition;
\item For conference papers: Authors, Paper Tittle, Conference Tittle, Place (Country and/or City) where the conference was held, Year of the conference;
\item For reports,  private communications and Lecture Notes: Authors, Tittle, Place issued (Country and/or City and Institution where the document was originated), Year;
\item For PhD Thesis and MSc Dissertations: Author, Tittle, Institution (University and Department/School), Year.
\end{enumerate}  
Thus, for example:
\begin{enumerate}[label={[\arabic*]}]
\item P.L. Houtekamer and L. Mitchell, $\lq$Data Assimilation Using an Ensemble Kalman Filter Technique', {\it Monthly Weather Review}, 126:796-811, 1998.
\item K. Pruess, $\lq$Numerical Modelling of Gas Migration at a Proposed Repository for Low and Intermediate Level Nuclear Wastes', Technical Report LBL-25413, Lawrence Berkeley Laboratory, Berkeley (USA), 1990.
\item K. Aziz, A. Settari, {\it Fundamentals of Reservoir Simulation}, Elsevier Applied Science Publishers, New York (USA), 1986.
\item R.B. Lowrie, $\lq$Compact higher-Order Numerical Methods for Hyperbolic Conservation Laws', PhD Thesis, Department of Aerospace Engineering and Scientific Computing, University of Michigan (USA), 1996.
\end{enumerate}
%
\end{enumerate}

\item Oral Presentation:
\begin{enumerate}
%
\item Do NOT read from notes and/or screen. Look at and interact with your audience.
%
\item Good literature review.
%
\item Graphics used appropriately to illustrate technical concepts to a general audience. 
%
\item Nice slide design.
%
\item Text in figures too small.
%
\item Be more careful when type setting equations (particularly subscripts).
%
\item Perhaps too much swapping between speakers.
%
\item Good time keeping.
%
\item Be more enthusiastic, try to burst with enthusiasm, if you are not, your audience will not be enthusiastic to listen to you.
%
\item Some equations shown but not explained. Everything on a slide should be explained.
%
\end{enumerate}

\end{enumerate}


\clearpage


%%%%%% 
%%%%%%
%%%%%%
\noindent{\bfseries\large EG3521 -- Engineering Thermodynamics (Continuous Assessment)  \hfill April, 2015}

\bigskip

\begin{center}
  {\Large Comments on the {\it Continuous Assessment Activities} -- Group 15}
\end{center}

\begin{enumerate}

\item Report:
\begin{enumerate}
%
\item The main aim of {\it Abstracts} is to briefly describe the work undertaken by the author. In general {\it Abstracts} are divided in 4 parts: (i) motivation, (ii) main objectives, (iii) summary of the main procedures / techniques / technologies (optional) and (iv) main findings. 
%
\item The main {\it Introduction} section usually has the same (but more in-depth and descriptive) four parts of the {\it Abstract} and a brief summary of the remaining of the work. In addition, it is \underline{always} expected a few clear statements -re main background (thus recent innovations related to the main topic), initial literature review and, most of all, technological / scientific gaps in the current understanding. Also, it is expected a summary of the remaining sections at the end of the {\it Introduction}.
%
\item Nice looking report with clear figures and simple design.
%
\item Figure captions should explain what is shown in the figure.
%
\item Reference the authors of the paper, not the university where the authors were based - the bricks and mortar of the University of Ontario hasn't written anything.
%
\item There's no need to say a paper was published in the $\lq$International Journal of Energy Research' in the main text -- just give authors and date, and leave the rest to the bibliography.
%
\item Unnecessary change in font size at end of summary.
%
\item The amount of technical thermodynamic content is limited.
%
\item Inconsistent font usage in the bibliography.
%
\item It's not clear how the figures in the appendices related to the main text.
%
%\item Quality of a few figures are very poor. The font size for some of them is too small and no captions are present in some of them. Numbering is also not consistent. Also, several figures are $\lq$floating' with no explanation/description in the main text.
%
\item Avoid using {\it colloquial (informal / personal)} writing.
%
\item Regardless of the chosen citation style (e.g., ACS, AIP, AMS, IEEE, AIAA, etc) any reference {\bf must} contain the following fields: 
\begin{enumerate}
\item For journal papers: Authors, Paper Tittle, Journal Name, Volume, Pages, Year of publication;
\item For books: Authors, Book Tittle, Publisher, Year or Edition;
\item For book chapters: Authors, Chapter Tittle, Book Tittle, Editors, Publisher, Year or Edition;
\item For conference papers: Authors, Paper Tittle, Conference Tittle, Place (Country and/or City) where the conference was held, Year of the conference;
\item For reports,  private communications and Lecture Notes: Authors, Tittle, Place issued (Country and/or City and Institution where the document was originated), Year;
\item For PhD Thesis and MSc Dissertations: Author, Tittle, Institution (University and Department/School), Year.
\end{enumerate}  
Thus, for example:
\begin{enumerate}[label={[\arabic*]}]
\item P.L. Houtekamer and L. Mitchell, $\lq$Data Assimilation Using an Ensemble Kalman Filter Technique', {\it Monthly Weather Review}, 126:796-811, 1998.
\item K. Pruess, $\lq$Numerical Modelling of Gas Migration at a Proposed Repository for Low and Intermediate Level Nuclear Wastes', Technical Report LBL-25413, Lawrence Berkeley Laboratory, Berkeley (USA), 1990.
\item K. Aziz, A. Settari, {\it Fundamentals of Reservoir Simulation}, Elsevier Applied Science Publishers, New York (USA), 1986.
\item R.B. Lowrie, $\lq$Compact higher-Order Numerical Methods for Hyperbolic Conservation Laws', PhD Thesis, Department of Aerospace Engineering and Scientific Computing, University of Michigan (USA), 1996.
\end{enumerate}
%
\end{enumerate}

\item Oral Presentation:
\begin{enumerate}
%
\item Good time keeping.
%
\item Confident presentation but some speakers reading notes.
%
\item Graphics not used appropriately to illustrate technical concepts (e.g. Rankine Cycle) .
%
\item Text on figures not clear.
%
\item Everything you put on a slide should be explained (e.g. overall process digram).
%
\item Table not clear that values in $\lq$dual' column are per boiler.
%
\item Be more enthusiastic, try to burst with enthusiasm, if you are not, your audience will not be enthusiastic to listen to you.
%
\end{enumerate}

\end{enumerate}


\clearpage


\clearpage


%%%%%% 
%%%%%%
%%%%%%
\noindent{\bfseries\large EG3521 -- Engineering Thermodynamics (Continuous Assessment)  \hfill April, 2015}

\bigskip

\begin{center}
  {\Large Comments on the {\it Continuous Assessment Activities} -- Group 16}
\end{center}

\begin{enumerate}

\item Report:
\begin{enumerate}
%
\item The main aim of {\it Abstracts} is to briefly describe the work undertaken by the author. In general {\it Abstracts} are divided in 4 parts: (i) motivation, (ii) main objectives, (iii) summary of the main procedures / techniques / technologies (optional) and (iv) main findings. 
%
\item The main {\it Introduction} section usually has the same (but more in-depth and descriptive) four parts of the {\it Abstract} and a brief summary of the remaining of the work. In addition, it is \underline{always} expected a few clear statements -re main background (thus recent innovations related to the main topic), initial literature review and, most of all, technological / scientific gaps in the current understanding. Also, it is expected a summary of the remaining sections at the end of the {\it Introduction}.
%
\item It's not made clear who the authors of the original paper actually are.
%
\item I would have written down the simplest equations in the paper (1), (2) and (3) rather than diving in at (4), (5) and (6).
%
\item I appreciate the equations here are complicated, but you should at least try to state the physical conservation law that underpins the initial equations, and what all the variables mean.
%
\item The discussion of numerical solutions techniques is good.
%
\item References in the text should all have the date published, i.e. Groves {\it et al.} (1972) not Groves {\it at el.} This particular refence doesn't appear in the bibliography and it should.
%
\item The number of references in the bibliography is limited.
%
\item A good attempt at a difficult paper.
%
%\item Quality of a few figures are very poor. The font size for some of them is too small and no captions are present in some of them. Numbering is also not consistent. Also, several figures are $\lq$floating' with no explanation/description in the main text.
%
\item Avoid using {\it colloquial (informal / personal)} writing.
%
\item Regardless of the chosen citation style (e.g., ACS, AIP, AMS, IEEE, AIAA, etc) any reference {\bf must} contain the following fields: 
\begin{enumerate}
\item For journal papers: Authors, Paper Tittle, Journal Name, Volume, Pages, Year of publication;
\item For books: Authors, Book Tittle, Publisher, Year or Edition;
\item For book chapters: Authors, Chapter Tittle, Book Tittle, Editors, Publisher, Year or Edition;
\item For conference papers: Authors, Paper Tittle, Conference Tittle, Place (Country and/or City) where the conference was held, Year of the conference;
\item For reports,  private communications and Lecture Notes: Authors, Tittle, Place issued (Country and/or City and Institution where the document was originated), Year;
\item For PhD Thesis and MSc Dissertations: Author, Tittle, Institution (University and Department/School), Year.
\end{enumerate}  
Thus, for example:
\begin{enumerate}[label={[\arabic*]}]
\item P.L. Houtekamer and L. Mitchell, $\lq$Data Assimilation Using an Ensemble Kalman Filter Technique', {\it Monthly Weather Review}, 126:796-811, 1998.
\item K. Pruess, $\lq$Numerical Modelling of Gas Migration at a Proposed Repository for Low and Intermediate Level Nuclear Wastes', Technical Report LBL-25413, Lawrence Berkeley Laboratory, Berkeley (USA), 1990.
\item K. Aziz, A. Settari, {\it Fundamentals of Reservoir Simulation}, Elsevier Applied Science Publishers, New York (USA), 1986.
\item R.B. Lowrie, $\lq$Compact higher-Order Numerical Methods for Hyperbolic Conservation Laws', PhD Thesis, Department of Aerospace Engineering and Scientific Computing, University of Michigan (USA), 1996.
\end{enumerate}
%
\end{enumerate}

\item Oral Presentation:
\begin{enumerate}
%
\item Do NOT read from notes and/or screen. Look at and interact with your audience.
%
\item No Graphics used to illustrate thermodynamic or numerical concepts to a general audience.
%
\item Do not put something in slides that you are not going to explain or something that you are not sure of.
%
\item Good time keeping.
%
\item Hard paper but creditable effort.
%
\item Be more enthusiastic, try to burst with enthusiasm, if you are not, your audience will not be enthusiastic to listen to you.
%
\end{enumerate}

\end{enumerate}

\clearpage


%%%%%% 
%%%%%%
%%%%%%
\noindent{\bfseries\large EG3521 -- Engineering Thermodynamics (Continuous Assessment)  \hfill April, 2015}

\bigskip

\begin{center}
  {\Large Comments on the {\it Continuous Assessment Activities} -- Group 17}
\end{center}

\begin{enumerate}

\item Report:
\begin{enumerate}
%
\item The main aim of {\it Abstracts} is to briefly describe the work undertaken by the author. In general {\it Abstracts} are divided in 4 parts: (i) motivation, (ii) main objectives, (iii) summary of the main procedures / techniques / technologies (optional) and (iv) main findings. 
%
\item The main {\it Introduction} section usually has the same (but more in-depth and descriptive) four parts of the {\it Abstract} and a brief summary of the remaining of the work. In addition, it is \underline{always} expected a few clear statements -re main background (thus recent innovations related to the main topic), initial literature review and, most of all, technological / scientific gaps in the current understanding. Also, it is expected a summary of the remaining sections at the end of the {\it Introduction}.
%
\item Nice looking report.
%
\item Figures and tables should have a number and a caption.
%
\item Avoid all apostrophes in scientific writing - particularly greengrocer's apostrophes $\lq$lower composition's [sic]'.
%
\item Take care typesetting chemical formula.
%
\item You've squashed the plant figure and altered its aspect ratio.
%
\item It could be clearer where the paragraphs end in the $\lq$summary of the manuscript'.
%
\item Make sure you typeset subscripts in formulae correctly.
%
\item Discussion of results is good.
%
\item Critical analysis of the paper text is good.
%
\item Having 64 references in the bibliography is impressive, but you should have read all of them, which seems improbable. Each reference should be linked to the point in the text, where a fact from the referenced is used.
%
\item For every reference I would expect to see, the authors, the article title, the date, volume number, page numbers. A lot of this is missing and there is some inconsistent font usage.
%
%\item Quality of a few figures are very poor. The font size for some of them is too small and no captions are present in some of them. Numbering is also not consistent. Also, several figures are $\lq$floating' with no explanation/description in the main text.
%
\item Avoid using {\it colloquial (informal / personal)} writing.
%
\item Regardless of the chosen citation style (e.g., ACS, AIP, AMS, IEEE, AIAA, etc) any reference {\bf must} contain the following fields: 
\begin{enumerate}
\item For journal papers: Authors, Paper Tittle, Journal Name, Volume, Pages, Year of publication;
\item For books: Authors, Book Tittle, Publisher, Year or Edition;
\item For book chapters: Authors, Chapter Tittle, Book Tittle, Editors, Publisher, Year or Edition;
\item For conference papers: Authors, Paper Tittle, Conference Tittle, Place (Country and/or City) where the conference was held, Year of the conference;
\item For reports,  private communications and Lecture Notes: Authors, Tittle, Place issued (Country and/or City and Institution where the document was originated), Year;
\item For PhD Thesis and MSc Dissertations: Author, Tittle, Institution (University and Department/School), Year.
\end{enumerate}  
Thus, for example:
\begin{enumerate}[label={[\arabic*]}]
\item P.L. Houtekamer and L. Mitchell, $\lq$Data Assimilation Using an Ensemble Kalman Filter Technique', {\it Monthly Weather Review}, 126:796-811, 1998.
\item K. Pruess, $\lq$Numerical Modelling of Gas Migration at a Proposed Repository for Low and Intermediate Level Nuclear Wastes', Technical Report LBL-25413, Lawrence Berkeley Laboratory, Berkeley (USA), 1990.
\item K. Aziz, A. Settari, {\it Fundamentals of Reservoir Simulation}, Elsevier Applied Science Publishers, New York (USA), 1986.
\item R.B. Lowrie, $\lq$Compact higher-Order Numerical Methods for Hyperbolic Conservation Laws', PhD Thesis, Department of Aerospace Engineering and Scientific Computing, University of Michigan (USA), 1996.
\end{enumerate}
%
\end{enumerate}

\item Oral Presentation:
\begin{enumerate}
%
\item Do NOT read from notes and/or screen. Look at and interact with your audience.
%
\item Good background knowledge on cement production.
%
\item Neat slides but body text size must be consistent.
%
\item Cue cards are supposed to have words on them that will remind the speaker what they want to say. They are not to be read off of. This defeats their purpose.
%
\item Graphics not used approprately to illustrate technical concepts to a general audience.
%
\item Speaker and slides not always synchronizing.
%
\item Be more enthusiastic, try to burst with enthusiasm, if you are not, your audience will not be enthusiastic to listen to you.
%
\end{enumerate}

\end{enumerate}


\clearpage


%%%%%% 
%%%%%%
%%%%%%
\noindent{\bfseries\large EG3521 -- Engineering Thermodynamics (Continuous Assessment)  \hfill April, 2015}

\bigskip

\begin{center}
  {\Large Comments on the {\it Continuous Assessment Activities} -- Group 18}
\end{center}

\begin{enumerate}

\item Report:
\begin{enumerate}
%
\item The main aim of {\it Abstracts} is to briefly describe the work undertaken by the author. In general {\it Abstracts} are divided in 4 parts: (i) motivation, (ii) main objectives, (iii) summary of the main procedures / techniques / technologies (optional) and (iv) main findings. 
%
\item The main {\it Introduction} section usually has the same (but more in-depth and descriptive) four parts of the {\it Abstract} and a brief summary of the remaining of the work. In addition, it is \underline{always} expected a few clear statements -re main background (thus recent innovations related to the main topic), initial literature review and, most of all, technological / scientific gaps in the current understanding. Also, it is expected a summary of the remaining sections at the end of the {\it Introduction}.
%
\item Nice looking report.
%
\item Avoid writing in one sentence paragraphs.
%
\item There are no equations, figures, tables or graphs, which I would expect on a report on thermodynamics.
%
\item If you're going to talk about equation 2, then you should include this equation in your report.
%
\item Discussion of wider literature is good, but the number of references is limited.
%
\item Don't mix author date reference i.e. Nouri-Borujerd and Ziaei-Rad (2009) and number references [1].
%
\item Careful with units 14K not 14k.
%
\item Either abbreviate all journal names i.e. "J. Fluid Eng." or write them all out in full i.e. "International Journal of Heat and Mass Transfer", not a mixture of both.
%
%\item Quality of a few figures are very poor. The font size for some of them is too small and no captions are present in some of them. Numbering is also not consistent. Also, several figures are $\lq$floating' with no explanation/description in the main text.
%
\item Avoid using {\it colloquial (informal / personal)} writing.
%
\item Regardless of the chosen citation style (e.g., ACS, AIP, AMS, IEEE, AIAA, etc) any reference {\bf must} contain the following fields: 
\begin{enumerate}
\item For journal papers: Authors, Paper Tittle, Journal Name, Volume, Pages, Year of publication;
\item For books: Authors, Book Tittle, Publisher, Year or Edition;
\item For book chapters: Authors, Chapter Tittle, Book Tittle, Editors, Publisher, Year or Edition;
\item For conference papers: Authors, Paper Tittle, Conference Tittle, Place (Country and/or City) where the conference was held, Year of the conference;
\item For reports,  private communications and Lecture Notes: Authors, Tittle, Place issued (Country and/or City and Institution where the document was originated), Year;
\item For PhD Thesis and MSc Dissertations: Author, Tittle, Institution (University and Department/School), Year.
\end{enumerate}  
Thus, for example:
\begin{enumerate}[label={[\arabic*]}]
\item P.L. Houtekamer and L. Mitchell, $\lq$Data Assimilation Using an Ensemble Kalman Filter Technique', {\it Monthly Weather Review}, 126:796-811, 1998.
\item K. Pruess, $\lq$Numerical Modelling of Gas Migration at a Proposed Repository for Low and Intermediate Level Nuclear Wastes', Technical Report LBL-25413, Lawrence Berkeley Laboratory, Berkeley (USA), 1990.
\item K. Aziz, A. Settari, {\it Fundamentals of Reservoir Simulation}, Elsevier Applied Science Publishers, New York (USA), 1986.
\item R.B. Lowrie, $\lq$Compact higher-Order Numerical Methods for Hyperbolic Conservation Laws', PhD Thesis, Department of Aerospace Engineering and Scientific Computing, University of Michigan (USA), 1996.
\end{enumerate}
%
\end{enumerate}

\item Oral Presentation:
\begin{enumerate}
%
\item Do NOT read from notes and/or screen. Look at and interact with your audience.
%
\item Good discussion of equations.
%
\item Neat consistent slides.
%
\item Some speakers had stuttered delivery, probably due to nerves.
%
\item Graphics used appropriately to illustrate technical concepts to a general audience.
%
\item Take care when typing equations.
%
\item Be more enthusiastic, try to burst with enthusiasm, if you are not, your audience will not be enthusiastic to listen to you.
%
\end{enumerate}

\end{enumerate}


\clearpage

%%%%%%%
%%%%%%%
%%%%%%%

%\lipsum % Text before
\afterpage{%
    \clearpage% Flush earlier floats (otherwise order might not be correct)
    \thispagestyle{empty}% empty page style (?)
    \begin{landscape}% Landscape page
        \centering % Center table

%\begin{center}
\Huge{MEng Study Assessment -- Winter Report (EG4013)}\\
\huge{(Review + Feedback)}\\
\huge{January 2015}
%\end{center}
\normalsize



\bigskip

\begin{center}
\begin{tabular}{l c c }
\hline
Michael Ewen       & 17  & B1 \\
Kristoffer Ritchie & 15  & B3 \\
Giulia Marzetti    & 14  & C1 \\
David Andrew       & 16  & B2 \\
Don Stuart         & 10  & D2 \\
\hline
\end{tabular}
\end{center}


    \end{landscape}
    \clearpage% Flush page
}

%\lipsum % Text after

\vfill

\clearpage




%%%%%% 
%%%%%%
%%%%%%


\noindent{\bfseries\large EG4013 -- MEng Engineering Report (Winter Progress Report) \hfill January, 2015}

\bigskip

\begin{center}
  {\Large Review of the {\it Winter Report} $\lq$A Feasibility Study of Carbon Dioxide Capture and Storage' by Michael Ewen}
\end{center}

The manuscript investigates currently available technologies for CO$_{2}$ capture, transport and storage for mitigating GHG emissions in concentrated flow streams (i.e., carbon-based power stations). Mr Ewen undertook  literature review on current technologies for capture (adsorption, absorption within pre-, post-combustion and oxy-fuel technologies), transport (pipelines) and storage of CO$_{2}$ in geological formations.

The manuscript is relatively well-written with a small number of typos and unrevised sentences. Few sentences are confusing and disconnected with no clear objectives and inter-connectivities. Most of all, the paper is well-structured with clear division, although with diffuse linkages between sections, leading to a relatively easy and smooth reading. A few general comments,

\begin{enumerate}
%
\item The main aim of {\it Abstracts} is to briefly describe the work undertaken by the author. In general {\it Abstracts} are divided in 4 parts: (i) motivation, (ii) main objectives, (iii) summary of the main procedures / techniques / technologies (optional) and (iv) main findings. The current {\it Abstract} encompass all of them. 
%
\item The main {\it Introduction} section usually has the same (but more in-depth and descriptive) four parts of the {\it Abstract} and a brief summary of the remaining of the work. In addition, it is \underline{always} expected a few clear statements -re main background (thus recent innovations related to the main topic), initial literature review and, most of all, technological / scientific gaps in the current understanding. Also, it is expected a summary of the remaining sections at the end of the {\it Introduction}.  Current {\it Introduction} covered (in some extent) all of above but lacked explain/summarise the main state-of-the-art aspects of the subject area. 
%
\item You \underline{must} avoid use {\it colloquial (informal / personal)} writing.  
%
\item A few {\it References} follows different standards with missing fields and no clear distinction between articles, conference proceedings, reports (internal or external), book chapters, books, communications (internal or external) etc.  A few {\it references} used in the manuscript are incomplete and/or wrong. Regardless of the chosen citation style (e.g., ACS, AIP, AMS, IEEE, AIAA, etc) any reference {\bf must} contain the following fields: 
\begin{enumerate}
\item For journal papers: Authors, Paper Tittle, Journal Name, Volume, Pages, Year of publication;
\item For books: Authors, Book Tittle, Publisher, Year or Edition;
\item For book chapters: Authors, Chapter Tittle, Book Tittle, Editors, Publisher, Year or Edition;
\item For conference papers: Authors, Paper Tittle, Conference Tittle, Place (Country and/or City) where the conference was held, Year of the conference;
\item For reports,  private communications and Lecture Notes: Authors, Tittle, Place issued (Country and/or City and Institution where the document was originated), Year;
\item For PhD Thesis and MSc Dissertations: Author, Tittle, Institution (University and Department/School), Year.
\end{enumerate}  
Thus, for example:
\begin{enumerate}[label={[\arabic*]}]
\item P.L. Houtekamer and L. Mitchell, $\lq$Data Assimilation Using an Ensemble Kalman Filter Technique', {\it Monthly Weather Review}, 126:796-811, 1998.
\item K. Pruess, $\lq$Numerical Modelling of Gas Migration at a Proposed Repository for Low and Intermediate Level Nuclear Wastes', Technical Report LBL-25413, Lawrence Berkeley Laboratory, Berkeley (USA), 1990.
\item K. Aziz, A. Settari, {\it Fundamentals of Reservoir Simulation}, Elsevier Applied Science Publishers, New York (USA), 1986.
\item R.B. Lowrie, $\lq$Compact higher-Order Numerical Methods for Hyperbolic Conservation Laws', PhD Thesis, Department of Aerospace Engineering and Scientific Computing, University of Michigan (USA), 1996.
\end{enumerate} 
%
\item Quality of a few figures is poor. Also, figures and tables {\bf must} be referenced in the main text -- they can not just $\lq$float around'! In addition, figure/table captions should be self-contained, i.e., with a good description of the figure/table highlighting the most relevant aspects/information that the author wants to convene. 
% 
\item The main objectives of the Winter report are:
\begin{enumerate} 
\item Student can get familiar with:    
\begin{enumerate}
\item fundamental science and technologies of the main subject areas (through an in-depth literature review);
\item main techniques to assess/investigate the problem that will be used during the Spring term.
\end{enumerate}
\item Student can narrow the project towards his main interests. With this in mind he can plan his research activities during the Spring.
\end{enumerate}
Mr Ewen decided that his main focus during the Spring is to investigate storage technologies and the energy/exergy impact on the whole power generation process. However there is no specific plans on how this will be achieved.
% 
\end{enumerate}

The paper describes technologies related to CCS work-flow. Although there is no clear plan for activities/tasks to be undertaken during the Spring term, Mr Ewen managed to make a relatively in-depth review of the current technologies that he will use in the second part of his project.

In the attached scanned document:
\begin{itemize}
\item {\bf PE:} Poor English;
\item {\bf SC:} Sentence(s) is/are very confusing and do(es) not make much/any sense.   
\end{itemize}
\medskip


%%%%%%
%%%%%%
%%%%%%



\clearpage

\noindent{\bfseries\large EG4013 -- MEng Engineering Report (Winter Progress Report) \hfill January, 2015}

\bigskip

\begin{center}
  {\Large Review of the {\it Winter Report} $\lq$Work-Flow for Reservoir Simulation: From Mapping to Simulation' by Kristoffer Ritchie}
\end{center}

The manuscript reports current technologies used by the oil and gas industry to predict reservoir production behaviour and performance. Mr Ritchie divided the reservoir simulation workflow in nine stages, from core and well testing to history matching procedures. The primary engineering aim of the project is to review the main technologies in the reservoir simulation workflow in a (as much as it is possible) comprehensive and interconnected way. 

The manuscript is relatively well-written with a number of typos and unrevised sentences. Most of all, the paper is well-structured with clear division, although with diffuse linkages between sections, leading to a relatively smooth reading. A few general comments,

\begin{enumerate}
%
\item The main aim of {\it Abstracts} is to briefly describe the work undertaken by the author. In general {\it Abstracts} are divided in 4 parts: (i) motivation, (ii) main objectives, (iii) summary of the main procedures / techniques / technologies (optional) and (iv) main findings. The current {\it Abstract} encompass (i) and partially (ii). 
%
\item The main {\it Introduction} section usually has the same (but more in-depth and descriptive) four parts of the {\it Abstract} and a brief summary of the remaining of the work. In addition, it is \underline{always} expected a few clear statements -re main background (thus recent innovations related to the main topic), initial literature review and, most of all, technological / scientific gaps in the current understanding. Also, it is expected a summary of the remaining sections at the end of the {\it Introduction}.  Current {\it Introduction} covered only (ii) and (iii) above but lacked explain/summarise the main state-of-the-art aspects of the subject area. In fact, the {\it Introduction} section introduces and describes the main project objectives and the general workflow structure that is outlined in Sections 2-10. However, a linkage between the Introduction and the remaining of the owrk is missing.
%
\item You \underline{must} avoid use {\it colloquial (informal / personal)} writing.  
%
\item A few {\it References} follows different standards with missing fields and no clear distinction between articles, conference proceedings, reports (internal or external), book chapters, books, communications (internal or external) etc.  A few {\it references} used in the manuscript are incomplete and/or wrong. Regardless of the chosen citation style (e.g., ACS, AIP, AMS, IEEE, AIAA, etc) any reference {\bf must} contain the following fields: 
\begin{enumerate}
\item For journal papers: Authors, Paper Tittle, Journal Name, Volume, Pages, Year of publication;
\item For books: Authors, Book Tittle, Publisher, Year or Edition;
\item For book chapters: Authors, Chapter Tittle, Book Tittle, Editors, Publisher, Year or Edition;
\item For conference papers: Authors, Paper Tittle, Conference Tittle, Place (Country and/or City) where the conference was held, Year of the conference;
\item For reports,  private communications and Lecture Notes: Authors, Tittle, Place issued (Country and/or City and Institution where the document was originated), Year;
\item For PhD Thesis and MSc Dissertations: Author, Tittle, Institution (University and Department/School), Year.
\end{enumerate}  
Thus, for example:
\begin{enumerate}[label={[\arabic*]}]
\item P.L. Houtekamer and L. Mitchell, $\lq$Data Assimilation Using an Ensemble Kalman Filter Technique', {\it Monthly Weather Review}, 126:796-811, 1998.
\item K. Pruess, $\lq$Numerical Modelling of Gas Migration at a Proposed Repository for Low and Intermediate Level Nuclear Wastes', Technical Report LBL-25413, Lawrence Berkeley Laboratory, Berkeley (USA), 1990.
\item K. Aziz, A. Settari, {\it Fundamentals of Reservoir Simulation}, Elsevier Applied Science Publishers, New York (USA), 1986.
\item R.B. Lowrie, $\lq$Compact higher-Order Numerical Methods for Hyperbolic Conservation Laws', PhD Thesis, Department of Aerospace Engineering and Scientific Computing, University of Michigan (USA), 1996.
\end{enumerate} 
%
\item Quality of figures are poor. Also, figures and tables {\bf must} be referenced in the main text -- they can not just $\lq$float around'! Also, figure/table captions should be self-contained, i.e., with a good description of the figure/table highlighting the most relevant aspects/information that the author wants to convene. 
%
\item Equations \underline{must} be placed in a separated line (centered aligned with uniform font) followed by numbers (rhs aligned). All terms used must be defined afterwards as part of the main text.
% 
\item The main objectives of the Winter report are:
\begin{enumerate} 
\item Student can get familiar with:    
\begin{enumerate}
\item fundamental science and technologies of the main subject areas (through an in-depth literature review);
\item main techniques to assess/investigate the problem that will be used during the Spring term.
\end{enumerate}
\item Student can narrow the project towards his main interests. With this in mind he can plan his research activities during the Spring.
\end{enumerate}
Mr Ritchie decided that his main focus during the Spring is to investigate fluids properties and models used in industry-standard reservoir simulations. However there is no plan of how this will be achieved.
% 
\end{enumerate}

The paper describes technologies related to reservoir simulation workflow (and hydrocabon exploration). Although there is no clear plan for actiuvities/tasks to be undertaken during the Spring term, Mr Ritchie managed to make an in-depth review of the current technologies that he will use in the second part of his project.

In the attached scanned document:
\begin{itemize}
\item {\bf PE:} Poor English;
\item {\bf SC:} Sentence(s) is/are very confusing and do(es) not make much/any sense.   
\end{itemize}
\medskip


%%%%%%
%%%%%%
%%%%%%


\clearpage

\noindent{\bfseries\large EG4013 -- MEng Engineering Report (Winter Progress Report) \hfill January, 2015}

\bigskip

\begin{center}
  {\Large Review of the {\it Winter Report} $\lq$Optimisation of Energy Systems in Smart Cities' by Don Stuart}
\end{center}

The manuscript reports initial literature review and analysis of power, heating and cooling systems. An overview of the existing thermodynamic power cyle technologies relevant to (C)CHP was undertaken by Mr Stuart including: (a) organic Rankine, (b) gas-turbine Brayton and (c) vapour-compression refrigeration cycles.

The report is relatively well-written with a number of typos and unrevised sentences. Several sentences are confusing and disconnected with no clear objectives and inter-connectivities. My main concern is that there is no clear indication of actual objectives (project's and report's) in the paper, except by generic sentences in {\it Introduction} and {\it Conclusion} sections. In summary, the report outlines (in a very superficial way) a few developments on thermodynamic cycles with no clear links with the main subject areas, sustainability (a crucial concept and motivation for $\lq${\it smart cities}) and optimisation (energy integration coupled with low GHG emissions). A few general comments,
\begin{enumerate}
%
\item Dissertations and thesis are divided into chapters, reports however are \underline{always} divided into sections. 
%
\item The main aim of {\it Abstracts} is to briefly describe the work undertaken by the author. In general {\it Abstracts} are divided in 4 parts: (i) motivation, (ii) main objectives, (iii) summary of the main procedures / techniques / technologies (optional) and (iv) main findings. In this report, there is {\bf\underline{no} {\it Abstract}}. 
%
\item The main {\it Introduction} section usually has the same (but more in-depth and descriptive) four parts of the {\it Abstract} and a brief summary of the remaining of the work. In addition, it is \underline{always} expected a few clear statements -re main background (thus recent innovations related to the main topic), initial literature review and, most of all, technological / scientific gaps in the current understanding. Also, it is expected a summary of the remaining sections at the end of the {\it Introduction}.  Current {\it Introduction} covered only (i) and (ii) (in part) above but lacked explain/summarise the main state-of-the-art aspects of the subject areas. In fact, the {\it Introduction} section barely introduces the main motivation for the work -- optimal use of energy resources for sustainable cities. 
%
\item Quality of figures are very poor. Also, figures {\bf must} be referenced in the main text -- they can not just $\lq$float around'! Also, figure/table captions should be self-contained, i.e., with a good description of the figure/table highlighting the most relevant aspects/information that the author wants to convene. 
%
\item Nomenclature tables must contain the relevant units associated the main symbols.

\item You \underline{must} avoid use {\it colloquial (informal)} writing.  
%
\item The main objectives of the Winter report are:
\begin{enumerate} 
\item Student can get familiar with:    
\begin{enumerate}
\item fundamental science and technologies of the main subject areas (through an in-depth literature review);
\item main techniques to assess/investigate the problem that will be used during the Spring term.
\end{enumerate}
\item Student can narrow the project towards his main interests. With this in mind he can plan his research activities during the Spring.
\end{enumerate}
% 
\end{enumerate}

The paper is a superficial review of power and refrigeration thermodynamic cycles that can be used in combined power, heating and cooling systems. 

In the attached scanned document:
\begin{itemize}
\item {\bf PE:} Poor English;
\item {\bf SC:} Sentence(s) is/are very confusing and do(es) not make much/any sense.   
\end{itemize}
\medskip


%%%%%%
%%%%%%
%%%%%%

\clearpage

\noindent{\bfseries\large EG4013 -- MEng Engineering Report (Winter Progress Report) \hfill January, 2015}

\bigskip

\begin{center}
  {\Large Review of the {\it Winter Report} $\lq$Optimisation of Energy Systems in Smart Cities' by David Andrew}
\end{center}

The manuscript reports initial literature review and analysis of power, heating and cooling systems. A relatively in-depth overview of the existing power cyle technologies relevant to (C)CHP was undertaken by Mr Andrew including: (a) water/steam-based (standard Rankine), (b) organic Rankine, (c) gas-turbine (Brayton), (d) combined and (e) refrigeration cycles.

The report is relatively well-written with a number of typos and unrevised sentences. Several sentences are confusing and disconnected with no clear objectives and inter-connectivities. Most of all, the paper is well-structured with clear division and linkages between sections, leading to a relatively easy and smooth reading. My main concern is that there is no clear indication of actual objectives (project's and report's) neither in the {\it Abstract} nor in the {\it General Introduction} Sections. A few general comments,
\begin{enumerate}
%
\item The main aim of {\it Abstracts} is to briefly describe the work undertaken by the author. In general {\it Abstracts} are divided in 4 parts: (i) motivation, (ii) main objectives, (iii) summary of the main procedures / techniques / technologies (optional) and (iv) main findings. The current {\it Abstract} encompass only (ii), but very superficially. 
%
\item The main {\it Introduction} section usually has the same (but more in-depth and descriptive) four parts of the {\it Abstract} and a brief summary of the remaining of the work. In addition, it is \underline{always} expected a few clear statements -re main background (thus recent innovations related to the main topic), initial literature review and, most of all, technological / scientific gaps in the current understanding. Also, it is expected a summary of the remaining sections at the end of the {\it Introduction}.  Current {\it Introduction} covered only (i) above but lacked explain/summarise the main state-of-the-art aspects of the subject area. In fact, the {\it Introduction} section introduces and describes the main motivation for the work -- the dual sustainability and smart cities, however no explanation is given to the terms and the plots/figures. 
%
\item Quality of figures are very poor. Also, several figures are $\lq$floating' with no explanation/description in the main text.
%
\item The {\it References} have  a few missing fields and no clear distinction between articles, conference proceedings, reports (internal or external), book chapters, books, communications (internal or external) etc.  A few {\it references} used in the manuscript are incomplete and/or wrong. Regardless of the chosen citation style (e.g., ACS, AIP, AMS, IEEE, AIAA, etc) any reference {\bf must} contain the following fields: 
\begin{enumerate}
\item For journal papers: Authors, Paper Tittle, Journal Name, Volume, Pages, Year of publication;
\item For books: Authors, Book Tittle, Publisher, Year or Edition;
\item For book chapters: Authors, Chapter Tittle, Book Tittle, Editors, Publisher, Year or Edition;
\item For conference papers: Authors, Paper Tittle, Conference Tittle, Place (Country and/or City) where the conference was held, Year of the conference;
\item For reports,  private communications and Lecture Notes: Authors, Tittle, Place issued (Country and/or City and Institution where the document was originated), Year;
\item For PhD Thesis and MSc Dissertations: Author, Tittle, Institution (University and Department/School), Year.
\end{enumerate}  
Thus, for example:
\begin{enumerate}[label={[\arabic*]}]
\item P.L. Houtekamer and L. Mitchell, $\lq$Data Assimilation Using an Ensemble Kalman Filter Technique', {\it Monthly Weather Review}, 126:796-811, 1998.
\item K. Pruess, $\lq$Numerical Modelling of Gas Migration at a Proposed Repository for Low and Intermediate Level Nuclear Wastes', Technical Report LBL-25413, Lawrence Berkeley Laboratory, Berkeley (USA), 1990.
\item K. Aziz, A. Settari, {\it Fundamentals of Reservoir Simulation}, Elsevier Applied Science Publishers, New York (USA), 1986.
\item R.B. Lowrie, $\lq$Compact higher-Order Numerical Methods for Hyperbolic Conservation Laws', PhD Thesis, Department of Aerospace Engineering and Scientific Computing, University of Michigan (USA), 1996.
\end{enumerate}
%
\item Nomenclature tables must contain the relevant units associated the main symbols.

\item You \underline{must} avoid use {\it colloquial (informal)} writing.  
%
\item Equations \underline{must} be placed in a separated line (centered aligned with uniform font) followed by numbers (rhs aligned). All terms used must be defined afterwards as part of the main text.
% 
\end{enumerate}

The paper is a good review of power, heating and cooling systems. Although there is no clear objectives associated with either the report or the project, Mr Andrew managed to make an in-depth review of the current technologies that he will use in the second part of his project.


In the attached scanned document:
\begin{itemize}
\item {\bf PE:} Poor English;
\item {\bf SC:} Sentence(s) is/are very confusing and do(es) not make much/any sense.   
\end{itemize}
\medskip


%%%%%%
%%%%%%
%%%%%%


\clearpage

\noindent{\bfseries\large EG4013 -- MEng Engineering Report (Winter Progress Report) \hfill January, 2015}

\bigskip

\begin{center}
  {\Large Review of the {\it Winter Report} $\lq$Study of Multiscale Waterflooding Mechanisms in Heterogeneous Reservoir Simulations' by Giulia Marzetti}
\end{center}

The manuscript reports initial investigation of viscous fluid instabilities in waterflooding processes for oil and gas exploration.  A brief overview of the main topics on reservoir engineering and simulation relevant to EOR was undertaken by Ms Marzetti including, (a) main terminology, (b) EOR techniques and reservoir morphological properties, (c) heterogeneity and (d) fundamental equation for reservoir simulation.  

The report is relatively well-written with a number of typos and unrevised sentences. Several sentences are confusing and disconnected with no clear objectives and inter-connectivities. Most of all, the paper is well-structured with clear division and linkages between sections, leading to a relatively easy and smooth reading. However the numbering of a few sections are mixed up. A few general comments,
\begin{enumerate}
%
\item Dissertations and thesis are divided into chapters, reports however are always divided into sections. 
%
\item The main aim of {\it Abstracts} is to briefly describe the work undertaken by the author. In general {\it Abstracts} are divided in 4 parts: (i) motivation, (ii) main objectives, (iii) summary of the main procedures / techniques / technologies (optional) and (iv) main findings. The current {\it Abstract} encompass only (i). In fact, the whole {\it Abstract} was written a short summary of the report.
%
\item The main {\it Introduction} section usually has the same (but more in-depth and descriptive) four parts of the {\it Abstract} and a brief summary of the remaining of the work. In addition, it is \underline{always} expected a few clear statements -re main background (thus recent innovations related to the main topic), initial literature review and, most of all, technological / scientific gaps in the current understanding. Also, it is expected a summary of the remaining sections at the end of the {\it Introduction}.  Current {\it Introduction} covered (i-ii), (iv) above but lacked explain/sumarise the main state-of-the-art aspects of the subject area.
%
\item The {\it References} have  a fewh missing fields and no clear distinction between articles, conference proceedings, reports (internal or external), book chapters, books, communications (internal or external) etc.  A few {\it references} used in the manuscript are incomplete and/or wrong. Regardless of the chosen citation style (e.g., ACS, AIP, AMS, IEEE, AIAA, etc) any reference {\bf must} contain the following fields: 
\begin{enumerate}
\item For journal papers: Authors, Paper Tittle, Journal Name, Volume, Pages, Year of publication;
\item For books: Authors, Book Tittle, Publisher, Year or Edition;
\item For book chapters: Authors, Chapter Tittle, Book Tittle, Editors, Publisher, Year or Edition;
\item For conference papers: Authors, Paper Tittle, Conference Tittle, Place (Country and/or City) where the conference was held, Year of the conference;
\item For reports,  private communications and Lecture Notes: Authors, Tittle, Place issued (Country and/or City and Institution where the document was originated), Year;
\item For PhD Thesis and MSc Dissertations: Author, Tittle, Institution (University and Department/School), Year.
\end{enumerate}  
Thus, for example:
\begin{enumerate}[label={[\arabic*]}]
\item P.L. Houtekamer and L. Mitchell, $\lq$Data Assimilation Using an Ensemble Kalman Filter Technique', {\it Monthly Weather Review}, 126:796-811, 1998.
\item K. Pruess, $\lq$Numerical Modelling of Gas Migration at a Proposed Repository for Low and Intermediate Level Nuclear Wastes', Technical Report LBL-25413, Lawrence Berkeley Laboratory, Berkeley (USA), 1990.
\item K. Aziz, A. Settari, {\it Fundamentals of Reservoir Simulation}, Elsevier Applied Science Publishers, New York (USA), 1986.
\item R.B. Lowrie, $\lq$Compact higher-Order Numerical Methods for Hyperbolic Conservation Laws', PhD Thesis, Department of Aerospace Engineering and Scientific Computing, University of Michigan (USA), 1996.
\end{enumerate}
%
\item A few terms were used before being defined/explained, e.g., {\it sweep efficiency} was firstly used in page 10 whereas the definition was found in page 15. 
%
\item Equations \underline{must} be placed in a separated line (centered aligned with uniform font) followed by its number (rhs aligned). All terms used must be defined afterwards as part of the main text.
% 
\end{enumerate}

The paper is a good review of the fundamentals of reservoir engineering and simulation, but does not cover the literature review of the main subject area, fluid instabilities that leads to the fingering phenomena. In addition, as a winter report it was expected a work plan for the activities that will be undertaken during the spring (e.g., Gantt chart, list of activities with appropriate time frame work).


In the attached scanned document:
\begin{itemize}
\item {\bf PE:} Poor English;
\item {\bf SC:} Sentence(s) is/are very confusing and do(es) not make much/any sense.   
\end{itemize}
\medskip

\clearpage

%%%%%%
%%%%%%
%%%%%%

\begin{center}
\Huge{Advanced Topics for MEng Study (EG5085)}\\
\huge{1$^{st}$ and 2$^{nd}$ Papers (Review + Feedback)}\\
\huge{November 2014 --  January 2015}
\end{center}

\vfill

\clearpage

\noindent{\bfseries\large EG5085 -- Advanced Topics for MEng Study $\left(\text{1}^{\text{st}}\text{ Paper}\right)$\hfill December, 2014}

\bigskip

\begin{center}
  {\Large Review of the 2$^{\text{nd}}$ Paper $\lq$Precipitation and Thermodynamic Analysis of Asphaltenes in Crude Oils' by MarySandra Oluchi Anunobi}
\end{center}

The paper assesses thermodynamic mechanisms of asphaltene precipitation in crude heavy oils under reservoir and transport in pipeline conditions. The student investigated (a) traditional and state-of-the-art technologies (i.e., designed equations of state and formulations for heavy macro-molecules) used to predict onset of asphaltene precipitation; (b) current methods to experimentally assess precipitation / deposition; (c) mechanisms that drives precipitation, floculation and deposition of asphaltenes under prescribed conditions and; (d) current technologies to mitigate / remediate heavy hydrocarbon precipitation.

The paper is relatively well-written with a number of typos and unrevised sentences. Several sentences are confusing and disconnected with no clear objectives and inter-connectivities. Most of all, the paper is well-structured with clear division and linkages between sections and paragraphs, leading to an easy and smooth reading. However the numbering of several sections are mixed up. A few general comments,
\begin{enumerate}
%
\item The main aim of {\it Abstracts} is to briefly describe the work undertaken by the author. In general {\it Abstracts} are divided in 4 parts: (i) motivation, (ii) main objectives, (iii) summary of the main procedures / techniques / technologies (optional) and (iv) main findings. The current {\it Abstract} encompass all of them.
%
\item The main {\it Introduction} section usually has the same (but more in-depth and descriptive) four parts of the {\it Abstract} and a brief summary of the remaining of the work. In addition, it is always expected a few clear statements -re main background (thus recent innovations related to the main topic), initial literature review and, most of all, technological / scientific gaps in the current understanding. Also, it is expected a summary of the remaining sections at the end of the {\it Introduction}. 
%
\item The {\it References} follow different standards with missing fields and no clear distinction between articles, conference proceedings, reports (internal or external), book chapters, books, communications (internal or external) etc.  A few {\it references} used in the manuscript are incomplete and/or wrong. Regardless of the chosen citation style (e.g., ACS, AIP, AMS, IEEE, AIAA, etc) any reference {\bf must} contain the following fields: 
\begin{enumerate}
\item For journal papers: Authors, Paper Tittle, Journal Name, Volume, Pages, Year of publication;
\item For books: Authors, Book Tittle, Publisher, Year or Edition;
\item For book chapters: Authors, Chapter Tittle, Book Tittle, Editors, Publisher, Year or Edition;
\item For conference papers: Authors, Paper Tittle, Conference Tittle, Place (Country and/or City) where the conference was held, Year of the conference;
\item For reports,  private communications and Lecture Notes: Authors, Tittle, Place issued (Country and/or City and Institution where the document was originated), Year;
\item For PhD Thesis and MSc Dissertations: Author, Tittle, Institution (University and Department/School), Year.
\end{enumerate}  
Thus, for example:
\begin{enumerate}[label={[\arabic*]}]
\item P.L. Houtekamer and L. Mitchell, $\lq$Data Assimilation Using an Ensemble Kalman Filter Technique', {\it Monthly Weather Review}, 126:796-811, 1998.
\item K. Pruess, $\lq$Numerical Modelling of Gas Migration at a Proposed Repository for Low and Intermediate Level Nuclear Wastes', Technical Report LBL-25413, Lawrence Berkeley Laboratory, Berkeley (USA), 1990.
\item K. Aziz, A. Settari, {\it Fundamentals of Reservoir Simulation}, Elsevier Applied Science Publishers, New York (USA), 1986.
\item R.B. Lowrie, $\lq$Compact higher-Order Numerical Methods for Hyperbolic Conservation Laws', PhD Thesis, Department of Aerospace Engineering and Scientific Computing, University of Michigan (USA), 1996.
\end{enumerate}
%
\item There is a mismatch of indices in Section 3 -- thus $n_{k}^{ij}$ (Eqn. 4) is the number of moles of species $k$ in phases $i$ and $j$. This is more correctly expressed as $n_{k}^{j} \;\;\; \forall j={1,2,\cdots, \mathcal{N}_{p}}$ $\left(\text{where }\mathcal{N}_{p}\text{ is the total number of phases}\right)$. 
%
\item Terms in Eqns. 10 and 11 were not defined.
%
\item No discussion on the pros $\&$ cons on the model formulations (Sections 5 and 6). As thermodynamic formulations (and strategies) to assess the onset of precipitation (and maybe deposition) of asphaltenes are the main aim of the paper, I would expect a more in-depth discussion and analysis of (or at least one of) those formulations.    
% 
\end{enumerate}

The topic is very relevant for the energy sector and each section has been the focus of several academic- and industrial-based studies worldwide with clear cross-fertilisation with physics and chemistry (thermodynamics, kinetics, macromolecules, fluid mechanics, etc), geology $\&$ geophysics (e.g., lithography, petrology, geochemistry, etc) and computer science (e.g., software engineering, algorithms, parallel processing, etc).  The student demonstrated that she had a solid understanding of the main fundamental physics and engineering concepts involved in the technologies for this project.


%%%%%%
%%%%%%
%%%%%%

\clearpage

\noindent{\bfseries\large EG5085 -- Advanced Topics for MEng Study $\left(\text{1}^{\text{st}}\text{ Paper}\right)$\hfill December, 2014}

\bigskip

\begin{center}
  {\Large Review of the 2$^{\text{nd}}$ Paper $\lq$An Energy and Exergy Analysis of Geothermal Power Systems' by Douglas Russell}
\end{center}

The paper reviews technologies on power generation from geothermal energy sources. The paper also reports the thermodynamic analysis of a US-based geothermal power plant. The student investigated (a) current available geothermal energy system technologies and (b) energy and exergy analysis of Rankine and Kalina thermodynamic cycles.   

The paper is well-written with a small number of typos; it is well-structured with clear division and linkages between sections and paragraphs, leading to an easy and smooth reading. However the numbering of a couple of sections / figures are mixed up. A few general comments,
\begin{enumerate}
%
\item The main aim of {\it Abstracts} is to briefly describe the work undertaken by the author. In general {\it Abstracts} are divided in 4 parts: (i) motivation, (ii) main objectives, (iii) summary of the main procedures / techniques / technologies (optional) and (iv) main findings. The current {\it Abstract} encompass (i-ii) and (iv).
%
\item The main {\it Introduction} section usually has the same (but more in-depth and descriptive) four parts of the {\it Abstract} and a brief summary of the remaining of the work. In addition, it is always expected a few clear statements -re main background (thus recent innovations related to the main topic), initial literature review and, most of all, technological / scientific gaps in the current understanding. Also, it is expected a summary of the remaining sections at the end of the {\it Introduction}. In this paper, the literature review is spread over the whole paper, and the objectives are listed in a specific section (Section 2), however there is no summary of the remaining sections in the main {\it Introduction} section.   
%
\item A table containing all initial data (with references) and assumptions of the energy/exergy analysis would easier the reading and make {\it Section 4} more comprehensible. 
%
\item As a simplification of the heat and flow through porous media, Eqn. 20 states that the groundwater mass flow rate is constant (steady-state). The simplification, although acceptable for the overall heat balance calculation performed by Mr Russell, should be followed by an explanation of the real physics phenomena involving Darcy equation for the flow transport.  
% 
\end{enumerate}

The topic is very relevant for the energy sector and each section has been the focus of several academic- and industrial-based studies worldwide with clear cross-fertilisation with physics and chemistry (thermodynamics, fluid mechanics, etc), geology $\&$ geophysics (e.g., lithography, petrology, geochemistry, etc) and computer science (e.g., software engineering, algorithms, parallel processing, etc).  The student demonstrated that he had a very solid understanding of the main fundamental physics and engineering concepts involved in the technologies for this project.

\clearpage

%%%%%%%
%%%%%%%
%%%%%%%

\noindent{\bfseries\large EG5085 -- Advanced Topics for MEng Study $\left(\text{1}^{\text{st}}\text{ Paper}\right)$\hfill December, 2014}

\bigskip

\begin{center}
  {\Large Review of the 2$^{\text{nd}}$ Paper $\lq$A Review of Well Allocation in the Context of Flow Assurance and Well Management' by Lauren Crane}
\end{center}

The paper investigates multiphase flow meters (MPFM) in the O$\&$G sector and their optimal use in well allocation. In this paper, Ms Crane studied (a) flow assurance problems associated  oil, gas and water transport in pipelines and reservoirs, (b) underlying technology in MPFM and (c) current industry-standard well allocation methods.

The paper is relatively well-written with a number of typos and unrevised sentences. Several sentences are confusing and disconnected with no clear objectives and inter-connectivities. Most of all, the paper lacks a good structured with clear division and linkages between sections and paragraphs -- for example, Section 2 could be readily subdivided into 4 subsections on flow regimes, hydrate formation, asphaltene and wax deposition and operation shutdown. This would lead to an easier an smoother reading. 

My main concern is that there is no clear indication of the actual objectives neither in the {\it Abstract} nor in the {\it General Introduction} Sections. A few comments,
\begin{enumerate}
%
\item The main aim of {\it Abstracts} is to briefly describe the work undertaken by the author. In general {\it Abstracts} are divided in 4 parts: (i) motivation, (ii) main objectives, (iii) summary of the main procedures / techniques / technologies (optional) and (iv) main findings. The current {\it Abstract} encompass (i), (iii) and (iv).
%
\item The main {\it Introduction} section usually has the same (but more in-depth and descriptive) four parts of the {\it Abstract} and a brief summary of the remaining of the work. In addition, it is always expected a few clear statements -re main background (thus recent innovations related to the main topic), initial literature review and, most of all, technological / scientific gaps in the current understanding. In a {\it review paper} the later is absolutely crucial. Also, it is expected a summary of the remaining sections at the end of the {\it Introduction}. The current {\it Introduction} section, although short, is well-written, but lacks stating the objective of the work and how the remaining work will be divided.
%
\item Equations are written down throughout the paper with very little comments on them. Also, a few terms introduced by the equations were not defined, e.g., in Eqns. 4 and 5, $\Delta P$, $\rho$, $C$, $N$, $x$ and $\mu$.
%
\item A few figures are $\lq$floating' with no clear explanation/description in the main text (e.g., Fig. 4).   
% 
\end{enumerate}

The topic is very relevant for the energy sector and each section has been the focus of several academic- and industrial-based studies worldwide with clear cross-fertilisation with physics and chemistry (thermodynamics, kinetics, macromolecules, fluid mechanics, etc), geology $\&$ geophysics (e.g., lithography, petrology, geochemistry, etc) and computer science (e.g., software engineering, algorithms, parallel processing, etc).  The student demonstrated that she had a good understanding of the main fundamental physics and engineering concepts involved in the technologies for this project.

\clearpage
%%%%%%%
%%%%%%%
%%%%%%%

\noindent{\bfseries\large EG5085 -- Advanced Topics for MEng Study $\left(\text{1}^{\text{st}}\text{ Paper}\right)$\hfill December, 2014}

\bigskip

\begin{center}
  {\Large Review of the 2$^{\text{nd}}$ Paper $\lq$An Investigation into Impact of Retro-Fitting a Carbon Capture Plant to a Gas-Turbine Power Generation System' by Calum Murdoch}
\end{center}

The paper reports the design project of an integrated carbon-capture plant and dual power generation (Brayton and Rankine cycles) systems. The design project aims to investigate the energy feasibility of well-established technologies of gas and steam power cycles integrated with amine-based absorption column for CO$_{2}$ capture.  In this paper, Mr Murdoch studied (a) process engineering design and optimisation, (b) CO$_{2}$-capture technologies and (c) power generation processes.

The paper is well-written with a small number of typos and unrevised sentences. A few sentences are confusing and disconnected with no clear objectives and inter-connectivities. Overall, the paper is well-structured with clear division and linkages between sections and paragraphs, leading to an easy and smooth reading. My main concern is that there is no clear indication of the actual objectives neither in the {\it Abstract} nor in the {\it General Introduction} Sections. Indeed, the only part where the objective of the paper is in the first paragraph of Section 2.1 -- $\lq$... the overall objective for this design project is to maximise the export power whilst meeting all design constraints'.  A few comments,

\begin{enumerate}
%
\item The main aim of {\it Abstracts} is to briefly describe the work undertaken by the author. In general {\it Abstracts} are divided in 4 parts: (i) motivation, (ii) main objectives, (iii) summary of the main procedures / techniques / technologies (optional) and (iv) main findings. The current {\it Abstract} encompass (i), (iii) and (iv).
%
\item The main {\it Introduction} section usually has the same (but more in-depth and descriptive) four parts of the {\it Abstract} and a brief summary of the remaining of the work. In addition, it is always expected a few clear statements -re main background (thus recent innovations related to the main topic), initial literature review and, most of all, technological / scientific gaps in the current understanding. Also, it is expected a summary of the remaining sections at the end of the {\it Introduction}. The current {\it Introduction} section is well-written, but lacks stating the objective of the work and how the remaining work will be divided. 
%
\item A few terms are not defined (e.g., K$_{4}$ in Section 3.3) or have double-definition (e.g., $m$ in Eqns. 3.2.15 and 3.3.6).
% 
\end{enumerate}

The topic is very relevant for the energy sector and each section has been the focus of several academic- and industrial-based studies worldwide with clear cross-fertilisation with physics and chemistry (thermodynamics, kinetics, fluid mechanics, heat and mass transfers, etc), engineering (process engineering, design, optimisation, etc) and computer science (e.g., software engineering, algorithms, parallel processing, etc).  The student demonstrated that she had a good understanding of the main fundamental physics and engineering concepts involved in the technologies for this project.



\clearpage
%%%%%%%
%%%%%%%
%%%%%%%

\noindent{\bfseries\large EG5085 -- Advanced Topics for MEng Study $\left(\text{1}^{\text{st}}\text{ Paper}\right)$\hfill December, 2014}

\bigskip

\begin{center}
  {\Large Review of the 2$^{\text{nd}}$ Paper $\lq$Advanced Reservoir Management: History Matching Workflow' by Steven Miskelly}
\end{center}

The paper reviews current computational / numerical methods used by O$\&$G industry for history matching studies in reservoirs.  In this paper, Mr Miskelly studied: (a) all main stages of the O$\&$G exploration relevant to reservoir simulations, (b) uncertainties analysis, (c) optimisation methods and (d) fluid flow in porous media.

The paper is well-written with a small number of typos and unrevised sentences. A few sentences are confusing and disconnected with no clear objectives and inter-connectivities. Overall, the paper is well-structured with clear division and linkages between sections and paragraphs, leading to an easy and smooth reading. A few comments,

\begin{enumerate}
%
\item The main aim of {\it Abstracts} is to briefly describe the work undertaken by the author. In general {\it Abstracts} are divided in 4 parts: (i) motivation, (ii) main objectives, (iii) summary of the main procedures / techniques / technologies (optional) and (iv) main findings. The current {\it Abstract} encompass (i), (iii) and (iv).
%
\item The main {\it Introduction} section usually has the same (but more in-depth and descriptive) four parts of the {\it Abstract} and a brief summary of the remaining of the work. In addition, it is always expected a few clear statements -re main background (thus recent innovations related to the main topic), initial literature review and, most of all, technological / scientific gaps in the current understanding.  In a {\it review paper} the later is absolutely crucial. Also, it is expected a summary of the remaining sections at the end of the {\it Introduction}. The current {\it Introduction} section, although short, is well-written, but lacks a short summary of the remaining of the work.
%
\item The {\it References} follow different standards with missing fields and no clear distinction between articles, conference proceedings, reports (internal or external), book chapters, books, communications (internal or external) etc.  A few {\it references} used in the manuscript are incomplete and/or wrong. Regardless of the chosen citation style (e.g., ACS, AIP, AMS, IEEE, AIAA, etc) any reference {\bf must} contain the following fields: 
\begin{enumerate}
\item For journal papers: Authors, Paper Tittle, Journal Name, Volume, Pages, Year of publication;
\item For books: Authors, Book Tittle, Publisher, Year or Edition;
\item For book chapters: Authors, Chapter Tittle, Book Tittle, Editors, Publisher, Year or Edition;
\item For conference papers: Authors, Paper Tittle, Conference Tittle, Place (Country and/or City) where the conference was held, Year of the conference;
\item For reports,  private communications and Lecture Notes: Authors, Tittle, Place issued (Country and/or City and Institution where the document was originated), Year;
\item For PhD Thesis and MSc Dissertations: Author, Tittle, Institution (University and Department/School), Year.
\end{enumerate}  
Thus, for example:
\begin{enumerate}[label={[\arabic*]}]
\item P.L. Houtekamer and L. Mitchell, $\lq$Data Assimilation Using an Ensemble Kalman Filter Technique', {\it Monthly Weather Review}, 126:796-811, 1998.
\item K. Pruess, $\lq$Numerical Modelling of Gas Migration at a Proposed Repository for Low and Intermediate Level Nuclear Wastes', Technical Report LBL-25413, Lawrence Berkeley Laboratory, Berkeley (USA), 1990.
\item K. Aziz, A. Settari, {\it Fundamentals of Reservoir Simulation}, Elsevier Applied Science Publishers, New York (USA), 1986.
\item R.B. Lowrie, $\lq$Compact higher-Order Numerical Methods for Hyperbolic Conservation Laws', PhD Thesis, Department of Aerospace Engineering and Scientific Computing, University of Michigan (USA), 1996.
\end{enumerate}
%
\item Most of the figures used in the paper were obtained from journal articles and books with low resolution (quality).
%
\item A few symbols / acronyms were used before being defined, e.g., GOR, WOR (gas-oil and water-oil ratios, Secion 2.7) and $Y_{i}$ (Section 3.1) is a vector of observed and simulates values.
\end{enumerate}

The topic is very relevant for the energy and environmental science sectors and each section has been the focus of several academic- and industrial-based studies worldwide with clear cross-fertilisation with physics (thermodynamics, fluid mechanics, cloud physics, shock-physics etc), geology $\&$ geophysics (e.g., lithography, petrology, geochemistry, etc) and computer science (e.g., software engineering, algorithms, parallel processing, etc). The student demonstrated that he had a solid understanding of the main fundamental physics and engineering concepts involved in the technologies for this project.



%%%%%%%%
%%%%%%%%
%%%%%%%%

\clearpage
\noindent{\bfseries\large EG5085 -- Advanced Topics for MEng Study $\left(\text{1}^{\text{st}}\text{ Paper}\right)$\hfill November, 2014}

\bigskip

\begin{center}
  {\Large Review of the 1$^{\text{st}}$ Paper $\lq$Development and Performance Analysis of the Stochastic Torus Algorithm for Global Optimisation Problems' by Michael Drew}
\end{center}
The paper aims to assess and compare the performance of 3 global optimisation algorithms -- torus, simulated annealing and cuckoo search. Global optimisation methods are widely used in industrial engineering applications. Mr Draw studied two main subject areas within the main topic (torus algorithm): stochastic and metaheuristic methods.

The paper is reasonably well-written with a small number of typos and unrevised sentences. A few sentences are very confusing and disconnected with no clear objectives. In general, the paper is well-structured with clear division and linkages between sections and paragraphs, leading to an easy and smooth reading. The paper, although focused on $\lq$generic' (and rather superficial) description of the three optimisation methods was interesting to read with good content to enable discussion and analysis. A few extra comments:
\begin{enumerate}
\item The main aim of {\it Abstracts} is to briefly describe the work undertaken by the author. In general {\it Abstracts} are divided in 4 parts: (i) motivation, (ii) main objectives, (iii) summary of the main procedures / techniques / technologies (optional) and (iv) main findings. The current {\it Abstract} encompasses (i-ii) and (iv).
%
\item The {\it Introduction} section usually has the same (but more in-depth and descriptive) four parts of the {\it Abstract} and a brief summary of the remaining of the work. In addition, it is always expected a few clear statements -re main background (thus recent innovations related to the main topic), initial literature review and, most of all, technological / scientific gaps in the current understanding. In this paper, the {\it Introduction} section focused mostly on the motivation and application of optimisation algorithms with very little attention to fundamentals underlying these algorithms and novelties (e.g., adaptive/optimal search in SA and CS). 
%
\item A few {\it References} are missing important fields, e.g., page numbers and volumes. Also, there are different standards -re the authors' names/initials.
%
\item Literature review was the focus of {\it Section 2}. The student divided the section into three parts, each of them with brief (and slightly superficial) description of the algorithms with limited critical analysis of the work undertaken by authors. 
%
\item The student outlined the generic algorithm (block diagram) along with the pseudo-code of the three methods. These helped reading and understanding the differences between the algorithms. However, the $\lq$data input deck' of each algorithm is missing -- this is crucial to understand why a method is more used than the others. For example, whereas the CS needs no more than 4-5 parameters, the cooling schedule of the SA requires more than a dozen parameters (similar to the TA).   
%
\item The discussion and conclusions sections, although interesting and insightful, were very superficial and lacked analysis and data (that could be gathered from other publications). 
\end{enumerate}



In the attached scanned document:
\begin{itemize}
\item {\bf PE:} Poor English;
\item {\bf SC:} Sentence(s) is/are very confusing and do(es) not make much/any sense.   
\end{itemize}
\medskip


\clearpage


\noindent{\bfseries\large EG5085 -- Advanced Topics for MEng Study $\left(\text{1}^{\text{st}}\text{ Paper}\right)$\hfill November, 2014}

\bigskip

\begin{center}
  {\Large Review of the 1$^{\text{st}}$ Paper $\lq$Near-Well Upscaling for Waterflooding' by Blair Ward}
\end{center}
The main objective of the paper is to investigate methods and technologies current performed by industry to upscale hydrocarbon producer fields. Upscaling can be considered as a crucial stage on hydrocarbon exploration, and as such has attracted attention from industry and academia towards novel methods that can help to statistically represent geological oil and gas reservoirs. Mr Ward studied two main subject areas within the main topic (upscale methods): geological properties of reservoirs, fluid flow in porous media.

The paper is reasonably well-written with a small number of typos and unrevised sentences. A few sentences are very confusing and disconnected with no clear objectives. In general, the paper is well-structured with clear division and linkages between sections and paragraphs, leading to an easy and smooth reading. The paper attempts to briefly describe the main techniques for field upscaling, sensitivity of the methods and parameters that are often $\lq$averaged'. This led to an interesting to read with good content to enable discussion and analysis. A few extra comments:
\begin{enumerate}
\item The main aim of {\it Abstracts} is to briefly describe the work undertaken by the author. In general {\it Abstracts} are divided in 4 parts: (i) motivation, (ii) main objectives, (iii) summary of the main procedures / techniques / technologies (optional) and (iv) main findings. The current {\it Abstract} encompasses (i-ii) and (iv).
%
\item The {\it Introduction} section usually has the same (but more in-depth and descriptive) four parts of the {\it Abstract} and a brief summary of the remaining of the work. In addition, it is always expected a few clear statements -re main background (thus recent innovations related to the main topic), initial literature review and, most of all, technological / scientific gaps in the current understanding. In this paper, the {\it Introduction} section focused mostly on the motivation and main background (in a very superficial way) for upscaling techniques.
%
\item Equations are written down throughout the paper with very little comments on them (see comments on Eqns. 1-3). Also, in Eqns 6 and 7, there are no commas between $\phi$ and $\rho$, and between $\rho$ and ${\bf u}$, as these are just normal multiplication between scalars and between scalars and vectors. $\left(\rho,{\bf u}\right)$ and $\left(\phi,\rho\right)$ represent $\displaystyle\int\rho d{\bf u}$ and $\displaystyle\int \phi d\rho$, respectively, which is not a correctly description of the continuity and pressure equations.  Same comment applies to Eqn. 12.
%
\item The purpose of the paper is to discuss and assess upscaling methods in near-well regions, however there is nearly no mention about how the methods used in Sections 2 and 3 would be applied to this particular region. Section 3 focus on different treatment for boundary conditions but with no (explicit) comment on how this is applied to near-well region.
%
\item In Section 5 the upscaling procedures for either permeability and porosity were (very) briefly summarised and these were supposed to be one of the main objectives of the paper.  
%
\item Sections 6 and 7 (conclusions), although interesting and insightful, were very superficial and lacked analysis and data (e.g., how these techniques have been used in industry?) that could be gathered from other publications. 
\end{enumerate}

In the attached scanned document:
\begin{itemize}
\item {\bf PE:} Poor English;
\item {\bf SC:} Sentence(s) is/are very confusing and do(es) not make much/any sense.   
\end{itemize}
%\medskip

\clearpage


\noindent{\bfseries\large EG5085 -- Advanced Topics for MEng Study $\left(\text{1}^{\text{st}}\text{ Paper}\right)$\hfill November, 2014}

\bigskip

\begin{center}
  {\Large Review of the 1$^{\text{st}}$ Paper $\lq$Assessment of In-Situ Adaptive Tabulation Algorithms for Modelling and Simulation of Reactive Flow Problems' by Eamon Baker}
\end{center}
The paper summarises past and current development on computational methods to solve reactive flows. The problem investigated by the paper can be defined into two parts, solving systems of ordinary differential equations that represent chemical kinetics and {\it in-situ adaptive tabulation} (ISAT) method for fast coupling between kinetics and transport equations. Mr Baker studied two main subject areas within the main topic (ISAT methods): reaction mechanisms and numerical time-integration.

The paper is reasonably well-written with a small number of typos and unrevised sentences. A few sentences are very confusing and disconnected with no clear objectives. In general, the paper is well-structured with clear division and linkages between sections and paragraphs, leading to an easy and smooth reading. From the fundamental mass conservation of species equation (partial differential equation), the paper summarises the procedures to solve this equation in time (time-splitting methods) and space, and how to save computational (CPU) time by storing solutions, boundary and initial conditions into a k-ary tree. This led to an interesting paper to read with good content to enable discussion and analysis. A few extra comments:
\begin{enumerate}
\item The main aim of {\it Abstracts} is to briefly describe the work undertaken by the author. In general {\it Abstracts} are divided in 4 parts: (i) motivation, (ii) main objectives, (iii) summary of the main procedures / techniques / technologies (optional) and (iv) main findings. The current {\it Abstract} encompasses only (i-ii).
%
\item The {\it Introduction} section usually has the same (but more in-depth and descriptive) four parts of the {\it Abstract} and a brief summary of the remaining of the work. In addition, it is always expected a few clear statements -re main background (thus recent innovations related to the main topic), initial literature review and, most of all, technological / scientific gaps in the current understanding. In this paper, the {\it Introduction} section was very poor and did not cover any of the main points above.
%
\item The {\it Problem Statement} section was very interesting to read and provided a useful link to the remaining of the paper. However, there are very few references (or actually just one) to support the procedure described in this section.
%
\item A few symbols used in Fig. 2 were not defined anywhere.
%
\item The search space explanation would have been clear with the use of diagrams showing the rotating and  shrinking-expanded hyper-ellipsoid.
%
\item Acronyms are used before definition.
%
\item Although ISAT and the main solver method within it were well summarised, there was a very limited analysis on the current state-of-the art methods and procedures for general applications.  
%
\item Sections 6 (conclusions) was very superficial and lacked analysis, data (e.g., how these techniques have been used in industry?) and recommendations that could be gathered from other publications. 
\end{enumerate}

In the attached scanned document:
\begin{itemize}
\item {\bf PE:} Poor English;
\item {\bf SC:} Sentence(s) is/are very confusing and do(es) not make much/any sense.   
\end{itemize}
%\medskip

\clearpage


%%%%%%%
%%%%%%%
%%%%%%%

%\lipsum % Text before
\afterpage{%
    \clearpage% Flush earlier floats (otherwise order might not be correct)
    \thispagestyle{empty}% empty page style (?)
    \begin{landscape}% Landscape page
        \centering % Center table

%\begin{center}
\Huge{MSc Oil and Gas Engineering Dissertations (EG5908)}\\
\huge{(Review + Feedback)}\\
\huge{September 2014}
%\end{center}
\normalsize



\bigskip

%\begin{center}
\begin{tabular}{||c| c c c |c| c c||}
\hline\hline
                           & {\bf Presentation and Style of} & {\bf Technical Content and}           & {\bf Evidence of Critical} & {\bf Average} & {\bf Averaged} & {\bf Oral}\\
                           & {\bf Writing (20$\%$)}          & {\bf Merit of Dissertation (50$\%$)}  & {\bf Reasoning (30$\%$)}   &               & {\bf CAS Mark} & {\bf Presentation} \\
\hline
Efthymios Efthymiou        &         80                      &           80                          &          80                &    80.00      &    18          & 16              \\
Alexandros Pilichos        &         73                      &           68                          &          73                &    70.50      &    16          & 14              \\
Reshma Koshy               &         78                      &           77                          &          70                &    75.10      &    15          & 17              \\
Adarsh Gopinadham          &         69                      &           68                          &          62                &    66.40      &    14          & 16              \\
Georgios Neofytidis        &         83                      &           86                          &          80                &    83.60      &    17          & 14              \\
Pat Limpuangthip           &         82                      &           86                          &          84                &    84.60      &    18          & 18              \\
Alberto Diez Rojo          &         83                      &           85                          &          82                &    83.70      &    18          & 14              \\
Srikanth Ramani            &         71                      &           69                          &          67                &    68.80      &    13          & 15              \\
\hline\hline
\end{tabular}
%\end{center}
    \end{landscape}
    \clearpage% Flush page
}

%\lipsum % Text after



\vfill

\clearpage

%%%%%
%%%%%
%%%%%

\noindent{\bfseries\large MSc in Oil $\&$ Gas Engineering\hfill September, 2014}

\bigskip

\begin{center}
{\Large Review of the MSc Dissertation $\lq$Coupling Geomechanical and Fluid Flow Modelling: Effect of Reservoir Parameters' by Georgios Neofytidis}
\end{center}

\medskip

This dissertation assesses current methods for fluid and solid mechanics for hydrocarbon reservoir applications. The manuscript describes with a good level of details/depth computational methods used in the current- and next-generation of geomechanical models. In addition, the student also investigated coupling methods for fluid and solid models that have been studied (mostly by academics) focusing on 1- and 2-ways interactions. He also assessed the impact of Young modulus and Poisson ratio in compaction phenomena in a flow/subduction simulation using synthetic data.

The manuscript is relatively well-written with a small number of typos and unrevised sentences. Few sentences are very confusing and disconnected with no clear objectives. In general, the dissertation is well-structured with clear division and linkages between chapters, sections and paragraphs, leading to an easy and smooth reading. The dissertation, although heavily focused on mathematical description of solid mechanics (stress/strain-based PDEs) was interesting to read with good content to enable discussion and analysis. A few extra comments:
\begin{enumerate}
\item The main aim of {\it Abstracts} is to briefly describe the work undertaken by the author. In general {\it Abstracts} are divided in 4 parts: (i) motivation, (ii) main objectives, (iii) summary of the main procedures / techniques / technologies (optional) and (iv) main findings. The current {\it Abstract} encompasses all of them.
%
\item The main {\it Introduction} section usually has the same (but more in-depth and descriptive) four parts of the {\it Abstract} and a brief summary of the remaining of the work. In addition, it is always expected a few clear statements -re main background (thus recent innovations related to the main topic), initial literature review and, most of all, technological / scientific gaps in the current understanding. In this dissertation, the {\it Introduction} section focused only on the motivation and objectives of the work. Literature review is spread over the remaining chapters with an in-depth critical analysis of the work undertaken by several authors. 
%
\item A few {\it References} are missing important fields, e.g., indication of the nature of the publication (e.g., reports, thesis, manuals etc), page numbers and volumes. Also, references for the commercial software used is also missing.
%
\item  Calculations and initial model development (initial and boundary conditions) undertaken for the simulation were clearly indicated with a good level of detail. However, an in-depth discussion of the methods used in the simulations (along with assumptions) is missing. 
% 
\end{enumerate}

The topic is very relevant for the O$\&$GE (and energy) sector, and each sub-topic has been the focus of several academic- and industrial-based studies worldwide with clear cross-fertilisation with physics (fluid mechanics, solid mechanics, material science etc), geology $\&$ geophysics (e.g., lithography, petrology, geochemistry, etc) and petroleum / mechanical / civil engineering (well completion and design, geomechanics etc). The student demonstrated that he had a excellent understanding of the main technologies involved in this project.


\clearpage

%%%%%
%%%%%
%%%%%

\noindent{\bfseries\large MSc in Oil $\&$ Gas Engineering\hfill September, 2014}

\bigskip

\begin{center}
{\Large Review of the MSc Dissertation $\lq$Pressure and Heat Transfer Control for Extraction of Coal without Mining' by Alberto Diez Rojo}
\end{center}

\medskip

This dissertation investigates a prototype process to extract CH$_{4}$ from coal bed mines. The manuscript describes existing technologies used in the energy sector (oil $\&$ gas, coal mining, chemical processing) that can be readily redesigned and deployed to underground coal devolatilisation. The student undertook extensive literature review on coal and well technologies, fluid flows in porous media and heat transfer. He also developed a simplified model for the initial design and assessment of the process. 

The manuscript is well-written with a few number of typos and unrevised sentences. In general, the dissertation is well-structured with clear division and linkages between chapters, sections and paragraphs, leading to an easy and smooth reading. A few general comments,
\begin{enumerate}
\item The main aim of {\it Abstracts} is to briefly describe the work undertaken by the author. In general {\it Abstracts} are divided in 4 parts: (i) motivation, (ii) main objectives, (iii) summary of the main procedures / techniques / technologies (optional) and (iv) main findings. The current {\it Abstract} encompasses all of them.
%
\item The main {\it Introduction} section usually has the same (but more in-depth and descriptive) four parts of the {\it Abstract} and a brief summary of the remaining of the work. In addition, it is always expected a few clear statements -re main background (thus recent innovations related to the main topic), initial literature review and, most of all, technological / scientific gaps in the current understanding. In this dissertation, the {\it Introduction} section focused only on the motivation and objectives of the work. Literature review is spread over the remaining chapters with an in-depth critical analysis of the work undertaken by several authors. 
%
\item A few {\it References} are missing important fields, e.g., indication of the nature of the publication (e.g., reports, thesis, manuals etc), page numbers and volumes. Also, references for the commercial software used is also missing.
%
\item Calculations developed for the project were clearly indicated with a good level of detail. However, an in-depth discussion of the main alternatives for the large number of (initial) assumptions is missing. 
% 
\end{enumerate}

The topic is very relevant for the O$\&$GE and coal (and energy) sectors, and each sub-topic has been the focus of several academic- and industrial-based studies worldwide with clear cross-fertilisation with physics (thermodynamics, fluid mechanics, surface chemistry, material science etc), geology $\&$ geophysics (e.g., lithography, petrology, geochemistry, etc) and mining / petroleum / chemical / mechanical / processing engineering chemical (well completion and design, kinetics engineering, etc). The student demonstrated that he had a excellent understanding of the main technologies involved in this project.


\bigskip
\begin{flushleft}
{\bf Comments from the Industrial Supervisor}
\end{flushleft}
\medskip

The project is part of a series of short projects looking at the feasibility of developing a technique for the devolatilisation of coal while in situ by using controlled heat. The techniques are based on the use of oilfield technologies combined with the some underground coal mining aspects. 

The student was set the task of studying the well system from a thermodynamics point of view so that basic energy balance could be derived. 
 
His approach was to diligently explore the literature to develop his understanding of the issues.  I was impressed with his ability to seek out experts where necessary so that he could short cut his studies.  In some places he went beyond what was required seeking out novel ways of conserving heat energy which demonstrated that there many areas that could improve the technique as presently conceived.

The model that the student has developed is an excellent start to a complex issue and there will be errors in the assumptions made because of a lack of experimental data available.  The thesis flows well although there are a few minor omissions in referencing. 



\clearpage

%%%%%
%%%%%
%%%%%

\noindent{\bfseries\large MSc in Oil $\&$ Gas Engineering\hfill September, 2014}

\bigskip

\begin{center}
{\Large Review of the MSc Dissertation $\lq$Study of Multiscale Waterflooding Mechanisms in Heterogeneous Reservoir Simulations' by Pat Limpuangthip}
\end{center}

\medskip

This dissertation investigates technologies to improve hydrocarbon production in heterogeneous porous media. The manuscript focuses on methods developed over the past 50 years to: (a) assess and improve sweep efficiency; (b) investigate mechanisms of fluids displacement in oil reservoirs. The student undertook extensive literature review on multiphase flows in porous media, reservoir simulation and fluid instability mechanisms.

The manuscript is well-written but with a few number of typos and unrevised sentences. In general, the dissertation is well-structured with clear division and linkages between chapters, sections and paragraphs, leading to an easy and smooth reading. A few general comments,

\begin{enumerate}
\item The main aim of {\it Abstracts} is to briefly describe the work undertaken by the author. In general {\it Abstracts} are divided in 4 parts: (i) motivation, (ii) main objectives, (iii) summary of the main procedures / techniques / technologies (optional) and (iv) main findings. The current {\it Abstract} encompasses all of them.
%
\item The main {\it Introduction} section usually has the same (but more in-depth and descriptive) four parts of the {\it Abstract} and a brief summary of the remaining of the work. In addition, it is always expected a few clear statements -re main background (thus recent innovations related to the main topic), initial literature review and, most of all, technological / scientific gaps in the current understanding. In this dissertation, the {\it Introduction} section focused only on the motivation and objectives of the work. Literature review is spread over the remaining chapters with an in-depth critical analysis of the work undertaken by several authors. 
%
\item A few {\it References} are missing important fields, e.g., page numbers and volumes.
%
\item Chapter 2 -- {\it Literature Review}, is rather long and convoluted. It would have improved the reading experience if this chapter was split in a more rational way.
% 
\end{enumerate}

The topic is very relevant for the O$\&$GE (and energy) sector, and each sub-topic has been the focus of several academic- and industrial-based studies worldwide with clear cross-fertilisation with physics (thermodynamics, fluid mechanics, surface chemistry, etc), geology $\&$ geophysics (e.g., lithography, petrology, geochemistry, etc) and petroleum / chemical engineering (EOR, CCS, water production, etc). The student demonstrated that he had a excellent understanding of the main technologies for this project.


\clearpage

%%%%%
%%%%%
%%%%%

\noindent{\bfseries\large MSc in Oil $\&$ Gas Engineering\hfill September, 2014}

\bigskip

\begin{center}
{\Large Review of the MSc Dissertation $\lq$Exploring the Potential for CO$_{2}$ Sequestration using Char obtained from Low Temperature Pyrolysis' by Adarsh Gopinadhan}
\end{center}

\medskip

This dissertation investigates char processing and its potential use for carbon dioxide sequestration in coalmines. The manuscript focused on twofold description, currently available technologies for carbon capture, transport, sequestration and storage and laboratory methods to investigate the potential adsorption of CO$_{2}$ by char. The student undertook extensive literature review on coal technology, CCS and adsorption processes. 

The manuscript is relatively well-written with a small number of typos and unrevised sentences. Few sentences are very confusing and disconnected with no clear objectives. In general, the dissertation was interesting to read with good content to enable discussion and analysis. However, the analysis of the main topics -- low-temperature pyrolysis, coal characterisation and surface chemistry (adsorption mechanisms),  was very superficial. A few extra comments:
\begin{enumerate}
\item The main aim of {\it Abstracts} is to briefly describe the work undertaken by the author. In general {\it Abstracts} are divided in 4 parts: (i) motivation, (ii) main objectives, (iii) summary of the main procedures / techniques / technologies (optional) and (iv) main findings. The current {\it Abstract} encompasses all of them.
%
\item The main {\it Introduction} section usually has the same (but more in-depth and descriptive) four parts of the {\it Abstract} and a brief summary of the remaining of the work. In addition, it is always expected a few clear statements -re main background (thus recent innovations related to the main topic), initial literature review and, most of all, technological / scientific gaps in the current understanding. In this dissertation, the {\it Introduction} section focused only on the motivation and objectives of the work. Literature review is spread over the remaining chapters with limited critical analysis of the work undertaken by several authors. 
%
\item The manuscript was very confusing with no real focus on the goals described in Section 1.8 ({\it Aims and Objectives}) -- must of all, the level of analysis of the main technical subjects was extremely superficial. 
% 
\end{enumerate}

The topic is very relevant for the O$\&$GE (and energy) and environmental science sectors, and each chapter has been the focus of several academic- and industrial-based studies worldwide with clear cross-fertilisation with physics (thermodynamics, fluid mechanics, surface chemistry, chemical kinetics etc), geology $\&$ geophysics (e.g., lithography, petrology, geochemistry, etc) and environmental sciences (environmental chemistry, geosciences, geo-monitoring etc). The student demonstrated that he had a good understanding of the main fundamental physics and technologies for this project.


\medskip
\begin{flushleft}
{\bf Comments from the Industrial Supervisor}
\end{flushleft}
\medskip 

The project is part of a series of short projects looking at the feasibility of developing a technique for the devolatilisation of coal while in situ by using controlled heat. The techniques are based on the use of oilfield technologies combined with the some underground coal mining aspects.  It also draws on nearly 200 years of knowledge about pyrolysis of coal. Ultimately the aim is to leave a high carbon residue which has been identified as a potential repository for CO2 for carbon sequestration.

The student was set the task of studying the possibility of using the residue after the devolatilisation process has been completed called char as a repository for CO2. The student quickly identified that the char could be useful in this respect but that there was a lack of data from char produced in a similar way to the slow low temperature pyrolysis being proposed.  It was suggested to the student that he should look at how this lack of data could be rectified.

His approach was to diligently explore the literature to develop his understanding of the issues.  I was impressed with his ability to seek out experts where necessary so that he could short cut his studies.  In some places he went beyond what was required seeking out novel ways of conserving heat energy which demonstrated that there many areas that could improve the technique as presently conceived.

The model that the student has developed is an excellent start to a complex issue and there will be errors in the assumptions made because of a lack of experimental data available. The thesis flows well although there are a few minor omissions in referencing. 


\clearpage

%%%%%
%%%%%
%%%%%

\noindent{\bfseries\large MSc in Oil $\&$ Gas Engineering\hfill September, 2014}

\bigskip

\begin{center}
{\Large Review of the MSc Dissertation $\lq$Assessment of History Matching Techniques in Reservoir Management' by Reshma Koshy}
\end{center}

\medskip

This dissertation reviews current history matching methods and technologies for reservoir management. In particular, the manuscript focuses on self-adaptive computational methods for data assimilation as part of the history matching procedure undertaken for model and software quality assurance and prediction/forecasting of reservoir production. The student undertook extensive literature review on current technologies for reservoir modelling, data assimilation and model optimisation. 

The manuscript is relatively well-written with a relatively number of typos and unrevised sentences. A few sentences are very confusing and disconnected with no clear objectives. In general, the dissertation was interesting to read with good content to enable discussion and analysis. A few general comments,
\begin{enumerate}
\item The main aim of {\it Abstracts} is to briefly describe the work undertaken by the author. In general {\it Abstracts} are divided in 4 parts: (i) motivation, (ii) main objectives, (iii) summary of the main procedures / techniques / technologies (optional) and (iv) main findings. The current {\it Abstract} encompasses (i-ii) and (iv).
%
\item The main {\it Introduction} section usually has the same (but more in-depth and descriptive) four parts of the {\it Abstract} and a brief summary of the remaining of the work. In addition, it is always expected a few clear statements -re main background (thus recent innovations related to the main topic), initial literature review and, most of all, technological / scientific gaps in the current understanding. In this dissertation, the {\it Introduction} section focused only on the motivation and objectives of the work. Literature review is spread over the remaining chapters with limited critical analysis of the work undertaken by several authors. 
%
\item A few {\it References} are missing important fields, e.g., page numbers and volumes. Also, there are different standards -re the authors' names/initials.
%
\item A few terms introduced by the equations were not defined. 
%
\item The manuscript undertakes a good review of the current state-of-the-art methods for history matching and, in particular, for data assimilation. However, there was no attempt to assess/investigate current industry-standard software for history matching, which would help in the discussion and applications.  
% 
\end{enumerate}

The topic is very relevant for the O$\&$GE (and energy) and environmental science sectors, and each chapter has been the focus of several academic- and industrial-based studies worldwide with clear cross-fertilisation with physics (thermodynamics, fluid mechanics, cloud physics, shock-physics etc), geology $\&$ geophysics (e.g., lithography, petrology, geochemistry, etc) and computer science (e.g., software engineering, algorithms, parallel processing, etc). The student demonstrated that she had a very good understanding of the main fundamental physics and technologies for this project.


\clearpage

%%%%%
%%%%%
%%%%%

\noindent{\bfseries\large MSc in Oil $\&$ Gas Engineering\hfill September, 2014}

\bigskip

\begin{center}
{\Large Review of the MSc Dissertation $\lq$Energy/Exergy Feasibility Study of Carbon Dioxide Enhanced Oil Recovery' by Alexandros Pilichos}
\end{center}

\medskip

The dissertation investigates currently available technologies for CO$_{2}$ capture, transport and storage for mitigating GHG emissions in concentrated flow streams (i.e., carbon-based power stations). Most of all, the manuscript focuses on assessing the energy costs of injection of CO$_{2}$ in geological formation. The student undertook extensive literature review on current technologies for capture, transport and storage of CO$_{2}$, exploration facilities and case-studies. The student also developed a simplified model to assess the energy/exergy budget for a CO$_{2}$-EOR unit.

The manuscript is relatively well-written with a number of typos and unrevised sentences. A few sentences are very confusing and disconnected with no clear objectives. In general, the dissertation was interesting to read with enough content to enable discussion and analysis. A few general comments,
\begin{enumerate}
\item The main aim of {\it Abstracts} is to briefly describe the work undertaken by the author. In general {\it Abstracts} are divided in 4 parts: (i) motivation, (ii) main objectives, (iii) summary of the main procedures / techniques / technologies (optional) and (iv) main findings. The current {\it Abstract} encompasses (i-ii) and (iv).
%
\item The main {\it Introduction} section usually has the same (but more in-depth and descriptive) four parts of the {\it Abstract} and a brief summary of the remaining of the work. In addition, it is always expected a few clear statements -re main background (thus recent innovations related to the main topic), initial literature review and, most of all, technological / scientific gaps in the current understanding. In this dissertation, the {\it Introduction} section focused only on the motivation and objectives of the work. Literature review is spread over the remaining chapters with limited critical analysis of the work undertaken by several authors. 
%
\item A few {\it References} are missing important fields, e.g., page numbers and volumes. Also, there are different standards -re the authors' names/initials.
%
\item The student developed his own Excel spreadsheet model with data from a specified field and with a number of correlation found in the open-literature. There was a good discussion of the obtained results although this was slightly confusing.
%
\item A few terms introduced by the equations were not defined. 
% 
\end{enumerate}

The topic is very relevant for the O$\&$GE (and energy) and environmental science sectors. The student demonstrated that he had a good understanding of some of the main fundamental physics and technologies for this project.

\clearpage

%%%%
%%%%
%%%%
\noindent{\bfseries\large MSc in Oil $\&$ Gas Engineering\hfill September, 2014}

\bigskip

\begin{center}
{\Large Review of the MSc Dissertation $\lq$Formation and Stability of Natural Gas Clathrate Hydrates in Pipelines' by Efthymios Efthymiou}
\end{center}

\medskip

The dissertation investigates the formation and stability of hydrates in the energy sector. The student proceeded with an extensive literature of: (a) the morphological structure of natural gas clathrate hydrates, (b) chemical kinetics models and (c) thermodynamic models. The student also developed a simplified model of hydrate formation and compared his numerical results against industry-standard models and publicly available experimental data.

The manuscript is well-written with very few typos and unrevised sentences. Most of all, the dissertation is very well-structured with clear division and linkages between chapters, sections and paragraphs, leading to an easy and smooth reading. A few general comments,
\begin{enumerate}
%
\item The main aim of {\it Abstracts} is to briefly describe the work undertaken by the author. In general {\it Abstracts} are divided in 4 parts: (i) motivation, (ii) main objectives, (iii) summary of the main procedures / techniques / technologies (optional) and (iv) main findings. The current {\it Abstract} encompass all of them.
%
\item The main {\it Introduction} section usually has the same (but more in-depth and descriptive) four parts of the {\it Abstract} and a brief summary of the remaining of the work. In addition, it is always expected a few clear statements -re main background (thus recent innovations related to the main topic), initial literature review and, most of all, technological / scientific gaps in the current understanding. In this dissertation, the {\it Introduction} section focused only on the motivation and objectives of the work. Literature review is spread over the remaining chapters with good critical analysis of the work undertaken by several authors. 
%
\item A few {\it References} are missing important fields, e.g., page numbers and volumes. Also, there are different standards -re the authors' names/initials.
%
\item The student developed his own Matlab code based on one of the previously described model. However, no specific comments on the results obtained -- Figs. 6.2-5 and Tables 6.3-4, were made. The conclusions/discussions of this Chapter (6) were neatly done on Section 6.3, but the specific discussions are missing. Also, no comments were done on the methods used by the student to solve the model and the initial conditions.
%
\item A few terms introduced by the equations were not defined. 
% 
\end{enumerate}

The topic is very relevant for the O$\&$GE (and energy) sector and each chapter has been the focus of several academic- and industrial-based studies worldwide with clear cross-fertilisation with physics and chemistry (thermodynamics, kinetics, fluid mechanics, etc), geology $\&$ geophysics (e.g., lithography, petrology, geochemistry, etc) and computer science (e.g., software engineering, algorithms, parallel processing, etc).  The student demonstrated that he had an excellent understanding of the main fundamental physics and technologies for this project.

\clearpage


\noindent{\bfseries\large MSc in Oil $\&$ Gas Engineering\hfill September, 2014}

\bigskip

\begin{center}
{\Large Review of the MSc Dissertation $\lq$Precipitation and Thermodynamic Stability Analysis of Asphaltenes in Crude Oils' by Srikanth Ramani}
\end{center}

\medskip

The dissertation assess thermodynamic mechanisms of asphaltene precipitation in crude heavy oils under reservoir and pipelines conditions. The student investigated (a) traditional and state-of-the-art technologies (i.e., designed equations of state and formulations for heavy molecules) used to predict onset of asphaltene precipitation; (b) current methods to experimentally assess precipitation / deposition; (c) commercial software for solid-liquid-vapour equilibrium and comparison against given experimental data and; (d) current technologies to mitigate / remediate heavy hydrocarbon precipitation.

The dissertation is well-written with few typos and unrevised sentences. A few sentences are confusing and disconnected with no clear objectives and inter-connectivities. Most of all, the dissertation is very well-structured with clear division and linkages between chapters, sections and paragraphs, leading to an easy and smooth reading. A few general comments,
\begin{enumerate}
%
\item The main aim of {\it Abstracts} is to briefly describe the work undertaken by the author. In general {\it Abstracts} are divided in 4 parts: (i) motivation, (ii) main objectives, (iii) summary of the main procedures / techniques / technologies (optional) and (iv) main findings. The current {\it Abstract} encompass all of them.
%
\item The main {\it Introduction} section usually has the same (but more in-depth and descriptive) four parts of the {\it Abstract} and a brief summary of the remaining of the work. In addition, it is always expected a few clear statements -re main background (thus recent innovations related to the main topic), initial literature review and, most of all, technological / scientific gaps in the current understanding. The current chapters 1 ({\it Introduction}) and 2 ({\it Asphaltene Precipitation}) are well-written and  but lacks demonstration that the student investigated past work on the subject (i.e., fundamental thermodynamic and kinetic mechanisms for precipitation). Also, it is expected a summary of the remaining chapters at the end of the chapter. 
%
\item The {\it References} follow different standards with missing fields and no clear distinction between articles, conference proceedings, reports (internal or external), book chapters, books, communications (internal or external) etc.  
%
\item One of the stated objectives of the work is optimisation techniques / methods, but this was not covered in this dissertation.
%
\item Equations in Chapter 4 were not correctly numbered. In addition, several terms were not defined either in the nomenclature table or throughout the main text.
% 
\end{enumerate}

The topic is very relevant for the O$\&$GE (and energy) sector and each chapter has been the focus of several academic- and industrial-based studies worldwide with clear cross-fertilisation with physics and chemistry (thermodynamics, kinetics, macromolecules, fluid mechanics, etc), geology $\&$ geophysics (e.g., lithography, petrology, geochemistry, etc) and computer science (e.g., software engineering, algorithms, parallel processing, etc).  The student demonstrated that he had a solid understanding of the main fundamental physics and engineering concepts involved in the technologies for this project.

\clearpage

%%%%%%%
%%%%%%%
%%%%%%%
\begin{center}
\Huge{SwB Summer Project}\\
\huge{(Letter / Feedback)}\\
\huge{September 2014}
\end{center}

\vfill

\clearpage

\noindent{\bfseries\large SwB Summer Project \hfill September, 2014}

\medskip

\begin{center}
{\Large Review of the SwB Summer Project $\lq$Numerical Simulations of Multi-Fluid Flows in Heterogeneous Porous Media' by William C. Rad\"unz}
\end{center}

William Rad\"unz's project focused on numerically investigating fluid displacement in heterogeneous porous media. This project lasted from November 2013 to September 2014. During this period, Mr Rad\"unz performed the following tasks:
\begin{enumerate}
\item Literature review of the main subject areas (e.g., multiphase flows in porous media, viscous flow instabilities, partial differential equations, finite element methods, mesh generation etc);
\item Generated structured/unstructured mesh grids for fluid flow simulations in geological formations;
\item Performed initial model and software quality assurance (validation);
\item Performed a number of numerical simulations of multiphase flows in heterogeneous porous media as described in his report. 
\end{enumerate} 

On March 2014, he submitted a full paper to the \href{http://www.wccm-eccm-ecfd2014.org/frontal/default.asp}{XI Word Congress on Computational Mechanics} (WCCM XI organised by the \href{http://www.eccomas.org/}{ECCOMAS}/\href{http://iacm.info/}{IACM}) that was held in Barcelona on July.  The manuscript -- $\lq$A Multi-Scale Model of Multi-Fluid Flows Transport in Dual Saturated-Unsaturated Heterogeneous Porous Media' (in attachment), encompassed his scientific activities on multi-fluid modelling. During the summer, Mr Rad\"unz has been engaged in extend the manuscript towards a journal paper ({\it Advances in Water Resources}) focusing on twofolds objectives, demonstrating the numerical accuracy of control volume finite element methods (CVFEM) and investigation of viscous instabilities in heterogeneous flows.  The later can be readily applied to the study of {\it waterflooding} for enhanced oil recovery (EOR) and pollutant dispersion in rock matrices.



Mr Rad\"unz is a hard-working, tenacious and bright undergraduate student. He proved he was well able to apply his good knowledge of computational mathematics and mechanical engineering to real world challenges. He was fully committed to the project and although the topic was very broad, he managed to deliver an outstanding conference paper. During the late stages of his project -- May-September he worked in close collaboration with PhD students to design and perform numerical simulations and write a journal paper on viscous instabilities. 
\begin{center}
%\begin{figure} 
%\includegraphics[width=4.0cm,height=2.cm]{/data2/Dropbox/Admin/ScannedSignature}\\
{Dr Jefferson Gomes}\\
{Project Supervisor}\\
%\end{figure}
\end{center}



\clearpage

%%%
%%%
%%%

\noindent{\bfseries\large SwB Summer Project \hfill September, 2014}

\medskip

\begin{center}
{\Large Review of the SwB Summer Project $\lq$Numerical Simulation of a Dam-Break Problem' by Luciana Renata Carvalho Pedreira}
\end{center}

The main objective of Luciana Pedreira's project is to apply high-order compressive advection schemes into a new multi-fluid flows formulation to study shock wave based problems. The project started on March to September 2014. During this period, Ms Pedreira performed the following tasks:
\begin{enumerate}
\item Literature review of the main subject areas -- single and multiphase flows formulations, multi-components flow sub-models, partial differential equations, finite element methods, mesh generation, signal analysis, etc;
\item Generated structured/unstructured mesh grids for fluid flow simulations in several idealised geometries;
\item Performed initial model and software quality assurance (validation);
\item Performed a number of numerical simulations of oscillatory waves and dam-break problems with a range of materials/species as described in his report. 
\end{enumerate} 

Ms Pedreira is a bright undergraduate student. She proved she was well able to apply her good knowledge of computational mathematics and fluid mechanics to challenging engineering problems. She was fully committed to the project and although the topic was very difficult, she managed to deliver an impressive set of numerical results that may lead to future publications. 
\vspace{2.cm}
\begin{center}
%\begin{figure}
%\includegraphics[width=4.0cm,height=2.cm]{/data2/Dropbox/Admin/ScannedSignature}\\
{Dr Jefferson Gomes}\\
{Project Supervisor}\\
%\end{figure}
\end{center}

\clearpage



\noindent{\bfseries\large SwB Summer Project \hfill May-September, 2013}

\medskip

\begin{center}
{\large Review of the SwB Summer Project $\lq$Integrated Modelling Framework for Non-Linear Optimisation Methods' by Felipe Rocha Andrade and Estev\~ao Zappone Carlos}
\end{center}

Felipe Andrade (forth-year mechanical engineering student from the Federal University of Sergipe, UFS) and Estev\~ao Carlos (fifth-year production engineering student from S\~ao Paulo University, USP) worked together in this project from May to September 2013. The project focused on developing a computational framework for model and software quality assurance (QA) of non-linear optimisation algorithms. During this summer they performed the following tasks: 
\begin{enumerate}
\item Literature review of the main subject areas (e.g., software engineering, stochastic optimisation techniques, computational linear algebra, etc);
\item Familiarisation with compiled (e.g., Fortran and C) and functional (e.g., Python) programming languages and algorithm development techniques;
\item Implementation of the stochastic evolutionary algorithm-based {\it Cuckoo-Search algorithm} (CAS);
\item Development and implementation of the QA framework and graphical-user interface for model and software validation including 100+ non-linear test functions;
\item Assessment of performance of the CAS against the stochastic Simulated Annealing algorithms (SAA) in the test functions;
\item Assessment of performance of the CAS in highly non-linear polymer thermodynamic problems (vapour-liquid equilibrium calculations through global minimisation of the free Gibbs energy);
%\item Attending weekly meeting where they presented their progress to a select audience (often MSc and PhD students).
\end{enumerate}

%\medskip

On August 2013, they presented their development work in the Environmental and Industrial Fluid Mechanics Research Group seminar at the School of Engineering. On late August 2013, they submitted a full paper to the \href{http://nbcgib.uesc.br/emc2013/}{XVI Conference on Computational Modelling (XVI EMC)} that was held in Ilh\'eus (Bahia, Brazil) on October 2013. The manuscript -- `A Computational Framework for Non-Linear Optimisation Problems: The Cuckoo Search Algorithm' described their research activities during the summer and was orally presented by Mr Andrade with excellent feedback from the audience. 
%\bigskip 

%\clearpage
Mr Andrade and Mr Carlos were two very bright, engaging and enthusiastic young students who were able to strongly drive the project since the early stages. They both were fully committed to the project and although the main topic was very broad and complex, they manage to deliver outstanding conference paper and presentations.
 
\begin{center}
%\begin{figure}
%\includegraphics[width=4.0cm,height=2.cm]{/data2/Dropbox/Admin/ScannedSignature}\\
\vspace{-.5cm}
{Dr Jefferson Gomes (Project Supervisor)}\\
%\end{figure}
\end{center}



\clearpage


%%%%%%
%%%%%%
%%%%%%

\begin{center}
\Huge{MSc Oil and Gas Engineering Dissertation (EG5908)}\\
\huge{(Generic Feedback to all supervised students)}\\
\huge{June-August 2014}
\end{center}

\vfill

\clearpage

\noindent{\bfseries\large MSc in Oil $\&$ Gas Engineering\hfill August, 2014}


\noindent
Below, a few comments on the attached draft of the dissertation. 
\begin{enumerate}
%
\item Dissertations and thesis are always divided into chapters rather than sections (commonly used in reports). 
%
\item Text becomes increasingly more readable if it is clearly divided (and numbered) into sections, subsections etc.  
%
\item In the begining of each chapter you should include a paragraph (or two) summarising previous relevant chapters and, most of all, the main aspects of the current chapter. This summary should indicate what the reader should expect from the following sections and how the chapter relates to previous chapters and to the overall thesis' subject.
%
\item Similarly, at the end of each chapter, it is expected a short section summarising the main aspects/results/conclusions of the chapter and how this can be linked with the overall thesis' subject and the following chapter.
%
\item If your thesis contain a large number of symbols or non-common terms you shoul consider including a {\it Nomenclature} table that would contain all symbols (and units) used in the work. 
%
\item Figures and Tables {\bf must} be referenced in the main text -- they can not just $\lq$float around'! Also, figure/table captions should be self-contained, i.e., with a good description of the figure/table highlighting the most relevant aspects/information that the author wants to convene. 
%
\item The main aim of {\it Abstracts} is to briefly describe the work undertaken by the author. In general {\it Abstracts} are divided in 4 parts: (i) motivation, (ii) main objectives, (iii) summary of the main procedures / techniques / technologies (optional) and (iv) main findings. 
%
\item The main {\it Introduction} section usually has the same (but more in-depth and descriptive) four parts of the {\it Abstract} and a brief summary of the remaining of the work. In addition, it is always expected a few clear statements -re main background (thus recent innovations related to the main topic), initial literature review and, most of all, technological / scientific gaps in the current understanding. 
%
\item Appendices are used to convey complementary (and not crucial) information of the main chapters and need to be referenced in the main text.
%
\item {\it References}: regardless of the chosen citation style (e.g., ACS, AIP, AMS, IEEE, AIAA, etc) any reference {\bf must} contain the following fields: 
\begin{enumerate}
\item For journal papers: Authors, Paper Tittle, Journal Name, Volume, Pages, Year of publication;
\item For books: Authors, Book Tittle, Publisher, Year or Edition;
\item For book chapters: Authors, Chapter Tittle, Book Tittle, Editors, Publisher, Year or Edition;
\item For conference papers: Authors, Paper Tittle, Conference Tittle, Place (Country and/or City) where the conference was held, Year of the conference;
\item For reports,  private communications and Lecture Notes: Authors, Tittle, Place issued (Country and/or City and Institution where the document was originated), Year;
\item For PhD Thesis and MSc Dissertations: Author, Tittle, Institution (University and Department/School), Year.
\end{enumerate}  
Thus, for example:\\
\noindent
[39] P.L. Houtekamer and L. Mitchell, $\lq$Data Assimilation Using an Ensemble Kalman Filter Technique', {\it Monthly Weather Review}, 126:796-811, 1998.\\
\noindent
[40] K. Pruess, $\lq$Numerical Modelling of Gas Migration at a Proposed Repository for Low and Intermediate Level Nuclear Wastes', Technical Report LBL-25413, Lawrence Berkeley Laboratory, Berkeley (USA), 1990.\\
\noindent
[41] K. Aziz, A. Settari, {\it Fundamentals of Reservoir Simulation}, Elsevier Applied Science Publishers, New York (USA), 1986.\\
\noindent
[42] R.B. Lowrie, $\lq$Compact higher-Order Numerical Methods for Hyperbolic Conservation Laws', PhD Thesis, Department of Aerospace Engineering and Scientific Computing, University of Michigan (USA), 1996.
%
\end{enumerate}

\clearpage

%%%%%%
%%%%%%
%%%%%%

\begin{center}
\Huge{MEng Study Assessment (EG4515)}\\
\huge{(Review + Feedback)}\\
\huge{May-June 2014}
\end{center}

\vfill


\clearpage


\noindent{\bfseries\large EG4515 -- MEng Thesis Assessment \hfill November, 2014}
\bigskip

\begin{center}
  {\Large Review of the MEng Thesis $\lq$Evaluation of the Performance of Bioelectrochemical Systems for the Production of Added Value Commodity Chemicals from Glycerol' by MarySandra Oluchi Anunobi }
\end{center}

\medskip

The dissertation investigates the production of 1,3-propanediol from glycerol through bio-electrochemical conversion system.  Ms Anunobi performed conversion experiments in a H-type bio-electrochemical reactor followed by analysis of the products to assess overall performance. The dissertation encompasses three main subject areas within the main topic (biochemical conversion of polyol into diol organic compounds): microbiology $\&$ enzymology (metabolic conversion route) and electro-chemistry (i.e., induced oxi-reduction mechanisms). 

The manuscript is well-written with a small number of typos and unrevised sentences. A few sentences are confusing and disconnected with no clear objectives, however in general, the dissertation was interesting to read with enough content to enable discussion and analysis. A few general comments,
\begin{enumerate}
\item The main aim of {\it Abstracts} is to briefly describe the work undertaken by the author. In general {\it Abstracts} are divided in 4 parts: (i) motivation, (ii) main objectives, (iii) summary of the main procedures / techniques / technologies (optional) and (iv) main findings. The current {\it Abstract} encompasses all parts, {\bf except (ii)}, which is the main aim of an abstract.
%
\item The main {\it Introduction} section usually has the same (but more in-depth and descriptive) four parts of the {\it Abstract} and a brief summary of the remaining of the work. In addition, it is always expected a few clear statements -re main background (thus recent innovations related to the main topic), initial literature review and, most of all, technological / scientific gaps in the current understanding. The current {\it Introduction} section, although short, is well-written and cover {\bf (i)} and, in very limited way {\bf (ii)} ($\lq$... therefore the purpose of this project was to investigate the potential for 1,3-PDO production enhancement via the bioelectrochemical route using MECs and the role of microbes in the process'). Literature review is the focus of Chapter 2 instead. However, the student failed to demonstrate that she managed to clearly identify the main technological gaps in the different routes to produce 1,3-PDO (that led to her research on enhanced bio-electrochemical route).
%
\item Quality of figures are really good with clear and self-contained legends and captions. However, a few figures are $\lq$floating' with no explanation/description in the main text. For example, Fig. 3 is crucial to understand the several biochemical routes to produce valuable chemical products from glycerol, including the reduction into 1,3-PDO. However, this figure (or indeed this route) is only explained a few pages later when the reductive and oxidative pathways are described along with Fig. 5.
%
\item Also, it would easy the reading (and to fully understand synthesis routes) if the organic reactions (e.g., the reductive pathway from glycerol to 1,3-PDO) were explicitly written/described.  
%
\item Section 6.3 is incomplete.
%
\item Excellent Future Work section.
\end{enumerate}

The topic of the paper, organic biochemical conversion, is very relevant for the polymer (PTT polyester) processing sector. Individual aspects have been investigated by a number of researchers (academics and industrial) worldwide with clear cross-fertilisation with microbiology and enzymology (synthesis of valuable chemical compounds through metabolic route, biochemical kinetics etc), electro-chemistry (induced oxi-reduction reactions) and organic chemistry (chemical mechanisms of enzyme catalysis). The student demonstrated that she had understood the concepts involved in the technologies for the project with a solid comprehension on fundamentals of engineering and chemistry and the impact in polymer business.    


\bigskip
\noindent
{\Large Comments from Second Marker}
\begin{itemize}
\item $\lq$Introduction: Solid. Covers all of the main issues, is well-structured and well-referenced and provides reader with key challenges associated with this research area. An objectives/mains page is absent;
\item Results are, in the main well-described and presented and indicate a reasonable amount of data was obtained.
\item Discussion places results in context of prior knowledge based on through evaluation of literature. This is a mature well-written piece of work with evidence of critical thought ans assessment'    
\end{itemize}


\clearpage


\noindent{\bfseries\large EG4515 -- MEng Thesis Assessment \hfill June, 2014}
\bigskip

\begin{center}
  {\Large Review of the MEng Thesis $\lq$Lattice Boltzmann Simulations of the Influencing Factors of Density Driven Particle Segregation in Micro-Fluidised Beds' by Michael Drew}
\end{center}

\medskip

The dissertation numerically investigates particle segregation in micro fluidised beds through lattice Boltzmann methods. Mr Drew effectively performed a set of sensitivity analysis on the flow simulations.  The manuscript is reasonably well-written with a number of typos and unrevised sentences. Few sentences are very long, confusing and disconnected with no clear objectives. In general, the dissertation was interesting to read with enough content to enable discussion and analysis. A few general comments,
\begin{enumerate}
\item The main aim of {\it Abstracts} is to briefly describe the work undertaken by the author. In general {\it Abstracts} are divided in 4 parts: (i) motivation, (ii) main objectives, (iii) summary of the main procedures / techniques / technologies (optional) and (iv) main findings. The current {\it Abstract} successfully encompasses all 4 parts.
%
\item A few {\it references} used in the manuscript are incomplete and/or wrong. Regardless of the chosen citation style (e.g., ACS, AIP, AMS, IEEE, AIAA, etc) any reference {\bf must} contain the following fields: 
\begin{enumerate}
\item For journal papers: Authors, Paper Tittle, Journal Name, Volume, Pages, Year of publication;
\item For books: Authors, Book Tittle, Publisher, Year or Edition;
\item For book chapters: Authors, Chapter Tittle, Book Tittle, Editors, Publisher, Year or Edition;
\item For conference papers: Authors, Paper Tittle, Conference Tittle, Place (Country and/or City) where the conference was held, Year of the conference;
\item For reports,  private communications and Lecture Notes: Authors, Tittle, Place issued (Country and/or City and Institution where the document was originated), Year;
\item For PhD Thesis and MSc Dissertations: Author, Tittle, Institution (University and Department/School), Year.
\end{enumerate}  
Thus, for example:\\
\noindent
[39] P.L. Houtekamer and L. Mitchell, $\lq$Data Assimilation Using an Ensemble Kalman Filter Technique', {\it Monthly Weather Review}, 126:796-811, 1998.\\
\noindent
[40] K. Pruess, $\lq$Numerical Modelling of Gas Migration at a Proposed Repository for Low and Intermediate Level Nuclear Wastes', Technical Report LBL-25413, Lawrence Berkeley Laboratory, Berkeley (USA), 1990.\\
\noindent
[41] K. Aziz, A. Settari, {\it Fundamentals of Reservoir Simulation}, Elsevier Applied Science Publishers, New York (USA), 1986.\\
\noindent
[42] R.B. Lowrie, $\lq$Compact higher-Order Numerical Methods for Hyperbolic Conservation Laws', PhD Thesis, Department of Aerospace Engineering and Scientific Computing, University of Michigan (USA), 1996.
%
\item The main {\it Introduction} section usually has the same (but more in-depth and descriptive) four parts of the {\it Abstract} and a brief summary of the remaining of the work. In addition, it is always expected a few clear statements -re main background (thus recent innovations related to the main topic), initial literature review and, most of all, technological / scientific gaps in the current understanding. The current {\it Introduction} section is well-written and cover all the aforementioned points (though the literature review is the focus of chapter 2 instead). %However, it failed to demonstrate that the student managed to identify the main technological gaps (although this was clear in the following chapters).
%
\item Quality of figures are very poor. The font size for some of them is too small and no legends are present in some of them. Also, several figures are $\lq$floating' with no explanation/description in the main text.
%
\item All appendices (except for Table 7.2) are not referred in the main text and all symbols used within are not defined.
%
\item The manuscript focus on lattice Boltzmann methods, but this was never fully (or partially) explained. So what equations are actually been solved?
%
\item Excellent Conclusions and Future Work sections.
%
%\item The contents within the Sections 3.2 and 6 were crucial to fully understand and appreciate the work. Unfortunately, the later was superficial and incomplete -- for example, first paragraph of page 41, how the models were built and what they are based upon. Most of the available models are generic and volume-based, were there any attempt to detailed modelling of chemical absorption in packed beds? What is $\lq$dynamic modelling'? 
%
%\item I guess Fig. 1 was copied from some public available source. If this was the case, the student should have cited the reference. In any case, the $\lq${\it Eclipse Simulator}' should be replaced by $\lq${\it Simulator}', as {\it Eclipse} is a commercial simulator developed by Schlumberger. Also after the $\lq$time-step', the flow chart should indicate returning to the $\lq$Simulator'.
\end{enumerate}

The topic of the paper, fluidisation technologies, is very relevant for the energy and environmental sectors. Individual aspects have been investigated by a number of researchers (academics and industrial) worldwide with clear cross-fertilisation with mathematics (e.g., solution of partial differential equations, discrete methods, numerical integration, optimisation, etc), physics/chemistry (fluid mechanics, material sciences, phase change, etc), computer science  and chemical / mechanical / petroleum engineering (industrial processing, safety, facilities, etc). The student demonstrated that he had understood the concepts involved in the technologies for the project with a solid comprehension on fundamentals of engineering .    



\clearpage

%%%%%%
%%%%%%
%%%%%%


\noindent{\bfseries\large EG4515 -- MEng Thesis Assessment \hfill May, 2014}
\bigskip

\begin{center}
  {\Large Review of the MEng Thesis $\lq$Synthesis, Characterisation and Sorption Studies on Imogolite: a Nano-Tubular Aluminium Silicate Mineral' by Creshia Jones}
\end{center}

\medskip

The dissertation investigates currently available methods for synthesis and characterisation of imogolite that can potentially be used for radionuclides immobilisation in geological disposal facilities (GDF). Ms Jones studied three subject areas within the main topic (radionuclide immobilisation): transport of cations and chemical reaction kinetics and mechanisms.  

The manuscript is well-written with a small number of typos and unrevised sentences. A few sentences are confusing and disconnected with no clear objectives. In general, the dissertation was interesting to read with enough content to enable discussion and analysis. A few general comments,
\begin{enumerate}
\item The main aim of {\it Abstracts} is to briefly describe the work undertaken by the author. In general {\it Abstracts} are divided in 4 parts: (i) motivation, (ii) main objectives, (iii) summary of the main procedures / techniques / technologies (optional) and (iv) main findings. The current {\it Abstract} successfully encompasses all 4 parts, although the transition between sentences/paragraphs were not very smooth.
%
\item A few {\it references} used in the manuscript are incomplete and/or wrong. Regardless of the chosen citation style (e.g., ACS, AIP, AMS, IEEE, AIAA, etc) any reference {\bf must} contain the following fields: 
\begin{enumerate}
\item For journal papers: Authors, Paper Tittle, Journal Name, Volume, Pages, Year of publication;
\item For books: Authors, Book Tittle, Publisher, Year or Edition;
\item For book chapters: Authors, Chapter Tittle, Book Tittle, Editors, Publisher, Year or Edition;
\item For conference papers: Authors, Paper Tittle, Conference Tittle, Place (Country and/or City) where the conference was held, Year of the conference;
\item For reports,  private communications and Lecture Notes: Authors, Tittle, Place issued (Country and/or City and Institution where the document was originated), Year;
\item For PhD Thesis and MSc Dissertations: Author, Tittle, Institution (University and Department/School), Year.
\end{enumerate}  
Thus, for example:\\
\noindent
[39] P.L. Houtekamer and L. Mitchell, $\lq$Data Assimilation Using an Ensemble Kalman Filter Technique', {\it Monthly Weather Review}, 126:796-811, 1998.\\
\noindent
[40] K. Pruess, $\lq$Numerical Modelling of Gas Migration at a Proposed Repository for Low and Intermediate Level Nuclear Wastes', Technical Report LBL-25413, Lawrence Berkeley Laboratory, Berkeley (USA), 1990.\\
\noindent
[41] K. Aziz, A. Settari, {\it Fundamentals of Reservoir Simulation}, Elsevier Applied Science Publishers, New York (USA), 1986.\\
\noindent
[42] R.B. Lowrie, $\lq$Compact higher-Order Numerical Methods for Hyperbolic Conservation Laws', PhD Thesis, Department of Aerospace Engineering and Scientific Computing, University of Michigan (USA), 1996.
%
\item The main {\it Introduction} section usually has the same (but more in-depth and descriptive) four parts of the {\it Abstract} and a brief summary of the remaining of the work. In addition, it is always expected a few clear statements -re main background (thus recent innovations related to the main topic), initial literature review and, most of all, technological / scientific gaps in the current understanding. The current {\it Introduction} section is well-written and cover most of the aforementioned points. However, it failed to demonstrate that the student managed to identify the main technological gaps (although this was clear in the following chapters).
%
\item Some of the chemical reactions in Chapter 2 are not properly balanced (or are wrongly described).
%
\item The topic is very timely as there are several R$\&$D initiatives worldwide on specifics aspects of development and safety assessment of nuclear waste repositories (in particular in immobilisation and storage of ILW and HLW). Novel technologies that accurately address the aforementioned technologies are crucial for the  nuclear sector as industry and NDA need to ensure all waste is accountable and secured. The procedure for synthesis and characterisation of imogolite is relatively complex and it would have helped the reader if a diagram of the work-flow was included in the dissertation. 
%
\item Figures and Tables {\bf must} be referenced in the main text -- they can not just $\lq$float around' (e.g., Table 5.2 and Fig. 5.2). 
%
\item First paragraph of page 58 refer to Fig. 7.5 which does not exist in the document, although a few conclusions are drawn based on this Figure. 
%
\end{enumerate}

The topic of the paper, new materials for radionuclide immobilisation, is very relevant for the energy and environmental sectors. Individual aspects have been investigated by a number of researchers (academics and industrial) worldwide with clear cross-fertilisation with physics/chemistry (material sciences, reactions, surface chemistry, etc) and chemical / nuclear / civil engineering (industrial processing, safety, etc). The student demonstrated that she had understood the concepts involved in the technologies for the project with a solid comprehension on fundamentals of engineering, chemistry and physics and the impact on energy and environmental business.    

\medskip
\noindent
{\Large Comments from Second Supervisor -- Prof Glasser}\\
\noindent
$\lq$It is an excellent piece of work showing great maturity of thought, concept, design and interpretation. This is especially so as I was ill for several crucial weeks towards the final stages of experimental data collection and integration. Some of the experiments described were new to me: if you did not suggest them, she must have designed them herself. And, as they are entirely appropriate to the  the topic, they are welcome additions which lift the quality of the work from very good to excellent. 

\noindent
Much of the work is entirely original and, of the many approaches which could have been taken to experimental design, hers are well chosen and well implemented. There were many experimental and instrumental barriers to overcome in the course of the work in order to yield results and I am amazed to see how much energy she must have expended to overcome these barriers.

\noindent
The Nuclear Decommissioning Agency (NDA) was aware of this work and I had promised, if the work warranted,  to send them the results. I would have no hesitation in doing so. The style and level of professionalism is such that very little rewriting is needed. I have made a few notes on post-its  which I would be glad to discuss with her before the Dissertation is sent to NDA (subject to any necessary consents) but no more than I would expect from a document prepared by one of my Post-Docs.  

%So this leads me to the one minor disagreement with your marks: I would have suggested at least an 80 for the criterion marked  "Evidence of innagination..."Otherwise
\noindent
(...) It is clear that you, like I, appreciated her enthusiasm for the work and that the enthusiasm was disciplined and focused. I completely agree with the rest of your scores: although high, they are well justified.'

\medskip
\noindent
{\Large Comments from Second Examiner}
\begin{enumerate}
\item $\lq$Well written thesis. Use of first person should be avoided. 
\item Well presented, good  looking and logically structured. 
\item Mature technical narrative for level 4 student. 
\item Context and importance of topic clearly set put.
\item Extensive reference listing supporting a through review of the topic.
\item Student demonstrated good reasoning and insights towards theory and interpreting results.
\item Conclusions and recommendations are well wide.'
\end{enumerate}


\clearpage


\noindent{\bfseries\large EG4515 -- MEng Thesis Assessment \hfill May, 2014}
\bigskip

\begin{center}
  {\Large Review of the MEng Thesis $\lq$Feasibility Study of Carbon Dioxide Capture and Storage' by Sophie Svensen}
\end{center}

\medskip

The dissertation investigates currently available technologies for CO$_{2}$ capture, transport and storage for mitigating GHG emissions in concentrated flow streams (i.e., carbon-based power stations). The student studied three subject areas within the main topic (CCS processes): capture technologies (adsorption, absorption, membranes within pre-, post-combustion and oxy-fuel technologies),  storage technologies and energy analysis $\&$ integration in power plants. 

The manuscript is well-written with a small number of typos and unrevised sentences. A few sentences are very confusing and disconnected with no clear objectives. In general, the dissertation was interesting to read with enough content to enable discussion and analysis. A few general comments,
\begin{enumerate}
\item The main aim of {\it Abstracts} is to briefly describe the work undertaken by the author. In general {\it Abstracts} are divided in 4 parts: (i) motivation, (ii) main objectives, (iii) summary of the main procedures / techniques / technologies (optional) and (iv) main findings. The current {\it Abstract} successfully encompasses all 4 parts.
%
\item A few {\it references} used in the manuscript are incomplete and/or wrong. Regardless of the chosen citation style (e.g., ACS, AIP, AMS, IEEE, AIAA, etc) any reference {\bf must} contain the following fields: 
\begin{enumerate}
\item For journal papers: Authors, Paper Tittle, Journal Name, Volume, Pages, Year of publication;
\item For books: Authors, Book Tittle, Publisher, Year or Edition;
\item For book chapters: Authors, Chapter Tittle, Book Tittle, Editors, Publisher, Year or Edition;
\item For conference papers: Authors, Paper Tittle, Conference Tittle, Place (Country and/or City) where the conference was held, Year of the conference;
\item For reports,  private communications and Lecture Notes: Authors, Tittle, Place issued (Country and/or City and Institution where the document was originated), Year;
\item For PhD Thesis and MSc Dissertations: Author, Tittle, Institution (University and Department/School), Year.
\end{enumerate}  
Thus, for example:\\
\noindent
[39] P.L. Houtekamer and L. Mitchell, $\lq$Data Assimilation Using an Ensemble Kalman Filter Technique', {\it Monthly Weather Review}, 126:796-811, 1998.\\
\noindent
[40] K. Pruess, $\lq$Numerical Modelling of Gas Migration at a Proposed Repository for Low and Intermediate Level Nuclear Wastes', Technical Report LBL-25413, Lawrence Berkeley Laboratory, Berkeley (USA), 1990.\\
\noindent
[41] K. Aziz, A. Settari, {\it Fundamentals of Reservoir Simulation}, Elsevier Applied Science Publishers, New York (USA), 1986.\\
\noindent
[42] R.B. Lowrie, $\lq$Compact higher-Order Numerical Methods for Hyperbolic Conservation Laws', PhD Thesis, Department of Aerospace Engineering and Scientific Computing, University of Michigan (USA), 1996.
%
\item The main {\it Introduction} section usually has the same (but more in-depth and descriptive) four parts of the {\it Abstract} and a brief summary of the remaining of the work. In addition, it is always expected a few clear statements -re main background (thus recent innovations related to the main topic), initial literature review and, most of all, technological / scientific gaps in the current understanding. The current {\it Introduction} section is well-written and cover all the aforementioned points. However, it failed to demonstrate that the student managed to identify the main technological gaps (although this was clear in the following chapters).
%
\item The topic is very timely as there are several R$\&$D initiatives worldwide on specifics aspects of CCS (in particular in capture and storage technologies and proof-of-concept). Novel technologies that accurately address the aforementioned technologies are crucial for energy and environmental management (including decision-making). Several aspects on these topics were covered in the manuscript, but an in-depth idealised or field example would help the reader to fully understand the subject. The calculations described in the report were interesting but it would have helped if the flow diagrams of the investigated plants were shown and the methods for analysis were fully described. 
%
\item The contents within the Sections 3.2 and 6 were crucial to fully understand and appreciate the work. Unfortunately, the later was superficial and incomplete -- for example, first paragraph of page 41, how the models were built and what they are based upon. Most of the available models are generic and volume-based, were there any attempt to detailed modelling of chemical absorption in packed beds? What is $\lq$dynamic modelling'? 
%
%\item I guess Fig. 1 was copied from some public available source. If this was the case, the student should have cited the reference. In any case, the $\lq${\it Eclipse Simulator}' should be replaced by $\lq${\it Simulator}', as {\it Eclipse} is a commercial simulator developed by Schlumberger. Also after the $\lq$time-step', the flow chart should indicate returning to the $\lq$Simulator'.
\end{enumerate}

The topic of the paper, CCS technologies, is very relevant for the energy and environmental sectors. Individual aspects have been investigated by a number of researchers (academics and industrial) worldwide with clear cross-fertilisation with mathematics (e.g., solution of partial differential equations, optimisation, etc), physics/chemistry (fluid mechanics, material sciences, reactions, signal analysis, phase change, etc), chemical / mechanical / electrical / petroleum engineering (industrial processing, safety, facilities, power grid, etc), economy (project management, financial forecasting etc). The student demonstrated that she had understood the concepts involved in the technologies for the project with a solid comprehension on fundamentals of engineering and the impact on energy and environmental business.    


\medskip
\noindent
{\Large Comments from Second Examiner}
\begin{description}
\item[Presentation] $\lq$Well written. Diagrams simple but neatly presented. Well referenced and draws from a wide variety of sources including journals.'
\item[Technical] $\lq$Given the broad nature of the topic I don't think it was feasible to go into greater depth for any particular technology. The depth and breadth were appropriate demonstrating a good grasp of the technical concepts. That said I have tried to reflect the $\lq$difficulty' factor in assessing the thesis.'
\item[Critical Reasoning]$\lq$The ability to critically evaluate was limited by availability of data. To begin with the student relied heavily on material taken directly from the literature with little critical analysis. That being said the balance was redressed somewhat towards the end.'
\end{description}
\clearpage



\noindent{\bfseries\large EG4515 -- MEng Thesis Assessment \hfill May, 2014}

\bigskip

\begin{center}
  {\Large Review of the MEng Thesis $\lq$An Investigation into the Thermal Conductivity of Nanofluids utilising Event-Driven Molecular Dynamics Simulation' by Calun Iain Murdoch}
\end{center}

\medskip

The thesis numerically investigates numerical models for thermal conductivity predictions through molecular dynamics. The manuscript is very well-written with a small number of typos and unrevised sentences. In general, the manuscript was interesting to read with a wide range of content to enable discussion and analysis. A few general comments,
\begin{enumerate}
%
\item Format of {\bf all} journal paper {\it references} used in the manuscript is {\bf wrong} -- format for chapters in books were used instead. 
%
\item The main aim of {\it Abstracts} is to briefly describe the work undertaken by the author. In general {\it Abstracts} are divided in 4 parts: (i) motivation, (ii) main objectives, (iii) summary of the main procedures / techniques / technologies (optional) and (iv) main findings. The current {\it Abstract} encompasses all parts.
%
\item The main {\it Introduction} section usually has the same (but more in-depth and descriptive) four parts of the {\it Abstract} and a brief summary of the remaining of the work. In addition, it is always expected a few clear statements -re main background (thus recent innovations related to the main topic), initial literature review and, most of all, technological / scientific gaps in the current understanding. The current {\it Introduction} section is well-written but lacks demonstration that the student investigated past work on the subject topics (e.g., molecular dynamics, Monte-Carlo methods, kinetic theory, nanofluids dynamics etc) but rather relied on a few reference sources based on the software manual.
%
\end{enumerate}

The topic of the paper, molecular dynamics, is very relevant for the energy and environmental sectors. Individual aspects have been investigated by a number of researchers (academics and industrial) worldwide with clear cross-fertilisation with mathematics (e.g., solution of partial/ordinary differential equations, optimisation, numerical integration etc), physics/chemistry/biology (fluid mechanics, material sciences, reactions, signal analysis, phase change, heat and mass transfers etc) and chemical / mechanical engineering (industrial processing, safety, facilities, etc). The student demonstrated that he had understood the concepts involved in the methods and technologies for the project with a solid comprehension on fundamentals of maths and engineering.    




\clearpage




\noindent{\bfseries\large EG4515 -- MEng Thesis Assessment \hfill May, 2014}

\bigskip

\begin{center}
  {\Large Review of the MEng Thesis $\lq$A Review of the Reaction Kinetics $\&$ Reactor Design for the Selective Hydrogenation of Acetylene from Ethylene Streams' by Lesley-Anne Rowand}
\end{center}

\medskip

The dissertation investigates chemical kinetics in selective hydrogenation reactions of acetylene. The manuscript reviews past and currently experimental and theoretical studies of organic chemical reaction mechanisms and proposes an experimental selectivity analysis on a set of supported catalysts and inflow ratio $\left(\text{H}_{2}:\text{C}_{2}\text{H}_{2}\right)$. Ms Rowand studied three subject areas within the main topic (reaction kinetics technologies: organic chemistry reaction mechanisms, heterogeneous catalysis (surface chemistry) and process engineering.

The manuscript is reasonably well-written with a small number of typos and unrevised sentences. A few sentences are very confusing and disconnected with no clear objectives. In general, the paper was interesting to read with enough content to enable discussion and analysis. A few general comments,
\begin{enumerate}
\item The main aim of {\it Abstracts} is to briefly describe the work undertaken by the author. In general {\it Abstracts} are divided in 4 parts: (i) motivation, (ii) main objectives, (iii) summary of the main procedures / techniques / technologies (optional) and (iv) main findings. The current {\it Abstract} encompasses (iii) and (iv).
%
\item Format of {\bf all} journal paper {\it references} used in the manuscript is {\bf wrong} -- format for chapters in books were used instead. 
%
\item The main {\it Introduction} section usually has the same (but more in-depth and descriptive) four parts of the {\it Abstract} and a brief summary of the remaining of the work. In addition, it is always expected a few clear statements -re main background (thus recent innovations related to the main topic), initial literature review and, most of all, technological / scientific gaps in the current understanding. The current {\it Introduction} section, although short (i.e., an extended {\it Abstract}) is well-written but does not cover most of the aforementioned parts. 
%
\item The topic is very timely as there are several R$\&$D initiatives worldwide on efficient methods for industrial hydrogenation of acetylene. Ms Rowand demonstrated that she understood the importance of the topic and undertook a through review.
%
\item Quality of figures are very poor, in particular those describing chemical mechanisms reactions (a critical part of the manuscript). Also, several figures are $\lq$floating' with no explanation/description in the main text (e.g., Figs. 16-19).
%
\item Section 7.1.2 was very superficial with no explanation of the actual experimental procedure.
\end{enumerate}

The topic of the paper, hydrogenation of acetylene, is very relevant for the energy and environmental sectors. Individual aspects have been investigated by a number of researchers (academics and industrial) worldwide with clear cross-fertilisation with physics/chemistry (e.g., surface physics, chemical kinetics, thermodynamics, fluid mechanics, material sciences, signal analysis, etc) and chemical / petroleum engineering (industrial processing, safety, etc). The student demonstrated that she had understood the concepts involved in the technologies for the project with a solid comprehension on fundamentals of engineering and the impact on energy business.    


\clearpage

%%%%%%
%%%%%%
%%%%%%

\begin{center}
\Huge{BEng Level 4 Thesis Assessment (EG4012)}\\
\huge{(Review + Feedback)}\\
\huge{May 2014}
\end{center}

\vfill

\clearpage



%%%%%%
%%%%%%
%%%%%%

\noindent{\bfseries\large EG4012 -- BEng Level 4 Thesis Assessment \hfill May, 2014}

\bigskip

\begin{center}
  {\Large Review of the BEng Thesis $\lq$Work Flow for Reservoir Simulation Engineering: From Mapping to Simulation' by Orawanya Boonklong}
\end{center}

\medskip

The dissertation focuses on study all stages involved on reservoir engineering, from characterising and mapping hydrocarbons fields to flow simulation using commercial software. Ms Boonklong studied three subject areas within the main topic (reservoir simulator): log-/core-analysis, static and dynamic models and history matching. 


The manuscript is reasonably well-written with a large number of typos and unrevised sentences. Some paragraphs and sections are very confusing and disconnected with no clear objectives and inter-connectivities. A few general comments,
\begin{enumerate}
\item The main aim of {\it Abstracts} is to briefly describe the work undertaken by the author. In general {\it Abstracts} are divided in 4 parts: (i) motivation, (ii) main objectives, (iii) summary of the main procedures / techniques / technologies (optional) and (iv) main findings. This manuscript's {\it Abstract} covered (i) and (ii), however the transitions between sentences/paragraphs were not very smooth.
%
\item The main {\it Introduction} section usually has the same (but more in-depth and descriptive) four parts of the {\it Abstract} and a brief summary of the remaining of the work. In addition, it is always expected a few clear statements -re main background (thus recent innovations related to the main topic), initial literature review and, most of all, technological /successfully encompasses all scientific gaps in the current understanding. It is also expected a summary of the remaining sections. There is {\bf no} {\it Introduction} in this work -- this was rather replaced by an $\lq$Introduction to Reservoir Simulation' section where a brief review of the main concepts and definitions of flow simulators were outlined. 
%
\item Figures and Tables {\bf must} be referenced in the main text -- they can not just $\lq$float around'! Also, figure/table captions should be self-contained, i.e., with a good description of the figure/table highlighting the most relevant aspects/information that the author wants to convene. 
%
\item The {\it References} follow different standards with missing fields and no clear distinction between articles, conference proceedings, reports (internal or external), book chapters, books, communications (internal or external) etc.  Regardless of the chosen citation style (e.g., ACS, AIP, AMS, IEEE, AIAA, etc) any reference {\bf must} contain the following fields: 
\begin{enumerate}
\item For journal papers: Authors, Paper Tittle, Journal Name, Volume, Pages, Year of publication;
\item For books: Authors, Book Tittle, Publisher, Year or Edition;
\item For book chapters: Authors, Chapter Tittle, Book Tittle, Editors, Publisher, Year or Edition;
\item For conference papers: Authors, Paper Tittle, Conference Tittle, Place (Country and/or City) where the conference was held, Year of the conference;
\item For reports,  private communications and Lecture Notes: Authors, Tittle, Place issued (Country and/or City and Institution where the document was originated), Year;
\item For PhD Thesis and MSc Dissertations: Author, Tittle, Institution (University and Department/School), Year.
\end{enumerate}  
Thus, for example:\\
\noindent
[39] P.L. Houtekamer and L. Mitchell, $\lq$Data Assimilation Using an Ensemble Kalman Filter Technique', {\it Monthly Weather Review}, 126:796-811, 1998.\\
\noindent
[40] K. Pruess, $\lq$Numerical Modelling of Gas Migration at a Proposed Repository for Low and Intermediate Level Nuclear Wastes', Technical Report LBL-25413, Lawrence Berkeley Laboratory, Berkeley (USA), 1990.\\
\noindent
[41] K. Aziz, A. Settari, {\it Fundamentals of Reservoir Simulation}, Elsevier Applied Science Publishers, New York (USA), 1986.\\
\noindent
[42] R.B. Lowrie, $\lq$Compact higher-Order Numerical Methods for Hyperbolic Conservation Laws', PhD Thesis, Department of Aerospace Engineering and Scientific Computing, University of Michigan (USA), 1996.
%
\item Some concepts, definitions and equations are wrong and/or incomplete, e.g., 
\begin{enumerate}
\item Equation 6 should read as,
\begin{displaymath}
\frac{\partial}{\partial t }\left(\phi\rho_{\alpha}S_{\alpha}\right) = -\nabla\left(\rho_{\alpha}u_{\alpha}S_{\alpha}\right)+q_{\alpha}
\end{displaymath}
\item Table 8 shows a few example of EOS that represent the PVT behaviour of fluids;
\item etc
\end{enumerate}

\end{enumerate}

The topic of this dissertation, workflow of reservoir simulators, is very relevant for the O$\&$GE (and energy) sector.  Individual aspects (within each chapter) have been investigated by a number of researchers (academics and industrial) worldwide with clear cross-fertilisation with mathematics (e.g., inverse theory, mesh generation, solution of partial differential equations, uncertainty quantification etc), physics/chemistry (fluid and solid mechanics, signal processing, phase change, etc), geology $\&$ geophysics (e.g., lithography, petrology, geochemistry, etc), computer science (e.g., software engineering, algorithms, parallel processing, etc) and chemical / petroleum engineering (industrial processing, safety, EOR, CCS etc). Ms Boonklong  demonstrated that she had understood the concepts involved in the technologies for the project with a reasonable comprehension on fundamentals of engineering and the impact on  the O$\&$G exploration business.   


\bigskip

\noindent
{\large Comments from Second Examiner}

\noindent
$\lq$The structure and style of the thesis is ok, but with poor attention to details. Unclear writing makes it hard to follow, and there are many mistakes, for instance in the chapter on the software package {\it Eclipse}, the word {\it Eclipse} is spelt in three different ways.  Equations are copied and pasted from {\it pdf} documents, and not well explained.  Concepts, such as auto-correlation and deconvolution are mentioned but not explained, showing poor critical thinking. Some simulation work was done, and the results explained briefly. This suggest some level of competence' 



\clearpage


\noindent{\bfseries\large EG4012 -- BEng Level 4 Thesis Assessment \hfill May, 2014}

\bigskip

\begin{center}
  {\Large Review of the BEng Thesis $\lq$Phase Behaviour of Simple Molecular Models' by Daniel Criag McKechnie}
\end{center}

\medskip

The manuscript is badly written with {\bf no clear indication} of the real objective(s) of the work. In the {\it Abstract}, the objective is stated as $\lq$(...) A look into understanding the link between fundamental liquid-gas behaviour and simulation data throughout the report'.  However this was not effectively tried at any part of this manuscript. From the title, it is expected an {\it in-depth} review of thermodynamic equilibrium calculations, however the student only {\bf briefly} summarised basic concepts of thermodynamics, e.g., first and second laws (though, not entirely right), properties and EOS. There are a number of typos and unrevised sentences. Several paragraphs and sentences are very confusing and disconnected with no clear objectives. A few general comments,
\begin{enumerate}
\item The main aim of {\it Abstracts} is to briefly describe the work undertaken by the author. In general {\it Abstracts} are divided in 4 parts: (i) motivation, (ii) main objectives, (iii) summary of the main procedures / techniques / technologies (optional) and (iv) main findings. This paper's {\it Abstract} covered (i) and (ii).
%
\item The main {\it Introduction} section usually has the same (but more in-depth and descriptive) four parts of the {\it Abstract} and a brief summary of the remaining of the work. In addition, it is always expected a few clear statements -re main background (thus recent innovations related to the main topic), initial literature review and, most of all, technological / scientific gaps in the current understanding. It is also expected a summary of the remaining sections. There is {\bf no} {\it Introduction} in this work -- this was rather replaced by an $\lq$Introduction to Thermodynamics' section where a brief review of the main concepts in classical thermodynamics was attempted. 
%
\item Figures and Tables {\bf must} be referenced in the main text -- they can not just $\lq$float around'! Also, figure/table captions should be self-contained, i.e., with a good description of the figure/table highlighting the most relevant aspects/information that the author wants to convene. 
%
\item The (very few) {\it References} follow different standards with missing fields and no clear distinction between articles, conference proceedings, reports (internal or external), book chapters, books, communications (internal or external) etc.  
%
\item Some concepts and definitions are wrong and/or incomplete, e.g., entropy (Clausius inequalities), Helmholtz energy, etc;
%
\item Expressions in vdW and SRK EOS are wrong;
%
\item As the manuscript's objective is to study phase behaviour, I would expect an in-depth discussion of EOS (derivation and calculations) for pure components, how accurate they are, polynomial- (Virial) and statistically-based etc. The presentation and discussion were very disappointing. 
%
\item Molecular simulations are the main focus of {\it Section 4}, although no real explanation of the fundamentals, theory and procedures -- and this (just one page) is the main objective of his work.
%
\item A numerical calculation of VLE for octane is attempted in Page 30, but no details, results or analysis are given.

\end{enumerate}

The topic of the paper, phase behaviour is very relevant for the energy and chemistry sectors. Individual aspects have been investigated by a number of researchers (academics and industrial) worldwide with clear cross-fertilisation with physics/chemistry (material sciences, reactions, phase change, corrosion, etc), chemical / petroleum engineering (industrial processing, safety, etc) and maths (optimisation, etc). The student did not demonstrate that he had understood the basic concepts involved in the topics for this project.    


\clearpage

%%%%%%
%%%%%%
%%%%%%

\begin{center}
\Huge{Advanced Topics for MEng Study (EG5085)}\\
\huge{2$^{nd}$ Paper (Review + Feedback)}\\
\huge{January 2014}
\end{center}

\vfill

\clearpage


\noindent{\bfseries\large EG5085 -- Advanced Topics for MEng Study $\left(\text{2}^{\text{nd}}\text{ Paper}\right)$\hfill January, 2014}

\bigskip

\begin{center}
  {\Large Review of the 2$^{\text{nd}}$ Paper $\lq$Assessment of Polymer Flooding: Fluid Properties; Reservoir Condition Economical and Environmental Impact' by Ejikemeuwa Eric Nnanna}
\end{center}

\medskip

The manuscript investigates technical conditions (thermo-fluid, transport, chemical, geo-physics, etc) for enhanced oil recovery using polymer solution injection in geological/reservoir formations.  The student studied three subject areas within the main topic (polymer flooding): polymer thermo-physical properties, transport phenomena in saturated reservoirs, physico-chemistry interactions (i.e., contact chemistry).

The paper is reasonably well-written with a small number of typos and unrevised sentences. A few sentences are very confusing and disconnected with no clear objectives. In general, the paper was very interesting to read with enough content to enable discussion and analysis. A few general comments,
\begin{enumerate}
\item The main aim of {\it Abstracts} is to briefly describe the work undertaken by the author. In general {\it Abstracts} are divided in 4 parts: (i) motivation, (ii) main objectives, (iii) summary of the main procedures / techniques / technologies (optional) and (iv) main findings. This paper's {\it Abstract} only targeted on (ii).
%
\item The main {\it Introduction} section usually has the same (but more in-depth and descriptive) four parts of the {\it Abstract} and a brief summary of the remaining of the work. In addition, it is always expected a few clear statements -re main background (thus recent innovations related to the main topic), initial literature review and, most of all, technological / scientific gaps in the current understanding. The current {\it Introduction} section is well-written although it does not cover the aforementioned items. It is also expected a summary of the remaining sections. 
%
\item The topic is very timely as there are several R$\&$D initiatives worldwide on enhanced oil recovery techniques aiming to improve hydrocarbon production. Novel technologies that accurately address chemical EOR are crucial for reservoir management (including decision-making). Several aspects of these technologies were covered in the paper, but an idealised or field example would help the reader to fully understand the subject.
%
\item Key properties definitions are missing and symbols/acronyms were used with no prior description. For example, WAG, HCPV, MMP, HPAM etc.
%
\item Figures and Tables {\bf must} be referenced in the main text -- they can not just $\lq$float around'! Also, figure/table captions should be self-contained, i.e., with a good description of the figure/table highlighting the most relevant aspects/information that the author wants to convene. 
%
\item References used different styles, and for most of them it is not clear if they are either conference papers, journal papers, books or book chapters. Regardless of the chosen citation style (e.g., ACS, AIP, AMS, IEEE, AIAA, etc) any reference {\bf must} contain the following fields: 
\begin{enumerate}
\item For journal papers: Authors, Paper Tittle, Journal Name, Volume, Pages, Year of publication;
\item For books: Authors, Book Tittle, Publisher, Year or Edition;
\item For book chapters: Authors, Chapter Tittle, Book Tittle, Editors, Publisher, Year or Edition;
\item For conference papers: Authors, Paper Tittle, Conference Tittle, Place (Country and/or City) where the conference was held, Year of the conference.
\end{enumerate}  
Thus, for example:

[3] P.L. Houtekamer and L. Mitchell (1998) Data Assimilation Using an Ensemble Kalman Filter Technique, {\it Monthly Weather Review}, 126:796-811, 1998.
%
\end{enumerate}

The topic of the paper, polymer flooding, is very relevant for the energy sector. Individual aspects have been investigated by a number of researchers (academics and industrial) worldwide with clear cross-fertilisation with physics/chemistry (material sciences, reactions, signal analysis, phase change, corrosion, etc), chemical / mechanical / electrical / petroleum engineering (industrial processing, safety, facilities, maintenance, etc), economy (project management, financial forecasting etc). The student demonstrated that he had understood the concepts involved in the technologies for the project with a solid comprehension on fundamentals of engineering and the impact on energy business.    


\clearpage

%%%%%%
%%%%%%
%%%%%%



\noindent{\bfseries\large EG5085 -- Advanced Topics for MEng Study $\left(\text{2}^{\text{nd}}\text{ Paper}\right)$\hfill January, 2014}

\bigskip

\begin{center}
{\Large Review of the 2$^{\text{nd}}$ Paper $\lq$The Use and Limitations of Composite Repairs in the Oil and Gas Industry' by Martin Irvane Murawiecki}
\end{center}

\medskip

The manuscript investigates the use of composite materials in repair procedures in hydrocarbon-related production facilities, in particular pipelines. The student studied three subject areas within the main topic (composite materials): surface treatment, performance and behaviour, industry standards (best-practice).

The paper is reasonably well-written with a small number of typos and unrevised sentences. A few sentences are very confusing and disconnected with no clear objectives. In general, the paper was interesting to read with enough content to enable discussion and analysis. A few general comments,
\begin{enumerate}
\item The main aim of {\it Abstracts} is to briefly describe the work undertaken by the author. In general {\it Abstracts} are divided in 4 parts: (i) motivation, (ii) main objectives, (iii) summary of the main procedures / techniques / technologies (optional) and (iv) main findings. The current {\it Abstract} encompasses (ii) and (iv).
%
\item The main {\it Introduction} section usually has the same (but more in-depth and descriptive) four parts of the {\it Abstract} and a brief summary of the remaining of the work. In addition, it is always expected a few clear statements -re main background (thus recent innovations related to the main topic), initial literature review and, most of all, technological / scientific gaps in the current understanding. The current {\it Introduction} section is well-written but lacks demonstration that the student investigated past work on the subject topics (e.g., surface and materials science, adhesion, etc). Also, it is expected a summary of the remaining sections. Most of all, it absolutely unclear the main objectives of the work.
%
\item The topic is very timely as there are several R$\&$D initiatives worldwide on surface and material sciences with applications spanning industries. Novel technologies that accurately address composite materials for repairs in pipelines are crucial for oil and gas industry, and several aspects on this topic were very superficially covered in the paper. 
%
\item Key properties definitions are missing and symbols/acronyms were used with no prior description. For example, matrices were classified as {\it thermosets} and {\it thermoplastics}, but no definition were given for any of them.
%
\item Figures and Tables {\bf must} be referenced in the main text -- they can not just $\lq$float around'! Also, figure/table captions should be self-contained, i.e., with a good description of the figure/table, highlighting the most relevant aspects/information that the author wants to convene. 
%
\item I guess the info contained in Tables 1-11 were copied from publicly available sources. If this was the case, the student should have cited the reference.
\end{enumerate}

The topic of the paper, composite materials, is very relevant for the energy and environmental sectors. Individual aspects have been investigated by a number of researchers (academics and industrial) worldwide with clear cross-fertilisation with physics/chemistry (material sciences, reactions, signal analysis, phase change, corrosion, etc), chemical / mechanical / electrical / petroleum engineering (industrial processing, safety, facilities, maintenance, etc), economy (project management, financial forecasting etc). The student demonstrated that he had understood the concepts involved in the technologies for the project with a solid comprehension on fundamentals of engineering and the impact on energy business.    


\clearpage

%%%%%%
%%%%%%
%%%%%%



\noindent{\bfseries\large EG5085 -- Advanced Topics for MEng Study $\left(\text{2}^{\text{nd}}\text{ Paper}\right)$\hfill January, 2014}

\bigskip

\begin{center}
{\Large Review of the 2$^{\text{nd}}$ Paper $\lq$Advanced Reservoir Management: History-Matching Workflow' by Ian William Mitchell}
\end{center}

\medskip

The paper investigates numerical/computational tools for history-matching for reservoir flow studies. The student studied three subject areas within the main topic (history-matching technologies): overall workflow for reservoir simulation, optimisation techniques and fluid flows in porous media.

The manuscript is reasonably well-written with a small number of typos and unrevised sentences. A few sentences are very confusing and disconnected with no clear objectives. In general, the paper was interesting to read with enough content to enable discussion and analysis. A few general comments,
\begin{enumerate}
\item The main aim of {\it Abstracts} is to briefly describe the work undertaken by the author. In general {\it Abstracts} are divided in 4 parts: (i) motivation, (ii) main objectives, (iii) summary of the main procedures / techniques / technologies (optional) and (iv) main findings. The current {\it Abstract} successfully encompasses all 4 parts.
%
\item {\bf All} journal paper {\it references} used in the paper are incomplete and/or wrong. Regardless of the chosen citation style (e.g., ACS, AIP, AMS, IEEE, AIAA, etc) any reference {\bf must} contain the following fields: 
\begin{enumerate}
\item For journal papers: Authors, Paper Tittle, Journal Name, Volume, Pages, Year of publication;
\item For books: Authors, Book Tittle, Publisher, Year or Edition;
\item For book chapters: Authors, Chapter Tittle, Book Tittle, Editors, Publisher, Year or Edition;
\item For conference papers: Authors, Paper Tittle, Conference Tittle, Place (Country and/or City) where the conference was held, Year of the conference.
\end{enumerate}  
Thus, for example:\\
\noindent
[39] P.L. Houtek and L. Mitchell, $\lq$Data Assimilation Using an Ensemble Kalman Filter Technique', American Meteorological Society, 1998.\\
\noindent
should read as,\\
\noindent
[39] P.L. Houtekamer and L. Mitchell, $\lq$Data Assimilation Using an Ensemble Kalman Filter Technique', {\it Monthly Weather Review}, 126:796-811, 1998.
%
\item The main {\it Introduction} section usually has the same (but more in-depth and descriptive) four parts of the {\it Abstract} and a brief summary of the remaining of the work. In addition, it is always expected a few clear statements -re main background (thus recent innovations related to the main topic), initial literature review and, most of all, technological / scientific gaps in the current understanding. The current {\it Introduction} section is well-written but lacks demonstration that the student investigated past work on the subject topics (e.g., optimisation techniques, flow simulation, Kalman filters, EnKF, data assimilation etc).
%
\item The topic is very timely as there are several R$\&$D initiatives worldwide on history matching and data assimilation aiming to forecast both hydrocarbon production (with updated potential reserves) and wells/fields performance. Novel technologies that accurately address the aforementioned technologies are crucial for reservoir management (including decision-making). Several aspects on these topics were covered in the paper, but an idealised or field example (also available in the literature) would help the reader to fully understand the subject.
%
\item The contents within the Sections $\lq$Extended Darcy Law' and $\lq$Reservoir Simulation' were crucial to fully understand and appreciate the work. Unfortunately, the former was very superficial and incomplete -- for example, the extended Darcy equation (i.e., the multiphase porous media flow) can only be fulfilled by complementing Eqn. (2) with the global mass conservation equation (i.e., saturation equation) that links fluid flux $\left(u,\;u_{\alpha}\;\rightarrow\;\text{velocity of phase }\alpha\right)$, saturations, densities, relative and absolute permeabilities. Neither the equations nor the physics associated with miscible/immiscible fluid flows were fully explained. Although well-written, the first two paragraphs of the $\lq$Reservoir Simulation' section were very superficial with continuous reference to computing power (and $\lq$improvements') without explaining which developments enabled the new high-performance computing (HPC) technologies.
%
\item I guess Fig. 1 was copied from some public available source. If this was the case, the student should have cited the reference. In any case, the $\lq${\it Eclipse Simulator}' should be replaced by $\lq${\it Simulator}', as {\it Eclipse} is a commercial simulator developed by Schlumberger. Also after the $\lq$time-step', the flow chart should indicate returning to the $\lq$Simulator'.
\end{enumerate}

The topic of the paper, history-matching technology, is very relevant for the energy and environmental sectors. Individual aspects have been investigated by a number of researchers (academics and industrial) worldwide with clear cross-fertilisation with mathematics (e.g., solution of partial differential equations, optimisation, etc), physics/chemistry (fluid mechanics, material sciences, reactions, signal analysis, phase change, etc), chemical / mechanical / electrical / petroleum engineering (reservoir management, industrial processing, safety, facilities, power grid, etc), economy (project management, financial forecasting etc). The student demonstrated that he had understood the concepts involved in the technologies for the project with a solid comprehension on fundamentals of engineering and the impact on energy business.    


\clearpage

%%%%%%
%%%%%%
%%%%%%

\begin{center}
\Huge{Advanced Topics for MEng Study (EG5085)}\\
\huge{1$^{st}$ Paper (Review + Feedback)}\\
\huge{November 2013}
\end{center}

\vfill

\clearpage


\noindent{\bfseries\large EG5085 -- Advanced Topics for MEng Study $\left(\text{1}^{\text{st}}\text{ Paper}\right)$\hfill November, 2013}

\bigskip

\begin{center}
{\Large Review of the 1$^{\text{st}}$ Paper $\lq$Optimisation of Energy Systems in Smart Cities' by Grant Milne}
\end{center}

\medskip

The paper aims to investigate optimal energy technologies for urban environments and, in particular co-/tri-generation systems (CHP and CCHP) and applications in cities. The student assessed four subject areas within the main topic (optimal energy usage in $\lq$smart cities'): co-/tri-generation systems, optimal energy integration, UK energy policy and applications of (C)CHP in British cities/towns.

The manuscript is reasonably well-written with a small number of typos and unrevised sentences. Some paragraphs are very confusing and disconnected with no clear objectives. In general, the paper was interesting to read with enough content to enable discussion, although with no in-depth data collection and analysis. A few general comments,
\begin{enumerate}
\item The main aim of {\it Abstracts} is to briefly describe the work undertaken by the author. In general {\it Abstracts} are divided in 4 parts: (i) motivation, (ii) main objectives, (iii) summary of the main procedures / techniques / technologies (optional) and (iv) main findings. The current {\it Abstract} only encompass (i) and (iii).
%
\item A large number of {\it references} used in the paper are from web-pages -- which are usually unreliable sources of information.
%
\item The main {\it Introduction} section usually has the same (but more in-depth and descriptive) four parts of the {\it Abstract} and a brief summary of the remaining of the work. In addition, it is always expected a few clear statements -re main background (thus recent innovations related to the main topic), initial literature review and, most of all, technological / scientific gaps in the current understanding. The current {\it Introduction} section is well-written but lacks demonstration that the student investigated past work on the subject topics (e.g., environmental impact of standard large-scale fossil-fuel plants and local (C)CHP, energy integration on thermal-power plants, etc).
%
\item Figures {\bf must} be referenced in the main text -- they can not just $\lq$float around'! Also, figure captions should be self-contained, i.e., with a good description of the figure, highlighting the most relevant aspects/information that the author wants to convene. 
%
\item The topic is very timely as there are several initiatives worldwide on energy optimisation to both mitigate GHG emissions and improve thermal efficiency at low cost. Several aspects on this topic were covered in the paper: (a) co-/tri-generation systems; (b) optimal energy integration; (c) EU/UK energy and environmental policies and regulations; (d) case-studies of optimal integrated energy systems. The paper superficially covered most of these aspects,
\begin{enumerate}
\item Energy generation systems: good overview of the refrigeration cycle but no attempt to describe the integrated cycles -- thermal and refrigeration (which was the main objective of the proposed work). These cycles are the thermal engineering core of the (C)CHP systems and should have been described and discussed. 
% 
\item Energy integration: ideally Section 3 should have covered energy and exergy analysis and the relationship with modern energy integration techniques (e.g., pinch and optimal design methods). This section should have focused on the thermal engineering aspects of the paper, leading to an in-depth understanding of the challenges involved on designing and optimisation of sustainable energy technologies. A very superficial and non-connected description of exergy and pinch technology was presented instead.   
%
\item EU/UK energy and environmental policies and regulations: the paper partially focused on cost regulations (in particular to economic costing barriers for the introduction of small electricity providers) and the impact of incentives and government subsidises (IGS) in the transmission sector (Section 2.2 and 3.3). No further analysis on costing and IGS for electricity and/or heat generation. Also, one of the main concepts behind $\lq$smart cities' is {\it sustainability} (i.e., bridging the gaps between new and innovative technologies for GHG mitigation, optimal use of energy, low wasting, thermal building efficiency, etc), which was not mentioned in the paper.
%
\item Section 5 (case-studies of locally generated power-heat) was very insightful and very well-researched with a clear vision of the potential of this technology to mitigate GHG emissions and lower electricity/heating costs whilst easing the loading in national power grid. This section briefly reviewed three cases in which (C)CHP were used in UK -- Sheffield, London (Pimlico) and Aberdeenshire, to generate electricity and/or heat.
\end{enumerate} 
\end{enumerate}

The topic of the paper, optimal energy integration, is very relevant for the energy and environmental sectors. Individual aspects have been investigated by a number of researchers (academics and industrial) worldwide with clear cross-fertilisation with mathematics (e.g., solution of partial differential equations, optimisation, etc), physics/chemistry (fluid mechanics, material sciences, reactions, signal analysis, phase change, etc), chemical/mechanical/electrical engineering (industrial processing, safety, facilities, power grid, etc), economy (project management, financial forecasting etc). The student demonstrated that he had understood the concepts involved in the technologies for the project with a solid comprehension on fundamentals of engineering and the impact on energy business.    


\clearpage


\begin{center}
\Huge{MSc Oil and Gas Engineering Dissertations (EG5908)}\\
\huge{(Review + Feedback)}\\
\huge{September 2013}
\end{center}

\vfill

\clearpage


\noindent{\bfseries\large MSc in Oil $\&$ Gas Engineering\hfill September, 2013}

\bigskip

\begin{center}
{\Large Review of the MSc Dissertation $\lq$Profitability Assessment of Gas to Power for National Grid, Gulf of Guinea, Nigeria' by Elefin Mayowa Olurunfemi}
\end{center}

\medskip

The dissertation assess the financial feasibility of an ambitious offshore-based energy project in Nigeria. The engineering aim of the project is to use Combined-Cycle Gas Turbine technology to produce electricity offshore from flaring gas. The produced electricity is then connected into the main Nigerian power grid. The dissertation focuses on the preliminary financial engineering for this proposed project with clear simplified assumptions.

The dissertation is well-written with very few typos. Most of all, it is very well-structured with clear division and linkages between chapters, sections and paragraphs, leading to an easy and smooth reading. A few general comments,
\begin{enumerate}
\item Overall presentation: (a) Headers of the first pages are missing, (b) tables in the appendix are larger than the size of the A4 page (therefore they are not readable) and (c) a few tables and figures and not numbered and referenced correctly. Although these do not impact in the quality and comprehensiveness of the work, it indicates a lack of care with the appearance.  
\item KV $\rightarrow$ kV; and KW $\rightarrow$ kW.
\item In general, the literature review was good, although I would expect further analysis and comparison with similar initiatives to either diversify the energy matrix or to introduce new energy sources.
\item Equations in Chapter 4 were not correctly numbered (and referenced in later Sections). In addition, the definition of IRR introduced in Section 4.2.6 (Eqn.1.5) is not correct.
\item In Table 5.2, it seems that Year 0 (2015) is missing, and most of the discussion in first paragraph of Section 5.1.2 is based upon Year 0. 
\end{enumerate}

My main (and maybe only) criticism of the work is that there is no analysis on the consumption market for electricity -- the author assumed that there is a natural demand for power in Nigeria, but no numbers (or time series indicating the growth of consumption) were shown. In the analysis, he assumed that all electricity generated by the proposed power plant would be entirely used. 
%{\it My main concern on recommending this work for publication is that there is no clear indication of the authors' contribution to either particle science and technology or numerical/computational formulation and methods}. Most of the work (if not all) is available in particle technology textbooks and/or specialised journals (e.g., Powder Technology, International Journal of Multiphase Flow, Numerical Heat Transfer etc). %Therefore I would suggest that the authors revise the manuscript and resubmit.

\clearpage



\noindent{\bfseries\large MSc in Oil $\&$ Gas Engineering\hfill September, 2013}

\bigskip

\begin{center}
{\Large Review of the MSc Dissertation $\lq$Enhanced Oil Recovery' by Aleksandr Poljakov}
\end{center}

\medskip

This dissertation aims to review past and current technologies for oil recovery and, in particular, for EOR. The student presented a short review on the main technical factors that impact in the fluid flow behaviour under reservoir conditions. The primary engineering aim of the project is to review EOR techniques and to assess the business case.

The manuscript is reasonably well-written with a number of typos and unrevised sentences. Some paragraphs/sections are very confusing and with no clear objectives. A few general comments,
\begin{enumerate}
\item The main aim of {\it Abstracts} is to briefly describe the work undertaken by the author. In general {\it Abstracts} are divided in 4 parts: (i) motivation, (ii) main objectives, (iii) summary of the main procedures / techniques / technologies (optional) and (iv) main findings. The current {\it Abstract} only encompass (ii) (in part) and (iv).
\item Main {\it Introduction} sections usually have the same (but more in-depth) four parts of the {\it Abstract} and a brief description of the remaining of the work. In addition, it is always expected a few clear statements -re main background (thus recent innovations related to the main topic), initial literature review and, most of all, technological / scientific gaps in the current understanding. The current {\it Introduction} section is reasonably well-written but lacks (i) connectivity between paragraphs and (ii) demonstration of deep understanding of underlying issues on EOR that led the student to engage into the work.
\item It is customary for Dissertations to include a list of figures and tables. In addition, they are always divided in chapters rather than  sections. Finally, as stated in the Guidelines, equations must be numbered.
\end{enumerate}

The topic is very relevant for the O$\&$GE (and energy) sector and has been the main focus of several academic- and industrial-based studies worldwide with clear cross-fertilisation with other environmental (e.g., CCS) and energy (e.g., similarities with UCG for in-situ combustion) areas. The student demonstrated a sound understanding of the main technologies used in EOR but with a superficial discussion on both, fundamentals and impact on O$\&$G exploration business.    



\clearpage

\noindent{\bfseries\large MSc in Oil $\&$ Gas Engineering\hfill September, 2013}

\bigskip

\begin{center}
{\Large Review of the MSc Dissertation $\lq$Assessment of Reservoir Simulators: Workflow and Quality Assurance' by Mohamed Sherif Elkiki}
\end{center}

\medskip

This dissertation reviews current technologies used by the oil and gas industry to predict reservoir production behaviour and performance. The student divided the reservoir simulation workflow in eight stages, from core and well testing to history matching procedures. The primary engineering aim of the project is to review the major technologies in the workflow in a (as much as it is possible) comprehensive and interconnected way. 

The manuscript is reasonably well-written with a number of typos and unrevised sentences. Some paragraphs are very confusing and disconnected with no clear objectives. A few general comments,
\begin{enumerate}
\item The main aim of {\it Abstracts} is to briefly describe the work undertaken by the author. In general {\it Abstracts} are divided in 4 parts: (i) motivation, (ii) main objectives, (iii) summary of the main procedures / techniques / technologies (optional) and (iv) main findings. The current {\it Abstract} only encompass (ii) (in part), (iii) and (iv) (also in part).
\item The {\it References} follow different standards with missing fields.
\item The main {\it Introduction} section usually has the same (but more in-depth and descriptive) four parts of the {\it Abstract} and a brief summary of the remaining of the work. In addition, it is always expected a few clear statements -re main background (thus recent innovations related to the main topic), initial literature review and, most of all, technological / scientific gaps in the current understanding. The current {\it Introduction} section is reasonably well-written but lacks demonstration that the student investigated past work on the subject(s). Also, it is expected a summary of the remaining chapters at the end of the section.
\item A {\it Nomenclature} table should contain (most of) all symbols (and units) used in the work. Several symbols were used throughout the text with no prior definition (e.g., $Z=\rho v$ in the top of page 18).
\item Captions of figures are in different colour, font and font size from the main text. In addition, they are a very poor description of the figure (i.e., not self-contained). Also some figures exceeded the maximum page size.
\item Tables must be allocated in a single page, and should not (except in very specific cases) span over 2-3 pages. Finally, as stated in the Guidelines, equations must be numbered.
\item Chapter 3 -- {\it Seismic Analysis} was a very insightful chapter with excellent linkage with the remaining of the work and, most of all, brought to light an exciting and evolving technology. However, the chapter lacked a (in-depth) description/discussion of seismic analysis, i.e., production of waves, physics of reflection, data collection, processing (i.e., signal processing and inverse theory) and analysis. Also, Chapter 4 -- {\it Grid Creation} was very informative, but some crucial information was missed. For example, what is the difference between structured and unstructured grid? Also, the chapter lacked the important transition from map and grid. 
\item Appendices are used to convey complementary (and not crucial) information of the main chapters and need to be referenced in the main text.
\end{enumerate}

The topic is very relevant for the O$\&$GE (and energy) sector and each chapter has been the focus of several academic- and industrial-based studies worldwide with clear cross-fertilisation with mathematics (e.g., inverse theory, mesh generation, solution of partial differential equations, uncertainty quantification etc), physics (fluid and solid mechanics, signal processing, etc), geology $\&$ geophysics (e.g., lithography, petrology, geochemistry, etc) and computer science (e.g., software engineering, algorithms, parallel processing, etc). The student demonstrated a good understanding of the main available technologies but with a superficial discussion on fundamentals, engineering and the impact on O$\&$G exploration business.    


\clearpage

\noindent{\bfseries\large MSc in Oil $\&$ Gas Engineering\hfill September, 2013}

\medskip

\begin{center}
{\Large Review of the MSc Dissertation $\lq$Virtual Training for O$\&$G Industry: Assessing Existing and Upcoming Technologies in LNG Plants' by Christoforos Constantinou}
\end{center}

\medskip

The dissertation focuses in twofolds topics, review of current technologies LNG plants and training solutions (i.e., operations and safety) using virtual environments. The student investigated the complete workflow for LNG: natural gas production, purification, liquefaction, transport and regasification on on-/off-shore plants. He also preliminary investigated current and future financial feasibility of LNG expansion. Finally, the workflow was subdivided to propose virtual reality training modules and facilities, focusing on operations and safety.

The manuscript is reasonably well-written with a large number of typos and unrevised sentences. Some paragraphs and sections are very confusing and disconnected with no clear objectives and inter-connectivities. A few general comments,
\begin{enumerate}
%
\item The main aim of {\it Abstracts} is to briefly describe the work undertaken by the author. In general {\it Abstracts} are divided in 4 parts: (i) motivation, (ii) main objectives, (iii) summary of the main procedures / techniques / technologies (optional) and (iv) main findings. The current {\it Abstract} only (partially) encompass (iii) and (iv).
%
\item The {\it References} follow different standards with missing fields and no clear distinction between articles, conference proceedings, reports (internal or external), book chapters, books, communications (internal or external) etc.  
%
\item The main {\it Introduction} section usually has the same (but more in-depth and descriptive) four parts of the {\it Abstract} and a brief summary of the remaining of the work. In addition, it is always expected a few clear statements -re main background (thus recent innovations related to the main topic), initial literature review and, most of all, technological / scientific gaps in the current understanding. The current {\it Introduction} section is reasonably well-written but lacks demonstration that the student investigated past work on the subject (i.e., LNG production technology and the use of virtual reality as an industrial training environment). Also, it is expected a summary of the remaining chapters at the end of the section. Most of all, it absolutely unclear the main objectives of the work -- $\lq$The main purpose of this project is to investigate and provide techniques to a company called LanguageLab'.
%
\item Most of the figures are not referenced in the main text and are of very poor quality (most of them are nearly unreadable).
%
\item As the whole project gravitates around LNG technologies, however there was no physical explanation (basic thermodynamics) of LNG processing (or the liquefaction cycle). The description of one of the main technology (Section 3.1) was very superficial -- I would expect a larger overview of the main technologies.  
%
\item The second major focus of the dissertation is the virtual reality as a safe training environment for O$\&$G (and LNG) industry personnel. The description of the main VR technologies are very superficial (Sections 6.1-2) and the student has not demonstrate an understanding of how simulations (or the virtual reality or simulated environment) work and can be manipulated for training purposes. However, as a positive aspect, it is very clear how he (and LanguageLab) envisaged to use the technology to support specialised training and, most of all, designed specific scenarios that can be potentially useful for the industry sector.   
%
\item However, this was not clearly highlighted in his conclusion section.
%
\end{enumerate}

The topic of the dissertation is very relevant for the O$\&$GE (and energy) sector. The two main subjects -- processing and transport of LNG and industrial applications for VR have been the focus of several academic- and industrial-based studies worldwide with clear cross-fertilisation with mathematics (e.g., CFD, mesh generation, solution of partial differential equations, non-linear optimisation, uncertainty quantification etc), physics (fluid mechanics, material sciences, etc), geology $\&$ geophysics (e.g., lithography, petrology, geochemistry, etc), computer science (e.g., software engineering, algorithms, parallel processing, artificial intelligence etc) and chemical engineering (industrial processing, safety, etc). The student demonstrated that he had understood the basic concepts involved in the technologies for the project, but with a superficial discussion on fundamentals, engineering and the impact on LNG business.    


\clearpage

\noindent{\bfseries\large MSc in Oil $\&$ Gas Engineering\hfill September, 2013}

\medskip

\begin{center}
{\Large Review of the MSc Dissertation $\lq$The Liberation of Volatiles from Coal under Temperature Controlled Conditions in an Anaerobic Atmosphere' by Craig Donaldson}
\end{center}

\medskip

The dissertation aims to review current unconventional coal technologies and, in particular, to assess the feasibility of {\it in situ} low-temperature coal pyrolysis to recover remaining energy and chemicals (volatiles) stored in underground deposits. The engineering aim of the dissertation is to introduce the preliminary design of the process based on data collected during the literature review. The proposed novel process was introduced as a hybrid of current CBM and UCG technologies and, due to the lack of data on the prescribed conditions, a lab-scale model was proposed.

The dissertation is very well-written with very few typos. Most of all, it is very well-structured with clear division and linkages between chapters, sections and paragraphs, leading to an easy and smooth reading. A few general comments,
\begin{enumerate}
%
\item The {\it overall presentation} was excellent with clear figures with comprehensive and self-contained captions. However, a couple of them are of poor quality. 
%
\item Chapter 2 -- {\it Background Theory}, introduced an excellent overview on basic coal chemistry, geological origins and occurrence (UK and overseas). However, the chapter lacked a further analysis on future trends on coal market and an initial financial analysis on cost of the volatiles in comparison with the current commodities market.  
%
\item Chapters 3 and 4 are really good with a very smooth transition from current unconventional coal technologies to the proposed one based on low-temperature pyrolysis. However, there was no formal definition (in any chapter) of the pyrolysis process. I can understand that there was limited time for the dissertation, but I believe a small section/paragraph on fundamentals of pyrolysis (as this is a very well-studied topic -- also stated in Chp. 4) would be useful.  
%
\end{enumerate}
My main (and maybe only) criticism of the work is on Chapter 5. The contents of Appendix A should be part of this chapter as the numerical solution is important for the understanding of Chapter 7.  Additionally, I would expect a more formal model and that the assumptions are clearly stated, e.g., incompressible and inviscid fluid flow with constant thermo-physical solid properties. Also the result should be validated against the multiphase thermal equation (instead of the proposed global equation),
\begin{equation}
\displaystyle\frac{\partial}{\partial t}\left(C_{k}\varepsilon_{k}\rho_{k}T_{k}\right) = -p_{k}\nabla\left(\varepsilon_{k}v_{k}\right) + \nabla \cdot \left(\varepsilon_{f}\kappa_{k}\nabla T_{k}\right) + \alpha\left(T_{k'}-T_{k}\right) + \Omega_{wk}\label{thermal} 
\end{equation}
where $C_{k}$, $\varepsilon_{k}$, $\rho_{k}$, $\kappa_{k}$ and $v_{k}$ are the heat capacity, volume fraction, density, thermal conductivity and velocity of phase $k$ (solid or fluid). $\alpha$ and $\Omega_{wk}$ are the volumetric interphase and wall-phase heat transfer coefficients. This equation can be simplified (assuming incompressible and inviscid flows with constant granular properties, no thermal source or sink terms and constant fluid velocity and inlet temperature) to, 
\begin{equation}
\displaystyle\frac{\partial}{\partial t}\left(\eta T_{f}\right)+ \psi\left(T_{f}-T_{\text{(inlet)}}\right) = 0\label{simple}
\end{equation} 
with
\begin{displaymath}
\eta=\rho_{s}C_{s}V_{s}+\rho_{f}C_{f}V_{f} \;\;\;\;\text{ and } \;\;\;\; \psi=\rho_{f}C_{f}v_{f}\overline{A}
\end{displaymath}
where $V_{k}$ and $\overline{A}$ are the volume of phase $k$ and the cross-section area, respectively. Equation \ref{simple} has an exponential (in time) analytical solution. Alternatively, Eq. \ref{thermal} could be easily solved (with similar assumptions as before) in 1D for $T\left(x,t\right)$.

The topic of the dissertation is very relevant for the O$\&$GE (and energy) sector. The two main subjects -- coal technology and pyrolysis have been the focus of several academic- and industrial-based studies worldwide with clear cross-fertilisation with mathematics (e.g., CFD, solution of partial differential equations, etc), physics (fluid mechanics, material sciences, etc), geology $\&$ geophysics (e.g., lithography, petrology, geochemistry, etc) and chemical engineering (industrial processing, safety, etc). The student demonstrated that he had an excellent understanding of the main fundamental physical and engineering concepts involved in the technologies for this project.    




\clearpage

\noindent{\bfseries\large MSc in Oil $\&$ Gas Engineering\hfill September, 2013}

\medskip

\begin{center}
{\Large Review of the MSc Dissertation $\lq$Thermodynamic Analysis of Clathrate Hydrates Formation and Stability' by Warinthon Lertpornsuksawat}
\end{center}

\medskip

The dissertation aims to assess the current understanding of formation and thermodynamic stability of hydrates in natural gas. The student investigated the typical workflow for thermodynamic equilibrium analysis of clathrate of hydrates: chemistry of lattice (basic quantum mechanics), multi-component and multiphase equilibrium conditions (i.e., problem formulation) and optimisation problem. She also studied a few deterministic and stochastic optimisation techniques currently used in engineering and financial problems.

The manuscript is reasonably well-written with a large number of typos and unrevised sentences. Some paragraphs and sections are very confusing and disconnected with no clear objectives and inter-connectivities. A few general comments,
\begin{enumerate}
%
\item The main aim of {\it Abstracts} is to briefly describe the work undertaken by the author. In general {\it Abstracts} are divided in 4 parts: (i) motivation, (ii) main objectives, (iii) summary of the main procedures / techniques / technologies (optional) and (iv) main findings. The current {\it Abstract} only (partially) encompass (i), (iii) and (iv).
%
\item The {\it References} follow different standards with missing fields and no clear distinction between articles, conference proceedings, reports (internal or external), book chapters, books, communications (internal or external) etc.  
%
\item The main {\it Introduction} section usually has the same (but more in-depth and descriptive) four parts of the {\it Abstract} and a brief summary of the remaining of the work. In addition, it is always expected a few clear statements -re main background (thus recent innovations related to the main topic), initial literature review and, most of all, technological / scientific gaps in the current understanding. The current {\it Introduction} section is reasonably well-written but lacks demonstration that the student investigated past work on the subject (i.e., hydrate thermodynamic formulations other than Sloan and Ballard). Also, it is unclear what the main objectives of the work are.
%
\item A few figures are not referenced in the main text.
%
\item Equations are of very poor quality.
%
\item A good chunk of Chapter 2 ({\it PVT Behaviour of Hydrates: Thermodynamics Stability and formation}) focused on EOS of hydrates in the lattice cavities (i.e, statistical state of empty and filled cavities). However the dissertation did not discussed the (also important) EOS for the guest molecules (also used in Ballard and Sloan work).  
%
\item Chapter 3 ({\it Free Gibbs Energy Formulation} is very insightful and with a good explanation of the mathematical formulation of thermodynamics extensive properties. However it was not clear the link between the free Gibbs energy and Eqns. 3.84-5. The latter is not the Gibbs free energy, but a functional that represents the residual of the variation of $G$.
%
\item Definitions of stochastic and deterministics optimisation methods are wrong. 
%
\end{enumerate}

The topic of the dissertation is very relevant for the O$\&$GE (and energy) sector. The two main subjects -- thermodynamics formulation for hydrate formation and stability and optimisation techniques  have been the focus of several academic- and industrial-based studies worldwide with clear cross-fertilisation with mathematics (e.g., non-linear optimisation, Monte Carlo methods, solution of partial differential equations, etc), physics/chemistry (fluid mechanics, material sciences, quantum mechanics, etc), computer science (e.g., software engineering, algorithms, parallel processing, artificial intelligence etc) and chemical engineering (industrial processing, safety, flow assurance etc). The student demonstrated that she had understood the concepts involved in the technologies for the project with a basic comprehension on fundamentals of engineering and chemistry, and the impact on O$\&$G business.    


\clearpage


\noindent{\bfseries\large MSc in Oil $\&$ Gas Engineering\hfill September, 2013}

\medskip

\begin{center}
{\Large Review of the MSc Dissertation $\lq$Numerical Investigation of Compositional Flows in Porous Media' by Oluwatosin Anuluwapo}
\end{center}

\medskip

The dissertation investigates the use of numerical models to simulate compositional flows in homogeneous porous media. The student assessed and compared the individual components of commercial- and academic-based flow simulators: mesh characteristics, allocation of rock and fluid properties in the grid, discretisation and solver methods for PDEs. She also investigated methods used by Eclipse to solve multi-component and multiphase flows in porous media.

The manuscript is reasonably well-written with a small number of typos and unrevised sentences. A few paragraphs and sections are very confusing and disconnected with no clear objectives. A few general comments,
\begin{enumerate}
%
%\item The main aim of {\it Abstracts} is to briefly describe the work undertaken by the author. In general {\it Abstracts} are divided in 4 parts: (i) motivation, (ii) main objectives, (iii) summary of the main procedures / techniques / technologies (optional) and (iv) main findings. The current {\it Abstract} only (partially) encompass (i), (iii) and (iv).
%
\item The {\it References} follow different standards with missing fields and no clear distinction between articles, conference proceedings, reports (internal or external), book chapters, books, communications (internal or external) etc.  
%
%\item The main {\it Introduction} section usually has the same (but more in-depth and descriptive) four parts of the {\it Abstract} and a brief summary of the remaining of the work. In addition, it is always expected a few clear statements -re main background (thus recent innovations related to the main topic), initial literature review and, most of all, technological / scientific gaps in the current understanding. The current {\it Introduction} section is reasonably well-written but lacks demonstration that the student investigated past work on the subject (i.e., hydrate thermodynamic formulations other than Sloan and Ballard). Also, it is unclear what the main objectives of the work are.
%
\item A few figures are not referenced in the main text.
%
\item A few equations are not correctly labelled.
%
\item Pages 35 and 36 are missing.
%
\item All flow simulators, regardless the main application (e.g., porous media, pipes, shallow waters, vessel and reactors, atmospheric, ocean circulation, etc) and the designed solving method (FDM, FVM, FEM and/or their hybrids) have very similar workflow: 
\begin{enumerate}
\item geometry setup (pre-processing);
\item \label{mesh}dimension setup $\longrightarrow$  choice of mesh grid type (quads, triangles, hex, tets, etc) $\longrightarrow$  mesh generation (pre-processing);
\item \label{pde}choice of the PDEs that will be solved (i.e., parabolic, hyperbolic or elliptic) and complexity (simplified or 'full-blow' equations);
\item \label{allocation}allocation of material and system properties (saturation, volume and mass/molar fractions, temperature, pressure, etc) in the mesh. This will depend on the mesh type, corner- or centre-point (ans the derived families);
\item \label{choice}choice of:
\begin{enumerate}
\item \label{disc}spatial- and time-discretisation schemes (application dependent);
\item \label{solver}pre-conditioners and linear solvers (for the resulting set of system of algebraic equations) -- dependent of the type of PDEs solved, discretisation schemes, single or multiple (parallel) processors;
\end{enumerate}
\item initial and boundary conditions setup.
\end{enumerate}
A good review of main (academic and commercial) simulators (Section 2.4) should cover \ref{mesh}-\ref{choice}, as these will be pivotal for the accuracy and performance of any flow simulator.
%
\item Several terms and methods (Chapter 2) were not explained, e.g., dual porosity dual permeability (DPDP), fractured multimodal porous media (FMPM), maximum reservoir contacts (MRC), actual difference between Eclipse and Intersect, streamline simulation, $\lq${\it benchmark of simulation}' etc.
%
\item In multiphase models, a cubic EOS (Section 2.2.2.6) is defined in terms of the  compressibility factor $\left(Z\right)$ as,
\begin{displaymath}
Z^{3}+C_{0}Z^{2}+C_{1}Z+C_{2}=0
\end{displaymath}
where $C_{j}$ are coefficients that arise from the specific EOS, thermo-physical properties of the fluids, temperature and pressure. This equation can be solved either numerically or analytically, and has 3 roots (real and/or complex). In VLE problems, the largest real (and positive) root is compressibility factor of the vapour phase whereas the smallest real (and positive) root represents the liquid phase. $Z_{k}$ ({\it k} is the phase) is then used to fugacities and chemical potentials (i.e., Gibbs free energy) for each component distributed in all phases. 
%
\item Time-discretisation scheme is crucial for solving the extended Darcy law and was not described / explained. Also, the boundary conditions (for pressure, saturation and components) were not described.
%
\item Equations 2.9-10 -- governing equations for the multiphase Darcy flows are wrong.
%
\item This may be in the missing pages, but I would expect a full explanation of the multi-components equations (for the full blow and the reduced models) as this is the focus point of the dissertation.  
%\item Definitions of stochastic and deterministics optimisation methods are wrong. 
%
\end{enumerate}

The topic of the dissertation is very relevant for the O$\&$GE (and energy) sector. The two main subjects -- multiphase and multi-component flows in porous media and flow simulators have been the focus of several academic- and industrial-based studies worldwide with clear cross-fertilisation with mathematics (e.g., solution of partial differential equations, algebraic topology, computational methods, optimisation, etc), physics/chemistry (fluid mechanics, material sciences, thermodynamics, etc), computer science (e.g., software engineering, algorithms, parallel processing, etc) and chemical/petroleum engineering (industrial processing, safety, oil recovery, CCS,  etc). The student demonstrated that she had understood the concepts involved in the technologies for the project with a basic comprehension on engineering $\&$ physics fundamentals, and the impact on O$\&$G business.    




\clearpage

\noindent{\bfseries\large MSc in Oil $\&$ Gas Engineering\hfill September, 2013}

\medskip

\begin{center}
{\Large Review of the MSc Dissertation $\lq$Creating Artificial Gas Storage in the Rock Salt Layers by Application International Experience, Technical and Financial Costs Estimation and Ways of Costs Decreasing' by Farhad Akbarov}
\end{center}

\medskip

The dissertation describes the procedure to develop an underground gas storage facility in a leached rock-salt cavity. The project focuses in three aspects:
\begin{itemize}
\item geological formation and feasibility for storage purposes;
\item preliminary (basic) engineering design, including potential and practical safety procedures;
\item simplified financial engineering analysis.
\end{itemize} 
The student investigated the workflow for UGS development in Azerbaijan, including good engineering practice during cavities' creation and operation. He also suggested financial alternatives for gas supplies and caverns' creation.

The manuscript is reasonably well-written with a large number of typos and unrevised sentences. Some paragraphs and sections are very confusing and disconnected with no clear objectives. A few general comments,
\begin{enumerate}
%
\item The main aim of {\it Abstracts} is to briefly describe the work undertaken by the author. In general {\it Abstracts} are divided in 4 parts: (i) motivation, (ii) main objectives, (iii) summary of the main procedures / techniques / technologies (optional) and (iv) main findings. The current {\it Abstract} only (partially) encompass (ii) and (iv).
%
\item The {\it References} follow different standards with missing fields and no clear distinction between articles, conference proceedings, reports (internal or external), book chapters, books, communications (internal or external) etc.  Additionally, most of the references used in the dissertation are from web-pages -- which are usually considered as unreliable source of information. As one of the objectives of the dissertation (also stated in the tittle) is to investigate international experiences on development and operation of UGS sites, I would expect a detailed literature review on past published work on the subject, e.g., geology and geophysics related to rock salt layers, structural solid mechanics related to underground cavities, geochemistry (water and salt formation/dissolution and reaction with hydrocarbons), drilling and well engineering challenges, automated and continuous monitoring of critical parameters (e.g., pressure, temperature, hydrates, stress, leakage etc), etc.  
%
\item The main {\it Introduction} section usually has the same (but more in-depth and descriptive) four parts of the {\it Abstract} and a brief summary of the remaining of the work. In addition, it is always expected a few clear statements -re main background (thus recent innovations related to the main topic), initial literature review and, most of all, technological / scientific gaps in the current understanding. The current {\it Introduction} section is reasonably well-written but lacks demonstration that the student investigated past work on the subject.
%
\item A few figures are not referenced in the main text. All Tables (except one in page 24) are located in the appendix. However Table 1 appears twice with different contents.
%
\item A few equations are not correctly labelled. And some terms were not defined.
%
\item Equations are of very poor quality.
%
\end{enumerate}

The topic of the dissertation is very relevant for the O$\&$GE (and energy) sector. The main subject -- development and operation of UGS is a relatively new application field. However, individual aspects have been investigated by a number of researchers (academics and industrial) worldwide with clear cross-fertilisation with mathematics (e.g., optimisation, Monte Carlo methods, solution of partial differential equations, inverse methods, etc), physics/chemistry (fluid mechanics, material sciences, reactions, signal analysis, etc), chemical/petroleum/mechanical/mining engineering (industrial processing, safety, flow assurance, etc), economy (financial maths, market evaluation, project management, financial forecasting, econometrics etc). The student demonstrated that he had understood the concepts involved in the technologies for the project with a basic comprehension on fundamentals of engineering and chemistry, and the impact on O$\&$G business.    




\clearpage

\noindent{\bfseries\large MSc in Oil $\&$ Gas Engineering\hfill September, 2013}

\medskip

\begin{center}
{\Large Review of the MSc Dissertation $\lq$Topsides Sand Handling' by Francis Okeke}
\end{center}

\medskip

The dissertation describes the workflow for sand production in hydrocarbons exploration. The student reviewed:
\begin{itemize}
%
\item impact of sand in industrial facilities;
%
\item methods to control/mitigate sand production;
%
\item fluid-solid separation methods.
%
\end{itemize}

The manuscript is reasonably well-written with a number of typos and unrevised sentences. A few paragraphs and sections are very confusing and disconnected with no clear objectives. A few general comments,
\begin{enumerate}
%
\item The main aim of {\it Abstracts} is to briefly describe the work undertaken by the author. In general {\it Abstracts} are divided in 4 parts: (i) motivation, (ii) main objectives, (iii) summary of the main procedures / techniques / technologies (optional) and (iv) main findings. The current {\it Abstract} only (partially) encompass (i-iii).
%
\item The {\it References} follow different standards with missing fields and no clear distinction between articles, conference proceedings, reports (internal or external), book chapters, books, communications (internal or external) etc.  In fact, most of the {\it references} are incomplete and could not be checked.  
%
\item The main {\it Introduction} section usually has the same (but more in-depth and descriptive) four parts of the {\it Abstract} and a brief summary of the remaining of the work. In addition, it is always expected a few clear statements -re main background (thus recent innovations related to the main topic), initial literature review and, most of all, technological / scientific gaps in the current understanding. The current {\it Introduction} section is reasonably well-written but lacks demonstration that the student investigated past work (or current available technologies) on the subject.
%
\item Figures {\bf must} be referenced in the main text -- they can not just $\lq$float around'! Also, figure captions should be self-contained, i.e., with a good description of the figure, highlighting the most relevant aspects/information that the author wants to convene. 
%
\item Appendices {\bf must} have tittles and they are not designed to hold figures and tables that do not fit in the main text. They are used as a complementary source of information (though non critical) that readers can use for a full understanding of the subject. Most of all, figures and tables in appendices {\bf must} be referenced in the main text. 
%
\item Empirical correlations are used in most (if not all) engineering applications and they are designed to be used in specific situations and conditions -- range of: particle diameter, fluid/particle velocities (or flow rates), thermo-physical properties etc. As the student included a few correlations (Eqns. 3.2-6) that are used to investigate sand erosion rate, I would expect (a) range of validity and  (b) accuracy of the correlation (or at least a valid reference).
%
\item The description and assessment of fluid-solid separation techniques seem overly superficial. The discussion of engineering/physics principles involving pressure- (or centripetal force-) or gravity-based separators were non-existent. The main criteria for performance analysis for both system are terminal and settling velocities, and none of them were addressed.
%
\item The critical analysis of fluid-solid separators undertaken by the student was very interesting and insightful. A few points though (bear in mind it is not a biased opinion as I have worked with gravity settlers, (hydro)cyclones and press-filters in the past): (a) cyclones and filters are generally more expansive as they require power to operate (set of pumps) and continuous maintenance, (b) gravity settlers are usually operated over long periods of time and require more volumetric space (although pre-separators and chemicals could be used to improve efficiency), (c) there is not much difference on the environmental impact for all of them (except for the noise in cyclones and press-filters).   
%
\end{enumerate}

The topic of the dissertation, management of particulates (sand) in hydrocarbon production fields, is very relevant for the O$\&$GE (and energy) sector. Individual aspects have been investigated by a number of researchers (academics and industrial) worldwide with clear cross-fertilisation with mathematics (e.g., solution of partial differential equations, optimisation, etc), physics/chemistry (fluid mechanics, material sciences, reactions, signal analysis, etc), chemical/petroleum/mechanical/mining engineering (industrial processing, safety, flow assurance, etc), economy (project management, financial forecasting etc). The student demonstrated that he had understood the concepts involved in the technologies for the project with a basic comprehension on fundamentals of engineering and the impact on O$\&$G business.    



\clearpage

\noindent{\bfseries\large MSc in Oil $\&$ Gas Engineering\hfill September, 2013}

\medskip

\begin{center}
{\Large Review of the MSc Dissertation $\lq$An Examination of the Nature and Performance of Drilling Fluids in Recent Use' by Enefiok Peter Effiong}
\end{center}

\medskip
The dissertation outlines drilling fluids (types, properties etc) used in O$\&$G and CBM industry worldwide. The project particularly focuses on (relatively) recent industrial experience on drilling and on the developments of reservoir-tailored drilling fluids (DF). The student investigated proprietary  DF's properties and their main field applications.  

The manuscript is reasonably well-written with a large number of typos and unrevised sentences. Some paragraphs and sections are very confusing and disconnected with no clear objectives. A few general comments,
\begin{enumerate}
%
\item The main aim of {\it Abstracts} is to briefly (and clearly) describe the work undertaken by the author. In general {\it Abstracts} are divided in 4 parts: (i) motivation, (ii) main objectives, (iii) summary of the main procedures / techniques / technologies (optional) and (iv) main findings. 
%
\item The {\it References} follow different standards with missing fields and no clear distinction between articles, conference proceedings, reports (internal or external), book chapters, books, communications (internal or external) etc.  Additionally, most of the references used in the dissertation are from web-pages -- which are usually considered as unreliable source of information. As one of the objectives of the dissertation (also stated in the tittle) is to investigate DFs, I would expect an initial overview of the chemistry and surface chemistry on the several families of DF (water-, oil, synthetic-, air-based, etc) and the interaction with the solids (rocks, cuttings, etc) and fluids (hydrocarbons, water, additives etc), followed by a formal classification wrt drilling field type.
%
\item The main {\it Introduction} section usually has the same (but more in-depth and descriptive) four parts of the {\it Abstract} and a brief summary of the remaining of the work. In addition, it is always expected a few clear statements -re main background (thus recent innovations related to the main topic), initial literature review and, most of all, technological / scientific gaps in the current understanding. Current {\it Introduction} and {\it Literature Review} sections are reasonably well-written but lacks demonstration that the student investigated past work on the subject -- just relying on commercial material.  There are plenty of SPE material on drilling fluids available.
%
\item In Chapter 4 (also in 3), the student used the Grunberg-Nissan (GN) Equation,
\begin{displaymath}
\ln \mu_{\text{mixture}} = \sum\limits_{i=1}^{2} x_{i}\ln\mu_{i} + x_{1}x_{2}\gamma_{12} 
\end{displaymath} 
$\gamma_{12}$ is a well-known fitting parameter that represents the dependency of the bulk viscosity on the composition and temperature, i.e., 
\begin{displaymath}
\gamma_{ij}=\gamma_{ij}\left(x_{i},x_{j}, T\right)
\end{displaymath} 
Assuming that both chemical species are in thermal equilibrium. The interpolation procedure in Chp. 4 is used across fields to find linearly dependent parameters, such as $\gamma$ (see Totten {\it et al.} (2003) 'Fuels and Lubricants Handbook: Technology, Properties, Performance, and Testing').
%
\end{enumerate}

The topic of the dissertation is very relevant for the O$\&$GE (and energy) sector. The main subject -- review of the performance of DF,  has been investigated by a number of researchers (academics and industrial) worldwide with clear cross-fertilisation with physics/chemistry (fluid mechanics, material sciences, reactions, organic formulations and synthesis, etc), chemical/petroleum/mechanical/mining engineering (industrial processing, safety, flow assurance, drilling, etc) and economy (market evaluation, project management, etc). The student demonstrated that he had understood the concepts involved in the technologies for the project with a basic comprehension on fundamentals of engineering and physics, and the impact on O$\&$G business.    

\vfill

%%%%%%
%%%%%%
%%%%%%


\clearpage

\noindent{\bfseries\large General Comments and Tips for MSc and MEng Thesis \hfill \today}

 
\begin{enumerate}
%
\item Dissertations and thesis are always divided into chapters rather than sections (commonly used in reports). 
%
\item Text becomes increasingly more readable if it is clearly divided (and numbered) into sections, subsections etc.  
%
\item In the begining of each chapter you should include a paragraph (or two) summarising previous relevant chapters and, most of all, the main aspects of the current chapter. This summary should indicate what the reader should expect from the following sections and how the chapter relates to previous chapters and to the overall thesis' subject.
%
\item Similarly, at the end of each chapter, it is expected a short section summarising the main aspects/results/conclusions of the chapter and how this can be linked with the overall thesis' subject and the following chapter.
%
\item If your thesis contain a large number of symbols or non-common terms you shoul consider including a {\it Nomenclature} table that would contain all symbols (and units) used in the work. 
%
\item Figures and Tables {\bf must} be referenced in the main text -- they can not just `{\it float around}'! Also, figure/table captions should be self-contained, i.e., with a good description of the figure/table highlighting the most relevant aspects/information that the author wants to convene. 
%
\item Figures, if obtained from third-party, make sure that,
\begin{itemize}
\item Figures are of high-resolution and,
\item Source/reference is in the caption.
\end{itemize}
Also, scanned tables {\bf are not} figures.  
%
\item The main aim of {\it Abstracts} is to briefly describe the work undertaken by the author. In general {\it Abstracts} are divided in 4 parts: (i) motivation, (ii) main objectives, (iii) summary of the main procedures / techniques / technologies (optional) and (iv) main findings. 
%
\item The main {\it Introduction} section usually has the same (but more in-depth and descriptive) four parts of the {\it Abstract} and a brief summary of the remaining of the work. In addition, it is always expected a few clear statements -re main background (thus recent innovations related to the main topic), initial literature review and, most of all, technological / scientific gaps in the current understanding.
%
\item You \underline{must} avoid use {\it colloquial (informal / personal)} writing. Also, try to avoid long sentences.  
%
\item Appendices are used to convey complementary (and not crucial) information of the main chapters and {\bf must} be referenced in the main text.
%
\item {\it References}: regardless of the chosen citation style (e.g., ACS, AIP, AMS, IEEE, AIAA, etc) any reference {\bf must} contain the following fields: 
\begin{enumerate}
\item For journal papers: Authors, Paper Tittle, Journal Name, Volume, Pages, Year of publication;
\item For books: Authors, Book Tittle, Publisher, Year or Edition;
\item For book chapters: Authors, Chapter Tittle, Book Tittle, Editors, Publisher, Year or Edition;
\item For conference papers: Authors, Paper Tittle, Conference Tittle, Place (Country and/or City) where the conference was held, Year of the conference;
\item For reports,  private communications and Lecture Notes: Authors, Tittle, Place issued (Country and/or City and Institution where the document was originated), Year;
\item For PhD Thesis and MSc Dissertations: Author, Tittle, Institution (University and Department/School), Year.
\end{enumerate}  
Thus, for example:
\begin{enumerate}[label={[\arabic*]}]
\item P.L. Houtekamer and L. Mitchell, $\lq$Data Assimilation Using an Ensemble Kalman Filter Technique', {\it Monthly Weather Review}, 126:796-811, 1998.
\item K. Pruess, $\lq$Numerical Modelling of Gas Migration at a Proposed Repository for Low and Intermediate Level Nuclear Wastes', Technical Report LBL-25413, Lawrence Berkeley Laboratory, Berkeley (USA), 1990.
\item K. Aziz, A. Settari, {\it Fundamentals of Reservoir Simulation}, Elsevier Applied Science Publishers, New York (USA), 1986.
\item R.B. Lowrie, $\lq$Compact Higher-Order Numerical Methods for Hyperbolic Conservation Laws', PhD Thesis, Department of Aerospace Engineering and Scientific Computing, University of Michigan (USA), 1996.
\end{enumerate} 
%
\end{enumerate}

\clearpage

\vfill 

%%%%%
%%%%%  GENERAL COMMENTS
%%%%%

\noindent{\bfseries\large MEng Chem/Mech/Pet Eng -- MSc O$\&$G, Renewable and Pet Engineering  \hfill September, 2015}

\bigskip

\begin{center}
{\Large General Comments for MEng and MSc Dissertations }
\end{center}

\medskip

\begin{enumerate}
\item The main aim of {\it Abstracts} is to briefly describe the work undertaken by the author. In general {\it Abstracts} are divided in 4 parts: (i) motivation, (ii) main objectives, (iii) summary of the main procedures / techniques / technologies (optional) and (iv) main findings. 
%
\item The main {\it Introduction} section usually has the same (but more in-depth and descriptive) four parts of the {\it Abstract} and a brief summary of the remaining of the work. In addition, it is always expected a few clear statements -re main background (thus recent innovations related to the main topic), initial literature review and, most of all, technological / scientific gaps in the current understanding. Also, it is expected a summary of the remaining sections at the end of the {\it Introduction}. 
%
\item Literature review is often gathered within a specific chapter covering the main subjects cover by the dissertation. It is always expected comments / analysis on each citation.
%
\item Equations \underline{must} be explained in full and all terms used must be defined afterwards as part of the main text. Also, the font size used in the equation must be the same as the main text.
%
\item A few {\it References} \underline{must} follow {\bf one} style with clear distinction between articles, conference proceedings, reports (internal or external), book chapters, books, communications (internal or external) etc.  Regardless of the chosen citation style (e.g., ACS, AIP, AMS, IEEE, AIAA, etc) any reference {\bf must} contain the following fields: 
\begin{enumerate}
\item For journal papers: Authors, Paper Tittle, Journal Name, Volume, Pages, Year of publication;
\item For books: Authors, Book Tittle, Publisher, Year or Edition;
\item For book chapters: Authors, Chapter Tittle, Book Tittle, Editors, Publisher, Year or Edition;
\item For conference papers: Authors, Paper Tittle, Conference Tittle, Place (Country and/or City) where the conference was held, Year of the conference;
\item For reports,  private communications and Lecture Notes: Authors, Tittle, Place issued (Country and/or City and Institution where the document was originated), Year;
\item For PhD Thesis and MSc Dissertations: Author, Tittle, Institution (University and Department/School), Year.
\end{enumerate}  
Thus, for example:
\begin{enumerate}[label={[\arabic*]}]
\item P.L. Houtekamer and L. Mitchell, $\lq$Data Assimilation Using an Ensemble Kalman Filter Technique', {\it Monthly Weather Review}, 126:796-811, 1998.
\item K. Pruess, $\lq$Numerical Modelling of Gas Migration at a Proposed Repository for Low and Intermediate Level Nuclear Wastes', Technical Report LBL-25413, Lawrence Berkeley Laboratory, Berkeley (USA), 1990.
\item K. Aziz, A. Settari, {\it Fundamentals of Reservoir Simulation}, Elsevier Applied Science Publishers, New York (USA), 1986.
\item R.B. Lowrie, $\lq$Compact higher-Order Numerical Methods for Hyperbolic Conservation Laws', PhD Thesis, Department of Aerospace Engineering and Scientific Computing, University of Michigan (USA), 1996.
\end{enumerate} 
% 
\item Figures need to be designed/drawn/scanned with {\bf high quality} with consistent legend. In addition, figure/table captions should be self-contained, i.e., with a good description of the figure/table highlighting the most relevant aspects/information that the author wants to convene. 
%
\item  Figures and tables can not be `floating', i.e., they {\bf must} be referenced in the main text.
%
\item You \underline{must} avoid use {\it colloquial (informal / personal)} writing. Also, try to avoid long (and confusing) sentences.  
%
\item Dissertations and thesis are always divided into chapters $\rightarrow$ sections, whereas reports are divided into sections.
%
%
\end{enumerate}




%%%
%%% Appendix
%%%
%{
%  \includepdf[pages=-,fitpaper, angle=0]{Scan_Review_Tripathy_2013}
%}

   
\end{document}
