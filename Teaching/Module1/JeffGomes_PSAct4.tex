\documentclass[14pt,twoside]{article}

\usepackage{amsfonts,amsmath,amssymb,stmaryrd,indentfirst}
\usepackage{epsfig,graphicx,times,psfrag}
\usepackage{natbib,enumerate}

\usepackage[pdftex,bookmarks,colorlinks=true,urlcolor=blue,citecolor=blue]{hyperref}
\pagestyle{myheadings}
\markboth{\hfill  \hfill \today}{\today \hfill Post-Session Activity 4 \hfill}
%\markboth{\hfill Jeff Gomes \hfill \today}{\today \hfill Jeff Gomes \hfill}
%\markboth{\hfill Book's Title \hfill Author's Name}{Author's Name \hfill Book's Title \hfill}

%\usepackage{fancyhdr} %%%%
%\pagestyle{fancy}%%%%
\def\newblock{\hskip .11em plus .33em minus .07em}

\setlength\textwidth      {16.5cm}
\setlength\textheight     {22.0cm}
\setlength\oddsidemargin  {-0.3cm}
\setlength\evensidemargin {-0.3cm}

\setlength\headheight{0in} 
\setlength\topmargin{0.cm}
\setlength\headsep{1.cm}
\setlength\footskip{1.cm}
\setlength\parskip{0pt}

%%%
%%% Headers and Footers
%\lhead[\text{\small{IMPERIAL COLLEGE LONDON}}] {\text{\small{Applied Modelling and Computation Group - AMCG}}} 
%%\chead[\text{\small{AMCG}}] {\text{\small{ }}}
%\rhead[\text{\small{c.pain@imperial.ac.uk}}]{\text{\small{c.pain@imperial.ac.uk}} }
%\rfoot[\thepage]{\thepage}
%\cfoot[\text{\small{April 2005}}] {\text{\small{April 2005}}}
%\lfoot [\text{\small{http://amcg.ese.imperial.ac.uk}}]{\text{\small{http://amcg.ese.imperial.ac.uk}}}
%\renewcommand{\headrulewidth}{0.8pt}

%%%
%%% space between lines
%%%
\renewcommand {\baselinestretch}{1.5}

\begin{document}

\begin{center}
{\Large Post-Session Activity 4: Critical Evaluation of Professional Development on Teaching for Learning}
\end{center}

My past teaching related experience encompass the following activities: 
\begin{enumerate}
%
\item\label{UoA_1} Lecturing (also course coordinator), supervision and tutoring undergraduate students. (UoA, 2012-today);
%
\item\label{UoA_2} Course coordinator of a MSc Programme (UoA, 2012-today);
%
\item\label{IC_1} Supervision and mentoring of PhD students (Imperial College, 2006-12);
%
\item\label{UFRJ} Lecturing undergraduate engineering students and supervising final-year projects (Brazil, 2004-5);
%
\item\label{Aula} Private tuition for undergraduate and high-school students (Brazil, 1988-96). 
%
\end{enumerate}
In all these activities, I have the opportunity to support the continuous learning of young students. Most of the above teaching and mentoring activities (\ref{IC_1}-\ref{Aula}) were undertaken with no formal training, but rather engaging on experimentation and heuristics. More recently, I was enrolled in a programme on $\lq$teaching for learning' (T4L) at Imperial for early career academics (2012). This started raining my awareness of new technologies and methods aimed to aid the learning processes in individuals, small and large groups, and enhanced my teaching activities (\ref{UoA_1}-\ref{IC_1}). My T4L understanding was consolidated during the current training programme held at the Centre of Academic Development aiming to pursue a PG Certificate in $\lq$Higher Education Learning and Teaching'. 


I envisage that one of my academic roles in the future will be as a $\lq$learning facilitator', i.e., to make sure that students have the best learning experience during their undergraduate studies, and that during this time they are able to develop creative and critical thinking attributes -- crucial for young engineers and scientists. In order to apply to the Associate Fellowship of the AFHEA, I chose to be assessed on the following areas of activity: 
\begin{enumerate}[(a)]
\item\label{Act1} design and plan learning activities and/or programme of study and;
\item\label{Act2} teach and/or support learning.
\end{enumerate}
As (A) course coordinator (and main contributor) of $\lq$Engineering Thermodynamics' (Level 3, UG) and (B) contributor of $\lq$Energy Technologies' (one-off 2 hours lectures, MSc Programme), I intend to use the lessons learnt during the T4L training course to improve both courses. For (A), I will have the opportunity to partially redesign the course $\left(\text{2}^{\text{nd}}\text{ half session}\right)$ and change the way the course has been delivered (some of the ideas below were introduced in the last academic year and can be improved based upon the students feedback -- SCEF):
\begin{enumerate}
%
\item\label{CourseMaterial} Consistent and uniform course delivery across both contributors comprising: 
\begin{enumerate}
\item lectures notes containing in-class fill-in spaces (promoting participation during the lectures); 
\item industrial applications of the fundamentals learnt in class (as films or invited talks);
\item problem-solving sessions to help students connecting disciplines learnt in previous levels applied to the current course;
\item comprehensive feedback on problems introduced during tutorial sessions;
\end{enumerate}
\item continuous assessment (group work) on two-fold activities based on scientific / technical papers related to the main course: 
\begin{enumerate}
\item technical report comprising literature review and critical analysis on the main technical subject of the paper related to the course;
\item oral presentation containing summary of the paper, 1-2 slides designed to a broad (and non-technical) audience, 1-2 slides on $\lq$how to bring the contents of the paper into a lecture'.  
\end{enumerate}
Comprehensive feedback is given within 2 weeks of report submission and presentations.  
%
\end{enumerate} 
The objective of this set of activities is to (i) promote student engagement, (ii) raise awareness of continuous self-learning, (iii) improve soft engineering skills (communicating ideas, team-player, leadership) and (c) show that learning is a continuous and inter-connected process.
\medskip

For the one-off lecture (B), I will teach recent advances in nuclear technology to students enrolled in the MSc in Oil and Gas Engineering Programme. The syllabus of this lecture comprises (i) to introduce energy generation workflow; (ii) demand and consumption of energy; (iii) mitigation of greenhouse gas emissions using advanced energy technologies; (iv) established and novel nuclear technologies (v) nuclear waste disposal and safety; and (vi) future technology trends. At the end of the lecture, the students should be able to (a) identify {\it energy mix} for electricity generation in international markets, (b) critically assess risks (economics and safety) of novel nuclear technologies and (c) identify novel and derived nuclear energy technologies. The lecture (2 hours) and tutorial (1 hour) are designed to:
\begin{enumerate}
\item contain short films produced by nuclear operators showing fundamental physics and stages of nuclear power plant design and operations for non-nuclear experts;
\item show data (and references) on the main worldwide energy statistics;
\item in the tutorial session, I intend to divide the class in groups (5-7) for discussion of specific topics covered during the lecture (e.g., economics, media and public perception of nuclear energy in the world, accidents, etc). They can (individually) write a 1-page essay describing their discussion and main conclusions, with expected feedback in 15 days.
\end{enumerate}  
These activities will (a) promote student engagement and awareness of energy technologies and associated risks; (b) improve their abilities to discuss about unfamiliar topics; and (c) improve their critical view on the complete workflow of energy cycles (i.e., industrial processes, economics, safety, etc).   




\end{document}

