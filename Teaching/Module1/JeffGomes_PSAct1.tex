\documentclass[14pt,twoside]{article}

\usepackage{amsfonts,amsmath,amssymb,stmaryrd,indentfirst}
\usepackage{epsfig,graphicx,times,psfrag}
\usepackage{natbib,enumerate}

\usepackage[pdftex,bookmarks,colorlinks=true,urlcolor=blue,citecolor=blue]{hyperref}


%\usepackage{fancyhdr} %%%%
%\pagestyle{fancy}%%%%
\def\newblock{\hskip .11em plus .33em minus .07em}

\setlength\textwidth      {16.5cm}
\setlength\textheight     {22.0cm}
\setlength\oddsidemargin  {-0.3cm}
\setlength\evensidemargin {-0.3cm}

\setlength\headheight{0in} 
\setlength\topmargin{0.cm}
\setlength\headsep{1.cm}
\setlength\footskip{1.cm}
\setlength\parskip{0pt}

%%%
%%% Headers and Footers
%\lhead[\text{\small{IMPERIAL COLLEGE LONDON}}] {\text{\small{Applied Modelling and Computation Group - AMCG}}} 
%%\chead[\text{\small{AMCG}}] {\text{\small{ }}}
%\rhead[\text{\small{c.pain@imperial.ac.uk}}]{\text{\small{c.pain@imperial.ac.uk}} }
%\rfoot[\thepage]{\thepage}
%\cfoot[\text{\small{April 2005}}] {\text{\small{April 2005}}}
%\lfoot [\text{\small{http://amcg.ese.imperial.ac.uk}}]{\text{\small{http://amcg.ese.imperial.ac.uk}}}
%\renewcommand{\headrulewidth}{0.8pt}

%%%
%%% space between lines
%%%
\renewcommand {\baselinestretch}{1.5}

\begin{document}

\begin{center}
{\Large Post-Session Activity 1: My Understanding of Effective Learning}
\end{center}
I would define {\it learning} as sequences of individual processes that deal with unknown (or not clearly defined) experiences (e.g., knowledge, skills). Here, a process (e.g., memorisation, familiarisation) is a set of linked (either continuous or discontinuous) operations or activities. In my opinion, {\it effective learning} is a set of new sequences that aims to consolidate the understanding of a discipline. 

From the above, expectation, memorisation and initial understanding are three examples of processes that are part of the {\it learning experience}. Each process can be subdivided into specific operations/activities (e.g., techniques to memorise or to comprehend/understand a subject). Also, each process is connected to one or more processes (of different weights) summing as individual sequences that are also inter-connected. {\it Effective learning} is when the vast majority of these sequences of {\it learning experience} are connected and an individual is able to readily apply to specific problems.   

\medskip

I currently teach Thermodynamics for 3$^{\text{rd}}$ year Engineering students (as well as supervision of UG and PG students). My usual approach is based on the following sequence: 
\begin{itemize}
\item Motivation, or why learning a particular subject is important in the professional life (using examples, if possible);
\item Main background of the subject, i.e., review the main scientific aspects that are crucial to fully understand the subject;
\item The main subject is then introduced as a natural consequence of the main background;
\item Examples are introduced, starting from simple cases with direct (engineering) applications. Other (more complex) examples involving previous learnt (and inter-connected) subjects are introduced. I believe this may help individuals to (a) understand the recent learnt subject (initial learning consolidation); (b) raise awareness that knowledge is a continuous and cumulative process;
\item As part of the continuous assessment (but also for the continuous learning process), I split the class in small groups and assign to each group a scientific paper related to a particular (and applied) aspect of the course. They are asked to write a report followed by an oral presentation containing:
\begin{enumerate}[(a)]
\item summary of the paper;
\item a short (but comprehensive) literature review on the main (or more relevant) subject related to the course;
\item 1 page document (and 1-2 slides) summarising the paper for a wide (but not technical) audience;
\item 1-2 slides on how the paper could fit in a thermodynamic lecture.
\end{enumerate}
\end{itemize} 



\end{document}

