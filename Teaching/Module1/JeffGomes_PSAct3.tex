\documentclass[14pt,twoside]{article}

\usepackage{amsfonts,amsmath,amssymb,stmaryrd,indentfirst}
\usepackage{epsfig,graphicx,times,psfrag}
\usepackage{natbib,enumerate}

\usepackage[pdftex,bookmarks,colorlinks=true,urlcolor=blue,citecolor=blue]{hyperref}
\pagestyle{myheadings}
\markboth{\hfill  \hfill \today}{\today \hfill Post-Session Activity 3 \hfill}
%\markboth{\hfill Jeff Gomes \hfill \today}{\today \hfill Jeff Gomes \hfill}
%\markboth{\hfill Book's Title \hfill Author's Name}{Author's Name \hfill Book's Title \hfill}

%\usepackage{fancyhdr} %%%%
%\pagestyle{fancy}%%%%
\def\newblock{\hskip .11em plus .33em minus .07em}

\setlength\textwidth      {16.5cm}
\setlength\textheight     {22.0cm}
\setlength\oddsidemargin  {-0.3cm}
\setlength\evensidemargin {-0.3cm}

\setlength\headheight{0in} 
\setlength\topmargin{0.cm}
\setlength\headsep{1.cm}
\setlength\footskip{1.cm}
\setlength\parskip{0pt}

%%%
%%% Headers and Footers
%\lhead[\text{\small{IMPERIAL COLLEGE LONDON}}] {\text{\small{Applied Modelling and Computation Group - AMCG}}} 
%%\chead[\text{\small{AMCG}}] {\text{\small{ }}}
%\rhead[\text{\small{c.pain@imperial.ac.uk}}]{\text{\small{c.pain@imperial.ac.uk}} }
%\rfoot[\thepage]{\thepage}
%\cfoot[\text{\small{April 2005}}] {\text{\small{April 2005}}}
%\lfoot [\text{\small{http://amcg.ese.imperial.ac.uk}}]{\text{\small{http://amcg.ese.imperial.ac.uk}}}
%\renewcommand{\headrulewidth}{0.8pt}

%%%
%%% space between lines
%%%
\renewcommand {\baselinestretch}{1.5}

\begin{document}

\begin{flushright}
Dr Jefferson Gomes\\
School of Engineering, \\
University of Aberdeen\\
Aberdeen, AB24 3UE
\end{flushright}

\begin{flushleft}
Director of the Education Learning and Teaching Programme\\
Centre of Academic Development\\
University of Aberdeen\\
Aberdeen, AB24 3UE\\
{\bf Subject:} Application for AFHEA
\end{flushleft}

\noindent
Dear Director of the Education Learning and Teaching Programme,

My name is Jefferson Gomes and since December 2012 I am a lecturer in the School of Engineering (SoE). Since I arrived in the SoE I have been engaged in teaching (course coordinator and contributor) and supervision (BEng, MEng and MSc programmes) of students. 

Prior to my arrival in Aberdeen, I acted as PhD supervisor and mentor (Imperial College, 2006-12), UG lecturer and project supervisor (Brazil, 2004-05) and as private tutor (Brazil, 1988-96). In all these teaching- and learning-related activities, I was fortunate for the opportunity to help young pupils and scientists to reach their academic and professional objectives, and to improve my teaching skills. More recently, I was enrolled in a programme on $\lq$teaching for learning' at Imperial for early career academics (2012). This rather formal training raised my awareness to new technologies and methods aimed to aid the learning processes in individuals, small and large groups. 

I am currently undertaking a similar $\lq$teaching for learning' (T4L) training programme at the Centre of Academic Development (at UoA) aiming to pursue a PG Certificate in $\lq$Higher Education Learning and Teaching'. I believe that a more in-depth training on T4L will help
\begin{enumerate} [(a)]
\item improving my teaching and communication skills; 
\item to ensure that graduate attributes/competences are correctly embedded in my courses and are continuously developed by students in the modules/courses I coordinate / contribute / supervise; 
\item raising my awareness on key-aspects of students learning (and therefore helping students benefit from their experiences as undergraduates); 
\item to appreciate elements of good course design, delivery and assessment and;
\item to be aware of different methods and technologies to interactive small and large group teaching.
\end{enumerate} 

I envisage that one of my academic roles in the future will be as a $\lq$learning facilitator', i.e., to make sure that students have the best learning experience during their undergraduate studies, and that during this time they are able to develop creative and critical thinking attributes -- crucial for young engineers and scientists. 

To this end, I would like to apply to an Associate Fellowship of the AFHEA through a professional interview. As required for this chartered qualification, I would like to be assessed on the following areas of activity, (a) design and plan learning activities and/or programme of study and (b) teach and/or support learning. 

I believe that my short teaching experience and exposure have helped understand the structure necessary to create, foster, develop, coordinate and teach courses that ensure graduate attributes are embedded, and that the technical contents are learnt in the most efficient and continuous way. Additionally, novel assisting technologies are now available and I am keen to learn how to better utilise them to enhance students' learning experiences and to promote effective knowledge (and/or experience) exchange between the students and me.



\end{document}

